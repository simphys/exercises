%%%%%%%% Klassen-Optionen
\documentclass[12pt,a4paper]{scrartcl}

%%%%%%%% PAKETE: unverzichtbare Pakete mit Einstellungen
\usepackage[left=2.5cm, right=2cm, top=3cm, bottom=3cm, a4paper]{geometry} %Seitenrände
\usepackage[utf8x]{inputenc} % utf8-Kodierung und direkte Eingabe von Sonderzeichen
\usepackage{fixltx2e} % Verbessert einige Kernkompetenzen von LaTeX2e

%%%%%%%% PAKETE: AMS-Pakete
\usepackage{amsmath} % Mathe-Erweiterung
\usepackage{amsfonts} % Schrift-Erweiterung
\usepackage{amssymb} % Sonderzeichen-Erweiterung

%%%%%%%% PAKETE: Sonstiges
\usepackage[colorlinks, citecolor=black, filecolor=black, linkcolor=black, urlcolor=black]{hyperref} % Links
\usepackage{wrapfig} % ausgeklügekte Floatumgebung
\usepackage{float} % normale Floatumgebung
\restylefloat{figure} % ermöglicht die Verwendung von "H" (ist noch stärker als "h!")
\usepackage[small,it,singlelinecheck=false]{caption} % Bildunterschriften formatieren
\usepackage{multirow} % ermöglich Verbinden von Tabellenzeilen
\usepackage{multicol} % ermöglicht Spalten
\usepackage{fancyhdr} % ermöglicht Kopf- und Fußzeilen
\usepackage{graphicx} % Einbinden von Bildern möglich
\usepackage{units} % Einheiten
\usepackage{subcaption}

%%%%%%%% DEFINITIONEN: Titelseite
\author{April Cooper, Patrick Kreissl und Sebastian Weber}
\title{Worksheet 3: Diffusion processes and atomistic water model properties}
\publishers{University of Stuttgart}
\date{\today}

%%%%%%%% ANPASSUNGEN: Kopf-und Fußzeile
\fancypagestyle{plain}{} % redefine the plain pagestyle to match the fancy layout
\pagestyle{fancy} % aktiviere eigenen Seitenstil
\fancyhf{} % alle Kopf- und Fußzeilen bereinigen
\fancyhead[L]{Worksheet 3: Diffusion processes and atomistic water model properties}
\fancyhead[R]{\today}
\renewcommand{\headrulewidth}{0.6pt} % obere Trennlinie
\fancyfoot[L]{April Cooper, Patrick Kreissl und Sebastian Weber}
\fancyfoot[R]{Page \thepage}
\renewcommand{\footrulewidth}{0.6pt} % untere Trennlinie

%%%%%%%% ANPASSUNGEN: Absätze
\setlength{\parindent}{0em} % keine Absatzeinzüge
\setlength{\parskip}{0.5em} % Absatz-Abstand

%%%%%%%% ANPASSUNGEN: Abbildungsverzeichnis
\usepackage{tocloft} % Zum Anpassen der Verzeichnisse
%\renewcommand{\cftfigpresnum}{Abb. }
%\renewcommand{\cfttabpresnum}{Tab. }
\renewcommand{\cftfigaftersnum}{:}
\renewcommand{\cfttabaftersnum}{:}
\setlength{\cftfignumwidth}{2cm}
\setlength{\cfttabnumwidth}{2cm}
\setlength{\cftfigindent}{0cm}
\setlength{\cfttabindent}{0cm}

%%%%%%%% SONSTIGES
\usepackage{pdfpages}
\usepackage{pgf}
%\usepackage{subfigure}
\usepackage{graphicx}
\usepackage{caption}
\usepackage{subcaption}


% NÜTZLICH: http://truben.no/latex/table/

% Anfang des eigentlichen Dokuments
\begin{document}

\maketitle
\tableofcontents
\newpage

\section{Short Questions - Short Answers}


\subsubsection*{What are the main differences between various atomistic water models?}
\begin{itemize}
\item Geometry - some are planar, some tetrahedral, also the location and size of partial charges can differ 
\item Polarizability - some models take it into account some don't
\item Rigidness - some have fixed atom positions, others model atoms connected by "springs"
\end{itemize}

\subsubsection*{What is the difference between the SPC and the SPC/E water model? }
The SPC/E model takes the averaged polarization effects  into account, SPC doesn't.

\subsubsection*{What are the typical terms in an atomistic classical force field?}
Typical terms for the potential are: $E_{bond}$,$E_{torsion}$,$E_{angular}$,$E_{van-der-Waals}$, $E_{LJ}$ and 
$E_{coulomb}$

\subsubsection*{How is the Pauli exclusion principle incorporated into a classical force field?}
It is incorporated into the energy expression of the Lennard-Jones interactions $E_{LJ}$. If two (non-bonded) atoms get too close to each other their electron clouds overlap which results due to Pauli repulsion in a very strong repulsive force between these atoms. In the Lennard-Jones  potential the $r^{-12}$- term describes this strong (Pauli -) repulsion.

\section{Theoretical Task: Langevin equation - Calculation of particle
positions and velocities}
In this theoretical task, the Langevin equation describing the Brownian motion has to be solved:

\begin{equation}
	\text{d}v = - \gamma v \text ~ \text{d}t + \frac{\Gamma}{m}~\text{d}W
\end{equation}

The first term on the right hand side describes the dissipative force, the second the stochastic force.

\subsection{Velocities of the particle}
Since the average force in the Langevin equation is already included in the first force term, the stochastic second one has to be zero on average: $\langle \text{d}W(t) \rangle = 0$. Therefore the second term can be neglected if one is only interested in computing the average force (force term one):
\begin{equation}
	\text{d} v = - \gamma v ~ \text{d} t
\end{equation}
This differential equation can be easily solved by separation of variables, which leads to to following solution (with $v_0 = v(t=0)$):
\begin{equation} 
	v = v_0 \cdot \exp{(-\gamma t)}
\end{equation}

\end{document}


% =============== Comments ============
\begin{comment}
\verb{x_init {}}

\begin{figure}[H]
	\resizebox{1\textwidth}{!{\input{../plots/GGA_mesh.pdf}}
	\caption{CAPTION}\label{fig:NAME}
\end{figure}
\end{comment}
