%%%%%%%% Klassen-Optionen
\documentclass[12pt,a4paper]{scrartcl}

%%%%%%%% PAKETE: unverzichtbare Pakete mit Einstellungen
\usepackage[left=2.5cm, right=2cm, top=3cm, bottom=3cm, a4paper]{geometry} %Seitenrände
\usepackage[utf8x]{inputenc} % utf8-Kodierung und direkte Eingabe von Sonderzeichen
\usepackage{fixltx2e} % Verbessert einige Kernkompetenzen von LaTeX2e

%%%%%%%% PAKETE: AMS-Pakete
\usepackage{amsmath} % Mathe-Erweiterung
\usepackage{amsfonts} % Schrift-Erweiterung
\usepackage{amssymb} % Sonderzeichen-Erweiterung

%%%%%%%% PAKETE: Sonstiges
\usepackage[colorlinks, citecolor=black, filecolor=black, linkcolor=black, urlcolor=black]{hyperref} % Links
\usepackage{wrapfig} % ausgeklügekte Floatumgebung
\usepackage{float} % normale Floatumgebung
\restylefloat{figure} % ermöglicht die Verwendung von "H" (ist noch stärker als "h!")
\usepackage[small,it,singlelinecheck=false]{caption} % Bildunterschriften formatieren
\usepackage{multirow} % ermöglich Verbinden von Tabellenzeilen
\usepackage{multicol} % ermöglicht Spalten
\usepackage{fancyhdr} % ermöglicht Kopf- und Fußzeilen
\usepackage{graphicx} % Einbinden von Bildern möglich
\usepackage{units} % Einheiten
\usepackage{subcaption}

%%%%%%%% DEFINITIONEN: Titelseite
\author{April Cooper, Patrick Kreissl und Sebastian Weber}
\title{Worksheet 3: Diffusion processes and atomistic water model properties}
\publishers{University of Stuttgart}
\date{\today}

%%%%%%%% ANPASSUNGEN: Kopf-und Fußzeile
\fancypagestyle{plain}{} % redefine the plain pagestyle to match the fancy layout
\pagestyle{fancy} % aktiviere eigenen Seitenstil
\fancyhf{} % alle Kopf- und Fußzeilen bereinigen
\fancyhead[L]{Worksheet 3: Diffusion processes and atomistic water model properties}
\fancyhead[R]{\today}
\renewcommand{\headrulewidth}{0.6pt} % obere Trennlinie
\fancyfoot[L]{April Cooper, Patrick Kreissl und Sebastian Weber}
\fancyfoot[R]{Page \thepage}
\renewcommand{\footrulewidth}{0.6pt} % untere Trennlinie

%%%%%%%% ANPASSUNGEN: Absätze
\setlength{\parindent}{0em} % keine Absatzeinzüge
\setlength{\parskip}{0.5em} % Absatz-Abstand

%%%%%%%% ANPASSUNGEN: Abbildungsverzeichnis
\usepackage{tocloft} % Zum Anpassen der Verzeichnisse
%\renewcommand{\cftfigpresnum}{Abb. }
%\renewcommand{\cfttabpresnum}{Tab. }
\renewcommand{\cftfigaftersnum}{:}
\renewcommand{\cfttabaftersnum}{:}
\setlength{\cftfignumwidth}{2cm}
\setlength{\cfttabnumwidth}{2cm}
\setlength{\cftfigindent}{0cm}
\setlength{\cfttabindent}{0cm}

%%%%%%%% SONSTIGES
\usepackage{pdfpages}
\usepackage{pgf}
%\usepackage{subfigure}
\usepackage{graphicx}
\usepackage{caption}
\usepackage{subcaption}


% NÜTZLICH: http://truben.no/latex/table/

% Anfang des eigentlichen Dokuments
\begin{document}

\maketitle
\tableofcontents
\newpage

\section{Short Questions - Short Answers}


\subsubsection*{What are the main differences between various atomistic water models?}
\begin{itemize}
\item Geometry - some are planar, some tetrahedral, also the location and size of partial charges can differ 
\item Polarizability - some models take it into account some don't
\item Rigidness - some have fixed atom positions, others model atoms connected by "springs"
\end{itemize}

\subsubsection*{What is the difference between the SPC and the SPC/E water model? }
The SPC/E model takes the averaged polarization effects  into account, SPC doesn't.

\subsubsection*{What are the typical terms in an atomistic classical force field?}
Typical terms for the potential are: $E_{bond}$,$E_{torsion}$,$E_{angular}$,$E_{van-der-Waals}$, $E_{LJ}$ and 
$E_{coulomb}$

\subsubsection*{How is the Pauli exclusion principle incorporated into a classical force field?}
It is incorporated into the energy expression of the Lennard-Jones interactions $E_{LJ}$. If two (non-bonded) atoms get too close to each other their electron clouds overlap which results due to Pauli repulsion in a very strong repulsive force between these atoms. In the Lennard-Jones  potential the $r^{-12}$- term describes this strong (Pauli -) repulsion.

\section{Theoretical Task: Langevin equation - Calculation of particle
positions and velocities}
In this theoretical task, the Langevin equation describing the Brownian motion has to be solved:

\begin{equation}
	\text{d}v = - \gamma v \text ~ \text{d}t + \frac{\Gamma}{m}~\text{d}W
\end{equation}

The first term on the right hand side describes the dissipative force, the second the stochastic force.

\newpage
\subsection{Velocities of the particle}
Since the average force in the Langevin equation is already included in the first force term, the stochastic second one has to be zero on average: 
\begin{equation}
	\langle \text{d}W(t) \rangle = 0
\end{equation}
Therefore the second term can be neglected if one is only interested in computing the average force (force term one):
\begin{equation}
	\text{d} v = - \gamma v ~ \text{d} t
\end{equation}
This differential equation can be easily solved by separation of variables, which leads to to following solution (with $v_0 = v(t=0)$):
\begin{equation} 
	v(t) = v_0 \cdot \text{e}^{(-\gamma t)}
\end{equation}
The stochastic fluctuations of the second term also fulfill the following relation:
\begin{equation}
	\langle\text{d}W(t_1)\text{d}W(t_2)\rangle =  \delta_{t_1,t_2} ~ \text{d}t_1
\end{equation}
An explicit formal solution can be obtained as
\begin{equation}
	v(t) = v_0 \cdot \text{e}^{-\gamma t} + \frac{\Gamma}{m} \int_0^t \text{e}^ {-\gamma(t-s)}~\text{d}W(s)
	\label{eqn:v}
\end{equation}
Now one can calculate:
\begin{align}
	\langle v(t_1)v(t_2) \rangle &= \langle v_0 \cdot  v_0\rangle \cdot \text{e}^{-\gamma (t_1+t_2)}
 + \left( \frac{\Gamma}{m} \right)^2 \int_0^{t_1} \int_0^{t_2} \text{e}^ {-\gamma\cdot(t_1+t_2-(s_1+s_2))}~\langle\text{d}W(s_2)\text{d}W(s_1)\rangle\\
	&= \langle v_0^2\rangle \cdot \text{e}^{-\gamma\cdot (t_1+t_2)}	
	 + \left( \frac{\Gamma}{m} \right)^2  \int_0^{\min{t_1, t_2}} \text{e}^ {-\gamma\cdot(t_1+t_2-2s)}~\text{d}s\\
	&= \langle v_0^2\rangle \cdot \text{e}^{-\gamma\cdot (t_1+t_2)}+ \left( \frac{\Gamma}{m} \right)^2 \frac{1}{2 \gamma} \text{e}^ {-\gamma\cdot(t_1+t_2)}\left( \text{e}^ {2\gamma \min{t_1, t_2}} - 1 \right)
\end{align}
For $t_1=t_2=t$ this results in
\begin{align}
	\langle v(t)^2\rangle &= \langle v_0(t)^2\rangle \cdot \text{e}^{-2\gamma t}+ \left( \frac{\Gamma}{m} \right)^2 \frac{1}{2 \gamma} \text{e}^ {-2\gamma t}\left( \text{e}^ {2\gamma t} - 1 \right)\\
	&=\langle v_0(t)^2\rangle \cdot \text{e}^{-2\gamma t} - \frac{\Gamma}{2 m\gamma} \text{e}^ {-2\gamma t} + \left( \frac{\Gamma}{m} \right)^2 \frac{1}{2 \gamma}\\
	&=\left( \langle v_0(t)^2\rangle - \frac{\Gamma}{2 m\gamma} \right) \text{e}^ {-2\gamma t} + \left( \frac{\Gamma}{m} \right)^2 \frac{1}{2 \gamma} \label{eq:velsqrt}
\end{align}
Since in equilibrium we must have the equipartition theorem for one dimension ($\langle v(t)^2\rangle_{\text{eq}} = k_\text{B} \cdot T / m$, in three dimensions the right hand side would have to be multiplied by three) $\Gamma$ can be calculated (for $t \rightarrow \infty$ the first part of equation \ref{eq:velsqrt} vanishes):
\begin{equation}
	\Gamma = \sqrt{2 \gamma k_\text{B} T m}
\end{equation}

\subsection{Position of the particle}

By integrating equation \ref{eqn:v} we get an expression for the position.

\begin{align}
x(t) &= x_0 +  \int_0^t \left[ v_0 \cdot \text{e}^{-\gamma u} + \frac{\Gamma}{m} \int_0^u \text{e}^ {-\gamma(u-s)}~\text{d}W(s) \right] du \\
&= x_0 + \frac{v_0}{\gamma} \left(1-\text{e}^{-\gamma t} \right) + \frac{\Gamma}{m} \int_0^t \text{d}W(s) \int_s^t \text{e}^{-\gamma (u-s)} du \\
&= x_0 + \frac{v_0}{\gamma} \left(1-\text{e}^{-\gamma t} \right) +\frac{\Gamma}{m \gamma} \int_0^t \left(1-\text{e}^{-\gamma (t-s)} \right) ~\text{d}W(s) \label{eqn:x}
\end{align}
For the mean position, the last term gets zero because of it dependents on $\text{d}W(s)$. So the result is:
\begin{equation}
\langle x(t) \rangle = x_0 + \frac{v_0}{\gamma} \left(1-\text{e}^{-\gamma t} \right)
\end{equation}
This gives us in the limit of ...
\begin{itemize}
\item short time scales ($\gamma t << 1$) : $\langle x(t) \rangle \approx x_0 + v_0 \cdot t$
\item long time scales ($\tfrac{1}{\gamma t} << 1$) : $\langle x(t) \rangle \approx x_0 + \frac{v_0}{\gamma}$
\end{itemize} \vspace{2em}

In addition, equation \ref{eqn:x} results in the following mean-square displacement:
\begin{align}
\langle \Delta x(t)^2 \rangle &= \langle \left( x(t) - \langle x(t) \rangle \right)^2 \rangle = \langle x(t)^2 \rangle - \left( \langle x(t)\rangle \right)^2 \\
&= \left(\frac{\Gamma}{m \gamma} \right)^2 \int_0^t \int_0^t \left(1-\text{e}^{-\gamma (t-s_1)} \right) \left(1-\text{e}^{-\gamma (t-s_2)} \right) \langle ~\text{d}W(s_1) ~\text{d}W(s_2) \rangle \\
&= \left( \frac{\Gamma}{m \gamma} \right)^2 \int_0^t \left(1-\text{e}^{-\gamma (t-s)} \right)^2 ~\text{d}s \\
&= \frac{2 k_B T}{m \gamma} \int_0^t 1+\text{e}^{-2\gamma (t-s)}-2\text{e}^{-\gamma (t-s)} ~\text{d}s \\
&= \frac{2 k_B T}{m \gamma} \left( t + \frac{4\text{e}^{-\gamma t}-\text{e}^{-2\gamma t}}{2 \gamma} - \frac{3}{2g}\right)
%&= \frac{2 k_B T}{m \gamma} \left( t + \frac{1}{2 \gamma} (1-\text{e}^{-2 \gamma t}) - \frac{2}{\gamma} (1-\text{e}^{-\gamma t} ) \right)
%&\approx \frac{2 k_B T}{m \gamma} \left( t + \frac{1}{2 \gamma} (2 \gamma t + \frac{(2 \gamma t)^2}{2}) - \frac{2}{\gamma} (-\gamma t+ \frac{(\gamma t)^2}{2}) \right) \\
%&=\frac{2 k_B T}{m \gamma} \left( t + \frac{1}{2 \gamma} (2 \gamma t - \frac{(2 \gamma t)^2}{2}) - \frac{2}{\gamma} (\gamma t- \frac{(\gamma t)^2}{2}) \right)
\end{align}

%This results in the limit of ...
%\begin{itemize}
%\item short time scales: $\langle \Delta x(t)^2 \rangle  \varpropto \frac{k_B T}{m} t^2$
%\item long time scales: $\langle \Delta x(t)^2 \rangle \varpropto \frac{2 k_B T}{m \gamma} t$
%\end{itemize}

For long time scales the exponential functions approx zero. It results in:
\begin{equation}
\langle \Delta x(t)^2 \rangle \varpropto \frac{2 k_B T}{m \gamma} t
\end{equation}

With the equation  $\langle \Delta x(t)^2 \rangle = 2 D t$ (for three dimensions $6 D t$) you get the following diffusion constant:
\begin{equation}
D = \frac{k_B T}{m \gamma}
\end{equation}


\section{Computational Task: Atomistic water simulations with GROMACS}

\subsection{Radial distribution function}
\begin{figure}[H]
	\resizebox{\linewidth}{!}{%% Creator: Matplotlib, PGF backend
%%
%% To include the figure in your LaTeX document, write
%%   \input{<filename>.pgf}
%%
%% Make sure the required packages are loaded in your preamble
%%   \usepackage{pgf}
%%
%% Figures using additional raster images can only be included by \input if
%% they are in the same directory as the main LaTeX file. For loading figures
%% from other directories you can use the `import` package
%%   \usepackage{import}
%% and then include the figures with
%%   \import{<path to file>}{<filename>.pgf}
%%
%% Matplotlib used the following preamble
%%   \usepackage{fontspec}
%%   \setmainfont{DejaVu Serif}
%%   \setsansfont{DejaVu Sans}
%%   \setmonofont{DejaVu Sans Mono}
%%
\begingroup%
\makeatletter%
\begin{pgfpicture}%
\pgfpathrectangle{\pgfpointorigin}{\pgfqpoint{8.000000in}{3.500000in}}%
\pgfusepath{use as bounding box}%
\begin{pgfscope}%
\pgfsetrectcap%
\pgfsetroundjoin%
\definecolor{currentfill}{rgb}{1.000000,1.000000,1.000000}%
\pgfsetfillcolor{currentfill}%
\pgfsetlinewidth{0.000000pt}%
\definecolor{currentstroke}{rgb}{1.000000,1.000000,1.000000}%
\pgfsetstrokecolor{currentstroke}%
\pgfsetdash{}{0pt}%
\pgfpathmoveto{\pgfqpoint{0.000000in}{0.000000in}}%
\pgfpathlineto{\pgfqpoint{8.000000in}{0.000000in}}%
\pgfpathlineto{\pgfqpoint{8.000000in}{3.500000in}}%
\pgfpathlineto{\pgfqpoint{0.000000in}{3.500000in}}%
\pgfpathclose%
\pgfusepath{fill}%
\end{pgfscope}%
\begin{pgfscope}%
\pgfsetrectcap%
\pgfsetroundjoin%
\definecolor{currentfill}{rgb}{1.000000,1.000000,1.000000}%
\pgfsetfillcolor{currentfill}%
\pgfsetlinewidth{0.000000pt}%
\definecolor{currentstroke}{rgb}{0.000000,0.000000,0.000000}%
\pgfsetstrokecolor{currentstroke}%
\pgfsetdash{}{0pt}%
\pgfpathmoveto{\pgfqpoint{1.000000in}{0.350000in}}%
\pgfpathlineto{\pgfqpoint{7.200000in}{0.350000in}}%
\pgfpathlineto{\pgfqpoint{7.200000in}{3.150000in}}%
\pgfpathlineto{\pgfqpoint{1.000000in}{3.150000in}}%
\pgfpathclose%
\pgfusepath{fill}%
\end{pgfscope}%
\begin{pgfscope}%
\pgfpathrectangle{\pgfqpoint{1.000000in}{0.350000in}}{\pgfqpoint{6.200000in}{2.800000in}} %
\pgfusepath{clip}%
\pgfsetrectcap%
\pgfsetroundjoin%
\pgfsetlinewidth{1.003750pt}%
\definecolor{currentstroke}{rgb}{0.000000,0.000000,1.000000}%
\pgfsetstrokecolor{currentstroke}%
\pgfsetdash{}{0pt}%
\pgfpathmoveto{\pgfqpoint{1.000000in}{0.350000in}}%
\pgfpathlineto{\pgfqpoint{2.500400in}{0.350461in}}%
\pgfpathlineto{\pgfqpoint{2.512800in}{0.351452in}}%
\pgfpathlineto{\pgfqpoint{2.525200in}{0.354999in}}%
\pgfpathlineto{\pgfqpoint{2.537600in}{0.366397in}}%
\pgfpathlineto{\pgfqpoint{2.550000in}{0.389446in}}%
\pgfpathlineto{\pgfqpoint{2.562400in}{0.435840in}}%
\pgfpathlineto{\pgfqpoint{2.574800in}{0.517536in}}%
\pgfpathlineto{\pgfqpoint{2.587200in}{0.641090in}}%
\pgfpathlineto{\pgfqpoint{2.599600in}{0.817422in}}%
\pgfpathlineto{\pgfqpoint{2.612000in}{1.040588in}}%
\pgfpathlineto{\pgfqpoint{2.661600in}{2.095000in}}%
\pgfpathlineto{\pgfqpoint{2.674000in}{2.298888in}}%
\pgfpathlineto{\pgfqpoint{2.686400in}{2.462632in}}%
\pgfpathlineto{\pgfqpoint{2.698800in}{2.562552in}}%
\pgfpathlineto{\pgfqpoint{2.711200in}{2.616680in}}%
\pgfpathlineto{\pgfqpoint{2.723600in}{2.625680in}}%
\pgfpathlineto{\pgfqpoint{2.736000in}{2.585368in}}%
\pgfpathlineto{\pgfqpoint{2.748400in}{2.521704in}}%
\pgfpathlineto{\pgfqpoint{2.760800in}{2.426592in}}%
\pgfpathlineto{\pgfqpoint{2.785600in}{2.205016in}}%
\pgfpathlineto{\pgfqpoint{2.810400in}{1.959088in}}%
\pgfpathlineto{\pgfqpoint{2.822800in}{1.867264in}}%
\pgfpathlineto{\pgfqpoint{2.835200in}{1.741512in}}%
\pgfpathlineto{\pgfqpoint{2.847600in}{1.667328in}}%
\pgfpathlineto{\pgfqpoint{2.860000in}{1.577952in}}%
\pgfpathlineto{\pgfqpoint{2.884800in}{1.435064in}}%
\pgfpathlineto{\pgfqpoint{2.897200in}{1.374616in}}%
\pgfpathlineto{\pgfqpoint{2.909600in}{1.324680in}}%
\pgfpathlineto{\pgfqpoint{2.922000in}{1.284032in}}%
\pgfpathlineto{\pgfqpoint{2.946800in}{1.211288in}}%
\pgfpathlineto{\pgfqpoint{2.971600in}{1.159664in}}%
\pgfpathlineto{\pgfqpoint{3.008800in}{1.102794in}}%
\pgfpathlineto{\pgfqpoint{3.021200in}{1.096791in}}%
\pgfpathlineto{\pgfqpoint{3.033600in}{1.101372in}}%
\pgfpathlineto{\pgfqpoint{3.046000in}{1.079921in}}%
\pgfpathlineto{\pgfqpoint{3.058400in}{1.070282in}}%
\pgfpathlineto{\pgfqpoint{3.070800in}{1.072646in}}%
\pgfpathlineto{\pgfqpoint{3.083200in}{1.068967in}}%
\pgfpathlineto{\pgfqpoint{3.095600in}{1.067910in}}%
\pgfpathlineto{\pgfqpoint{3.108000in}{1.061810in}}%
\pgfpathlineto{\pgfqpoint{3.132800in}{1.057938in}}%
\pgfpathlineto{\pgfqpoint{3.157600in}{1.060258in}}%
\pgfpathlineto{\pgfqpoint{3.170000in}{1.059410in}}%
\pgfpathlineto{\pgfqpoint{3.182400in}{1.062355in}}%
\pgfpathlineto{\pgfqpoint{3.194800in}{1.058818in}}%
\pgfpathlineto{\pgfqpoint{3.207200in}{1.064017in}}%
\pgfpathlineto{\pgfqpoint{3.219600in}{1.063510in}}%
\pgfpathlineto{\pgfqpoint{3.232000in}{1.065397in}}%
\pgfpathlineto{\pgfqpoint{3.256800in}{1.072977in}}%
\pgfpathlineto{\pgfqpoint{3.269200in}{1.074415in}}%
\pgfpathlineto{\pgfqpoint{3.281600in}{1.078570in}}%
\pgfpathlineto{\pgfqpoint{3.294000in}{1.081422in}}%
\pgfpathlineto{\pgfqpoint{3.306400in}{1.075846in}}%
\pgfpathlineto{\pgfqpoint{3.318800in}{1.081504in}}%
\pgfpathlineto{\pgfqpoint{3.331200in}{1.082770in}}%
\pgfpathlineto{\pgfqpoint{3.343600in}{1.090947in}}%
\pgfpathlineto{\pgfqpoint{3.380800in}{1.099057in}}%
\pgfpathlineto{\pgfqpoint{3.405600in}{1.106490in}}%
\pgfpathlineto{\pgfqpoint{3.418000in}{1.107384in}}%
\pgfpathlineto{\pgfqpoint{3.430400in}{1.110536in}}%
\pgfpathlineto{\pgfqpoint{3.455200in}{1.123650in}}%
\pgfpathlineto{\pgfqpoint{3.467600in}{1.120113in}}%
\pgfpathlineto{\pgfqpoint{3.504800in}{1.138795in}}%
\pgfpathlineto{\pgfqpoint{3.517200in}{1.134814in}}%
\pgfpathlineto{\pgfqpoint{3.529600in}{1.143116in}}%
\pgfpathlineto{\pgfqpoint{3.542000in}{1.148637in}}%
\pgfpathlineto{\pgfqpoint{3.554400in}{1.155840in}}%
\pgfpathlineto{\pgfqpoint{3.566800in}{1.151752in}}%
\pgfpathlineto{\pgfqpoint{3.579200in}{1.163784in}}%
\pgfpathlineto{\pgfqpoint{3.591600in}{1.162496in}}%
\pgfpathlineto{\pgfqpoint{3.616400in}{1.168536in}}%
\pgfpathlineto{\pgfqpoint{3.628800in}{1.176264in}}%
\pgfpathlineto{\pgfqpoint{3.653600in}{1.175120in}}%
\pgfpathlineto{\pgfqpoint{3.666000in}{1.181872in}}%
\pgfpathlineto{\pgfqpoint{3.678400in}{1.182432in}}%
\pgfpathlineto{\pgfqpoint{3.690800in}{1.190448in}}%
\pgfpathlineto{\pgfqpoint{3.703200in}{1.187408in}}%
\pgfpathlineto{\pgfqpoint{3.740400in}{1.198480in}}%
\pgfpathlineto{\pgfqpoint{3.752800in}{1.200616in}}%
\pgfpathlineto{\pgfqpoint{3.765200in}{1.192864in}}%
\pgfpathlineto{\pgfqpoint{3.790000in}{1.199584in}}%
\pgfpathlineto{\pgfqpoint{3.802400in}{1.197072in}}%
\pgfpathlineto{\pgfqpoint{3.814800in}{1.200880in}}%
\pgfpathlineto{\pgfqpoint{3.839600in}{1.196776in}}%
\pgfpathlineto{\pgfqpoint{3.864400in}{1.195984in}}%
\pgfpathlineto{\pgfqpoint{3.876800in}{1.188208in}}%
\pgfpathlineto{\pgfqpoint{3.889200in}{1.197464in}}%
\pgfpathlineto{\pgfqpoint{3.901600in}{1.199304in}}%
\pgfpathlineto{\pgfqpoint{3.951200in}{1.188536in}}%
\pgfpathlineto{\pgfqpoint{3.963600in}{1.183136in}}%
\pgfpathlineto{\pgfqpoint{3.976000in}{1.185632in}}%
\pgfpathlineto{\pgfqpoint{3.988400in}{1.186008in}}%
\pgfpathlineto{\pgfqpoint{4.000800in}{1.184792in}}%
\pgfpathlineto{\pgfqpoint{4.013200in}{1.175744in}}%
\pgfpathlineto{\pgfqpoint{4.025600in}{1.176264in}}%
\pgfpathlineto{\pgfqpoint{4.038000in}{1.177904in}}%
\pgfpathlineto{\pgfqpoint{4.050400in}{1.171848in}}%
\pgfpathlineto{\pgfqpoint{4.062800in}{1.172392in}}%
\pgfpathlineto{\pgfqpoint{4.075200in}{1.167696in}}%
\pgfpathlineto{\pgfqpoint{4.087600in}{1.168456in}}%
\pgfpathlineto{\pgfqpoint{4.100000in}{1.163008in}}%
\pgfpathlineto{\pgfqpoint{4.112400in}{1.164136in}}%
\pgfpathlineto{\pgfqpoint{4.124800in}{1.155824in}}%
\pgfpathlineto{\pgfqpoint{4.137200in}{1.154272in}}%
\pgfpathlineto{\pgfqpoint{4.149600in}{1.155032in}}%
\pgfpathlineto{\pgfqpoint{4.162000in}{1.153808in}}%
\pgfpathlineto{\pgfqpoint{4.174400in}{1.150944in}}%
\pgfpathlineto{\pgfqpoint{4.186800in}{1.149984in}}%
\pgfpathlineto{\pgfqpoint{4.199200in}{1.147770in}}%
\pgfpathlineto{\pgfqpoint{4.211600in}{1.140789in}}%
\pgfpathlineto{\pgfqpoint{4.224000in}{1.138250in}}%
\pgfpathlineto{\pgfqpoint{4.236400in}{1.138426in}}%
\pgfpathlineto{\pgfqpoint{4.248800in}{1.135121in}}%
\pgfpathlineto{\pgfqpoint{4.261200in}{1.136905in}}%
\pgfpathlineto{\pgfqpoint{4.273600in}{1.129222in}}%
\pgfpathlineto{\pgfqpoint{4.286000in}{1.126374in}}%
\pgfpathlineto{\pgfqpoint{4.310800in}{1.126606in}}%
\pgfpathlineto{\pgfqpoint{4.348000in}{1.118412in}}%
\pgfpathlineto{\pgfqpoint{4.360400in}{1.117622in}}%
\pgfpathlineto{\pgfqpoint{4.385200in}{1.109515in}}%
\pgfpathlineto{\pgfqpoint{4.397600in}{1.108998in}}%
\pgfpathlineto{\pgfqpoint{4.410000in}{1.106697in}}%
\pgfpathlineto{\pgfqpoint{4.422400in}{1.102851in}}%
\pgfpathlineto{\pgfqpoint{4.434800in}{1.097734in}}%
\pgfpathlineto{\pgfqpoint{4.447200in}{1.102070in}}%
\pgfpathlineto{\pgfqpoint{4.459600in}{1.097085in}}%
\pgfpathlineto{\pgfqpoint{4.472000in}{1.096787in}}%
\pgfpathlineto{\pgfqpoint{4.496800in}{1.098658in}}%
\pgfpathlineto{\pgfqpoint{4.509200in}{1.092780in}}%
\pgfpathlineto{\pgfqpoint{4.521600in}{1.098177in}}%
\pgfpathlineto{\pgfqpoint{4.534000in}{1.092990in}}%
\pgfpathlineto{\pgfqpoint{4.546400in}{1.092944in}}%
\pgfpathlineto{\pgfqpoint{4.558800in}{1.097518in}}%
\pgfpathlineto{\pgfqpoint{4.571200in}{1.096010in}}%
\pgfpathlineto{\pgfqpoint{4.583600in}{1.098092in}}%
\pgfpathlineto{\pgfqpoint{4.608400in}{1.098692in}}%
\pgfpathlineto{\pgfqpoint{4.620800in}{1.101321in}}%
\pgfpathlineto{\pgfqpoint{4.633200in}{1.101790in}}%
\pgfpathlineto{\pgfqpoint{4.645600in}{1.104232in}}%
\pgfpathlineto{\pgfqpoint{4.658000in}{1.102274in}}%
\pgfpathlineto{\pgfqpoint{4.670400in}{1.104502in}}%
\pgfpathlineto{\pgfqpoint{4.682800in}{1.108556in}}%
\pgfpathlineto{\pgfqpoint{4.695200in}{1.107040in}}%
\pgfpathlineto{\pgfqpoint{4.720000in}{1.115915in}}%
\pgfpathlineto{\pgfqpoint{4.732400in}{1.113854in}}%
\pgfpathlineto{\pgfqpoint{4.744800in}{1.119067in}}%
\pgfpathlineto{\pgfqpoint{4.757200in}{1.118654in}}%
\pgfpathlineto{\pgfqpoint{4.769600in}{1.120197in}}%
\pgfpathlineto{\pgfqpoint{4.782000in}{1.126132in}}%
\pgfpathlineto{\pgfqpoint{4.794400in}{1.125161in}}%
\pgfpathlineto{\pgfqpoint{4.806800in}{1.131188in}}%
\pgfpathlineto{\pgfqpoint{4.819200in}{1.132588in}}%
\pgfpathlineto{\pgfqpoint{4.831600in}{1.129732in}}%
\pgfpathlineto{\pgfqpoint{4.844000in}{1.134229in}}%
\pgfpathlineto{\pgfqpoint{4.856400in}{1.134186in}}%
\pgfpathlineto{\pgfqpoint{4.868800in}{1.141806in}}%
\pgfpathlineto{\pgfqpoint{4.893600in}{1.146075in}}%
\pgfpathlineto{\pgfqpoint{4.906000in}{1.143191in}}%
\pgfpathlineto{\pgfqpoint{4.930800in}{1.157576in}}%
\pgfpathlineto{\pgfqpoint{4.943200in}{1.157896in}}%
\pgfpathlineto{\pgfqpoint{4.955600in}{1.153888in}}%
\pgfpathlineto{\pgfqpoint{4.968000in}{1.158136in}}%
\pgfpathlineto{\pgfqpoint{5.005200in}{1.162048in}}%
\pgfpathlineto{\pgfqpoint{5.017600in}{1.166112in}}%
\pgfpathlineto{\pgfqpoint{5.030000in}{1.164448in}}%
\pgfpathlineto{\pgfqpoint{5.042400in}{1.167920in}}%
\pgfpathlineto{\pgfqpoint{5.054800in}{1.165696in}}%
\pgfpathlineto{\pgfqpoint{5.067200in}{1.165184in}}%
\pgfpathlineto{\pgfqpoint{5.079600in}{1.168800in}}%
\pgfpathlineto{\pgfqpoint{5.116800in}{1.169920in}}%
\pgfpathlineto{\pgfqpoint{5.129200in}{1.174072in}}%
\pgfpathlineto{\pgfqpoint{5.141600in}{1.170832in}}%
\pgfpathlineto{\pgfqpoint{5.154000in}{1.175344in}}%
\pgfpathlineto{\pgfqpoint{5.166400in}{1.175896in}}%
\pgfpathlineto{\pgfqpoint{5.191200in}{1.180096in}}%
\pgfpathlineto{\pgfqpoint{5.216000in}{1.177176in}}%
\pgfpathlineto{\pgfqpoint{5.228400in}{1.181144in}}%
\pgfpathlineto{\pgfqpoint{5.240800in}{1.179296in}}%
\pgfpathlineto{\pgfqpoint{5.253200in}{1.181608in}}%
\pgfpathlineto{\pgfqpoint{5.265600in}{1.176320in}}%
\pgfpathlineto{\pgfqpoint{5.278000in}{1.181488in}}%
\pgfpathlineto{\pgfqpoint{5.302800in}{1.179960in}}%
\pgfpathlineto{\pgfqpoint{5.315200in}{1.178256in}}%
\pgfpathlineto{\pgfqpoint{5.327600in}{1.177992in}}%
\pgfpathlineto{\pgfqpoint{5.340000in}{1.180888in}}%
\pgfpathlineto{\pgfqpoint{5.352400in}{1.175720in}}%
\pgfpathlineto{\pgfqpoint{5.364800in}{1.177224in}}%
\pgfpathlineto{\pgfqpoint{5.377200in}{1.175736in}}%
\pgfpathlineto{\pgfqpoint{5.402000in}{1.174256in}}%
\pgfpathlineto{\pgfqpoint{5.414400in}{1.173040in}}%
\pgfpathlineto{\pgfqpoint{5.426800in}{1.170456in}}%
\pgfpathlineto{\pgfqpoint{5.439200in}{1.172696in}}%
\pgfpathlineto{\pgfqpoint{5.451600in}{1.169296in}}%
\pgfpathlineto{\pgfqpoint{5.476400in}{1.171232in}}%
\pgfpathlineto{\pgfqpoint{5.538400in}{1.164456in}}%
\pgfpathlineto{\pgfqpoint{5.550800in}{1.165704in}}%
\pgfpathlineto{\pgfqpoint{5.563200in}{1.159800in}}%
\pgfpathlineto{\pgfqpoint{5.575600in}{1.163512in}}%
\pgfpathlineto{\pgfqpoint{5.588000in}{1.160568in}}%
\pgfpathlineto{\pgfqpoint{5.600400in}{1.160496in}}%
\pgfpathlineto{\pgfqpoint{5.612800in}{1.156576in}}%
\pgfpathlineto{\pgfqpoint{5.625200in}{1.158232in}}%
\pgfpathlineto{\pgfqpoint{5.674800in}{1.153104in}}%
\pgfpathlineto{\pgfqpoint{5.687200in}{1.152728in}}%
\pgfpathlineto{\pgfqpoint{5.699600in}{1.150552in}}%
\pgfpathlineto{\pgfqpoint{5.712000in}{1.151872in}}%
\pgfpathlineto{\pgfqpoint{5.724400in}{1.148903in}}%
\pgfpathlineto{\pgfqpoint{5.736800in}{1.150776in}}%
\pgfpathlineto{\pgfqpoint{5.786400in}{1.144028in}}%
\pgfpathlineto{\pgfqpoint{5.798800in}{1.146785in}}%
\pgfpathlineto{\pgfqpoint{5.811200in}{1.146771in}}%
\pgfpathlineto{\pgfqpoint{5.836000in}{1.143955in}}%
\pgfpathlineto{\pgfqpoint{5.848400in}{1.143442in}}%
\pgfpathlineto{\pgfqpoint{5.860800in}{1.140650in}}%
\pgfpathlineto{\pgfqpoint{5.885600in}{1.141678in}}%
\pgfpathlineto{\pgfqpoint{5.898000in}{1.139192in}}%
\pgfpathlineto{\pgfqpoint{5.910400in}{1.138070in}}%
\pgfpathlineto{\pgfqpoint{5.947600in}{1.139078in}}%
\pgfpathlineto{\pgfqpoint{5.960000in}{1.136734in}}%
\pgfpathlineto{\pgfqpoint{5.972400in}{1.141408in}}%
\pgfpathlineto{\pgfqpoint{5.984800in}{1.139166in}}%
\pgfpathlineto{\pgfqpoint{5.997200in}{1.141811in}}%
\pgfpathlineto{\pgfqpoint{6.009600in}{1.138178in}}%
\pgfpathlineto{\pgfqpoint{6.022000in}{1.139374in}}%
\pgfpathlineto{\pgfqpoint{6.034400in}{1.136849in}}%
\pgfpathlineto{\pgfqpoint{6.059200in}{1.138397in}}%
\pgfpathlineto{\pgfqpoint{6.071600in}{1.136427in}}%
\pgfpathlineto{\pgfqpoint{6.096400in}{1.141136in}}%
\pgfpathlineto{\pgfqpoint{6.108800in}{1.138822in}}%
\pgfpathlineto{\pgfqpoint{6.121200in}{1.142399in}}%
\pgfpathlineto{\pgfqpoint{6.146000in}{1.139172in}}%
\pgfpathlineto{\pgfqpoint{6.158400in}{1.136557in}}%
\pgfpathlineto{\pgfqpoint{6.170800in}{1.140678in}}%
\pgfpathlineto{\pgfqpoint{6.183200in}{1.141116in}}%
\pgfpathlineto{\pgfqpoint{6.195600in}{1.143628in}}%
\pgfpathlineto{\pgfqpoint{6.220400in}{1.141944in}}%
\pgfpathlineto{\pgfqpoint{6.232800in}{1.140911in}}%
\pgfpathlineto{\pgfqpoint{6.245200in}{1.141946in}}%
\pgfpathlineto{\pgfqpoint{6.257600in}{1.144794in}}%
\pgfpathlineto{\pgfqpoint{6.270000in}{1.142543in}}%
\pgfpathlineto{\pgfqpoint{6.307200in}{1.144843in}}%
\pgfpathlineto{\pgfqpoint{6.319600in}{1.143163in}}%
\pgfpathlineto{\pgfqpoint{6.356800in}{1.144370in}}%
\pgfpathlineto{\pgfqpoint{6.369200in}{1.146700in}}%
\pgfpathlineto{\pgfqpoint{6.381600in}{1.144902in}}%
\pgfpathlineto{\pgfqpoint{6.394000in}{1.148912in}}%
\pgfpathlineto{\pgfqpoint{6.406400in}{1.148762in}}%
\pgfpathlineto{\pgfqpoint{6.418800in}{1.145104in}}%
\pgfpathlineto{\pgfqpoint{6.431200in}{1.145996in}}%
\pgfpathlineto{\pgfqpoint{6.443600in}{1.150544in}}%
\pgfpathlineto{\pgfqpoint{6.456000in}{1.148291in}}%
\pgfpathlineto{\pgfqpoint{6.480800in}{1.148051in}}%
\pgfpathlineto{\pgfqpoint{6.493200in}{1.148224in}}%
\pgfpathlineto{\pgfqpoint{6.505600in}{1.150304in}}%
\pgfpathlineto{\pgfqpoint{6.518000in}{1.148755in}}%
\pgfpathlineto{\pgfqpoint{6.530400in}{1.149467in}}%
\pgfpathlineto{\pgfqpoint{6.542800in}{1.148683in}}%
\pgfpathlineto{\pgfqpoint{6.580000in}{1.151544in}}%
\pgfpathlineto{\pgfqpoint{6.592400in}{1.148333in}}%
\pgfpathlineto{\pgfqpoint{6.604800in}{1.147322in}}%
\pgfpathlineto{\pgfqpoint{6.617200in}{1.151280in}}%
\pgfpathlineto{\pgfqpoint{6.642000in}{1.152352in}}%
\pgfpathlineto{\pgfqpoint{6.654400in}{1.150720in}}%
\pgfpathlineto{\pgfqpoint{6.691600in}{1.153952in}}%
\pgfpathlineto{\pgfqpoint{6.704000in}{1.151312in}}%
\pgfpathlineto{\pgfqpoint{6.704000in}{1.151312in}}%
\pgfusepath{stroke}%
\end{pgfscope}%
\begin{pgfscope}%
\pgfpathrectangle{\pgfqpoint{1.000000in}{0.350000in}}{\pgfqpoint{6.200000in}{2.800000in}} %
\pgfusepath{clip}%
\pgfsetrectcap%
\pgfsetroundjoin%
\pgfsetlinewidth{1.003750pt}%
\definecolor{currentstroke}{rgb}{0.000000,0.500000,0.000000}%
\pgfsetstrokecolor{currentstroke}%
\pgfsetdash{}{0pt}%
\pgfpathmoveto{\pgfqpoint{1.000000in}{0.350000in}}%
\pgfpathlineto{\pgfqpoint{2.500400in}{0.350872in}}%
\pgfpathlineto{\pgfqpoint{2.512800in}{0.352720in}}%
\pgfpathlineto{\pgfqpoint{2.525200in}{0.360123in}}%
\pgfpathlineto{\pgfqpoint{2.537600in}{0.376705in}}%
\pgfpathlineto{\pgfqpoint{2.550000in}{0.412981in}}%
\pgfpathlineto{\pgfqpoint{2.562400in}{0.493846in}}%
\pgfpathlineto{\pgfqpoint{2.574800in}{0.610342in}}%
\pgfpathlineto{\pgfqpoint{2.587200in}{0.782591in}}%
\pgfpathlineto{\pgfqpoint{2.599600in}{1.017948in}}%
\pgfpathlineto{\pgfqpoint{2.612000in}{1.297032in}}%
\pgfpathlineto{\pgfqpoint{2.649200in}{2.220072in}}%
\pgfpathlineto{\pgfqpoint{2.661600in}{2.453712in}}%
\pgfpathlineto{\pgfqpoint{2.674000in}{2.629112in}}%
\pgfpathlineto{\pgfqpoint{2.686400in}{2.783968in}}%
\pgfpathlineto{\pgfqpoint{2.698800in}{2.807664in}}%
\pgfpathlineto{\pgfqpoint{2.711200in}{2.810424in}}%
\pgfpathlineto{\pgfqpoint{2.723600in}{2.739184in}}%
\pgfpathlineto{\pgfqpoint{2.748400in}{2.518784in}}%
\pgfpathlineto{\pgfqpoint{2.798000in}{1.968040in}}%
\pgfpathlineto{\pgfqpoint{2.810400in}{1.817800in}}%
\pgfpathlineto{\pgfqpoint{2.835200in}{1.587016in}}%
\pgfpathlineto{\pgfqpoint{2.860000in}{1.409872in}}%
\pgfpathlineto{\pgfqpoint{2.872400in}{1.337568in}}%
\pgfpathlineto{\pgfqpoint{2.884800in}{1.272368in}}%
\pgfpathlineto{\pgfqpoint{2.909600in}{1.178328in}}%
\pgfpathlineto{\pgfqpoint{2.922000in}{1.138915in}}%
\pgfpathlineto{\pgfqpoint{2.934400in}{1.106167in}}%
\pgfpathlineto{\pgfqpoint{2.946800in}{1.079285in}}%
\pgfpathlineto{\pgfqpoint{2.959200in}{1.057297in}}%
\pgfpathlineto{\pgfqpoint{2.971600in}{1.046064in}}%
\pgfpathlineto{\pgfqpoint{2.984000in}{1.025803in}}%
\pgfpathlineto{\pgfqpoint{2.996400in}{1.020748in}}%
\pgfpathlineto{\pgfqpoint{3.008800in}{1.008582in}}%
\pgfpathlineto{\pgfqpoint{3.021200in}{1.005598in}}%
\pgfpathlineto{\pgfqpoint{3.033600in}{1.004106in}}%
\pgfpathlineto{\pgfqpoint{3.046000in}{1.000711in}}%
\pgfpathlineto{\pgfqpoint{3.058400in}{0.992346in}}%
\pgfpathlineto{\pgfqpoint{3.070800in}{0.992761in}}%
\pgfpathlineto{\pgfqpoint{3.083200in}{0.995970in}}%
\pgfpathlineto{\pgfqpoint{3.095600in}{1.004021in}}%
\pgfpathlineto{\pgfqpoint{3.108000in}{1.003232in}}%
\pgfpathlineto{\pgfqpoint{3.120400in}{1.007820in}}%
\pgfpathlineto{\pgfqpoint{3.132800in}{1.006679in}}%
\pgfpathlineto{\pgfqpoint{3.145200in}{1.011146in}}%
\pgfpathlineto{\pgfqpoint{3.157600in}{1.018055in}}%
\pgfpathlineto{\pgfqpoint{3.170000in}{1.017253in}}%
\pgfpathlineto{\pgfqpoint{3.194800in}{1.027322in}}%
\pgfpathlineto{\pgfqpoint{3.207200in}{1.031023in}}%
\pgfpathlineto{\pgfqpoint{3.219600in}{1.038656in}}%
\pgfpathlineto{\pgfqpoint{3.232000in}{1.033522in}}%
\pgfpathlineto{\pgfqpoint{3.244400in}{1.033419in}}%
\pgfpathlineto{\pgfqpoint{3.256800in}{1.055454in}}%
\pgfpathlineto{\pgfqpoint{3.269200in}{1.052287in}}%
\pgfpathlineto{\pgfqpoint{3.281600in}{1.059390in}}%
\pgfpathlineto{\pgfqpoint{3.294000in}{1.061436in}}%
\pgfpathlineto{\pgfqpoint{3.306400in}{1.069914in}}%
\pgfpathlineto{\pgfqpoint{3.318800in}{1.073108in}}%
\pgfpathlineto{\pgfqpoint{3.331200in}{1.079080in}}%
\pgfpathlineto{\pgfqpoint{3.343600in}{1.089822in}}%
\pgfpathlineto{\pgfqpoint{3.356000in}{1.090438in}}%
\pgfpathlineto{\pgfqpoint{3.368400in}{1.098053in}}%
\pgfpathlineto{\pgfqpoint{3.380800in}{1.101935in}}%
\pgfpathlineto{\pgfqpoint{3.393200in}{1.108361in}}%
\pgfpathlineto{\pgfqpoint{3.405600in}{1.117440in}}%
\pgfpathlineto{\pgfqpoint{3.418000in}{1.114439in}}%
\pgfpathlineto{\pgfqpoint{3.430400in}{1.131058in}}%
\pgfpathlineto{\pgfqpoint{3.442800in}{1.128233in}}%
\pgfpathlineto{\pgfqpoint{3.467600in}{1.144340in}}%
\pgfpathlineto{\pgfqpoint{3.480000in}{1.145658in}}%
\pgfpathlineto{\pgfqpoint{3.504800in}{1.162352in}}%
\pgfpathlineto{\pgfqpoint{3.529600in}{1.172000in}}%
\pgfpathlineto{\pgfqpoint{3.542000in}{1.172040in}}%
\pgfpathlineto{\pgfqpoint{3.554400in}{1.180240in}}%
\pgfpathlineto{\pgfqpoint{3.566800in}{1.180856in}}%
\pgfpathlineto{\pgfqpoint{3.579200in}{1.192952in}}%
\pgfpathlineto{\pgfqpoint{3.591600in}{1.197096in}}%
\pgfpathlineto{\pgfqpoint{3.604000in}{1.206088in}}%
\pgfpathlineto{\pgfqpoint{3.616400in}{1.208064in}}%
\pgfpathlineto{\pgfqpoint{3.628800in}{1.205208in}}%
\pgfpathlineto{\pgfqpoint{3.641200in}{1.216168in}}%
\pgfpathlineto{\pgfqpoint{3.653600in}{1.215024in}}%
\pgfpathlineto{\pgfqpoint{3.666000in}{1.219000in}}%
\pgfpathlineto{\pgfqpoint{3.678400in}{1.224192in}}%
\pgfpathlineto{\pgfqpoint{3.728000in}{1.232568in}}%
\pgfpathlineto{\pgfqpoint{3.740400in}{1.235056in}}%
\pgfpathlineto{\pgfqpoint{3.765200in}{1.231728in}}%
\pgfpathlineto{\pgfqpoint{3.777600in}{1.237024in}}%
\pgfpathlineto{\pgfqpoint{3.790000in}{1.240360in}}%
\pgfpathlineto{\pgfqpoint{3.802400in}{1.240048in}}%
\pgfpathlineto{\pgfqpoint{3.814800in}{1.241200in}}%
\pgfpathlineto{\pgfqpoint{3.827200in}{1.234896in}}%
\pgfpathlineto{\pgfqpoint{3.839600in}{1.238216in}}%
\pgfpathlineto{\pgfqpoint{3.852000in}{1.231632in}}%
\pgfpathlineto{\pgfqpoint{3.864400in}{1.229616in}}%
\pgfpathlineto{\pgfqpoint{3.876800in}{1.233808in}}%
\pgfpathlineto{\pgfqpoint{3.889200in}{1.229480in}}%
\pgfpathlineto{\pgfqpoint{3.901600in}{1.226408in}}%
\pgfpathlineto{\pgfqpoint{3.914000in}{1.220840in}}%
\pgfpathlineto{\pgfqpoint{3.926400in}{1.226512in}}%
\pgfpathlineto{\pgfqpoint{3.938800in}{1.216592in}}%
\pgfpathlineto{\pgfqpoint{3.951200in}{1.217616in}}%
\pgfpathlineto{\pgfqpoint{3.963600in}{1.214640in}}%
\pgfpathlineto{\pgfqpoint{3.976000in}{1.217296in}}%
\pgfpathlineto{\pgfqpoint{3.988400in}{1.207112in}}%
\pgfpathlineto{\pgfqpoint{4.000800in}{1.205088in}}%
\pgfpathlineto{\pgfqpoint{4.013200in}{1.197336in}}%
\pgfpathlineto{\pgfqpoint{4.025600in}{1.200256in}}%
\pgfpathlineto{\pgfqpoint{4.038000in}{1.193656in}}%
\pgfpathlineto{\pgfqpoint{4.050400in}{1.190200in}}%
\pgfpathlineto{\pgfqpoint{4.062800in}{1.183832in}}%
\pgfpathlineto{\pgfqpoint{4.087600in}{1.175872in}}%
\pgfpathlineto{\pgfqpoint{4.100000in}{1.169864in}}%
\pgfpathlineto{\pgfqpoint{4.112400in}{1.167496in}}%
\pgfpathlineto{\pgfqpoint{4.137200in}{1.155352in}}%
\pgfpathlineto{\pgfqpoint{4.149600in}{1.154520in}}%
\pgfpathlineto{\pgfqpoint{4.162000in}{1.144930in}}%
\pgfpathlineto{\pgfqpoint{4.174400in}{1.143098in}}%
\pgfpathlineto{\pgfqpoint{4.186800in}{1.144101in}}%
\pgfpathlineto{\pgfqpoint{4.199200in}{1.137533in}}%
\pgfpathlineto{\pgfqpoint{4.211600in}{1.133686in}}%
\pgfpathlineto{\pgfqpoint{4.236400in}{1.124146in}}%
\pgfpathlineto{\pgfqpoint{4.248800in}{1.125185in}}%
\pgfpathlineto{\pgfqpoint{4.273600in}{1.115172in}}%
\pgfpathlineto{\pgfqpoint{4.286000in}{1.113672in}}%
\pgfpathlineto{\pgfqpoint{4.298400in}{1.108087in}}%
\pgfpathlineto{\pgfqpoint{4.310800in}{1.104463in}}%
\pgfpathlineto{\pgfqpoint{4.323200in}{1.098729in}}%
\pgfpathlineto{\pgfqpoint{4.335600in}{1.099970in}}%
\pgfpathlineto{\pgfqpoint{4.348000in}{1.090779in}}%
\pgfpathlineto{\pgfqpoint{4.372800in}{1.087910in}}%
\pgfpathlineto{\pgfqpoint{4.385200in}{1.082652in}}%
\pgfpathlineto{\pgfqpoint{4.447200in}{1.073190in}}%
\pgfpathlineto{\pgfqpoint{4.459600in}{1.073275in}}%
\pgfpathlineto{\pgfqpoint{4.472000in}{1.069835in}}%
\pgfpathlineto{\pgfqpoint{4.496800in}{1.071303in}}%
\pgfpathlineto{\pgfqpoint{4.509200in}{1.064853in}}%
\pgfpathlineto{\pgfqpoint{4.521600in}{1.070933in}}%
\pgfpathlineto{\pgfqpoint{4.534000in}{1.068076in}}%
\pgfpathlineto{\pgfqpoint{4.546400in}{1.071161in}}%
\pgfpathlineto{\pgfqpoint{4.558800in}{1.071260in}}%
\pgfpathlineto{\pgfqpoint{4.583600in}{1.077916in}}%
\pgfpathlineto{\pgfqpoint{4.596000in}{1.082881in}}%
\pgfpathlineto{\pgfqpoint{4.608400in}{1.082369in}}%
\pgfpathlineto{\pgfqpoint{4.633200in}{1.089040in}}%
\pgfpathlineto{\pgfqpoint{4.645600in}{1.087387in}}%
\pgfpathlineto{\pgfqpoint{4.658000in}{1.093019in}}%
\pgfpathlineto{\pgfqpoint{4.670400in}{1.097329in}}%
\pgfpathlineto{\pgfqpoint{4.682800in}{1.097630in}}%
\pgfpathlineto{\pgfqpoint{4.695200in}{1.099874in}}%
\pgfpathlineto{\pgfqpoint{4.707600in}{1.104435in}}%
\pgfpathlineto{\pgfqpoint{4.720000in}{1.105465in}}%
\pgfpathlineto{\pgfqpoint{4.744800in}{1.115285in}}%
\pgfpathlineto{\pgfqpoint{4.757200in}{1.114692in}}%
\pgfpathlineto{\pgfqpoint{4.769600in}{1.116460in}}%
\pgfpathlineto{\pgfqpoint{4.782000in}{1.121412in}}%
\pgfpathlineto{\pgfqpoint{4.806800in}{1.127609in}}%
\pgfpathlineto{\pgfqpoint{4.819200in}{1.130264in}}%
\pgfpathlineto{\pgfqpoint{4.831600in}{1.135486in}}%
\pgfpathlineto{\pgfqpoint{4.844000in}{1.133481in}}%
\pgfpathlineto{\pgfqpoint{4.856400in}{1.139556in}}%
\pgfpathlineto{\pgfqpoint{4.881200in}{1.147874in}}%
\pgfpathlineto{\pgfqpoint{4.893600in}{1.150768in}}%
\pgfpathlineto{\pgfqpoint{4.906000in}{1.150696in}}%
\pgfpathlineto{\pgfqpoint{4.918400in}{1.152944in}}%
\pgfpathlineto{\pgfqpoint{4.930800in}{1.152944in}}%
\pgfpathlineto{\pgfqpoint{4.943200in}{1.160512in}}%
\pgfpathlineto{\pgfqpoint{4.955600in}{1.161256in}}%
\pgfpathlineto{\pgfqpoint{4.968000in}{1.164712in}}%
\pgfpathlineto{\pgfqpoint{4.980400in}{1.165176in}}%
\pgfpathlineto{\pgfqpoint{4.992800in}{1.167152in}}%
\pgfpathlineto{\pgfqpoint{5.005200in}{1.163168in}}%
\pgfpathlineto{\pgfqpoint{5.017600in}{1.171544in}}%
\pgfpathlineto{\pgfqpoint{5.030000in}{1.172896in}}%
\pgfpathlineto{\pgfqpoint{5.042400in}{1.177424in}}%
\pgfpathlineto{\pgfqpoint{5.067200in}{1.176776in}}%
\pgfpathlineto{\pgfqpoint{5.079600in}{1.176944in}}%
\pgfpathlineto{\pgfqpoint{5.092000in}{1.180400in}}%
\pgfpathlineto{\pgfqpoint{5.104400in}{1.180520in}}%
\pgfpathlineto{\pgfqpoint{5.116800in}{1.186216in}}%
\pgfpathlineto{\pgfqpoint{5.129200in}{1.184232in}}%
\pgfpathlineto{\pgfqpoint{5.141600in}{1.184344in}}%
\pgfpathlineto{\pgfqpoint{5.154000in}{1.182696in}}%
\pgfpathlineto{\pgfqpoint{5.166400in}{1.183872in}}%
\pgfpathlineto{\pgfqpoint{5.191200in}{1.184512in}}%
\pgfpathlineto{\pgfqpoint{5.203600in}{1.185208in}}%
\pgfpathlineto{\pgfqpoint{5.216000in}{1.182616in}}%
\pgfpathlineto{\pgfqpoint{5.228400in}{1.186832in}}%
\pgfpathlineto{\pgfqpoint{5.240800in}{1.184944in}}%
\pgfpathlineto{\pgfqpoint{5.253200in}{1.188776in}}%
\pgfpathlineto{\pgfqpoint{5.265600in}{1.189376in}}%
\pgfpathlineto{\pgfqpoint{5.278000in}{1.185424in}}%
\pgfpathlineto{\pgfqpoint{5.290400in}{1.188336in}}%
\pgfpathlineto{\pgfqpoint{5.302800in}{1.183352in}}%
\pgfpathlineto{\pgfqpoint{5.315200in}{1.187904in}}%
\pgfpathlineto{\pgfqpoint{5.327600in}{1.181848in}}%
\pgfpathlineto{\pgfqpoint{5.340000in}{1.183472in}}%
\pgfpathlineto{\pgfqpoint{5.352400in}{1.183504in}}%
\pgfpathlineto{\pgfqpoint{5.364800in}{1.181856in}}%
\pgfpathlineto{\pgfqpoint{5.377200in}{1.182072in}}%
\pgfpathlineto{\pgfqpoint{5.389600in}{1.177448in}}%
\pgfpathlineto{\pgfqpoint{5.414400in}{1.175136in}}%
\pgfpathlineto{\pgfqpoint{5.426800in}{1.175424in}}%
\pgfpathlineto{\pgfqpoint{5.439200in}{1.172728in}}%
\pgfpathlineto{\pgfqpoint{5.451600in}{1.173872in}}%
\pgfpathlineto{\pgfqpoint{5.464000in}{1.169728in}}%
\pgfpathlineto{\pgfqpoint{5.476400in}{1.174608in}}%
\pgfpathlineto{\pgfqpoint{5.488800in}{1.169144in}}%
\pgfpathlineto{\pgfqpoint{5.501200in}{1.169360in}}%
\pgfpathlineto{\pgfqpoint{5.513600in}{1.165456in}}%
\pgfpathlineto{\pgfqpoint{5.526000in}{1.163760in}}%
\pgfpathlineto{\pgfqpoint{5.550800in}{1.166072in}}%
\pgfpathlineto{\pgfqpoint{5.563200in}{1.160776in}}%
\pgfpathlineto{\pgfqpoint{5.588000in}{1.159720in}}%
\pgfpathlineto{\pgfqpoint{5.600400in}{1.155648in}}%
\pgfpathlineto{\pgfqpoint{5.612800in}{1.156448in}}%
\pgfpathlineto{\pgfqpoint{5.625200in}{1.158400in}}%
\pgfpathlineto{\pgfqpoint{5.637600in}{1.154120in}}%
\pgfpathlineto{\pgfqpoint{5.650000in}{1.152504in}}%
\pgfpathlineto{\pgfqpoint{5.662400in}{1.152792in}}%
\pgfpathlineto{\pgfqpoint{5.674800in}{1.147836in}}%
\pgfpathlineto{\pgfqpoint{5.699600in}{1.149461in}}%
\pgfpathlineto{\pgfqpoint{5.724400in}{1.146736in}}%
\pgfpathlineto{\pgfqpoint{5.736800in}{1.142926in}}%
\pgfpathlineto{\pgfqpoint{5.749200in}{1.142130in}}%
\pgfpathlineto{\pgfqpoint{5.761600in}{1.143359in}}%
\pgfpathlineto{\pgfqpoint{5.774000in}{1.142356in}}%
\pgfpathlineto{\pgfqpoint{5.798800in}{1.143210in}}%
\pgfpathlineto{\pgfqpoint{5.811200in}{1.139153in}}%
\pgfpathlineto{\pgfqpoint{5.848400in}{1.138659in}}%
\pgfpathlineto{\pgfqpoint{5.860800in}{1.142925in}}%
\pgfpathlineto{\pgfqpoint{5.873200in}{1.138745in}}%
\pgfpathlineto{\pgfqpoint{5.885600in}{1.142633in}}%
\pgfpathlineto{\pgfqpoint{5.898000in}{1.139682in}}%
\pgfpathlineto{\pgfqpoint{5.910400in}{1.139613in}}%
\pgfpathlineto{\pgfqpoint{5.922800in}{1.137184in}}%
\pgfpathlineto{\pgfqpoint{5.960000in}{1.136831in}}%
\pgfpathlineto{\pgfqpoint{5.972400in}{1.139958in}}%
\pgfpathlineto{\pgfqpoint{5.984800in}{1.137944in}}%
\pgfpathlineto{\pgfqpoint{6.022000in}{1.139514in}}%
\pgfpathlineto{\pgfqpoint{6.034400in}{1.137824in}}%
\pgfpathlineto{\pgfqpoint{6.046800in}{1.138607in}}%
\pgfpathlineto{\pgfqpoint{6.059200in}{1.140914in}}%
\pgfpathlineto{\pgfqpoint{6.071600in}{1.138831in}}%
\pgfpathlineto{\pgfqpoint{6.084000in}{1.140198in}}%
\pgfpathlineto{\pgfqpoint{6.096400in}{1.138636in}}%
\pgfpathlineto{\pgfqpoint{6.121200in}{1.139066in}}%
\pgfpathlineto{\pgfqpoint{6.146000in}{1.142501in}}%
\pgfpathlineto{\pgfqpoint{6.158400in}{1.142444in}}%
\pgfpathlineto{\pgfqpoint{6.170800in}{1.143513in}}%
\pgfpathlineto{\pgfqpoint{6.195600in}{1.141166in}}%
\pgfpathlineto{\pgfqpoint{6.232800in}{1.145003in}}%
\pgfpathlineto{\pgfqpoint{6.245200in}{1.143338in}}%
\pgfpathlineto{\pgfqpoint{6.257600in}{1.144828in}}%
\pgfpathlineto{\pgfqpoint{6.270000in}{1.143996in}}%
\pgfpathlineto{\pgfqpoint{6.307200in}{1.145100in}}%
\pgfpathlineto{\pgfqpoint{6.319600in}{1.148116in}}%
\pgfpathlineto{\pgfqpoint{6.344400in}{1.147170in}}%
\pgfpathlineto{\pgfqpoint{6.356800in}{1.151280in}}%
\pgfpathlineto{\pgfqpoint{6.381600in}{1.147877in}}%
\pgfpathlineto{\pgfqpoint{6.394000in}{1.147500in}}%
\pgfpathlineto{\pgfqpoint{6.406400in}{1.148243in}}%
\pgfpathlineto{\pgfqpoint{6.418800in}{1.146442in}}%
\pgfpathlineto{\pgfqpoint{6.431200in}{1.150280in}}%
\pgfpathlineto{\pgfqpoint{6.443600in}{1.151840in}}%
\pgfpathlineto{\pgfqpoint{6.456000in}{1.148431in}}%
\pgfpathlineto{\pgfqpoint{6.468400in}{1.149596in}}%
\pgfpathlineto{\pgfqpoint{6.480800in}{1.148087in}}%
\pgfpathlineto{\pgfqpoint{6.493200in}{1.150560in}}%
\pgfpathlineto{\pgfqpoint{6.505600in}{1.148707in}}%
\pgfpathlineto{\pgfqpoint{6.518000in}{1.150120in}}%
\pgfpathlineto{\pgfqpoint{6.530400in}{1.150144in}}%
\pgfpathlineto{\pgfqpoint{6.542800in}{1.152328in}}%
\pgfpathlineto{\pgfqpoint{6.555200in}{1.150136in}}%
\pgfpathlineto{\pgfqpoint{6.567600in}{1.149126in}}%
\pgfpathlineto{\pgfqpoint{6.580000in}{1.153344in}}%
\pgfpathlineto{\pgfqpoint{6.592400in}{1.151968in}}%
\pgfpathlineto{\pgfqpoint{6.604800in}{1.149290in}}%
\pgfpathlineto{\pgfqpoint{6.617200in}{1.151376in}}%
\pgfpathlineto{\pgfqpoint{6.629600in}{1.149829in}}%
\pgfpathlineto{\pgfqpoint{6.642000in}{1.151632in}}%
\pgfpathlineto{\pgfqpoint{6.654400in}{1.149699in}}%
\pgfpathlineto{\pgfqpoint{6.679200in}{1.149438in}}%
\pgfpathlineto{\pgfqpoint{6.691600in}{1.151512in}}%
\pgfpathlineto{\pgfqpoint{6.704000in}{1.150944in}}%
\pgfpathlineto{\pgfqpoint{6.704000in}{1.150944in}}%
\pgfusepath{stroke}%
\end{pgfscope}%
\begin{pgfscope}%
\pgfpathrectangle{\pgfqpoint{1.000000in}{0.350000in}}{\pgfqpoint{6.200000in}{2.800000in}} %
\pgfusepath{clip}%
\pgfsetrectcap%
\pgfsetroundjoin%
\pgfsetlinewidth{1.003750pt}%
\definecolor{currentstroke}{rgb}{1.000000,0.000000,0.000000}%
\pgfsetstrokecolor{currentstroke}%
\pgfsetdash{}{0pt}%
\pgfpathmoveto{\pgfqpoint{1.000000in}{0.350000in}}%
\pgfpathlineto{\pgfqpoint{2.500400in}{0.350824in}}%
\pgfpathlineto{\pgfqpoint{2.512800in}{0.352553in}}%
\pgfpathlineto{\pgfqpoint{2.525200in}{0.358864in}}%
\pgfpathlineto{\pgfqpoint{2.537600in}{0.372385in}}%
\pgfpathlineto{\pgfqpoint{2.550000in}{0.401380in}}%
\pgfpathlineto{\pgfqpoint{2.562400in}{0.455106in}}%
\pgfpathlineto{\pgfqpoint{2.574800in}{0.549347in}}%
\pgfpathlineto{\pgfqpoint{2.587200in}{0.675052in}}%
\pgfpathlineto{\pgfqpoint{2.599600in}{0.844681in}}%
\pgfpathlineto{\pgfqpoint{2.612000in}{1.058378in}}%
\pgfpathlineto{\pgfqpoint{2.649200in}{1.804592in}}%
\pgfpathlineto{\pgfqpoint{2.661600in}{2.021624in}}%
\pgfpathlineto{\pgfqpoint{2.674000in}{2.216936in}}%
\pgfpathlineto{\pgfqpoint{2.686400in}{2.353944in}}%
\pgfpathlineto{\pgfqpoint{2.698800in}{2.461632in}}%
\pgfpathlineto{\pgfqpoint{2.711200in}{2.510624in}}%
\pgfpathlineto{\pgfqpoint{2.723600in}{2.513168in}}%
\pgfpathlineto{\pgfqpoint{2.736000in}{2.474352in}}%
\pgfpathlineto{\pgfqpoint{2.748400in}{2.420424in}}%
\pgfpathlineto{\pgfqpoint{2.760800in}{2.343728in}}%
\pgfpathlineto{\pgfqpoint{2.773200in}{2.257952in}}%
\pgfpathlineto{\pgfqpoint{2.785600in}{2.163032in}}%
\pgfpathlineto{\pgfqpoint{2.810400in}{1.959928in}}%
\pgfpathlineto{\pgfqpoint{2.847600in}{1.689976in}}%
\pgfpathlineto{\pgfqpoint{2.884800in}{1.491280in}}%
\pgfpathlineto{\pgfqpoint{2.909600in}{1.391208in}}%
\pgfpathlineto{\pgfqpoint{2.922000in}{1.356512in}}%
\pgfpathlineto{\pgfqpoint{2.934400in}{1.311336in}}%
\pgfpathlineto{\pgfqpoint{2.946800in}{1.286496in}}%
\pgfpathlineto{\pgfqpoint{2.959200in}{1.248712in}}%
\pgfpathlineto{\pgfqpoint{2.971600in}{1.224728in}}%
\pgfpathlineto{\pgfqpoint{2.984000in}{1.211072in}}%
\pgfpathlineto{\pgfqpoint{2.996400in}{1.192344in}}%
\pgfpathlineto{\pgfqpoint{3.008800in}{1.181200in}}%
\pgfpathlineto{\pgfqpoint{3.021200in}{1.159784in}}%
\pgfpathlineto{\pgfqpoint{3.033600in}{1.153944in}}%
\pgfpathlineto{\pgfqpoint{3.046000in}{1.134162in}}%
\pgfpathlineto{\pgfqpoint{3.058400in}{1.132347in}}%
\pgfpathlineto{\pgfqpoint{3.070800in}{1.125364in}}%
\pgfpathlineto{\pgfqpoint{3.083200in}{1.119740in}}%
\pgfpathlineto{\pgfqpoint{3.095600in}{1.108876in}}%
\pgfpathlineto{\pgfqpoint{3.108000in}{1.110894in}}%
\pgfpathlineto{\pgfqpoint{3.120400in}{1.104726in}}%
\pgfpathlineto{\pgfqpoint{3.132800in}{1.105177in}}%
\pgfpathlineto{\pgfqpoint{3.157600in}{1.096452in}}%
\pgfpathlineto{\pgfqpoint{3.170000in}{1.105401in}}%
\pgfpathlineto{\pgfqpoint{3.182400in}{1.093952in}}%
\pgfpathlineto{\pgfqpoint{3.194800in}{1.100285in}}%
\pgfpathlineto{\pgfqpoint{3.207200in}{1.094132in}}%
\pgfpathlineto{\pgfqpoint{3.219600in}{1.089345in}}%
\pgfpathlineto{\pgfqpoint{3.232000in}{1.101030in}}%
\pgfpathlineto{\pgfqpoint{3.244400in}{1.101979in}}%
\pgfpathlineto{\pgfqpoint{3.256800in}{1.097018in}}%
\pgfpathlineto{\pgfqpoint{3.269200in}{1.096713in}}%
\pgfpathlineto{\pgfqpoint{3.281600in}{1.104290in}}%
\pgfpathlineto{\pgfqpoint{3.294000in}{1.098703in}}%
\pgfpathlineto{\pgfqpoint{3.306400in}{1.105471in}}%
\pgfpathlineto{\pgfqpoint{3.318800in}{1.099111in}}%
\pgfpathlineto{\pgfqpoint{3.331200in}{1.107222in}}%
\pgfpathlineto{\pgfqpoint{3.343600in}{1.104202in}}%
\pgfpathlineto{\pgfqpoint{3.368400in}{1.114428in}}%
\pgfpathlineto{\pgfqpoint{3.380800in}{1.110890in}}%
\pgfpathlineto{\pgfqpoint{3.393200in}{1.113786in}}%
\pgfpathlineto{\pgfqpoint{3.405600in}{1.114426in}}%
\pgfpathlineto{\pgfqpoint{3.418000in}{1.117141in}}%
\pgfpathlineto{\pgfqpoint{3.430400in}{1.115438in}}%
\pgfpathlineto{\pgfqpoint{3.467600in}{1.122854in}}%
\pgfpathlineto{\pgfqpoint{3.480000in}{1.122386in}}%
\pgfpathlineto{\pgfqpoint{3.492400in}{1.123835in}}%
\pgfpathlineto{\pgfqpoint{3.504800in}{1.132320in}}%
\pgfpathlineto{\pgfqpoint{3.517200in}{1.125370in}}%
\pgfpathlineto{\pgfqpoint{3.529600in}{1.130923in}}%
\pgfpathlineto{\pgfqpoint{3.542000in}{1.137786in}}%
\pgfpathlineto{\pgfqpoint{3.554400in}{1.137870in}}%
\pgfpathlineto{\pgfqpoint{3.566800in}{1.139780in}}%
\pgfpathlineto{\pgfqpoint{3.579200in}{1.145362in}}%
\pgfpathlineto{\pgfqpoint{3.591600in}{1.142786in}}%
\pgfpathlineto{\pgfqpoint{3.604000in}{1.145098in}}%
\pgfpathlineto{\pgfqpoint{3.616400in}{1.149760in}}%
\pgfpathlineto{\pgfqpoint{3.641200in}{1.143587in}}%
\pgfpathlineto{\pgfqpoint{3.653600in}{1.149308in}}%
\pgfpathlineto{\pgfqpoint{3.678400in}{1.147154in}}%
\pgfpathlineto{\pgfqpoint{3.703200in}{1.147983in}}%
\pgfpathlineto{\pgfqpoint{3.715600in}{1.156472in}}%
\pgfpathlineto{\pgfqpoint{3.728000in}{1.154216in}}%
\pgfpathlineto{\pgfqpoint{3.740400in}{1.147632in}}%
\pgfpathlineto{\pgfqpoint{3.752800in}{1.152152in}}%
\pgfpathlineto{\pgfqpoint{3.765200in}{1.151360in}}%
\pgfpathlineto{\pgfqpoint{3.777600in}{1.146029in}}%
\pgfpathlineto{\pgfqpoint{3.790000in}{1.152200in}}%
\pgfpathlineto{\pgfqpoint{3.802400in}{1.155296in}}%
\pgfpathlineto{\pgfqpoint{3.814800in}{1.156032in}}%
\pgfpathlineto{\pgfqpoint{3.827200in}{1.153264in}}%
\pgfpathlineto{\pgfqpoint{3.839600in}{1.156704in}}%
\pgfpathlineto{\pgfqpoint{3.876800in}{1.148515in}}%
\pgfpathlineto{\pgfqpoint{3.889200in}{1.153968in}}%
\pgfpathlineto{\pgfqpoint{3.901600in}{1.154648in}}%
\pgfpathlineto{\pgfqpoint{3.914000in}{1.149618in}}%
\pgfpathlineto{\pgfqpoint{3.938800in}{1.150920in}}%
\pgfpathlineto{\pgfqpoint{3.951200in}{1.150488in}}%
\pgfpathlineto{\pgfqpoint{3.963600in}{1.152968in}}%
\pgfpathlineto{\pgfqpoint{3.976000in}{1.149849in}}%
\pgfpathlineto{\pgfqpoint{4.000800in}{1.146242in}}%
\pgfpathlineto{\pgfqpoint{4.013200in}{1.146527in}}%
\pgfpathlineto{\pgfqpoint{4.025600in}{1.145439in}}%
\pgfpathlineto{\pgfqpoint{4.038000in}{1.148988in}}%
\pgfpathlineto{\pgfqpoint{4.050400in}{1.145886in}}%
\pgfpathlineto{\pgfqpoint{4.062800in}{1.152304in}}%
\pgfpathlineto{\pgfqpoint{4.075200in}{1.145982in}}%
\pgfpathlineto{\pgfqpoint{4.087600in}{1.149856in}}%
\pgfpathlineto{\pgfqpoint{4.100000in}{1.145469in}}%
\pgfpathlineto{\pgfqpoint{4.112400in}{1.144458in}}%
\pgfpathlineto{\pgfqpoint{4.124800in}{1.144567in}}%
\pgfpathlineto{\pgfqpoint{4.137200in}{1.148729in}}%
\pgfpathlineto{\pgfqpoint{4.149600in}{1.147969in}}%
\pgfpathlineto{\pgfqpoint{4.162000in}{1.144282in}}%
\pgfpathlineto{\pgfqpoint{4.174400in}{1.143095in}}%
\pgfpathlineto{\pgfqpoint{4.186800in}{1.146627in}}%
\pgfpathlineto{\pgfqpoint{4.199200in}{1.145049in}}%
\pgfpathlineto{\pgfqpoint{4.211600in}{1.145311in}}%
\pgfpathlineto{\pgfqpoint{4.224000in}{1.147084in}}%
\pgfpathlineto{\pgfqpoint{4.236400in}{1.151600in}}%
\pgfpathlineto{\pgfqpoint{4.248800in}{1.144186in}}%
\pgfpathlineto{\pgfqpoint{4.273600in}{1.147254in}}%
\pgfpathlineto{\pgfqpoint{4.286000in}{1.143282in}}%
\pgfpathlineto{\pgfqpoint{4.298400in}{1.143111in}}%
\pgfpathlineto{\pgfqpoint{4.310800in}{1.141822in}}%
\pgfpathlineto{\pgfqpoint{4.323200in}{1.146500in}}%
\pgfpathlineto{\pgfqpoint{4.335600in}{1.141994in}}%
\pgfpathlineto{\pgfqpoint{4.348000in}{1.141874in}}%
\pgfpathlineto{\pgfqpoint{4.360400in}{1.137528in}}%
\pgfpathlineto{\pgfqpoint{4.372800in}{1.139638in}}%
\pgfpathlineto{\pgfqpoint{4.385200in}{1.138189in}}%
\pgfpathlineto{\pgfqpoint{4.397600in}{1.140370in}}%
\pgfpathlineto{\pgfqpoint{4.422400in}{1.132490in}}%
\pgfpathlineto{\pgfqpoint{4.434800in}{1.132814in}}%
\pgfpathlineto{\pgfqpoint{4.447200in}{1.137898in}}%
\pgfpathlineto{\pgfqpoint{4.459600in}{1.129594in}}%
\pgfpathlineto{\pgfqpoint{4.472000in}{1.128982in}}%
\pgfpathlineto{\pgfqpoint{4.484400in}{1.126456in}}%
\pgfpathlineto{\pgfqpoint{4.496800in}{1.125448in}}%
\pgfpathlineto{\pgfqpoint{4.509200in}{1.128711in}}%
\pgfpathlineto{\pgfqpoint{4.521600in}{1.121410in}}%
\pgfpathlineto{\pgfqpoint{4.534000in}{1.125613in}}%
\pgfpathlineto{\pgfqpoint{4.546400in}{1.122402in}}%
\pgfpathlineto{\pgfqpoint{4.596000in}{1.122741in}}%
\pgfpathlineto{\pgfqpoint{4.608400in}{1.120300in}}%
\pgfpathlineto{\pgfqpoint{4.620800in}{1.121823in}}%
\pgfpathlineto{\pgfqpoint{4.633200in}{1.124862in}}%
\pgfpathlineto{\pgfqpoint{4.645600in}{1.125443in}}%
\pgfpathlineto{\pgfqpoint{4.658000in}{1.123974in}}%
\pgfpathlineto{\pgfqpoint{4.670400in}{1.125057in}}%
\pgfpathlineto{\pgfqpoint{4.682800in}{1.128093in}}%
\pgfpathlineto{\pgfqpoint{4.695200in}{1.127010in}}%
\pgfpathlineto{\pgfqpoint{4.732400in}{1.128258in}}%
\pgfpathlineto{\pgfqpoint{4.744800in}{1.132323in}}%
\pgfpathlineto{\pgfqpoint{4.757200in}{1.133912in}}%
\pgfpathlineto{\pgfqpoint{4.769600in}{1.132870in}}%
\pgfpathlineto{\pgfqpoint{4.782000in}{1.134040in}}%
\pgfpathlineto{\pgfqpoint{4.794400in}{1.137065in}}%
\pgfpathlineto{\pgfqpoint{4.806800in}{1.136978in}}%
\pgfpathlineto{\pgfqpoint{4.819200in}{1.140349in}}%
\pgfpathlineto{\pgfqpoint{4.831600in}{1.136548in}}%
\pgfpathlineto{\pgfqpoint{4.844000in}{1.143072in}}%
\pgfpathlineto{\pgfqpoint{4.856400in}{1.142604in}}%
\pgfpathlineto{\pgfqpoint{4.868800in}{1.145673in}}%
\pgfpathlineto{\pgfqpoint{4.881200in}{1.144209in}}%
\pgfpathlineto{\pgfqpoint{4.906000in}{1.148772in}}%
\pgfpathlineto{\pgfqpoint{4.943200in}{1.151144in}}%
\pgfpathlineto{\pgfqpoint{4.955600in}{1.152352in}}%
\pgfpathlineto{\pgfqpoint{4.980400in}{1.156824in}}%
\pgfpathlineto{\pgfqpoint{4.992800in}{1.155464in}}%
\pgfpathlineto{\pgfqpoint{5.005200in}{1.160784in}}%
\pgfpathlineto{\pgfqpoint{5.017600in}{1.157696in}}%
\pgfpathlineto{\pgfqpoint{5.030000in}{1.162728in}}%
\pgfpathlineto{\pgfqpoint{5.042400in}{1.160624in}}%
\pgfpathlineto{\pgfqpoint{5.054800in}{1.164728in}}%
\pgfpathlineto{\pgfqpoint{5.067200in}{1.164168in}}%
\pgfpathlineto{\pgfqpoint{5.079600in}{1.159472in}}%
\pgfpathlineto{\pgfqpoint{5.092000in}{1.167624in}}%
\pgfpathlineto{\pgfqpoint{5.104400in}{1.163968in}}%
\pgfpathlineto{\pgfqpoint{5.116800in}{1.164008in}}%
\pgfpathlineto{\pgfqpoint{5.129200in}{1.168408in}}%
\pgfpathlineto{\pgfqpoint{5.141600in}{1.167824in}}%
\pgfpathlineto{\pgfqpoint{5.154000in}{1.169888in}}%
\pgfpathlineto{\pgfqpoint{5.166400in}{1.165072in}}%
\pgfpathlineto{\pgfqpoint{5.178800in}{1.165360in}}%
\pgfpathlineto{\pgfqpoint{5.191200in}{1.168520in}}%
\pgfpathlineto{\pgfqpoint{5.203600in}{1.167168in}}%
\pgfpathlineto{\pgfqpoint{5.216000in}{1.168320in}}%
\pgfpathlineto{\pgfqpoint{5.240800in}{1.168744in}}%
\pgfpathlineto{\pgfqpoint{5.253200in}{1.170040in}}%
\pgfpathlineto{\pgfqpoint{5.265600in}{1.166000in}}%
\pgfpathlineto{\pgfqpoint{5.278000in}{1.169816in}}%
\pgfpathlineto{\pgfqpoint{5.290400in}{1.166304in}}%
\pgfpathlineto{\pgfqpoint{5.302800in}{1.168024in}}%
\pgfpathlineto{\pgfqpoint{5.315200in}{1.168400in}}%
\pgfpathlineto{\pgfqpoint{5.327600in}{1.165200in}}%
\pgfpathlineto{\pgfqpoint{5.352400in}{1.169104in}}%
\pgfpathlineto{\pgfqpoint{5.377200in}{1.164920in}}%
\pgfpathlineto{\pgfqpoint{5.414400in}{1.165928in}}%
\pgfpathlineto{\pgfqpoint{5.439200in}{1.159864in}}%
\pgfpathlineto{\pgfqpoint{5.464000in}{1.165376in}}%
\pgfpathlineto{\pgfqpoint{5.476400in}{1.161152in}}%
\pgfpathlineto{\pgfqpoint{5.488800in}{1.164504in}}%
\pgfpathlineto{\pgfqpoint{5.501200in}{1.163456in}}%
\pgfpathlineto{\pgfqpoint{5.513600in}{1.164144in}}%
\pgfpathlineto{\pgfqpoint{5.526000in}{1.163616in}}%
\pgfpathlineto{\pgfqpoint{5.538400in}{1.160400in}}%
\pgfpathlineto{\pgfqpoint{5.550800in}{1.159448in}}%
\pgfpathlineto{\pgfqpoint{5.563200in}{1.155880in}}%
\pgfpathlineto{\pgfqpoint{5.575600in}{1.157496in}}%
\pgfpathlineto{\pgfqpoint{5.600400in}{1.154600in}}%
\pgfpathlineto{\pgfqpoint{5.612800in}{1.156264in}}%
\pgfpathlineto{\pgfqpoint{5.625200in}{1.154384in}}%
\pgfpathlineto{\pgfqpoint{5.637600in}{1.154480in}}%
\pgfpathlineto{\pgfqpoint{5.650000in}{1.148988in}}%
\pgfpathlineto{\pgfqpoint{5.662400in}{1.153000in}}%
\pgfpathlineto{\pgfqpoint{5.674800in}{1.150024in}}%
\pgfpathlineto{\pgfqpoint{5.699600in}{1.149576in}}%
\pgfpathlineto{\pgfqpoint{5.712000in}{1.149424in}}%
\pgfpathlineto{\pgfqpoint{5.724400in}{1.146820in}}%
\pgfpathlineto{\pgfqpoint{5.749200in}{1.152536in}}%
\pgfpathlineto{\pgfqpoint{5.761600in}{1.149538in}}%
\pgfpathlineto{\pgfqpoint{5.774000in}{1.149287in}}%
\pgfpathlineto{\pgfqpoint{5.786400in}{1.146477in}}%
\pgfpathlineto{\pgfqpoint{5.798800in}{1.147527in}}%
\pgfpathlineto{\pgfqpoint{5.811200in}{1.145097in}}%
\pgfpathlineto{\pgfqpoint{5.873200in}{1.142197in}}%
\pgfpathlineto{\pgfqpoint{5.885600in}{1.144293in}}%
\pgfpathlineto{\pgfqpoint{5.910400in}{1.141171in}}%
\pgfpathlineto{\pgfqpoint{5.922800in}{1.141317in}}%
\pgfpathlineto{\pgfqpoint{5.935200in}{1.143106in}}%
\pgfpathlineto{\pgfqpoint{5.947600in}{1.141667in}}%
\pgfpathlineto{\pgfqpoint{5.960000in}{1.142322in}}%
\pgfpathlineto{\pgfqpoint{5.984800in}{1.140865in}}%
\pgfpathlineto{\pgfqpoint{5.997200in}{1.142651in}}%
\pgfpathlineto{\pgfqpoint{6.022000in}{1.141012in}}%
\pgfpathlineto{\pgfqpoint{6.034400in}{1.142010in}}%
\pgfpathlineto{\pgfqpoint{6.046800in}{1.137950in}}%
\pgfpathlineto{\pgfqpoint{6.059200in}{1.142191in}}%
\pgfpathlineto{\pgfqpoint{6.071600in}{1.139122in}}%
\pgfpathlineto{\pgfqpoint{6.108800in}{1.143423in}}%
\pgfpathlineto{\pgfqpoint{6.121200in}{1.143791in}}%
\pgfpathlineto{\pgfqpoint{6.133600in}{1.140961in}}%
\pgfpathlineto{\pgfqpoint{6.146000in}{1.141373in}}%
\pgfpathlineto{\pgfqpoint{6.158400in}{1.140286in}}%
\pgfpathlineto{\pgfqpoint{6.170800in}{1.142171in}}%
\pgfpathlineto{\pgfqpoint{6.183200in}{1.142134in}}%
\pgfpathlineto{\pgfqpoint{6.195600in}{1.145439in}}%
\pgfpathlineto{\pgfqpoint{6.208000in}{1.145030in}}%
\pgfpathlineto{\pgfqpoint{6.220400in}{1.139637in}}%
\pgfpathlineto{\pgfqpoint{6.232800in}{1.143934in}}%
\pgfpathlineto{\pgfqpoint{6.257600in}{1.145330in}}%
\pgfpathlineto{\pgfqpoint{6.270000in}{1.142975in}}%
\pgfpathlineto{\pgfqpoint{6.294800in}{1.142829in}}%
\pgfpathlineto{\pgfqpoint{6.307200in}{1.145161in}}%
\pgfpathlineto{\pgfqpoint{6.319600in}{1.143691in}}%
\pgfpathlineto{\pgfqpoint{6.332000in}{1.145426in}}%
\pgfpathlineto{\pgfqpoint{6.344400in}{1.145537in}}%
\pgfpathlineto{\pgfqpoint{6.356800in}{1.147719in}}%
\pgfpathlineto{\pgfqpoint{6.369200in}{1.147940in}}%
\pgfpathlineto{\pgfqpoint{6.381600in}{1.143962in}}%
\pgfpathlineto{\pgfqpoint{6.394000in}{1.146746in}}%
\pgfpathlineto{\pgfqpoint{6.418800in}{1.145334in}}%
\pgfpathlineto{\pgfqpoint{6.431200in}{1.146406in}}%
\pgfpathlineto{\pgfqpoint{6.456000in}{1.147287in}}%
\pgfpathlineto{\pgfqpoint{6.468400in}{1.151016in}}%
\pgfpathlineto{\pgfqpoint{6.493200in}{1.147399in}}%
\pgfpathlineto{\pgfqpoint{6.505600in}{1.149341in}}%
\pgfpathlineto{\pgfqpoint{6.530400in}{1.149168in}}%
\pgfpathlineto{\pgfqpoint{6.555200in}{1.150512in}}%
\pgfpathlineto{\pgfqpoint{6.567600in}{1.147386in}}%
\pgfpathlineto{\pgfqpoint{6.580000in}{1.149585in}}%
\pgfpathlineto{\pgfqpoint{6.629600in}{1.153800in}}%
\pgfpathlineto{\pgfqpoint{6.642000in}{1.151376in}}%
\pgfpathlineto{\pgfqpoint{6.666800in}{1.153680in}}%
\pgfpathlineto{\pgfqpoint{6.691600in}{1.151400in}}%
\pgfpathlineto{\pgfqpoint{6.704000in}{1.152440in}}%
\pgfpathlineto{\pgfqpoint{6.704000in}{1.152440in}}%
\pgfusepath{stroke}%
\end{pgfscope}%
\begin{pgfscope}%
\pgfpathrectangle{\pgfqpoint{1.000000in}{0.350000in}}{\pgfqpoint{6.200000in}{2.800000in}} %
\pgfusepath{clip}%
\pgfsetbuttcap%
\pgfsetroundjoin%
\pgfsetlinewidth{0.501875pt}%
\definecolor{currentstroke}{rgb}{0.000000,0.000000,0.000000}%
\pgfsetstrokecolor{currentstroke}%
\pgfsetdash{{1.000000pt}{3.000000pt}}{0.000000pt}%
\pgfpathmoveto{\pgfqpoint{1.000000in}{0.350000in}}%
\pgfpathlineto{\pgfqpoint{1.000000in}{3.150000in}}%
\pgfusepath{stroke}%
\end{pgfscope}%
\begin{pgfscope}%
\pgfsetbuttcap%
\pgfsetroundjoin%
\definecolor{currentfill}{rgb}{0.000000,0.000000,0.000000}%
\pgfsetfillcolor{currentfill}%
\pgfsetlinewidth{0.501875pt}%
\definecolor{currentstroke}{rgb}{0.000000,0.000000,0.000000}%
\pgfsetstrokecolor{currentstroke}%
\pgfsetdash{}{0pt}%
\pgfsys@defobject{currentmarker}{\pgfqpoint{0.000000in}{0.000000in}}{\pgfqpoint{0.000000in}{0.055556in}}{%
\pgfpathmoveto{\pgfqpoint{0.000000in}{0.000000in}}%
\pgfpathlineto{\pgfqpoint{0.000000in}{0.055556in}}%
\pgfusepath{stroke,fill}%
}%
\begin{pgfscope}%
\pgfsys@transformshift{1.000000in}{0.350000in}%
\pgfsys@useobject{currentmarker}{}%
\end{pgfscope}%
\end{pgfscope}%
\begin{pgfscope}%
\pgfsetbuttcap%
\pgfsetroundjoin%
\definecolor{currentfill}{rgb}{0.000000,0.000000,0.000000}%
\pgfsetfillcolor{currentfill}%
\pgfsetlinewidth{0.501875pt}%
\definecolor{currentstroke}{rgb}{0.000000,0.000000,0.000000}%
\pgfsetstrokecolor{currentstroke}%
\pgfsetdash{}{0pt}%
\pgfsys@defobject{currentmarker}{\pgfqpoint{0.000000in}{-0.055556in}}{\pgfqpoint{0.000000in}{0.000000in}}{%
\pgfpathmoveto{\pgfqpoint{0.000000in}{0.000000in}}%
\pgfpathlineto{\pgfqpoint{0.000000in}{-0.055556in}}%
\pgfusepath{stroke,fill}%
}%
\begin{pgfscope}%
\pgfsys@transformshift{1.000000in}{3.150000in}%
\pgfsys@useobject{currentmarker}{}%
\end{pgfscope}%
\end{pgfscope}%
\begin{pgfscope}%
\pgftext[left,bottom,x=0.867472in,y=0.168387in,rotate=0.000000]{{\sffamily\fontsize{12.000000}{14.400000}\selectfont 0.0}}
%
\end{pgfscope}%
\begin{pgfscope}%
\pgfpathrectangle{\pgfqpoint{1.000000in}{0.350000in}}{\pgfqpoint{6.200000in}{2.800000in}} %
\pgfusepath{clip}%
\pgfsetbuttcap%
\pgfsetroundjoin%
\pgfsetlinewidth{0.501875pt}%
\definecolor{currentstroke}{rgb}{0.000000,0.000000,0.000000}%
\pgfsetstrokecolor{currentstroke}%
\pgfsetdash{{1.000000pt}{3.000000pt}}{0.000000pt}%
\pgfpathmoveto{\pgfqpoint{2.240000in}{0.350000in}}%
\pgfpathlineto{\pgfqpoint{2.240000in}{3.150000in}}%
\pgfusepath{stroke}%
\end{pgfscope}%
\begin{pgfscope}%
\pgfsetbuttcap%
\pgfsetroundjoin%
\definecolor{currentfill}{rgb}{0.000000,0.000000,0.000000}%
\pgfsetfillcolor{currentfill}%
\pgfsetlinewidth{0.501875pt}%
\definecolor{currentstroke}{rgb}{0.000000,0.000000,0.000000}%
\pgfsetstrokecolor{currentstroke}%
\pgfsetdash{}{0pt}%
\pgfsys@defobject{currentmarker}{\pgfqpoint{0.000000in}{0.000000in}}{\pgfqpoint{0.000000in}{0.055556in}}{%
\pgfpathmoveto{\pgfqpoint{0.000000in}{0.000000in}}%
\pgfpathlineto{\pgfqpoint{0.000000in}{0.055556in}}%
\pgfusepath{stroke,fill}%
}%
\begin{pgfscope}%
\pgfsys@transformshift{2.240000in}{0.350000in}%
\pgfsys@useobject{currentmarker}{}%
\end{pgfscope}%
\end{pgfscope}%
\begin{pgfscope}%
\pgfsetbuttcap%
\pgfsetroundjoin%
\definecolor{currentfill}{rgb}{0.000000,0.000000,0.000000}%
\pgfsetfillcolor{currentfill}%
\pgfsetlinewidth{0.501875pt}%
\definecolor{currentstroke}{rgb}{0.000000,0.000000,0.000000}%
\pgfsetstrokecolor{currentstroke}%
\pgfsetdash{}{0pt}%
\pgfsys@defobject{currentmarker}{\pgfqpoint{0.000000in}{-0.055556in}}{\pgfqpoint{0.000000in}{0.000000in}}{%
\pgfpathmoveto{\pgfqpoint{0.000000in}{0.000000in}}%
\pgfpathlineto{\pgfqpoint{0.000000in}{-0.055556in}}%
\pgfusepath{stroke,fill}%
}%
\begin{pgfscope}%
\pgfsys@transformshift{2.240000in}{3.150000in}%
\pgfsys@useobject{currentmarker}{}%
\end{pgfscope}%
\end{pgfscope}%
\begin{pgfscope}%
\pgftext[left,bottom,x=2.107472in,y=0.168387in,rotate=0.000000]{{\sffamily\fontsize{12.000000}{14.400000}\selectfont 0.2}}
%
\end{pgfscope}%
\begin{pgfscope}%
\pgfpathrectangle{\pgfqpoint{1.000000in}{0.350000in}}{\pgfqpoint{6.200000in}{2.800000in}} %
\pgfusepath{clip}%
\pgfsetbuttcap%
\pgfsetroundjoin%
\pgfsetlinewidth{0.501875pt}%
\definecolor{currentstroke}{rgb}{0.000000,0.000000,0.000000}%
\pgfsetstrokecolor{currentstroke}%
\pgfsetdash{{1.000000pt}{3.000000pt}}{0.000000pt}%
\pgfpathmoveto{\pgfqpoint{3.480000in}{0.350000in}}%
\pgfpathlineto{\pgfqpoint{3.480000in}{3.150000in}}%
\pgfusepath{stroke}%
\end{pgfscope}%
\begin{pgfscope}%
\pgfsetbuttcap%
\pgfsetroundjoin%
\definecolor{currentfill}{rgb}{0.000000,0.000000,0.000000}%
\pgfsetfillcolor{currentfill}%
\pgfsetlinewidth{0.501875pt}%
\definecolor{currentstroke}{rgb}{0.000000,0.000000,0.000000}%
\pgfsetstrokecolor{currentstroke}%
\pgfsetdash{}{0pt}%
\pgfsys@defobject{currentmarker}{\pgfqpoint{0.000000in}{0.000000in}}{\pgfqpoint{0.000000in}{0.055556in}}{%
\pgfpathmoveto{\pgfqpoint{0.000000in}{0.000000in}}%
\pgfpathlineto{\pgfqpoint{0.000000in}{0.055556in}}%
\pgfusepath{stroke,fill}%
}%
\begin{pgfscope}%
\pgfsys@transformshift{3.480000in}{0.350000in}%
\pgfsys@useobject{currentmarker}{}%
\end{pgfscope}%
\end{pgfscope}%
\begin{pgfscope}%
\pgfsetbuttcap%
\pgfsetroundjoin%
\definecolor{currentfill}{rgb}{0.000000,0.000000,0.000000}%
\pgfsetfillcolor{currentfill}%
\pgfsetlinewidth{0.501875pt}%
\definecolor{currentstroke}{rgb}{0.000000,0.000000,0.000000}%
\pgfsetstrokecolor{currentstroke}%
\pgfsetdash{}{0pt}%
\pgfsys@defobject{currentmarker}{\pgfqpoint{0.000000in}{-0.055556in}}{\pgfqpoint{0.000000in}{0.000000in}}{%
\pgfpathmoveto{\pgfqpoint{0.000000in}{0.000000in}}%
\pgfpathlineto{\pgfqpoint{0.000000in}{-0.055556in}}%
\pgfusepath{stroke,fill}%
}%
\begin{pgfscope}%
\pgfsys@transformshift{3.480000in}{3.150000in}%
\pgfsys@useobject{currentmarker}{}%
\end{pgfscope}%
\end{pgfscope}%
\begin{pgfscope}%
\pgftext[left,bottom,x=3.347472in,y=0.168387in,rotate=0.000000]{{\sffamily\fontsize{12.000000}{14.400000}\selectfont 0.4}}
%
\end{pgfscope}%
\begin{pgfscope}%
\pgfpathrectangle{\pgfqpoint{1.000000in}{0.350000in}}{\pgfqpoint{6.200000in}{2.800000in}} %
\pgfusepath{clip}%
\pgfsetbuttcap%
\pgfsetroundjoin%
\pgfsetlinewidth{0.501875pt}%
\definecolor{currentstroke}{rgb}{0.000000,0.000000,0.000000}%
\pgfsetstrokecolor{currentstroke}%
\pgfsetdash{{1.000000pt}{3.000000pt}}{0.000000pt}%
\pgfpathmoveto{\pgfqpoint{4.720000in}{0.350000in}}%
\pgfpathlineto{\pgfqpoint{4.720000in}{3.150000in}}%
\pgfusepath{stroke}%
\end{pgfscope}%
\begin{pgfscope}%
\pgfsetbuttcap%
\pgfsetroundjoin%
\definecolor{currentfill}{rgb}{0.000000,0.000000,0.000000}%
\pgfsetfillcolor{currentfill}%
\pgfsetlinewidth{0.501875pt}%
\definecolor{currentstroke}{rgb}{0.000000,0.000000,0.000000}%
\pgfsetstrokecolor{currentstroke}%
\pgfsetdash{}{0pt}%
\pgfsys@defobject{currentmarker}{\pgfqpoint{0.000000in}{0.000000in}}{\pgfqpoint{0.000000in}{0.055556in}}{%
\pgfpathmoveto{\pgfqpoint{0.000000in}{0.000000in}}%
\pgfpathlineto{\pgfqpoint{0.000000in}{0.055556in}}%
\pgfusepath{stroke,fill}%
}%
\begin{pgfscope}%
\pgfsys@transformshift{4.720000in}{0.350000in}%
\pgfsys@useobject{currentmarker}{}%
\end{pgfscope}%
\end{pgfscope}%
\begin{pgfscope}%
\pgfsetbuttcap%
\pgfsetroundjoin%
\definecolor{currentfill}{rgb}{0.000000,0.000000,0.000000}%
\pgfsetfillcolor{currentfill}%
\pgfsetlinewidth{0.501875pt}%
\definecolor{currentstroke}{rgb}{0.000000,0.000000,0.000000}%
\pgfsetstrokecolor{currentstroke}%
\pgfsetdash{}{0pt}%
\pgfsys@defobject{currentmarker}{\pgfqpoint{0.000000in}{-0.055556in}}{\pgfqpoint{0.000000in}{0.000000in}}{%
\pgfpathmoveto{\pgfqpoint{0.000000in}{0.000000in}}%
\pgfpathlineto{\pgfqpoint{0.000000in}{-0.055556in}}%
\pgfusepath{stroke,fill}%
}%
\begin{pgfscope}%
\pgfsys@transformshift{4.720000in}{3.150000in}%
\pgfsys@useobject{currentmarker}{}%
\end{pgfscope}%
\end{pgfscope}%
\begin{pgfscope}%
\pgftext[left,bottom,x=4.587472in,y=0.168387in,rotate=0.000000]{{\sffamily\fontsize{12.000000}{14.400000}\selectfont 0.6}}
%
\end{pgfscope}%
\begin{pgfscope}%
\pgfpathrectangle{\pgfqpoint{1.000000in}{0.350000in}}{\pgfqpoint{6.200000in}{2.800000in}} %
\pgfusepath{clip}%
\pgfsetbuttcap%
\pgfsetroundjoin%
\pgfsetlinewidth{0.501875pt}%
\definecolor{currentstroke}{rgb}{0.000000,0.000000,0.000000}%
\pgfsetstrokecolor{currentstroke}%
\pgfsetdash{{1.000000pt}{3.000000pt}}{0.000000pt}%
\pgfpathmoveto{\pgfqpoint{5.960000in}{0.350000in}}%
\pgfpathlineto{\pgfqpoint{5.960000in}{3.150000in}}%
\pgfusepath{stroke}%
\end{pgfscope}%
\begin{pgfscope}%
\pgfsetbuttcap%
\pgfsetroundjoin%
\definecolor{currentfill}{rgb}{0.000000,0.000000,0.000000}%
\pgfsetfillcolor{currentfill}%
\pgfsetlinewidth{0.501875pt}%
\definecolor{currentstroke}{rgb}{0.000000,0.000000,0.000000}%
\pgfsetstrokecolor{currentstroke}%
\pgfsetdash{}{0pt}%
\pgfsys@defobject{currentmarker}{\pgfqpoint{0.000000in}{0.000000in}}{\pgfqpoint{0.000000in}{0.055556in}}{%
\pgfpathmoveto{\pgfqpoint{0.000000in}{0.000000in}}%
\pgfpathlineto{\pgfqpoint{0.000000in}{0.055556in}}%
\pgfusepath{stroke,fill}%
}%
\begin{pgfscope}%
\pgfsys@transformshift{5.960000in}{0.350000in}%
\pgfsys@useobject{currentmarker}{}%
\end{pgfscope}%
\end{pgfscope}%
\begin{pgfscope}%
\pgfsetbuttcap%
\pgfsetroundjoin%
\definecolor{currentfill}{rgb}{0.000000,0.000000,0.000000}%
\pgfsetfillcolor{currentfill}%
\pgfsetlinewidth{0.501875pt}%
\definecolor{currentstroke}{rgb}{0.000000,0.000000,0.000000}%
\pgfsetstrokecolor{currentstroke}%
\pgfsetdash{}{0pt}%
\pgfsys@defobject{currentmarker}{\pgfqpoint{0.000000in}{-0.055556in}}{\pgfqpoint{0.000000in}{0.000000in}}{%
\pgfpathmoveto{\pgfqpoint{0.000000in}{0.000000in}}%
\pgfpathlineto{\pgfqpoint{0.000000in}{-0.055556in}}%
\pgfusepath{stroke,fill}%
}%
\begin{pgfscope}%
\pgfsys@transformshift{5.960000in}{3.150000in}%
\pgfsys@useobject{currentmarker}{}%
\end{pgfscope}%
\end{pgfscope}%
\begin{pgfscope}%
\pgftext[left,bottom,x=5.827472in,y=0.168387in,rotate=0.000000]{{\sffamily\fontsize{12.000000}{14.400000}\selectfont 0.8}}
%
\end{pgfscope}%
\begin{pgfscope}%
\pgfpathrectangle{\pgfqpoint{1.000000in}{0.350000in}}{\pgfqpoint{6.200000in}{2.800000in}} %
\pgfusepath{clip}%
\pgfsetbuttcap%
\pgfsetroundjoin%
\pgfsetlinewidth{0.501875pt}%
\definecolor{currentstroke}{rgb}{0.000000,0.000000,0.000000}%
\pgfsetstrokecolor{currentstroke}%
\pgfsetdash{{1.000000pt}{3.000000pt}}{0.000000pt}%
\pgfpathmoveto{\pgfqpoint{7.200000in}{0.350000in}}%
\pgfpathlineto{\pgfqpoint{7.200000in}{3.150000in}}%
\pgfusepath{stroke}%
\end{pgfscope}%
\begin{pgfscope}%
\pgfsetbuttcap%
\pgfsetroundjoin%
\definecolor{currentfill}{rgb}{0.000000,0.000000,0.000000}%
\pgfsetfillcolor{currentfill}%
\pgfsetlinewidth{0.501875pt}%
\definecolor{currentstroke}{rgb}{0.000000,0.000000,0.000000}%
\pgfsetstrokecolor{currentstroke}%
\pgfsetdash{}{0pt}%
\pgfsys@defobject{currentmarker}{\pgfqpoint{0.000000in}{0.000000in}}{\pgfqpoint{0.000000in}{0.055556in}}{%
\pgfpathmoveto{\pgfqpoint{0.000000in}{0.000000in}}%
\pgfpathlineto{\pgfqpoint{0.000000in}{0.055556in}}%
\pgfusepath{stroke,fill}%
}%
\begin{pgfscope}%
\pgfsys@transformshift{7.200000in}{0.350000in}%
\pgfsys@useobject{currentmarker}{}%
\end{pgfscope}%
\end{pgfscope}%
\begin{pgfscope}%
\pgfsetbuttcap%
\pgfsetroundjoin%
\definecolor{currentfill}{rgb}{0.000000,0.000000,0.000000}%
\pgfsetfillcolor{currentfill}%
\pgfsetlinewidth{0.501875pt}%
\definecolor{currentstroke}{rgb}{0.000000,0.000000,0.000000}%
\pgfsetstrokecolor{currentstroke}%
\pgfsetdash{}{0pt}%
\pgfsys@defobject{currentmarker}{\pgfqpoint{0.000000in}{-0.055556in}}{\pgfqpoint{0.000000in}{0.000000in}}{%
\pgfpathmoveto{\pgfqpoint{0.000000in}{0.000000in}}%
\pgfpathlineto{\pgfqpoint{0.000000in}{-0.055556in}}%
\pgfusepath{stroke,fill}%
}%
\begin{pgfscope}%
\pgfsys@transformshift{7.200000in}{3.150000in}%
\pgfsys@useobject{currentmarker}{}%
\end{pgfscope}%
\end{pgfscope}%
\begin{pgfscope}%
\pgftext[left,bottom,x=7.067472in,y=0.168387in,rotate=0.000000]{{\sffamily\fontsize{12.000000}{14.400000}\selectfont 1.0}}
%
\end{pgfscope}%
\begin{pgfscope}%
\pgftext[left,bottom,x=3.842391in,y=-0.030045in,rotate=0.000000]{{\sffamily\fontsize{12.000000}{14.400000}\selectfont radius}}
%
\end{pgfscope}%
\begin{pgfscope}%
\pgfpathrectangle{\pgfqpoint{1.000000in}{0.350000in}}{\pgfqpoint{6.200000in}{2.800000in}} %
\pgfusepath{clip}%
\pgfsetbuttcap%
\pgfsetroundjoin%
\pgfsetlinewidth{0.501875pt}%
\definecolor{currentstroke}{rgb}{0.000000,0.000000,0.000000}%
\pgfsetstrokecolor{currentstroke}%
\pgfsetdash{{1.000000pt}{3.000000pt}}{0.000000pt}%
\pgfpathmoveto{\pgfqpoint{1.000000in}{0.350000in}}%
\pgfpathlineto{\pgfqpoint{7.200000in}{0.350000in}}%
\pgfusepath{stroke}%
\end{pgfscope}%
\begin{pgfscope}%
\pgfsetbuttcap%
\pgfsetroundjoin%
\definecolor{currentfill}{rgb}{0.000000,0.000000,0.000000}%
\pgfsetfillcolor{currentfill}%
\pgfsetlinewidth{0.501875pt}%
\definecolor{currentstroke}{rgb}{0.000000,0.000000,0.000000}%
\pgfsetstrokecolor{currentstroke}%
\pgfsetdash{}{0pt}%
\pgfsys@defobject{currentmarker}{\pgfqpoint{0.000000in}{0.000000in}}{\pgfqpoint{0.055556in}{0.000000in}}{%
\pgfpathmoveto{\pgfqpoint{0.000000in}{0.000000in}}%
\pgfpathlineto{\pgfqpoint{0.055556in}{0.000000in}}%
\pgfusepath{stroke,fill}%
}%
\begin{pgfscope}%
\pgfsys@transformshift{1.000000in}{0.350000in}%
\pgfsys@useobject{currentmarker}{}%
\end{pgfscope}%
\end{pgfscope}%
\begin{pgfscope}%
\pgfsetbuttcap%
\pgfsetroundjoin%
\definecolor{currentfill}{rgb}{0.000000,0.000000,0.000000}%
\pgfsetfillcolor{currentfill}%
\pgfsetlinewidth{0.501875pt}%
\definecolor{currentstroke}{rgb}{0.000000,0.000000,0.000000}%
\pgfsetstrokecolor{currentstroke}%
\pgfsetdash{}{0pt}%
\pgfsys@defobject{currentmarker}{\pgfqpoint{-0.055556in}{0.000000in}}{\pgfqpoint{0.000000in}{0.000000in}}{%
\pgfpathmoveto{\pgfqpoint{0.000000in}{0.000000in}}%
\pgfpathlineto{\pgfqpoint{-0.055556in}{0.000000in}}%
\pgfusepath{stroke,fill}%
}%
\begin{pgfscope}%
\pgfsys@transformshift{7.200000in}{0.350000in}%
\pgfsys@useobject{currentmarker}{}%
\end{pgfscope}%
\end{pgfscope}%
\begin{pgfscope}%
\pgftext[left,bottom,x=0.679389in,y=0.286971in,rotate=0.000000]{{\sffamily\fontsize{12.000000}{14.400000}\selectfont 0.0}}
%
\end{pgfscope}%
\begin{pgfscope}%
\pgfpathrectangle{\pgfqpoint{1.000000in}{0.350000in}}{\pgfqpoint{6.200000in}{2.800000in}} %
\pgfusepath{clip}%
\pgfsetbuttcap%
\pgfsetroundjoin%
\pgfsetlinewidth{0.501875pt}%
\definecolor{currentstroke}{rgb}{0.000000,0.000000,0.000000}%
\pgfsetstrokecolor{currentstroke}%
\pgfsetdash{{1.000000pt}{3.000000pt}}{0.000000pt}%
\pgfpathmoveto{\pgfqpoint{1.000000in}{0.750000in}}%
\pgfpathlineto{\pgfqpoint{7.200000in}{0.750000in}}%
\pgfusepath{stroke}%
\end{pgfscope}%
\begin{pgfscope}%
\pgfsetbuttcap%
\pgfsetroundjoin%
\definecolor{currentfill}{rgb}{0.000000,0.000000,0.000000}%
\pgfsetfillcolor{currentfill}%
\pgfsetlinewidth{0.501875pt}%
\definecolor{currentstroke}{rgb}{0.000000,0.000000,0.000000}%
\pgfsetstrokecolor{currentstroke}%
\pgfsetdash{}{0pt}%
\pgfsys@defobject{currentmarker}{\pgfqpoint{0.000000in}{0.000000in}}{\pgfqpoint{0.055556in}{0.000000in}}{%
\pgfpathmoveto{\pgfqpoint{0.000000in}{0.000000in}}%
\pgfpathlineto{\pgfqpoint{0.055556in}{0.000000in}}%
\pgfusepath{stroke,fill}%
}%
\begin{pgfscope}%
\pgfsys@transformshift{1.000000in}{0.750000in}%
\pgfsys@useobject{currentmarker}{}%
\end{pgfscope}%
\end{pgfscope}%
\begin{pgfscope}%
\pgfsetbuttcap%
\pgfsetroundjoin%
\definecolor{currentfill}{rgb}{0.000000,0.000000,0.000000}%
\pgfsetfillcolor{currentfill}%
\pgfsetlinewidth{0.501875pt}%
\definecolor{currentstroke}{rgb}{0.000000,0.000000,0.000000}%
\pgfsetstrokecolor{currentstroke}%
\pgfsetdash{}{0pt}%
\pgfsys@defobject{currentmarker}{\pgfqpoint{-0.055556in}{0.000000in}}{\pgfqpoint{0.000000in}{0.000000in}}{%
\pgfpathmoveto{\pgfqpoint{0.000000in}{0.000000in}}%
\pgfpathlineto{\pgfqpoint{-0.055556in}{0.000000in}}%
\pgfusepath{stroke,fill}%
}%
\begin{pgfscope}%
\pgfsys@transformshift{7.200000in}{0.750000in}%
\pgfsys@useobject{currentmarker}{}%
\end{pgfscope}%
\end{pgfscope}%
\begin{pgfscope}%
\pgftext[left,bottom,x=0.679389in,y=0.686971in,rotate=0.000000]{{\sffamily\fontsize{12.000000}{14.400000}\selectfont 0.5}}
%
\end{pgfscope}%
\begin{pgfscope}%
\pgfpathrectangle{\pgfqpoint{1.000000in}{0.350000in}}{\pgfqpoint{6.200000in}{2.800000in}} %
\pgfusepath{clip}%
\pgfsetbuttcap%
\pgfsetroundjoin%
\pgfsetlinewidth{0.501875pt}%
\definecolor{currentstroke}{rgb}{0.000000,0.000000,0.000000}%
\pgfsetstrokecolor{currentstroke}%
\pgfsetdash{{1.000000pt}{3.000000pt}}{0.000000pt}%
\pgfpathmoveto{\pgfqpoint{1.000000in}{1.150000in}}%
\pgfpathlineto{\pgfqpoint{7.200000in}{1.150000in}}%
\pgfusepath{stroke}%
\end{pgfscope}%
\begin{pgfscope}%
\pgfsetbuttcap%
\pgfsetroundjoin%
\definecolor{currentfill}{rgb}{0.000000,0.000000,0.000000}%
\pgfsetfillcolor{currentfill}%
\pgfsetlinewidth{0.501875pt}%
\definecolor{currentstroke}{rgb}{0.000000,0.000000,0.000000}%
\pgfsetstrokecolor{currentstroke}%
\pgfsetdash{}{0pt}%
\pgfsys@defobject{currentmarker}{\pgfqpoint{0.000000in}{0.000000in}}{\pgfqpoint{0.055556in}{0.000000in}}{%
\pgfpathmoveto{\pgfqpoint{0.000000in}{0.000000in}}%
\pgfpathlineto{\pgfqpoint{0.055556in}{0.000000in}}%
\pgfusepath{stroke,fill}%
}%
\begin{pgfscope}%
\pgfsys@transformshift{1.000000in}{1.150000in}%
\pgfsys@useobject{currentmarker}{}%
\end{pgfscope}%
\end{pgfscope}%
\begin{pgfscope}%
\pgfsetbuttcap%
\pgfsetroundjoin%
\definecolor{currentfill}{rgb}{0.000000,0.000000,0.000000}%
\pgfsetfillcolor{currentfill}%
\pgfsetlinewidth{0.501875pt}%
\definecolor{currentstroke}{rgb}{0.000000,0.000000,0.000000}%
\pgfsetstrokecolor{currentstroke}%
\pgfsetdash{}{0pt}%
\pgfsys@defobject{currentmarker}{\pgfqpoint{-0.055556in}{0.000000in}}{\pgfqpoint{0.000000in}{0.000000in}}{%
\pgfpathmoveto{\pgfqpoint{0.000000in}{0.000000in}}%
\pgfpathlineto{\pgfqpoint{-0.055556in}{0.000000in}}%
\pgfusepath{stroke,fill}%
}%
\begin{pgfscope}%
\pgfsys@transformshift{7.200000in}{1.150000in}%
\pgfsys@useobject{currentmarker}{}%
\end{pgfscope}%
\end{pgfscope}%
\begin{pgfscope}%
\pgftext[left,bottom,x=0.679389in,y=1.086971in,rotate=0.000000]{{\sffamily\fontsize{12.000000}{14.400000}\selectfont 1.0}}
%
\end{pgfscope}%
\begin{pgfscope}%
\pgfpathrectangle{\pgfqpoint{1.000000in}{0.350000in}}{\pgfqpoint{6.200000in}{2.800000in}} %
\pgfusepath{clip}%
\pgfsetbuttcap%
\pgfsetroundjoin%
\pgfsetlinewidth{0.501875pt}%
\definecolor{currentstroke}{rgb}{0.000000,0.000000,0.000000}%
\pgfsetstrokecolor{currentstroke}%
\pgfsetdash{{1.000000pt}{3.000000pt}}{0.000000pt}%
\pgfpathmoveto{\pgfqpoint{1.000000in}{1.550000in}}%
\pgfpathlineto{\pgfqpoint{7.200000in}{1.550000in}}%
\pgfusepath{stroke}%
\end{pgfscope}%
\begin{pgfscope}%
\pgfsetbuttcap%
\pgfsetroundjoin%
\definecolor{currentfill}{rgb}{0.000000,0.000000,0.000000}%
\pgfsetfillcolor{currentfill}%
\pgfsetlinewidth{0.501875pt}%
\definecolor{currentstroke}{rgb}{0.000000,0.000000,0.000000}%
\pgfsetstrokecolor{currentstroke}%
\pgfsetdash{}{0pt}%
\pgfsys@defobject{currentmarker}{\pgfqpoint{0.000000in}{0.000000in}}{\pgfqpoint{0.055556in}{0.000000in}}{%
\pgfpathmoveto{\pgfqpoint{0.000000in}{0.000000in}}%
\pgfpathlineto{\pgfqpoint{0.055556in}{0.000000in}}%
\pgfusepath{stroke,fill}%
}%
\begin{pgfscope}%
\pgfsys@transformshift{1.000000in}{1.550000in}%
\pgfsys@useobject{currentmarker}{}%
\end{pgfscope}%
\end{pgfscope}%
\begin{pgfscope}%
\pgfsetbuttcap%
\pgfsetroundjoin%
\definecolor{currentfill}{rgb}{0.000000,0.000000,0.000000}%
\pgfsetfillcolor{currentfill}%
\pgfsetlinewidth{0.501875pt}%
\definecolor{currentstroke}{rgb}{0.000000,0.000000,0.000000}%
\pgfsetstrokecolor{currentstroke}%
\pgfsetdash{}{0pt}%
\pgfsys@defobject{currentmarker}{\pgfqpoint{-0.055556in}{0.000000in}}{\pgfqpoint{0.000000in}{0.000000in}}{%
\pgfpathmoveto{\pgfqpoint{0.000000in}{0.000000in}}%
\pgfpathlineto{\pgfqpoint{-0.055556in}{0.000000in}}%
\pgfusepath{stroke,fill}%
}%
\begin{pgfscope}%
\pgfsys@transformshift{7.200000in}{1.550000in}%
\pgfsys@useobject{currentmarker}{}%
\end{pgfscope}%
\end{pgfscope}%
\begin{pgfscope}%
\pgftext[left,bottom,x=0.679389in,y=1.488070in,rotate=0.000000]{{\sffamily\fontsize{12.000000}{14.400000}\selectfont 1.5}}
%
\end{pgfscope}%
\begin{pgfscope}%
\pgfpathrectangle{\pgfqpoint{1.000000in}{0.350000in}}{\pgfqpoint{6.200000in}{2.800000in}} %
\pgfusepath{clip}%
\pgfsetbuttcap%
\pgfsetroundjoin%
\pgfsetlinewidth{0.501875pt}%
\definecolor{currentstroke}{rgb}{0.000000,0.000000,0.000000}%
\pgfsetstrokecolor{currentstroke}%
\pgfsetdash{{1.000000pt}{3.000000pt}}{0.000000pt}%
\pgfpathmoveto{\pgfqpoint{1.000000in}{1.950000in}}%
\pgfpathlineto{\pgfqpoint{7.200000in}{1.950000in}}%
\pgfusepath{stroke}%
\end{pgfscope}%
\begin{pgfscope}%
\pgfsetbuttcap%
\pgfsetroundjoin%
\definecolor{currentfill}{rgb}{0.000000,0.000000,0.000000}%
\pgfsetfillcolor{currentfill}%
\pgfsetlinewidth{0.501875pt}%
\definecolor{currentstroke}{rgb}{0.000000,0.000000,0.000000}%
\pgfsetstrokecolor{currentstroke}%
\pgfsetdash{}{0pt}%
\pgfsys@defobject{currentmarker}{\pgfqpoint{0.000000in}{0.000000in}}{\pgfqpoint{0.055556in}{0.000000in}}{%
\pgfpathmoveto{\pgfqpoint{0.000000in}{0.000000in}}%
\pgfpathlineto{\pgfqpoint{0.055556in}{0.000000in}}%
\pgfusepath{stroke,fill}%
}%
\begin{pgfscope}%
\pgfsys@transformshift{1.000000in}{1.950000in}%
\pgfsys@useobject{currentmarker}{}%
\end{pgfscope}%
\end{pgfscope}%
\begin{pgfscope}%
\pgfsetbuttcap%
\pgfsetroundjoin%
\definecolor{currentfill}{rgb}{0.000000,0.000000,0.000000}%
\pgfsetfillcolor{currentfill}%
\pgfsetlinewidth{0.501875pt}%
\definecolor{currentstroke}{rgb}{0.000000,0.000000,0.000000}%
\pgfsetstrokecolor{currentstroke}%
\pgfsetdash{}{0pt}%
\pgfsys@defobject{currentmarker}{\pgfqpoint{-0.055556in}{0.000000in}}{\pgfqpoint{0.000000in}{0.000000in}}{%
\pgfpathmoveto{\pgfqpoint{0.000000in}{0.000000in}}%
\pgfpathlineto{\pgfqpoint{-0.055556in}{0.000000in}}%
\pgfusepath{stroke,fill}%
}%
\begin{pgfscope}%
\pgfsys@transformshift{7.200000in}{1.950000in}%
\pgfsys@useobject{currentmarker}{}%
\end{pgfscope}%
\end{pgfscope}%
\begin{pgfscope}%
\pgftext[left,bottom,x=0.679389in,y=1.886971in,rotate=0.000000]{{\sffamily\fontsize{12.000000}{14.400000}\selectfont 2.0}}
%
\end{pgfscope}%
\begin{pgfscope}%
\pgfpathrectangle{\pgfqpoint{1.000000in}{0.350000in}}{\pgfqpoint{6.200000in}{2.800000in}} %
\pgfusepath{clip}%
\pgfsetbuttcap%
\pgfsetroundjoin%
\pgfsetlinewidth{0.501875pt}%
\definecolor{currentstroke}{rgb}{0.000000,0.000000,0.000000}%
\pgfsetstrokecolor{currentstroke}%
\pgfsetdash{{1.000000pt}{3.000000pt}}{0.000000pt}%
\pgfpathmoveto{\pgfqpoint{1.000000in}{2.350000in}}%
\pgfpathlineto{\pgfqpoint{7.200000in}{2.350000in}}%
\pgfusepath{stroke}%
\end{pgfscope}%
\begin{pgfscope}%
\pgfsetbuttcap%
\pgfsetroundjoin%
\definecolor{currentfill}{rgb}{0.000000,0.000000,0.000000}%
\pgfsetfillcolor{currentfill}%
\pgfsetlinewidth{0.501875pt}%
\definecolor{currentstroke}{rgb}{0.000000,0.000000,0.000000}%
\pgfsetstrokecolor{currentstroke}%
\pgfsetdash{}{0pt}%
\pgfsys@defobject{currentmarker}{\pgfqpoint{0.000000in}{0.000000in}}{\pgfqpoint{0.055556in}{0.000000in}}{%
\pgfpathmoveto{\pgfqpoint{0.000000in}{0.000000in}}%
\pgfpathlineto{\pgfqpoint{0.055556in}{0.000000in}}%
\pgfusepath{stroke,fill}%
}%
\begin{pgfscope}%
\pgfsys@transformshift{1.000000in}{2.350000in}%
\pgfsys@useobject{currentmarker}{}%
\end{pgfscope}%
\end{pgfscope}%
\begin{pgfscope}%
\pgfsetbuttcap%
\pgfsetroundjoin%
\definecolor{currentfill}{rgb}{0.000000,0.000000,0.000000}%
\pgfsetfillcolor{currentfill}%
\pgfsetlinewidth{0.501875pt}%
\definecolor{currentstroke}{rgb}{0.000000,0.000000,0.000000}%
\pgfsetstrokecolor{currentstroke}%
\pgfsetdash{}{0pt}%
\pgfsys@defobject{currentmarker}{\pgfqpoint{-0.055556in}{0.000000in}}{\pgfqpoint{0.000000in}{0.000000in}}{%
\pgfpathmoveto{\pgfqpoint{0.000000in}{0.000000in}}%
\pgfpathlineto{\pgfqpoint{-0.055556in}{0.000000in}}%
\pgfusepath{stroke,fill}%
}%
\begin{pgfscope}%
\pgfsys@transformshift{7.200000in}{2.350000in}%
\pgfsys@useobject{currentmarker}{}%
\end{pgfscope}%
\end{pgfscope}%
\begin{pgfscope}%
\pgftext[left,bottom,x=0.679389in,y=2.286971in,rotate=0.000000]{{\sffamily\fontsize{12.000000}{14.400000}\selectfont 2.5}}
%
\end{pgfscope}%
\begin{pgfscope}%
\pgfpathrectangle{\pgfqpoint{1.000000in}{0.350000in}}{\pgfqpoint{6.200000in}{2.800000in}} %
\pgfusepath{clip}%
\pgfsetbuttcap%
\pgfsetroundjoin%
\pgfsetlinewidth{0.501875pt}%
\definecolor{currentstroke}{rgb}{0.000000,0.000000,0.000000}%
\pgfsetstrokecolor{currentstroke}%
\pgfsetdash{{1.000000pt}{3.000000pt}}{0.000000pt}%
\pgfpathmoveto{\pgfqpoint{1.000000in}{2.750000in}}%
\pgfpathlineto{\pgfqpoint{7.200000in}{2.750000in}}%
\pgfusepath{stroke}%
\end{pgfscope}%
\begin{pgfscope}%
\pgfsetbuttcap%
\pgfsetroundjoin%
\definecolor{currentfill}{rgb}{0.000000,0.000000,0.000000}%
\pgfsetfillcolor{currentfill}%
\pgfsetlinewidth{0.501875pt}%
\definecolor{currentstroke}{rgb}{0.000000,0.000000,0.000000}%
\pgfsetstrokecolor{currentstroke}%
\pgfsetdash{}{0pt}%
\pgfsys@defobject{currentmarker}{\pgfqpoint{0.000000in}{0.000000in}}{\pgfqpoint{0.055556in}{0.000000in}}{%
\pgfpathmoveto{\pgfqpoint{0.000000in}{0.000000in}}%
\pgfpathlineto{\pgfqpoint{0.055556in}{0.000000in}}%
\pgfusepath{stroke,fill}%
}%
\begin{pgfscope}%
\pgfsys@transformshift{1.000000in}{2.750000in}%
\pgfsys@useobject{currentmarker}{}%
\end{pgfscope}%
\end{pgfscope}%
\begin{pgfscope}%
\pgfsetbuttcap%
\pgfsetroundjoin%
\definecolor{currentfill}{rgb}{0.000000,0.000000,0.000000}%
\pgfsetfillcolor{currentfill}%
\pgfsetlinewidth{0.501875pt}%
\definecolor{currentstroke}{rgb}{0.000000,0.000000,0.000000}%
\pgfsetstrokecolor{currentstroke}%
\pgfsetdash{}{0pt}%
\pgfsys@defobject{currentmarker}{\pgfqpoint{-0.055556in}{0.000000in}}{\pgfqpoint{0.000000in}{0.000000in}}{%
\pgfpathmoveto{\pgfqpoint{0.000000in}{0.000000in}}%
\pgfpathlineto{\pgfqpoint{-0.055556in}{0.000000in}}%
\pgfusepath{stroke,fill}%
}%
\begin{pgfscope}%
\pgfsys@transformshift{7.200000in}{2.750000in}%
\pgfsys@useobject{currentmarker}{}%
\end{pgfscope}%
\end{pgfscope}%
\begin{pgfscope}%
\pgftext[left,bottom,x=0.679389in,y=2.686971in,rotate=0.000000]{{\sffamily\fontsize{12.000000}{14.400000}\selectfont 3.0}}
%
\end{pgfscope}%
\begin{pgfscope}%
\pgfpathrectangle{\pgfqpoint{1.000000in}{0.350000in}}{\pgfqpoint{6.200000in}{2.800000in}} %
\pgfusepath{clip}%
\pgfsetbuttcap%
\pgfsetroundjoin%
\pgfsetlinewidth{0.501875pt}%
\definecolor{currentstroke}{rgb}{0.000000,0.000000,0.000000}%
\pgfsetstrokecolor{currentstroke}%
\pgfsetdash{{1.000000pt}{3.000000pt}}{0.000000pt}%
\pgfpathmoveto{\pgfqpoint{1.000000in}{3.150000in}}%
\pgfpathlineto{\pgfqpoint{7.200000in}{3.150000in}}%
\pgfusepath{stroke}%
\end{pgfscope}%
\begin{pgfscope}%
\pgfsetbuttcap%
\pgfsetroundjoin%
\definecolor{currentfill}{rgb}{0.000000,0.000000,0.000000}%
\pgfsetfillcolor{currentfill}%
\pgfsetlinewidth{0.501875pt}%
\definecolor{currentstroke}{rgb}{0.000000,0.000000,0.000000}%
\pgfsetstrokecolor{currentstroke}%
\pgfsetdash{}{0pt}%
\pgfsys@defobject{currentmarker}{\pgfqpoint{0.000000in}{0.000000in}}{\pgfqpoint{0.055556in}{0.000000in}}{%
\pgfpathmoveto{\pgfqpoint{0.000000in}{0.000000in}}%
\pgfpathlineto{\pgfqpoint{0.055556in}{0.000000in}}%
\pgfusepath{stroke,fill}%
}%
\begin{pgfscope}%
\pgfsys@transformshift{1.000000in}{3.150000in}%
\pgfsys@useobject{currentmarker}{}%
\end{pgfscope}%
\end{pgfscope}%
\begin{pgfscope}%
\pgfsetbuttcap%
\pgfsetroundjoin%
\definecolor{currentfill}{rgb}{0.000000,0.000000,0.000000}%
\pgfsetfillcolor{currentfill}%
\pgfsetlinewidth{0.501875pt}%
\definecolor{currentstroke}{rgb}{0.000000,0.000000,0.000000}%
\pgfsetstrokecolor{currentstroke}%
\pgfsetdash{}{0pt}%
\pgfsys@defobject{currentmarker}{\pgfqpoint{-0.055556in}{0.000000in}}{\pgfqpoint{0.000000in}{0.000000in}}{%
\pgfpathmoveto{\pgfqpoint{0.000000in}{0.000000in}}%
\pgfpathlineto{\pgfqpoint{-0.055556in}{0.000000in}}%
\pgfusepath{stroke,fill}%
}%
\begin{pgfscope}%
\pgfsys@transformshift{7.200000in}{3.150000in}%
\pgfsys@useobject{currentmarker}{}%
\end{pgfscope}%
\end{pgfscope}%
\begin{pgfscope}%
\pgftext[left,bottom,x=0.679389in,y=3.086971in,rotate=0.000000]{{\sffamily\fontsize{12.000000}{14.400000}\selectfont 3.5}}
%
\end{pgfscope}%
\begin{pgfscope}%
\pgftext[left,bottom,x=0.609945in,y=0.646118in,rotate=90.000000]{{\sffamily\fontsize{12.000000}{14.400000}\selectfont radial distribution function}}
%
\end{pgfscope}%
\begin{pgfscope}%
\pgfsetrectcap%
\pgfsetroundjoin%
\pgfsetlinewidth{1.003750pt}%
\definecolor{currentstroke}{rgb}{0.000000,0.000000,0.000000}%
\pgfsetstrokecolor{currentstroke}%
\pgfsetdash{}{0pt}%
\pgfpathmoveto{\pgfqpoint{1.000000in}{3.150000in}}%
\pgfpathlineto{\pgfqpoint{7.200000in}{3.150000in}}%
\pgfusepath{stroke}%
\end{pgfscope}%
\begin{pgfscope}%
\pgfsetrectcap%
\pgfsetroundjoin%
\pgfsetlinewidth{1.003750pt}%
\definecolor{currentstroke}{rgb}{0.000000,0.000000,0.000000}%
\pgfsetstrokecolor{currentstroke}%
\pgfsetdash{}{0pt}%
\pgfpathmoveto{\pgfqpoint{7.200000in}{0.350000in}}%
\pgfpathlineto{\pgfqpoint{7.200000in}{3.150000in}}%
\pgfusepath{stroke}%
\end{pgfscope}%
\begin{pgfscope}%
\pgfsetrectcap%
\pgfsetroundjoin%
\pgfsetlinewidth{1.003750pt}%
\definecolor{currentstroke}{rgb}{0.000000,0.000000,0.000000}%
\pgfsetstrokecolor{currentstroke}%
\pgfsetdash{}{0pt}%
\pgfpathmoveto{\pgfqpoint{1.000000in}{0.350000in}}%
\pgfpathlineto{\pgfqpoint{7.200000in}{0.350000in}}%
\pgfusepath{stroke}%
\end{pgfscope}%
\begin{pgfscope}%
\pgfsetrectcap%
\pgfsetroundjoin%
\pgfsetlinewidth{1.003750pt}%
\definecolor{currentstroke}{rgb}{0.000000,0.000000,0.000000}%
\pgfsetstrokecolor{currentstroke}%
\pgfsetdash{}{0pt}%
\pgfpathmoveto{\pgfqpoint{1.000000in}{0.350000in}}%
\pgfpathlineto{\pgfqpoint{1.000000in}{3.150000in}}%
\pgfusepath{stroke}%
\end{pgfscope}%
\begin{pgfscope}%
\pgfsetrectcap%
\pgfsetroundjoin%
\definecolor{currentfill}{rgb}{1.000000,1.000000,1.000000}%
\pgfsetfillcolor{currentfill}%
\pgfsetlinewidth{1.003750pt}%
\definecolor{currentstroke}{rgb}{0.000000,0.000000,0.000000}%
\pgfsetstrokecolor{currentstroke}%
\pgfsetdash{}{0pt}%
\pgfpathmoveto{\pgfqpoint{1.069417in}{2.427606in}}%
\pgfpathlineto{\pgfqpoint{1.926808in}{2.427606in}}%
\pgfpathlineto{\pgfqpoint{1.926808in}{3.080583in}}%
\pgfpathlineto{\pgfqpoint{1.069417in}{3.080583in}}%
\pgfpathlineto{\pgfqpoint{1.069417in}{2.427606in}}%
\pgfpathclose%
\pgfusepath{stroke,fill}%
\end{pgfscope}%
\begin{pgfscope}%
\pgfsetrectcap%
\pgfsetroundjoin%
\pgfsetlinewidth{1.003750pt}%
\definecolor{currentstroke}{rgb}{0.000000,0.000000,1.000000}%
\pgfsetstrokecolor{currentstroke}%
\pgfsetdash{}{0pt}%
\pgfpathmoveto{\pgfqpoint{1.166600in}{2.968161in}}%
\pgfpathlineto{\pgfqpoint{1.360967in}{2.968161in}}%
\pgfusepath{stroke}%
\end{pgfscope}%
\begin{pgfscope}%
\pgftext[left,bottom,x=1.513683in,y=2.890691in,rotate=0.000000]{{\sffamily\fontsize{9.996000}{11.995200}\selectfont spc}}
%
\end{pgfscope}%
\begin{pgfscope}%
\pgfsetrectcap%
\pgfsetroundjoin%
\pgfsetlinewidth{1.003750pt}%
\definecolor{currentstroke}{rgb}{0.000000,0.500000,0.000000}%
\pgfsetstrokecolor{currentstroke}%
\pgfsetdash{}{0pt}%
\pgfpathmoveto{\pgfqpoint{1.166600in}{2.764385in}}%
\pgfpathlineto{\pgfqpoint{1.360967in}{2.764385in}}%
\pgfusepath{stroke}%
\end{pgfscope}%
\begin{pgfscope}%
\pgftext[left,bottom,x=1.513683in,y=2.686915in,rotate=0.000000]{{\sffamily\fontsize{9.996000}{11.995200}\selectfont spce}}
%
\end{pgfscope}%
\begin{pgfscope}%
\pgfsetrectcap%
\pgfsetroundjoin%
\pgfsetlinewidth{1.003750pt}%
\definecolor{currentstroke}{rgb}{1.000000,0.000000,0.000000}%
\pgfsetstrokecolor{currentstroke}%
\pgfsetdash{}{0pt}%
\pgfpathmoveto{\pgfqpoint{1.166600in}{2.560609in}}%
\pgfpathlineto{\pgfqpoint{1.360967in}{2.560609in}}%
\pgfusepath{stroke}%
\end{pgfscope}%
\begin{pgfscope}%
\pgftext[left,bottom,x=1.513683in,y=2.483139in,rotate=0.000000]{{\sffamily\fontsize{9.996000}{11.995200}\selectfont tip3p}}
%
\end{pgfscope}%
\end{pgfpicture}%
\makeatother%
\endgroup%
}
    \caption{Radial distribution function. The peaks are marked with stars.} \label{fig:rdf}
\end{figure}

All water models give similar distances for the peaks. The height and visibility of the peaks is very different, however. \textit{SPC/E} produce the highest and \textit{TIP3P} the lowest maxima. \textit{SPC} is in between. 

\subsection{Hydrogen bond analysis}
\begin{figure}[H]
	\resizebox{\linewidth}{!}{%% Creator: Matplotlib, PGF backend
%%
%% To include the figure in your LaTeX document, write
%%   \input{<filename>.pgf}
%%
%% Make sure the required packages are loaded in your preamble
%%   \usepackage{pgf}
%%
%% Figures using additional raster images can only be included by \input if
%% they are in the same directory as the main LaTeX file. For loading figures
%% from other directories you can use the `import` package
%%   \usepackage{import}
%% and then include the figures with
%%   \import{<path to file>}{<filename>.pgf}
%%
%% Matplotlib used the following preamble
%%   \usepackage{fontspec}
%%   \setmainfont{DejaVu Serif}
%%   \setsansfont{DejaVu Sans}
%%   \setmonofont{DejaVu Sans Mono}
%%
\begingroup%
\makeatletter%
\begin{pgfpicture}%
\pgfpathrectangle{\pgfpointorigin}{\pgfqpoint{8.000000in}{3.500000in}}%
\pgfusepath{use as bounding box}%
\begin{pgfscope}%
\pgfsetrectcap%
\pgfsetroundjoin%
\definecolor{currentfill}{rgb}{1.000000,1.000000,1.000000}%
\pgfsetfillcolor{currentfill}%
\pgfsetlinewidth{0.000000pt}%
\definecolor{currentstroke}{rgb}{1.000000,1.000000,1.000000}%
\pgfsetstrokecolor{currentstroke}%
\pgfsetdash{}{0pt}%
\pgfpathmoveto{\pgfqpoint{0.000000in}{0.000000in}}%
\pgfpathlineto{\pgfqpoint{8.000000in}{0.000000in}}%
\pgfpathlineto{\pgfqpoint{8.000000in}{3.500000in}}%
\pgfpathlineto{\pgfqpoint{0.000000in}{3.500000in}}%
\pgfpathclose%
\pgfusepath{fill}%
\end{pgfscope}%
\begin{pgfscope}%
\pgfsetrectcap%
\pgfsetroundjoin%
\definecolor{currentfill}{rgb}{1.000000,1.000000,1.000000}%
\pgfsetfillcolor{currentfill}%
\pgfsetlinewidth{0.000000pt}%
\definecolor{currentstroke}{rgb}{0.000000,0.000000,0.000000}%
\pgfsetstrokecolor{currentstroke}%
\pgfsetdash{}{0pt}%
\pgfpathmoveto{\pgfqpoint{1.000000in}{0.350000in}}%
\pgfpathlineto{\pgfqpoint{7.200000in}{0.350000in}}%
\pgfpathlineto{\pgfqpoint{7.200000in}{3.150000in}}%
\pgfpathlineto{\pgfqpoint{1.000000in}{3.150000in}}%
\pgfpathclose%
\pgfusepath{fill}%
\end{pgfscope}%
\begin{pgfscope}%
\pgfpathrectangle{\pgfqpoint{1.000000in}{0.350000in}}{\pgfqpoint{6.200000in}{2.800000in}} %
\pgfusepath{clip}%
\pgfsetrectcap%
\pgfsetroundjoin%
\pgfsetlinewidth{1.003750pt}%
\definecolor{currentstroke}{rgb}{0.000000,0.000000,1.000000}%
\pgfsetstrokecolor{currentstroke}%
\pgfsetdash{}{0pt}%
\pgfpathmoveto{\pgfqpoint{1.000000in}{2.415000in}}%
\pgfpathlineto{\pgfqpoint{1.001240in}{2.170000in}}%
\pgfpathlineto{\pgfqpoint{1.002480in}{2.135000in}}%
\pgfpathlineto{\pgfqpoint{1.003720in}{2.695000in}}%
\pgfpathlineto{\pgfqpoint{1.004960in}{1.960000in}}%
\pgfpathlineto{\pgfqpoint{1.006200in}{2.030000in}}%
\pgfpathlineto{\pgfqpoint{1.007440in}{1.540000in}}%
\pgfpathlineto{\pgfqpoint{1.008680in}{1.855000in}}%
\pgfpathlineto{\pgfqpoint{1.009920in}{1.750000in}}%
\pgfpathlineto{\pgfqpoint{1.011160in}{1.750000in}}%
\pgfpathlineto{\pgfqpoint{1.012400in}{1.960000in}}%
\pgfpathlineto{\pgfqpoint{1.014880in}{1.785000in}}%
\pgfpathlineto{\pgfqpoint{1.016120in}{1.680000in}}%
\pgfpathlineto{\pgfqpoint{1.017360in}{1.785000in}}%
\pgfpathlineto{\pgfqpoint{1.018600in}{1.750000in}}%
\pgfpathlineto{\pgfqpoint{1.021080in}{2.205000in}}%
\pgfpathlineto{\pgfqpoint{1.022320in}{2.065000in}}%
\pgfpathlineto{\pgfqpoint{1.023560in}{1.540000in}}%
\pgfpathlineto{\pgfqpoint{1.024800in}{1.715000in}}%
\pgfpathlineto{\pgfqpoint{1.026040in}{2.135000in}}%
\pgfpathlineto{\pgfqpoint{1.027280in}{1.995000in}}%
\pgfpathlineto{\pgfqpoint{1.028520in}{2.345000in}}%
\pgfpathlineto{\pgfqpoint{1.031000in}{1.925000in}}%
\pgfpathlineto{\pgfqpoint{1.033480in}{2.240000in}}%
\pgfpathlineto{\pgfqpoint{1.034720in}{2.205000in}}%
\pgfpathlineto{\pgfqpoint{1.035960in}{1.855000in}}%
\pgfpathlineto{\pgfqpoint{1.038440in}{2.030000in}}%
\pgfpathlineto{\pgfqpoint{1.040920in}{1.855000in}}%
\pgfpathlineto{\pgfqpoint{1.042160in}{1.295000in}}%
\pgfpathlineto{\pgfqpoint{1.044640in}{1.925000in}}%
\pgfpathlineto{\pgfqpoint{1.045880in}{1.960000in}}%
\pgfpathlineto{\pgfqpoint{1.047120in}{1.295000in}}%
\pgfpathlineto{\pgfqpoint{1.052080in}{2.240000in}}%
\pgfpathlineto{\pgfqpoint{1.053320in}{1.680000in}}%
\pgfpathlineto{\pgfqpoint{1.055800in}{2.030000in}}%
\pgfpathlineto{\pgfqpoint{1.057040in}{2.100000in}}%
\pgfpathlineto{\pgfqpoint{1.058280in}{1.925000in}}%
\pgfpathlineto{\pgfqpoint{1.059520in}{2.170000in}}%
\pgfpathlineto{\pgfqpoint{1.060760in}{1.785000in}}%
\pgfpathlineto{\pgfqpoint{1.062000in}{1.995000in}}%
\pgfpathlineto{\pgfqpoint{1.064480in}{1.995000in}}%
\pgfpathlineto{\pgfqpoint{1.065720in}{2.065000in}}%
\pgfpathlineto{\pgfqpoint{1.066960in}{2.030000in}}%
\pgfpathlineto{\pgfqpoint{1.068200in}{2.100000in}}%
\pgfpathlineto{\pgfqpoint{1.070680in}{1.540000in}}%
\pgfpathlineto{\pgfqpoint{1.071920in}{1.505000in}}%
\pgfpathlineto{\pgfqpoint{1.073160in}{1.715000in}}%
\pgfpathlineto{\pgfqpoint{1.074400in}{2.135000in}}%
\pgfpathlineto{\pgfqpoint{1.075640in}{1.715000in}}%
\pgfpathlineto{\pgfqpoint{1.076880in}{2.240000in}}%
\pgfpathlineto{\pgfqpoint{1.078120in}{1.855000in}}%
\pgfpathlineto{\pgfqpoint{1.079360in}{1.855000in}}%
\pgfpathlineto{\pgfqpoint{1.080600in}{1.575000in}}%
\pgfpathlineto{\pgfqpoint{1.083080in}{2.100000in}}%
\pgfpathlineto{\pgfqpoint{1.084320in}{1.470000in}}%
\pgfpathlineto{\pgfqpoint{1.085560in}{2.240000in}}%
\pgfpathlineto{\pgfqpoint{1.088040in}{1.960000in}}%
\pgfpathlineto{\pgfqpoint{1.089280in}{1.855000in}}%
\pgfpathlineto{\pgfqpoint{1.091760in}{2.205000in}}%
\pgfpathlineto{\pgfqpoint{1.094240in}{1.855000in}}%
\pgfpathlineto{\pgfqpoint{1.095480in}{2.030000in}}%
\pgfpathlineto{\pgfqpoint{1.096720in}{1.960000in}}%
\pgfpathlineto{\pgfqpoint{1.099200in}{1.785000in}}%
\pgfpathlineto{\pgfqpoint{1.100440in}{2.030000in}}%
\pgfpathlineto{\pgfqpoint{1.101680in}{1.680000in}}%
\pgfpathlineto{\pgfqpoint{1.102920in}{1.715000in}}%
\pgfpathlineto{\pgfqpoint{1.104160in}{1.680000in}}%
\pgfpathlineto{\pgfqpoint{1.106640in}{1.995000in}}%
\pgfpathlineto{\pgfqpoint{1.107880in}{1.960000in}}%
\pgfpathlineto{\pgfqpoint{1.109120in}{2.065000in}}%
\pgfpathlineto{\pgfqpoint{1.110360in}{1.645000in}}%
\pgfpathlineto{\pgfqpoint{1.111600in}{2.205000in}}%
\pgfpathlineto{\pgfqpoint{1.112840in}{1.925000in}}%
\pgfpathlineto{\pgfqpoint{1.114080in}{1.925000in}}%
\pgfpathlineto{\pgfqpoint{1.115320in}{1.680000in}}%
\pgfpathlineto{\pgfqpoint{1.116560in}{1.750000in}}%
\pgfpathlineto{\pgfqpoint{1.117800in}{2.275000in}}%
\pgfpathlineto{\pgfqpoint{1.119040in}{1.855000in}}%
\pgfpathlineto{\pgfqpoint{1.121520in}{1.855000in}}%
\pgfpathlineto{\pgfqpoint{1.122760in}{1.680000in}}%
\pgfpathlineto{\pgfqpoint{1.124000in}{1.855000in}}%
\pgfpathlineto{\pgfqpoint{1.125240in}{1.750000in}}%
\pgfpathlineto{\pgfqpoint{1.127720in}{1.435000in}}%
\pgfpathlineto{\pgfqpoint{1.128960in}{1.890000in}}%
\pgfpathlineto{\pgfqpoint{1.130200in}{1.820000in}}%
\pgfpathlineto{\pgfqpoint{1.131440in}{1.925000in}}%
\pgfpathlineto{\pgfqpoint{1.132680in}{1.890000in}}%
\pgfpathlineto{\pgfqpoint{1.133920in}{2.170000in}}%
\pgfpathlineto{\pgfqpoint{1.135160in}{1.785000in}}%
\pgfpathlineto{\pgfqpoint{1.136400in}{1.820000in}}%
\pgfpathlineto{\pgfqpoint{1.137640in}{2.030000in}}%
\pgfpathlineto{\pgfqpoint{1.138880in}{1.610000in}}%
\pgfpathlineto{\pgfqpoint{1.140120in}{1.820000in}}%
\pgfpathlineto{\pgfqpoint{1.141360in}{1.680000in}}%
\pgfpathlineto{\pgfqpoint{1.142600in}{1.820000in}}%
\pgfpathlineto{\pgfqpoint{1.143840in}{1.785000in}}%
\pgfpathlineto{\pgfqpoint{1.145080in}{1.785000in}}%
\pgfpathlineto{\pgfqpoint{1.147560in}{2.275000in}}%
\pgfpathlineto{\pgfqpoint{1.148800in}{2.170000in}}%
\pgfpathlineto{\pgfqpoint{1.151280in}{1.715000in}}%
\pgfpathlineto{\pgfqpoint{1.152520in}{1.960000in}}%
\pgfpathlineto{\pgfqpoint{1.153760in}{1.260000in}}%
\pgfpathlineto{\pgfqpoint{1.156240in}{1.960000in}}%
\pgfpathlineto{\pgfqpoint{1.157480in}{1.925000in}}%
\pgfpathlineto{\pgfqpoint{1.158720in}{1.750000in}}%
\pgfpathlineto{\pgfqpoint{1.159960in}{1.960000in}}%
\pgfpathlineto{\pgfqpoint{1.161200in}{1.785000in}}%
\pgfpathlineto{\pgfqpoint{1.162440in}{1.855000in}}%
\pgfpathlineto{\pgfqpoint{1.163680in}{2.030000in}}%
\pgfpathlineto{\pgfqpoint{1.164920in}{1.820000in}}%
\pgfpathlineto{\pgfqpoint{1.166160in}{1.890000in}}%
\pgfpathlineto{\pgfqpoint{1.167400in}{1.890000in}}%
\pgfpathlineto{\pgfqpoint{1.168640in}{1.855000in}}%
\pgfpathlineto{\pgfqpoint{1.169880in}{1.890000in}}%
\pgfpathlineto{\pgfqpoint{1.171120in}{2.065000in}}%
\pgfpathlineto{\pgfqpoint{1.172360in}{1.960000in}}%
\pgfpathlineto{\pgfqpoint{1.173600in}{2.310000in}}%
\pgfpathlineto{\pgfqpoint{1.176080in}{1.995000in}}%
\pgfpathlineto{\pgfqpoint{1.177320in}{1.925000in}}%
\pgfpathlineto{\pgfqpoint{1.178560in}{2.030000in}}%
\pgfpathlineto{\pgfqpoint{1.179800in}{1.540000in}}%
\pgfpathlineto{\pgfqpoint{1.181040in}{1.610000in}}%
\pgfpathlineto{\pgfqpoint{1.182280in}{2.135000in}}%
\pgfpathlineto{\pgfqpoint{1.184760in}{1.610000in}}%
\pgfpathlineto{\pgfqpoint{1.187240in}{1.680000in}}%
\pgfpathlineto{\pgfqpoint{1.188480in}{1.680000in}}%
\pgfpathlineto{\pgfqpoint{1.189720in}{2.240000in}}%
\pgfpathlineto{\pgfqpoint{1.192200in}{1.680000in}}%
\pgfpathlineto{\pgfqpoint{1.193440in}{2.030000in}}%
\pgfpathlineto{\pgfqpoint{1.194680in}{1.820000in}}%
\pgfpathlineto{\pgfqpoint{1.195920in}{2.065000in}}%
\pgfpathlineto{\pgfqpoint{1.197160in}{2.065000in}}%
\pgfpathlineto{\pgfqpoint{1.198400in}{1.785000in}}%
\pgfpathlineto{\pgfqpoint{1.199640in}{1.820000in}}%
\pgfpathlineto{\pgfqpoint{1.200880in}{2.030000in}}%
\pgfpathlineto{\pgfqpoint{1.202120in}{1.995000in}}%
\pgfpathlineto{\pgfqpoint{1.203360in}{2.135000in}}%
\pgfpathlineto{\pgfqpoint{1.204600in}{2.135000in}}%
\pgfpathlineto{\pgfqpoint{1.205840in}{1.995000in}}%
\pgfpathlineto{\pgfqpoint{1.207080in}{2.205000in}}%
\pgfpathlineto{\pgfqpoint{1.209560in}{1.785000in}}%
\pgfpathlineto{\pgfqpoint{1.210800in}{2.170000in}}%
\pgfpathlineto{\pgfqpoint{1.212040in}{2.205000in}}%
\pgfpathlineto{\pgfqpoint{1.214520in}{1.960000in}}%
\pgfpathlineto{\pgfqpoint{1.215760in}{2.065000in}}%
\pgfpathlineto{\pgfqpoint{1.217000in}{1.645000in}}%
\pgfpathlineto{\pgfqpoint{1.220720in}{2.380000in}}%
\pgfpathlineto{\pgfqpoint{1.221960in}{1.995000in}}%
\pgfpathlineto{\pgfqpoint{1.223200in}{2.065000in}}%
\pgfpathlineto{\pgfqpoint{1.224440in}{2.345000in}}%
\pgfpathlineto{\pgfqpoint{1.225680in}{1.855000in}}%
\pgfpathlineto{\pgfqpoint{1.226920in}{2.065000in}}%
\pgfpathlineto{\pgfqpoint{1.228160in}{1.750000in}}%
\pgfpathlineto{\pgfqpoint{1.231880in}{2.485000in}}%
\pgfpathlineto{\pgfqpoint{1.235600in}{1.890000in}}%
\pgfpathlineto{\pgfqpoint{1.236840in}{2.030000in}}%
\pgfpathlineto{\pgfqpoint{1.238080in}{2.345000in}}%
\pgfpathlineto{\pgfqpoint{1.240560in}{2.100000in}}%
\pgfpathlineto{\pgfqpoint{1.241800in}{2.135000in}}%
\pgfpathlineto{\pgfqpoint{1.243040in}{2.100000in}}%
\pgfpathlineto{\pgfqpoint{1.245520in}{1.820000in}}%
\pgfpathlineto{\pgfqpoint{1.246760in}{1.995000in}}%
\pgfpathlineto{\pgfqpoint{1.250480in}{1.365000in}}%
\pgfpathlineto{\pgfqpoint{1.252960in}{1.750000in}}%
\pgfpathlineto{\pgfqpoint{1.254200in}{1.820000in}}%
\pgfpathlineto{\pgfqpoint{1.255440in}{1.505000in}}%
\pgfpathlineto{\pgfqpoint{1.256680in}{1.505000in}}%
\pgfpathlineto{\pgfqpoint{1.257920in}{1.855000in}}%
\pgfpathlineto{\pgfqpoint{1.259160in}{1.750000in}}%
\pgfpathlineto{\pgfqpoint{1.260400in}{1.960000in}}%
\pgfpathlineto{\pgfqpoint{1.261640in}{1.680000in}}%
\pgfpathlineto{\pgfqpoint{1.265360in}{2.380000in}}%
\pgfpathlineto{\pgfqpoint{1.266600in}{2.135000in}}%
\pgfpathlineto{\pgfqpoint{1.267840in}{2.170000in}}%
\pgfpathlineto{\pgfqpoint{1.269080in}{1.925000in}}%
\pgfpathlineto{\pgfqpoint{1.271560in}{2.310000in}}%
\pgfpathlineto{\pgfqpoint{1.272800in}{1.645000in}}%
\pgfpathlineto{\pgfqpoint{1.274040in}{1.890000in}}%
\pgfpathlineto{\pgfqpoint{1.275280in}{1.785000in}}%
\pgfpathlineto{\pgfqpoint{1.276520in}{2.030000in}}%
\pgfpathlineto{\pgfqpoint{1.277760in}{1.890000in}}%
\pgfpathlineto{\pgfqpoint{1.280240in}{2.240000in}}%
\pgfpathlineto{\pgfqpoint{1.281480in}{1.995000in}}%
\pgfpathlineto{\pgfqpoint{1.282720in}{1.995000in}}%
\pgfpathlineto{\pgfqpoint{1.283960in}{2.135000in}}%
\pgfpathlineto{\pgfqpoint{1.285200in}{1.820000in}}%
\pgfpathlineto{\pgfqpoint{1.286440in}{2.030000in}}%
\pgfpathlineto{\pgfqpoint{1.287680in}{1.855000in}}%
\pgfpathlineto{\pgfqpoint{1.290160in}{2.135000in}}%
\pgfpathlineto{\pgfqpoint{1.291400in}{1.820000in}}%
\pgfpathlineto{\pgfqpoint{1.292640in}{1.820000in}}%
\pgfpathlineto{\pgfqpoint{1.293880in}{1.995000in}}%
\pgfpathlineto{\pgfqpoint{1.295120in}{1.995000in}}%
\pgfpathlineto{\pgfqpoint{1.296360in}{2.135000in}}%
\pgfpathlineto{\pgfqpoint{1.297600in}{1.925000in}}%
\pgfpathlineto{\pgfqpoint{1.298840in}{2.415000in}}%
\pgfpathlineto{\pgfqpoint{1.300080in}{2.065000in}}%
\pgfpathlineto{\pgfqpoint{1.302560in}{2.275000in}}%
\pgfpathlineto{\pgfqpoint{1.303800in}{1.855000in}}%
\pgfpathlineto{\pgfqpoint{1.306280in}{2.345000in}}%
\pgfpathlineto{\pgfqpoint{1.307520in}{2.100000in}}%
\pgfpathlineto{\pgfqpoint{1.308760in}{1.575000in}}%
\pgfpathlineto{\pgfqpoint{1.312480in}{2.135000in}}%
\pgfpathlineto{\pgfqpoint{1.313720in}{1.505000in}}%
\pgfpathlineto{\pgfqpoint{1.316200in}{2.345000in}}%
\pgfpathlineto{\pgfqpoint{1.317440in}{1.645000in}}%
\pgfpathlineto{\pgfqpoint{1.319920in}{2.345000in}}%
\pgfpathlineto{\pgfqpoint{1.321160in}{1.925000in}}%
\pgfpathlineto{\pgfqpoint{1.322400in}{2.170000in}}%
\pgfpathlineto{\pgfqpoint{1.324880in}{1.610000in}}%
\pgfpathlineto{\pgfqpoint{1.326120in}{1.715000in}}%
\pgfpathlineto{\pgfqpoint{1.327360in}{1.785000in}}%
\pgfpathlineto{\pgfqpoint{1.328600in}{1.610000in}}%
\pgfpathlineto{\pgfqpoint{1.329840in}{1.855000in}}%
\pgfpathlineto{\pgfqpoint{1.332320in}{1.680000in}}%
\pgfpathlineto{\pgfqpoint{1.333560in}{2.170000in}}%
\pgfpathlineto{\pgfqpoint{1.334800in}{1.680000in}}%
\pgfpathlineto{\pgfqpoint{1.336040in}{1.855000in}}%
\pgfpathlineto{\pgfqpoint{1.337280in}{1.715000in}}%
\pgfpathlineto{\pgfqpoint{1.338520in}{1.820000in}}%
\pgfpathlineto{\pgfqpoint{1.339760in}{2.275000in}}%
\pgfpathlineto{\pgfqpoint{1.342240in}{1.505000in}}%
\pgfpathlineto{\pgfqpoint{1.347200in}{2.275000in}}%
\pgfpathlineto{\pgfqpoint{1.349680in}{1.680000in}}%
\pgfpathlineto{\pgfqpoint{1.350920in}{1.925000in}}%
\pgfpathlineto{\pgfqpoint{1.352160in}{1.680000in}}%
\pgfpathlineto{\pgfqpoint{1.353400in}{2.310000in}}%
\pgfpathlineto{\pgfqpoint{1.354640in}{1.890000in}}%
\pgfpathlineto{\pgfqpoint{1.357120in}{2.170000in}}%
\pgfpathlineto{\pgfqpoint{1.358360in}{1.715000in}}%
\pgfpathlineto{\pgfqpoint{1.359600in}{1.785000in}}%
\pgfpathlineto{\pgfqpoint{1.360840in}{2.100000in}}%
\pgfpathlineto{\pgfqpoint{1.363320in}{1.575000in}}%
\pgfpathlineto{\pgfqpoint{1.364560in}{1.680000in}}%
\pgfpathlineto{\pgfqpoint{1.365800in}{1.890000in}}%
\pgfpathlineto{\pgfqpoint{1.367040in}{1.890000in}}%
\pgfpathlineto{\pgfqpoint{1.368280in}{2.520000in}}%
\pgfpathlineto{\pgfqpoint{1.369520in}{2.065000in}}%
\pgfpathlineto{\pgfqpoint{1.370760in}{2.415000in}}%
\pgfpathlineto{\pgfqpoint{1.372000in}{2.205000in}}%
\pgfpathlineto{\pgfqpoint{1.373240in}{2.240000in}}%
\pgfpathlineto{\pgfqpoint{1.374480in}{2.310000in}}%
\pgfpathlineto{\pgfqpoint{1.375720in}{2.240000in}}%
\pgfpathlineto{\pgfqpoint{1.376960in}{1.855000in}}%
\pgfpathlineto{\pgfqpoint{1.378200in}{2.065000in}}%
\pgfpathlineto{\pgfqpoint{1.379440in}{1.960000in}}%
\pgfpathlineto{\pgfqpoint{1.380680in}{2.240000in}}%
\pgfpathlineto{\pgfqpoint{1.383160in}{2.240000in}}%
\pgfpathlineto{\pgfqpoint{1.385640in}{1.855000in}}%
\pgfpathlineto{\pgfqpoint{1.386880in}{2.030000in}}%
\pgfpathlineto{\pgfqpoint{1.388120in}{1.750000in}}%
\pgfpathlineto{\pgfqpoint{1.389360in}{1.785000in}}%
\pgfpathlineto{\pgfqpoint{1.390600in}{1.960000in}}%
\pgfpathlineto{\pgfqpoint{1.391840in}{1.925000in}}%
\pgfpathlineto{\pgfqpoint{1.393080in}{2.205000in}}%
\pgfpathlineto{\pgfqpoint{1.394320in}{2.205000in}}%
\pgfpathlineto{\pgfqpoint{1.396800in}{1.855000in}}%
\pgfpathlineto{\pgfqpoint{1.398040in}{2.170000in}}%
\pgfpathlineto{\pgfqpoint{1.399280in}{1.925000in}}%
\pgfpathlineto{\pgfqpoint{1.400520in}{1.960000in}}%
\pgfpathlineto{\pgfqpoint{1.403000in}{2.240000in}}%
\pgfpathlineto{\pgfqpoint{1.405480in}{1.365000in}}%
\pgfpathlineto{\pgfqpoint{1.406720in}{1.470000in}}%
\pgfpathlineto{\pgfqpoint{1.407960in}{1.470000in}}%
\pgfpathlineto{\pgfqpoint{1.410440in}{2.205000in}}%
\pgfpathlineto{\pgfqpoint{1.411680in}{2.310000in}}%
\pgfpathlineto{\pgfqpoint{1.414160in}{1.750000in}}%
\pgfpathlineto{\pgfqpoint{1.416640in}{1.995000in}}%
\pgfpathlineto{\pgfqpoint{1.417880in}{1.610000in}}%
\pgfpathlineto{\pgfqpoint{1.420360in}{2.170000in}}%
\pgfpathlineto{\pgfqpoint{1.421600in}{2.065000in}}%
\pgfpathlineto{\pgfqpoint{1.422840in}{1.680000in}}%
\pgfpathlineto{\pgfqpoint{1.424080in}{2.310000in}}%
\pgfpathlineto{\pgfqpoint{1.425320in}{1.820000in}}%
\pgfpathlineto{\pgfqpoint{1.427800in}{2.380000in}}%
\pgfpathlineto{\pgfqpoint{1.429040in}{2.485000in}}%
\pgfpathlineto{\pgfqpoint{1.430280in}{1.925000in}}%
\pgfpathlineto{\pgfqpoint{1.431520in}{2.100000in}}%
\pgfpathlineto{\pgfqpoint{1.432760in}{1.820000in}}%
\pgfpathlineto{\pgfqpoint{1.435240in}{1.995000in}}%
\pgfpathlineto{\pgfqpoint{1.437720in}{1.785000in}}%
\pgfpathlineto{\pgfqpoint{1.438960in}{2.100000in}}%
\pgfpathlineto{\pgfqpoint{1.440200in}{1.785000in}}%
\pgfpathlineto{\pgfqpoint{1.441440in}{1.785000in}}%
\pgfpathlineto{\pgfqpoint{1.442680in}{1.890000in}}%
\pgfpathlineto{\pgfqpoint{1.443920in}{1.470000in}}%
\pgfpathlineto{\pgfqpoint{1.446400in}{2.030000in}}%
\pgfpathlineto{\pgfqpoint{1.447640in}{1.925000in}}%
\pgfpathlineto{\pgfqpoint{1.448880in}{1.995000in}}%
\pgfpathlineto{\pgfqpoint{1.450120in}{1.645000in}}%
\pgfpathlineto{\pgfqpoint{1.452600in}{1.925000in}}%
\pgfpathlineto{\pgfqpoint{1.453840in}{1.890000in}}%
\pgfpathlineto{\pgfqpoint{1.455080in}{1.925000in}}%
\pgfpathlineto{\pgfqpoint{1.456320in}{2.205000in}}%
\pgfpathlineto{\pgfqpoint{1.458800in}{1.785000in}}%
\pgfpathlineto{\pgfqpoint{1.460040in}{1.820000in}}%
\pgfpathlineto{\pgfqpoint{1.461280in}{1.610000in}}%
\pgfpathlineto{\pgfqpoint{1.463760in}{2.240000in}}%
\pgfpathlineto{\pgfqpoint{1.465000in}{1.855000in}}%
\pgfpathlineto{\pgfqpoint{1.467480in}{2.030000in}}%
\pgfpathlineto{\pgfqpoint{1.468720in}{1.645000in}}%
\pgfpathlineto{\pgfqpoint{1.469960in}{1.890000in}}%
\pgfpathlineto{\pgfqpoint{1.471200in}{1.645000in}}%
\pgfpathlineto{\pgfqpoint{1.472440in}{1.890000in}}%
\pgfpathlineto{\pgfqpoint{1.473680in}{1.890000in}}%
\pgfpathlineto{\pgfqpoint{1.474920in}{2.100000in}}%
\pgfpathlineto{\pgfqpoint{1.476160in}{1.995000in}}%
\pgfpathlineto{\pgfqpoint{1.477400in}{1.505000in}}%
\pgfpathlineto{\pgfqpoint{1.479880in}{2.450000in}}%
\pgfpathlineto{\pgfqpoint{1.481120in}{1.610000in}}%
\pgfpathlineto{\pgfqpoint{1.483600in}{1.855000in}}%
\pgfpathlineto{\pgfqpoint{1.484840in}{1.715000in}}%
\pgfpathlineto{\pgfqpoint{1.486080in}{2.310000in}}%
\pgfpathlineto{\pgfqpoint{1.489800in}{1.400000in}}%
\pgfpathlineto{\pgfqpoint{1.491040in}{2.135000in}}%
\pgfpathlineto{\pgfqpoint{1.492280in}{1.505000in}}%
\pgfpathlineto{\pgfqpoint{1.494760in}{1.925000in}}%
\pgfpathlineto{\pgfqpoint{1.496000in}{1.610000in}}%
\pgfpathlineto{\pgfqpoint{1.498480in}{1.995000in}}%
\pgfpathlineto{\pgfqpoint{1.500960in}{1.400000in}}%
\pgfpathlineto{\pgfqpoint{1.502200in}{2.135000in}}%
\pgfpathlineto{\pgfqpoint{1.503440in}{1.750000in}}%
\pgfpathlineto{\pgfqpoint{1.504680in}{1.960000in}}%
\pgfpathlineto{\pgfqpoint{1.505920in}{1.855000in}}%
\pgfpathlineto{\pgfqpoint{1.507160in}{2.345000in}}%
\pgfpathlineto{\pgfqpoint{1.509640in}{1.680000in}}%
\pgfpathlineto{\pgfqpoint{1.512120in}{2.030000in}}%
\pgfpathlineto{\pgfqpoint{1.514600in}{1.575000in}}%
\pgfpathlineto{\pgfqpoint{1.515840in}{1.680000in}}%
\pgfpathlineto{\pgfqpoint{1.517080in}{1.540000in}}%
\pgfpathlineto{\pgfqpoint{1.518320in}{1.575000in}}%
\pgfpathlineto{\pgfqpoint{1.519560in}{1.400000in}}%
\pgfpathlineto{\pgfqpoint{1.520800in}{1.540000in}}%
\pgfpathlineto{\pgfqpoint{1.522040in}{1.890000in}}%
\pgfpathlineto{\pgfqpoint{1.523280in}{1.820000in}}%
\pgfpathlineto{\pgfqpoint{1.524520in}{1.855000in}}%
\pgfpathlineto{\pgfqpoint{1.527000in}{1.680000in}}%
\pgfpathlineto{\pgfqpoint{1.528240in}{1.505000in}}%
\pgfpathlineto{\pgfqpoint{1.530720in}{2.030000in}}%
\pgfpathlineto{\pgfqpoint{1.533200in}{1.890000in}}%
\pgfpathlineto{\pgfqpoint{1.534440in}{2.135000in}}%
\pgfpathlineto{\pgfqpoint{1.536920in}{1.680000in}}%
\pgfpathlineto{\pgfqpoint{1.538160in}{1.855000in}}%
\pgfpathlineto{\pgfqpoint{1.540640in}{1.715000in}}%
\pgfpathlineto{\pgfqpoint{1.541880in}{2.100000in}}%
\pgfpathlineto{\pgfqpoint{1.544360in}{1.785000in}}%
\pgfpathlineto{\pgfqpoint{1.545600in}{1.855000in}}%
\pgfpathlineto{\pgfqpoint{1.546840in}{1.855000in}}%
\pgfpathlineto{\pgfqpoint{1.548080in}{1.925000in}}%
\pgfpathlineto{\pgfqpoint{1.549320in}{1.925000in}}%
\pgfpathlineto{\pgfqpoint{1.550560in}{2.030000in}}%
\pgfpathlineto{\pgfqpoint{1.551800in}{1.750000in}}%
\pgfpathlineto{\pgfqpoint{1.553040in}{2.170000in}}%
\pgfpathlineto{\pgfqpoint{1.555520in}{1.680000in}}%
\pgfpathlineto{\pgfqpoint{1.556760in}{1.610000in}}%
\pgfpathlineto{\pgfqpoint{1.558000in}{2.205000in}}%
\pgfpathlineto{\pgfqpoint{1.559240in}{2.135000in}}%
\pgfpathlineto{\pgfqpoint{1.560480in}{2.170000in}}%
\pgfpathlineto{\pgfqpoint{1.561720in}{2.170000in}}%
\pgfpathlineto{\pgfqpoint{1.564200in}{1.435000in}}%
\pgfpathlineto{\pgfqpoint{1.566680in}{2.030000in}}%
\pgfpathlineto{\pgfqpoint{1.567920in}{1.610000in}}%
\pgfpathlineto{\pgfqpoint{1.569160in}{1.680000in}}%
\pgfpathlineto{\pgfqpoint{1.571640in}{2.135000in}}%
\pgfpathlineto{\pgfqpoint{1.572880in}{1.365000in}}%
\pgfpathlineto{\pgfqpoint{1.574120in}{1.995000in}}%
\pgfpathlineto{\pgfqpoint{1.576600in}{1.715000in}}%
\pgfpathlineto{\pgfqpoint{1.577840in}{2.450000in}}%
\pgfpathlineto{\pgfqpoint{1.579080in}{1.470000in}}%
\pgfpathlineto{\pgfqpoint{1.580320in}{1.995000in}}%
\pgfpathlineto{\pgfqpoint{1.582800in}{1.785000in}}%
\pgfpathlineto{\pgfqpoint{1.584040in}{1.925000in}}%
\pgfpathlineto{\pgfqpoint{1.585280in}{1.925000in}}%
\pgfpathlineto{\pgfqpoint{1.586520in}{1.610000in}}%
\pgfpathlineto{\pgfqpoint{1.590240in}{1.995000in}}%
\pgfpathlineto{\pgfqpoint{1.591480in}{1.925000in}}%
\pgfpathlineto{\pgfqpoint{1.592720in}{1.960000in}}%
\pgfpathlineto{\pgfqpoint{1.593960in}{1.715000in}}%
\pgfpathlineto{\pgfqpoint{1.595200in}{1.960000in}}%
\pgfpathlineto{\pgfqpoint{1.596440in}{1.890000in}}%
\pgfpathlineto{\pgfqpoint{1.597680in}{1.400000in}}%
\pgfpathlineto{\pgfqpoint{1.600160in}{1.890000in}}%
\pgfpathlineto{\pgfqpoint{1.601400in}{1.890000in}}%
\pgfpathlineto{\pgfqpoint{1.602640in}{1.995000in}}%
\pgfpathlineto{\pgfqpoint{1.603880in}{1.505000in}}%
\pgfpathlineto{\pgfqpoint{1.606360in}{1.855000in}}%
\pgfpathlineto{\pgfqpoint{1.607600in}{1.855000in}}%
\pgfpathlineto{\pgfqpoint{1.610080in}{1.960000in}}%
\pgfpathlineto{\pgfqpoint{1.611320in}{1.470000in}}%
\pgfpathlineto{\pgfqpoint{1.612560in}{1.505000in}}%
\pgfpathlineto{\pgfqpoint{1.615040in}{1.785000in}}%
\pgfpathlineto{\pgfqpoint{1.616280in}{2.310000in}}%
\pgfpathlineto{\pgfqpoint{1.617520in}{1.855000in}}%
\pgfpathlineto{\pgfqpoint{1.618760in}{2.240000in}}%
\pgfpathlineto{\pgfqpoint{1.620000in}{1.960000in}}%
\pgfpathlineto{\pgfqpoint{1.621240in}{1.995000in}}%
\pgfpathlineto{\pgfqpoint{1.622480in}{1.680000in}}%
\pgfpathlineto{\pgfqpoint{1.623720in}{1.890000in}}%
\pgfpathlineto{\pgfqpoint{1.624960in}{1.820000in}}%
\pgfpathlineto{\pgfqpoint{1.626200in}{2.100000in}}%
\pgfpathlineto{\pgfqpoint{1.627440in}{1.785000in}}%
\pgfpathlineto{\pgfqpoint{1.629920in}{2.100000in}}%
\pgfpathlineto{\pgfqpoint{1.631160in}{1.960000in}}%
\pgfpathlineto{\pgfqpoint{1.632400in}{1.960000in}}%
\pgfpathlineto{\pgfqpoint{1.633640in}{2.135000in}}%
\pgfpathlineto{\pgfqpoint{1.634880in}{1.715000in}}%
\pgfpathlineto{\pgfqpoint{1.636120in}{1.785000in}}%
\pgfpathlineto{\pgfqpoint{1.637360in}{2.065000in}}%
\pgfpathlineto{\pgfqpoint{1.639840in}{1.575000in}}%
\pgfpathlineto{\pgfqpoint{1.641080in}{1.855000in}}%
\pgfpathlineto{\pgfqpoint{1.643560in}{1.750000in}}%
\pgfpathlineto{\pgfqpoint{1.644800in}{1.540000in}}%
\pgfpathlineto{\pgfqpoint{1.646040in}{2.275000in}}%
\pgfpathlineto{\pgfqpoint{1.647280in}{1.995000in}}%
\pgfpathlineto{\pgfqpoint{1.648520in}{2.380000in}}%
\pgfpathlineto{\pgfqpoint{1.651000in}{1.540000in}}%
\pgfpathlineto{\pgfqpoint{1.653480in}{2.205000in}}%
\pgfpathlineto{\pgfqpoint{1.655960in}{1.855000in}}%
\pgfpathlineto{\pgfqpoint{1.657200in}{1.575000in}}%
\pgfpathlineto{\pgfqpoint{1.658440in}{2.065000in}}%
\pgfpathlineto{\pgfqpoint{1.659680in}{1.820000in}}%
\pgfpathlineto{\pgfqpoint{1.660920in}{2.310000in}}%
\pgfpathlineto{\pgfqpoint{1.662160in}{1.960000in}}%
\pgfpathlineto{\pgfqpoint{1.663400in}{1.925000in}}%
\pgfpathlineto{\pgfqpoint{1.664640in}{2.170000in}}%
\pgfpathlineto{\pgfqpoint{1.665880in}{1.855000in}}%
\pgfpathlineto{\pgfqpoint{1.668360in}{1.925000in}}%
\pgfpathlineto{\pgfqpoint{1.669600in}{1.925000in}}%
\pgfpathlineto{\pgfqpoint{1.670840in}{1.645000in}}%
\pgfpathlineto{\pgfqpoint{1.673320in}{2.100000in}}%
\pgfpathlineto{\pgfqpoint{1.675800in}{1.575000in}}%
\pgfpathlineto{\pgfqpoint{1.677040in}{1.645000in}}%
\pgfpathlineto{\pgfqpoint{1.679520in}{2.240000in}}%
\pgfpathlineto{\pgfqpoint{1.680760in}{1.610000in}}%
\pgfpathlineto{\pgfqpoint{1.682000in}{1.995000in}}%
\pgfpathlineto{\pgfqpoint{1.683240in}{1.960000in}}%
\pgfpathlineto{\pgfqpoint{1.684480in}{2.205000in}}%
\pgfpathlineto{\pgfqpoint{1.686960in}{1.960000in}}%
\pgfpathlineto{\pgfqpoint{1.688200in}{1.925000in}}%
\pgfpathlineto{\pgfqpoint{1.689440in}{2.065000in}}%
\pgfpathlineto{\pgfqpoint{1.690680in}{2.345000in}}%
\pgfpathlineto{\pgfqpoint{1.691920in}{1.680000in}}%
\pgfpathlineto{\pgfqpoint{1.693160in}{2.030000in}}%
\pgfpathlineto{\pgfqpoint{1.695640in}{1.890000in}}%
\pgfpathlineto{\pgfqpoint{1.696880in}{2.205000in}}%
\pgfpathlineto{\pgfqpoint{1.698120in}{1.540000in}}%
\pgfpathlineto{\pgfqpoint{1.699360in}{2.345000in}}%
\pgfpathlineto{\pgfqpoint{1.701840in}{1.645000in}}%
\pgfpathlineto{\pgfqpoint{1.703080in}{1.960000in}}%
\pgfpathlineto{\pgfqpoint{1.704320in}{1.890000in}}%
\pgfpathlineto{\pgfqpoint{1.705560in}{1.575000in}}%
\pgfpathlineto{\pgfqpoint{1.706800in}{1.575000in}}%
\pgfpathlineto{\pgfqpoint{1.708040in}{2.030000in}}%
\pgfpathlineto{\pgfqpoint{1.709280in}{1.435000in}}%
\pgfpathlineto{\pgfqpoint{1.710520in}{1.960000in}}%
\pgfpathlineto{\pgfqpoint{1.711760in}{1.610000in}}%
\pgfpathlineto{\pgfqpoint{1.713000in}{1.645000in}}%
\pgfpathlineto{\pgfqpoint{1.714240in}{1.960000in}}%
\pgfpathlineto{\pgfqpoint{1.716720in}{1.435000in}}%
\pgfpathlineto{\pgfqpoint{1.717960in}{1.575000in}}%
\pgfpathlineto{\pgfqpoint{1.720440in}{2.030000in}}%
\pgfpathlineto{\pgfqpoint{1.721680in}{1.645000in}}%
\pgfpathlineto{\pgfqpoint{1.722920in}{1.680000in}}%
\pgfpathlineto{\pgfqpoint{1.724160in}{1.645000in}}%
\pgfpathlineto{\pgfqpoint{1.727880in}{2.170000in}}%
\pgfpathlineto{\pgfqpoint{1.730360in}{1.540000in}}%
\pgfpathlineto{\pgfqpoint{1.731600in}{1.960000in}}%
\pgfpathlineto{\pgfqpoint{1.732840in}{1.715000in}}%
\pgfpathlineto{\pgfqpoint{1.734080in}{2.205000in}}%
\pgfpathlineto{\pgfqpoint{1.736560in}{1.715000in}}%
\pgfpathlineto{\pgfqpoint{1.737800in}{2.310000in}}%
\pgfpathlineto{\pgfqpoint{1.739040in}{1.960000in}}%
\pgfpathlineto{\pgfqpoint{1.740280in}{2.030000in}}%
\pgfpathlineto{\pgfqpoint{1.742760in}{1.785000in}}%
\pgfpathlineto{\pgfqpoint{1.745240in}{2.100000in}}%
\pgfpathlineto{\pgfqpoint{1.746480in}{1.995000in}}%
\pgfpathlineto{\pgfqpoint{1.747720in}{2.065000in}}%
\pgfpathlineto{\pgfqpoint{1.748960in}{2.345000in}}%
\pgfpathlineto{\pgfqpoint{1.750200in}{2.065000in}}%
\pgfpathlineto{\pgfqpoint{1.751440in}{2.380000in}}%
\pgfpathlineto{\pgfqpoint{1.752680in}{2.170000in}}%
\pgfpathlineto{\pgfqpoint{1.753920in}{2.310000in}}%
\pgfpathlineto{\pgfqpoint{1.755160in}{2.240000in}}%
\pgfpathlineto{\pgfqpoint{1.757640in}{1.820000in}}%
\pgfpathlineto{\pgfqpoint{1.758880in}{1.960000in}}%
\pgfpathlineto{\pgfqpoint{1.761360in}{1.820000in}}%
\pgfpathlineto{\pgfqpoint{1.762600in}{1.890000in}}%
\pgfpathlineto{\pgfqpoint{1.763840in}{1.750000in}}%
\pgfpathlineto{\pgfqpoint{1.765080in}{2.065000in}}%
\pgfpathlineto{\pgfqpoint{1.767560in}{1.715000in}}%
\pgfpathlineto{\pgfqpoint{1.768800in}{1.820000in}}%
\pgfpathlineto{\pgfqpoint{1.770040in}{1.820000in}}%
\pgfpathlineto{\pgfqpoint{1.772520in}{2.310000in}}%
\pgfpathlineto{\pgfqpoint{1.773760in}{1.855000in}}%
\pgfpathlineto{\pgfqpoint{1.775000in}{1.960000in}}%
\pgfpathlineto{\pgfqpoint{1.776240in}{1.890000in}}%
\pgfpathlineto{\pgfqpoint{1.777480in}{1.680000in}}%
\pgfpathlineto{\pgfqpoint{1.778720in}{1.715000in}}%
\pgfpathlineto{\pgfqpoint{1.779960in}{1.855000in}}%
\pgfpathlineto{\pgfqpoint{1.781200in}{1.540000in}}%
\pgfpathlineto{\pgfqpoint{1.782440in}{2.030000in}}%
\pgfpathlineto{\pgfqpoint{1.783680in}{2.065000in}}%
\pgfpathlineto{\pgfqpoint{1.784920in}{1.890000in}}%
\pgfpathlineto{\pgfqpoint{1.786160in}{2.065000in}}%
\pgfpathlineto{\pgfqpoint{1.787400in}{1.925000in}}%
\pgfpathlineto{\pgfqpoint{1.788640in}{2.030000in}}%
\pgfpathlineto{\pgfqpoint{1.789880in}{2.240000in}}%
\pgfpathlineto{\pgfqpoint{1.791120in}{2.065000in}}%
\pgfpathlineto{\pgfqpoint{1.792360in}{2.275000in}}%
\pgfpathlineto{\pgfqpoint{1.793600in}{2.100000in}}%
\pgfpathlineto{\pgfqpoint{1.794840in}{1.750000in}}%
\pgfpathlineto{\pgfqpoint{1.796080in}{1.820000in}}%
\pgfpathlineto{\pgfqpoint{1.797320in}{2.135000in}}%
\pgfpathlineto{\pgfqpoint{1.798560in}{1.890000in}}%
\pgfpathlineto{\pgfqpoint{1.799800in}{2.240000in}}%
\pgfpathlineto{\pgfqpoint{1.802280in}{1.890000in}}%
\pgfpathlineto{\pgfqpoint{1.803520in}{2.100000in}}%
\pgfpathlineto{\pgfqpoint{1.804760in}{1.575000in}}%
\pgfpathlineto{\pgfqpoint{1.806000in}{1.925000in}}%
\pgfpathlineto{\pgfqpoint{1.807240in}{1.820000in}}%
\pgfpathlineto{\pgfqpoint{1.808480in}{1.890000in}}%
\pgfpathlineto{\pgfqpoint{1.809720in}{1.820000in}}%
\pgfpathlineto{\pgfqpoint{1.810960in}{1.365000in}}%
\pgfpathlineto{\pgfqpoint{1.813440in}{1.925000in}}%
\pgfpathlineto{\pgfqpoint{1.814680in}{1.785000in}}%
\pgfpathlineto{\pgfqpoint{1.815920in}{1.995000in}}%
\pgfpathlineto{\pgfqpoint{1.817160in}{1.820000in}}%
\pgfpathlineto{\pgfqpoint{1.818400in}{1.470000in}}%
\pgfpathlineto{\pgfqpoint{1.819640in}{1.995000in}}%
\pgfpathlineto{\pgfqpoint{1.820880in}{1.820000in}}%
\pgfpathlineto{\pgfqpoint{1.823360in}{2.030000in}}%
\pgfpathlineto{\pgfqpoint{1.824600in}{1.505000in}}%
\pgfpathlineto{\pgfqpoint{1.825840in}{2.100000in}}%
\pgfpathlineto{\pgfqpoint{1.827080in}{1.855000in}}%
\pgfpathlineto{\pgfqpoint{1.828320in}{1.890000in}}%
\pgfpathlineto{\pgfqpoint{1.829560in}{2.240000in}}%
\pgfpathlineto{\pgfqpoint{1.830800in}{1.995000in}}%
\pgfpathlineto{\pgfqpoint{1.832040in}{1.505000in}}%
\pgfpathlineto{\pgfqpoint{1.833280in}{2.275000in}}%
\pgfpathlineto{\pgfqpoint{1.834520in}{1.645000in}}%
\pgfpathlineto{\pgfqpoint{1.835760in}{2.100000in}}%
\pgfpathlineto{\pgfqpoint{1.837000in}{2.065000in}}%
\pgfpathlineto{\pgfqpoint{1.838240in}{2.205000in}}%
\pgfpathlineto{\pgfqpoint{1.839480in}{1.785000in}}%
\pgfpathlineto{\pgfqpoint{1.840720in}{2.345000in}}%
\pgfpathlineto{\pgfqpoint{1.843200in}{1.680000in}}%
\pgfpathlineto{\pgfqpoint{1.844440in}{1.750000in}}%
\pgfpathlineto{\pgfqpoint{1.846920in}{2.135000in}}%
\pgfpathlineto{\pgfqpoint{1.848160in}{1.680000in}}%
\pgfpathlineto{\pgfqpoint{1.849400in}{2.135000in}}%
\pgfpathlineto{\pgfqpoint{1.850640in}{1.680000in}}%
\pgfpathlineto{\pgfqpoint{1.851880in}{1.645000in}}%
\pgfpathlineto{\pgfqpoint{1.854360in}{1.505000in}}%
\pgfpathlineto{\pgfqpoint{1.855600in}{1.960000in}}%
\pgfpathlineto{\pgfqpoint{1.858080in}{1.645000in}}%
\pgfpathlineto{\pgfqpoint{1.859320in}{1.645000in}}%
\pgfpathlineto{\pgfqpoint{1.860560in}{1.960000in}}%
\pgfpathlineto{\pgfqpoint{1.861800in}{1.715000in}}%
\pgfpathlineto{\pgfqpoint{1.864280in}{1.785000in}}%
\pgfpathlineto{\pgfqpoint{1.865520in}{1.435000in}}%
\pgfpathlineto{\pgfqpoint{1.866760in}{1.855000in}}%
\pgfpathlineto{\pgfqpoint{1.868000in}{1.575000in}}%
\pgfpathlineto{\pgfqpoint{1.870480in}{2.275000in}}%
\pgfpathlineto{\pgfqpoint{1.871720in}{1.785000in}}%
\pgfpathlineto{\pgfqpoint{1.874200in}{2.030000in}}%
\pgfpathlineto{\pgfqpoint{1.875440in}{2.030000in}}%
\pgfpathlineto{\pgfqpoint{1.876680in}{1.715000in}}%
\pgfpathlineto{\pgfqpoint{1.877920in}{1.995000in}}%
\pgfpathlineto{\pgfqpoint{1.879160in}{1.785000in}}%
\pgfpathlineto{\pgfqpoint{1.881640in}{2.240000in}}%
\pgfpathlineto{\pgfqpoint{1.882880in}{1.540000in}}%
\pgfpathlineto{\pgfqpoint{1.885360in}{2.030000in}}%
\pgfpathlineto{\pgfqpoint{1.886600in}{1.925000in}}%
\pgfpathlineto{\pgfqpoint{1.889080in}{1.365000in}}%
\pgfpathlineto{\pgfqpoint{1.891560in}{2.135000in}}%
\pgfpathlineto{\pgfqpoint{1.892800in}{1.435000in}}%
\pgfpathlineto{\pgfqpoint{1.895280in}{1.750000in}}%
\pgfpathlineto{\pgfqpoint{1.896520in}{1.750000in}}%
\pgfpathlineto{\pgfqpoint{1.897760in}{1.995000in}}%
\pgfpathlineto{\pgfqpoint{1.899000in}{1.680000in}}%
\pgfpathlineto{\pgfqpoint{1.900240in}{1.715000in}}%
\pgfpathlineto{\pgfqpoint{1.901480in}{1.645000in}}%
\pgfpathlineto{\pgfqpoint{1.902720in}{1.785000in}}%
\pgfpathlineto{\pgfqpoint{1.905200in}{2.415000in}}%
\pgfpathlineto{\pgfqpoint{1.906440in}{1.715000in}}%
\pgfpathlineto{\pgfqpoint{1.907680in}{1.785000in}}%
\pgfpathlineto{\pgfqpoint{1.908920in}{1.785000in}}%
\pgfpathlineto{\pgfqpoint{1.910160in}{1.470000in}}%
\pgfpathlineto{\pgfqpoint{1.912640in}{1.855000in}}%
\pgfpathlineto{\pgfqpoint{1.913880in}{1.890000in}}%
\pgfpathlineto{\pgfqpoint{1.915120in}{1.820000in}}%
\pgfpathlineto{\pgfqpoint{1.916360in}{1.925000in}}%
\pgfpathlineto{\pgfqpoint{1.917600in}{1.925000in}}%
\pgfpathlineto{\pgfqpoint{1.918840in}{1.890000in}}%
\pgfpathlineto{\pgfqpoint{1.920080in}{1.645000in}}%
\pgfpathlineto{\pgfqpoint{1.921320in}{1.750000in}}%
\pgfpathlineto{\pgfqpoint{1.922560in}{1.610000in}}%
\pgfpathlineto{\pgfqpoint{1.923800in}{1.785000in}}%
\pgfpathlineto{\pgfqpoint{1.925040in}{1.680000in}}%
\pgfpathlineto{\pgfqpoint{1.926280in}{1.820000in}}%
\pgfpathlineto{\pgfqpoint{1.927520in}{1.435000in}}%
\pgfpathlineto{\pgfqpoint{1.931240in}{2.345000in}}%
\pgfpathlineto{\pgfqpoint{1.933720in}{1.750000in}}%
\pgfpathlineto{\pgfqpoint{1.934960in}{1.610000in}}%
\pgfpathlineto{\pgfqpoint{1.936200in}{1.820000in}}%
\pgfpathlineto{\pgfqpoint{1.937440in}{1.715000in}}%
\pgfpathlineto{\pgfqpoint{1.938680in}{1.715000in}}%
\pgfpathlineto{\pgfqpoint{1.939920in}{1.225000in}}%
\pgfpathlineto{\pgfqpoint{1.942400in}{1.855000in}}%
\pgfpathlineto{\pgfqpoint{1.943640in}{1.855000in}}%
\pgfpathlineto{\pgfqpoint{1.944880in}{1.960000in}}%
\pgfpathlineto{\pgfqpoint{1.946120in}{1.925000in}}%
\pgfpathlineto{\pgfqpoint{1.947360in}{1.995000in}}%
\pgfpathlineto{\pgfqpoint{1.948600in}{1.645000in}}%
\pgfpathlineto{\pgfqpoint{1.951080in}{2.380000in}}%
\pgfpathlineto{\pgfqpoint{1.954800in}{1.330000in}}%
\pgfpathlineto{\pgfqpoint{1.956040in}{1.750000in}}%
\pgfpathlineto{\pgfqpoint{1.957280in}{1.575000in}}%
\pgfpathlineto{\pgfqpoint{1.958520in}{1.925000in}}%
\pgfpathlineto{\pgfqpoint{1.959760in}{1.785000in}}%
\pgfpathlineto{\pgfqpoint{1.961000in}{1.960000in}}%
\pgfpathlineto{\pgfqpoint{1.962240in}{1.890000in}}%
\pgfpathlineto{\pgfqpoint{1.963480in}{1.750000in}}%
\pgfpathlineto{\pgfqpoint{1.964720in}{1.855000in}}%
\pgfpathlineto{\pgfqpoint{1.965960in}{1.645000in}}%
\pgfpathlineto{\pgfqpoint{1.967200in}{1.890000in}}%
\pgfpathlineto{\pgfqpoint{1.968440in}{1.890000in}}%
\pgfpathlineto{\pgfqpoint{1.969680in}{1.960000in}}%
\pgfpathlineto{\pgfqpoint{1.970920in}{1.925000in}}%
\pgfpathlineto{\pgfqpoint{1.972160in}{1.820000in}}%
\pgfpathlineto{\pgfqpoint{1.973400in}{2.205000in}}%
\pgfpathlineto{\pgfqpoint{1.974640in}{2.030000in}}%
\pgfpathlineto{\pgfqpoint{1.975880in}{1.505000in}}%
\pgfpathlineto{\pgfqpoint{1.980840in}{2.100000in}}%
\pgfpathlineto{\pgfqpoint{1.982080in}{1.680000in}}%
\pgfpathlineto{\pgfqpoint{1.983320in}{1.960000in}}%
\pgfpathlineto{\pgfqpoint{1.985800in}{1.470000in}}%
\pgfpathlineto{\pgfqpoint{1.988280in}{1.855000in}}%
\pgfpathlineto{\pgfqpoint{1.989520in}{1.855000in}}%
\pgfpathlineto{\pgfqpoint{1.990760in}{2.170000in}}%
\pgfpathlineto{\pgfqpoint{1.992000in}{1.820000in}}%
\pgfpathlineto{\pgfqpoint{1.993240in}{2.170000in}}%
\pgfpathlineto{\pgfqpoint{1.994480in}{1.995000in}}%
\pgfpathlineto{\pgfqpoint{1.995720in}{1.400000in}}%
\pgfpathlineto{\pgfqpoint{1.998200in}{1.820000in}}%
\pgfpathlineto{\pgfqpoint{2.000680in}{1.610000in}}%
\pgfpathlineto{\pgfqpoint{2.001920in}{1.680000in}}%
\pgfpathlineto{\pgfqpoint{2.004400in}{2.030000in}}%
\pgfpathlineto{\pgfqpoint{2.005640in}{1.820000in}}%
\pgfpathlineto{\pgfqpoint{2.008120in}{2.240000in}}%
\pgfpathlineto{\pgfqpoint{2.009360in}{1.925000in}}%
\pgfpathlineto{\pgfqpoint{2.010600in}{2.030000in}}%
\pgfpathlineto{\pgfqpoint{2.011840in}{1.995000in}}%
\pgfpathlineto{\pgfqpoint{2.014320in}{1.715000in}}%
\pgfpathlineto{\pgfqpoint{2.015560in}{1.960000in}}%
\pgfpathlineto{\pgfqpoint{2.016800in}{1.890000in}}%
\pgfpathlineto{\pgfqpoint{2.018040in}{2.240000in}}%
\pgfpathlineto{\pgfqpoint{2.019280in}{1.610000in}}%
\pgfpathlineto{\pgfqpoint{2.020520in}{2.065000in}}%
\pgfpathlineto{\pgfqpoint{2.021760in}{1.855000in}}%
\pgfpathlineto{\pgfqpoint{2.023000in}{1.925000in}}%
\pgfpathlineto{\pgfqpoint{2.024240in}{2.205000in}}%
\pgfpathlineto{\pgfqpoint{2.025480in}{2.135000in}}%
\pgfpathlineto{\pgfqpoint{2.026720in}{2.135000in}}%
\pgfpathlineto{\pgfqpoint{2.029200in}{1.925000in}}%
\pgfpathlineto{\pgfqpoint{2.030440in}{2.240000in}}%
\pgfpathlineto{\pgfqpoint{2.031680in}{2.205000in}}%
\pgfpathlineto{\pgfqpoint{2.034160in}{1.750000in}}%
\pgfpathlineto{\pgfqpoint{2.035400in}{1.925000in}}%
\pgfpathlineto{\pgfqpoint{2.036640in}{1.610000in}}%
\pgfpathlineto{\pgfqpoint{2.037880in}{2.100000in}}%
\pgfpathlineto{\pgfqpoint{2.040360in}{1.715000in}}%
\pgfpathlineto{\pgfqpoint{2.041600in}{1.855000in}}%
\pgfpathlineto{\pgfqpoint{2.042840in}{1.855000in}}%
\pgfpathlineto{\pgfqpoint{2.044080in}{2.030000in}}%
\pgfpathlineto{\pgfqpoint{2.045320in}{1.785000in}}%
\pgfpathlineto{\pgfqpoint{2.046560in}{2.030000in}}%
\pgfpathlineto{\pgfqpoint{2.047800in}{1.960000in}}%
\pgfpathlineto{\pgfqpoint{2.049040in}{1.750000in}}%
\pgfpathlineto{\pgfqpoint{2.050280in}{1.925000in}}%
\pgfpathlineto{\pgfqpoint{2.051520in}{1.925000in}}%
\pgfpathlineto{\pgfqpoint{2.055240in}{1.365000in}}%
\pgfpathlineto{\pgfqpoint{2.057720in}{2.030000in}}%
\pgfpathlineto{\pgfqpoint{2.058960in}{1.400000in}}%
\pgfpathlineto{\pgfqpoint{2.060200in}{1.820000in}}%
\pgfpathlineto{\pgfqpoint{2.062680in}{1.365000in}}%
\pgfpathlineto{\pgfqpoint{2.063920in}{1.785000in}}%
\pgfpathlineto{\pgfqpoint{2.065160in}{1.820000in}}%
\pgfpathlineto{\pgfqpoint{2.066400in}{2.030000in}}%
\pgfpathlineto{\pgfqpoint{2.068880in}{1.645000in}}%
\pgfpathlineto{\pgfqpoint{2.070120in}{1.995000in}}%
\pgfpathlineto{\pgfqpoint{2.071360in}{1.750000in}}%
\pgfpathlineto{\pgfqpoint{2.072600in}{1.890000in}}%
\pgfpathlineto{\pgfqpoint{2.075080in}{1.750000in}}%
\pgfpathlineto{\pgfqpoint{2.076320in}{1.785000in}}%
\pgfpathlineto{\pgfqpoint{2.078800in}{1.960000in}}%
\pgfpathlineto{\pgfqpoint{2.080040in}{1.785000in}}%
\pgfpathlineto{\pgfqpoint{2.081280in}{2.065000in}}%
\pgfpathlineto{\pgfqpoint{2.082520in}{1.960000in}}%
\pgfpathlineto{\pgfqpoint{2.083760in}{2.100000in}}%
\pgfpathlineto{\pgfqpoint{2.085000in}{1.820000in}}%
\pgfpathlineto{\pgfqpoint{2.086240in}{1.855000in}}%
\pgfpathlineto{\pgfqpoint{2.087480in}{1.995000in}}%
\pgfpathlineto{\pgfqpoint{2.088720in}{1.855000in}}%
\pgfpathlineto{\pgfqpoint{2.089960in}{1.540000in}}%
\pgfpathlineto{\pgfqpoint{2.092440in}{2.100000in}}%
\pgfpathlineto{\pgfqpoint{2.094920in}{1.750000in}}%
\pgfpathlineto{\pgfqpoint{2.096160in}{1.855000in}}%
\pgfpathlineto{\pgfqpoint{2.097400in}{2.065000in}}%
\pgfpathlineto{\pgfqpoint{2.098640in}{1.855000in}}%
\pgfpathlineto{\pgfqpoint{2.099880in}{2.030000in}}%
\pgfpathlineto{\pgfqpoint{2.101120in}{1.925000in}}%
\pgfpathlineto{\pgfqpoint{2.102360in}{2.170000in}}%
\pgfpathlineto{\pgfqpoint{2.103600in}{2.170000in}}%
\pgfpathlineto{\pgfqpoint{2.104840in}{1.925000in}}%
\pgfpathlineto{\pgfqpoint{2.106080in}{1.995000in}}%
\pgfpathlineto{\pgfqpoint{2.107320in}{1.680000in}}%
\pgfpathlineto{\pgfqpoint{2.108560in}{1.890000in}}%
\pgfpathlineto{\pgfqpoint{2.109800in}{1.890000in}}%
\pgfpathlineto{\pgfqpoint{2.112280in}{1.960000in}}%
\pgfpathlineto{\pgfqpoint{2.113520in}{1.610000in}}%
\pgfpathlineto{\pgfqpoint{2.116000in}{2.065000in}}%
\pgfpathlineto{\pgfqpoint{2.117240in}{1.890000in}}%
\pgfpathlineto{\pgfqpoint{2.118480in}{1.890000in}}%
\pgfpathlineto{\pgfqpoint{2.119720in}{1.855000in}}%
\pgfpathlineto{\pgfqpoint{2.120960in}{1.435000in}}%
\pgfpathlineto{\pgfqpoint{2.124680in}{2.240000in}}%
\pgfpathlineto{\pgfqpoint{2.127160in}{1.820000in}}%
\pgfpathlineto{\pgfqpoint{2.128400in}{1.820000in}}%
\pgfpathlineto{\pgfqpoint{2.130880in}{2.380000in}}%
\pgfpathlineto{\pgfqpoint{2.132120in}{1.610000in}}%
\pgfpathlineto{\pgfqpoint{2.134600in}{1.960000in}}%
\pgfpathlineto{\pgfqpoint{2.135840in}{1.750000in}}%
\pgfpathlineto{\pgfqpoint{2.138320in}{2.030000in}}%
\pgfpathlineto{\pgfqpoint{2.139560in}{1.715000in}}%
\pgfpathlineto{\pgfqpoint{2.140800in}{2.135000in}}%
\pgfpathlineto{\pgfqpoint{2.142040in}{1.855000in}}%
\pgfpathlineto{\pgfqpoint{2.143280in}{1.960000in}}%
\pgfpathlineto{\pgfqpoint{2.147000in}{1.855000in}}%
\pgfpathlineto{\pgfqpoint{2.148240in}{1.855000in}}%
\pgfpathlineto{\pgfqpoint{2.149480in}{1.820000in}}%
\pgfpathlineto{\pgfqpoint{2.150720in}{1.890000in}}%
\pgfpathlineto{\pgfqpoint{2.151960in}{2.030000in}}%
\pgfpathlineto{\pgfqpoint{2.153200in}{1.855000in}}%
\pgfpathlineto{\pgfqpoint{2.154440in}{2.065000in}}%
\pgfpathlineto{\pgfqpoint{2.155680in}{1.610000in}}%
\pgfpathlineto{\pgfqpoint{2.159400in}{2.100000in}}%
\pgfpathlineto{\pgfqpoint{2.160640in}{1.435000in}}%
\pgfpathlineto{\pgfqpoint{2.161880in}{2.170000in}}%
\pgfpathlineto{\pgfqpoint{2.164360in}{1.820000in}}%
\pgfpathlineto{\pgfqpoint{2.165600in}{1.890000in}}%
\pgfpathlineto{\pgfqpoint{2.166840in}{2.310000in}}%
\pgfpathlineto{\pgfqpoint{2.168080in}{1.960000in}}%
\pgfpathlineto{\pgfqpoint{2.170560in}{2.240000in}}%
\pgfpathlineto{\pgfqpoint{2.171800in}{1.925000in}}%
\pgfpathlineto{\pgfqpoint{2.173040in}{2.380000in}}%
\pgfpathlineto{\pgfqpoint{2.174280in}{2.065000in}}%
\pgfpathlineto{\pgfqpoint{2.175520in}{2.205000in}}%
\pgfpathlineto{\pgfqpoint{2.176760in}{1.925000in}}%
\pgfpathlineto{\pgfqpoint{2.178000in}{1.995000in}}%
\pgfpathlineto{\pgfqpoint{2.180480in}{2.275000in}}%
\pgfpathlineto{\pgfqpoint{2.181720in}{1.995000in}}%
\pgfpathlineto{\pgfqpoint{2.182960in}{2.170000in}}%
\pgfpathlineto{\pgfqpoint{2.185440in}{1.820000in}}%
\pgfpathlineto{\pgfqpoint{2.186680in}{1.925000in}}%
\pgfpathlineto{\pgfqpoint{2.187920in}{1.890000in}}%
\pgfpathlineto{\pgfqpoint{2.189160in}{2.030000in}}%
\pgfpathlineto{\pgfqpoint{2.190400in}{1.925000in}}%
\pgfpathlineto{\pgfqpoint{2.191640in}{1.680000in}}%
\pgfpathlineto{\pgfqpoint{2.192880in}{2.135000in}}%
\pgfpathlineto{\pgfqpoint{2.194120in}{1.785000in}}%
\pgfpathlineto{\pgfqpoint{2.195360in}{2.135000in}}%
\pgfpathlineto{\pgfqpoint{2.196600in}{2.100000in}}%
\pgfpathlineto{\pgfqpoint{2.197840in}{1.960000in}}%
\pgfpathlineto{\pgfqpoint{2.200320in}{2.240000in}}%
\pgfpathlineto{\pgfqpoint{2.201560in}{2.030000in}}%
\pgfpathlineto{\pgfqpoint{2.202800in}{2.065000in}}%
\pgfpathlineto{\pgfqpoint{2.204040in}{1.645000in}}%
\pgfpathlineto{\pgfqpoint{2.207760in}{2.310000in}}%
\pgfpathlineto{\pgfqpoint{2.210240in}{1.715000in}}%
\pgfpathlineto{\pgfqpoint{2.211480in}{2.310000in}}%
\pgfpathlineto{\pgfqpoint{2.212720in}{2.310000in}}%
\pgfpathlineto{\pgfqpoint{2.213960in}{2.065000in}}%
\pgfpathlineto{\pgfqpoint{2.215200in}{2.205000in}}%
\pgfpathlineto{\pgfqpoint{2.216440in}{1.820000in}}%
\pgfpathlineto{\pgfqpoint{2.217680in}{1.890000in}}%
\pgfpathlineto{\pgfqpoint{2.218920in}{1.680000in}}%
\pgfpathlineto{\pgfqpoint{2.220160in}{1.925000in}}%
\pgfpathlineto{\pgfqpoint{2.221400in}{1.785000in}}%
\pgfpathlineto{\pgfqpoint{2.223880in}{2.240000in}}%
\pgfpathlineto{\pgfqpoint{2.226360in}{1.995000in}}%
\pgfpathlineto{\pgfqpoint{2.227600in}{2.275000in}}%
\pgfpathlineto{\pgfqpoint{2.230080in}{1.610000in}}%
\pgfpathlineto{\pgfqpoint{2.232560in}{2.310000in}}%
\pgfpathlineto{\pgfqpoint{2.236280in}{1.750000in}}%
\pgfpathlineto{\pgfqpoint{2.237520in}{1.855000in}}%
\pgfpathlineto{\pgfqpoint{2.238760in}{1.785000in}}%
\pgfpathlineto{\pgfqpoint{2.241240in}{2.065000in}}%
\pgfpathlineto{\pgfqpoint{2.243720in}{1.820000in}}%
\pgfpathlineto{\pgfqpoint{2.244960in}{2.135000in}}%
\pgfpathlineto{\pgfqpoint{2.246200in}{1.610000in}}%
\pgfpathlineto{\pgfqpoint{2.248680in}{2.275000in}}%
\pgfpathlineto{\pgfqpoint{2.249920in}{2.135000in}}%
\pgfpathlineto{\pgfqpoint{2.251160in}{1.820000in}}%
\pgfpathlineto{\pgfqpoint{2.252400in}{1.855000in}}%
\pgfpathlineto{\pgfqpoint{2.253640in}{1.855000in}}%
\pgfpathlineto{\pgfqpoint{2.254880in}{1.470000in}}%
\pgfpathlineto{\pgfqpoint{2.257360in}{2.205000in}}%
\pgfpathlineto{\pgfqpoint{2.258600in}{1.575000in}}%
\pgfpathlineto{\pgfqpoint{2.259840in}{1.995000in}}%
\pgfpathlineto{\pgfqpoint{2.262320in}{1.505000in}}%
\pgfpathlineto{\pgfqpoint{2.264800in}{1.855000in}}%
\pgfpathlineto{\pgfqpoint{2.266040in}{1.820000in}}%
\pgfpathlineto{\pgfqpoint{2.267280in}{1.680000in}}%
\pgfpathlineto{\pgfqpoint{2.268520in}{2.100000in}}%
\pgfpathlineto{\pgfqpoint{2.269760in}{2.065000in}}%
\pgfpathlineto{\pgfqpoint{2.271000in}{1.715000in}}%
\pgfpathlineto{\pgfqpoint{2.274720in}{2.030000in}}%
\pgfpathlineto{\pgfqpoint{2.277200in}{1.855000in}}%
\pgfpathlineto{\pgfqpoint{2.278440in}{2.030000in}}%
\pgfpathlineto{\pgfqpoint{2.279680in}{2.030000in}}%
\pgfpathlineto{\pgfqpoint{2.280920in}{1.890000in}}%
\pgfpathlineto{\pgfqpoint{2.282160in}{1.610000in}}%
\pgfpathlineto{\pgfqpoint{2.283400in}{1.715000in}}%
\pgfpathlineto{\pgfqpoint{2.285880in}{1.960000in}}%
\pgfpathlineto{\pgfqpoint{2.287120in}{1.610000in}}%
\pgfpathlineto{\pgfqpoint{2.288360in}{1.785000in}}%
\pgfpathlineto{\pgfqpoint{2.289600in}{1.645000in}}%
\pgfpathlineto{\pgfqpoint{2.292080in}{2.065000in}}%
\pgfpathlineto{\pgfqpoint{2.293320in}{1.890000in}}%
\pgfpathlineto{\pgfqpoint{2.295800in}{2.310000in}}%
\pgfpathlineto{\pgfqpoint{2.297040in}{2.030000in}}%
\pgfpathlineto{\pgfqpoint{2.299520in}{2.345000in}}%
\pgfpathlineto{\pgfqpoint{2.300760in}{1.995000in}}%
\pgfpathlineto{\pgfqpoint{2.302000in}{2.275000in}}%
\pgfpathlineto{\pgfqpoint{2.303240in}{2.170000in}}%
\pgfpathlineto{\pgfqpoint{2.304480in}{2.170000in}}%
\pgfpathlineto{\pgfqpoint{2.305720in}{1.960000in}}%
\pgfpathlineto{\pgfqpoint{2.306960in}{2.415000in}}%
\pgfpathlineto{\pgfqpoint{2.309440in}{1.855000in}}%
\pgfpathlineto{\pgfqpoint{2.310680in}{1.750000in}}%
\pgfpathlineto{\pgfqpoint{2.311920in}{1.890000in}}%
\pgfpathlineto{\pgfqpoint{2.313160in}{1.890000in}}%
\pgfpathlineto{\pgfqpoint{2.314400in}{1.925000in}}%
\pgfpathlineto{\pgfqpoint{2.316880in}{1.680000in}}%
\pgfpathlineto{\pgfqpoint{2.319360in}{2.450000in}}%
\pgfpathlineto{\pgfqpoint{2.320600in}{1.820000in}}%
\pgfpathlineto{\pgfqpoint{2.321840in}{1.890000in}}%
\pgfpathlineto{\pgfqpoint{2.323080in}{2.170000in}}%
\pgfpathlineto{\pgfqpoint{2.324320in}{2.065000in}}%
\pgfpathlineto{\pgfqpoint{2.325560in}{2.310000in}}%
\pgfpathlineto{\pgfqpoint{2.326800in}{2.205000in}}%
\pgfpathlineto{\pgfqpoint{2.328040in}{2.205000in}}%
\pgfpathlineto{\pgfqpoint{2.329280in}{2.135000in}}%
\pgfpathlineto{\pgfqpoint{2.330520in}{2.170000in}}%
\pgfpathlineto{\pgfqpoint{2.331760in}{2.345000in}}%
\pgfpathlineto{\pgfqpoint{2.333000in}{1.855000in}}%
\pgfpathlineto{\pgfqpoint{2.334240in}{2.205000in}}%
\pgfpathlineto{\pgfqpoint{2.335480in}{1.960000in}}%
\pgfpathlineto{\pgfqpoint{2.336720in}{1.960000in}}%
\pgfpathlineto{\pgfqpoint{2.337960in}{2.065000in}}%
\pgfpathlineto{\pgfqpoint{2.339200in}{2.065000in}}%
\pgfpathlineto{\pgfqpoint{2.340440in}{2.205000in}}%
\pgfpathlineto{\pgfqpoint{2.344160in}{1.750000in}}%
\pgfpathlineto{\pgfqpoint{2.345400in}{2.240000in}}%
\pgfpathlineto{\pgfqpoint{2.346640in}{2.170000in}}%
\pgfpathlineto{\pgfqpoint{2.347880in}{1.855000in}}%
\pgfpathlineto{\pgfqpoint{2.349120in}{2.030000in}}%
\pgfpathlineto{\pgfqpoint{2.351600in}{1.680000in}}%
\pgfpathlineto{\pgfqpoint{2.352840in}{1.680000in}}%
\pgfpathlineto{\pgfqpoint{2.355320in}{2.030000in}}%
\pgfpathlineto{\pgfqpoint{2.356560in}{2.100000in}}%
\pgfpathlineto{\pgfqpoint{2.357800in}{2.100000in}}%
\pgfpathlineto{\pgfqpoint{2.359040in}{2.135000in}}%
\pgfpathlineto{\pgfqpoint{2.360280in}{1.855000in}}%
\pgfpathlineto{\pgfqpoint{2.361520in}{2.030000in}}%
\pgfpathlineto{\pgfqpoint{2.362760in}{1.995000in}}%
\pgfpathlineto{\pgfqpoint{2.364000in}{1.855000in}}%
\pgfpathlineto{\pgfqpoint{2.365240in}{2.450000in}}%
\pgfpathlineto{\pgfqpoint{2.366480in}{1.855000in}}%
\pgfpathlineto{\pgfqpoint{2.367720in}{2.170000in}}%
\pgfpathlineto{\pgfqpoint{2.370200in}{1.890000in}}%
\pgfpathlineto{\pgfqpoint{2.371440in}{1.715000in}}%
\pgfpathlineto{\pgfqpoint{2.372680in}{1.855000in}}%
\pgfpathlineto{\pgfqpoint{2.373920in}{1.855000in}}%
\pgfpathlineto{\pgfqpoint{2.375160in}{1.645000in}}%
\pgfpathlineto{\pgfqpoint{2.376400in}{1.925000in}}%
\pgfpathlineto{\pgfqpoint{2.377640in}{1.925000in}}%
\pgfpathlineto{\pgfqpoint{2.378880in}{1.715000in}}%
\pgfpathlineto{\pgfqpoint{2.380120in}{1.960000in}}%
\pgfpathlineto{\pgfqpoint{2.381360in}{1.645000in}}%
\pgfpathlineto{\pgfqpoint{2.383840in}{1.960000in}}%
\pgfpathlineto{\pgfqpoint{2.385080in}{2.240000in}}%
\pgfpathlineto{\pgfqpoint{2.386320in}{2.205000in}}%
\pgfpathlineto{\pgfqpoint{2.388800in}{1.820000in}}%
\pgfpathlineto{\pgfqpoint{2.390040in}{1.995000in}}%
\pgfpathlineto{\pgfqpoint{2.391280in}{1.925000in}}%
\pgfpathlineto{\pgfqpoint{2.392520in}{2.135000in}}%
\pgfpathlineto{\pgfqpoint{2.393760in}{1.750000in}}%
\pgfpathlineto{\pgfqpoint{2.395000in}{1.855000in}}%
\pgfpathlineto{\pgfqpoint{2.396240in}{1.855000in}}%
\pgfpathlineto{\pgfqpoint{2.397480in}{1.715000in}}%
\pgfpathlineto{\pgfqpoint{2.398720in}{1.890000in}}%
\pgfpathlineto{\pgfqpoint{2.399960in}{1.855000in}}%
\pgfpathlineto{\pgfqpoint{2.401200in}{2.100000in}}%
\pgfpathlineto{\pgfqpoint{2.402440in}{1.610000in}}%
\pgfpathlineto{\pgfqpoint{2.403680in}{1.960000in}}%
\pgfpathlineto{\pgfqpoint{2.404920in}{1.995000in}}%
\pgfpathlineto{\pgfqpoint{2.406160in}{2.135000in}}%
\pgfpathlineto{\pgfqpoint{2.408640in}{1.540000in}}%
\pgfpathlineto{\pgfqpoint{2.409880in}{1.715000in}}%
\pgfpathlineto{\pgfqpoint{2.411120in}{1.645000in}}%
\pgfpathlineto{\pgfqpoint{2.412360in}{1.785000in}}%
\pgfpathlineto{\pgfqpoint{2.413600in}{2.275000in}}%
\pgfpathlineto{\pgfqpoint{2.416080in}{1.960000in}}%
\pgfpathlineto{\pgfqpoint{2.417320in}{1.890000in}}%
\pgfpathlineto{\pgfqpoint{2.418560in}{1.995000in}}%
\pgfpathlineto{\pgfqpoint{2.419800in}{1.785000in}}%
\pgfpathlineto{\pgfqpoint{2.421040in}{1.960000in}}%
\pgfpathlineto{\pgfqpoint{2.422280in}{1.820000in}}%
\pgfpathlineto{\pgfqpoint{2.423520in}{1.995000in}}%
\pgfpathlineto{\pgfqpoint{2.424760in}{1.960000in}}%
\pgfpathlineto{\pgfqpoint{2.426000in}{1.750000in}}%
\pgfpathlineto{\pgfqpoint{2.427240in}{1.925000in}}%
\pgfpathlineto{\pgfqpoint{2.428480in}{1.890000in}}%
\pgfpathlineto{\pgfqpoint{2.430960in}{2.205000in}}%
\pgfpathlineto{\pgfqpoint{2.433440in}{1.610000in}}%
\pgfpathlineto{\pgfqpoint{2.435920in}{1.995000in}}%
\pgfpathlineto{\pgfqpoint{2.437160in}{2.030000in}}%
\pgfpathlineto{\pgfqpoint{2.438400in}{1.925000in}}%
\pgfpathlineto{\pgfqpoint{2.439640in}{1.995000in}}%
\pgfpathlineto{\pgfqpoint{2.440880in}{1.855000in}}%
\pgfpathlineto{\pgfqpoint{2.444600in}{1.960000in}}%
\pgfpathlineto{\pgfqpoint{2.445840in}{1.890000in}}%
\pgfpathlineto{\pgfqpoint{2.447080in}{1.680000in}}%
\pgfpathlineto{\pgfqpoint{2.448320in}{2.100000in}}%
\pgfpathlineto{\pgfqpoint{2.449560in}{1.785000in}}%
\pgfpathlineto{\pgfqpoint{2.450800in}{2.170000in}}%
\pgfpathlineto{\pgfqpoint{2.452040in}{1.785000in}}%
\pgfpathlineto{\pgfqpoint{2.453280in}{1.925000in}}%
\pgfpathlineto{\pgfqpoint{2.454520in}{1.610000in}}%
\pgfpathlineto{\pgfqpoint{2.457000in}{1.715000in}}%
\pgfpathlineto{\pgfqpoint{2.458240in}{1.295000in}}%
\pgfpathlineto{\pgfqpoint{2.459480in}{2.030000in}}%
\pgfpathlineto{\pgfqpoint{2.460720in}{1.575000in}}%
\pgfpathlineto{\pgfqpoint{2.461960in}{1.890000in}}%
\pgfpathlineto{\pgfqpoint{2.463200in}{1.645000in}}%
\pgfpathlineto{\pgfqpoint{2.464440in}{1.820000in}}%
\pgfpathlineto{\pgfqpoint{2.465680in}{1.715000in}}%
\pgfpathlineto{\pgfqpoint{2.466920in}{1.715000in}}%
\pgfpathlineto{\pgfqpoint{2.468160in}{1.680000in}}%
\pgfpathlineto{\pgfqpoint{2.469400in}{1.505000in}}%
\pgfpathlineto{\pgfqpoint{2.470640in}{1.540000in}}%
\pgfpathlineto{\pgfqpoint{2.473120in}{2.415000in}}%
\pgfpathlineto{\pgfqpoint{2.474360in}{1.610000in}}%
\pgfpathlineto{\pgfqpoint{2.475600in}{2.065000in}}%
\pgfpathlineto{\pgfqpoint{2.476840in}{2.030000in}}%
\pgfpathlineto{\pgfqpoint{2.478080in}{1.715000in}}%
\pgfpathlineto{\pgfqpoint{2.479320in}{1.750000in}}%
\pgfpathlineto{\pgfqpoint{2.480560in}{2.240000in}}%
\pgfpathlineto{\pgfqpoint{2.485520in}{1.750000in}}%
\pgfpathlineto{\pgfqpoint{2.486760in}{2.345000in}}%
\pgfpathlineto{\pgfqpoint{2.489240in}{1.645000in}}%
\pgfpathlineto{\pgfqpoint{2.492960in}{2.415000in}}%
\pgfpathlineto{\pgfqpoint{2.494200in}{1.785000in}}%
\pgfpathlineto{\pgfqpoint{2.495440in}{2.030000in}}%
\pgfpathlineto{\pgfqpoint{2.496680in}{1.890000in}}%
\pgfpathlineto{\pgfqpoint{2.497920in}{1.575000in}}%
\pgfpathlineto{\pgfqpoint{2.499160in}{1.855000in}}%
\pgfpathlineto{\pgfqpoint{2.500400in}{1.820000in}}%
\pgfpathlineto{\pgfqpoint{2.501640in}{1.855000in}}%
\pgfpathlineto{\pgfqpoint{2.502880in}{1.960000in}}%
\pgfpathlineto{\pgfqpoint{2.505360in}{1.190000in}}%
\pgfpathlineto{\pgfqpoint{2.506600in}{1.540000in}}%
\pgfpathlineto{\pgfqpoint{2.507840in}{1.505000in}}%
\pgfpathlineto{\pgfqpoint{2.510320in}{2.065000in}}%
\pgfpathlineto{\pgfqpoint{2.511560in}{1.855000in}}%
\pgfpathlineto{\pgfqpoint{2.512800in}{1.890000in}}%
\pgfpathlineto{\pgfqpoint{2.514040in}{2.100000in}}%
\pgfpathlineto{\pgfqpoint{2.516520in}{1.820000in}}%
\pgfpathlineto{\pgfqpoint{2.517760in}{2.135000in}}%
\pgfpathlineto{\pgfqpoint{2.520240in}{1.470000in}}%
\pgfpathlineto{\pgfqpoint{2.521480in}{1.470000in}}%
\pgfpathlineto{\pgfqpoint{2.522720in}{1.785000in}}%
\pgfpathlineto{\pgfqpoint{2.523960in}{1.715000in}}%
\pgfpathlineto{\pgfqpoint{2.525200in}{1.470000in}}%
\pgfpathlineto{\pgfqpoint{2.527680in}{2.100000in}}%
\pgfpathlineto{\pgfqpoint{2.528920in}{1.785000in}}%
\pgfpathlineto{\pgfqpoint{2.530160in}{2.240000in}}%
\pgfpathlineto{\pgfqpoint{2.531400in}{1.890000in}}%
\pgfpathlineto{\pgfqpoint{2.532640in}{1.890000in}}%
\pgfpathlineto{\pgfqpoint{2.533880in}{1.925000in}}%
\pgfpathlineto{\pgfqpoint{2.535120in}{1.715000in}}%
\pgfpathlineto{\pgfqpoint{2.536360in}{1.785000in}}%
\pgfpathlineto{\pgfqpoint{2.537600in}{2.240000in}}%
\pgfpathlineto{\pgfqpoint{2.538840in}{1.855000in}}%
\pgfpathlineto{\pgfqpoint{2.540080in}{2.100000in}}%
\pgfpathlineto{\pgfqpoint{2.543800in}{1.855000in}}%
\pgfpathlineto{\pgfqpoint{2.545040in}{1.820000in}}%
\pgfpathlineto{\pgfqpoint{2.546280in}{1.680000in}}%
\pgfpathlineto{\pgfqpoint{2.548760in}{2.065000in}}%
\pgfpathlineto{\pgfqpoint{2.551240in}{1.715000in}}%
\pgfpathlineto{\pgfqpoint{2.552480in}{1.750000in}}%
\pgfpathlineto{\pgfqpoint{2.553720in}{2.135000in}}%
\pgfpathlineto{\pgfqpoint{2.554960in}{1.890000in}}%
\pgfpathlineto{\pgfqpoint{2.556200in}{2.065000in}}%
\pgfpathlineto{\pgfqpoint{2.557440in}{1.785000in}}%
\pgfpathlineto{\pgfqpoint{2.558680in}{1.960000in}}%
\pgfpathlineto{\pgfqpoint{2.559920in}{1.610000in}}%
\pgfpathlineto{\pgfqpoint{2.562400in}{1.890000in}}%
\pgfpathlineto{\pgfqpoint{2.563640in}{2.030000in}}%
\pgfpathlineto{\pgfqpoint{2.564880in}{1.680000in}}%
\pgfpathlineto{\pgfqpoint{2.566120in}{1.750000in}}%
\pgfpathlineto{\pgfqpoint{2.567360in}{1.680000in}}%
\pgfpathlineto{\pgfqpoint{2.568600in}{1.680000in}}%
\pgfpathlineto{\pgfqpoint{2.569840in}{1.715000in}}%
\pgfpathlineto{\pgfqpoint{2.571080in}{1.890000in}}%
\pgfpathlineto{\pgfqpoint{2.572320in}{1.785000in}}%
\pgfpathlineto{\pgfqpoint{2.574800in}{2.170000in}}%
\pgfpathlineto{\pgfqpoint{2.576040in}{1.820000in}}%
\pgfpathlineto{\pgfqpoint{2.578520in}{2.100000in}}%
\pgfpathlineto{\pgfqpoint{2.579760in}{1.575000in}}%
\pgfpathlineto{\pgfqpoint{2.581000in}{1.855000in}}%
\pgfpathlineto{\pgfqpoint{2.582240in}{1.785000in}}%
\pgfpathlineto{\pgfqpoint{2.583480in}{1.610000in}}%
\pgfpathlineto{\pgfqpoint{2.585960in}{2.100000in}}%
\pgfpathlineto{\pgfqpoint{2.587200in}{1.540000in}}%
\pgfpathlineto{\pgfqpoint{2.590920in}{2.240000in}}%
\pgfpathlineto{\pgfqpoint{2.592160in}{2.310000in}}%
\pgfpathlineto{\pgfqpoint{2.593400in}{1.820000in}}%
\pgfpathlineto{\pgfqpoint{2.594640in}{2.100000in}}%
\pgfpathlineto{\pgfqpoint{2.597120in}{2.030000in}}%
\pgfpathlineto{\pgfqpoint{2.598360in}{1.960000in}}%
\pgfpathlineto{\pgfqpoint{2.599600in}{1.470000in}}%
\pgfpathlineto{\pgfqpoint{2.600840in}{1.925000in}}%
\pgfpathlineto{\pgfqpoint{2.602080in}{1.890000in}}%
\pgfpathlineto{\pgfqpoint{2.604560in}{2.240000in}}%
\pgfpathlineto{\pgfqpoint{2.605800in}{2.205000in}}%
\pgfpathlineto{\pgfqpoint{2.608280in}{1.995000in}}%
\pgfpathlineto{\pgfqpoint{2.609520in}{2.240000in}}%
\pgfpathlineto{\pgfqpoint{2.610760in}{2.100000in}}%
\pgfpathlineto{\pgfqpoint{2.613240in}{1.785000in}}%
\pgfpathlineto{\pgfqpoint{2.614480in}{2.065000in}}%
\pgfpathlineto{\pgfqpoint{2.615720in}{1.785000in}}%
\pgfpathlineto{\pgfqpoint{2.616960in}{1.785000in}}%
\pgfpathlineto{\pgfqpoint{2.618200in}{1.540000in}}%
\pgfpathlineto{\pgfqpoint{2.620680in}{1.960000in}}%
\pgfpathlineto{\pgfqpoint{2.621920in}{2.065000in}}%
\pgfpathlineto{\pgfqpoint{2.623160in}{2.030000in}}%
\pgfpathlineto{\pgfqpoint{2.624400in}{1.400000in}}%
\pgfpathlineto{\pgfqpoint{2.625640in}{1.645000in}}%
\pgfpathlineto{\pgfqpoint{2.626880in}{2.275000in}}%
\pgfpathlineto{\pgfqpoint{2.629360in}{1.610000in}}%
\pgfpathlineto{\pgfqpoint{2.630600in}{1.435000in}}%
\pgfpathlineto{\pgfqpoint{2.631840in}{2.205000in}}%
\pgfpathlineto{\pgfqpoint{2.633080in}{1.785000in}}%
\pgfpathlineto{\pgfqpoint{2.634320in}{2.135000in}}%
\pgfpathlineto{\pgfqpoint{2.635560in}{1.785000in}}%
\pgfpathlineto{\pgfqpoint{2.636800in}{2.030000in}}%
\pgfpathlineto{\pgfqpoint{2.638040in}{1.715000in}}%
\pgfpathlineto{\pgfqpoint{2.639280in}{1.925000in}}%
\pgfpathlineto{\pgfqpoint{2.641760in}{1.645000in}}%
\pgfpathlineto{\pgfqpoint{2.643000in}{1.995000in}}%
\pgfpathlineto{\pgfqpoint{2.644240in}{1.960000in}}%
\pgfpathlineto{\pgfqpoint{2.645480in}{2.135000in}}%
\pgfpathlineto{\pgfqpoint{2.646720in}{1.750000in}}%
\pgfpathlineto{\pgfqpoint{2.647960in}{1.715000in}}%
\pgfpathlineto{\pgfqpoint{2.649200in}{2.030000in}}%
\pgfpathlineto{\pgfqpoint{2.651680in}{1.540000in}}%
\pgfpathlineto{\pgfqpoint{2.652920in}{1.855000in}}%
\pgfpathlineto{\pgfqpoint{2.654160in}{1.435000in}}%
\pgfpathlineto{\pgfqpoint{2.655400in}{2.065000in}}%
\pgfpathlineto{\pgfqpoint{2.656640in}{2.065000in}}%
\pgfpathlineto{\pgfqpoint{2.657880in}{1.855000in}}%
\pgfpathlineto{\pgfqpoint{2.659120in}{1.960000in}}%
\pgfpathlineto{\pgfqpoint{2.661600in}{1.470000in}}%
\pgfpathlineto{\pgfqpoint{2.662840in}{1.610000in}}%
\pgfpathlineto{\pgfqpoint{2.664080in}{1.925000in}}%
\pgfpathlineto{\pgfqpoint{2.665320in}{1.785000in}}%
\pgfpathlineto{\pgfqpoint{2.666560in}{1.855000in}}%
\pgfpathlineto{\pgfqpoint{2.667800in}{2.100000in}}%
\pgfpathlineto{\pgfqpoint{2.669040in}{2.065000in}}%
\pgfpathlineto{\pgfqpoint{2.670280in}{1.960000in}}%
\pgfpathlineto{\pgfqpoint{2.671520in}{1.435000in}}%
\pgfpathlineto{\pgfqpoint{2.674000in}{2.100000in}}%
\pgfpathlineto{\pgfqpoint{2.675240in}{1.575000in}}%
\pgfpathlineto{\pgfqpoint{2.677720in}{2.170000in}}%
\pgfpathlineto{\pgfqpoint{2.678960in}{2.065000in}}%
\pgfpathlineto{\pgfqpoint{2.681440in}{1.610000in}}%
\pgfpathlineto{\pgfqpoint{2.683920in}{1.995000in}}%
\pgfpathlineto{\pgfqpoint{2.685160in}{1.925000in}}%
\pgfpathlineto{\pgfqpoint{2.686400in}{1.785000in}}%
\pgfpathlineto{\pgfqpoint{2.687640in}{1.785000in}}%
\pgfpathlineto{\pgfqpoint{2.688880in}{2.170000in}}%
\pgfpathlineto{\pgfqpoint{2.691360in}{2.065000in}}%
\pgfpathlineto{\pgfqpoint{2.692600in}{1.785000in}}%
\pgfpathlineto{\pgfqpoint{2.693840in}{2.170000in}}%
\pgfpathlineto{\pgfqpoint{2.696320in}{1.470000in}}%
\pgfpathlineto{\pgfqpoint{2.697560in}{1.890000in}}%
\pgfpathlineto{\pgfqpoint{2.698800in}{1.890000in}}%
\pgfpathlineto{\pgfqpoint{2.700040in}{1.680000in}}%
\pgfpathlineto{\pgfqpoint{2.701280in}{2.380000in}}%
\pgfpathlineto{\pgfqpoint{2.703760in}{1.855000in}}%
\pgfpathlineto{\pgfqpoint{2.705000in}{2.345000in}}%
\pgfpathlineto{\pgfqpoint{2.706240in}{1.610000in}}%
\pgfpathlineto{\pgfqpoint{2.707480in}{2.135000in}}%
\pgfpathlineto{\pgfqpoint{2.709960in}{1.785000in}}%
\pgfpathlineto{\pgfqpoint{2.711200in}{1.855000in}}%
\pgfpathlineto{\pgfqpoint{2.712440in}{1.995000in}}%
\pgfpathlineto{\pgfqpoint{2.713680in}{1.750000in}}%
\pgfpathlineto{\pgfqpoint{2.714920in}{1.925000in}}%
\pgfpathlineto{\pgfqpoint{2.717400in}{1.785000in}}%
\pgfpathlineto{\pgfqpoint{2.718640in}{2.310000in}}%
\pgfpathlineto{\pgfqpoint{2.722360in}{1.715000in}}%
\pgfpathlineto{\pgfqpoint{2.723600in}{1.680000in}}%
\pgfpathlineto{\pgfqpoint{2.726080in}{1.470000in}}%
\pgfpathlineto{\pgfqpoint{2.727320in}{2.065000in}}%
\pgfpathlineto{\pgfqpoint{2.728560in}{1.575000in}}%
\pgfpathlineto{\pgfqpoint{2.729800in}{1.680000in}}%
\pgfpathlineto{\pgfqpoint{2.732280in}{2.065000in}}%
\pgfpathlineto{\pgfqpoint{2.733520in}{1.715000in}}%
\pgfpathlineto{\pgfqpoint{2.734760in}{1.890000in}}%
\pgfpathlineto{\pgfqpoint{2.736000in}{1.820000in}}%
\pgfpathlineto{\pgfqpoint{2.737240in}{1.820000in}}%
\pgfpathlineto{\pgfqpoint{2.738480in}{1.645000in}}%
\pgfpathlineto{\pgfqpoint{2.740960in}{2.135000in}}%
\pgfpathlineto{\pgfqpoint{2.742200in}{1.820000in}}%
\pgfpathlineto{\pgfqpoint{2.743440in}{1.855000in}}%
\pgfpathlineto{\pgfqpoint{2.745920in}{1.575000in}}%
\pgfpathlineto{\pgfqpoint{2.747160in}{1.960000in}}%
\pgfpathlineto{\pgfqpoint{2.748400in}{1.960000in}}%
\pgfpathlineto{\pgfqpoint{2.752120in}{1.610000in}}%
\pgfpathlineto{\pgfqpoint{2.753360in}{1.645000in}}%
\pgfpathlineto{\pgfqpoint{2.754600in}{2.065000in}}%
\pgfpathlineto{\pgfqpoint{2.755840in}{2.065000in}}%
\pgfpathlineto{\pgfqpoint{2.758320in}{1.785000in}}%
\pgfpathlineto{\pgfqpoint{2.759560in}{2.205000in}}%
\pgfpathlineto{\pgfqpoint{2.760800in}{1.960000in}}%
\pgfpathlineto{\pgfqpoint{2.762040in}{2.310000in}}%
\pgfpathlineto{\pgfqpoint{2.764520in}{1.995000in}}%
\pgfpathlineto{\pgfqpoint{2.765760in}{1.680000in}}%
\pgfpathlineto{\pgfqpoint{2.769480in}{2.345000in}}%
\pgfpathlineto{\pgfqpoint{2.771960in}{1.925000in}}%
\pgfpathlineto{\pgfqpoint{2.773200in}{1.995000in}}%
\pgfpathlineto{\pgfqpoint{2.774440in}{2.170000in}}%
\pgfpathlineto{\pgfqpoint{2.775680in}{2.030000in}}%
\pgfpathlineto{\pgfqpoint{2.776920in}{2.030000in}}%
\pgfpathlineto{\pgfqpoint{2.779400in}{1.785000in}}%
\pgfpathlineto{\pgfqpoint{2.780640in}{1.995000in}}%
\pgfpathlineto{\pgfqpoint{2.781880in}{1.645000in}}%
\pgfpathlineto{\pgfqpoint{2.784360in}{1.890000in}}%
\pgfpathlineto{\pgfqpoint{2.785600in}{1.925000in}}%
\pgfpathlineto{\pgfqpoint{2.786840in}{2.310000in}}%
\pgfpathlineto{\pgfqpoint{2.788080in}{1.610000in}}%
\pgfpathlineto{\pgfqpoint{2.790560in}{2.485000in}}%
\pgfpathlineto{\pgfqpoint{2.791800in}{1.960000in}}%
\pgfpathlineto{\pgfqpoint{2.793040in}{2.135000in}}%
\pgfpathlineto{\pgfqpoint{2.794280in}{1.855000in}}%
\pgfpathlineto{\pgfqpoint{2.795520in}{2.170000in}}%
\pgfpathlineto{\pgfqpoint{2.798000in}{1.890000in}}%
\pgfpathlineto{\pgfqpoint{2.799240in}{2.275000in}}%
\pgfpathlineto{\pgfqpoint{2.801720in}{1.785000in}}%
\pgfpathlineto{\pgfqpoint{2.802960in}{2.135000in}}%
\pgfpathlineto{\pgfqpoint{2.804200in}{1.645000in}}%
\pgfpathlineto{\pgfqpoint{2.805440in}{1.750000in}}%
\pgfpathlineto{\pgfqpoint{2.806680in}{2.030000in}}%
\pgfpathlineto{\pgfqpoint{2.807920in}{1.785000in}}%
\pgfpathlineto{\pgfqpoint{2.809160in}{1.890000in}}%
\pgfpathlineto{\pgfqpoint{2.810400in}{1.785000in}}%
\pgfpathlineto{\pgfqpoint{2.811640in}{2.205000in}}%
\pgfpathlineto{\pgfqpoint{2.812880in}{2.100000in}}%
\pgfpathlineto{\pgfqpoint{2.814120in}{2.135000in}}%
\pgfpathlineto{\pgfqpoint{2.815360in}{2.275000in}}%
\pgfpathlineto{\pgfqpoint{2.816600in}{1.960000in}}%
\pgfpathlineto{\pgfqpoint{2.820320in}{2.170000in}}%
\pgfpathlineto{\pgfqpoint{2.821560in}{1.750000in}}%
\pgfpathlineto{\pgfqpoint{2.824040in}{2.170000in}}%
\pgfpathlineto{\pgfqpoint{2.826520in}{1.820000in}}%
\pgfpathlineto{\pgfqpoint{2.827760in}{2.240000in}}%
\pgfpathlineto{\pgfqpoint{2.829000in}{1.960000in}}%
\pgfpathlineto{\pgfqpoint{2.830240in}{2.345000in}}%
\pgfpathlineto{\pgfqpoint{2.831480in}{1.960000in}}%
\pgfpathlineto{\pgfqpoint{2.833960in}{2.030000in}}%
\pgfpathlineto{\pgfqpoint{2.835200in}{2.030000in}}%
\pgfpathlineto{\pgfqpoint{2.836440in}{2.275000in}}%
\pgfpathlineto{\pgfqpoint{2.837680in}{1.855000in}}%
\pgfpathlineto{\pgfqpoint{2.838920in}{2.065000in}}%
\pgfpathlineto{\pgfqpoint{2.840160in}{1.645000in}}%
\pgfpathlineto{\pgfqpoint{2.842640in}{1.960000in}}%
\pgfpathlineto{\pgfqpoint{2.843880in}{1.785000in}}%
\pgfpathlineto{\pgfqpoint{2.845120in}{1.925000in}}%
\pgfpathlineto{\pgfqpoint{2.846360in}{1.645000in}}%
\pgfpathlineto{\pgfqpoint{2.847600in}{1.960000in}}%
\pgfpathlineto{\pgfqpoint{2.848840in}{1.155000in}}%
\pgfpathlineto{\pgfqpoint{2.850080in}{2.310000in}}%
\pgfpathlineto{\pgfqpoint{2.851320in}{1.680000in}}%
\pgfpathlineto{\pgfqpoint{2.852560in}{2.170000in}}%
\pgfpathlineto{\pgfqpoint{2.853800in}{1.715000in}}%
\pgfpathlineto{\pgfqpoint{2.855040in}{1.960000in}}%
\pgfpathlineto{\pgfqpoint{2.856280in}{2.450000in}}%
\pgfpathlineto{\pgfqpoint{2.857520in}{2.485000in}}%
\pgfpathlineto{\pgfqpoint{2.858760in}{1.575000in}}%
\pgfpathlineto{\pgfqpoint{2.860000in}{1.680000in}}%
\pgfpathlineto{\pgfqpoint{2.861240in}{1.330000in}}%
\pgfpathlineto{\pgfqpoint{2.863720in}{1.890000in}}%
\pgfpathlineto{\pgfqpoint{2.866200in}{1.645000in}}%
\pgfpathlineto{\pgfqpoint{2.867440in}{1.680000in}}%
\pgfpathlineto{\pgfqpoint{2.868680in}{1.960000in}}%
\pgfpathlineto{\pgfqpoint{2.869920in}{1.540000in}}%
\pgfpathlineto{\pgfqpoint{2.871160in}{1.610000in}}%
\pgfpathlineto{\pgfqpoint{2.872400in}{1.890000in}}%
\pgfpathlineto{\pgfqpoint{2.873640in}{1.680000in}}%
\pgfpathlineto{\pgfqpoint{2.874880in}{1.820000in}}%
\pgfpathlineto{\pgfqpoint{2.876120in}{2.240000in}}%
\pgfpathlineto{\pgfqpoint{2.877360in}{1.995000in}}%
\pgfpathlineto{\pgfqpoint{2.878600in}{1.995000in}}%
\pgfpathlineto{\pgfqpoint{2.879840in}{1.960000in}}%
\pgfpathlineto{\pgfqpoint{2.881080in}{1.960000in}}%
\pgfpathlineto{\pgfqpoint{2.882320in}{2.030000in}}%
\pgfpathlineto{\pgfqpoint{2.883560in}{1.995000in}}%
\pgfpathlineto{\pgfqpoint{2.884800in}{1.820000in}}%
\pgfpathlineto{\pgfqpoint{2.886040in}{1.995000in}}%
\pgfpathlineto{\pgfqpoint{2.888520in}{1.750000in}}%
\pgfpathlineto{\pgfqpoint{2.891000in}{1.820000in}}%
\pgfpathlineto{\pgfqpoint{2.892240in}{1.680000in}}%
\pgfpathlineto{\pgfqpoint{2.893480in}{2.310000in}}%
\pgfpathlineto{\pgfqpoint{2.894720in}{1.295000in}}%
\pgfpathlineto{\pgfqpoint{2.897200in}{1.820000in}}%
\pgfpathlineto{\pgfqpoint{2.898440in}{1.715000in}}%
\pgfpathlineto{\pgfqpoint{2.899680in}{1.715000in}}%
\pgfpathlineto{\pgfqpoint{2.900920in}{1.645000in}}%
\pgfpathlineto{\pgfqpoint{2.902160in}{2.100000in}}%
\pgfpathlineto{\pgfqpoint{2.903400in}{1.820000in}}%
\pgfpathlineto{\pgfqpoint{2.904640in}{1.820000in}}%
\pgfpathlineto{\pgfqpoint{2.905880in}{1.890000in}}%
\pgfpathlineto{\pgfqpoint{2.907120in}{2.100000in}}%
\pgfpathlineto{\pgfqpoint{2.908360in}{1.575000in}}%
\pgfpathlineto{\pgfqpoint{2.909600in}{2.135000in}}%
\pgfpathlineto{\pgfqpoint{2.910840in}{2.065000in}}%
\pgfpathlineto{\pgfqpoint{2.912080in}{1.750000in}}%
\pgfpathlineto{\pgfqpoint{2.913320in}{2.170000in}}%
\pgfpathlineto{\pgfqpoint{2.914560in}{2.135000in}}%
\pgfpathlineto{\pgfqpoint{2.917040in}{1.785000in}}%
\pgfpathlineto{\pgfqpoint{2.918280in}{1.960000in}}%
\pgfpathlineto{\pgfqpoint{2.919520in}{1.680000in}}%
\pgfpathlineto{\pgfqpoint{2.920760in}{2.030000in}}%
\pgfpathlineto{\pgfqpoint{2.922000in}{1.575000in}}%
\pgfpathlineto{\pgfqpoint{2.923240in}{2.030000in}}%
\pgfpathlineto{\pgfqpoint{2.924480in}{1.855000in}}%
\pgfpathlineto{\pgfqpoint{2.925720in}{2.205000in}}%
\pgfpathlineto{\pgfqpoint{2.928200in}{1.750000in}}%
\pgfpathlineto{\pgfqpoint{2.931920in}{2.415000in}}%
\pgfpathlineto{\pgfqpoint{2.933160in}{1.925000in}}%
\pgfpathlineto{\pgfqpoint{2.934400in}{2.065000in}}%
\pgfpathlineto{\pgfqpoint{2.935640in}{2.030000in}}%
\pgfpathlineto{\pgfqpoint{2.936880in}{1.960000in}}%
\pgfpathlineto{\pgfqpoint{2.938120in}{1.715000in}}%
\pgfpathlineto{\pgfqpoint{2.940600in}{1.890000in}}%
\pgfpathlineto{\pgfqpoint{2.941840in}{1.890000in}}%
\pgfpathlineto{\pgfqpoint{2.943080in}{1.540000in}}%
\pgfpathlineto{\pgfqpoint{2.944320in}{1.960000in}}%
\pgfpathlineto{\pgfqpoint{2.945560in}{1.890000in}}%
\pgfpathlineto{\pgfqpoint{2.946800in}{1.995000in}}%
\pgfpathlineto{\pgfqpoint{2.948040in}{1.540000in}}%
\pgfpathlineto{\pgfqpoint{2.949280in}{1.890000in}}%
\pgfpathlineto{\pgfqpoint{2.950520in}{1.505000in}}%
\pgfpathlineto{\pgfqpoint{2.951760in}{1.820000in}}%
\pgfpathlineto{\pgfqpoint{2.953000in}{1.470000in}}%
\pgfpathlineto{\pgfqpoint{2.954240in}{1.505000in}}%
\pgfpathlineto{\pgfqpoint{2.956720in}{1.995000in}}%
\pgfpathlineto{\pgfqpoint{2.959200in}{1.610000in}}%
\pgfpathlineto{\pgfqpoint{2.960440in}{1.855000in}}%
\pgfpathlineto{\pgfqpoint{2.961680in}{1.820000in}}%
\pgfpathlineto{\pgfqpoint{2.962920in}{1.365000in}}%
\pgfpathlineto{\pgfqpoint{2.965400in}{2.065000in}}%
\pgfpathlineto{\pgfqpoint{2.966640in}{1.260000in}}%
\pgfpathlineto{\pgfqpoint{2.967880in}{2.030000in}}%
\pgfpathlineto{\pgfqpoint{2.970360in}{1.855000in}}%
\pgfpathlineto{\pgfqpoint{2.971600in}{2.030000in}}%
\pgfpathlineto{\pgfqpoint{2.972840in}{1.575000in}}%
\pgfpathlineto{\pgfqpoint{2.974080in}{1.645000in}}%
\pgfpathlineto{\pgfqpoint{2.975320in}{2.030000in}}%
\pgfpathlineto{\pgfqpoint{2.976560in}{2.065000in}}%
\pgfpathlineto{\pgfqpoint{2.977800in}{1.995000in}}%
\pgfpathlineto{\pgfqpoint{2.979040in}{1.750000in}}%
\pgfpathlineto{\pgfqpoint{2.980280in}{2.065000in}}%
\pgfpathlineto{\pgfqpoint{2.982760in}{1.610000in}}%
\pgfpathlineto{\pgfqpoint{2.984000in}{1.890000in}}%
\pgfpathlineto{\pgfqpoint{2.985240in}{1.820000in}}%
\pgfpathlineto{\pgfqpoint{2.986480in}{2.135000in}}%
\pgfpathlineto{\pgfqpoint{2.987720in}{2.135000in}}%
\pgfpathlineto{\pgfqpoint{2.988960in}{1.855000in}}%
\pgfpathlineto{\pgfqpoint{2.990200in}{2.310000in}}%
\pgfpathlineto{\pgfqpoint{2.992680in}{1.540000in}}%
\pgfpathlineto{\pgfqpoint{2.993920in}{2.450000in}}%
\pgfpathlineto{\pgfqpoint{2.996400in}{1.540000in}}%
\pgfpathlineto{\pgfqpoint{3.001360in}{1.995000in}}%
\pgfpathlineto{\pgfqpoint{3.002600in}{1.925000in}}%
\pgfpathlineto{\pgfqpoint{3.003840in}{2.030000in}}%
\pgfpathlineto{\pgfqpoint{3.005080in}{1.645000in}}%
\pgfpathlineto{\pgfqpoint{3.007560in}{2.065000in}}%
\pgfpathlineto{\pgfqpoint{3.010040in}{1.750000in}}%
\pgfpathlineto{\pgfqpoint{3.012520in}{2.310000in}}%
\pgfpathlineto{\pgfqpoint{3.013760in}{1.960000in}}%
\pgfpathlineto{\pgfqpoint{3.015000in}{1.925000in}}%
\pgfpathlineto{\pgfqpoint{3.016240in}{1.680000in}}%
\pgfpathlineto{\pgfqpoint{3.017480in}{2.065000in}}%
\pgfpathlineto{\pgfqpoint{3.018720in}{1.785000in}}%
\pgfpathlineto{\pgfqpoint{3.021200in}{2.100000in}}%
\pgfpathlineto{\pgfqpoint{3.023680in}{1.575000in}}%
\pgfpathlineto{\pgfqpoint{3.024920in}{2.135000in}}%
\pgfpathlineto{\pgfqpoint{3.026160in}{2.135000in}}%
\pgfpathlineto{\pgfqpoint{3.027400in}{1.820000in}}%
\pgfpathlineto{\pgfqpoint{3.028640in}{2.170000in}}%
\pgfpathlineto{\pgfqpoint{3.029880in}{2.135000in}}%
\pgfpathlineto{\pgfqpoint{3.031120in}{2.380000in}}%
\pgfpathlineto{\pgfqpoint{3.032360in}{1.575000in}}%
\pgfpathlineto{\pgfqpoint{3.033600in}{2.065000in}}%
\pgfpathlineto{\pgfqpoint{3.034840in}{2.065000in}}%
\pgfpathlineto{\pgfqpoint{3.036080in}{2.485000in}}%
\pgfpathlineto{\pgfqpoint{3.037320in}{1.540000in}}%
\pgfpathlineto{\pgfqpoint{3.039800in}{1.715000in}}%
\pgfpathlineto{\pgfqpoint{3.041040in}{1.680000in}}%
\pgfpathlineto{\pgfqpoint{3.042280in}{1.680000in}}%
\pgfpathlineto{\pgfqpoint{3.043520in}{2.100000in}}%
\pgfpathlineto{\pgfqpoint{3.044760in}{2.030000in}}%
\pgfpathlineto{\pgfqpoint{3.046000in}{2.205000in}}%
\pgfpathlineto{\pgfqpoint{3.047240in}{1.785000in}}%
\pgfpathlineto{\pgfqpoint{3.049720in}{2.205000in}}%
\pgfpathlineto{\pgfqpoint{3.050960in}{1.855000in}}%
\pgfpathlineto{\pgfqpoint{3.052200in}{2.065000in}}%
\pgfpathlineto{\pgfqpoint{3.053440in}{1.890000in}}%
\pgfpathlineto{\pgfqpoint{3.054680in}{2.100000in}}%
\pgfpathlineto{\pgfqpoint{3.057160in}{1.330000in}}%
\pgfpathlineto{\pgfqpoint{3.058400in}{1.995000in}}%
\pgfpathlineto{\pgfqpoint{3.059640in}{2.030000in}}%
\pgfpathlineto{\pgfqpoint{3.060880in}{2.380000in}}%
\pgfpathlineto{\pgfqpoint{3.064600in}{1.890000in}}%
\pgfpathlineto{\pgfqpoint{3.065840in}{1.995000in}}%
\pgfpathlineto{\pgfqpoint{3.067080in}{1.680000in}}%
\pgfpathlineto{\pgfqpoint{3.068320in}{1.960000in}}%
\pgfpathlineto{\pgfqpoint{3.069560in}{1.890000in}}%
\pgfpathlineto{\pgfqpoint{3.070800in}{2.030000in}}%
\pgfpathlineto{\pgfqpoint{3.072040in}{2.625000in}}%
\pgfpathlineto{\pgfqpoint{3.075760in}{1.400000in}}%
\pgfpathlineto{\pgfqpoint{3.077000in}{1.715000in}}%
\pgfpathlineto{\pgfqpoint{3.078240in}{1.470000in}}%
\pgfpathlineto{\pgfqpoint{3.079480in}{1.820000in}}%
\pgfpathlineto{\pgfqpoint{3.080720in}{1.610000in}}%
\pgfpathlineto{\pgfqpoint{3.081960in}{1.925000in}}%
\pgfpathlineto{\pgfqpoint{3.084440in}{1.750000in}}%
\pgfpathlineto{\pgfqpoint{3.085680in}{1.855000in}}%
\pgfpathlineto{\pgfqpoint{3.086920in}{1.715000in}}%
\pgfpathlineto{\pgfqpoint{3.088160in}{2.345000in}}%
\pgfpathlineto{\pgfqpoint{3.089400in}{1.855000in}}%
\pgfpathlineto{\pgfqpoint{3.090640in}{2.100000in}}%
\pgfpathlineto{\pgfqpoint{3.091880in}{2.065000in}}%
\pgfpathlineto{\pgfqpoint{3.094360in}{1.750000in}}%
\pgfpathlineto{\pgfqpoint{3.095600in}{1.750000in}}%
\pgfpathlineto{\pgfqpoint{3.096840in}{2.030000in}}%
\pgfpathlineto{\pgfqpoint{3.098080in}{1.995000in}}%
\pgfpathlineto{\pgfqpoint{3.099320in}{1.680000in}}%
\pgfpathlineto{\pgfqpoint{3.100560in}{1.960000in}}%
\pgfpathlineto{\pgfqpoint{3.101800in}{1.925000in}}%
\pgfpathlineto{\pgfqpoint{3.104280in}{1.680000in}}%
\pgfpathlineto{\pgfqpoint{3.106760in}{2.135000in}}%
\pgfpathlineto{\pgfqpoint{3.109240in}{1.540000in}}%
\pgfpathlineto{\pgfqpoint{3.110480in}{2.030000in}}%
\pgfpathlineto{\pgfqpoint{3.111720in}{1.715000in}}%
\pgfpathlineto{\pgfqpoint{3.114200in}{2.065000in}}%
\pgfpathlineto{\pgfqpoint{3.115440in}{1.855000in}}%
\pgfpathlineto{\pgfqpoint{3.116680in}{2.275000in}}%
\pgfpathlineto{\pgfqpoint{3.117920in}{2.205000in}}%
\pgfpathlineto{\pgfqpoint{3.119160in}{1.960000in}}%
\pgfpathlineto{\pgfqpoint{3.120400in}{2.065000in}}%
\pgfpathlineto{\pgfqpoint{3.122880in}{2.485000in}}%
\pgfpathlineto{\pgfqpoint{3.124120in}{2.415000in}}%
\pgfpathlineto{\pgfqpoint{3.125360in}{1.785000in}}%
\pgfpathlineto{\pgfqpoint{3.126600in}{2.065000in}}%
\pgfpathlineto{\pgfqpoint{3.127840in}{1.855000in}}%
\pgfpathlineto{\pgfqpoint{3.129080in}{1.995000in}}%
\pgfpathlineto{\pgfqpoint{3.130320in}{1.785000in}}%
\pgfpathlineto{\pgfqpoint{3.131560in}{1.925000in}}%
\pgfpathlineto{\pgfqpoint{3.134040in}{1.505000in}}%
\pgfpathlineto{\pgfqpoint{3.135280in}{1.750000in}}%
\pgfpathlineto{\pgfqpoint{3.136520in}{1.750000in}}%
\pgfpathlineto{\pgfqpoint{3.137760in}{1.925000in}}%
\pgfpathlineto{\pgfqpoint{3.139000in}{1.890000in}}%
\pgfpathlineto{\pgfqpoint{3.140240in}{2.380000in}}%
\pgfpathlineto{\pgfqpoint{3.141480in}{2.205000in}}%
\pgfpathlineto{\pgfqpoint{3.142720in}{2.275000in}}%
\pgfpathlineto{\pgfqpoint{3.143960in}{1.820000in}}%
\pgfpathlineto{\pgfqpoint{3.145200in}{2.170000in}}%
\pgfpathlineto{\pgfqpoint{3.146440in}{2.100000in}}%
\pgfpathlineto{\pgfqpoint{3.147680in}{2.240000in}}%
\pgfpathlineto{\pgfqpoint{3.148920in}{1.995000in}}%
\pgfpathlineto{\pgfqpoint{3.150160in}{2.135000in}}%
\pgfpathlineto{\pgfqpoint{3.151400in}{1.715000in}}%
\pgfpathlineto{\pgfqpoint{3.153880in}{2.415000in}}%
\pgfpathlineto{\pgfqpoint{3.155120in}{1.820000in}}%
\pgfpathlineto{\pgfqpoint{3.156360in}{2.100000in}}%
\pgfpathlineto{\pgfqpoint{3.157600in}{2.100000in}}%
\pgfpathlineto{\pgfqpoint{3.160080in}{1.715000in}}%
\pgfpathlineto{\pgfqpoint{3.161320in}{1.890000in}}%
\pgfpathlineto{\pgfqpoint{3.163800in}{1.295000in}}%
\pgfpathlineto{\pgfqpoint{3.165040in}{1.715000in}}%
\pgfpathlineto{\pgfqpoint{3.167520in}{1.540000in}}%
\pgfpathlineto{\pgfqpoint{3.168760in}{1.575000in}}%
\pgfpathlineto{\pgfqpoint{3.170000in}{1.925000in}}%
\pgfpathlineto{\pgfqpoint{3.171240in}{1.750000in}}%
\pgfpathlineto{\pgfqpoint{3.172480in}{1.960000in}}%
\pgfpathlineto{\pgfqpoint{3.173720in}{1.575000in}}%
\pgfpathlineto{\pgfqpoint{3.174960in}{1.925000in}}%
\pgfpathlineto{\pgfqpoint{3.176200in}{1.890000in}}%
\pgfpathlineto{\pgfqpoint{3.177440in}{1.785000in}}%
\pgfpathlineto{\pgfqpoint{3.179920in}{1.330000in}}%
\pgfpathlineto{\pgfqpoint{3.181160in}{1.960000in}}%
\pgfpathlineto{\pgfqpoint{3.182400in}{1.575000in}}%
\pgfpathlineto{\pgfqpoint{3.183640in}{1.960000in}}%
\pgfpathlineto{\pgfqpoint{3.184880in}{1.645000in}}%
\pgfpathlineto{\pgfqpoint{3.186120in}{1.960000in}}%
\pgfpathlineto{\pgfqpoint{3.187360in}{1.575000in}}%
\pgfpathlineto{\pgfqpoint{3.188600in}{1.575000in}}%
\pgfpathlineto{\pgfqpoint{3.189840in}{1.540000in}}%
\pgfpathlineto{\pgfqpoint{3.191080in}{1.470000in}}%
\pgfpathlineto{\pgfqpoint{3.192320in}{1.995000in}}%
\pgfpathlineto{\pgfqpoint{3.193560in}{1.925000in}}%
\pgfpathlineto{\pgfqpoint{3.194800in}{2.135000in}}%
\pgfpathlineto{\pgfqpoint{3.196040in}{1.890000in}}%
\pgfpathlineto{\pgfqpoint{3.198520in}{2.240000in}}%
\pgfpathlineto{\pgfqpoint{3.199760in}{2.205000in}}%
\pgfpathlineto{\pgfqpoint{3.201000in}{1.820000in}}%
\pgfpathlineto{\pgfqpoint{3.202240in}{2.100000in}}%
\pgfpathlineto{\pgfqpoint{3.203480in}{2.030000in}}%
\pgfpathlineto{\pgfqpoint{3.204720in}{1.750000in}}%
\pgfpathlineto{\pgfqpoint{3.207200in}{1.995000in}}%
\pgfpathlineto{\pgfqpoint{3.208440in}{1.715000in}}%
\pgfpathlineto{\pgfqpoint{3.209680in}{1.785000in}}%
\pgfpathlineto{\pgfqpoint{3.210920in}{1.925000in}}%
\pgfpathlineto{\pgfqpoint{3.212160in}{2.240000in}}%
\pgfpathlineto{\pgfqpoint{3.213400in}{1.715000in}}%
\pgfpathlineto{\pgfqpoint{3.214640in}{1.855000in}}%
\pgfpathlineto{\pgfqpoint{3.215880in}{2.205000in}}%
\pgfpathlineto{\pgfqpoint{3.217120in}{1.890000in}}%
\pgfpathlineto{\pgfqpoint{3.218360in}{2.170000in}}%
\pgfpathlineto{\pgfqpoint{3.220840in}{2.030000in}}%
\pgfpathlineto{\pgfqpoint{3.222080in}{1.610000in}}%
\pgfpathlineto{\pgfqpoint{3.223320in}{2.100000in}}%
\pgfpathlineto{\pgfqpoint{3.224560in}{1.890000in}}%
\pgfpathlineto{\pgfqpoint{3.225800in}{1.960000in}}%
\pgfpathlineto{\pgfqpoint{3.228280in}{1.400000in}}%
\pgfpathlineto{\pgfqpoint{3.230760in}{1.820000in}}%
\pgfpathlineto{\pgfqpoint{3.232000in}{1.855000in}}%
\pgfpathlineto{\pgfqpoint{3.233240in}{1.120000in}}%
\pgfpathlineto{\pgfqpoint{3.234480in}{1.960000in}}%
\pgfpathlineto{\pgfqpoint{3.235720in}{1.820000in}}%
\pgfpathlineto{\pgfqpoint{3.236960in}{2.135000in}}%
\pgfpathlineto{\pgfqpoint{3.239440in}{1.470000in}}%
\pgfpathlineto{\pgfqpoint{3.240680in}{1.785000in}}%
\pgfpathlineto{\pgfqpoint{3.241920in}{1.610000in}}%
\pgfpathlineto{\pgfqpoint{3.244400in}{1.995000in}}%
\pgfpathlineto{\pgfqpoint{3.245640in}{1.750000in}}%
\pgfpathlineto{\pgfqpoint{3.246880in}{2.030000in}}%
\pgfpathlineto{\pgfqpoint{3.249360in}{1.610000in}}%
\pgfpathlineto{\pgfqpoint{3.250600in}{1.435000in}}%
\pgfpathlineto{\pgfqpoint{3.251840in}{1.855000in}}%
\pgfpathlineto{\pgfqpoint{3.254320in}{1.575000in}}%
\pgfpathlineto{\pgfqpoint{3.256800in}{1.715000in}}%
\pgfpathlineto{\pgfqpoint{3.258040in}{1.855000in}}%
\pgfpathlineto{\pgfqpoint{3.260520in}{1.505000in}}%
\pgfpathlineto{\pgfqpoint{3.261760in}{2.065000in}}%
\pgfpathlineto{\pgfqpoint{3.263000in}{1.890000in}}%
\pgfpathlineto{\pgfqpoint{3.264240in}{2.135000in}}%
\pgfpathlineto{\pgfqpoint{3.265480in}{1.750000in}}%
\pgfpathlineto{\pgfqpoint{3.267960in}{1.890000in}}%
\pgfpathlineto{\pgfqpoint{3.269200in}{1.400000in}}%
\pgfpathlineto{\pgfqpoint{3.270440in}{1.400000in}}%
\pgfpathlineto{\pgfqpoint{3.271680in}{1.750000in}}%
\pgfpathlineto{\pgfqpoint{3.272920in}{1.750000in}}%
\pgfpathlineto{\pgfqpoint{3.274160in}{2.100000in}}%
\pgfpathlineto{\pgfqpoint{3.275400in}{1.995000in}}%
\pgfpathlineto{\pgfqpoint{3.276640in}{1.995000in}}%
\pgfpathlineto{\pgfqpoint{3.277880in}{1.820000in}}%
\pgfpathlineto{\pgfqpoint{3.280360in}{2.345000in}}%
\pgfpathlineto{\pgfqpoint{3.281600in}{1.610000in}}%
\pgfpathlineto{\pgfqpoint{3.282840in}{2.170000in}}%
\pgfpathlineto{\pgfqpoint{3.284080in}{1.855000in}}%
\pgfpathlineto{\pgfqpoint{3.285320in}{2.135000in}}%
\pgfpathlineto{\pgfqpoint{3.286560in}{1.960000in}}%
\pgfpathlineto{\pgfqpoint{3.287800in}{2.065000in}}%
\pgfpathlineto{\pgfqpoint{3.289040in}{1.890000in}}%
\pgfpathlineto{\pgfqpoint{3.290280in}{2.100000in}}%
\pgfpathlineto{\pgfqpoint{3.292760in}{2.030000in}}%
\pgfpathlineto{\pgfqpoint{3.295240in}{1.785000in}}%
\pgfpathlineto{\pgfqpoint{3.296480in}{2.170000in}}%
\pgfpathlineto{\pgfqpoint{3.297720in}{1.855000in}}%
\pgfpathlineto{\pgfqpoint{3.298960in}{2.205000in}}%
\pgfpathlineto{\pgfqpoint{3.301440in}{1.995000in}}%
\pgfpathlineto{\pgfqpoint{3.302680in}{2.415000in}}%
\pgfpathlineto{\pgfqpoint{3.303920in}{1.960000in}}%
\pgfpathlineto{\pgfqpoint{3.305160in}{1.995000in}}%
\pgfpathlineto{\pgfqpoint{3.306400in}{1.785000in}}%
\pgfpathlineto{\pgfqpoint{3.307640in}{2.030000in}}%
\pgfpathlineto{\pgfqpoint{3.308880in}{1.365000in}}%
\pgfpathlineto{\pgfqpoint{3.311360in}{1.820000in}}%
\pgfpathlineto{\pgfqpoint{3.313840in}{2.065000in}}%
\pgfpathlineto{\pgfqpoint{3.315080in}{1.575000in}}%
\pgfpathlineto{\pgfqpoint{3.316320in}{2.030000in}}%
\pgfpathlineto{\pgfqpoint{3.317560in}{1.820000in}}%
\pgfpathlineto{\pgfqpoint{3.318800in}{1.855000in}}%
\pgfpathlineto{\pgfqpoint{3.320040in}{1.925000in}}%
\pgfpathlineto{\pgfqpoint{3.322520in}{1.750000in}}%
\pgfpathlineto{\pgfqpoint{3.323760in}{1.190000in}}%
\pgfpathlineto{\pgfqpoint{3.325000in}{1.435000in}}%
\pgfpathlineto{\pgfqpoint{3.326240in}{1.925000in}}%
\pgfpathlineto{\pgfqpoint{3.327480in}{1.540000in}}%
\pgfpathlineto{\pgfqpoint{3.328720in}{1.750000in}}%
\pgfpathlineto{\pgfqpoint{3.329960in}{2.205000in}}%
\pgfpathlineto{\pgfqpoint{3.331200in}{1.890000in}}%
\pgfpathlineto{\pgfqpoint{3.333680in}{2.240000in}}%
\pgfpathlineto{\pgfqpoint{3.334920in}{1.890000in}}%
\pgfpathlineto{\pgfqpoint{3.337400in}{2.065000in}}%
\pgfpathlineto{\pgfqpoint{3.339880in}{1.365000in}}%
\pgfpathlineto{\pgfqpoint{3.341120in}{2.065000in}}%
\pgfpathlineto{\pgfqpoint{3.342360in}{1.855000in}}%
\pgfpathlineto{\pgfqpoint{3.343600in}{2.205000in}}%
\pgfpathlineto{\pgfqpoint{3.346080in}{1.995000in}}%
\pgfpathlineto{\pgfqpoint{3.347320in}{1.750000in}}%
\pgfpathlineto{\pgfqpoint{3.348560in}{1.750000in}}%
\pgfpathlineto{\pgfqpoint{3.352280in}{2.030000in}}%
\pgfpathlineto{\pgfqpoint{3.353520in}{1.470000in}}%
\pgfpathlineto{\pgfqpoint{3.354760in}{1.855000in}}%
\pgfpathlineto{\pgfqpoint{3.356000in}{1.785000in}}%
\pgfpathlineto{\pgfqpoint{3.357240in}{1.960000in}}%
\pgfpathlineto{\pgfqpoint{3.358480in}{1.960000in}}%
\pgfpathlineto{\pgfqpoint{3.359720in}{1.505000in}}%
\pgfpathlineto{\pgfqpoint{3.360960in}{1.645000in}}%
\pgfpathlineto{\pgfqpoint{3.362200in}{1.925000in}}%
\pgfpathlineto{\pgfqpoint{3.364680in}{1.680000in}}%
\pgfpathlineto{\pgfqpoint{3.365920in}{1.750000in}}%
\pgfpathlineto{\pgfqpoint{3.367160in}{1.890000in}}%
\pgfpathlineto{\pgfqpoint{3.369640in}{1.680000in}}%
\pgfpathlineto{\pgfqpoint{3.372120in}{2.030000in}}%
\pgfpathlineto{\pgfqpoint{3.373360in}{1.820000in}}%
\pgfpathlineto{\pgfqpoint{3.374600in}{1.925000in}}%
\pgfpathlineto{\pgfqpoint{3.375840in}{2.135000in}}%
\pgfpathlineto{\pgfqpoint{3.377080in}{1.610000in}}%
\pgfpathlineto{\pgfqpoint{3.378320in}{1.995000in}}%
\pgfpathlineto{\pgfqpoint{3.379560in}{1.890000in}}%
\pgfpathlineto{\pgfqpoint{3.380800in}{1.400000in}}%
\pgfpathlineto{\pgfqpoint{3.382040in}{1.960000in}}%
\pgfpathlineto{\pgfqpoint{3.383280in}{1.820000in}}%
\pgfpathlineto{\pgfqpoint{3.384520in}{1.820000in}}%
\pgfpathlineto{\pgfqpoint{3.385760in}{2.030000in}}%
\pgfpathlineto{\pgfqpoint{3.388240in}{1.470000in}}%
\pgfpathlineto{\pgfqpoint{3.389480in}{1.435000in}}%
\pgfpathlineto{\pgfqpoint{3.390720in}{1.820000in}}%
\pgfpathlineto{\pgfqpoint{3.391960in}{1.470000in}}%
\pgfpathlineto{\pgfqpoint{3.393200in}{1.470000in}}%
\pgfpathlineto{\pgfqpoint{3.395680in}{2.030000in}}%
\pgfpathlineto{\pgfqpoint{3.396920in}{1.645000in}}%
\pgfpathlineto{\pgfqpoint{3.398160in}{2.065000in}}%
\pgfpathlineto{\pgfqpoint{3.399400in}{2.030000in}}%
\pgfpathlineto{\pgfqpoint{3.400640in}{1.610000in}}%
\pgfpathlineto{\pgfqpoint{3.401880in}{1.925000in}}%
\pgfpathlineto{\pgfqpoint{3.405600in}{1.715000in}}%
\pgfpathlineto{\pgfqpoint{3.406840in}{1.925000in}}%
\pgfpathlineto{\pgfqpoint{3.409320in}{1.575000in}}%
\pgfpathlineto{\pgfqpoint{3.410560in}{1.785000in}}%
\pgfpathlineto{\pgfqpoint{3.411800in}{2.310000in}}%
\pgfpathlineto{\pgfqpoint{3.413040in}{2.100000in}}%
\pgfpathlineto{\pgfqpoint{3.414280in}{2.100000in}}%
\pgfpathlineto{\pgfqpoint{3.415520in}{1.820000in}}%
\pgfpathlineto{\pgfqpoint{3.416760in}{1.820000in}}%
\pgfpathlineto{\pgfqpoint{3.418000in}{1.645000in}}%
\pgfpathlineto{\pgfqpoint{3.419240in}{2.100000in}}%
\pgfpathlineto{\pgfqpoint{3.420480in}{1.855000in}}%
\pgfpathlineto{\pgfqpoint{3.421720in}{2.240000in}}%
\pgfpathlineto{\pgfqpoint{3.422960in}{1.715000in}}%
\pgfpathlineto{\pgfqpoint{3.424200in}{1.820000in}}%
\pgfpathlineto{\pgfqpoint{3.425440in}{1.820000in}}%
\pgfpathlineto{\pgfqpoint{3.426680in}{1.995000in}}%
\pgfpathlineto{\pgfqpoint{3.427920in}{1.855000in}}%
\pgfpathlineto{\pgfqpoint{3.429160in}{1.960000in}}%
\pgfpathlineto{\pgfqpoint{3.431640in}{1.260000in}}%
\pgfpathlineto{\pgfqpoint{3.432880in}{2.065000in}}%
\pgfpathlineto{\pgfqpoint{3.434120in}{1.505000in}}%
\pgfpathlineto{\pgfqpoint{3.435360in}{1.995000in}}%
\pgfpathlineto{\pgfqpoint{3.436600in}{1.785000in}}%
\pgfpathlineto{\pgfqpoint{3.437840in}{1.820000in}}%
\pgfpathlineto{\pgfqpoint{3.439080in}{1.820000in}}%
\pgfpathlineto{\pgfqpoint{3.440320in}{1.995000in}}%
\pgfpathlineto{\pgfqpoint{3.441560in}{1.750000in}}%
\pgfpathlineto{\pgfqpoint{3.442800in}{1.820000in}}%
\pgfpathlineto{\pgfqpoint{3.444040in}{1.645000in}}%
\pgfpathlineto{\pgfqpoint{3.445280in}{2.170000in}}%
\pgfpathlineto{\pgfqpoint{3.446520in}{1.610000in}}%
\pgfpathlineto{\pgfqpoint{3.447760in}{1.855000in}}%
\pgfpathlineto{\pgfqpoint{3.449000in}{1.820000in}}%
\pgfpathlineto{\pgfqpoint{3.450240in}{1.890000in}}%
\pgfpathlineto{\pgfqpoint{3.451480in}{2.065000in}}%
\pgfpathlineto{\pgfqpoint{3.452720in}{1.960000in}}%
\pgfpathlineto{\pgfqpoint{3.453960in}{1.715000in}}%
\pgfpathlineto{\pgfqpoint{3.455200in}{1.785000in}}%
\pgfpathlineto{\pgfqpoint{3.456440in}{2.170000in}}%
\pgfpathlineto{\pgfqpoint{3.457680in}{1.820000in}}%
\pgfpathlineto{\pgfqpoint{3.458920in}{2.170000in}}%
\pgfpathlineto{\pgfqpoint{3.460160in}{1.890000in}}%
\pgfpathlineto{\pgfqpoint{3.462640in}{1.995000in}}%
\pgfpathlineto{\pgfqpoint{3.463880in}{1.890000in}}%
\pgfpathlineto{\pgfqpoint{3.465120in}{2.170000in}}%
\pgfpathlineto{\pgfqpoint{3.467600in}{1.995000in}}%
\pgfpathlineto{\pgfqpoint{3.468840in}{1.960000in}}%
\pgfpathlineto{\pgfqpoint{3.470080in}{1.960000in}}%
\pgfpathlineto{\pgfqpoint{3.471320in}{1.890000in}}%
\pgfpathlineto{\pgfqpoint{3.472560in}{1.750000in}}%
\pgfpathlineto{\pgfqpoint{3.473800in}{2.030000in}}%
\pgfpathlineto{\pgfqpoint{3.475040in}{2.030000in}}%
\pgfpathlineto{\pgfqpoint{3.476280in}{1.960000in}}%
\pgfpathlineto{\pgfqpoint{3.477520in}{1.295000in}}%
\pgfpathlineto{\pgfqpoint{3.478760in}{2.170000in}}%
\pgfpathlineto{\pgfqpoint{3.480000in}{1.715000in}}%
\pgfpathlineto{\pgfqpoint{3.481240in}{1.960000in}}%
\pgfpathlineto{\pgfqpoint{3.483720in}{1.680000in}}%
\pgfpathlineto{\pgfqpoint{3.484960in}{1.610000in}}%
\pgfpathlineto{\pgfqpoint{3.486200in}{1.750000in}}%
\pgfpathlineto{\pgfqpoint{3.487440in}{2.205000in}}%
\pgfpathlineto{\pgfqpoint{3.488680in}{2.030000in}}%
\pgfpathlineto{\pgfqpoint{3.489920in}{1.645000in}}%
\pgfpathlineto{\pgfqpoint{3.492400in}{2.240000in}}%
\pgfpathlineto{\pgfqpoint{3.493640in}{1.925000in}}%
\pgfpathlineto{\pgfqpoint{3.494880in}{2.100000in}}%
\pgfpathlineto{\pgfqpoint{3.496120in}{1.960000in}}%
\pgfpathlineto{\pgfqpoint{3.497360in}{1.995000in}}%
\pgfpathlineto{\pgfqpoint{3.498600in}{2.205000in}}%
\pgfpathlineto{\pgfqpoint{3.499840in}{2.030000in}}%
\pgfpathlineto{\pgfqpoint{3.501080in}{2.030000in}}%
\pgfpathlineto{\pgfqpoint{3.502320in}{2.065000in}}%
\pgfpathlineto{\pgfqpoint{3.503560in}{2.345000in}}%
\pgfpathlineto{\pgfqpoint{3.504800in}{1.610000in}}%
\pgfpathlineto{\pgfqpoint{3.507280in}{2.030000in}}%
\pgfpathlineto{\pgfqpoint{3.508520in}{1.820000in}}%
\pgfpathlineto{\pgfqpoint{3.509760in}{1.400000in}}%
\pgfpathlineto{\pgfqpoint{3.512240in}{2.030000in}}%
\pgfpathlineto{\pgfqpoint{3.514720in}{1.890000in}}%
\pgfpathlineto{\pgfqpoint{3.515960in}{1.995000in}}%
\pgfpathlineto{\pgfqpoint{3.517200in}{1.855000in}}%
\pgfpathlineto{\pgfqpoint{3.518440in}{1.925000in}}%
\pgfpathlineto{\pgfqpoint{3.519680in}{2.205000in}}%
\pgfpathlineto{\pgfqpoint{3.522160in}{1.750000in}}%
\pgfpathlineto{\pgfqpoint{3.523400in}{1.785000in}}%
\pgfpathlineto{\pgfqpoint{3.524640in}{2.100000in}}%
\pgfpathlineto{\pgfqpoint{3.527120in}{1.785000in}}%
\pgfpathlineto{\pgfqpoint{3.528360in}{1.680000in}}%
\pgfpathlineto{\pgfqpoint{3.529600in}{1.680000in}}%
\pgfpathlineto{\pgfqpoint{3.530840in}{1.925000in}}%
\pgfpathlineto{\pgfqpoint{3.532080in}{1.750000in}}%
\pgfpathlineto{\pgfqpoint{3.534560in}{2.065000in}}%
\pgfpathlineto{\pgfqpoint{3.535800in}{2.100000in}}%
\pgfpathlineto{\pgfqpoint{3.538280in}{1.645000in}}%
\pgfpathlineto{\pgfqpoint{3.539520in}{2.100000in}}%
\pgfpathlineto{\pgfqpoint{3.540760in}{1.645000in}}%
\pgfpathlineto{\pgfqpoint{3.542000in}{1.855000in}}%
\pgfpathlineto{\pgfqpoint{3.543240in}{1.715000in}}%
\pgfpathlineto{\pgfqpoint{3.544480in}{1.435000in}}%
\pgfpathlineto{\pgfqpoint{3.545720in}{2.135000in}}%
\pgfpathlineto{\pgfqpoint{3.548200in}{1.855000in}}%
\pgfpathlineto{\pgfqpoint{3.549440in}{1.925000in}}%
\pgfpathlineto{\pgfqpoint{3.550680in}{1.820000in}}%
\pgfpathlineto{\pgfqpoint{3.551920in}{1.435000in}}%
\pgfpathlineto{\pgfqpoint{3.553160in}{1.995000in}}%
\pgfpathlineto{\pgfqpoint{3.554400in}{1.540000in}}%
\pgfpathlineto{\pgfqpoint{3.555640in}{1.505000in}}%
\pgfpathlineto{\pgfqpoint{3.558120in}{1.855000in}}%
\pgfpathlineto{\pgfqpoint{3.559360in}{1.680000in}}%
\pgfpathlineto{\pgfqpoint{3.560600in}{2.030000in}}%
\pgfpathlineto{\pgfqpoint{3.561840in}{2.030000in}}%
\pgfpathlineto{\pgfqpoint{3.563080in}{1.715000in}}%
\pgfpathlineto{\pgfqpoint{3.564320in}{1.995000in}}%
\pgfpathlineto{\pgfqpoint{3.565560in}{1.855000in}}%
\pgfpathlineto{\pgfqpoint{3.566800in}{2.170000in}}%
\pgfpathlineto{\pgfqpoint{3.568040in}{2.170000in}}%
\pgfpathlineto{\pgfqpoint{3.569280in}{1.750000in}}%
\pgfpathlineto{\pgfqpoint{3.571760in}{2.310000in}}%
\pgfpathlineto{\pgfqpoint{3.574240in}{1.680000in}}%
\pgfpathlineto{\pgfqpoint{3.575480in}{1.995000in}}%
\pgfpathlineto{\pgfqpoint{3.577960in}{1.750000in}}%
\pgfpathlineto{\pgfqpoint{3.580440in}{2.065000in}}%
\pgfpathlineto{\pgfqpoint{3.581680in}{2.065000in}}%
\pgfpathlineto{\pgfqpoint{3.582920in}{2.240000in}}%
\pgfpathlineto{\pgfqpoint{3.584160in}{1.960000in}}%
\pgfpathlineto{\pgfqpoint{3.585400in}{2.240000in}}%
\pgfpathlineto{\pgfqpoint{3.586640in}{1.715000in}}%
\pgfpathlineto{\pgfqpoint{3.587880in}{1.680000in}}%
\pgfpathlineto{\pgfqpoint{3.590360in}{1.925000in}}%
\pgfpathlineto{\pgfqpoint{3.592840in}{1.680000in}}%
\pgfpathlineto{\pgfqpoint{3.594080in}{1.995000in}}%
\pgfpathlineto{\pgfqpoint{3.595320in}{1.785000in}}%
\pgfpathlineto{\pgfqpoint{3.597800in}{2.030000in}}%
\pgfpathlineto{\pgfqpoint{3.600280in}{1.750000in}}%
\pgfpathlineto{\pgfqpoint{3.602760in}{1.995000in}}%
\pgfpathlineto{\pgfqpoint{3.604000in}{2.485000in}}%
\pgfpathlineto{\pgfqpoint{3.605240in}{2.100000in}}%
\pgfpathlineto{\pgfqpoint{3.607720in}{2.380000in}}%
\pgfpathlineto{\pgfqpoint{3.608960in}{1.890000in}}%
\pgfpathlineto{\pgfqpoint{3.610200in}{2.590000in}}%
\pgfpathlineto{\pgfqpoint{3.611440in}{2.135000in}}%
\pgfpathlineto{\pgfqpoint{3.612680in}{2.170000in}}%
\pgfpathlineto{\pgfqpoint{3.613920in}{2.275000in}}%
\pgfpathlineto{\pgfqpoint{3.617640in}{2.065000in}}%
\pgfpathlineto{\pgfqpoint{3.618880in}{2.205000in}}%
\pgfpathlineto{\pgfqpoint{3.621360in}{1.925000in}}%
\pgfpathlineto{\pgfqpoint{3.622600in}{1.960000in}}%
\pgfpathlineto{\pgfqpoint{3.625080in}{1.750000in}}%
\pgfpathlineto{\pgfqpoint{3.626320in}{1.820000in}}%
\pgfpathlineto{\pgfqpoint{3.627560in}{1.960000in}}%
\pgfpathlineto{\pgfqpoint{3.628800in}{1.925000in}}%
\pgfpathlineto{\pgfqpoint{3.630040in}{1.855000in}}%
\pgfpathlineto{\pgfqpoint{3.631280in}{1.505000in}}%
\pgfpathlineto{\pgfqpoint{3.632520in}{2.065000in}}%
\pgfpathlineto{\pgfqpoint{3.633760in}{1.750000in}}%
\pgfpathlineto{\pgfqpoint{3.636240in}{2.205000in}}%
\pgfpathlineto{\pgfqpoint{3.639960in}{1.610000in}}%
\pgfpathlineto{\pgfqpoint{3.641200in}{2.345000in}}%
\pgfpathlineto{\pgfqpoint{3.642440in}{1.890000in}}%
\pgfpathlineto{\pgfqpoint{3.644920in}{2.100000in}}%
\pgfpathlineto{\pgfqpoint{3.646160in}{2.345000in}}%
\pgfpathlineto{\pgfqpoint{3.647400in}{2.310000in}}%
\pgfpathlineto{\pgfqpoint{3.648640in}{1.855000in}}%
\pgfpathlineto{\pgfqpoint{3.651120in}{1.855000in}}%
\pgfpathlineto{\pgfqpoint{3.653600in}{2.345000in}}%
\pgfpathlineto{\pgfqpoint{3.654840in}{1.855000in}}%
\pgfpathlineto{\pgfqpoint{3.656080in}{2.275000in}}%
\pgfpathlineto{\pgfqpoint{3.657320in}{1.855000in}}%
\pgfpathlineto{\pgfqpoint{3.658560in}{1.890000in}}%
\pgfpathlineto{\pgfqpoint{3.659800in}{1.680000in}}%
\pgfpathlineto{\pgfqpoint{3.663520in}{2.100000in}}%
\pgfpathlineto{\pgfqpoint{3.664760in}{2.100000in}}%
\pgfpathlineto{\pgfqpoint{3.666000in}{1.680000in}}%
\pgfpathlineto{\pgfqpoint{3.667240in}{2.030000in}}%
\pgfpathlineto{\pgfqpoint{3.669720in}{1.470000in}}%
\pgfpathlineto{\pgfqpoint{3.673440in}{2.380000in}}%
\pgfpathlineto{\pgfqpoint{3.674680in}{2.275000in}}%
\pgfpathlineto{\pgfqpoint{3.678400in}{1.925000in}}%
\pgfpathlineto{\pgfqpoint{3.679640in}{2.310000in}}%
\pgfpathlineto{\pgfqpoint{3.680880in}{1.855000in}}%
\pgfpathlineto{\pgfqpoint{3.683360in}{1.995000in}}%
\pgfpathlineto{\pgfqpoint{3.684600in}{1.995000in}}%
\pgfpathlineto{\pgfqpoint{3.687080in}{1.715000in}}%
\pgfpathlineto{\pgfqpoint{3.689560in}{2.030000in}}%
\pgfpathlineto{\pgfqpoint{3.690800in}{2.030000in}}%
\pgfpathlineto{\pgfqpoint{3.692040in}{2.100000in}}%
\pgfpathlineto{\pgfqpoint{3.694520in}{1.715000in}}%
\pgfpathlineto{\pgfqpoint{3.695760in}{2.310000in}}%
\pgfpathlineto{\pgfqpoint{3.698240in}{1.855000in}}%
\pgfpathlineto{\pgfqpoint{3.699480in}{1.680000in}}%
\pgfpathlineto{\pgfqpoint{3.700720in}{1.995000in}}%
\pgfpathlineto{\pgfqpoint{3.701960in}{1.785000in}}%
\pgfpathlineto{\pgfqpoint{3.703200in}{2.100000in}}%
\pgfpathlineto{\pgfqpoint{3.704440in}{1.820000in}}%
\pgfpathlineto{\pgfqpoint{3.705680in}{2.135000in}}%
\pgfpathlineto{\pgfqpoint{3.706920in}{2.065000in}}%
\pgfpathlineto{\pgfqpoint{3.709400in}{1.540000in}}%
\pgfpathlineto{\pgfqpoint{3.711880in}{1.610000in}}%
\pgfpathlineto{\pgfqpoint{3.714360in}{2.345000in}}%
\pgfpathlineto{\pgfqpoint{3.715600in}{1.820000in}}%
\pgfpathlineto{\pgfqpoint{3.716840in}{2.240000in}}%
\pgfpathlineto{\pgfqpoint{3.718080in}{2.240000in}}%
\pgfpathlineto{\pgfqpoint{3.719320in}{1.435000in}}%
\pgfpathlineto{\pgfqpoint{3.721800in}{2.065000in}}%
\pgfpathlineto{\pgfqpoint{3.723040in}{1.995000in}}%
\pgfpathlineto{\pgfqpoint{3.724280in}{1.680000in}}%
\pgfpathlineto{\pgfqpoint{3.725520in}{1.925000in}}%
\pgfpathlineto{\pgfqpoint{3.726760in}{1.575000in}}%
\pgfpathlineto{\pgfqpoint{3.729240in}{2.030000in}}%
\pgfpathlineto{\pgfqpoint{3.731720in}{1.680000in}}%
\pgfpathlineto{\pgfqpoint{3.732960in}{1.680000in}}%
\pgfpathlineto{\pgfqpoint{3.734200in}{1.785000in}}%
\pgfpathlineto{\pgfqpoint{3.736680in}{1.680000in}}%
\pgfpathlineto{\pgfqpoint{3.740400in}{2.485000in}}%
\pgfpathlineto{\pgfqpoint{3.741640in}{1.890000in}}%
\pgfpathlineto{\pgfqpoint{3.742880in}{1.995000in}}%
\pgfpathlineto{\pgfqpoint{3.744120in}{2.240000in}}%
\pgfpathlineto{\pgfqpoint{3.746600in}{1.575000in}}%
\pgfpathlineto{\pgfqpoint{3.749080in}{1.925000in}}%
\pgfpathlineto{\pgfqpoint{3.750320in}{1.890000in}}%
\pgfpathlineto{\pgfqpoint{3.751560in}{1.820000in}}%
\pgfpathlineto{\pgfqpoint{3.752800in}{2.065000in}}%
\pgfpathlineto{\pgfqpoint{3.754040in}{1.785000in}}%
\pgfpathlineto{\pgfqpoint{3.755280in}{1.820000in}}%
\pgfpathlineto{\pgfqpoint{3.756520in}{1.995000in}}%
\pgfpathlineto{\pgfqpoint{3.757760in}{2.380000in}}%
\pgfpathlineto{\pgfqpoint{3.761480in}{1.855000in}}%
\pgfpathlineto{\pgfqpoint{3.762720in}{2.450000in}}%
\pgfpathlineto{\pgfqpoint{3.765200in}{1.820000in}}%
\pgfpathlineto{\pgfqpoint{3.766440in}{2.100000in}}%
\pgfpathlineto{\pgfqpoint{3.770160in}{1.540000in}}%
\pgfpathlineto{\pgfqpoint{3.771400in}{1.995000in}}%
\pgfpathlineto{\pgfqpoint{3.772640in}{1.750000in}}%
\pgfpathlineto{\pgfqpoint{3.773880in}{1.855000in}}%
\pgfpathlineto{\pgfqpoint{3.775120in}{1.680000in}}%
\pgfpathlineto{\pgfqpoint{3.776360in}{2.205000in}}%
\pgfpathlineto{\pgfqpoint{3.777600in}{1.890000in}}%
\pgfpathlineto{\pgfqpoint{3.780080in}{2.135000in}}%
\pgfpathlineto{\pgfqpoint{3.781320in}{1.820000in}}%
\pgfpathlineto{\pgfqpoint{3.783800in}{2.135000in}}%
\pgfpathlineto{\pgfqpoint{3.786280in}{1.645000in}}%
\pgfpathlineto{\pgfqpoint{3.791240in}{2.380000in}}%
\pgfpathlineto{\pgfqpoint{3.793720in}{1.750000in}}%
\pgfpathlineto{\pgfqpoint{3.794960in}{2.065000in}}%
\pgfpathlineto{\pgfqpoint{3.797440in}{1.680000in}}%
\pgfpathlineto{\pgfqpoint{3.798680in}{1.785000in}}%
\pgfpathlineto{\pgfqpoint{3.799920in}{1.610000in}}%
\pgfpathlineto{\pgfqpoint{3.802400in}{2.345000in}}%
\pgfpathlineto{\pgfqpoint{3.804880in}{1.540000in}}%
\pgfpathlineto{\pgfqpoint{3.806120in}{1.925000in}}%
\pgfpathlineto{\pgfqpoint{3.807360in}{1.890000in}}%
\pgfpathlineto{\pgfqpoint{3.808600in}{2.135000in}}%
\pgfpathlineto{\pgfqpoint{3.811080in}{1.680000in}}%
\pgfpathlineto{\pgfqpoint{3.812320in}{1.750000in}}%
\pgfpathlineto{\pgfqpoint{3.813560in}{1.680000in}}%
\pgfpathlineto{\pgfqpoint{3.814800in}{1.890000in}}%
\pgfpathlineto{\pgfqpoint{3.816040in}{1.435000in}}%
\pgfpathlineto{\pgfqpoint{3.818520in}{1.890000in}}%
\pgfpathlineto{\pgfqpoint{3.819760in}{1.645000in}}%
\pgfpathlineto{\pgfqpoint{3.822240in}{1.960000in}}%
\pgfpathlineto{\pgfqpoint{3.823480in}{1.960000in}}%
\pgfpathlineto{\pgfqpoint{3.824720in}{1.750000in}}%
\pgfpathlineto{\pgfqpoint{3.825960in}{1.785000in}}%
\pgfpathlineto{\pgfqpoint{3.827200in}{2.100000in}}%
\pgfpathlineto{\pgfqpoint{3.828440in}{1.820000in}}%
\pgfpathlineto{\pgfqpoint{3.829680in}{1.855000in}}%
\pgfpathlineto{\pgfqpoint{3.830920in}{1.540000in}}%
\pgfpathlineto{\pgfqpoint{3.833400in}{2.240000in}}%
\pgfpathlineto{\pgfqpoint{3.835880in}{1.995000in}}%
\pgfpathlineto{\pgfqpoint{3.837120in}{1.680000in}}%
\pgfpathlineto{\pgfqpoint{3.838360in}{1.750000in}}%
\pgfpathlineto{\pgfqpoint{3.839600in}{1.715000in}}%
\pgfpathlineto{\pgfqpoint{3.840840in}{1.400000in}}%
\pgfpathlineto{\pgfqpoint{3.842080in}{2.240000in}}%
\pgfpathlineto{\pgfqpoint{3.843320in}{1.540000in}}%
\pgfpathlineto{\pgfqpoint{3.844560in}{1.925000in}}%
\pgfpathlineto{\pgfqpoint{3.845800in}{1.925000in}}%
\pgfpathlineto{\pgfqpoint{3.847040in}{1.995000in}}%
\pgfpathlineto{\pgfqpoint{3.848280in}{1.925000in}}%
\pgfpathlineto{\pgfqpoint{3.849520in}{2.345000in}}%
\pgfpathlineto{\pgfqpoint{3.850760in}{1.890000in}}%
\pgfpathlineto{\pgfqpoint{3.852000in}{1.890000in}}%
\pgfpathlineto{\pgfqpoint{3.853240in}{1.820000in}}%
\pgfpathlineto{\pgfqpoint{3.854480in}{1.855000in}}%
\pgfpathlineto{\pgfqpoint{3.855720in}{1.435000in}}%
\pgfpathlineto{\pgfqpoint{3.856960in}{1.855000in}}%
\pgfpathlineto{\pgfqpoint{3.858200in}{1.855000in}}%
\pgfpathlineto{\pgfqpoint{3.859440in}{1.645000in}}%
\pgfpathlineto{\pgfqpoint{3.860680in}{2.100000in}}%
\pgfpathlineto{\pgfqpoint{3.861920in}{1.645000in}}%
\pgfpathlineto{\pgfqpoint{3.863160in}{1.680000in}}%
\pgfpathlineto{\pgfqpoint{3.864400in}{1.680000in}}%
\pgfpathlineto{\pgfqpoint{3.865640in}{2.275000in}}%
\pgfpathlineto{\pgfqpoint{3.866880in}{1.855000in}}%
\pgfpathlineto{\pgfqpoint{3.868120in}{2.030000in}}%
\pgfpathlineto{\pgfqpoint{3.870600in}{1.470000in}}%
\pgfpathlineto{\pgfqpoint{3.871840in}{2.065000in}}%
\pgfpathlineto{\pgfqpoint{3.873080in}{1.785000in}}%
\pgfpathlineto{\pgfqpoint{3.874320in}{1.820000in}}%
\pgfpathlineto{\pgfqpoint{3.875560in}{1.540000in}}%
\pgfpathlineto{\pgfqpoint{3.878040in}{1.960000in}}%
\pgfpathlineto{\pgfqpoint{3.879280in}{2.100000in}}%
\pgfpathlineto{\pgfqpoint{3.880520in}{1.610000in}}%
\pgfpathlineto{\pgfqpoint{3.881760in}{2.065000in}}%
\pgfpathlineto{\pgfqpoint{3.883000in}{1.995000in}}%
\pgfpathlineto{\pgfqpoint{3.884240in}{2.205000in}}%
\pgfpathlineto{\pgfqpoint{3.885480in}{1.995000in}}%
\pgfpathlineto{\pgfqpoint{3.886720in}{2.275000in}}%
\pgfpathlineto{\pgfqpoint{3.887960in}{2.065000in}}%
\pgfpathlineto{\pgfqpoint{3.889200in}{2.065000in}}%
\pgfpathlineto{\pgfqpoint{3.890440in}{2.205000in}}%
\pgfpathlineto{\pgfqpoint{3.891680in}{1.820000in}}%
\pgfpathlineto{\pgfqpoint{3.892920in}{1.820000in}}%
\pgfpathlineto{\pgfqpoint{3.894160in}{1.925000in}}%
\pgfpathlineto{\pgfqpoint{3.895400in}{1.680000in}}%
\pgfpathlineto{\pgfqpoint{3.897880in}{2.275000in}}%
\pgfpathlineto{\pgfqpoint{3.901600in}{1.820000in}}%
\pgfpathlineto{\pgfqpoint{3.904080in}{2.100000in}}%
\pgfpathlineto{\pgfqpoint{3.906560in}{1.890000in}}%
\pgfpathlineto{\pgfqpoint{3.907800in}{1.995000in}}%
\pgfpathlineto{\pgfqpoint{3.909040in}{1.995000in}}%
\pgfpathlineto{\pgfqpoint{3.910280in}{1.680000in}}%
\pgfpathlineto{\pgfqpoint{3.911520in}{2.170000in}}%
\pgfpathlineto{\pgfqpoint{3.914000in}{1.645000in}}%
\pgfpathlineto{\pgfqpoint{3.915240in}{1.715000in}}%
\pgfpathlineto{\pgfqpoint{3.917720in}{2.240000in}}%
\pgfpathlineto{\pgfqpoint{3.918960in}{1.505000in}}%
\pgfpathlineto{\pgfqpoint{3.921440in}{1.995000in}}%
\pgfpathlineto{\pgfqpoint{3.922680in}{1.645000in}}%
\pgfpathlineto{\pgfqpoint{3.923920in}{2.065000in}}%
\pgfpathlineto{\pgfqpoint{3.926400in}{1.750000in}}%
\pgfpathlineto{\pgfqpoint{3.927640in}{1.715000in}}%
\pgfpathlineto{\pgfqpoint{3.928880in}{1.995000in}}%
\pgfpathlineto{\pgfqpoint{3.930120in}{1.820000in}}%
\pgfpathlineto{\pgfqpoint{3.931360in}{2.170000in}}%
\pgfpathlineto{\pgfqpoint{3.933840in}{1.715000in}}%
\pgfpathlineto{\pgfqpoint{3.935080in}{1.785000in}}%
\pgfpathlineto{\pgfqpoint{3.936320in}{2.100000in}}%
\pgfpathlineto{\pgfqpoint{3.937560in}{2.100000in}}%
\pgfpathlineto{\pgfqpoint{3.938800in}{2.170000in}}%
\pgfpathlineto{\pgfqpoint{3.940040in}{1.505000in}}%
\pgfpathlineto{\pgfqpoint{3.941280in}{1.470000in}}%
\pgfpathlineto{\pgfqpoint{3.943760in}{1.820000in}}%
\pgfpathlineto{\pgfqpoint{3.945000in}{1.750000in}}%
\pgfpathlineto{\pgfqpoint{3.946240in}{1.400000in}}%
\pgfpathlineto{\pgfqpoint{3.947480in}{1.715000in}}%
\pgfpathlineto{\pgfqpoint{3.948720in}{1.680000in}}%
\pgfpathlineto{\pgfqpoint{3.949960in}{1.715000in}}%
\pgfpathlineto{\pgfqpoint{3.951200in}{2.065000in}}%
\pgfpathlineto{\pgfqpoint{3.952440in}{2.065000in}}%
\pgfpathlineto{\pgfqpoint{3.954920in}{1.610000in}}%
\pgfpathlineto{\pgfqpoint{3.957400in}{2.170000in}}%
\pgfpathlineto{\pgfqpoint{3.958640in}{1.680000in}}%
\pgfpathlineto{\pgfqpoint{3.961120in}{2.205000in}}%
\pgfpathlineto{\pgfqpoint{3.962360in}{2.275000in}}%
\pgfpathlineto{\pgfqpoint{3.963600in}{1.610000in}}%
\pgfpathlineto{\pgfqpoint{3.964840in}{2.275000in}}%
\pgfpathlineto{\pgfqpoint{3.966080in}{2.240000in}}%
\pgfpathlineto{\pgfqpoint{3.967320in}{2.065000in}}%
\pgfpathlineto{\pgfqpoint{3.968560in}{1.645000in}}%
\pgfpathlineto{\pgfqpoint{3.969800in}{1.715000in}}%
\pgfpathlineto{\pgfqpoint{3.971040in}{2.135000in}}%
\pgfpathlineto{\pgfqpoint{3.972280in}{1.890000in}}%
\pgfpathlineto{\pgfqpoint{3.973520in}{1.960000in}}%
\pgfpathlineto{\pgfqpoint{3.974760in}{1.820000in}}%
\pgfpathlineto{\pgfqpoint{3.977240in}{1.995000in}}%
\pgfpathlineto{\pgfqpoint{3.978480in}{1.995000in}}%
\pgfpathlineto{\pgfqpoint{3.979720in}{1.785000in}}%
\pgfpathlineto{\pgfqpoint{3.980960in}{2.240000in}}%
\pgfpathlineto{\pgfqpoint{3.982200in}{2.100000in}}%
\pgfpathlineto{\pgfqpoint{3.983440in}{2.100000in}}%
\pgfpathlineto{\pgfqpoint{3.985920in}{2.415000in}}%
\pgfpathlineto{\pgfqpoint{3.987160in}{1.435000in}}%
\pgfpathlineto{\pgfqpoint{3.989640in}{2.065000in}}%
\pgfpathlineto{\pgfqpoint{3.990880in}{1.960000in}}%
\pgfpathlineto{\pgfqpoint{3.992120in}{2.065000in}}%
\pgfpathlineto{\pgfqpoint{3.993360in}{1.540000in}}%
\pgfpathlineto{\pgfqpoint{3.994600in}{1.925000in}}%
\pgfpathlineto{\pgfqpoint{3.995840in}{1.435000in}}%
\pgfpathlineto{\pgfqpoint{3.998320in}{2.100000in}}%
\pgfpathlineto{\pgfqpoint{4.000800in}{1.575000in}}%
\pgfpathlineto{\pgfqpoint{4.002040in}{1.680000in}}%
\pgfpathlineto{\pgfqpoint{4.003280in}{1.295000in}}%
\pgfpathlineto{\pgfqpoint{4.004520in}{2.030000in}}%
\pgfpathlineto{\pgfqpoint{4.007000in}{1.680000in}}%
\pgfpathlineto{\pgfqpoint{4.008240in}{1.645000in}}%
\pgfpathlineto{\pgfqpoint{4.009480in}{1.785000in}}%
\pgfpathlineto{\pgfqpoint{4.010720in}{1.540000in}}%
\pgfpathlineto{\pgfqpoint{4.013200in}{1.925000in}}%
\pgfpathlineto{\pgfqpoint{4.014440in}{1.855000in}}%
\pgfpathlineto{\pgfqpoint{4.016920in}{2.240000in}}%
\pgfpathlineto{\pgfqpoint{4.019400in}{2.240000in}}%
\pgfpathlineto{\pgfqpoint{4.024360in}{1.575000in}}%
\pgfpathlineto{\pgfqpoint{4.025600in}{1.575000in}}%
\pgfpathlineto{\pgfqpoint{4.026840in}{1.925000in}}%
\pgfpathlineto{\pgfqpoint{4.028080in}{1.960000in}}%
\pgfpathlineto{\pgfqpoint{4.029320in}{1.785000in}}%
\pgfpathlineto{\pgfqpoint{4.030560in}{1.960000in}}%
\pgfpathlineto{\pgfqpoint{4.031800in}{1.750000in}}%
\pgfpathlineto{\pgfqpoint{4.033040in}{1.785000in}}%
\pgfpathlineto{\pgfqpoint{4.035520in}{2.100000in}}%
\pgfpathlineto{\pgfqpoint{4.036760in}{2.030000in}}%
\pgfpathlineto{\pgfqpoint{4.038000in}{1.645000in}}%
\pgfpathlineto{\pgfqpoint{4.040480in}{2.310000in}}%
\pgfpathlineto{\pgfqpoint{4.041720in}{1.750000in}}%
\pgfpathlineto{\pgfqpoint{4.042960in}{1.960000in}}%
\pgfpathlineto{\pgfqpoint{4.045440in}{1.505000in}}%
\pgfpathlineto{\pgfqpoint{4.047920in}{2.065000in}}%
\pgfpathlineto{\pgfqpoint{4.049160in}{1.960000in}}%
\pgfpathlineto{\pgfqpoint{4.050400in}{1.715000in}}%
\pgfpathlineto{\pgfqpoint{4.052880in}{2.170000in}}%
\pgfpathlineto{\pgfqpoint{4.054120in}{1.540000in}}%
\pgfpathlineto{\pgfqpoint{4.055360in}{1.855000in}}%
\pgfpathlineto{\pgfqpoint{4.056600in}{1.610000in}}%
\pgfpathlineto{\pgfqpoint{4.057840in}{1.855000in}}%
\pgfpathlineto{\pgfqpoint{4.059080in}{1.505000in}}%
\pgfpathlineto{\pgfqpoint{4.060320in}{2.100000in}}%
\pgfpathlineto{\pgfqpoint{4.062800in}{1.610000in}}%
\pgfpathlineto{\pgfqpoint{4.065280in}{2.170000in}}%
\pgfpathlineto{\pgfqpoint{4.067760in}{1.960000in}}%
\pgfpathlineto{\pgfqpoint{4.069000in}{1.960000in}}%
\pgfpathlineto{\pgfqpoint{4.070240in}{2.240000in}}%
\pgfpathlineto{\pgfqpoint{4.072720in}{1.470000in}}%
\pgfpathlineto{\pgfqpoint{4.073960in}{2.135000in}}%
\pgfpathlineto{\pgfqpoint{4.076440in}{1.785000in}}%
\pgfpathlineto{\pgfqpoint{4.077680in}{2.030000in}}%
\pgfpathlineto{\pgfqpoint{4.080160in}{1.820000in}}%
\pgfpathlineto{\pgfqpoint{4.081400in}{1.890000in}}%
\pgfpathlineto{\pgfqpoint{4.082640in}{1.855000in}}%
\pgfpathlineto{\pgfqpoint{4.083880in}{1.750000in}}%
\pgfpathlineto{\pgfqpoint{4.086360in}{1.960000in}}%
\pgfpathlineto{\pgfqpoint{4.088840in}{1.680000in}}%
\pgfpathlineto{\pgfqpoint{4.090080in}{1.785000in}}%
\pgfpathlineto{\pgfqpoint{4.091320in}{1.715000in}}%
\pgfpathlineto{\pgfqpoint{4.092560in}{1.470000in}}%
\pgfpathlineto{\pgfqpoint{4.093800in}{1.575000in}}%
\pgfpathlineto{\pgfqpoint{4.095040in}{2.100000in}}%
\pgfpathlineto{\pgfqpoint{4.096280in}{1.225000in}}%
\pgfpathlineto{\pgfqpoint{4.097520in}{2.030000in}}%
\pgfpathlineto{\pgfqpoint{4.098760in}{2.100000in}}%
\pgfpathlineto{\pgfqpoint{4.100000in}{2.100000in}}%
\pgfpathlineto{\pgfqpoint{4.101240in}{2.555000in}}%
\pgfpathlineto{\pgfqpoint{4.103720in}{1.995000in}}%
\pgfpathlineto{\pgfqpoint{4.106200in}{1.855000in}}%
\pgfpathlineto{\pgfqpoint{4.107440in}{1.855000in}}%
\pgfpathlineto{\pgfqpoint{4.108680in}{1.295000in}}%
\pgfpathlineto{\pgfqpoint{4.111160in}{1.995000in}}%
\pgfpathlineto{\pgfqpoint{4.112400in}{1.820000in}}%
\pgfpathlineto{\pgfqpoint{4.113640in}{2.030000in}}%
\pgfpathlineto{\pgfqpoint{4.116120in}{1.785000in}}%
\pgfpathlineto{\pgfqpoint{4.117360in}{2.170000in}}%
\pgfpathlineto{\pgfqpoint{4.119840in}{1.575000in}}%
\pgfpathlineto{\pgfqpoint{4.121080in}{1.785000in}}%
\pgfpathlineto{\pgfqpoint{4.122320in}{2.380000in}}%
\pgfpathlineto{\pgfqpoint{4.123560in}{2.100000in}}%
\pgfpathlineto{\pgfqpoint{4.124800in}{2.100000in}}%
\pgfpathlineto{\pgfqpoint{4.126040in}{2.205000in}}%
\pgfpathlineto{\pgfqpoint{4.127280in}{1.715000in}}%
\pgfpathlineto{\pgfqpoint{4.128520in}{1.995000in}}%
\pgfpathlineto{\pgfqpoint{4.129760in}{1.680000in}}%
\pgfpathlineto{\pgfqpoint{4.131000in}{2.205000in}}%
\pgfpathlineto{\pgfqpoint{4.132240in}{2.205000in}}%
\pgfpathlineto{\pgfqpoint{4.133480in}{1.855000in}}%
\pgfpathlineto{\pgfqpoint{4.134720in}{2.205000in}}%
\pgfpathlineto{\pgfqpoint{4.135960in}{2.205000in}}%
\pgfpathlineto{\pgfqpoint{4.137200in}{1.995000in}}%
\pgfpathlineto{\pgfqpoint{4.138440in}{2.030000in}}%
\pgfpathlineto{\pgfqpoint{4.139680in}{1.575000in}}%
\pgfpathlineto{\pgfqpoint{4.140920in}{1.855000in}}%
\pgfpathlineto{\pgfqpoint{4.143400in}{1.470000in}}%
\pgfpathlineto{\pgfqpoint{4.144640in}{1.890000in}}%
\pgfpathlineto{\pgfqpoint{4.145880in}{1.890000in}}%
\pgfpathlineto{\pgfqpoint{4.147120in}{1.855000in}}%
\pgfpathlineto{\pgfqpoint{4.148360in}{1.750000in}}%
\pgfpathlineto{\pgfqpoint{4.149600in}{1.785000in}}%
\pgfpathlineto{\pgfqpoint{4.150840in}{2.240000in}}%
\pgfpathlineto{\pgfqpoint{4.154560in}{1.540000in}}%
\pgfpathlineto{\pgfqpoint{4.155800in}{1.960000in}}%
\pgfpathlineto{\pgfqpoint{4.157040in}{1.785000in}}%
\pgfpathlineto{\pgfqpoint{4.158280in}{1.785000in}}%
\pgfpathlineto{\pgfqpoint{4.159520in}{1.855000in}}%
\pgfpathlineto{\pgfqpoint{4.160760in}{2.170000in}}%
\pgfpathlineto{\pgfqpoint{4.162000in}{1.750000in}}%
\pgfpathlineto{\pgfqpoint{4.163240in}{2.240000in}}%
\pgfpathlineto{\pgfqpoint{4.164480in}{2.205000in}}%
\pgfpathlineto{\pgfqpoint{4.166960in}{1.575000in}}%
\pgfpathlineto{\pgfqpoint{4.169440in}{1.960000in}}%
\pgfpathlineto{\pgfqpoint{4.171920in}{2.415000in}}%
\pgfpathlineto{\pgfqpoint{4.173160in}{1.995000in}}%
\pgfpathlineto{\pgfqpoint{4.174400in}{2.065000in}}%
\pgfpathlineto{\pgfqpoint{4.176880in}{1.820000in}}%
\pgfpathlineto{\pgfqpoint{4.178120in}{2.170000in}}%
\pgfpathlineto{\pgfqpoint{4.180600in}{1.785000in}}%
\pgfpathlineto{\pgfqpoint{4.181840in}{1.470000in}}%
\pgfpathlineto{\pgfqpoint{4.183080in}{2.135000in}}%
\pgfpathlineto{\pgfqpoint{4.186800in}{1.330000in}}%
\pgfpathlineto{\pgfqpoint{4.188040in}{2.205000in}}%
\pgfpathlineto{\pgfqpoint{4.190520in}{1.330000in}}%
\pgfpathlineto{\pgfqpoint{4.193000in}{1.960000in}}%
\pgfpathlineto{\pgfqpoint{4.194240in}{1.925000in}}%
\pgfpathlineto{\pgfqpoint{4.195480in}{1.925000in}}%
\pgfpathlineto{\pgfqpoint{4.196720in}{1.890000in}}%
\pgfpathlineto{\pgfqpoint{4.197960in}{2.170000in}}%
\pgfpathlineto{\pgfqpoint{4.199200in}{2.100000in}}%
\pgfpathlineto{\pgfqpoint{4.200440in}{1.645000in}}%
\pgfpathlineto{\pgfqpoint{4.202920in}{1.890000in}}%
\pgfpathlineto{\pgfqpoint{4.204160in}{1.610000in}}%
\pgfpathlineto{\pgfqpoint{4.206640in}{1.890000in}}%
\pgfpathlineto{\pgfqpoint{4.207880in}{2.030000in}}%
\pgfpathlineto{\pgfqpoint{4.209120in}{2.310000in}}%
\pgfpathlineto{\pgfqpoint{4.210360in}{1.330000in}}%
\pgfpathlineto{\pgfqpoint{4.211600in}{1.960000in}}%
\pgfpathlineto{\pgfqpoint{4.212840in}{1.470000in}}%
\pgfpathlineto{\pgfqpoint{4.214080in}{1.575000in}}%
\pgfpathlineto{\pgfqpoint{4.216560in}{2.030000in}}%
\pgfpathlineto{\pgfqpoint{4.217800in}{1.785000in}}%
\pgfpathlineto{\pgfqpoint{4.219040in}{1.855000in}}%
\pgfpathlineto{\pgfqpoint{4.220280in}{1.855000in}}%
\pgfpathlineto{\pgfqpoint{4.221520in}{2.065000in}}%
\pgfpathlineto{\pgfqpoint{4.222760in}{1.855000in}}%
\pgfpathlineto{\pgfqpoint{4.225240in}{1.995000in}}%
\pgfpathlineto{\pgfqpoint{4.226480in}{1.820000in}}%
\pgfpathlineto{\pgfqpoint{4.227720in}{1.925000in}}%
\pgfpathlineto{\pgfqpoint{4.228960in}{2.170000in}}%
\pgfpathlineto{\pgfqpoint{4.231440in}{1.750000in}}%
\pgfpathlineto{\pgfqpoint{4.232680in}{2.345000in}}%
\pgfpathlineto{\pgfqpoint{4.233920in}{1.960000in}}%
\pgfpathlineto{\pgfqpoint{4.235160in}{1.925000in}}%
\pgfpathlineto{\pgfqpoint{4.237640in}{1.925000in}}%
\pgfpathlineto{\pgfqpoint{4.238880in}{2.135000in}}%
\pgfpathlineto{\pgfqpoint{4.240120in}{1.890000in}}%
\pgfpathlineto{\pgfqpoint{4.241360in}{2.065000in}}%
\pgfpathlineto{\pgfqpoint{4.242600in}{1.925000in}}%
\pgfpathlineto{\pgfqpoint{4.245080in}{2.030000in}}%
\pgfpathlineto{\pgfqpoint{4.247560in}{1.960000in}}%
\pgfpathlineto{\pgfqpoint{4.248800in}{2.345000in}}%
\pgfpathlineto{\pgfqpoint{4.250040in}{1.995000in}}%
\pgfpathlineto{\pgfqpoint{4.251280in}{2.135000in}}%
\pgfpathlineto{\pgfqpoint{4.252520in}{2.030000in}}%
\pgfpathlineto{\pgfqpoint{4.253760in}{1.575000in}}%
\pgfpathlineto{\pgfqpoint{4.256240in}{1.960000in}}%
\pgfpathlineto{\pgfqpoint{4.257480in}{1.925000in}}%
\pgfpathlineto{\pgfqpoint{4.258720in}{1.925000in}}%
\pgfpathlineto{\pgfqpoint{4.259960in}{1.960000in}}%
\pgfpathlineto{\pgfqpoint{4.261200in}{1.960000in}}%
\pgfpathlineto{\pgfqpoint{4.263680in}{1.400000in}}%
\pgfpathlineto{\pgfqpoint{4.264920in}{1.505000in}}%
\pgfpathlineto{\pgfqpoint{4.267400in}{2.030000in}}%
\pgfpathlineto{\pgfqpoint{4.268640in}{2.100000in}}%
\pgfpathlineto{\pgfqpoint{4.269880in}{2.345000in}}%
\pgfpathlineto{\pgfqpoint{4.272360in}{1.645000in}}%
\pgfpathlineto{\pgfqpoint{4.273600in}{2.065000in}}%
\pgfpathlineto{\pgfqpoint{4.274840in}{2.065000in}}%
\pgfpathlineto{\pgfqpoint{4.276080in}{2.520000in}}%
\pgfpathlineto{\pgfqpoint{4.277320in}{2.275000in}}%
\pgfpathlineto{\pgfqpoint{4.278560in}{1.785000in}}%
\pgfpathlineto{\pgfqpoint{4.279800in}{2.275000in}}%
\pgfpathlineto{\pgfqpoint{4.283520in}{1.785000in}}%
\pgfpathlineto{\pgfqpoint{4.284760in}{2.170000in}}%
\pgfpathlineto{\pgfqpoint{4.286000in}{2.170000in}}%
\pgfpathlineto{\pgfqpoint{4.287240in}{1.960000in}}%
\pgfpathlineto{\pgfqpoint{4.289720in}{2.205000in}}%
\pgfpathlineto{\pgfqpoint{4.290960in}{2.065000in}}%
\pgfpathlineto{\pgfqpoint{4.292200in}{2.100000in}}%
\pgfpathlineto{\pgfqpoint{4.293440in}{2.275000in}}%
\pgfpathlineto{\pgfqpoint{4.294680in}{1.925000in}}%
\pgfpathlineto{\pgfqpoint{4.295920in}{1.960000in}}%
\pgfpathlineto{\pgfqpoint{4.297160in}{1.960000in}}%
\pgfpathlineto{\pgfqpoint{4.298400in}{2.205000in}}%
\pgfpathlineto{\pgfqpoint{4.300880in}{1.820000in}}%
\pgfpathlineto{\pgfqpoint{4.302120in}{1.680000in}}%
\pgfpathlineto{\pgfqpoint{4.304600in}{1.925000in}}%
\pgfpathlineto{\pgfqpoint{4.305840in}{1.505000in}}%
\pgfpathlineto{\pgfqpoint{4.307080in}{1.680000in}}%
\pgfpathlineto{\pgfqpoint{4.308320in}{2.065000in}}%
\pgfpathlineto{\pgfqpoint{4.309560in}{2.100000in}}%
\pgfpathlineto{\pgfqpoint{4.310800in}{1.925000in}}%
\pgfpathlineto{\pgfqpoint{4.312040in}{2.240000in}}%
\pgfpathlineto{\pgfqpoint{4.313280in}{2.240000in}}%
\pgfpathlineto{\pgfqpoint{4.314520in}{2.170000in}}%
\pgfpathlineto{\pgfqpoint{4.315760in}{2.275000in}}%
\pgfpathlineto{\pgfqpoint{4.317000in}{2.065000in}}%
\pgfpathlineto{\pgfqpoint{4.318240in}{2.065000in}}%
\pgfpathlineto{\pgfqpoint{4.320720in}{1.750000in}}%
\pgfpathlineto{\pgfqpoint{4.321960in}{2.135000in}}%
\pgfpathlineto{\pgfqpoint{4.323200in}{1.995000in}}%
\pgfpathlineto{\pgfqpoint{4.324440in}{2.135000in}}%
\pgfpathlineto{\pgfqpoint{4.325680in}{2.135000in}}%
\pgfpathlineto{\pgfqpoint{4.326920in}{1.750000in}}%
\pgfpathlineto{\pgfqpoint{4.329400in}{2.030000in}}%
\pgfpathlineto{\pgfqpoint{4.330640in}{1.645000in}}%
\pgfpathlineto{\pgfqpoint{4.331880in}{1.785000in}}%
\pgfpathlineto{\pgfqpoint{4.333120in}{1.785000in}}%
\pgfpathlineto{\pgfqpoint{4.334360in}{1.610000in}}%
\pgfpathlineto{\pgfqpoint{4.335600in}{2.065000in}}%
\pgfpathlineto{\pgfqpoint{4.336840in}{2.100000in}}%
\pgfpathlineto{\pgfqpoint{4.339320in}{1.610000in}}%
\pgfpathlineto{\pgfqpoint{4.340560in}{1.855000in}}%
\pgfpathlineto{\pgfqpoint{4.341800in}{1.715000in}}%
\pgfpathlineto{\pgfqpoint{4.343040in}{2.065000in}}%
\pgfpathlineto{\pgfqpoint{4.344280in}{1.540000in}}%
\pgfpathlineto{\pgfqpoint{4.346760in}{1.855000in}}%
\pgfpathlineto{\pgfqpoint{4.348000in}{1.505000in}}%
\pgfpathlineto{\pgfqpoint{4.349240in}{1.470000in}}%
\pgfpathlineto{\pgfqpoint{4.350480in}{1.750000in}}%
\pgfpathlineto{\pgfqpoint{4.351720in}{1.365000in}}%
\pgfpathlineto{\pgfqpoint{4.354200in}{1.785000in}}%
\pgfpathlineto{\pgfqpoint{4.356680in}{1.505000in}}%
\pgfpathlineto{\pgfqpoint{4.357920in}{1.540000in}}%
\pgfpathlineto{\pgfqpoint{4.359160in}{1.715000in}}%
\pgfpathlineto{\pgfqpoint{4.360400in}{1.715000in}}%
\pgfpathlineto{\pgfqpoint{4.361640in}{1.540000in}}%
\pgfpathlineto{\pgfqpoint{4.362880in}{1.890000in}}%
\pgfpathlineto{\pgfqpoint{4.364120in}{1.435000in}}%
\pgfpathlineto{\pgfqpoint{4.366600in}{1.995000in}}%
\pgfpathlineto{\pgfqpoint{4.367840in}{1.820000in}}%
\pgfpathlineto{\pgfqpoint{4.369080in}{1.820000in}}%
\pgfpathlineto{\pgfqpoint{4.371560in}{1.995000in}}%
\pgfpathlineto{\pgfqpoint{4.372800in}{1.925000in}}%
\pgfpathlineto{\pgfqpoint{4.374040in}{1.680000in}}%
\pgfpathlineto{\pgfqpoint{4.376520in}{2.030000in}}%
\pgfpathlineto{\pgfqpoint{4.377760in}{1.925000in}}%
\pgfpathlineto{\pgfqpoint{4.379000in}{1.470000in}}%
\pgfpathlineto{\pgfqpoint{4.381480in}{1.855000in}}%
\pgfpathlineto{\pgfqpoint{4.382720in}{2.310000in}}%
\pgfpathlineto{\pgfqpoint{4.385200in}{1.995000in}}%
\pgfpathlineto{\pgfqpoint{4.386440in}{1.995000in}}%
\pgfpathlineto{\pgfqpoint{4.387680in}{1.435000in}}%
\pgfpathlineto{\pgfqpoint{4.390160in}{2.170000in}}%
\pgfpathlineto{\pgfqpoint{4.391400in}{1.995000in}}%
\pgfpathlineto{\pgfqpoint{4.392640in}{2.030000in}}%
\pgfpathlineto{\pgfqpoint{4.393880in}{1.820000in}}%
\pgfpathlineto{\pgfqpoint{4.397600in}{2.310000in}}%
\pgfpathlineto{\pgfqpoint{4.398840in}{2.205000in}}%
\pgfpathlineto{\pgfqpoint{4.400080in}{1.750000in}}%
\pgfpathlineto{\pgfqpoint{4.402560in}{2.100000in}}%
\pgfpathlineto{\pgfqpoint{4.403800in}{1.960000in}}%
\pgfpathlineto{\pgfqpoint{4.405040in}{2.100000in}}%
\pgfpathlineto{\pgfqpoint{4.406280in}{1.680000in}}%
\pgfpathlineto{\pgfqpoint{4.407520in}{1.995000in}}%
\pgfpathlineto{\pgfqpoint{4.410000in}{1.855000in}}%
\pgfpathlineto{\pgfqpoint{4.411240in}{1.995000in}}%
\pgfpathlineto{\pgfqpoint{4.412480in}{1.540000in}}%
\pgfpathlineto{\pgfqpoint{4.413720in}{1.890000in}}%
\pgfpathlineto{\pgfqpoint{4.416200in}{1.575000in}}%
\pgfpathlineto{\pgfqpoint{4.418680in}{2.065000in}}%
\pgfpathlineto{\pgfqpoint{4.419920in}{1.855000in}}%
\pgfpathlineto{\pgfqpoint{4.421160in}{2.170000in}}%
\pgfpathlineto{\pgfqpoint{4.422400in}{2.100000in}}%
\pgfpathlineto{\pgfqpoint{4.423640in}{2.275000in}}%
\pgfpathlineto{\pgfqpoint{4.426120in}{1.680000in}}%
\pgfpathlineto{\pgfqpoint{4.428600in}{2.100000in}}%
\pgfpathlineto{\pgfqpoint{4.431080in}{1.750000in}}%
\pgfpathlineto{\pgfqpoint{4.432320in}{1.715000in}}%
\pgfpathlineto{\pgfqpoint{4.434800in}{1.995000in}}%
\pgfpathlineto{\pgfqpoint{4.436040in}{1.995000in}}%
\pgfpathlineto{\pgfqpoint{4.437280in}{1.855000in}}%
\pgfpathlineto{\pgfqpoint{4.438520in}{2.065000in}}%
\pgfpathlineto{\pgfqpoint{4.439760in}{1.890000in}}%
\pgfpathlineto{\pgfqpoint{4.441000in}{2.170000in}}%
\pgfpathlineto{\pgfqpoint{4.442240in}{1.995000in}}%
\pgfpathlineto{\pgfqpoint{4.443480in}{2.450000in}}%
\pgfpathlineto{\pgfqpoint{4.444720in}{1.890000in}}%
\pgfpathlineto{\pgfqpoint{4.445960in}{2.065000in}}%
\pgfpathlineto{\pgfqpoint{4.448440in}{1.575000in}}%
\pgfpathlineto{\pgfqpoint{4.449680in}{2.275000in}}%
\pgfpathlineto{\pgfqpoint{4.450920in}{1.890000in}}%
\pgfpathlineto{\pgfqpoint{4.452160in}{1.855000in}}%
\pgfpathlineto{\pgfqpoint{4.453400in}{1.995000in}}%
\pgfpathlineto{\pgfqpoint{4.454640in}{1.610000in}}%
\pgfpathlineto{\pgfqpoint{4.455880in}{1.820000in}}%
\pgfpathlineto{\pgfqpoint{4.457120in}{1.715000in}}%
\pgfpathlineto{\pgfqpoint{4.459600in}{1.820000in}}%
\pgfpathlineto{\pgfqpoint{4.462080in}{1.645000in}}%
\pgfpathlineto{\pgfqpoint{4.463320in}{1.680000in}}%
\pgfpathlineto{\pgfqpoint{4.465800in}{2.205000in}}%
\pgfpathlineto{\pgfqpoint{4.468280in}{1.820000in}}%
\pgfpathlineto{\pgfqpoint{4.469520in}{2.240000in}}%
\pgfpathlineto{\pgfqpoint{4.472000in}{1.610000in}}%
\pgfpathlineto{\pgfqpoint{4.473240in}{2.030000in}}%
\pgfpathlineto{\pgfqpoint{4.476960in}{1.645000in}}%
\pgfpathlineto{\pgfqpoint{4.478200in}{1.785000in}}%
\pgfpathlineto{\pgfqpoint{4.479440in}{1.505000in}}%
\pgfpathlineto{\pgfqpoint{4.480680in}{1.925000in}}%
\pgfpathlineto{\pgfqpoint{4.481920in}{1.855000in}}%
\pgfpathlineto{\pgfqpoint{4.483160in}{2.240000in}}%
\pgfpathlineto{\pgfqpoint{4.484400in}{1.610000in}}%
\pgfpathlineto{\pgfqpoint{4.488120in}{2.135000in}}%
\pgfpathlineto{\pgfqpoint{4.490600in}{1.855000in}}%
\pgfpathlineto{\pgfqpoint{4.493080in}{2.135000in}}%
\pgfpathlineto{\pgfqpoint{4.494320in}{1.785000in}}%
\pgfpathlineto{\pgfqpoint{4.495560in}{2.275000in}}%
\pgfpathlineto{\pgfqpoint{4.498040in}{1.960000in}}%
\pgfpathlineto{\pgfqpoint{4.499280in}{2.030000in}}%
\pgfpathlineto{\pgfqpoint{4.501760in}{1.890000in}}%
\pgfpathlineto{\pgfqpoint{4.503000in}{2.030000in}}%
\pgfpathlineto{\pgfqpoint{4.505480in}{1.435000in}}%
\pgfpathlineto{\pgfqpoint{4.507960in}{1.750000in}}%
\pgfpathlineto{\pgfqpoint{4.509200in}{1.365000in}}%
\pgfpathlineto{\pgfqpoint{4.510440in}{2.065000in}}%
\pgfpathlineto{\pgfqpoint{4.511680in}{1.995000in}}%
\pgfpathlineto{\pgfqpoint{4.512920in}{2.135000in}}%
\pgfpathlineto{\pgfqpoint{4.515400in}{1.925000in}}%
\pgfpathlineto{\pgfqpoint{4.517880in}{2.100000in}}%
\pgfpathlineto{\pgfqpoint{4.519120in}{2.485000in}}%
\pgfpathlineto{\pgfqpoint{4.521600in}{1.890000in}}%
\pgfpathlineto{\pgfqpoint{4.522840in}{2.030000in}}%
\pgfpathlineto{\pgfqpoint{4.524080in}{1.890000in}}%
\pgfpathlineto{\pgfqpoint{4.525320in}{2.345000in}}%
\pgfpathlineto{\pgfqpoint{4.526560in}{1.785000in}}%
\pgfpathlineto{\pgfqpoint{4.527800in}{2.205000in}}%
\pgfpathlineto{\pgfqpoint{4.529040in}{2.205000in}}%
\pgfpathlineto{\pgfqpoint{4.530280in}{1.715000in}}%
\pgfpathlineto{\pgfqpoint{4.532760in}{2.100000in}}%
\pgfpathlineto{\pgfqpoint{4.534000in}{1.890000in}}%
\pgfpathlineto{\pgfqpoint{4.535240in}{2.065000in}}%
\pgfpathlineto{\pgfqpoint{4.536480in}{1.890000in}}%
\pgfpathlineto{\pgfqpoint{4.537720in}{2.100000in}}%
\pgfpathlineto{\pgfqpoint{4.538960in}{1.645000in}}%
\pgfpathlineto{\pgfqpoint{4.540200in}{1.715000in}}%
\pgfpathlineto{\pgfqpoint{4.541440in}{1.715000in}}%
\pgfpathlineto{\pgfqpoint{4.543920in}{2.100000in}}%
\pgfpathlineto{\pgfqpoint{4.545160in}{2.030000in}}%
\pgfpathlineto{\pgfqpoint{4.546400in}{1.785000in}}%
\pgfpathlineto{\pgfqpoint{4.547640in}{1.785000in}}%
\pgfpathlineto{\pgfqpoint{4.548880in}{1.715000in}}%
\pgfpathlineto{\pgfqpoint{4.551360in}{2.310000in}}%
\pgfpathlineto{\pgfqpoint{4.555080in}{1.610000in}}%
\pgfpathlineto{\pgfqpoint{4.556320in}{1.680000in}}%
\pgfpathlineto{\pgfqpoint{4.560040in}{2.065000in}}%
\pgfpathlineto{\pgfqpoint{4.561280in}{2.030000in}}%
\pgfpathlineto{\pgfqpoint{4.562520in}{1.645000in}}%
\pgfpathlineto{\pgfqpoint{4.565000in}{2.520000in}}%
\pgfpathlineto{\pgfqpoint{4.566240in}{1.820000in}}%
\pgfpathlineto{\pgfqpoint{4.567480in}{2.065000in}}%
\pgfpathlineto{\pgfqpoint{4.569960in}{1.820000in}}%
\pgfpathlineto{\pgfqpoint{4.571200in}{1.820000in}}%
\pgfpathlineto{\pgfqpoint{4.572440in}{1.960000in}}%
\pgfpathlineto{\pgfqpoint{4.573680in}{1.925000in}}%
\pgfpathlineto{\pgfqpoint{4.574920in}{2.380000in}}%
\pgfpathlineto{\pgfqpoint{4.576160in}{1.820000in}}%
\pgfpathlineto{\pgfqpoint{4.577400in}{2.030000in}}%
\pgfpathlineto{\pgfqpoint{4.578640in}{1.715000in}}%
\pgfpathlineto{\pgfqpoint{4.579880in}{2.275000in}}%
\pgfpathlineto{\pgfqpoint{4.582360in}{1.785000in}}%
\pgfpathlineto{\pgfqpoint{4.583600in}{1.540000in}}%
\pgfpathlineto{\pgfqpoint{4.586080in}{1.925000in}}%
\pgfpathlineto{\pgfqpoint{4.587320in}{1.820000in}}%
\pgfpathlineto{\pgfqpoint{4.588560in}{2.065000in}}%
\pgfpathlineto{\pgfqpoint{4.589800in}{1.330000in}}%
\pgfpathlineto{\pgfqpoint{4.591040in}{1.715000in}}%
\pgfpathlineto{\pgfqpoint{4.592280in}{1.645000in}}%
\pgfpathlineto{\pgfqpoint{4.593520in}{1.365000in}}%
\pgfpathlineto{\pgfqpoint{4.596000in}{2.135000in}}%
\pgfpathlineto{\pgfqpoint{4.599720in}{1.575000in}}%
\pgfpathlineto{\pgfqpoint{4.600960in}{1.785000in}}%
\pgfpathlineto{\pgfqpoint{4.602200in}{1.785000in}}%
\pgfpathlineto{\pgfqpoint{4.603440in}{1.505000in}}%
\pgfpathlineto{\pgfqpoint{4.604680in}{2.170000in}}%
\pgfpathlineto{\pgfqpoint{4.607160in}{1.505000in}}%
\pgfpathlineto{\pgfqpoint{4.608400in}{1.855000in}}%
\pgfpathlineto{\pgfqpoint{4.610880in}{1.540000in}}%
\pgfpathlineto{\pgfqpoint{4.612120in}{1.470000in}}%
\pgfpathlineto{\pgfqpoint{4.614600in}{2.135000in}}%
\pgfpathlineto{\pgfqpoint{4.615840in}{2.170000in}}%
\pgfpathlineto{\pgfqpoint{4.617080in}{1.890000in}}%
\pgfpathlineto{\pgfqpoint{4.618320in}{2.345000in}}%
\pgfpathlineto{\pgfqpoint{4.619560in}{2.100000in}}%
\pgfpathlineto{\pgfqpoint{4.620800in}{2.520000in}}%
\pgfpathlineto{\pgfqpoint{4.623280in}{1.820000in}}%
\pgfpathlineto{\pgfqpoint{4.625760in}{2.030000in}}%
\pgfpathlineto{\pgfqpoint{4.627000in}{1.925000in}}%
\pgfpathlineto{\pgfqpoint{4.628240in}{2.310000in}}%
\pgfpathlineto{\pgfqpoint{4.629480in}{2.100000in}}%
\pgfpathlineto{\pgfqpoint{4.630720in}{2.205000in}}%
\pgfpathlineto{\pgfqpoint{4.631960in}{1.960000in}}%
\pgfpathlineto{\pgfqpoint{4.633200in}{2.065000in}}%
\pgfpathlineto{\pgfqpoint{4.634440in}{1.855000in}}%
\pgfpathlineto{\pgfqpoint{4.639400in}{2.275000in}}%
\pgfpathlineto{\pgfqpoint{4.641880in}{1.925000in}}%
\pgfpathlineto{\pgfqpoint{4.643120in}{1.890000in}}%
\pgfpathlineto{\pgfqpoint{4.645600in}{1.995000in}}%
\pgfpathlineto{\pgfqpoint{4.646840in}{1.960000in}}%
\pgfpathlineto{\pgfqpoint{4.648080in}{1.750000in}}%
\pgfpathlineto{\pgfqpoint{4.651800in}{2.030000in}}%
\pgfpathlineto{\pgfqpoint{4.653040in}{1.715000in}}%
\pgfpathlineto{\pgfqpoint{4.655520in}{1.995000in}}%
\pgfpathlineto{\pgfqpoint{4.656760in}{1.890000in}}%
\pgfpathlineto{\pgfqpoint{4.659240in}{1.365000in}}%
\pgfpathlineto{\pgfqpoint{4.660480in}{1.470000in}}%
\pgfpathlineto{\pgfqpoint{4.661720in}{1.820000in}}%
\pgfpathlineto{\pgfqpoint{4.662960in}{1.400000in}}%
\pgfpathlineto{\pgfqpoint{4.665440in}{1.645000in}}%
\pgfpathlineto{\pgfqpoint{4.666680in}{1.925000in}}%
\pgfpathlineto{\pgfqpoint{4.669160in}{1.470000in}}%
\pgfpathlineto{\pgfqpoint{4.670400in}{2.030000in}}%
\pgfpathlineto{\pgfqpoint{4.671640in}{2.065000in}}%
\pgfpathlineto{\pgfqpoint{4.672880in}{1.715000in}}%
\pgfpathlineto{\pgfqpoint{4.674120in}{2.100000in}}%
\pgfpathlineto{\pgfqpoint{4.675360in}{1.820000in}}%
\pgfpathlineto{\pgfqpoint{4.676600in}{1.995000in}}%
\pgfpathlineto{\pgfqpoint{4.677840in}{1.855000in}}%
\pgfpathlineto{\pgfqpoint{4.679080in}{2.135000in}}%
\pgfpathlineto{\pgfqpoint{4.680320in}{1.785000in}}%
\pgfpathlineto{\pgfqpoint{4.681560in}{1.820000in}}%
\pgfpathlineto{\pgfqpoint{4.682800in}{2.065000in}}%
\pgfpathlineto{\pgfqpoint{4.684040in}{1.645000in}}%
\pgfpathlineto{\pgfqpoint{4.685280in}{1.715000in}}%
\pgfpathlineto{\pgfqpoint{4.686520in}{2.205000in}}%
\pgfpathlineto{\pgfqpoint{4.687760in}{2.240000in}}%
\pgfpathlineto{\pgfqpoint{4.689000in}{2.240000in}}%
\pgfpathlineto{\pgfqpoint{4.690240in}{2.170000in}}%
\pgfpathlineto{\pgfqpoint{4.692720in}{1.995000in}}%
\pgfpathlineto{\pgfqpoint{4.693960in}{2.135000in}}%
\pgfpathlineto{\pgfqpoint{4.695200in}{1.330000in}}%
\pgfpathlineto{\pgfqpoint{4.696440in}{2.065000in}}%
\pgfpathlineto{\pgfqpoint{4.697680in}{2.030000in}}%
\pgfpathlineto{\pgfqpoint{4.700160in}{1.750000in}}%
\pgfpathlineto{\pgfqpoint{4.702640in}{1.750000in}}%
\pgfpathlineto{\pgfqpoint{4.703880in}{1.820000in}}%
\pgfpathlineto{\pgfqpoint{4.706360in}{1.505000in}}%
\pgfpathlineto{\pgfqpoint{4.707600in}{2.205000in}}%
\pgfpathlineto{\pgfqpoint{4.708840in}{1.680000in}}%
\pgfpathlineto{\pgfqpoint{4.710080in}{1.925000in}}%
\pgfpathlineto{\pgfqpoint{4.712560in}{1.680000in}}%
\pgfpathlineto{\pgfqpoint{4.713800in}{1.505000in}}%
\pgfpathlineto{\pgfqpoint{4.715040in}{1.785000in}}%
\pgfpathlineto{\pgfqpoint{4.716280in}{1.575000in}}%
\pgfpathlineto{\pgfqpoint{4.717520in}{1.890000in}}%
\pgfpathlineto{\pgfqpoint{4.718760in}{1.785000in}}%
\pgfpathlineto{\pgfqpoint{4.720000in}{1.575000in}}%
\pgfpathlineto{\pgfqpoint{4.722480in}{1.960000in}}%
\pgfpathlineto{\pgfqpoint{4.723720in}{1.680000in}}%
\pgfpathlineto{\pgfqpoint{4.724960in}{2.100000in}}%
\pgfpathlineto{\pgfqpoint{4.727440in}{1.610000in}}%
\pgfpathlineto{\pgfqpoint{4.729920in}{1.890000in}}%
\pgfpathlineto{\pgfqpoint{4.731160in}{1.610000in}}%
\pgfpathlineto{\pgfqpoint{4.732400in}{2.030000in}}%
\pgfpathlineto{\pgfqpoint{4.733640in}{1.855000in}}%
\pgfpathlineto{\pgfqpoint{4.734880in}{1.505000in}}%
\pgfpathlineto{\pgfqpoint{4.736120in}{1.575000in}}%
\pgfpathlineto{\pgfqpoint{4.738600in}{2.275000in}}%
\pgfpathlineto{\pgfqpoint{4.739840in}{2.170000in}}%
\pgfpathlineto{\pgfqpoint{4.741080in}{1.715000in}}%
\pgfpathlineto{\pgfqpoint{4.742320in}{2.135000in}}%
\pgfpathlineto{\pgfqpoint{4.743560in}{1.470000in}}%
\pgfpathlineto{\pgfqpoint{4.746040in}{1.855000in}}%
\pgfpathlineto{\pgfqpoint{4.747280in}{1.960000in}}%
\pgfpathlineto{\pgfqpoint{4.748520in}{1.785000in}}%
\pgfpathlineto{\pgfqpoint{4.751000in}{2.100000in}}%
\pgfpathlineto{\pgfqpoint{4.752240in}{1.715000in}}%
\pgfpathlineto{\pgfqpoint{4.753480in}{1.750000in}}%
\pgfpathlineto{\pgfqpoint{4.754720in}{1.925000in}}%
\pgfpathlineto{\pgfqpoint{4.755960in}{1.855000in}}%
\pgfpathlineto{\pgfqpoint{4.759680in}{2.310000in}}%
\pgfpathlineto{\pgfqpoint{4.760920in}{1.855000in}}%
\pgfpathlineto{\pgfqpoint{4.762160in}{1.995000in}}%
\pgfpathlineto{\pgfqpoint{4.763400in}{1.750000in}}%
\pgfpathlineto{\pgfqpoint{4.764640in}{1.855000in}}%
\pgfpathlineto{\pgfqpoint{4.765880in}{1.750000in}}%
\pgfpathlineto{\pgfqpoint{4.768360in}{1.995000in}}%
\pgfpathlineto{\pgfqpoint{4.769600in}{1.995000in}}%
\pgfpathlineto{\pgfqpoint{4.770840in}{2.240000in}}%
\pgfpathlineto{\pgfqpoint{4.772080in}{1.750000in}}%
\pgfpathlineto{\pgfqpoint{4.773320in}{1.750000in}}%
\pgfpathlineto{\pgfqpoint{4.774560in}{1.540000in}}%
\pgfpathlineto{\pgfqpoint{4.775800in}{1.785000in}}%
\pgfpathlineto{\pgfqpoint{4.777040in}{1.470000in}}%
\pgfpathlineto{\pgfqpoint{4.778280in}{1.750000in}}%
\pgfpathlineto{\pgfqpoint{4.779520in}{1.575000in}}%
\pgfpathlineto{\pgfqpoint{4.780760in}{2.135000in}}%
\pgfpathlineto{\pgfqpoint{4.783240in}{1.575000in}}%
\pgfpathlineto{\pgfqpoint{4.784480in}{2.030000in}}%
\pgfpathlineto{\pgfqpoint{4.785720in}{1.330000in}}%
\pgfpathlineto{\pgfqpoint{4.788200in}{2.100000in}}%
\pgfpathlineto{\pgfqpoint{4.790680in}{1.715000in}}%
\pgfpathlineto{\pgfqpoint{4.791920in}{1.960000in}}%
\pgfpathlineto{\pgfqpoint{4.793160in}{1.960000in}}%
\pgfpathlineto{\pgfqpoint{4.794400in}{1.785000in}}%
\pgfpathlineto{\pgfqpoint{4.796880in}{2.205000in}}%
\pgfpathlineto{\pgfqpoint{4.799360in}{1.505000in}}%
\pgfpathlineto{\pgfqpoint{4.800600in}{1.785000in}}%
\pgfpathlineto{\pgfqpoint{4.801840in}{1.505000in}}%
\pgfpathlineto{\pgfqpoint{4.803080in}{1.645000in}}%
\pgfpathlineto{\pgfqpoint{4.804320in}{1.575000in}}%
\pgfpathlineto{\pgfqpoint{4.805560in}{1.960000in}}%
\pgfpathlineto{\pgfqpoint{4.806800in}{1.785000in}}%
\pgfpathlineto{\pgfqpoint{4.809280in}{2.170000in}}%
\pgfpathlineto{\pgfqpoint{4.810520in}{1.820000in}}%
\pgfpathlineto{\pgfqpoint{4.813000in}{2.240000in}}%
\pgfpathlineto{\pgfqpoint{4.814240in}{2.240000in}}%
\pgfpathlineto{\pgfqpoint{4.815480in}{2.030000in}}%
\pgfpathlineto{\pgfqpoint{4.816720in}{2.380000in}}%
\pgfpathlineto{\pgfqpoint{4.819200in}{1.750000in}}%
\pgfpathlineto{\pgfqpoint{4.820440in}{1.890000in}}%
\pgfpathlineto{\pgfqpoint{4.821680in}{2.170000in}}%
\pgfpathlineto{\pgfqpoint{4.822920in}{1.995000in}}%
\pgfpathlineto{\pgfqpoint{4.824160in}{2.065000in}}%
\pgfpathlineto{\pgfqpoint{4.825400in}{1.575000in}}%
\pgfpathlineto{\pgfqpoint{4.826640in}{2.030000in}}%
\pgfpathlineto{\pgfqpoint{4.827880in}{1.610000in}}%
\pgfpathlineto{\pgfqpoint{4.829120in}{1.645000in}}%
\pgfpathlineto{\pgfqpoint{4.830360in}{1.785000in}}%
\pgfpathlineto{\pgfqpoint{4.832840in}{1.575000in}}%
\pgfpathlineto{\pgfqpoint{4.834080in}{1.750000in}}%
\pgfpathlineto{\pgfqpoint{4.835320in}{2.135000in}}%
\pgfpathlineto{\pgfqpoint{4.836560in}{2.170000in}}%
\pgfpathlineto{\pgfqpoint{4.837800in}{1.785000in}}%
\pgfpathlineto{\pgfqpoint{4.839040in}{1.995000in}}%
\pgfpathlineto{\pgfqpoint{4.840280in}{1.750000in}}%
\pgfpathlineto{\pgfqpoint{4.841520in}{1.785000in}}%
\pgfpathlineto{\pgfqpoint{4.842760in}{1.540000in}}%
\pgfpathlineto{\pgfqpoint{4.844000in}{1.645000in}}%
\pgfpathlineto{\pgfqpoint{4.845240in}{1.960000in}}%
\pgfpathlineto{\pgfqpoint{4.846480in}{1.960000in}}%
\pgfpathlineto{\pgfqpoint{4.847720in}{1.645000in}}%
\pgfpathlineto{\pgfqpoint{4.851440in}{1.995000in}}%
\pgfpathlineto{\pgfqpoint{4.853920in}{1.715000in}}%
\pgfpathlineto{\pgfqpoint{4.855160in}{1.890000in}}%
\pgfpathlineto{\pgfqpoint{4.856400in}{1.680000in}}%
\pgfpathlineto{\pgfqpoint{4.858880in}{1.925000in}}%
\pgfpathlineto{\pgfqpoint{4.860120in}{1.610000in}}%
\pgfpathlineto{\pgfqpoint{4.862600in}{1.785000in}}%
\pgfpathlineto{\pgfqpoint{4.863840in}{2.170000in}}%
\pgfpathlineto{\pgfqpoint{4.865080in}{1.715000in}}%
\pgfpathlineto{\pgfqpoint{4.866320in}{1.785000in}}%
\pgfpathlineto{\pgfqpoint{4.867560in}{1.400000in}}%
\pgfpathlineto{\pgfqpoint{4.868800in}{2.100000in}}%
\pgfpathlineto{\pgfqpoint{4.870040in}{2.030000in}}%
\pgfpathlineto{\pgfqpoint{4.871280in}{1.680000in}}%
\pgfpathlineto{\pgfqpoint{4.873760in}{2.065000in}}%
\pgfpathlineto{\pgfqpoint{4.875000in}{1.995000in}}%
\pgfpathlineto{\pgfqpoint{4.877480in}{1.820000in}}%
\pgfpathlineto{\pgfqpoint{4.878720in}{1.785000in}}%
\pgfpathlineto{\pgfqpoint{4.879960in}{1.645000in}}%
\pgfpathlineto{\pgfqpoint{4.881200in}{1.820000in}}%
\pgfpathlineto{\pgfqpoint{4.882440in}{1.680000in}}%
\pgfpathlineto{\pgfqpoint{4.884920in}{1.960000in}}%
\pgfpathlineto{\pgfqpoint{4.886160in}{1.645000in}}%
\pgfpathlineto{\pgfqpoint{4.887400in}{1.960000in}}%
\pgfpathlineto{\pgfqpoint{4.888640in}{1.750000in}}%
\pgfpathlineto{\pgfqpoint{4.889880in}{2.100000in}}%
\pgfpathlineto{\pgfqpoint{4.891120in}{2.100000in}}%
\pgfpathlineto{\pgfqpoint{4.892360in}{1.820000in}}%
\pgfpathlineto{\pgfqpoint{4.894840in}{2.100000in}}%
\pgfpathlineto{\pgfqpoint{4.896080in}{1.890000in}}%
\pgfpathlineto{\pgfqpoint{4.897320in}{2.030000in}}%
\pgfpathlineto{\pgfqpoint{4.898560in}{1.610000in}}%
\pgfpathlineto{\pgfqpoint{4.899800in}{1.925000in}}%
\pgfpathlineto{\pgfqpoint{4.901040in}{1.610000in}}%
\pgfpathlineto{\pgfqpoint{4.902280in}{2.170000in}}%
\pgfpathlineto{\pgfqpoint{4.903520in}{2.030000in}}%
\pgfpathlineto{\pgfqpoint{4.904760in}{2.100000in}}%
\pgfpathlineto{\pgfqpoint{4.906000in}{1.400000in}}%
\pgfpathlineto{\pgfqpoint{4.908480in}{2.100000in}}%
\pgfpathlineto{\pgfqpoint{4.909720in}{2.205000in}}%
\pgfpathlineto{\pgfqpoint{4.910960in}{2.170000in}}%
\pgfpathlineto{\pgfqpoint{4.912200in}{1.820000in}}%
\pgfpathlineto{\pgfqpoint{4.913440in}{2.030000in}}%
\pgfpathlineto{\pgfqpoint{4.914680in}{1.680000in}}%
\pgfpathlineto{\pgfqpoint{4.915920in}{2.030000in}}%
\pgfpathlineto{\pgfqpoint{4.917160in}{1.960000in}}%
\pgfpathlineto{\pgfqpoint{4.918400in}{2.135000in}}%
\pgfpathlineto{\pgfqpoint{4.919640in}{1.575000in}}%
\pgfpathlineto{\pgfqpoint{4.920880in}{1.820000in}}%
\pgfpathlineto{\pgfqpoint{4.923360in}{1.540000in}}%
\pgfpathlineto{\pgfqpoint{4.925840in}{2.100000in}}%
\pgfpathlineto{\pgfqpoint{4.928320in}{1.855000in}}%
\pgfpathlineto{\pgfqpoint{4.929560in}{1.995000in}}%
\pgfpathlineto{\pgfqpoint{4.930800in}{1.470000in}}%
\pgfpathlineto{\pgfqpoint{4.932040in}{2.170000in}}%
\pgfpathlineto{\pgfqpoint{4.933280in}{1.645000in}}%
\pgfpathlineto{\pgfqpoint{4.934520in}{2.275000in}}%
\pgfpathlineto{\pgfqpoint{4.935760in}{1.925000in}}%
\pgfpathlineto{\pgfqpoint{4.937000in}{2.065000in}}%
\pgfpathlineto{\pgfqpoint{4.939480in}{2.065000in}}%
\pgfpathlineto{\pgfqpoint{4.940720in}{1.575000in}}%
\pgfpathlineto{\pgfqpoint{4.943200in}{1.925000in}}%
\pgfpathlineto{\pgfqpoint{4.944440in}{1.750000in}}%
\pgfpathlineto{\pgfqpoint{4.945680in}{1.855000in}}%
\pgfpathlineto{\pgfqpoint{4.946920in}{1.470000in}}%
\pgfpathlineto{\pgfqpoint{4.948160in}{1.470000in}}%
\pgfpathlineto{\pgfqpoint{4.949400in}{1.995000in}}%
\pgfpathlineto{\pgfqpoint{4.950640in}{1.960000in}}%
\pgfpathlineto{\pgfqpoint{4.951880in}{1.960000in}}%
\pgfpathlineto{\pgfqpoint{4.954360in}{1.575000in}}%
\pgfpathlineto{\pgfqpoint{4.956840in}{1.785000in}}%
\pgfpathlineto{\pgfqpoint{4.958080in}{1.680000in}}%
\pgfpathlineto{\pgfqpoint{4.959320in}{1.995000in}}%
\pgfpathlineto{\pgfqpoint{4.961800in}{1.470000in}}%
\pgfpathlineto{\pgfqpoint{4.963040in}{1.400000in}}%
\pgfpathlineto{\pgfqpoint{4.966760in}{2.030000in}}%
\pgfpathlineto{\pgfqpoint{4.968000in}{1.995000in}}%
\pgfpathlineto{\pgfqpoint{4.969240in}{1.960000in}}%
\pgfpathlineto{\pgfqpoint{4.971720in}{2.100000in}}%
\pgfpathlineto{\pgfqpoint{4.972960in}{2.065000in}}%
\pgfpathlineto{\pgfqpoint{4.974200in}{1.750000in}}%
\pgfpathlineto{\pgfqpoint{4.976680in}{2.030000in}}%
\pgfpathlineto{\pgfqpoint{4.977920in}{1.645000in}}%
\pgfpathlineto{\pgfqpoint{4.979160in}{1.645000in}}%
\pgfpathlineto{\pgfqpoint{4.980400in}{2.275000in}}%
\pgfpathlineto{\pgfqpoint{4.981640in}{2.205000in}}%
\pgfpathlineto{\pgfqpoint{4.985360in}{1.505000in}}%
\pgfpathlineto{\pgfqpoint{4.989080in}{1.855000in}}%
\pgfpathlineto{\pgfqpoint{4.990320in}{1.610000in}}%
\pgfpathlineto{\pgfqpoint{4.994040in}{2.240000in}}%
\pgfpathlineto{\pgfqpoint{4.995280in}{1.820000in}}%
\pgfpathlineto{\pgfqpoint{4.996520in}{2.310000in}}%
\pgfpathlineto{\pgfqpoint{4.999000in}{1.960000in}}%
\pgfpathlineto{\pgfqpoint{5.000240in}{2.030000in}}%
\pgfpathlineto{\pgfqpoint{5.001480in}{1.820000in}}%
\pgfpathlineto{\pgfqpoint{5.002720in}{1.820000in}}%
\pgfpathlineto{\pgfqpoint{5.003960in}{2.100000in}}%
\pgfpathlineto{\pgfqpoint{5.005200in}{1.505000in}}%
\pgfpathlineto{\pgfqpoint{5.006440in}{1.505000in}}%
\pgfpathlineto{\pgfqpoint{5.008920in}{2.310000in}}%
\pgfpathlineto{\pgfqpoint{5.010160in}{2.310000in}}%
\pgfpathlineto{\pgfqpoint{5.011400in}{1.890000in}}%
\pgfpathlineto{\pgfqpoint{5.012640in}{2.100000in}}%
\pgfpathlineto{\pgfqpoint{5.013880in}{1.750000in}}%
\pgfpathlineto{\pgfqpoint{5.015120in}{2.030000in}}%
\pgfpathlineto{\pgfqpoint{5.016360in}{1.785000in}}%
\pgfpathlineto{\pgfqpoint{5.018840in}{2.135000in}}%
\pgfpathlineto{\pgfqpoint{5.020080in}{2.345000in}}%
\pgfpathlineto{\pgfqpoint{5.021320in}{1.610000in}}%
\pgfpathlineto{\pgfqpoint{5.022560in}{1.995000in}}%
\pgfpathlineto{\pgfqpoint{5.023800in}{1.890000in}}%
\pgfpathlineto{\pgfqpoint{5.026280in}{2.240000in}}%
\pgfpathlineto{\pgfqpoint{5.027520in}{2.205000in}}%
\pgfpathlineto{\pgfqpoint{5.028760in}{2.275000in}}%
\pgfpathlineto{\pgfqpoint{5.030000in}{1.855000in}}%
\pgfpathlineto{\pgfqpoint{5.031240in}{1.820000in}}%
\pgfpathlineto{\pgfqpoint{5.032480in}{1.925000in}}%
\pgfpathlineto{\pgfqpoint{5.033720in}{1.925000in}}%
\pgfpathlineto{\pgfqpoint{5.034960in}{1.960000in}}%
\pgfpathlineto{\pgfqpoint{5.037440in}{1.715000in}}%
\pgfpathlineto{\pgfqpoint{5.038680in}{2.065000in}}%
\pgfpathlineto{\pgfqpoint{5.039920in}{1.890000in}}%
\pgfpathlineto{\pgfqpoint{5.041160in}{2.135000in}}%
\pgfpathlineto{\pgfqpoint{5.042400in}{1.855000in}}%
\pgfpathlineto{\pgfqpoint{5.043640in}{1.890000in}}%
\pgfpathlineto{\pgfqpoint{5.044880in}{2.030000in}}%
\pgfpathlineto{\pgfqpoint{5.046120in}{1.680000in}}%
\pgfpathlineto{\pgfqpoint{5.048600in}{2.135000in}}%
\pgfpathlineto{\pgfqpoint{5.049840in}{2.170000in}}%
\pgfpathlineto{\pgfqpoint{5.051080in}{1.645000in}}%
\pgfpathlineto{\pgfqpoint{5.052320in}{2.170000in}}%
\pgfpathlineto{\pgfqpoint{5.053560in}{1.785000in}}%
\pgfpathlineto{\pgfqpoint{5.054800in}{2.275000in}}%
\pgfpathlineto{\pgfqpoint{5.056040in}{1.435000in}}%
\pgfpathlineto{\pgfqpoint{5.057280in}{1.925000in}}%
\pgfpathlineto{\pgfqpoint{5.058520in}{1.890000in}}%
\pgfpathlineto{\pgfqpoint{5.059760in}{2.240000in}}%
\pgfpathlineto{\pgfqpoint{5.061000in}{1.960000in}}%
\pgfpathlineto{\pgfqpoint{5.062240in}{2.240000in}}%
\pgfpathlineto{\pgfqpoint{5.063480in}{1.995000in}}%
\pgfpathlineto{\pgfqpoint{5.064720in}{2.275000in}}%
\pgfpathlineto{\pgfqpoint{5.065960in}{1.575000in}}%
\pgfpathlineto{\pgfqpoint{5.068440in}{2.170000in}}%
\pgfpathlineto{\pgfqpoint{5.072160in}{1.645000in}}%
\pgfpathlineto{\pgfqpoint{5.073400in}{1.925000in}}%
\pgfpathlineto{\pgfqpoint{5.074640in}{1.750000in}}%
\pgfpathlineto{\pgfqpoint{5.077120in}{2.205000in}}%
\pgfpathlineto{\pgfqpoint{5.078360in}{2.240000in}}%
\pgfpathlineto{\pgfqpoint{5.080840in}{1.820000in}}%
\pgfpathlineto{\pgfqpoint{5.082080in}{2.135000in}}%
\pgfpathlineto{\pgfqpoint{5.083320in}{1.820000in}}%
\pgfpathlineto{\pgfqpoint{5.084560in}{1.995000in}}%
\pgfpathlineto{\pgfqpoint{5.085800in}{1.575000in}}%
\pgfpathlineto{\pgfqpoint{5.087040in}{2.170000in}}%
\pgfpathlineto{\pgfqpoint{5.088280in}{2.135000in}}%
\pgfpathlineto{\pgfqpoint{5.089520in}{1.750000in}}%
\pgfpathlineto{\pgfqpoint{5.090760in}{2.065000in}}%
\pgfpathlineto{\pgfqpoint{5.092000in}{1.610000in}}%
\pgfpathlineto{\pgfqpoint{5.093240in}{1.925000in}}%
\pgfpathlineto{\pgfqpoint{5.094480in}{1.750000in}}%
\pgfpathlineto{\pgfqpoint{5.095720in}{1.890000in}}%
\pgfpathlineto{\pgfqpoint{5.096960in}{2.310000in}}%
\pgfpathlineto{\pgfqpoint{5.098200in}{1.715000in}}%
\pgfpathlineto{\pgfqpoint{5.099440in}{2.205000in}}%
\pgfpathlineto{\pgfqpoint{5.100680in}{1.820000in}}%
\pgfpathlineto{\pgfqpoint{5.101920in}{1.960000in}}%
\pgfpathlineto{\pgfqpoint{5.104400in}{1.645000in}}%
\pgfpathlineto{\pgfqpoint{5.106880in}{2.205000in}}%
\pgfpathlineto{\pgfqpoint{5.110600in}{1.715000in}}%
\pgfpathlineto{\pgfqpoint{5.113080in}{2.030000in}}%
\pgfpathlineto{\pgfqpoint{5.114320in}{1.925000in}}%
\pgfpathlineto{\pgfqpoint{5.116800in}{2.065000in}}%
\pgfpathlineto{\pgfqpoint{5.119280in}{1.855000in}}%
\pgfpathlineto{\pgfqpoint{5.120520in}{2.275000in}}%
\pgfpathlineto{\pgfqpoint{5.123000in}{2.065000in}}%
\pgfpathlineto{\pgfqpoint{5.124240in}{1.820000in}}%
\pgfpathlineto{\pgfqpoint{5.127960in}{2.170000in}}%
\pgfpathlineto{\pgfqpoint{5.129200in}{2.135000in}}%
\pgfpathlineto{\pgfqpoint{5.130440in}{1.855000in}}%
\pgfpathlineto{\pgfqpoint{5.131680in}{2.380000in}}%
\pgfpathlineto{\pgfqpoint{5.132920in}{1.960000in}}%
\pgfpathlineto{\pgfqpoint{5.134160in}{1.925000in}}%
\pgfpathlineto{\pgfqpoint{5.136640in}{2.380000in}}%
\pgfpathlineto{\pgfqpoint{5.139120in}{2.100000in}}%
\pgfpathlineto{\pgfqpoint{5.141600in}{1.925000in}}%
\pgfpathlineto{\pgfqpoint{5.144080in}{2.275000in}}%
\pgfpathlineto{\pgfqpoint{5.145320in}{2.100000in}}%
\pgfpathlineto{\pgfqpoint{5.147800in}{2.345000in}}%
\pgfpathlineto{\pgfqpoint{5.150280in}{1.890000in}}%
\pgfpathlineto{\pgfqpoint{5.151520in}{1.715000in}}%
\pgfpathlineto{\pgfqpoint{5.152760in}{2.065000in}}%
\pgfpathlineto{\pgfqpoint{5.155240in}{1.750000in}}%
\pgfpathlineto{\pgfqpoint{5.157720in}{1.925000in}}%
\pgfpathlineto{\pgfqpoint{5.158960in}{1.575000in}}%
\pgfpathlineto{\pgfqpoint{5.160200in}{1.645000in}}%
\pgfpathlineto{\pgfqpoint{5.161440in}{1.470000in}}%
\pgfpathlineto{\pgfqpoint{5.162680in}{1.960000in}}%
\pgfpathlineto{\pgfqpoint{5.163920in}{1.750000in}}%
\pgfpathlineto{\pgfqpoint{5.165160in}{1.785000in}}%
\pgfpathlineto{\pgfqpoint{5.166400in}{1.995000in}}%
\pgfpathlineto{\pgfqpoint{5.167640in}{1.960000in}}%
\pgfpathlineto{\pgfqpoint{5.170120in}{1.750000in}}%
\pgfpathlineto{\pgfqpoint{5.171360in}{1.890000in}}%
\pgfpathlineto{\pgfqpoint{5.172600in}{1.785000in}}%
\pgfpathlineto{\pgfqpoint{5.173840in}{1.855000in}}%
\pgfpathlineto{\pgfqpoint{5.175080in}{1.820000in}}%
\pgfpathlineto{\pgfqpoint{5.176320in}{1.820000in}}%
\pgfpathlineto{\pgfqpoint{5.177560in}{1.890000in}}%
\pgfpathlineto{\pgfqpoint{5.181280in}{1.645000in}}%
\pgfpathlineto{\pgfqpoint{5.183760in}{1.925000in}}%
\pgfpathlineto{\pgfqpoint{5.185000in}{1.890000in}}%
\pgfpathlineto{\pgfqpoint{5.186240in}{2.240000in}}%
\pgfpathlineto{\pgfqpoint{5.187480in}{1.820000in}}%
\pgfpathlineto{\pgfqpoint{5.188720in}{2.100000in}}%
\pgfpathlineto{\pgfqpoint{5.189960in}{1.295000in}}%
\pgfpathlineto{\pgfqpoint{5.191200in}{1.750000in}}%
\pgfpathlineto{\pgfqpoint{5.192440in}{1.645000in}}%
\pgfpathlineto{\pgfqpoint{5.194920in}{1.960000in}}%
\pgfpathlineto{\pgfqpoint{5.196160in}{1.610000in}}%
\pgfpathlineto{\pgfqpoint{5.199880in}{1.995000in}}%
\pgfpathlineto{\pgfqpoint{5.201120in}{1.785000in}}%
\pgfpathlineto{\pgfqpoint{5.202360in}{1.820000in}}%
\pgfpathlineto{\pgfqpoint{5.203600in}{1.820000in}}%
\pgfpathlineto{\pgfqpoint{5.204840in}{2.170000in}}%
\pgfpathlineto{\pgfqpoint{5.206080in}{2.065000in}}%
\pgfpathlineto{\pgfqpoint{5.207320in}{1.820000in}}%
\pgfpathlineto{\pgfqpoint{5.208560in}{1.925000in}}%
\pgfpathlineto{\pgfqpoint{5.209800in}{1.855000in}}%
\pgfpathlineto{\pgfqpoint{5.211040in}{2.240000in}}%
\pgfpathlineto{\pgfqpoint{5.214760in}{1.680000in}}%
\pgfpathlineto{\pgfqpoint{5.216000in}{1.715000in}}%
\pgfpathlineto{\pgfqpoint{5.217240in}{1.785000in}}%
\pgfpathlineto{\pgfqpoint{5.218480in}{1.680000in}}%
\pgfpathlineto{\pgfqpoint{5.219720in}{1.715000in}}%
\pgfpathlineto{\pgfqpoint{5.220960in}{1.715000in}}%
\pgfpathlineto{\pgfqpoint{5.222200in}{1.855000in}}%
\pgfpathlineto{\pgfqpoint{5.223440in}{1.820000in}}%
\pgfpathlineto{\pgfqpoint{5.224680in}{1.855000in}}%
\pgfpathlineto{\pgfqpoint{5.225920in}{1.365000in}}%
\pgfpathlineto{\pgfqpoint{5.227160in}{1.890000in}}%
\pgfpathlineto{\pgfqpoint{5.228400in}{1.505000in}}%
\pgfpathlineto{\pgfqpoint{5.229640in}{1.715000in}}%
\pgfpathlineto{\pgfqpoint{5.230880in}{1.680000in}}%
\pgfpathlineto{\pgfqpoint{5.232120in}{1.890000in}}%
\pgfpathlineto{\pgfqpoint{5.233360in}{1.645000in}}%
\pgfpathlineto{\pgfqpoint{5.234600in}{2.030000in}}%
\pgfpathlineto{\pgfqpoint{5.235840in}{1.925000in}}%
\pgfpathlineto{\pgfqpoint{5.237080in}{2.170000in}}%
\pgfpathlineto{\pgfqpoint{5.238320in}{1.855000in}}%
\pgfpathlineto{\pgfqpoint{5.239560in}{2.275000in}}%
\pgfpathlineto{\pgfqpoint{5.240800in}{1.960000in}}%
\pgfpathlineto{\pgfqpoint{5.243280in}{2.240000in}}%
\pgfpathlineto{\pgfqpoint{5.244520in}{2.030000in}}%
\pgfpathlineto{\pgfqpoint{5.245760in}{2.590000in}}%
\pgfpathlineto{\pgfqpoint{5.247000in}{1.995000in}}%
\pgfpathlineto{\pgfqpoint{5.248240in}{1.960000in}}%
\pgfpathlineto{\pgfqpoint{5.249480in}{2.310000in}}%
\pgfpathlineto{\pgfqpoint{5.251960in}{2.030000in}}%
\pgfpathlineto{\pgfqpoint{5.253200in}{2.065000in}}%
\pgfpathlineto{\pgfqpoint{5.255680in}{2.450000in}}%
\pgfpathlineto{\pgfqpoint{5.256920in}{1.680000in}}%
\pgfpathlineto{\pgfqpoint{5.258160in}{1.680000in}}%
\pgfpathlineto{\pgfqpoint{5.259400in}{1.750000in}}%
\pgfpathlineto{\pgfqpoint{5.260640in}{2.275000in}}%
\pgfpathlineto{\pgfqpoint{5.263120in}{1.750000in}}%
\pgfpathlineto{\pgfqpoint{5.268080in}{2.555000in}}%
\pgfpathlineto{\pgfqpoint{5.270560in}{2.240000in}}%
\pgfpathlineto{\pgfqpoint{5.271800in}{2.275000in}}%
\pgfpathlineto{\pgfqpoint{5.273040in}{2.170000in}}%
\pgfpathlineto{\pgfqpoint{5.274280in}{1.855000in}}%
\pgfpathlineto{\pgfqpoint{5.276760in}{2.240000in}}%
\pgfpathlineto{\pgfqpoint{5.278000in}{1.960000in}}%
\pgfpathlineto{\pgfqpoint{5.279240in}{2.065000in}}%
\pgfpathlineto{\pgfqpoint{5.280480in}{1.995000in}}%
\pgfpathlineto{\pgfqpoint{5.281720in}{1.715000in}}%
\pgfpathlineto{\pgfqpoint{5.284200in}{2.030000in}}%
\pgfpathlineto{\pgfqpoint{5.285440in}{1.785000in}}%
\pgfpathlineto{\pgfqpoint{5.286680in}{1.960000in}}%
\pgfpathlineto{\pgfqpoint{5.289160in}{1.645000in}}%
\pgfpathlineto{\pgfqpoint{5.290400in}{2.345000in}}%
\pgfpathlineto{\pgfqpoint{5.291640in}{1.855000in}}%
\pgfpathlineto{\pgfqpoint{5.292880in}{1.820000in}}%
\pgfpathlineto{\pgfqpoint{5.294120in}{1.715000in}}%
\pgfpathlineto{\pgfqpoint{5.295360in}{1.785000in}}%
\pgfpathlineto{\pgfqpoint{5.296600in}{1.750000in}}%
\pgfpathlineto{\pgfqpoint{5.297840in}{1.750000in}}%
\pgfpathlineto{\pgfqpoint{5.299080in}{1.540000in}}%
\pgfpathlineto{\pgfqpoint{5.301560in}{1.925000in}}%
\pgfpathlineto{\pgfqpoint{5.302800in}{1.470000in}}%
\pgfpathlineto{\pgfqpoint{5.305280in}{1.575000in}}%
\pgfpathlineto{\pgfqpoint{5.309000in}{2.380000in}}%
\pgfpathlineto{\pgfqpoint{5.311480in}{1.960000in}}%
\pgfpathlineto{\pgfqpoint{5.312720in}{2.100000in}}%
\pgfpathlineto{\pgfqpoint{5.313960in}{2.065000in}}%
\pgfpathlineto{\pgfqpoint{5.315200in}{2.100000in}}%
\pgfpathlineto{\pgfqpoint{5.316440in}{2.380000in}}%
\pgfpathlineto{\pgfqpoint{5.320160in}{1.400000in}}%
\pgfpathlineto{\pgfqpoint{5.321400in}{2.415000in}}%
\pgfpathlineto{\pgfqpoint{5.325120in}{1.680000in}}%
\pgfpathlineto{\pgfqpoint{5.327600in}{1.925000in}}%
\pgfpathlineto{\pgfqpoint{5.330080in}{2.100000in}}%
\pgfpathlineto{\pgfqpoint{5.331320in}{1.435000in}}%
\pgfpathlineto{\pgfqpoint{5.332560in}{1.890000in}}%
\pgfpathlineto{\pgfqpoint{5.333800in}{1.750000in}}%
\pgfpathlineto{\pgfqpoint{5.335040in}{2.065000in}}%
\pgfpathlineto{\pgfqpoint{5.336280in}{1.925000in}}%
\pgfpathlineto{\pgfqpoint{5.338760in}{2.135000in}}%
\pgfpathlineto{\pgfqpoint{5.342480in}{1.785000in}}%
\pgfpathlineto{\pgfqpoint{5.343720in}{2.275000in}}%
\pgfpathlineto{\pgfqpoint{5.344960in}{1.785000in}}%
\pgfpathlineto{\pgfqpoint{5.346200in}{1.890000in}}%
\pgfpathlineto{\pgfqpoint{5.347440in}{1.330000in}}%
\pgfpathlineto{\pgfqpoint{5.349920in}{2.275000in}}%
\pgfpathlineto{\pgfqpoint{5.351160in}{2.170000in}}%
\pgfpathlineto{\pgfqpoint{5.353640in}{1.855000in}}%
\pgfpathlineto{\pgfqpoint{5.354880in}{2.310000in}}%
\pgfpathlineto{\pgfqpoint{5.357360in}{1.820000in}}%
\pgfpathlineto{\pgfqpoint{5.358600in}{2.450000in}}%
\pgfpathlineto{\pgfqpoint{5.359840in}{1.610000in}}%
\pgfpathlineto{\pgfqpoint{5.361080in}{1.540000in}}%
\pgfpathlineto{\pgfqpoint{5.362320in}{1.820000in}}%
\pgfpathlineto{\pgfqpoint{5.363560in}{1.785000in}}%
\pgfpathlineto{\pgfqpoint{5.364800in}{1.890000in}}%
\pgfpathlineto{\pgfqpoint{5.366040in}{1.610000in}}%
\pgfpathlineto{\pgfqpoint{5.368520in}{2.170000in}}%
\pgfpathlineto{\pgfqpoint{5.371000in}{1.715000in}}%
\pgfpathlineto{\pgfqpoint{5.373480in}{1.960000in}}%
\pgfpathlineto{\pgfqpoint{5.374720in}{2.030000in}}%
\pgfpathlineto{\pgfqpoint{5.377200in}{1.715000in}}%
\pgfpathlineto{\pgfqpoint{5.378440in}{1.785000in}}%
\pgfpathlineto{\pgfqpoint{5.379680in}{1.715000in}}%
\pgfpathlineto{\pgfqpoint{5.380920in}{1.785000in}}%
\pgfpathlineto{\pgfqpoint{5.382160in}{1.575000in}}%
\pgfpathlineto{\pgfqpoint{5.384640in}{2.240000in}}%
\pgfpathlineto{\pgfqpoint{5.385880in}{2.275000in}}%
\pgfpathlineto{\pgfqpoint{5.387120in}{2.170000in}}%
\pgfpathlineto{\pgfqpoint{5.388360in}{1.855000in}}%
\pgfpathlineto{\pgfqpoint{5.389600in}{2.555000in}}%
\pgfpathlineto{\pgfqpoint{5.390840in}{2.450000in}}%
\pgfpathlineto{\pgfqpoint{5.392080in}{2.450000in}}%
\pgfpathlineto{\pgfqpoint{5.393320in}{2.345000in}}%
\pgfpathlineto{\pgfqpoint{5.394560in}{1.925000in}}%
\pgfpathlineto{\pgfqpoint{5.395800in}{2.100000in}}%
\pgfpathlineto{\pgfqpoint{5.398280in}{1.715000in}}%
\pgfpathlineto{\pgfqpoint{5.400760in}{2.205000in}}%
\pgfpathlineto{\pgfqpoint{5.402000in}{1.820000in}}%
\pgfpathlineto{\pgfqpoint{5.405720in}{2.030000in}}%
\pgfpathlineto{\pgfqpoint{5.408200in}{1.820000in}}%
\pgfpathlineto{\pgfqpoint{5.409440in}{1.820000in}}%
\pgfpathlineto{\pgfqpoint{5.410680in}{1.505000in}}%
\pgfpathlineto{\pgfqpoint{5.411920in}{2.170000in}}%
\pgfpathlineto{\pgfqpoint{5.413160in}{2.170000in}}%
\pgfpathlineto{\pgfqpoint{5.414400in}{1.715000in}}%
\pgfpathlineto{\pgfqpoint{5.415640in}{1.680000in}}%
\pgfpathlineto{\pgfqpoint{5.416880in}{1.610000in}}%
\pgfpathlineto{\pgfqpoint{5.418120in}{1.715000in}}%
\pgfpathlineto{\pgfqpoint{5.419360in}{1.715000in}}%
\pgfpathlineto{\pgfqpoint{5.420600in}{1.470000in}}%
\pgfpathlineto{\pgfqpoint{5.423080in}{2.135000in}}%
\pgfpathlineto{\pgfqpoint{5.424320in}{1.925000in}}%
\pgfpathlineto{\pgfqpoint{5.425560in}{2.100000in}}%
\pgfpathlineto{\pgfqpoint{5.426800in}{1.960000in}}%
\pgfpathlineto{\pgfqpoint{5.428040in}{2.030000in}}%
\pgfpathlineto{\pgfqpoint{5.429280in}{1.890000in}}%
\pgfpathlineto{\pgfqpoint{5.430520in}{1.890000in}}%
\pgfpathlineto{\pgfqpoint{5.431760in}{2.240000in}}%
\pgfpathlineto{\pgfqpoint{5.434240in}{1.890000in}}%
\pgfpathlineto{\pgfqpoint{5.435480in}{2.205000in}}%
\pgfpathlineto{\pgfqpoint{5.436720in}{2.100000in}}%
\pgfpathlineto{\pgfqpoint{5.437960in}{2.240000in}}%
\pgfpathlineto{\pgfqpoint{5.439200in}{1.995000in}}%
\pgfpathlineto{\pgfqpoint{5.440440in}{2.450000in}}%
\pgfpathlineto{\pgfqpoint{5.442920in}{1.995000in}}%
\pgfpathlineto{\pgfqpoint{5.444160in}{2.345000in}}%
\pgfpathlineto{\pgfqpoint{5.445400in}{1.890000in}}%
\pgfpathlineto{\pgfqpoint{5.446640in}{2.065000in}}%
\pgfpathlineto{\pgfqpoint{5.447880in}{1.820000in}}%
\pgfpathlineto{\pgfqpoint{5.449120in}{1.995000in}}%
\pgfpathlineto{\pgfqpoint{5.450360in}{1.680000in}}%
\pgfpathlineto{\pgfqpoint{5.451600in}{1.680000in}}%
\pgfpathlineto{\pgfqpoint{5.454080in}{2.100000in}}%
\pgfpathlineto{\pgfqpoint{5.459040in}{1.645000in}}%
\pgfpathlineto{\pgfqpoint{5.460280in}{2.065000in}}%
\pgfpathlineto{\pgfqpoint{5.461520in}{2.065000in}}%
\pgfpathlineto{\pgfqpoint{5.462760in}{1.995000in}}%
\pgfpathlineto{\pgfqpoint{5.465240in}{1.610000in}}%
\pgfpathlineto{\pgfqpoint{5.467720in}{1.470000in}}%
\pgfpathlineto{\pgfqpoint{5.468960in}{1.995000in}}%
\pgfpathlineto{\pgfqpoint{5.470200in}{1.645000in}}%
\pgfpathlineto{\pgfqpoint{5.472680in}{1.820000in}}%
\pgfpathlineto{\pgfqpoint{5.473920in}{2.205000in}}%
\pgfpathlineto{\pgfqpoint{5.475160in}{1.820000in}}%
\pgfpathlineto{\pgfqpoint{5.477640in}{2.275000in}}%
\pgfpathlineto{\pgfqpoint{5.478880in}{1.925000in}}%
\pgfpathlineto{\pgfqpoint{5.480120in}{1.960000in}}%
\pgfpathlineto{\pgfqpoint{5.481360in}{1.435000in}}%
\pgfpathlineto{\pgfqpoint{5.482600in}{2.170000in}}%
\pgfpathlineto{\pgfqpoint{5.483840in}{1.820000in}}%
\pgfpathlineto{\pgfqpoint{5.485080in}{2.100000in}}%
\pgfpathlineto{\pgfqpoint{5.487560in}{1.470000in}}%
\pgfpathlineto{\pgfqpoint{5.488800in}{1.645000in}}%
\pgfpathlineto{\pgfqpoint{5.490040in}{1.365000in}}%
\pgfpathlineto{\pgfqpoint{5.491280in}{2.135000in}}%
\pgfpathlineto{\pgfqpoint{5.493760in}{1.435000in}}%
\pgfpathlineto{\pgfqpoint{5.495000in}{1.820000in}}%
\pgfpathlineto{\pgfqpoint{5.496240in}{1.680000in}}%
\pgfpathlineto{\pgfqpoint{5.497480in}{2.135000in}}%
\pgfpathlineto{\pgfqpoint{5.498720in}{1.715000in}}%
\pgfpathlineto{\pgfqpoint{5.499960in}{2.205000in}}%
\pgfpathlineto{\pgfqpoint{5.501200in}{2.135000in}}%
\pgfpathlineto{\pgfqpoint{5.502440in}{1.680000in}}%
\pgfpathlineto{\pgfqpoint{5.503680in}{1.715000in}}%
\pgfpathlineto{\pgfqpoint{5.504920in}{1.785000in}}%
\pgfpathlineto{\pgfqpoint{5.506160in}{1.435000in}}%
\pgfpathlineto{\pgfqpoint{5.507400in}{1.785000in}}%
\pgfpathlineto{\pgfqpoint{5.508640in}{1.645000in}}%
\pgfpathlineto{\pgfqpoint{5.509880in}{1.680000in}}%
\pgfpathlineto{\pgfqpoint{5.511120in}{1.540000in}}%
\pgfpathlineto{\pgfqpoint{5.513600in}{1.960000in}}%
\pgfpathlineto{\pgfqpoint{5.514840in}{1.785000in}}%
\pgfpathlineto{\pgfqpoint{5.516080in}{2.135000in}}%
\pgfpathlineto{\pgfqpoint{5.517320in}{1.785000in}}%
\pgfpathlineto{\pgfqpoint{5.518560in}{2.170000in}}%
\pgfpathlineto{\pgfqpoint{5.519800in}{1.435000in}}%
\pgfpathlineto{\pgfqpoint{5.521040in}{2.135000in}}%
\pgfpathlineto{\pgfqpoint{5.522280in}{2.065000in}}%
\pgfpathlineto{\pgfqpoint{5.523520in}{1.925000in}}%
\pgfpathlineto{\pgfqpoint{5.526000in}{1.330000in}}%
\pgfpathlineto{\pgfqpoint{5.527240in}{2.135000in}}%
\pgfpathlineto{\pgfqpoint{5.528480in}{1.750000in}}%
\pgfpathlineto{\pgfqpoint{5.529720in}{1.715000in}}%
\pgfpathlineto{\pgfqpoint{5.530960in}{1.610000in}}%
\pgfpathlineto{\pgfqpoint{5.532200in}{1.960000in}}%
\pgfpathlineto{\pgfqpoint{5.533440in}{1.925000in}}%
\pgfpathlineto{\pgfqpoint{5.534680in}{1.715000in}}%
\pgfpathlineto{\pgfqpoint{5.537160in}{2.205000in}}%
\pgfpathlineto{\pgfqpoint{5.539640in}{1.890000in}}%
\pgfpathlineto{\pgfqpoint{5.540880in}{1.960000in}}%
\pgfpathlineto{\pgfqpoint{5.542120in}{1.645000in}}%
\pgfpathlineto{\pgfqpoint{5.543360in}{2.100000in}}%
\pgfpathlineto{\pgfqpoint{5.544600in}{2.135000in}}%
\pgfpathlineto{\pgfqpoint{5.547080in}{1.610000in}}%
\pgfpathlineto{\pgfqpoint{5.548320in}{2.310000in}}%
\pgfpathlineto{\pgfqpoint{5.549560in}{2.240000in}}%
\pgfpathlineto{\pgfqpoint{5.550800in}{1.960000in}}%
\pgfpathlineto{\pgfqpoint{5.552040in}{1.995000in}}%
\pgfpathlineto{\pgfqpoint{5.553280in}{2.135000in}}%
\pgfpathlineto{\pgfqpoint{5.554520in}{1.610000in}}%
\pgfpathlineto{\pgfqpoint{5.555760in}{1.925000in}}%
\pgfpathlineto{\pgfqpoint{5.557000in}{1.575000in}}%
\pgfpathlineto{\pgfqpoint{5.558240in}{1.820000in}}%
\pgfpathlineto{\pgfqpoint{5.559480in}{1.470000in}}%
\pgfpathlineto{\pgfqpoint{5.561960in}{2.240000in}}%
\pgfpathlineto{\pgfqpoint{5.563200in}{1.960000in}}%
\pgfpathlineto{\pgfqpoint{5.564440in}{2.205000in}}%
\pgfpathlineto{\pgfqpoint{5.565680in}{1.540000in}}%
\pgfpathlineto{\pgfqpoint{5.569400in}{2.135000in}}%
\pgfpathlineto{\pgfqpoint{5.570640in}{1.820000in}}%
\pgfpathlineto{\pgfqpoint{5.571880in}{1.925000in}}%
\pgfpathlineto{\pgfqpoint{5.574360in}{2.240000in}}%
\pgfpathlineto{\pgfqpoint{5.575600in}{2.415000in}}%
\pgfpathlineto{\pgfqpoint{5.576840in}{2.030000in}}%
\pgfpathlineto{\pgfqpoint{5.579320in}{2.485000in}}%
\pgfpathlineto{\pgfqpoint{5.581800in}{1.855000in}}%
\pgfpathlineto{\pgfqpoint{5.583040in}{1.785000in}}%
\pgfpathlineto{\pgfqpoint{5.584280in}{2.135000in}}%
\pgfpathlineto{\pgfqpoint{5.586760in}{1.820000in}}%
\pgfpathlineto{\pgfqpoint{5.589240in}{2.170000in}}%
\pgfpathlineto{\pgfqpoint{5.590480in}{1.645000in}}%
\pgfpathlineto{\pgfqpoint{5.591720in}{2.100000in}}%
\pgfpathlineto{\pgfqpoint{5.594200in}{1.960000in}}%
\pgfpathlineto{\pgfqpoint{5.595440in}{2.100000in}}%
\pgfpathlineto{\pgfqpoint{5.596680in}{1.750000in}}%
\pgfpathlineto{\pgfqpoint{5.597920in}{1.925000in}}%
\pgfpathlineto{\pgfqpoint{5.599160in}{2.310000in}}%
\pgfpathlineto{\pgfqpoint{5.601640in}{2.065000in}}%
\pgfpathlineto{\pgfqpoint{5.602880in}{1.995000in}}%
\pgfpathlineto{\pgfqpoint{5.604120in}{2.170000in}}%
\pgfpathlineto{\pgfqpoint{5.605360in}{1.890000in}}%
\pgfpathlineto{\pgfqpoint{5.606600in}{1.890000in}}%
\pgfpathlineto{\pgfqpoint{5.607840in}{1.680000in}}%
\pgfpathlineto{\pgfqpoint{5.609080in}{2.030000in}}%
\pgfpathlineto{\pgfqpoint{5.611560in}{1.855000in}}%
\pgfpathlineto{\pgfqpoint{5.612800in}{1.890000in}}%
\pgfpathlineto{\pgfqpoint{5.614040in}{1.960000in}}%
\pgfpathlineto{\pgfqpoint{5.615280in}{1.925000in}}%
\pgfpathlineto{\pgfqpoint{5.616520in}{2.240000in}}%
\pgfpathlineto{\pgfqpoint{5.619000in}{1.855000in}}%
\pgfpathlineto{\pgfqpoint{5.621480in}{2.170000in}}%
\pgfpathlineto{\pgfqpoint{5.622720in}{1.890000in}}%
\pgfpathlineto{\pgfqpoint{5.623960in}{2.170000in}}%
\pgfpathlineto{\pgfqpoint{5.625200in}{2.135000in}}%
\pgfpathlineto{\pgfqpoint{5.626440in}{1.575000in}}%
\pgfpathlineto{\pgfqpoint{5.627680in}{2.065000in}}%
\pgfpathlineto{\pgfqpoint{5.628920in}{1.890000in}}%
\pgfpathlineto{\pgfqpoint{5.630160in}{2.135000in}}%
\pgfpathlineto{\pgfqpoint{5.631400in}{1.610000in}}%
\pgfpathlineto{\pgfqpoint{5.632640in}{1.680000in}}%
\pgfpathlineto{\pgfqpoint{5.635120in}{2.135000in}}%
\pgfpathlineto{\pgfqpoint{5.637600in}{1.540000in}}%
\pgfpathlineto{\pgfqpoint{5.640080in}{1.995000in}}%
\pgfpathlineto{\pgfqpoint{5.641320in}{1.960000in}}%
\pgfpathlineto{\pgfqpoint{5.642560in}{1.610000in}}%
\pgfpathlineto{\pgfqpoint{5.643800in}{1.680000in}}%
\pgfpathlineto{\pgfqpoint{5.645040in}{1.995000in}}%
\pgfpathlineto{\pgfqpoint{5.646280in}{1.435000in}}%
\pgfpathlineto{\pgfqpoint{5.648760in}{2.205000in}}%
\pgfpathlineto{\pgfqpoint{5.650000in}{2.170000in}}%
\pgfpathlineto{\pgfqpoint{5.651240in}{2.310000in}}%
\pgfpathlineto{\pgfqpoint{5.653720in}{1.925000in}}%
\pgfpathlineto{\pgfqpoint{5.654960in}{2.275000in}}%
\pgfpathlineto{\pgfqpoint{5.656200in}{2.135000in}}%
\pgfpathlineto{\pgfqpoint{5.657440in}{2.555000in}}%
\pgfpathlineto{\pgfqpoint{5.658680in}{2.030000in}}%
\pgfpathlineto{\pgfqpoint{5.659920in}{1.995000in}}%
\pgfpathlineto{\pgfqpoint{5.663640in}{1.540000in}}%
\pgfpathlineto{\pgfqpoint{5.666120in}{2.170000in}}%
\pgfpathlineto{\pgfqpoint{5.668600in}{1.645000in}}%
\pgfpathlineto{\pgfqpoint{5.669840in}{1.820000in}}%
\pgfpathlineto{\pgfqpoint{5.671080in}{1.820000in}}%
\pgfpathlineto{\pgfqpoint{5.672320in}{2.030000in}}%
\pgfpathlineto{\pgfqpoint{5.673560in}{1.645000in}}%
\pgfpathlineto{\pgfqpoint{5.674800in}{1.995000in}}%
\pgfpathlineto{\pgfqpoint{5.676040in}{1.750000in}}%
\pgfpathlineto{\pgfqpoint{5.677280in}{1.785000in}}%
\pgfpathlineto{\pgfqpoint{5.678520in}{1.470000in}}%
\pgfpathlineto{\pgfqpoint{5.681000in}{1.820000in}}%
\pgfpathlineto{\pgfqpoint{5.682240in}{1.645000in}}%
\pgfpathlineto{\pgfqpoint{5.685960in}{2.135000in}}%
\pgfpathlineto{\pgfqpoint{5.687200in}{1.995000in}}%
\pgfpathlineto{\pgfqpoint{5.688440in}{1.400000in}}%
\pgfpathlineto{\pgfqpoint{5.689680in}{1.925000in}}%
\pgfpathlineto{\pgfqpoint{5.690920in}{1.225000in}}%
\pgfpathlineto{\pgfqpoint{5.692160in}{1.960000in}}%
\pgfpathlineto{\pgfqpoint{5.693400in}{1.855000in}}%
\pgfpathlineto{\pgfqpoint{5.694640in}{1.855000in}}%
\pgfpathlineto{\pgfqpoint{5.697120in}{2.345000in}}%
\pgfpathlineto{\pgfqpoint{5.699600in}{1.435000in}}%
\pgfpathlineto{\pgfqpoint{5.700840in}{1.855000in}}%
\pgfpathlineto{\pgfqpoint{5.702080in}{1.680000in}}%
\pgfpathlineto{\pgfqpoint{5.703320in}{1.820000in}}%
\pgfpathlineto{\pgfqpoint{5.704560in}{1.645000in}}%
\pgfpathlineto{\pgfqpoint{5.705800in}{1.750000in}}%
\pgfpathlineto{\pgfqpoint{5.707040in}{2.065000in}}%
\pgfpathlineto{\pgfqpoint{5.708280in}{1.890000in}}%
\pgfpathlineto{\pgfqpoint{5.709520in}{1.890000in}}%
\pgfpathlineto{\pgfqpoint{5.710760in}{1.750000in}}%
\pgfpathlineto{\pgfqpoint{5.713240in}{2.100000in}}%
\pgfpathlineto{\pgfqpoint{5.714480in}{1.365000in}}%
\pgfpathlineto{\pgfqpoint{5.715720in}{1.855000in}}%
\pgfpathlineto{\pgfqpoint{5.716960in}{1.505000in}}%
\pgfpathlineto{\pgfqpoint{5.718200in}{1.680000in}}%
\pgfpathlineto{\pgfqpoint{5.719440in}{2.345000in}}%
\pgfpathlineto{\pgfqpoint{5.721920in}{1.330000in}}%
\pgfpathlineto{\pgfqpoint{5.723160in}{1.645000in}}%
\pgfpathlineto{\pgfqpoint{5.724400in}{1.330000in}}%
\pgfpathlineto{\pgfqpoint{5.726880in}{1.890000in}}%
\pgfpathlineto{\pgfqpoint{5.728120in}{1.785000in}}%
\pgfpathlineto{\pgfqpoint{5.729360in}{1.820000in}}%
\pgfpathlineto{\pgfqpoint{5.730600in}{1.645000in}}%
\pgfpathlineto{\pgfqpoint{5.731840in}{1.715000in}}%
\pgfpathlineto{\pgfqpoint{5.733080in}{1.890000in}}%
\pgfpathlineto{\pgfqpoint{5.734320in}{1.715000in}}%
\pgfpathlineto{\pgfqpoint{5.736800in}{2.310000in}}%
\pgfpathlineto{\pgfqpoint{5.738040in}{1.960000in}}%
\pgfpathlineto{\pgfqpoint{5.739280in}{2.065000in}}%
\pgfpathlineto{\pgfqpoint{5.741760in}{1.750000in}}%
\pgfpathlineto{\pgfqpoint{5.743000in}{1.750000in}}%
\pgfpathlineto{\pgfqpoint{5.744240in}{2.170000in}}%
\pgfpathlineto{\pgfqpoint{5.746720in}{1.435000in}}%
\pgfpathlineto{\pgfqpoint{5.750440in}{2.170000in}}%
\pgfpathlineto{\pgfqpoint{5.751680in}{1.820000in}}%
\pgfpathlineto{\pgfqpoint{5.754160in}{2.240000in}}%
\pgfpathlineto{\pgfqpoint{5.756640in}{1.645000in}}%
\pgfpathlineto{\pgfqpoint{5.757880in}{1.680000in}}%
\pgfpathlineto{\pgfqpoint{5.760360in}{1.960000in}}%
\pgfpathlineto{\pgfqpoint{5.761600in}{1.960000in}}%
\pgfpathlineto{\pgfqpoint{5.762840in}{1.995000in}}%
\pgfpathlineto{\pgfqpoint{5.764080in}{1.925000in}}%
\pgfpathlineto{\pgfqpoint{5.765320in}{1.925000in}}%
\pgfpathlineto{\pgfqpoint{5.766560in}{1.715000in}}%
\pgfpathlineto{\pgfqpoint{5.767800in}{1.960000in}}%
\pgfpathlineto{\pgfqpoint{5.770280in}{1.610000in}}%
\pgfpathlineto{\pgfqpoint{5.771520in}{1.785000in}}%
\pgfpathlineto{\pgfqpoint{5.772760in}{1.750000in}}%
\pgfpathlineto{\pgfqpoint{5.774000in}{1.995000in}}%
\pgfpathlineto{\pgfqpoint{5.775240in}{1.855000in}}%
\pgfpathlineto{\pgfqpoint{5.776480in}{2.065000in}}%
\pgfpathlineto{\pgfqpoint{5.778960in}{1.750000in}}%
\pgfpathlineto{\pgfqpoint{5.780200in}{1.960000in}}%
\pgfpathlineto{\pgfqpoint{5.785160in}{1.540000in}}%
\pgfpathlineto{\pgfqpoint{5.786400in}{1.575000in}}%
\pgfpathlineto{\pgfqpoint{5.787640in}{1.435000in}}%
\pgfpathlineto{\pgfqpoint{5.788880in}{1.890000in}}%
\pgfpathlineto{\pgfqpoint{5.791360in}{1.610000in}}%
\pgfpathlineto{\pgfqpoint{5.795080in}{2.520000in}}%
\pgfpathlineto{\pgfqpoint{5.796320in}{1.680000in}}%
\pgfpathlineto{\pgfqpoint{5.797560in}{1.785000in}}%
\pgfpathlineto{\pgfqpoint{5.798800in}{2.275000in}}%
\pgfpathlineto{\pgfqpoint{5.800040in}{2.170000in}}%
\pgfpathlineto{\pgfqpoint{5.801280in}{1.855000in}}%
\pgfpathlineto{\pgfqpoint{5.803760in}{2.170000in}}%
\pgfpathlineto{\pgfqpoint{5.807480in}{1.645000in}}%
\pgfpathlineto{\pgfqpoint{5.808720in}{1.750000in}}%
\pgfpathlineto{\pgfqpoint{5.809960in}{1.680000in}}%
\pgfpathlineto{\pgfqpoint{5.811200in}{1.960000in}}%
\pgfpathlineto{\pgfqpoint{5.812440in}{1.540000in}}%
\pgfpathlineto{\pgfqpoint{5.813680in}{1.855000in}}%
\pgfpathlineto{\pgfqpoint{5.814920in}{1.785000in}}%
\pgfpathlineto{\pgfqpoint{5.816160in}{1.400000in}}%
\pgfpathlineto{\pgfqpoint{5.817400in}{1.645000in}}%
\pgfpathlineto{\pgfqpoint{5.818640in}{1.610000in}}%
\pgfpathlineto{\pgfqpoint{5.819880in}{1.925000in}}%
\pgfpathlineto{\pgfqpoint{5.822360in}{1.610000in}}%
\pgfpathlineto{\pgfqpoint{5.823600in}{1.680000in}}%
\pgfpathlineto{\pgfqpoint{5.824840in}{1.470000in}}%
\pgfpathlineto{\pgfqpoint{5.827320in}{2.100000in}}%
\pgfpathlineto{\pgfqpoint{5.828560in}{2.135000in}}%
\pgfpathlineto{\pgfqpoint{5.829800in}{1.925000in}}%
\pgfpathlineto{\pgfqpoint{5.831040in}{1.295000in}}%
\pgfpathlineto{\pgfqpoint{5.832280in}{1.995000in}}%
\pgfpathlineto{\pgfqpoint{5.833520in}{1.680000in}}%
\pgfpathlineto{\pgfqpoint{5.834760in}{1.750000in}}%
\pgfpathlineto{\pgfqpoint{5.836000in}{2.240000in}}%
\pgfpathlineto{\pgfqpoint{5.837240in}{2.135000in}}%
\pgfpathlineto{\pgfqpoint{5.838480in}{1.925000in}}%
\pgfpathlineto{\pgfqpoint{5.839720in}{2.415000in}}%
\pgfpathlineto{\pgfqpoint{5.842200in}{1.680000in}}%
\pgfpathlineto{\pgfqpoint{5.843440in}{1.610000in}}%
\pgfpathlineto{\pgfqpoint{5.844680in}{2.030000in}}%
\pgfpathlineto{\pgfqpoint{5.845920in}{1.645000in}}%
\pgfpathlineto{\pgfqpoint{5.847160in}{1.750000in}}%
\pgfpathlineto{\pgfqpoint{5.848400in}{1.715000in}}%
\pgfpathlineto{\pgfqpoint{5.849640in}{2.205000in}}%
\pgfpathlineto{\pgfqpoint{5.850880in}{2.240000in}}%
\pgfpathlineto{\pgfqpoint{5.852120in}{2.135000in}}%
\pgfpathlineto{\pgfqpoint{5.854600in}{1.645000in}}%
\pgfpathlineto{\pgfqpoint{5.855840in}{2.065000in}}%
\pgfpathlineto{\pgfqpoint{5.857080in}{1.855000in}}%
\pgfpathlineto{\pgfqpoint{5.858320in}{1.890000in}}%
\pgfpathlineto{\pgfqpoint{5.859560in}{1.960000in}}%
\pgfpathlineto{\pgfqpoint{5.860800in}{1.925000in}}%
\pgfpathlineto{\pgfqpoint{5.862040in}{1.960000in}}%
\pgfpathlineto{\pgfqpoint{5.863280in}{1.785000in}}%
\pgfpathlineto{\pgfqpoint{5.864520in}{1.960000in}}%
\pgfpathlineto{\pgfqpoint{5.865760in}{1.750000in}}%
\pgfpathlineto{\pgfqpoint{5.867000in}{1.750000in}}%
\pgfpathlineto{\pgfqpoint{5.868240in}{1.470000in}}%
\pgfpathlineto{\pgfqpoint{5.870720in}{2.100000in}}%
\pgfpathlineto{\pgfqpoint{5.873200in}{1.715000in}}%
\pgfpathlineto{\pgfqpoint{5.875680in}{2.135000in}}%
\pgfpathlineto{\pgfqpoint{5.876920in}{1.645000in}}%
\pgfpathlineto{\pgfqpoint{5.878160in}{1.890000in}}%
\pgfpathlineto{\pgfqpoint{5.879400in}{1.855000in}}%
\pgfpathlineto{\pgfqpoint{5.880640in}{2.240000in}}%
\pgfpathlineto{\pgfqpoint{5.881880in}{1.995000in}}%
\pgfpathlineto{\pgfqpoint{5.883120in}{2.030000in}}%
\pgfpathlineto{\pgfqpoint{5.884360in}{1.995000in}}%
\pgfpathlineto{\pgfqpoint{5.885600in}{2.065000in}}%
\pgfpathlineto{\pgfqpoint{5.886840in}{1.715000in}}%
\pgfpathlineto{\pgfqpoint{5.888080in}{2.065000in}}%
\pgfpathlineto{\pgfqpoint{5.889320in}{2.030000in}}%
\pgfpathlineto{\pgfqpoint{5.891800in}{1.645000in}}%
\pgfpathlineto{\pgfqpoint{5.894280in}{2.205000in}}%
\pgfpathlineto{\pgfqpoint{5.895520in}{1.715000in}}%
\pgfpathlineto{\pgfqpoint{5.896760in}{2.065000in}}%
\pgfpathlineto{\pgfqpoint{5.898000in}{2.065000in}}%
\pgfpathlineto{\pgfqpoint{5.899240in}{1.715000in}}%
\pgfpathlineto{\pgfqpoint{5.900480in}{1.750000in}}%
\pgfpathlineto{\pgfqpoint{5.901720in}{1.365000in}}%
\pgfpathlineto{\pgfqpoint{5.905440in}{2.100000in}}%
\pgfpathlineto{\pgfqpoint{5.906680in}{1.855000in}}%
\pgfpathlineto{\pgfqpoint{5.907920in}{2.065000in}}%
\pgfpathlineto{\pgfqpoint{5.909160in}{1.925000in}}%
\pgfpathlineto{\pgfqpoint{5.910400in}{2.135000in}}%
\pgfpathlineto{\pgfqpoint{5.912880in}{1.715000in}}%
\pgfpathlineto{\pgfqpoint{5.914120in}{1.785000in}}%
\pgfpathlineto{\pgfqpoint{5.916600in}{2.170000in}}%
\pgfpathlineto{\pgfqpoint{5.919080in}{1.750000in}}%
\pgfpathlineto{\pgfqpoint{5.920320in}{2.170000in}}%
\pgfpathlineto{\pgfqpoint{5.921560in}{1.750000in}}%
\pgfpathlineto{\pgfqpoint{5.924040in}{2.100000in}}%
\pgfpathlineto{\pgfqpoint{5.926520in}{1.505000in}}%
\pgfpathlineto{\pgfqpoint{5.927760in}{1.610000in}}%
\pgfpathlineto{\pgfqpoint{5.929000in}{2.030000in}}%
\pgfpathlineto{\pgfqpoint{5.930240in}{2.065000in}}%
\pgfpathlineto{\pgfqpoint{5.931480in}{2.170000in}}%
\pgfpathlineto{\pgfqpoint{5.933960in}{1.680000in}}%
\pgfpathlineto{\pgfqpoint{5.935200in}{1.715000in}}%
\pgfpathlineto{\pgfqpoint{5.936440in}{2.065000in}}%
\pgfpathlineto{\pgfqpoint{5.937680in}{1.890000in}}%
\pgfpathlineto{\pgfqpoint{5.938920in}{1.925000in}}%
\pgfpathlineto{\pgfqpoint{5.940160in}{2.030000in}}%
\pgfpathlineto{\pgfqpoint{5.941400in}{1.995000in}}%
\pgfpathlineto{\pgfqpoint{5.942640in}{2.065000in}}%
\pgfpathlineto{\pgfqpoint{5.943880in}{1.785000in}}%
\pgfpathlineto{\pgfqpoint{5.945120in}{2.135000in}}%
\pgfpathlineto{\pgfqpoint{5.946360in}{1.645000in}}%
\pgfpathlineto{\pgfqpoint{5.951320in}{2.205000in}}%
\pgfpathlineto{\pgfqpoint{5.953800in}{1.610000in}}%
\pgfpathlineto{\pgfqpoint{5.955040in}{1.400000in}}%
\pgfpathlineto{\pgfqpoint{5.956280in}{1.645000in}}%
\pgfpathlineto{\pgfqpoint{5.957520in}{1.190000in}}%
\pgfpathlineto{\pgfqpoint{5.958760in}{1.610000in}}%
\pgfpathlineto{\pgfqpoint{5.960000in}{1.365000in}}%
\pgfpathlineto{\pgfqpoint{5.962480in}{2.135000in}}%
\pgfpathlineto{\pgfqpoint{5.963720in}{1.855000in}}%
\pgfpathlineto{\pgfqpoint{5.964960in}{1.995000in}}%
\pgfpathlineto{\pgfqpoint{5.966200in}{1.645000in}}%
\pgfpathlineto{\pgfqpoint{5.967440in}{1.785000in}}%
\pgfpathlineto{\pgfqpoint{5.968680in}{1.435000in}}%
\pgfpathlineto{\pgfqpoint{5.969920in}{1.470000in}}%
\pgfpathlineto{\pgfqpoint{5.972400in}{1.925000in}}%
\pgfpathlineto{\pgfqpoint{5.973640in}{1.435000in}}%
\pgfpathlineto{\pgfqpoint{5.974880in}{1.960000in}}%
\pgfpathlineto{\pgfqpoint{5.977360in}{1.470000in}}%
\pgfpathlineto{\pgfqpoint{5.978600in}{2.100000in}}%
\pgfpathlineto{\pgfqpoint{5.981080in}{1.925000in}}%
\pgfpathlineto{\pgfqpoint{5.982320in}{1.680000in}}%
\pgfpathlineto{\pgfqpoint{5.984800in}{1.925000in}}%
\pgfpathlineto{\pgfqpoint{5.986040in}{1.505000in}}%
\pgfpathlineto{\pgfqpoint{5.987280in}{1.540000in}}%
\pgfpathlineto{\pgfqpoint{5.988520in}{2.135000in}}%
\pgfpathlineto{\pgfqpoint{5.991000in}{1.575000in}}%
\pgfpathlineto{\pgfqpoint{5.992240in}{1.925000in}}%
\pgfpathlineto{\pgfqpoint{5.993480in}{1.435000in}}%
\pgfpathlineto{\pgfqpoint{5.995960in}{1.995000in}}%
\pgfpathlineto{\pgfqpoint{5.997200in}{1.925000in}}%
\pgfpathlineto{\pgfqpoint{5.998440in}{1.785000in}}%
\pgfpathlineto{\pgfqpoint{5.999680in}{2.030000in}}%
\pgfpathlineto{\pgfqpoint{6.000920in}{2.590000in}}%
\pgfpathlineto{\pgfqpoint{6.002160in}{1.645000in}}%
\pgfpathlineto{\pgfqpoint{6.003400in}{2.065000in}}%
\pgfpathlineto{\pgfqpoint{6.004640in}{1.995000in}}%
\pgfpathlineto{\pgfqpoint{6.005880in}{1.855000in}}%
\pgfpathlineto{\pgfqpoint{6.007120in}{2.065000in}}%
\pgfpathlineto{\pgfqpoint{6.009600in}{1.820000in}}%
\pgfpathlineto{\pgfqpoint{6.010840in}{1.505000in}}%
\pgfpathlineto{\pgfqpoint{6.012080in}{2.100000in}}%
\pgfpathlineto{\pgfqpoint{6.014560in}{1.855000in}}%
\pgfpathlineto{\pgfqpoint{6.015800in}{2.100000in}}%
\pgfpathlineto{\pgfqpoint{6.017040in}{1.750000in}}%
\pgfpathlineto{\pgfqpoint{6.018280in}{1.820000in}}%
\pgfpathlineto{\pgfqpoint{6.019520in}{2.240000in}}%
\pgfpathlineto{\pgfqpoint{6.020760in}{2.170000in}}%
\pgfpathlineto{\pgfqpoint{6.022000in}{2.030000in}}%
\pgfpathlineto{\pgfqpoint{6.023240in}{1.610000in}}%
\pgfpathlineto{\pgfqpoint{6.024480in}{1.960000in}}%
\pgfpathlineto{\pgfqpoint{6.025720in}{1.680000in}}%
\pgfpathlineto{\pgfqpoint{6.026960in}{1.750000in}}%
\pgfpathlineto{\pgfqpoint{6.029440in}{2.135000in}}%
\pgfpathlineto{\pgfqpoint{6.030680in}{2.240000in}}%
\pgfpathlineto{\pgfqpoint{6.031920in}{1.890000in}}%
\pgfpathlineto{\pgfqpoint{6.033160in}{1.960000in}}%
\pgfpathlineto{\pgfqpoint{6.035640in}{1.505000in}}%
\pgfpathlineto{\pgfqpoint{6.039360in}{2.170000in}}%
\pgfpathlineto{\pgfqpoint{6.040600in}{1.890000in}}%
\pgfpathlineto{\pgfqpoint{6.041840in}{2.030000in}}%
\pgfpathlineto{\pgfqpoint{6.043080in}{1.680000in}}%
\pgfpathlineto{\pgfqpoint{6.045560in}{2.100000in}}%
\pgfpathlineto{\pgfqpoint{6.046800in}{1.890000in}}%
\pgfpathlineto{\pgfqpoint{6.048040in}{2.380000in}}%
\pgfpathlineto{\pgfqpoint{6.049280in}{2.135000in}}%
\pgfpathlineto{\pgfqpoint{6.050520in}{2.170000in}}%
\pgfpathlineto{\pgfqpoint{6.053000in}{1.995000in}}%
\pgfpathlineto{\pgfqpoint{6.054240in}{1.680000in}}%
\pgfpathlineto{\pgfqpoint{6.055480in}{1.820000in}}%
\pgfpathlineto{\pgfqpoint{6.056720in}{1.820000in}}%
\pgfpathlineto{\pgfqpoint{6.057960in}{2.135000in}}%
\pgfpathlineto{\pgfqpoint{6.059200in}{2.135000in}}%
\pgfpathlineto{\pgfqpoint{6.060440in}{2.065000in}}%
\pgfpathlineto{\pgfqpoint{6.061680in}{1.890000in}}%
\pgfpathlineto{\pgfqpoint{6.064160in}{2.030000in}}%
\pgfpathlineto{\pgfqpoint{6.065400in}{1.960000in}}%
\pgfpathlineto{\pgfqpoint{6.066640in}{1.540000in}}%
\pgfpathlineto{\pgfqpoint{6.067880in}{1.645000in}}%
\pgfpathlineto{\pgfqpoint{6.069120in}{1.435000in}}%
\pgfpathlineto{\pgfqpoint{6.070360in}{1.680000in}}%
\pgfpathlineto{\pgfqpoint{6.071600in}{1.575000in}}%
\pgfpathlineto{\pgfqpoint{6.072840in}{2.240000in}}%
\pgfpathlineto{\pgfqpoint{6.075320in}{1.540000in}}%
\pgfpathlineto{\pgfqpoint{6.076560in}{1.925000in}}%
\pgfpathlineto{\pgfqpoint{6.077800in}{1.400000in}}%
\pgfpathlineto{\pgfqpoint{6.079040in}{1.890000in}}%
\pgfpathlineto{\pgfqpoint{6.082760in}{1.365000in}}%
\pgfpathlineto{\pgfqpoint{6.085240in}{1.925000in}}%
\pgfpathlineto{\pgfqpoint{6.086480in}{1.680000in}}%
\pgfpathlineto{\pgfqpoint{6.087720in}{1.680000in}}%
\pgfpathlineto{\pgfqpoint{6.088960in}{1.785000in}}%
\pgfpathlineto{\pgfqpoint{6.090200in}{1.750000in}}%
\pgfpathlineto{\pgfqpoint{6.091440in}{1.680000in}}%
\pgfpathlineto{\pgfqpoint{6.092680in}{1.785000in}}%
\pgfpathlineto{\pgfqpoint{6.093920in}{2.240000in}}%
\pgfpathlineto{\pgfqpoint{6.095160in}{1.855000in}}%
\pgfpathlineto{\pgfqpoint{6.096400in}{1.890000in}}%
\pgfpathlineto{\pgfqpoint{6.097640in}{1.610000in}}%
\pgfpathlineto{\pgfqpoint{6.098880in}{1.750000in}}%
\pgfpathlineto{\pgfqpoint{6.100120in}{1.680000in}}%
\pgfpathlineto{\pgfqpoint{6.102600in}{1.960000in}}%
\pgfpathlineto{\pgfqpoint{6.103840in}{1.855000in}}%
\pgfpathlineto{\pgfqpoint{6.105080in}{1.960000in}}%
\pgfpathlineto{\pgfqpoint{6.106320in}{1.960000in}}%
\pgfpathlineto{\pgfqpoint{6.107560in}{2.135000in}}%
\pgfpathlineto{\pgfqpoint{6.108800in}{1.855000in}}%
\pgfpathlineto{\pgfqpoint{6.110040in}{2.170000in}}%
\pgfpathlineto{\pgfqpoint{6.112520in}{1.785000in}}%
\pgfpathlineto{\pgfqpoint{6.113760in}{2.030000in}}%
\pgfpathlineto{\pgfqpoint{6.115000in}{1.925000in}}%
\pgfpathlineto{\pgfqpoint{6.116240in}{1.995000in}}%
\pgfpathlineto{\pgfqpoint{6.117480in}{1.785000in}}%
\pgfpathlineto{\pgfqpoint{6.118720in}{1.925000in}}%
\pgfpathlineto{\pgfqpoint{6.119960in}{2.345000in}}%
\pgfpathlineto{\pgfqpoint{6.121200in}{1.680000in}}%
\pgfpathlineto{\pgfqpoint{6.123680in}{2.030000in}}%
\pgfpathlineto{\pgfqpoint{6.124920in}{2.030000in}}%
\pgfpathlineto{\pgfqpoint{6.126160in}{1.995000in}}%
\pgfpathlineto{\pgfqpoint{6.127400in}{1.715000in}}%
\pgfpathlineto{\pgfqpoint{6.128640in}{1.785000in}}%
\pgfpathlineto{\pgfqpoint{6.129880in}{2.030000in}}%
\pgfpathlineto{\pgfqpoint{6.131120in}{1.960000in}}%
\pgfpathlineto{\pgfqpoint{6.132360in}{1.995000in}}%
\pgfpathlineto{\pgfqpoint{6.133600in}{1.995000in}}%
\pgfpathlineto{\pgfqpoint{6.134840in}{1.855000in}}%
\pgfpathlineto{\pgfqpoint{6.136080in}{2.135000in}}%
\pgfpathlineto{\pgfqpoint{6.137320in}{1.470000in}}%
\pgfpathlineto{\pgfqpoint{6.138560in}{1.750000in}}%
\pgfpathlineto{\pgfqpoint{6.139800in}{1.680000in}}%
\pgfpathlineto{\pgfqpoint{6.143520in}{2.240000in}}%
\pgfpathlineto{\pgfqpoint{6.144760in}{2.135000in}}%
\pgfpathlineto{\pgfqpoint{6.146000in}{2.205000in}}%
\pgfpathlineto{\pgfqpoint{6.147240in}{1.820000in}}%
\pgfpathlineto{\pgfqpoint{6.149720in}{2.170000in}}%
\pgfpathlineto{\pgfqpoint{6.150960in}{2.135000in}}%
\pgfpathlineto{\pgfqpoint{6.152200in}{2.205000in}}%
\pgfpathlineto{\pgfqpoint{6.153440in}{2.170000in}}%
\pgfpathlineto{\pgfqpoint{6.154680in}{2.170000in}}%
\pgfpathlineto{\pgfqpoint{6.155920in}{2.135000in}}%
\pgfpathlineto{\pgfqpoint{6.157160in}{2.170000in}}%
\pgfpathlineto{\pgfqpoint{6.158400in}{1.995000in}}%
\pgfpathlineto{\pgfqpoint{6.159640in}{2.205000in}}%
\pgfpathlineto{\pgfqpoint{6.160880in}{1.645000in}}%
\pgfpathlineto{\pgfqpoint{6.163360in}{2.310000in}}%
\pgfpathlineto{\pgfqpoint{6.164600in}{1.995000in}}%
\pgfpathlineto{\pgfqpoint{6.165840in}{2.310000in}}%
\pgfpathlineto{\pgfqpoint{6.167080in}{1.785000in}}%
\pgfpathlineto{\pgfqpoint{6.168320in}{1.855000in}}%
\pgfpathlineto{\pgfqpoint{6.169560in}{1.820000in}}%
\pgfpathlineto{\pgfqpoint{6.170800in}{2.065000in}}%
\pgfpathlineto{\pgfqpoint{6.172040in}{2.030000in}}%
\pgfpathlineto{\pgfqpoint{6.173280in}{2.135000in}}%
\pgfpathlineto{\pgfqpoint{6.174520in}{1.820000in}}%
\pgfpathlineto{\pgfqpoint{6.175760in}{2.030000in}}%
\pgfpathlineto{\pgfqpoint{6.177000in}{1.855000in}}%
\pgfpathlineto{\pgfqpoint{6.179480in}{2.345000in}}%
\pgfpathlineto{\pgfqpoint{6.180720in}{1.890000in}}%
\pgfpathlineto{\pgfqpoint{6.181960in}{1.960000in}}%
\pgfpathlineto{\pgfqpoint{6.183200in}{1.820000in}}%
\pgfpathlineto{\pgfqpoint{6.184440in}{1.995000in}}%
\pgfpathlineto{\pgfqpoint{6.185680in}{1.890000in}}%
\pgfpathlineto{\pgfqpoint{6.186920in}{1.925000in}}%
\pgfpathlineto{\pgfqpoint{6.188160in}{1.505000in}}%
\pgfpathlineto{\pgfqpoint{6.189400in}{1.715000in}}%
\pgfpathlineto{\pgfqpoint{6.190640in}{1.540000in}}%
\pgfpathlineto{\pgfqpoint{6.195600in}{1.960000in}}%
\pgfpathlineto{\pgfqpoint{6.196840in}{1.750000in}}%
\pgfpathlineto{\pgfqpoint{6.198080in}{2.205000in}}%
\pgfpathlineto{\pgfqpoint{6.200560in}{1.680000in}}%
\pgfpathlineto{\pgfqpoint{6.201800in}{2.030000in}}%
\pgfpathlineto{\pgfqpoint{6.204280in}{1.575000in}}%
\pgfpathlineto{\pgfqpoint{6.206760in}{2.065000in}}%
\pgfpathlineto{\pgfqpoint{6.208000in}{2.100000in}}%
\pgfpathlineto{\pgfqpoint{6.210480in}{1.785000in}}%
\pgfpathlineto{\pgfqpoint{6.211720in}{1.680000in}}%
\pgfpathlineto{\pgfqpoint{6.212960in}{2.030000in}}%
\pgfpathlineto{\pgfqpoint{6.215440in}{1.435000in}}%
\pgfpathlineto{\pgfqpoint{6.217920in}{1.925000in}}%
\pgfpathlineto{\pgfqpoint{6.219160in}{1.225000in}}%
\pgfpathlineto{\pgfqpoint{6.222880in}{2.065000in}}%
\pgfpathlineto{\pgfqpoint{6.224120in}{1.785000in}}%
\pgfpathlineto{\pgfqpoint{6.225360in}{1.925000in}}%
\pgfpathlineto{\pgfqpoint{6.226600in}{1.575000in}}%
\pgfpathlineto{\pgfqpoint{6.229080in}{1.960000in}}%
\pgfpathlineto{\pgfqpoint{6.231560in}{1.680000in}}%
\pgfpathlineto{\pgfqpoint{6.234040in}{2.100000in}}%
\pgfpathlineto{\pgfqpoint{6.235280in}{1.855000in}}%
\pgfpathlineto{\pgfqpoint{6.236520in}{2.100000in}}%
\pgfpathlineto{\pgfqpoint{6.237760in}{1.995000in}}%
\pgfpathlineto{\pgfqpoint{6.239000in}{2.100000in}}%
\pgfpathlineto{\pgfqpoint{6.240240in}{1.890000in}}%
\pgfpathlineto{\pgfqpoint{6.241480in}{1.435000in}}%
\pgfpathlineto{\pgfqpoint{6.247680in}{1.925000in}}%
\pgfpathlineto{\pgfqpoint{6.248920in}{1.680000in}}%
\pgfpathlineto{\pgfqpoint{6.251400in}{2.100000in}}%
\pgfpathlineto{\pgfqpoint{6.252640in}{1.680000in}}%
\pgfpathlineto{\pgfqpoint{6.256360in}{2.485000in}}%
\pgfpathlineto{\pgfqpoint{6.257600in}{2.170000in}}%
\pgfpathlineto{\pgfqpoint{6.258840in}{2.170000in}}%
\pgfpathlineto{\pgfqpoint{6.261320in}{1.995000in}}%
\pgfpathlineto{\pgfqpoint{6.262560in}{2.030000in}}%
\pgfpathlineto{\pgfqpoint{6.263800in}{1.820000in}}%
\pgfpathlineto{\pgfqpoint{6.265040in}{2.030000in}}%
\pgfpathlineto{\pgfqpoint{6.266280in}{1.785000in}}%
\pgfpathlineto{\pgfqpoint{6.267520in}{1.890000in}}%
\pgfpathlineto{\pgfqpoint{6.268760in}{2.135000in}}%
\pgfpathlineto{\pgfqpoint{6.270000in}{1.890000in}}%
\pgfpathlineto{\pgfqpoint{6.271240in}{2.065000in}}%
\pgfpathlineto{\pgfqpoint{6.272480in}{1.435000in}}%
\pgfpathlineto{\pgfqpoint{6.273720in}{2.100000in}}%
\pgfpathlineto{\pgfqpoint{6.276200in}{1.785000in}}%
\pgfpathlineto{\pgfqpoint{6.277440in}{1.995000in}}%
\pgfpathlineto{\pgfqpoint{6.278680in}{1.505000in}}%
\pgfpathlineto{\pgfqpoint{6.281160in}{2.030000in}}%
\pgfpathlineto{\pgfqpoint{6.282400in}{1.505000in}}%
\pgfpathlineto{\pgfqpoint{6.283640in}{2.065000in}}%
\pgfpathlineto{\pgfqpoint{6.284880in}{1.750000in}}%
\pgfpathlineto{\pgfqpoint{6.286120in}{2.065000in}}%
\pgfpathlineto{\pgfqpoint{6.287360in}{2.065000in}}%
\pgfpathlineto{\pgfqpoint{6.288600in}{2.450000in}}%
\pgfpathlineto{\pgfqpoint{6.291080in}{1.855000in}}%
\pgfpathlineto{\pgfqpoint{6.292320in}{1.890000in}}%
\pgfpathlineto{\pgfqpoint{6.293560in}{2.030000in}}%
\pgfpathlineto{\pgfqpoint{6.296040in}{1.855000in}}%
\pgfpathlineto{\pgfqpoint{6.297280in}{1.750000in}}%
\pgfpathlineto{\pgfqpoint{6.298520in}{2.030000in}}%
\pgfpathlineto{\pgfqpoint{6.299760in}{1.575000in}}%
\pgfpathlineto{\pgfqpoint{6.301000in}{1.855000in}}%
\pgfpathlineto{\pgfqpoint{6.302240in}{1.855000in}}%
\pgfpathlineto{\pgfqpoint{6.304720in}{1.680000in}}%
\pgfpathlineto{\pgfqpoint{6.305960in}{1.680000in}}%
\pgfpathlineto{\pgfqpoint{6.307200in}{1.435000in}}%
\pgfpathlineto{\pgfqpoint{6.308440in}{1.680000in}}%
\pgfpathlineto{\pgfqpoint{6.309680in}{1.540000in}}%
\pgfpathlineto{\pgfqpoint{6.312160in}{1.750000in}}%
\pgfpathlineto{\pgfqpoint{6.313400in}{1.750000in}}%
\pgfpathlineto{\pgfqpoint{6.314640in}{2.135000in}}%
\pgfpathlineto{\pgfqpoint{6.315880in}{1.680000in}}%
\pgfpathlineto{\pgfqpoint{6.318360in}{1.680000in}}%
\pgfpathlineto{\pgfqpoint{6.319600in}{2.030000in}}%
\pgfpathlineto{\pgfqpoint{6.320840in}{1.995000in}}%
\pgfpathlineto{\pgfqpoint{6.322080in}{1.995000in}}%
\pgfpathlineto{\pgfqpoint{6.323320in}{2.275000in}}%
\pgfpathlineto{\pgfqpoint{6.324560in}{1.680000in}}%
\pgfpathlineto{\pgfqpoint{6.325800in}{1.680000in}}%
\pgfpathlineto{\pgfqpoint{6.327040in}{1.890000in}}%
\pgfpathlineto{\pgfqpoint{6.328280in}{1.785000in}}%
\pgfpathlineto{\pgfqpoint{6.329520in}{1.470000in}}%
\pgfpathlineto{\pgfqpoint{6.330760in}{1.750000in}}%
\pgfpathlineto{\pgfqpoint{6.332000in}{1.680000in}}%
\pgfpathlineto{\pgfqpoint{6.333240in}{2.135000in}}%
\pgfpathlineto{\pgfqpoint{6.334480in}{1.960000in}}%
\pgfpathlineto{\pgfqpoint{6.335720in}{2.170000in}}%
\pgfpathlineto{\pgfqpoint{6.336960in}{2.065000in}}%
\pgfpathlineto{\pgfqpoint{6.338200in}{1.645000in}}%
\pgfpathlineto{\pgfqpoint{6.339440in}{2.170000in}}%
\pgfpathlineto{\pgfqpoint{6.340680in}{1.505000in}}%
\pgfpathlineto{\pgfqpoint{6.341920in}{1.960000in}}%
\pgfpathlineto{\pgfqpoint{6.344400in}{1.680000in}}%
\pgfpathlineto{\pgfqpoint{6.345640in}{1.890000in}}%
\pgfpathlineto{\pgfqpoint{6.346880in}{2.310000in}}%
\pgfpathlineto{\pgfqpoint{6.349360in}{2.030000in}}%
\pgfpathlineto{\pgfqpoint{6.353080in}{1.645000in}}%
\pgfpathlineto{\pgfqpoint{6.354320in}{1.890000in}}%
\pgfpathlineto{\pgfqpoint{6.355560in}{1.855000in}}%
\pgfpathlineto{\pgfqpoint{6.356800in}{1.645000in}}%
\pgfpathlineto{\pgfqpoint{6.358040in}{2.065000in}}%
\pgfpathlineto{\pgfqpoint{6.359280in}{1.645000in}}%
\pgfpathlineto{\pgfqpoint{6.363000in}{2.135000in}}%
\pgfpathlineto{\pgfqpoint{6.364240in}{1.995000in}}%
\pgfpathlineto{\pgfqpoint{6.365480in}{2.030000in}}%
\pgfpathlineto{\pgfqpoint{6.366720in}{1.960000in}}%
\pgfpathlineto{\pgfqpoint{6.367960in}{2.065000in}}%
\pgfpathlineto{\pgfqpoint{6.369200in}{1.925000in}}%
\pgfpathlineto{\pgfqpoint{6.370440in}{2.135000in}}%
\pgfpathlineto{\pgfqpoint{6.371680in}{2.065000in}}%
\pgfpathlineto{\pgfqpoint{6.372920in}{2.170000in}}%
\pgfpathlineto{\pgfqpoint{6.374160in}{2.065000in}}%
\pgfpathlineto{\pgfqpoint{6.376640in}{1.540000in}}%
\pgfpathlineto{\pgfqpoint{6.377880in}{1.925000in}}%
\pgfpathlineto{\pgfqpoint{6.379120in}{1.575000in}}%
\pgfpathlineto{\pgfqpoint{6.381600in}{1.925000in}}%
\pgfpathlineto{\pgfqpoint{6.382840in}{1.890000in}}%
\pgfpathlineto{\pgfqpoint{6.384080in}{1.540000in}}%
\pgfpathlineto{\pgfqpoint{6.387800in}{1.925000in}}%
\pgfpathlineto{\pgfqpoint{6.389040in}{1.925000in}}%
\pgfpathlineto{\pgfqpoint{6.390280in}{2.170000in}}%
\pgfpathlineto{\pgfqpoint{6.392760in}{1.680000in}}%
\pgfpathlineto{\pgfqpoint{6.395240in}{1.855000in}}%
\pgfpathlineto{\pgfqpoint{6.396480in}{1.715000in}}%
\pgfpathlineto{\pgfqpoint{6.397720in}{1.890000in}}%
\pgfpathlineto{\pgfqpoint{6.398960in}{1.680000in}}%
\pgfpathlineto{\pgfqpoint{6.400200in}{1.925000in}}%
\pgfpathlineto{\pgfqpoint{6.401440in}{1.715000in}}%
\pgfpathlineto{\pgfqpoint{6.402680in}{2.170000in}}%
\pgfpathlineto{\pgfqpoint{6.403920in}{1.295000in}}%
\pgfpathlineto{\pgfqpoint{6.405160in}{1.855000in}}%
\pgfpathlineto{\pgfqpoint{6.407640in}{1.680000in}}%
\pgfpathlineto{\pgfqpoint{6.408880in}{1.715000in}}%
\pgfpathlineto{\pgfqpoint{6.410120in}{1.925000in}}%
\pgfpathlineto{\pgfqpoint{6.411360in}{1.190000in}}%
\pgfpathlineto{\pgfqpoint{6.413840in}{1.890000in}}%
\pgfpathlineto{\pgfqpoint{6.415080in}{1.890000in}}%
\pgfpathlineto{\pgfqpoint{6.416320in}{1.680000in}}%
\pgfpathlineto{\pgfqpoint{6.417560in}{1.715000in}}%
\pgfpathlineto{\pgfqpoint{6.420040in}{1.330000in}}%
\pgfpathlineto{\pgfqpoint{6.423760in}{2.135000in}}%
\pgfpathlineto{\pgfqpoint{6.426240in}{1.785000in}}%
\pgfpathlineto{\pgfqpoint{6.427480in}{2.415000in}}%
\pgfpathlineto{\pgfqpoint{6.428720in}{1.785000in}}%
\pgfpathlineto{\pgfqpoint{6.429960in}{1.855000in}}%
\pgfpathlineto{\pgfqpoint{6.431200in}{1.750000in}}%
\pgfpathlineto{\pgfqpoint{6.432440in}{2.065000in}}%
\pgfpathlineto{\pgfqpoint{6.433680in}{1.890000in}}%
\pgfpathlineto{\pgfqpoint{6.434920in}{2.065000in}}%
\pgfpathlineto{\pgfqpoint{6.436160in}{2.730000in}}%
\pgfpathlineto{\pgfqpoint{6.437400in}{2.730000in}}%
\pgfpathlineto{\pgfqpoint{6.439880in}{2.135000in}}%
\pgfpathlineto{\pgfqpoint{6.441120in}{2.170000in}}%
\pgfpathlineto{\pgfqpoint{6.442360in}{2.065000in}}%
\pgfpathlineto{\pgfqpoint{6.443600in}{2.065000in}}%
\pgfpathlineto{\pgfqpoint{6.444840in}{1.890000in}}%
\pgfpathlineto{\pgfqpoint{6.447320in}{2.205000in}}%
\pgfpathlineto{\pgfqpoint{6.448560in}{2.240000in}}%
\pgfpathlineto{\pgfqpoint{6.449800in}{2.205000in}}%
\pgfpathlineto{\pgfqpoint{6.451040in}{2.240000in}}%
\pgfpathlineto{\pgfqpoint{6.452280in}{2.380000in}}%
\pgfpathlineto{\pgfqpoint{6.453520in}{2.380000in}}%
\pgfpathlineto{\pgfqpoint{6.454760in}{1.925000in}}%
\pgfpathlineto{\pgfqpoint{6.456000in}{2.450000in}}%
\pgfpathlineto{\pgfqpoint{6.457240in}{1.715000in}}%
\pgfpathlineto{\pgfqpoint{6.458480in}{1.645000in}}%
\pgfpathlineto{\pgfqpoint{6.459720in}{2.030000in}}%
\pgfpathlineto{\pgfqpoint{6.462200in}{1.855000in}}%
\pgfpathlineto{\pgfqpoint{6.464680in}{2.030000in}}%
\pgfpathlineto{\pgfqpoint{6.465920in}{1.890000in}}%
\pgfpathlineto{\pgfqpoint{6.467160in}{1.925000in}}%
\pgfpathlineto{\pgfqpoint{6.468400in}{2.135000in}}%
\pgfpathlineto{\pgfqpoint{6.469640in}{1.715000in}}%
\pgfpathlineto{\pgfqpoint{6.470880in}{1.715000in}}%
\pgfpathlineto{\pgfqpoint{6.472120in}{1.995000in}}%
\pgfpathlineto{\pgfqpoint{6.473360in}{1.960000in}}%
\pgfpathlineto{\pgfqpoint{6.474600in}{2.240000in}}%
\pgfpathlineto{\pgfqpoint{6.475840in}{1.925000in}}%
\pgfpathlineto{\pgfqpoint{6.478320in}{2.205000in}}%
\pgfpathlineto{\pgfqpoint{6.479560in}{2.135000in}}%
\pgfpathlineto{\pgfqpoint{6.480800in}{2.310000in}}%
\pgfpathlineto{\pgfqpoint{6.482040in}{1.960000in}}%
\pgfpathlineto{\pgfqpoint{6.483280in}{2.030000in}}%
\pgfpathlineto{\pgfqpoint{6.484520in}{2.240000in}}%
\pgfpathlineto{\pgfqpoint{6.485760in}{2.100000in}}%
\pgfpathlineto{\pgfqpoint{6.487000in}{1.750000in}}%
\pgfpathlineto{\pgfqpoint{6.488240in}{2.100000in}}%
\pgfpathlineto{\pgfqpoint{6.489480in}{2.135000in}}%
\pgfpathlineto{\pgfqpoint{6.490720in}{2.240000in}}%
\pgfpathlineto{\pgfqpoint{6.493200in}{1.575000in}}%
\pgfpathlineto{\pgfqpoint{6.494440in}{1.715000in}}%
\pgfpathlineto{\pgfqpoint{6.495680in}{2.065000in}}%
\pgfpathlineto{\pgfqpoint{6.496920in}{1.645000in}}%
\pgfpathlineto{\pgfqpoint{6.498160in}{1.785000in}}%
\pgfpathlineto{\pgfqpoint{6.499400in}{1.715000in}}%
\pgfpathlineto{\pgfqpoint{6.500640in}{1.715000in}}%
\pgfpathlineto{\pgfqpoint{6.501880in}{1.785000in}}%
\pgfpathlineto{\pgfqpoint{6.503120in}{1.470000in}}%
\pgfpathlineto{\pgfqpoint{6.504360in}{1.540000in}}%
\pgfpathlineto{\pgfqpoint{6.505600in}{1.925000in}}%
\pgfpathlineto{\pgfqpoint{6.506840in}{1.610000in}}%
\pgfpathlineto{\pgfqpoint{6.509320in}{2.065000in}}%
\pgfpathlineto{\pgfqpoint{6.510560in}{2.065000in}}%
\pgfpathlineto{\pgfqpoint{6.511800in}{1.890000in}}%
\pgfpathlineto{\pgfqpoint{6.513040in}{1.995000in}}%
\pgfpathlineto{\pgfqpoint{6.515520in}{2.345000in}}%
\pgfpathlineto{\pgfqpoint{6.516760in}{1.890000in}}%
\pgfpathlineto{\pgfqpoint{6.518000in}{1.855000in}}%
\pgfpathlineto{\pgfqpoint{6.519240in}{1.260000in}}%
\pgfpathlineto{\pgfqpoint{6.521720in}{2.100000in}}%
\pgfpathlineto{\pgfqpoint{6.525440in}{1.855000in}}%
\pgfpathlineto{\pgfqpoint{6.526680in}{2.135000in}}%
\pgfpathlineto{\pgfqpoint{6.529160in}{1.960000in}}%
\pgfpathlineto{\pgfqpoint{6.530400in}{1.750000in}}%
\pgfpathlineto{\pgfqpoint{6.531640in}{1.960000in}}%
\pgfpathlineto{\pgfqpoint{6.532880in}{1.680000in}}%
\pgfpathlineto{\pgfqpoint{6.534120in}{1.680000in}}%
\pgfpathlineto{\pgfqpoint{6.536600in}{1.995000in}}%
\pgfpathlineto{\pgfqpoint{6.537840in}{1.855000in}}%
\pgfpathlineto{\pgfqpoint{6.539080in}{2.485000in}}%
\pgfpathlineto{\pgfqpoint{6.540320in}{2.240000in}}%
\pgfpathlineto{\pgfqpoint{6.541560in}{1.750000in}}%
\pgfpathlineto{\pgfqpoint{6.542800in}{1.890000in}}%
\pgfpathlineto{\pgfqpoint{6.544040in}{1.575000in}}%
\pgfpathlineto{\pgfqpoint{6.545280in}{2.415000in}}%
\pgfpathlineto{\pgfqpoint{6.546520in}{1.645000in}}%
\pgfpathlineto{\pgfqpoint{6.547760in}{2.030000in}}%
\pgfpathlineto{\pgfqpoint{6.549000in}{1.820000in}}%
\pgfpathlineto{\pgfqpoint{6.550240in}{2.065000in}}%
\pgfpathlineto{\pgfqpoint{6.551480in}{1.470000in}}%
\pgfpathlineto{\pgfqpoint{6.555200in}{2.065000in}}%
\pgfpathlineto{\pgfqpoint{6.556440in}{1.995000in}}%
\pgfpathlineto{\pgfqpoint{6.557680in}{1.680000in}}%
\pgfpathlineto{\pgfqpoint{6.562640in}{2.065000in}}%
\pgfpathlineto{\pgfqpoint{6.563880in}{2.100000in}}%
\pgfpathlineto{\pgfqpoint{6.568840in}{1.540000in}}%
\pgfpathlineto{\pgfqpoint{6.570080in}{1.820000in}}%
\pgfpathlineto{\pgfqpoint{6.571320in}{1.750000in}}%
\pgfpathlineto{\pgfqpoint{6.572560in}{1.260000in}}%
\pgfpathlineto{\pgfqpoint{6.573800in}{1.855000in}}%
\pgfpathlineto{\pgfqpoint{6.575040in}{1.855000in}}%
\pgfpathlineto{\pgfqpoint{6.576280in}{1.820000in}}%
\pgfpathlineto{\pgfqpoint{6.577520in}{1.120000in}}%
\pgfpathlineto{\pgfqpoint{6.578760in}{1.960000in}}%
\pgfpathlineto{\pgfqpoint{6.580000in}{1.610000in}}%
\pgfpathlineto{\pgfqpoint{6.581240in}{1.960000in}}%
\pgfpathlineto{\pgfqpoint{6.582480in}{1.925000in}}%
\pgfpathlineto{\pgfqpoint{6.583720in}{1.925000in}}%
\pgfpathlineto{\pgfqpoint{6.584960in}{2.450000in}}%
\pgfpathlineto{\pgfqpoint{6.586200in}{1.750000in}}%
\pgfpathlineto{\pgfqpoint{6.587440in}{1.890000in}}%
\pgfpathlineto{\pgfqpoint{6.589920in}{1.890000in}}%
\pgfpathlineto{\pgfqpoint{6.591160in}{1.715000in}}%
\pgfpathlineto{\pgfqpoint{6.592400in}{1.890000in}}%
\pgfpathlineto{\pgfqpoint{6.593640in}{1.855000in}}%
\pgfpathlineto{\pgfqpoint{6.596120in}{1.925000in}}%
\pgfpathlineto{\pgfqpoint{6.597360in}{1.750000in}}%
\pgfpathlineto{\pgfqpoint{6.598600in}{1.785000in}}%
\pgfpathlineto{\pgfqpoint{6.599840in}{2.135000in}}%
\pgfpathlineto{\pgfqpoint{6.602320in}{1.715000in}}%
\pgfpathlineto{\pgfqpoint{6.603560in}{1.925000in}}%
\pgfpathlineto{\pgfqpoint{6.604800in}{1.890000in}}%
\pgfpathlineto{\pgfqpoint{6.606040in}{1.925000in}}%
\pgfpathlineto{\pgfqpoint{6.607280in}{1.890000in}}%
\pgfpathlineto{\pgfqpoint{6.608520in}{1.890000in}}%
\pgfpathlineto{\pgfqpoint{6.609760in}{1.960000in}}%
\pgfpathlineto{\pgfqpoint{6.612240in}{1.960000in}}%
\pgfpathlineto{\pgfqpoint{6.613480in}{1.855000in}}%
\pgfpathlineto{\pgfqpoint{6.614720in}{2.275000in}}%
\pgfpathlineto{\pgfqpoint{6.615960in}{2.205000in}}%
\pgfpathlineto{\pgfqpoint{6.617200in}{1.575000in}}%
\pgfpathlineto{\pgfqpoint{6.619680in}{2.100000in}}%
\pgfpathlineto{\pgfqpoint{6.620920in}{1.575000in}}%
\pgfpathlineto{\pgfqpoint{6.622160in}{1.750000in}}%
\pgfpathlineto{\pgfqpoint{6.623400in}{2.100000in}}%
\pgfpathlineto{\pgfqpoint{6.625880in}{1.785000in}}%
\pgfpathlineto{\pgfqpoint{6.627120in}{1.995000in}}%
\pgfpathlineto{\pgfqpoint{6.628360in}{1.435000in}}%
\pgfpathlineto{\pgfqpoint{6.630840in}{2.205000in}}%
\pgfpathlineto{\pgfqpoint{6.632080in}{2.100000in}}%
\pgfpathlineto{\pgfqpoint{6.633320in}{2.170000in}}%
\pgfpathlineto{\pgfqpoint{6.634560in}{1.890000in}}%
\pgfpathlineto{\pgfqpoint{6.635800in}{1.925000in}}%
\pgfpathlineto{\pgfqpoint{6.637040in}{1.855000in}}%
\pgfpathlineto{\pgfqpoint{6.638280in}{1.960000in}}%
\pgfpathlineto{\pgfqpoint{6.639520in}{2.170000in}}%
\pgfpathlineto{\pgfqpoint{6.640760in}{1.995000in}}%
\pgfpathlineto{\pgfqpoint{6.642000in}{1.645000in}}%
\pgfpathlineto{\pgfqpoint{6.644480in}{1.995000in}}%
\pgfpathlineto{\pgfqpoint{6.645720in}{2.030000in}}%
\pgfpathlineto{\pgfqpoint{6.646960in}{1.960000in}}%
\pgfpathlineto{\pgfqpoint{6.648200in}{1.995000in}}%
\pgfpathlineto{\pgfqpoint{6.649440in}{1.995000in}}%
\pgfpathlineto{\pgfqpoint{6.650680in}{2.170000in}}%
\pgfpathlineto{\pgfqpoint{6.651920in}{1.995000in}}%
\pgfpathlineto{\pgfqpoint{6.653160in}{2.030000in}}%
\pgfpathlineto{\pgfqpoint{6.654400in}{1.610000in}}%
\pgfpathlineto{\pgfqpoint{6.656880in}{2.240000in}}%
\pgfpathlineto{\pgfqpoint{6.659360in}{1.995000in}}%
\pgfpathlineto{\pgfqpoint{6.660600in}{1.960000in}}%
\pgfpathlineto{\pgfqpoint{6.661840in}{2.240000in}}%
\pgfpathlineto{\pgfqpoint{6.663080in}{1.610000in}}%
\pgfpathlineto{\pgfqpoint{6.664320in}{1.960000in}}%
\pgfpathlineto{\pgfqpoint{6.665560in}{1.960000in}}%
\pgfpathlineto{\pgfqpoint{6.666800in}{1.890000in}}%
\pgfpathlineto{\pgfqpoint{6.668040in}{2.065000in}}%
\pgfpathlineto{\pgfqpoint{6.669280in}{2.065000in}}%
\pgfpathlineto{\pgfqpoint{6.670520in}{1.750000in}}%
\pgfpathlineto{\pgfqpoint{6.671760in}{1.890000in}}%
\pgfpathlineto{\pgfqpoint{6.673000in}{1.820000in}}%
\pgfpathlineto{\pgfqpoint{6.674240in}{1.855000in}}%
\pgfpathlineto{\pgfqpoint{6.675480in}{1.645000in}}%
\pgfpathlineto{\pgfqpoint{6.677960in}{2.100000in}}%
\pgfpathlineto{\pgfqpoint{6.679200in}{1.610000in}}%
\pgfpathlineto{\pgfqpoint{6.680440in}{1.610000in}}%
\pgfpathlineto{\pgfqpoint{6.681680in}{1.715000in}}%
\pgfpathlineto{\pgfqpoint{6.684160in}{1.610000in}}%
\pgfpathlineto{\pgfqpoint{6.686640in}{1.750000in}}%
\pgfpathlineto{\pgfqpoint{6.687880in}{1.470000in}}%
\pgfpathlineto{\pgfqpoint{6.690360in}{1.820000in}}%
\pgfpathlineto{\pgfqpoint{6.691600in}{1.785000in}}%
\pgfpathlineto{\pgfqpoint{6.692840in}{1.715000in}}%
\pgfpathlineto{\pgfqpoint{6.694080in}{1.890000in}}%
\pgfpathlineto{\pgfqpoint{6.695320in}{1.820000in}}%
\pgfpathlineto{\pgfqpoint{6.696560in}{1.855000in}}%
\pgfpathlineto{\pgfqpoint{6.697800in}{2.240000in}}%
\pgfpathlineto{\pgfqpoint{6.699040in}{1.995000in}}%
\pgfpathlineto{\pgfqpoint{6.700280in}{2.065000in}}%
\pgfpathlineto{\pgfqpoint{6.701520in}{2.065000in}}%
\pgfpathlineto{\pgfqpoint{6.702760in}{1.750000in}}%
\pgfpathlineto{\pgfqpoint{6.704000in}{2.065000in}}%
\pgfpathlineto{\pgfqpoint{6.705240in}{1.785000in}}%
\pgfpathlineto{\pgfqpoint{6.706480in}{2.205000in}}%
\pgfpathlineto{\pgfqpoint{6.707720in}{2.205000in}}%
\pgfpathlineto{\pgfqpoint{6.708960in}{1.890000in}}%
\pgfpathlineto{\pgfqpoint{6.710200in}{1.925000in}}%
\pgfpathlineto{\pgfqpoint{6.711440in}{2.065000in}}%
\pgfpathlineto{\pgfqpoint{6.712680in}{1.610000in}}%
\pgfpathlineto{\pgfqpoint{6.713920in}{1.855000in}}%
\pgfpathlineto{\pgfqpoint{6.715160in}{1.750000in}}%
\pgfpathlineto{\pgfqpoint{6.717640in}{1.995000in}}%
\pgfpathlineto{\pgfqpoint{6.718880in}{1.505000in}}%
\pgfpathlineto{\pgfqpoint{6.721360in}{2.100000in}}%
\pgfpathlineto{\pgfqpoint{6.722600in}{2.205000in}}%
\pgfpathlineto{\pgfqpoint{6.723840in}{1.785000in}}%
\pgfpathlineto{\pgfqpoint{6.726320in}{1.925000in}}%
\pgfpathlineto{\pgfqpoint{6.727560in}{1.855000in}}%
\pgfpathlineto{\pgfqpoint{6.728800in}{1.995000in}}%
\pgfpathlineto{\pgfqpoint{6.730040in}{1.855000in}}%
\pgfpathlineto{\pgfqpoint{6.731280in}{1.960000in}}%
\pgfpathlineto{\pgfqpoint{6.732520in}{1.925000in}}%
\pgfpathlineto{\pgfqpoint{6.733760in}{1.540000in}}%
\pgfpathlineto{\pgfqpoint{6.735000in}{1.820000in}}%
\pgfpathlineto{\pgfqpoint{6.736240in}{1.610000in}}%
\pgfpathlineto{\pgfqpoint{6.737480in}{1.645000in}}%
\pgfpathlineto{\pgfqpoint{6.738720in}{2.170000in}}%
\pgfpathlineto{\pgfqpoint{6.739960in}{2.100000in}}%
\pgfpathlineto{\pgfqpoint{6.741200in}{2.100000in}}%
\pgfpathlineto{\pgfqpoint{6.743680in}{1.575000in}}%
\pgfpathlineto{\pgfqpoint{6.744920in}{1.925000in}}%
\pgfpathlineto{\pgfqpoint{6.747400in}{1.645000in}}%
\pgfpathlineto{\pgfqpoint{6.748640in}{1.820000in}}%
\pgfpathlineto{\pgfqpoint{6.749880in}{1.610000in}}%
\pgfpathlineto{\pgfqpoint{6.751120in}{1.680000in}}%
\pgfpathlineto{\pgfqpoint{6.752360in}{1.575000in}}%
\pgfpathlineto{\pgfqpoint{6.753600in}{1.960000in}}%
\pgfpathlineto{\pgfqpoint{6.754840in}{1.995000in}}%
\pgfpathlineto{\pgfqpoint{6.756080in}{1.610000in}}%
\pgfpathlineto{\pgfqpoint{6.757320in}{2.065000in}}%
\pgfpathlineto{\pgfqpoint{6.758560in}{1.400000in}}%
\pgfpathlineto{\pgfqpoint{6.759800in}{1.645000in}}%
\pgfpathlineto{\pgfqpoint{6.761040in}{1.610000in}}%
\pgfpathlineto{\pgfqpoint{6.762280in}{1.610000in}}%
\pgfpathlineto{\pgfqpoint{6.763520in}{1.365000in}}%
\pgfpathlineto{\pgfqpoint{6.764760in}{1.960000in}}%
\pgfpathlineto{\pgfqpoint{6.766000in}{1.680000in}}%
\pgfpathlineto{\pgfqpoint{6.767240in}{1.890000in}}%
\pgfpathlineto{\pgfqpoint{6.768480in}{1.610000in}}%
\pgfpathlineto{\pgfqpoint{6.770960in}{1.855000in}}%
\pgfpathlineto{\pgfqpoint{6.772200in}{1.505000in}}%
\pgfpathlineto{\pgfqpoint{6.773440in}{1.610000in}}%
\pgfpathlineto{\pgfqpoint{6.774680in}{2.065000in}}%
\pgfpathlineto{\pgfqpoint{6.775920in}{1.645000in}}%
\pgfpathlineto{\pgfqpoint{6.777160in}{1.715000in}}%
\pgfpathlineto{\pgfqpoint{6.779640in}{2.065000in}}%
\pgfpathlineto{\pgfqpoint{6.780880in}{1.470000in}}%
\pgfpathlineto{\pgfqpoint{6.782120in}{1.680000in}}%
\pgfpathlineto{\pgfqpoint{6.783360in}{1.540000in}}%
\pgfpathlineto{\pgfqpoint{6.784600in}{1.610000in}}%
\pgfpathlineto{\pgfqpoint{6.785840in}{1.820000in}}%
\pgfpathlineto{\pgfqpoint{6.787080in}{1.540000in}}%
\pgfpathlineto{\pgfqpoint{6.788320in}{1.890000in}}%
\pgfpathlineto{\pgfqpoint{6.789560in}{1.715000in}}%
\pgfpathlineto{\pgfqpoint{6.790800in}{1.365000in}}%
\pgfpathlineto{\pgfqpoint{6.793280in}{1.995000in}}%
\pgfpathlineto{\pgfqpoint{6.794520in}{1.540000in}}%
\pgfpathlineto{\pgfqpoint{6.795760in}{2.345000in}}%
\pgfpathlineto{\pgfqpoint{6.797000in}{1.085000in}}%
\pgfpathlineto{\pgfqpoint{6.798240in}{1.260000in}}%
\pgfpathlineto{\pgfqpoint{6.799480in}{1.890000in}}%
\pgfpathlineto{\pgfqpoint{6.801960in}{1.715000in}}%
\pgfpathlineto{\pgfqpoint{6.803200in}{1.890000in}}%
\pgfpathlineto{\pgfqpoint{6.804440in}{1.820000in}}%
\pgfpathlineto{\pgfqpoint{6.805680in}{2.240000in}}%
\pgfpathlineto{\pgfqpoint{6.808160in}{1.715000in}}%
\pgfpathlineto{\pgfqpoint{6.809400in}{1.960000in}}%
\pgfpathlineto{\pgfqpoint{6.810640in}{1.680000in}}%
\pgfpathlineto{\pgfqpoint{6.813120in}{2.310000in}}%
\pgfpathlineto{\pgfqpoint{6.814360in}{1.715000in}}%
\pgfpathlineto{\pgfqpoint{6.815600in}{2.205000in}}%
\pgfpathlineto{\pgfqpoint{6.816840in}{2.065000in}}%
\pgfpathlineto{\pgfqpoint{6.818080in}{2.100000in}}%
\pgfpathlineto{\pgfqpoint{6.819320in}{1.750000in}}%
\pgfpathlineto{\pgfqpoint{6.820560in}{1.995000in}}%
\pgfpathlineto{\pgfqpoint{6.821800in}{1.610000in}}%
\pgfpathlineto{\pgfqpoint{6.823040in}{2.205000in}}%
\pgfpathlineto{\pgfqpoint{6.824280in}{1.575000in}}%
\pgfpathlineto{\pgfqpoint{6.825520in}{1.750000in}}%
\pgfpathlineto{\pgfqpoint{6.826760in}{1.610000in}}%
\pgfpathlineto{\pgfqpoint{6.830480in}{2.345000in}}%
\pgfpathlineto{\pgfqpoint{6.831720in}{1.855000in}}%
\pgfpathlineto{\pgfqpoint{6.832960in}{2.100000in}}%
\pgfpathlineto{\pgfqpoint{6.834200in}{1.540000in}}%
\pgfpathlineto{\pgfqpoint{6.835440in}{2.135000in}}%
\pgfpathlineto{\pgfqpoint{6.836680in}{2.135000in}}%
\pgfpathlineto{\pgfqpoint{6.837920in}{2.275000in}}%
\pgfpathlineto{\pgfqpoint{6.839160in}{2.030000in}}%
\pgfpathlineto{\pgfqpoint{6.841640in}{2.345000in}}%
\pgfpathlineto{\pgfqpoint{6.844120in}{1.645000in}}%
\pgfpathlineto{\pgfqpoint{6.845360in}{1.960000in}}%
\pgfpathlineto{\pgfqpoint{6.846600in}{1.575000in}}%
\pgfpathlineto{\pgfqpoint{6.847840in}{2.100000in}}%
\pgfpathlineto{\pgfqpoint{6.849080in}{1.785000in}}%
\pgfpathlineto{\pgfqpoint{6.850320in}{2.030000in}}%
\pgfpathlineto{\pgfqpoint{6.851560in}{1.820000in}}%
\pgfpathlineto{\pgfqpoint{6.852800in}{1.820000in}}%
\pgfpathlineto{\pgfqpoint{6.854040in}{1.645000in}}%
\pgfpathlineto{\pgfqpoint{6.856520in}{1.995000in}}%
\pgfpathlineto{\pgfqpoint{6.857760in}{1.960000in}}%
\pgfpathlineto{\pgfqpoint{6.859000in}{2.345000in}}%
\pgfpathlineto{\pgfqpoint{6.861480in}{1.995000in}}%
\pgfpathlineto{\pgfqpoint{6.862720in}{1.855000in}}%
\pgfpathlineto{\pgfqpoint{6.863960in}{2.065000in}}%
\pgfpathlineto{\pgfqpoint{6.865200in}{1.855000in}}%
\pgfpathlineto{\pgfqpoint{6.866440in}{2.100000in}}%
\pgfpathlineto{\pgfqpoint{6.867680in}{1.785000in}}%
\pgfpathlineto{\pgfqpoint{6.868920in}{2.065000in}}%
\pgfpathlineto{\pgfqpoint{6.870160in}{1.715000in}}%
\pgfpathlineto{\pgfqpoint{6.871400in}{1.750000in}}%
\pgfpathlineto{\pgfqpoint{6.872640in}{1.820000in}}%
\pgfpathlineto{\pgfqpoint{6.873880in}{1.960000in}}%
\pgfpathlineto{\pgfqpoint{6.875120in}{1.505000in}}%
\pgfpathlineto{\pgfqpoint{6.876360in}{2.100000in}}%
\pgfpathlineto{\pgfqpoint{6.877600in}{1.960000in}}%
\pgfpathlineto{\pgfqpoint{6.878840in}{2.205000in}}%
\pgfpathlineto{\pgfqpoint{6.880080in}{1.295000in}}%
\pgfpathlineto{\pgfqpoint{6.881320in}{2.030000in}}%
\pgfpathlineto{\pgfqpoint{6.882560in}{1.925000in}}%
\pgfpathlineto{\pgfqpoint{6.883800in}{1.715000in}}%
\pgfpathlineto{\pgfqpoint{6.885040in}{1.820000in}}%
\pgfpathlineto{\pgfqpoint{6.886280in}{1.750000in}}%
\pgfpathlineto{\pgfqpoint{6.887520in}{2.065000in}}%
\pgfpathlineto{\pgfqpoint{6.888760in}{1.960000in}}%
\pgfpathlineto{\pgfqpoint{6.891240in}{1.505000in}}%
\pgfpathlineto{\pgfqpoint{6.892480in}{1.715000in}}%
\pgfpathlineto{\pgfqpoint{6.893720in}{1.645000in}}%
\pgfpathlineto{\pgfqpoint{6.894960in}{1.505000in}}%
\pgfpathlineto{\pgfqpoint{6.896200in}{1.960000in}}%
\pgfpathlineto{\pgfqpoint{6.898680in}{1.470000in}}%
\pgfpathlineto{\pgfqpoint{6.899920in}{2.205000in}}%
\pgfpathlineto{\pgfqpoint{6.901160in}{1.715000in}}%
\pgfpathlineto{\pgfqpoint{6.902400in}{1.960000in}}%
\pgfpathlineto{\pgfqpoint{6.904880in}{1.575000in}}%
\pgfpathlineto{\pgfqpoint{6.907360in}{2.345000in}}%
\pgfpathlineto{\pgfqpoint{6.908600in}{2.345000in}}%
\pgfpathlineto{\pgfqpoint{6.912320in}{1.820000in}}%
\pgfpathlineto{\pgfqpoint{6.913560in}{1.890000in}}%
\pgfpathlineto{\pgfqpoint{6.914800in}{2.205000in}}%
\pgfpathlineto{\pgfqpoint{6.916040in}{1.750000in}}%
\pgfpathlineto{\pgfqpoint{6.917280in}{1.890000in}}%
\pgfpathlineto{\pgfqpoint{6.918520in}{2.205000in}}%
\pgfpathlineto{\pgfqpoint{6.919760in}{1.785000in}}%
\pgfpathlineto{\pgfqpoint{6.922240in}{2.100000in}}%
\pgfpathlineto{\pgfqpoint{6.923480in}{2.275000in}}%
\pgfpathlineto{\pgfqpoint{6.925960in}{2.065000in}}%
\pgfpathlineto{\pgfqpoint{6.927200in}{2.205000in}}%
\pgfpathlineto{\pgfqpoint{6.928440in}{1.715000in}}%
\pgfpathlineto{\pgfqpoint{6.929680in}{1.750000in}}%
\pgfpathlineto{\pgfqpoint{6.932160in}{1.890000in}}%
\pgfpathlineto{\pgfqpoint{6.934640in}{1.610000in}}%
\pgfpathlineto{\pgfqpoint{6.935880in}{2.135000in}}%
\pgfpathlineto{\pgfqpoint{6.937120in}{2.135000in}}%
\pgfpathlineto{\pgfqpoint{6.939600in}{1.680000in}}%
\pgfpathlineto{\pgfqpoint{6.940840in}{2.170000in}}%
\pgfpathlineto{\pgfqpoint{6.942080in}{2.065000in}}%
\pgfpathlineto{\pgfqpoint{6.943320in}{1.435000in}}%
\pgfpathlineto{\pgfqpoint{6.944560in}{1.750000in}}%
\pgfpathlineto{\pgfqpoint{6.945800in}{1.680000in}}%
\pgfpathlineto{\pgfqpoint{6.947040in}{2.345000in}}%
\pgfpathlineto{\pgfqpoint{6.949520in}{1.750000in}}%
\pgfpathlineto{\pgfqpoint{6.950760in}{1.925000in}}%
\pgfpathlineto{\pgfqpoint{6.952000in}{1.645000in}}%
\pgfpathlineto{\pgfqpoint{6.953240in}{1.925000in}}%
\pgfpathlineto{\pgfqpoint{6.954480in}{1.925000in}}%
\pgfpathlineto{\pgfqpoint{6.955720in}{2.240000in}}%
\pgfpathlineto{\pgfqpoint{6.956960in}{1.750000in}}%
\pgfpathlineto{\pgfqpoint{6.958200in}{1.785000in}}%
\pgfpathlineto{\pgfqpoint{6.960680in}{2.135000in}}%
\pgfpathlineto{\pgfqpoint{6.961920in}{2.275000in}}%
\pgfpathlineto{\pgfqpoint{6.963160in}{1.715000in}}%
\pgfpathlineto{\pgfqpoint{6.964400in}{2.030000in}}%
\pgfpathlineto{\pgfqpoint{6.965640in}{1.960000in}}%
\pgfpathlineto{\pgfqpoint{6.966880in}{2.065000in}}%
\pgfpathlineto{\pgfqpoint{6.968120in}{1.820000in}}%
\pgfpathlineto{\pgfqpoint{6.970600in}{1.995000in}}%
\pgfpathlineto{\pgfqpoint{6.971840in}{1.925000in}}%
\pgfpathlineto{\pgfqpoint{6.973080in}{2.065000in}}%
\pgfpathlineto{\pgfqpoint{6.974320in}{1.715000in}}%
\pgfpathlineto{\pgfqpoint{6.975560in}{2.135000in}}%
\pgfpathlineto{\pgfqpoint{6.976800in}{2.065000in}}%
\pgfpathlineto{\pgfqpoint{6.978040in}{2.275000in}}%
\pgfpathlineto{\pgfqpoint{6.979280in}{2.170000in}}%
\pgfpathlineto{\pgfqpoint{6.980520in}{1.715000in}}%
\pgfpathlineto{\pgfqpoint{6.981760in}{2.170000in}}%
\pgfpathlineto{\pgfqpoint{6.983000in}{2.065000in}}%
\pgfpathlineto{\pgfqpoint{6.984240in}{1.785000in}}%
\pgfpathlineto{\pgfqpoint{6.985480in}{1.820000in}}%
\pgfpathlineto{\pgfqpoint{6.987960in}{1.470000in}}%
\pgfpathlineto{\pgfqpoint{6.989200in}{1.505000in}}%
\pgfpathlineto{\pgfqpoint{6.990440in}{1.890000in}}%
\pgfpathlineto{\pgfqpoint{6.991680in}{1.680000in}}%
\pgfpathlineto{\pgfqpoint{6.992920in}{1.855000in}}%
\pgfpathlineto{\pgfqpoint{6.994160in}{1.505000in}}%
\pgfpathlineto{\pgfqpoint{6.996640in}{1.995000in}}%
\pgfpathlineto{\pgfqpoint{6.997880in}{1.855000in}}%
\pgfpathlineto{\pgfqpoint{6.999120in}{1.890000in}}%
\pgfpathlineto{\pgfqpoint{7.000360in}{1.820000in}}%
\pgfpathlineto{\pgfqpoint{7.001600in}{2.100000in}}%
\pgfpathlineto{\pgfqpoint{7.002840in}{1.680000in}}%
\pgfpathlineto{\pgfqpoint{7.004080in}{1.995000in}}%
\pgfpathlineto{\pgfqpoint{7.005320in}{1.785000in}}%
\pgfpathlineto{\pgfqpoint{7.006560in}{2.065000in}}%
\pgfpathlineto{\pgfqpoint{7.009040in}{1.505000in}}%
\pgfpathlineto{\pgfqpoint{7.011520in}{1.680000in}}%
\pgfpathlineto{\pgfqpoint{7.012760in}{1.610000in}}%
\pgfpathlineto{\pgfqpoint{7.014000in}{1.645000in}}%
\pgfpathlineto{\pgfqpoint{7.015240in}{1.960000in}}%
\pgfpathlineto{\pgfqpoint{7.016480in}{1.855000in}}%
\pgfpathlineto{\pgfqpoint{7.017720in}{1.960000in}}%
\pgfpathlineto{\pgfqpoint{7.018960in}{1.750000in}}%
\pgfpathlineto{\pgfqpoint{7.020200in}{2.100000in}}%
\pgfpathlineto{\pgfqpoint{7.021440in}{1.750000in}}%
\pgfpathlineto{\pgfqpoint{7.022680in}{1.855000in}}%
\pgfpathlineto{\pgfqpoint{7.023920in}{1.715000in}}%
\pgfpathlineto{\pgfqpoint{7.025160in}{1.820000in}}%
\pgfpathlineto{\pgfqpoint{7.026400in}{1.750000in}}%
\pgfpathlineto{\pgfqpoint{7.027640in}{1.330000in}}%
\pgfpathlineto{\pgfqpoint{7.028880in}{1.995000in}}%
\pgfpathlineto{\pgfqpoint{7.030120in}{1.960000in}}%
\pgfpathlineto{\pgfqpoint{7.031360in}{2.135000in}}%
\pgfpathlineto{\pgfqpoint{7.032600in}{2.030000in}}%
\pgfpathlineto{\pgfqpoint{7.033840in}{1.680000in}}%
\pgfpathlineto{\pgfqpoint{7.035080in}{1.680000in}}%
\pgfpathlineto{\pgfqpoint{7.036320in}{1.435000in}}%
\pgfpathlineto{\pgfqpoint{7.038800in}{2.135000in}}%
\pgfpathlineto{\pgfqpoint{7.040040in}{1.785000in}}%
\pgfpathlineto{\pgfqpoint{7.041280in}{2.170000in}}%
\pgfpathlineto{\pgfqpoint{7.042520in}{1.960000in}}%
\pgfpathlineto{\pgfqpoint{7.043760in}{2.030000in}}%
\pgfpathlineto{\pgfqpoint{7.045000in}{2.240000in}}%
\pgfpathlineto{\pgfqpoint{7.046240in}{1.505000in}}%
\pgfpathlineto{\pgfqpoint{7.048720in}{1.995000in}}%
\pgfpathlineto{\pgfqpoint{7.049960in}{1.750000in}}%
\pgfpathlineto{\pgfqpoint{7.051200in}{2.100000in}}%
\pgfpathlineto{\pgfqpoint{7.052440in}{1.960000in}}%
\pgfpathlineto{\pgfqpoint{7.053680in}{1.575000in}}%
\pgfpathlineto{\pgfqpoint{7.054920in}{2.100000in}}%
\pgfpathlineto{\pgfqpoint{7.056160in}{1.540000in}}%
\pgfpathlineto{\pgfqpoint{7.057400in}{1.995000in}}%
\pgfpathlineto{\pgfqpoint{7.058640in}{1.575000in}}%
\pgfpathlineto{\pgfqpoint{7.059880in}{2.065000in}}%
\pgfpathlineto{\pgfqpoint{7.061120in}{1.960000in}}%
\pgfpathlineto{\pgfqpoint{7.062360in}{2.135000in}}%
\pgfpathlineto{\pgfqpoint{7.063600in}{1.855000in}}%
\pgfpathlineto{\pgfqpoint{7.064840in}{2.100000in}}%
\pgfpathlineto{\pgfqpoint{7.066080in}{1.750000in}}%
\pgfpathlineto{\pgfqpoint{7.068560in}{2.030000in}}%
\pgfpathlineto{\pgfqpoint{7.069800in}{2.100000in}}%
\pgfpathlineto{\pgfqpoint{7.071040in}{2.100000in}}%
\pgfpathlineto{\pgfqpoint{7.072280in}{1.855000in}}%
\pgfpathlineto{\pgfqpoint{7.073520in}{2.275000in}}%
\pgfpathlineto{\pgfqpoint{7.077240in}{1.505000in}}%
\pgfpathlineto{\pgfqpoint{7.079720in}{1.715000in}}%
\pgfpathlineto{\pgfqpoint{7.080960in}{1.505000in}}%
\pgfpathlineto{\pgfqpoint{7.084680in}{2.100000in}}%
\pgfpathlineto{\pgfqpoint{7.087160in}{1.435000in}}%
\pgfpathlineto{\pgfqpoint{7.088400in}{1.680000in}}%
\pgfpathlineto{\pgfqpoint{7.089640in}{1.505000in}}%
\pgfpathlineto{\pgfqpoint{7.090880in}{2.135000in}}%
\pgfpathlineto{\pgfqpoint{7.093360in}{1.645000in}}%
\pgfpathlineto{\pgfqpoint{7.095840in}{1.890000in}}%
\pgfpathlineto{\pgfqpoint{7.098320in}{1.715000in}}%
\pgfpathlineto{\pgfqpoint{7.099560in}{2.170000in}}%
\pgfpathlineto{\pgfqpoint{7.100800in}{2.065000in}}%
\pgfpathlineto{\pgfqpoint{7.102040in}{1.750000in}}%
\pgfpathlineto{\pgfqpoint{7.104520in}{1.960000in}}%
\pgfpathlineto{\pgfqpoint{7.105760in}{1.470000in}}%
\pgfpathlineto{\pgfqpoint{7.107000in}{1.960000in}}%
\pgfpathlineto{\pgfqpoint{7.108240in}{1.610000in}}%
\pgfpathlineto{\pgfqpoint{7.110720in}{2.205000in}}%
\pgfpathlineto{\pgfqpoint{7.113200in}{1.750000in}}%
\pgfpathlineto{\pgfqpoint{7.114440in}{1.890000in}}%
\pgfpathlineto{\pgfqpoint{7.116920in}{1.435000in}}%
\pgfpathlineto{\pgfqpoint{7.118160in}{1.960000in}}%
\pgfpathlineto{\pgfqpoint{7.120640in}{1.785000in}}%
\pgfpathlineto{\pgfqpoint{7.121880in}{1.960000in}}%
\pgfpathlineto{\pgfqpoint{7.123120in}{1.645000in}}%
\pgfpathlineto{\pgfqpoint{7.124360in}{1.890000in}}%
\pgfpathlineto{\pgfqpoint{7.125600in}{1.820000in}}%
\pgfpathlineto{\pgfqpoint{7.126840in}{1.680000in}}%
\pgfpathlineto{\pgfqpoint{7.129320in}{1.820000in}}%
\pgfpathlineto{\pgfqpoint{7.130560in}{1.890000in}}%
\pgfpathlineto{\pgfqpoint{7.131800in}{2.135000in}}%
\pgfpathlineto{\pgfqpoint{7.133040in}{1.610000in}}%
\pgfpathlineto{\pgfqpoint{7.134280in}{1.820000in}}%
\pgfpathlineto{\pgfqpoint{7.135520in}{1.680000in}}%
\pgfpathlineto{\pgfqpoint{7.138000in}{1.890000in}}%
\pgfpathlineto{\pgfqpoint{7.139240in}{1.925000in}}%
\pgfpathlineto{\pgfqpoint{7.140480in}{1.575000in}}%
\pgfpathlineto{\pgfqpoint{7.141720in}{1.610000in}}%
\pgfpathlineto{\pgfqpoint{7.142960in}{2.100000in}}%
\pgfpathlineto{\pgfqpoint{7.144200in}{1.995000in}}%
\pgfpathlineto{\pgfqpoint{7.145440in}{2.030000in}}%
\pgfpathlineto{\pgfqpoint{7.146680in}{1.995000in}}%
\pgfpathlineto{\pgfqpoint{7.147920in}{2.030000in}}%
\pgfpathlineto{\pgfqpoint{7.149160in}{2.100000in}}%
\pgfpathlineto{\pgfqpoint{7.150400in}{1.855000in}}%
\pgfpathlineto{\pgfqpoint{7.151640in}{1.995000in}}%
\pgfpathlineto{\pgfqpoint{7.152880in}{1.890000in}}%
\pgfpathlineto{\pgfqpoint{7.155360in}{2.310000in}}%
\pgfpathlineto{\pgfqpoint{7.156600in}{2.240000in}}%
\pgfpathlineto{\pgfqpoint{7.159080in}{1.645000in}}%
\pgfpathlineto{\pgfqpoint{7.160320in}{1.925000in}}%
\pgfpathlineto{\pgfqpoint{7.161560in}{1.890000in}}%
\pgfpathlineto{\pgfqpoint{7.162800in}{2.205000in}}%
\pgfpathlineto{\pgfqpoint{7.164040in}{1.820000in}}%
\pgfpathlineto{\pgfqpoint{7.165280in}{2.415000in}}%
\pgfpathlineto{\pgfqpoint{7.167760in}{1.995000in}}%
\pgfpathlineto{\pgfqpoint{7.169000in}{1.960000in}}%
\pgfpathlineto{\pgfqpoint{7.170240in}{2.170000in}}%
\pgfpathlineto{\pgfqpoint{7.173960in}{1.890000in}}%
\pgfpathlineto{\pgfqpoint{7.175200in}{1.925000in}}%
\pgfpathlineto{\pgfqpoint{7.176440in}{1.855000in}}%
\pgfpathlineto{\pgfqpoint{7.178920in}{1.960000in}}%
\pgfpathlineto{\pgfqpoint{7.180160in}{1.995000in}}%
\pgfpathlineto{\pgfqpoint{7.181400in}{1.960000in}}%
\pgfpathlineto{\pgfqpoint{7.182640in}{1.540000in}}%
\pgfpathlineto{\pgfqpoint{7.183880in}{2.065000in}}%
\pgfpathlineto{\pgfqpoint{7.186360in}{1.785000in}}%
\pgfpathlineto{\pgfqpoint{7.187600in}{2.135000in}}%
\pgfpathlineto{\pgfqpoint{7.188840in}{2.030000in}}%
\pgfpathlineto{\pgfqpoint{7.190080in}{2.100000in}}%
\pgfpathlineto{\pgfqpoint{7.191320in}{1.995000in}}%
\pgfpathlineto{\pgfqpoint{7.192560in}{2.100000in}}%
\pgfpathlineto{\pgfqpoint{7.193800in}{2.100000in}}%
\pgfpathlineto{\pgfqpoint{7.195040in}{1.855000in}}%
\pgfpathlineto{\pgfqpoint{7.196280in}{1.855000in}}%
\pgfpathlineto{\pgfqpoint{7.197520in}{1.680000in}}%
\pgfpathlineto{\pgfqpoint{7.200000in}{2.065000in}}%
\pgfpathlineto{\pgfqpoint{7.200000in}{2.065000in}}%
\pgfusepath{stroke}%
\end{pgfscope}%
\begin{pgfscope}%
\pgfpathrectangle{\pgfqpoint{1.000000in}{0.350000in}}{\pgfqpoint{6.200000in}{2.800000in}} %
\pgfusepath{clip}%
\pgfsetrectcap%
\pgfsetroundjoin%
\pgfsetlinewidth{1.003750pt}%
\definecolor{currentstroke}{rgb}{0.000000,0.500000,0.000000}%
\pgfsetstrokecolor{currentstroke}%
\pgfsetdash{}{0pt}%
\pgfpathmoveto{\pgfqpoint{1.000000in}{2.415000in}}%
\pgfpathlineto{\pgfqpoint{1.001240in}{2.205000in}}%
\pgfpathlineto{\pgfqpoint{1.002480in}{2.450000in}}%
\pgfpathlineto{\pgfqpoint{1.003720in}{2.450000in}}%
\pgfpathlineto{\pgfqpoint{1.004960in}{2.275000in}}%
\pgfpathlineto{\pgfqpoint{1.007440in}{2.345000in}}%
\pgfpathlineto{\pgfqpoint{1.008680in}{2.275000in}}%
\pgfpathlineto{\pgfqpoint{1.009920in}{2.345000in}}%
\pgfpathlineto{\pgfqpoint{1.012400in}{2.170000in}}%
\pgfpathlineto{\pgfqpoint{1.013640in}{2.380000in}}%
\pgfpathlineto{\pgfqpoint{1.014880in}{2.240000in}}%
\pgfpathlineto{\pgfqpoint{1.016120in}{2.625000in}}%
\pgfpathlineto{\pgfqpoint{1.017360in}{2.590000in}}%
\pgfpathlineto{\pgfqpoint{1.018600in}{2.345000in}}%
\pgfpathlineto{\pgfqpoint{1.019840in}{2.590000in}}%
\pgfpathlineto{\pgfqpoint{1.021080in}{2.275000in}}%
\pgfpathlineto{\pgfqpoint{1.022320in}{2.415000in}}%
\pgfpathlineto{\pgfqpoint{1.024800in}{2.345000in}}%
\pgfpathlineto{\pgfqpoint{1.026040in}{2.380000in}}%
\pgfpathlineto{\pgfqpoint{1.027280in}{2.345000in}}%
\pgfpathlineto{\pgfqpoint{1.028520in}{2.520000in}}%
\pgfpathlineto{\pgfqpoint{1.029760in}{2.275000in}}%
\pgfpathlineto{\pgfqpoint{1.032240in}{2.695000in}}%
\pgfpathlineto{\pgfqpoint{1.034720in}{1.925000in}}%
\pgfpathlineto{\pgfqpoint{1.035960in}{2.590000in}}%
\pgfpathlineto{\pgfqpoint{1.037200in}{2.555000in}}%
\pgfpathlineto{\pgfqpoint{1.038440in}{2.590000in}}%
\pgfpathlineto{\pgfqpoint{1.039680in}{2.240000in}}%
\pgfpathlineto{\pgfqpoint{1.040920in}{2.380000in}}%
\pgfpathlineto{\pgfqpoint{1.042160in}{2.310000in}}%
\pgfpathlineto{\pgfqpoint{1.045880in}{2.765000in}}%
\pgfpathlineto{\pgfqpoint{1.047120in}{2.240000in}}%
\pgfpathlineto{\pgfqpoint{1.049600in}{2.485000in}}%
\pgfpathlineto{\pgfqpoint{1.052080in}{2.240000in}}%
\pgfpathlineto{\pgfqpoint{1.053320in}{2.240000in}}%
\pgfpathlineto{\pgfqpoint{1.054560in}{2.660000in}}%
\pgfpathlineto{\pgfqpoint{1.055800in}{2.520000in}}%
\pgfpathlineto{\pgfqpoint{1.057040in}{2.625000in}}%
\pgfpathlineto{\pgfqpoint{1.058280in}{2.520000in}}%
\pgfpathlineto{\pgfqpoint{1.059520in}{2.555000in}}%
\pgfpathlineto{\pgfqpoint{1.060760in}{1.995000in}}%
\pgfpathlineto{\pgfqpoint{1.063240in}{2.380000in}}%
\pgfpathlineto{\pgfqpoint{1.064480in}{2.100000in}}%
\pgfpathlineto{\pgfqpoint{1.066960in}{2.415000in}}%
\pgfpathlineto{\pgfqpoint{1.068200in}{2.485000in}}%
\pgfpathlineto{\pgfqpoint{1.069440in}{2.730000in}}%
\pgfpathlineto{\pgfqpoint{1.070680in}{2.590000in}}%
\pgfpathlineto{\pgfqpoint{1.071920in}{2.660000in}}%
\pgfpathlineto{\pgfqpoint{1.074400in}{2.065000in}}%
\pgfpathlineto{\pgfqpoint{1.075640in}{2.205000in}}%
\pgfpathlineto{\pgfqpoint{1.078120in}{2.555000in}}%
\pgfpathlineto{\pgfqpoint{1.079360in}{2.625000in}}%
\pgfpathlineto{\pgfqpoint{1.080600in}{2.590000in}}%
\pgfpathlineto{\pgfqpoint{1.081840in}{2.170000in}}%
\pgfpathlineto{\pgfqpoint{1.083080in}{2.695000in}}%
\pgfpathlineto{\pgfqpoint{1.085560in}{2.205000in}}%
\pgfpathlineto{\pgfqpoint{1.088040in}{2.625000in}}%
\pgfpathlineto{\pgfqpoint{1.089280in}{2.240000in}}%
\pgfpathlineto{\pgfqpoint{1.090520in}{2.590000in}}%
\pgfpathlineto{\pgfqpoint{1.091760in}{2.625000in}}%
\pgfpathlineto{\pgfqpoint{1.093000in}{2.555000in}}%
\pgfpathlineto{\pgfqpoint{1.094240in}{2.310000in}}%
\pgfpathlineto{\pgfqpoint{1.095480in}{2.415000in}}%
\pgfpathlineto{\pgfqpoint{1.096720in}{2.345000in}}%
\pgfpathlineto{\pgfqpoint{1.097960in}{2.415000in}}%
\pgfpathlineto{\pgfqpoint{1.099200in}{1.925000in}}%
\pgfpathlineto{\pgfqpoint{1.100440in}{2.485000in}}%
\pgfpathlineto{\pgfqpoint{1.101680in}{2.100000in}}%
\pgfpathlineto{\pgfqpoint{1.102920in}{2.345000in}}%
\pgfpathlineto{\pgfqpoint{1.104160in}{2.275000in}}%
\pgfpathlineto{\pgfqpoint{1.105400in}{2.450000in}}%
\pgfpathlineto{\pgfqpoint{1.106640in}{2.345000in}}%
\pgfpathlineto{\pgfqpoint{1.107880in}{2.415000in}}%
\pgfpathlineto{\pgfqpoint{1.109120in}{2.415000in}}%
\pgfpathlineto{\pgfqpoint{1.110360in}{2.485000in}}%
\pgfpathlineto{\pgfqpoint{1.111600in}{2.205000in}}%
\pgfpathlineto{\pgfqpoint{1.112840in}{2.310000in}}%
\pgfpathlineto{\pgfqpoint{1.114080in}{2.205000in}}%
\pgfpathlineto{\pgfqpoint{1.115320in}{2.380000in}}%
\pgfpathlineto{\pgfqpoint{1.116560in}{2.765000in}}%
\pgfpathlineto{\pgfqpoint{1.119040in}{2.555000in}}%
\pgfpathlineto{\pgfqpoint{1.120280in}{2.730000in}}%
\pgfpathlineto{\pgfqpoint{1.121520in}{2.135000in}}%
\pgfpathlineto{\pgfqpoint{1.122760in}{2.275000in}}%
\pgfpathlineto{\pgfqpoint{1.124000in}{2.240000in}}%
\pgfpathlineto{\pgfqpoint{1.125240in}{2.450000in}}%
\pgfpathlineto{\pgfqpoint{1.126480in}{2.275000in}}%
\pgfpathlineto{\pgfqpoint{1.130200in}{2.940000in}}%
\pgfpathlineto{\pgfqpoint{1.131440in}{2.520000in}}%
\pgfpathlineto{\pgfqpoint{1.132680in}{2.485000in}}%
\pgfpathlineto{\pgfqpoint{1.133920in}{2.275000in}}%
\pgfpathlineto{\pgfqpoint{1.138880in}{2.555000in}}%
\pgfpathlineto{\pgfqpoint{1.140120in}{2.205000in}}%
\pgfpathlineto{\pgfqpoint{1.141360in}{2.205000in}}%
\pgfpathlineto{\pgfqpoint{1.142600in}{2.590000in}}%
\pgfpathlineto{\pgfqpoint{1.143840in}{2.590000in}}%
\pgfpathlineto{\pgfqpoint{1.145080in}{2.240000in}}%
\pgfpathlineto{\pgfqpoint{1.146320in}{2.765000in}}%
\pgfpathlineto{\pgfqpoint{1.151280in}{2.030000in}}%
\pgfpathlineto{\pgfqpoint{1.152520in}{2.240000in}}%
\pgfpathlineto{\pgfqpoint{1.153760in}{2.765000in}}%
\pgfpathlineto{\pgfqpoint{1.157480in}{2.030000in}}%
\pgfpathlineto{\pgfqpoint{1.158720in}{2.380000in}}%
\pgfpathlineto{\pgfqpoint{1.159960in}{2.205000in}}%
\pgfpathlineto{\pgfqpoint{1.162440in}{2.765000in}}%
\pgfpathlineto{\pgfqpoint{1.164920in}{2.170000in}}%
\pgfpathlineto{\pgfqpoint{1.167400in}{2.590000in}}%
\pgfpathlineto{\pgfqpoint{1.168640in}{2.205000in}}%
\pgfpathlineto{\pgfqpoint{1.169880in}{2.170000in}}%
\pgfpathlineto{\pgfqpoint{1.171120in}{2.800000in}}%
\pgfpathlineto{\pgfqpoint{1.172360in}{2.625000in}}%
\pgfpathlineto{\pgfqpoint{1.173600in}{2.205000in}}%
\pgfpathlineto{\pgfqpoint{1.174840in}{2.310000in}}%
\pgfpathlineto{\pgfqpoint{1.176080in}{2.765000in}}%
\pgfpathlineto{\pgfqpoint{1.177320in}{2.170000in}}%
\pgfpathlineto{\pgfqpoint{1.178560in}{2.240000in}}%
\pgfpathlineto{\pgfqpoint{1.181040in}{2.660000in}}%
\pgfpathlineto{\pgfqpoint{1.182280in}{2.240000in}}%
\pgfpathlineto{\pgfqpoint{1.183520in}{2.240000in}}%
\pgfpathlineto{\pgfqpoint{1.184760in}{2.205000in}}%
\pgfpathlineto{\pgfqpoint{1.186000in}{1.995000in}}%
\pgfpathlineto{\pgfqpoint{1.187240in}{1.995000in}}%
\pgfpathlineto{\pgfqpoint{1.189720in}{2.345000in}}%
\pgfpathlineto{\pgfqpoint{1.190960in}{2.345000in}}%
\pgfpathlineto{\pgfqpoint{1.192200in}{2.135000in}}%
\pgfpathlineto{\pgfqpoint{1.193440in}{2.695000in}}%
\pgfpathlineto{\pgfqpoint{1.194680in}{2.555000in}}%
\pgfpathlineto{\pgfqpoint{1.195920in}{2.555000in}}%
\pgfpathlineto{\pgfqpoint{1.197160in}{2.380000in}}%
\pgfpathlineto{\pgfqpoint{1.198400in}{2.800000in}}%
\pgfpathlineto{\pgfqpoint{1.199640in}{2.835000in}}%
\pgfpathlineto{\pgfqpoint{1.200880in}{2.520000in}}%
\pgfpathlineto{\pgfqpoint{1.202120in}{2.730000in}}%
\pgfpathlineto{\pgfqpoint{1.203360in}{2.695000in}}%
\pgfpathlineto{\pgfqpoint{1.205840in}{2.100000in}}%
\pgfpathlineto{\pgfqpoint{1.207080in}{2.135000in}}%
\pgfpathlineto{\pgfqpoint{1.210800in}{2.590000in}}%
\pgfpathlineto{\pgfqpoint{1.213280in}{2.345000in}}%
\pgfpathlineto{\pgfqpoint{1.215760in}{2.905000in}}%
\pgfpathlineto{\pgfqpoint{1.217000in}{2.590000in}}%
\pgfpathlineto{\pgfqpoint{1.218240in}{2.835000in}}%
\pgfpathlineto{\pgfqpoint{1.219480in}{2.730000in}}%
\pgfpathlineto{\pgfqpoint{1.220720in}{2.940000in}}%
\pgfpathlineto{\pgfqpoint{1.221960in}{2.450000in}}%
\pgfpathlineto{\pgfqpoint{1.224440in}{2.520000in}}%
\pgfpathlineto{\pgfqpoint{1.225680in}{2.520000in}}%
\pgfpathlineto{\pgfqpoint{1.226920in}{1.855000in}}%
\pgfpathlineto{\pgfqpoint{1.228160in}{2.485000in}}%
\pgfpathlineto{\pgfqpoint{1.229400in}{2.100000in}}%
\pgfpathlineto{\pgfqpoint{1.230640in}{2.380000in}}%
\pgfpathlineto{\pgfqpoint{1.233120in}{2.170000in}}%
\pgfpathlineto{\pgfqpoint{1.234360in}{2.520000in}}%
\pgfpathlineto{\pgfqpoint{1.235600in}{2.415000in}}%
\pgfpathlineto{\pgfqpoint{1.236840in}{2.485000in}}%
\pgfpathlineto{\pgfqpoint{1.238080in}{2.205000in}}%
\pgfpathlineto{\pgfqpoint{1.239320in}{2.660000in}}%
\pgfpathlineto{\pgfqpoint{1.240560in}{2.345000in}}%
\pgfpathlineto{\pgfqpoint{1.241800in}{2.590000in}}%
\pgfpathlineto{\pgfqpoint{1.243040in}{2.520000in}}%
\pgfpathlineto{\pgfqpoint{1.244280in}{2.555000in}}%
\pgfpathlineto{\pgfqpoint{1.245520in}{2.345000in}}%
\pgfpathlineto{\pgfqpoint{1.246760in}{2.555000in}}%
\pgfpathlineto{\pgfqpoint{1.248000in}{2.415000in}}%
\pgfpathlineto{\pgfqpoint{1.249240in}{2.520000in}}%
\pgfpathlineto{\pgfqpoint{1.250480in}{2.345000in}}%
\pgfpathlineto{\pgfqpoint{1.251720in}{2.450000in}}%
\pgfpathlineto{\pgfqpoint{1.252960in}{2.835000in}}%
\pgfpathlineto{\pgfqpoint{1.255440in}{2.450000in}}%
\pgfpathlineto{\pgfqpoint{1.256680in}{2.520000in}}%
\pgfpathlineto{\pgfqpoint{1.257920in}{2.415000in}}%
\pgfpathlineto{\pgfqpoint{1.259160in}{2.100000in}}%
\pgfpathlineto{\pgfqpoint{1.261640in}{2.520000in}}%
\pgfpathlineto{\pgfqpoint{1.262880in}{2.520000in}}%
\pgfpathlineto{\pgfqpoint{1.264120in}{2.345000in}}%
\pgfpathlineto{\pgfqpoint{1.265360in}{2.765000in}}%
\pgfpathlineto{\pgfqpoint{1.266600in}{2.345000in}}%
\pgfpathlineto{\pgfqpoint{1.267840in}{2.345000in}}%
\pgfpathlineto{\pgfqpoint{1.269080in}{2.380000in}}%
\pgfpathlineto{\pgfqpoint{1.270320in}{2.205000in}}%
\pgfpathlineto{\pgfqpoint{1.271560in}{2.310000in}}%
\pgfpathlineto{\pgfqpoint{1.272800in}{2.205000in}}%
\pgfpathlineto{\pgfqpoint{1.274040in}{2.240000in}}%
\pgfpathlineto{\pgfqpoint{1.276520in}{2.590000in}}%
\pgfpathlineto{\pgfqpoint{1.277760in}{2.765000in}}%
\pgfpathlineto{\pgfqpoint{1.279000in}{2.310000in}}%
\pgfpathlineto{\pgfqpoint{1.280240in}{2.450000in}}%
\pgfpathlineto{\pgfqpoint{1.281480in}{2.345000in}}%
\pgfpathlineto{\pgfqpoint{1.282720in}{2.345000in}}%
\pgfpathlineto{\pgfqpoint{1.283960in}{2.485000in}}%
\pgfpathlineto{\pgfqpoint{1.285200in}{2.380000in}}%
\pgfpathlineto{\pgfqpoint{1.286440in}{2.415000in}}%
\pgfpathlineto{\pgfqpoint{1.287680in}{2.415000in}}%
\pgfpathlineto{\pgfqpoint{1.288920in}{2.555000in}}%
\pgfpathlineto{\pgfqpoint{1.290160in}{2.275000in}}%
\pgfpathlineto{\pgfqpoint{1.291400in}{2.660000in}}%
\pgfpathlineto{\pgfqpoint{1.292640in}{2.415000in}}%
\pgfpathlineto{\pgfqpoint{1.293880in}{2.555000in}}%
\pgfpathlineto{\pgfqpoint{1.296360in}{2.205000in}}%
\pgfpathlineto{\pgfqpoint{1.297600in}{2.240000in}}%
\pgfpathlineto{\pgfqpoint{1.298840in}{2.030000in}}%
\pgfpathlineto{\pgfqpoint{1.300080in}{2.205000in}}%
\pgfpathlineto{\pgfqpoint{1.301320in}{2.695000in}}%
\pgfpathlineto{\pgfqpoint{1.302560in}{2.660000in}}%
\pgfpathlineto{\pgfqpoint{1.303800in}{2.555000in}}%
\pgfpathlineto{\pgfqpoint{1.305040in}{2.800000in}}%
\pgfpathlineto{\pgfqpoint{1.306280in}{2.485000in}}%
\pgfpathlineto{\pgfqpoint{1.307520in}{2.695000in}}%
\pgfpathlineto{\pgfqpoint{1.308760in}{2.520000in}}%
\pgfpathlineto{\pgfqpoint{1.310000in}{2.695000in}}%
\pgfpathlineto{\pgfqpoint{1.311240in}{2.345000in}}%
\pgfpathlineto{\pgfqpoint{1.314960in}{2.870000in}}%
\pgfpathlineto{\pgfqpoint{1.318680in}{2.520000in}}%
\pgfpathlineto{\pgfqpoint{1.319920in}{2.590000in}}%
\pgfpathlineto{\pgfqpoint{1.321160in}{2.310000in}}%
\pgfpathlineto{\pgfqpoint{1.322400in}{2.380000in}}%
\pgfpathlineto{\pgfqpoint{1.323640in}{2.555000in}}%
\pgfpathlineto{\pgfqpoint{1.324880in}{2.415000in}}%
\pgfpathlineto{\pgfqpoint{1.327360in}{2.730000in}}%
\pgfpathlineto{\pgfqpoint{1.328600in}{2.555000in}}%
\pgfpathlineto{\pgfqpoint{1.329840in}{2.800000in}}%
\pgfpathlineto{\pgfqpoint{1.331080in}{2.555000in}}%
\pgfpathlineto{\pgfqpoint{1.332320in}{2.695000in}}%
\pgfpathlineto{\pgfqpoint{1.333560in}{2.555000in}}%
\pgfpathlineto{\pgfqpoint{1.334800in}{2.695000in}}%
\pgfpathlineto{\pgfqpoint{1.336040in}{2.975000in}}%
\pgfpathlineto{\pgfqpoint{1.337280in}{2.485000in}}%
\pgfpathlineto{\pgfqpoint{1.339760in}{2.765000in}}%
\pgfpathlineto{\pgfqpoint{1.342240in}{2.240000in}}%
\pgfpathlineto{\pgfqpoint{1.343480in}{2.695000in}}%
\pgfpathlineto{\pgfqpoint{1.344720in}{2.695000in}}%
\pgfpathlineto{\pgfqpoint{1.345960in}{2.345000in}}%
\pgfpathlineto{\pgfqpoint{1.348440in}{2.730000in}}%
\pgfpathlineto{\pgfqpoint{1.349680in}{2.520000in}}%
\pgfpathlineto{\pgfqpoint{1.350920in}{2.520000in}}%
\pgfpathlineto{\pgfqpoint{1.352160in}{2.800000in}}%
\pgfpathlineto{\pgfqpoint{1.353400in}{2.485000in}}%
\pgfpathlineto{\pgfqpoint{1.354640in}{2.765000in}}%
\pgfpathlineto{\pgfqpoint{1.357120in}{2.205000in}}%
\pgfpathlineto{\pgfqpoint{1.358360in}{2.205000in}}%
\pgfpathlineto{\pgfqpoint{1.360840in}{2.520000in}}%
\pgfpathlineto{\pgfqpoint{1.363320in}{2.065000in}}%
\pgfpathlineto{\pgfqpoint{1.364560in}{2.590000in}}%
\pgfpathlineto{\pgfqpoint{1.367040in}{2.275000in}}%
\pgfpathlineto{\pgfqpoint{1.368280in}{2.450000in}}%
\pgfpathlineto{\pgfqpoint{1.369520in}{2.310000in}}%
\pgfpathlineto{\pgfqpoint{1.370760in}{2.520000in}}%
\pgfpathlineto{\pgfqpoint{1.372000in}{2.100000in}}%
\pgfpathlineto{\pgfqpoint{1.376960in}{2.695000in}}%
\pgfpathlineto{\pgfqpoint{1.378200in}{2.065000in}}%
\pgfpathlineto{\pgfqpoint{1.379440in}{2.485000in}}%
\pgfpathlineto{\pgfqpoint{1.381920in}{2.135000in}}%
\pgfpathlineto{\pgfqpoint{1.383160in}{2.135000in}}%
\pgfpathlineto{\pgfqpoint{1.384400in}{1.855000in}}%
\pgfpathlineto{\pgfqpoint{1.385640in}{2.485000in}}%
\pgfpathlineto{\pgfqpoint{1.386880in}{2.415000in}}%
\pgfpathlineto{\pgfqpoint{1.388120in}{2.520000in}}%
\pgfpathlineto{\pgfqpoint{1.389360in}{2.170000in}}%
\pgfpathlineto{\pgfqpoint{1.390600in}{2.800000in}}%
\pgfpathlineto{\pgfqpoint{1.393080in}{2.555000in}}%
\pgfpathlineto{\pgfqpoint{1.394320in}{2.590000in}}%
\pgfpathlineto{\pgfqpoint{1.395560in}{2.485000in}}%
\pgfpathlineto{\pgfqpoint{1.396800in}{2.135000in}}%
\pgfpathlineto{\pgfqpoint{1.398040in}{2.170000in}}%
\pgfpathlineto{\pgfqpoint{1.399280in}{2.590000in}}%
\pgfpathlineto{\pgfqpoint{1.401760in}{2.135000in}}%
\pgfpathlineto{\pgfqpoint{1.404240in}{2.520000in}}%
\pgfpathlineto{\pgfqpoint{1.406720in}{2.450000in}}%
\pgfpathlineto{\pgfqpoint{1.407960in}{2.695000in}}%
\pgfpathlineto{\pgfqpoint{1.409200in}{2.695000in}}%
\pgfpathlineto{\pgfqpoint{1.410440in}{2.625000in}}%
\pgfpathlineto{\pgfqpoint{1.411680in}{2.625000in}}%
\pgfpathlineto{\pgfqpoint{1.412920in}{2.415000in}}%
\pgfpathlineto{\pgfqpoint{1.414160in}{2.765000in}}%
\pgfpathlineto{\pgfqpoint{1.416640in}{2.205000in}}%
\pgfpathlineto{\pgfqpoint{1.417880in}{2.380000in}}%
\pgfpathlineto{\pgfqpoint{1.419120in}{2.940000in}}%
\pgfpathlineto{\pgfqpoint{1.421600in}{2.100000in}}%
\pgfpathlineto{\pgfqpoint{1.425320in}{2.765000in}}%
\pgfpathlineto{\pgfqpoint{1.426560in}{2.555000in}}%
\pgfpathlineto{\pgfqpoint{1.427800in}{2.590000in}}%
\pgfpathlineto{\pgfqpoint{1.429040in}{3.010000in}}%
\pgfpathlineto{\pgfqpoint{1.431520in}{2.275000in}}%
\pgfpathlineto{\pgfqpoint{1.432760in}{2.450000in}}%
\pgfpathlineto{\pgfqpoint{1.434000in}{2.030000in}}%
\pgfpathlineto{\pgfqpoint{1.435240in}{2.380000in}}%
\pgfpathlineto{\pgfqpoint{1.436480in}{2.205000in}}%
\pgfpathlineto{\pgfqpoint{1.438960in}{2.555000in}}%
\pgfpathlineto{\pgfqpoint{1.441440in}{1.960000in}}%
\pgfpathlineto{\pgfqpoint{1.443920in}{2.380000in}}%
\pgfpathlineto{\pgfqpoint{1.445160in}{2.450000in}}%
\pgfpathlineto{\pgfqpoint{1.446400in}{2.170000in}}%
\pgfpathlineto{\pgfqpoint{1.447640in}{2.415000in}}%
\pgfpathlineto{\pgfqpoint{1.448880in}{2.065000in}}%
\pgfpathlineto{\pgfqpoint{1.450120in}{2.065000in}}%
\pgfpathlineto{\pgfqpoint{1.452600in}{2.310000in}}%
\pgfpathlineto{\pgfqpoint{1.455080in}{1.820000in}}%
\pgfpathlineto{\pgfqpoint{1.456320in}{1.995000in}}%
\pgfpathlineto{\pgfqpoint{1.457560in}{2.345000in}}%
\pgfpathlineto{\pgfqpoint{1.458800in}{2.275000in}}%
\pgfpathlineto{\pgfqpoint{1.460040in}{2.135000in}}%
\pgfpathlineto{\pgfqpoint{1.461280in}{2.275000in}}%
\pgfpathlineto{\pgfqpoint{1.462520in}{2.100000in}}%
\pgfpathlineto{\pgfqpoint{1.463760in}{2.275000in}}%
\pgfpathlineto{\pgfqpoint{1.465000in}{2.135000in}}%
\pgfpathlineto{\pgfqpoint{1.466240in}{2.205000in}}%
\pgfpathlineto{\pgfqpoint{1.467480in}{2.205000in}}%
\pgfpathlineto{\pgfqpoint{1.468720in}{2.380000in}}%
\pgfpathlineto{\pgfqpoint{1.469960in}{2.310000in}}%
\pgfpathlineto{\pgfqpoint{1.471200in}{2.485000in}}%
\pgfpathlineto{\pgfqpoint{1.472440in}{2.450000in}}%
\pgfpathlineto{\pgfqpoint{1.473680in}{2.485000in}}%
\pgfpathlineto{\pgfqpoint{1.474920in}{2.625000in}}%
\pgfpathlineto{\pgfqpoint{1.476160in}{2.275000in}}%
\pgfpathlineto{\pgfqpoint{1.477400in}{2.275000in}}%
\pgfpathlineto{\pgfqpoint{1.478640in}{2.660000in}}%
\pgfpathlineto{\pgfqpoint{1.481120in}{2.310000in}}%
\pgfpathlineto{\pgfqpoint{1.483600in}{2.030000in}}%
\pgfpathlineto{\pgfqpoint{1.486080in}{2.415000in}}%
\pgfpathlineto{\pgfqpoint{1.488560in}{2.345000in}}%
\pgfpathlineto{\pgfqpoint{1.491040in}{2.625000in}}%
\pgfpathlineto{\pgfqpoint{1.492280in}{2.590000in}}%
\pgfpathlineto{\pgfqpoint{1.494760in}{1.925000in}}%
\pgfpathlineto{\pgfqpoint{1.497240in}{2.170000in}}%
\pgfpathlineto{\pgfqpoint{1.498480in}{2.065000in}}%
\pgfpathlineto{\pgfqpoint{1.500960in}{2.345000in}}%
\pgfpathlineto{\pgfqpoint{1.503440in}{2.345000in}}%
\pgfpathlineto{\pgfqpoint{1.504680in}{2.590000in}}%
\pgfpathlineto{\pgfqpoint{1.505920in}{2.555000in}}%
\pgfpathlineto{\pgfqpoint{1.507160in}{2.555000in}}%
\pgfpathlineto{\pgfqpoint{1.509640in}{2.380000in}}%
\pgfpathlineto{\pgfqpoint{1.512120in}{2.555000in}}%
\pgfpathlineto{\pgfqpoint{1.513360in}{2.275000in}}%
\pgfpathlineto{\pgfqpoint{1.514600in}{2.590000in}}%
\pgfpathlineto{\pgfqpoint{1.517080in}{2.240000in}}%
\pgfpathlineto{\pgfqpoint{1.518320in}{2.345000in}}%
\pgfpathlineto{\pgfqpoint{1.520800in}{2.170000in}}%
\pgfpathlineto{\pgfqpoint{1.523280in}{2.730000in}}%
\pgfpathlineto{\pgfqpoint{1.524520in}{2.275000in}}%
\pgfpathlineto{\pgfqpoint{1.525760in}{2.240000in}}%
\pgfpathlineto{\pgfqpoint{1.527000in}{2.310000in}}%
\pgfpathlineto{\pgfqpoint{1.528240in}{2.275000in}}%
\pgfpathlineto{\pgfqpoint{1.529480in}{2.415000in}}%
\pgfpathlineto{\pgfqpoint{1.531960in}{2.415000in}}%
\pgfpathlineto{\pgfqpoint{1.533200in}{2.240000in}}%
\pgfpathlineto{\pgfqpoint{1.534440in}{2.415000in}}%
\pgfpathlineto{\pgfqpoint{1.535680in}{2.415000in}}%
\pgfpathlineto{\pgfqpoint{1.536920in}{2.345000in}}%
\pgfpathlineto{\pgfqpoint{1.538160in}{2.485000in}}%
\pgfpathlineto{\pgfqpoint{1.539400in}{2.450000in}}%
\pgfpathlineto{\pgfqpoint{1.540640in}{2.310000in}}%
\pgfpathlineto{\pgfqpoint{1.541880in}{2.310000in}}%
\pgfpathlineto{\pgfqpoint{1.543120in}{2.135000in}}%
\pgfpathlineto{\pgfqpoint{1.545600in}{2.660000in}}%
\pgfpathlineto{\pgfqpoint{1.546840in}{2.135000in}}%
\pgfpathlineto{\pgfqpoint{1.548080in}{2.555000in}}%
\pgfpathlineto{\pgfqpoint{1.549320in}{2.345000in}}%
\pgfpathlineto{\pgfqpoint{1.550560in}{2.450000in}}%
\pgfpathlineto{\pgfqpoint{1.553040in}{2.100000in}}%
\pgfpathlineto{\pgfqpoint{1.555520in}{2.695000in}}%
\pgfpathlineto{\pgfqpoint{1.556760in}{2.590000in}}%
\pgfpathlineto{\pgfqpoint{1.558000in}{2.135000in}}%
\pgfpathlineto{\pgfqpoint{1.559240in}{2.520000in}}%
\pgfpathlineto{\pgfqpoint{1.561720in}{2.275000in}}%
\pgfpathlineto{\pgfqpoint{1.564200in}{2.520000in}}%
\pgfpathlineto{\pgfqpoint{1.565440in}{2.415000in}}%
\pgfpathlineto{\pgfqpoint{1.566680in}{2.415000in}}%
\pgfpathlineto{\pgfqpoint{1.567920in}{2.835000in}}%
\pgfpathlineto{\pgfqpoint{1.569160in}{2.450000in}}%
\pgfpathlineto{\pgfqpoint{1.570400in}{2.520000in}}%
\pgfpathlineto{\pgfqpoint{1.571640in}{2.660000in}}%
\pgfpathlineto{\pgfqpoint{1.572880in}{2.520000in}}%
\pgfpathlineto{\pgfqpoint{1.574120in}{2.240000in}}%
\pgfpathlineto{\pgfqpoint{1.576600in}{2.485000in}}%
\pgfpathlineto{\pgfqpoint{1.577840in}{2.380000in}}%
\pgfpathlineto{\pgfqpoint{1.579080in}{2.415000in}}%
\pgfpathlineto{\pgfqpoint{1.581560in}{2.660000in}}%
\pgfpathlineto{\pgfqpoint{1.582800in}{2.485000in}}%
\pgfpathlineto{\pgfqpoint{1.584040in}{2.590000in}}%
\pgfpathlineto{\pgfqpoint{1.585280in}{2.590000in}}%
\pgfpathlineto{\pgfqpoint{1.587760in}{2.380000in}}%
\pgfpathlineto{\pgfqpoint{1.589000in}{2.450000in}}%
\pgfpathlineto{\pgfqpoint{1.590240in}{2.380000in}}%
\pgfpathlineto{\pgfqpoint{1.591480in}{2.205000in}}%
\pgfpathlineto{\pgfqpoint{1.592720in}{2.275000in}}%
\pgfpathlineto{\pgfqpoint{1.593960in}{2.275000in}}%
\pgfpathlineto{\pgfqpoint{1.595200in}{2.310000in}}%
\pgfpathlineto{\pgfqpoint{1.597680in}{2.660000in}}%
\pgfpathlineto{\pgfqpoint{1.600160in}{2.205000in}}%
\pgfpathlineto{\pgfqpoint{1.601400in}{2.415000in}}%
\pgfpathlineto{\pgfqpoint{1.602640in}{2.275000in}}%
\pgfpathlineto{\pgfqpoint{1.603880in}{2.520000in}}%
\pgfpathlineto{\pgfqpoint{1.605120in}{2.415000in}}%
\pgfpathlineto{\pgfqpoint{1.606360in}{2.625000in}}%
\pgfpathlineto{\pgfqpoint{1.607600in}{2.380000in}}%
\pgfpathlineto{\pgfqpoint{1.608840in}{2.800000in}}%
\pgfpathlineto{\pgfqpoint{1.610080in}{1.960000in}}%
\pgfpathlineto{\pgfqpoint{1.612560in}{2.520000in}}%
\pgfpathlineto{\pgfqpoint{1.613800in}{2.275000in}}%
\pgfpathlineto{\pgfqpoint{1.615040in}{2.310000in}}%
\pgfpathlineto{\pgfqpoint{1.617520in}{2.135000in}}%
\pgfpathlineto{\pgfqpoint{1.618760in}{2.345000in}}%
\pgfpathlineto{\pgfqpoint{1.620000in}{2.310000in}}%
\pgfpathlineto{\pgfqpoint{1.621240in}{2.345000in}}%
\pgfpathlineto{\pgfqpoint{1.622480in}{2.520000in}}%
\pgfpathlineto{\pgfqpoint{1.624960in}{1.890000in}}%
\pgfpathlineto{\pgfqpoint{1.627440in}{2.590000in}}%
\pgfpathlineto{\pgfqpoint{1.629920in}{2.240000in}}%
\pgfpathlineto{\pgfqpoint{1.631160in}{2.310000in}}%
\pgfpathlineto{\pgfqpoint{1.632400in}{2.170000in}}%
\pgfpathlineto{\pgfqpoint{1.634880in}{2.835000in}}%
\pgfpathlineto{\pgfqpoint{1.636120in}{2.730000in}}%
\pgfpathlineto{\pgfqpoint{1.637360in}{2.240000in}}%
\pgfpathlineto{\pgfqpoint{1.638600in}{2.310000in}}%
\pgfpathlineto{\pgfqpoint{1.639840in}{2.170000in}}%
\pgfpathlineto{\pgfqpoint{1.642320in}{2.450000in}}%
\pgfpathlineto{\pgfqpoint{1.643560in}{2.240000in}}%
\pgfpathlineto{\pgfqpoint{1.647280in}{2.730000in}}%
\pgfpathlineto{\pgfqpoint{1.649760in}{2.345000in}}%
\pgfpathlineto{\pgfqpoint{1.651000in}{2.835000in}}%
\pgfpathlineto{\pgfqpoint{1.652240in}{2.415000in}}%
\pgfpathlineto{\pgfqpoint{1.653480in}{2.520000in}}%
\pgfpathlineto{\pgfqpoint{1.654720in}{2.310000in}}%
\pgfpathlineto{\pgfqpoint{1.655960in}{2.380000in}}%
\pgfpathlineto{\pgfqpoint{1.657200in}{2.730000in}}%
\pgfpathlineto{\pgfqpoint{1.660920in}{2.450000in}}%
\pgfpathlineto{\pgfqpoint{1.662160in}{2.415000in}}%
\pgfpathlineto{\pgfqpoint{1.663400in}{2.555000in}}%
\pgfpathlineto{\pgfqpoint{1.664640in}{2.555000in}}%
\pgfpathlineto{\pgfqpoint{1.667120in}{2.275000in}}%
\pgfpathlineto{\pgfqpoint{1.669600in}{2.555000in}}%
\pgfpathlineto{\pgfqpoint{1.672080in}{2.415000in}}%
\pgfpathlineto{\pgfqpoint{1.673320in}{2.170000in}}%
\pgfpathlineto{\pgfqpoint{1.674560in}{2.590000in}}%
\pgfpathlineto{\pgfqpoint{1.675800in}{2.345000in}}%
\pgfpathlineto{\pgfqpoint{1.677040in}{2.450000in}}%
\pgfpathlineto{\pgfqpoint{1.678280in}{2.870000in}}%
\pgfpathlineto{\pgfqpoint{1.679520in}{2.800000in}}%
\pgfpathlineto{\pgfqpoint{1.680760in}{2.380000in}}%
\pgfpathlineto{\pgfqpoint{1.683240in}{2.590000in}}%
\pgfpathlineto{\pgfqpoint{1.684480in}{2.520000in}}%
\pgfpathlineto{\pgfqpoint{1.685720in}{2.590000in}}%
\pgfpathlineto{\pgfqpoint{1.686960in}{2.205000in}}%
\pgfpathlineto{\pgfqpoint{1.688200in}{2.170000in}}%
\pgfpathlineto{\pgfqpoint{1.689440in}{2.205000in}}%
\pgfpathlineto{\pgfqpoint{1.690680in}{2.590000in}}%
\pgfpathlineto{\pgfqpoint{1.693160in}{2.170000in}}%
\pgfpathlineto{\pgfqpoint{1.694400in}{2.205000in}}%
\pgfpathlineto{\pgfqpoint{1.695640in}{2.030000in}}%
\pgfpathlineto{\pgfqpoint{1.698120in}{2.555000in}}%
\pgfpathlineto{\pgfqpoint{1.699360in}{2.520000in}}%
\pgfpathlineto{\pgfqpoint{1.701840in}{2.030000in}}%
\pgfpathlineto{\pgfqpoint{1.703080in}{2.485000in}}%
\pgfpathlineto{\pgfqpoint{1.704320in}{2.520000in}}%
\pgfpathlineto{\pgfqpoint{1.705560in}{2.590000in}}%
\pgfpathlineto{\pgfqpoint{1.706800in}{2.170000in}}%
\pgfpathlineto{\pgfqpoint{1.708040in}{2.345000in}}%
\pgfpathlineto{\pgfqpoint{1.709280in}{2.275000in}}%
\pgfpathlineto{\pgfqpoint{1.710520in}{2.065000in}}%
\pgfpathlineto{\pgfqpoint{1.711760in}{2.730000in}}%
\pgfpathlineto{\pgfqpoint{1.713000in}{2.275000in}}%
\pgfpathlineto{\pgfqpoint{1.714240in}{2.380000in}}%
\pgfpathlineto{\pgfqpoint{1.716720in}{2.065000in}}%
\pgfpathlineto{\pgfqpoint{1.719200in}{2.520000in}}%
\pgfpathlineto{\pgfqpoint{1.720440in}{2.100000in}}%
\pgfpathlineto{\pgfqpoint{1.722920in}{2.660000in}}%
\pgfpathlineto{\pgfqpoint{1.724160in}{2.485000in}}%
\pgfpathlineto{\pgfqpoint{1.725400in}{2.730000in}}%
\pgfpathlineto{\pgfqpoint{1.726640in}{2.660000in}}%
\pgfpathlineto{\pgfqpoint{1.729120in}{2.275000in}}%
\pgfpathlineto{\pgfqpoint{1.731600in}{2.415000in}}%
\pgfpathlineto{\pgfqpoint{1.732840in}{2.415000in}}%
\pgfpathlineto{\pgfqpoint{1.734080in}{2.170000in}}%
\pgfpathlineto{\pgfqpoint{1.735320in}{2.485000in}}%
\pgfpathlineto{\pgfqpoint{1.736560in}{2.450000in}}%
\pgfpathlineto{\pgfqpoint{1.737800in}{2.485000in}}%
\pgfpathlineto{\pgfqpoint{1.739040in}{2.765000in}}%
\pgfpathlineto{\pgfqpoint{1.740280in}{2.415000in}}%
\pgfpathlineto{\pgfqpoint{1.741520in}{2.520000in}}%
\pgfpathlineto{\pgfqpoint{1.742760in}{2.065000in}}%
\pgfpathlineto{\pgfqpoint{1.745240in}{2.450000in}}%
\pgfpathlineto{\pgfqpoint{1.746480in}{2.415000in}}%
\pgfpathlineto{\pgfqpoint{1.747720in}{2.100000in}}%
\pgfpathlineto{\pgfqpoint{1.748960in}{2.520000in}}%
\pgfpathlineto{\pgfqpoint{1.750200in}{2.205000in}}%
\pgfpathlineto{\pgfqpoint{1.753920in}{2.765000in}}%
\pgfpathlineto{\pgfqpoint{1.755160in}{2.520000in}}%
\pgfpathlineto{\pgfqpoint{1.757640in}{2.765000in}}%
\pgfpathlineto{\pgfqpoint{1.758880in}{2.485000in}}%
\pgfpathlineto{\pgfqpoint{1.761360in}{2.835000in}}%
\pgfpathlineto{\pgfqpoint{1.763840in}{2.555000in}}%
\pgfpathlineto{\pgfqpoint{1.765080in}{2.555000in}}%
\pgfpathlineto{\pgfqpoint{1.767560in}{2.170000in}}%
\pgfpathlineto{\pgfqpoint{1.768800in}{2.730000in}}%
\pgfpathlineto{\pgfqpoint{1.771280in}{2.345000in}}%
\pgfpathlineto{\pgfqpoint{1.772520in}{2.660000in}}%
\pgfpathlineto{\pgfqpoint{1.773760in}{2.590000in}}%
\pgfpathlineto{\pgfqpoint{1.775000in}{2.590000in}}%
\pgfpathlineto{\pgfqpoint{1.776240in}{1.960000in}}%
\pgfpathlineto{\pgfqpoint{1.777480in}{2.555000in}}%
\pgfpathlineto{\pgfqpoint{1.778720in}{2.415000in}}%
\pgfpathlineto{\pgfqpoint{1.779960in}{2.555000in}}%
\pgfpathlineto{\pgfqpoint{1.781200in}{2.485000in}}%
\pgfpathlineto{\pgfqpoint{1.782440in}{2.730000in}}%
\pgfpathlineto{\pgfqpoint{1.783680in}{2.380000in}}%
\pgfpathlineto{\pgfqpoint{1.784920in}{2.485000in}}%
\pgfpathlineto{\pgfqpoint{1.786160in}{2.310000in}}%
\pgfpathlineto{\pgfqpoint{1.788640in}{2.590000in}}%
\pgfpathlineto{\pgfqpoint{1.791120in}{2.275000in}}%
\pgfpathlineto{\pgfqpoint{1.792360in}{2.310000in}}%
\pgfpathlineto{\pgfqpoint{1.794840in}{2.730000in}}%
\pgfpathlineto{\pgfqpoint{1.796080in}{1.995000in}}%
\pgfpathlineto{\pgfqpoint{1.797320in}{2.555000in}}%
\pgfpathlineto{\pgfqpoint{1.798560in}{2.590000in}}%
\pgfpathlineto{\pgfqpoint{1.801040in}{2.100000in}}%
\pgfpathlineto{\pgfqpoint{1.802280in}{2.135000in}}%
\pgfpathlineto{\pgfqpoint{1.803520in}{2.625000in}}%
\pgfpathlineto{\pgfqpoint{1.804760in}{2.415000in}}%
\pgfpathlineto{\pgfqpoint{1.806000in}{2.555000in}}%
\pgfpathlineto{\pgfqpoint{1.807240in}{2.170000in}}%
\pgfpathlineto{\pgfqpoint{1.808480in}{2.205000in}}%
\pgfpathlineto{\pgfqpoint{1.809720in}{2.170000in}}%
\pgfpathlineto{\pgfqpoint{1.812200in}{2.870000in}}%
\pgfpathlineto{\pgfqpoint{1.814680in}{2.310000in}}%
\pgfpathlineto{\pgfqpoint{1.815920in}{2.625000in}}%
\pgfpathlineto{\pgfqpoint{1.817160in}{2.065000in}}%
\pgfpathlineto{\pgfqpoint{1.819640in}{2.415000in}}%
\pgfpathlineto{\pgfqpoint{1.820880in}{2.380000in}}%
\pgfpathlineto{\pgfqpoint{1.822120in}{2.415000in}}%
\pgfpathlineto{\pgfqpoint{1.823360in}{2.380000in}}%
\pgfpathlineto{\pgfqpoint{1.824600in}{2.555000in}}%
\pgfpathlineto{\pgfqpoint{1.825840in}{2.310000in}}%
\pgfpathlineto{\pgfqpoint{1.827080in}{2.415000in}}%
\pgfpathlineto{\pgfqpoint{1.828320in}{2.205000in}}%
\pgfpathlineto{\pgfqpoint{1.830800in}{2.765000in}}%
\pgfpathlineto{\pgfqpoint{1.834520in}{2.240000in}}%
\pgfpathlineto{\pgfqpoint{1.835760in}{2.380000in}}%
\pgfpathlineto{\pgfqpoint{1.837000in}{2.345000in}}%
\pgfpathlineto{\pgfqpoint{1.839480in}{2.205000in}}%
\pgfpathlineto{\pgfqpoint{1.841960in}{2.380000in}}%
\pgfpathlineto{\pgfqpoint{1.843200in}{2.415000in}}%
\pgfpathlineto{\pgfqpoint{1.844440in}{2.765000in}}%
\pgfpathlineto{\pgfqpoint{1.845680in}{2.590000in}}%
\pgfpathlineto{\pgfqpoint{1.846920in}{2.135000in}}%
\pgfpathlineto{\pgfqpoint{1.848160in}{2.205000in}}%
\pgfpathlineto{\pgfqpoint{1.849400in}{2.590000in}}%
\pgfpathlineto{\pgfqpoint{1.850640in}{2.380000in}}%
\pgfpathlineto{\pgfqpoint{1.851880in}{3.045000in}}%
\pgfpathlineto{\pgfqpoint{1.853120in}{2.310000in}}%
\pgfpathlineto{\pgfqpoint{1.854360in}{2.415000in}}%
\pgfpathlineto{\pgfqpoint{1.855600in}{2.065000in}}%
\pgfpathlineto{\pgfqpoint{1.856840in}{2.275000in}}%
\pgfpathlineto{\pgfqpoint{1.858080in}{2.100000in}}%
\pgfpathlineto{\pgfqpoint{1.859320in}{2.485000in}}%
\pgfpathlineto{\pgfqpoint{1.860560in}{2.310000in}}%
\pgfpathlineto{\pgfqpoint{1.861800in}{1.960000in}}%
\pgfpathlineto{\pgfqpoint{1.863040in}{2.310000in}}%
\pgfpathlineto{\pgfqpoint{1.864280in}{2.240000in}}%
\pgfpathlineto{\pgfqpoint{1.865520in}{2.590000in}}%
\pgfpathlineto{\pgfqpoint{1.866760in}{2.170000in}}%
\pgfpathlineto{\pgfqpoint{1.869240in}{2.450000in}}%
\pgfpathlineto{\pgfqpoint{1.871720in}{2.135000in}}%
\pgfpathlineto{\pgfqpoint{1.872960in}{2.170000in}}%
\pgfpathlineto{\pgfqpoint{1.874200in}{1.960000in}}%
\pgfpathlineto{\pgfqpoint{1.875440in}{2.310000in}}%
\pgfpathlineto{\pgfqpoint{1.876680in}{1.925000in}}%
\pgfpathlineto{\pgfqpoint{1.877920in}{1.925000in}}%
\pgfpathlineto{\pgfqpoint{1.879160in}{2.450000in}}%
\pgfpathlineto{\pgfqpoint{1.881640in}{2.100000in}}%
\pgfpathlineto{\pgfqpoint{1.882880in}{2.380000in}}%
\pgfpathlineto{\pgfqpoint{1.887840in}{2.065000in}}%
\pgfpathlineto{\pgfqpoint{1.889080in}{2.275000in}}%
\pgfpathlineto{\pgfqpoint{1.890320in}{2.205000in}}%
\pgfpathlineto{\pgfqpoint{1.891560in}{2.345000in}}%
\pgfpathlineto{\pgfqpoint{1.894040in}{2.135000in}}%
\pgfpathlineto{\pgfqpoint{1.895280in}{2.205000in}}%
\pgfpathlineto{\pgfqpoint{1.896520in}{2.205000in}}%
\pgfpathlineto{\pgfqpoint{1.899000in}{1.855000in}}%
\pgfpathlineto{\pgfqpoint{1.900240in}{2.485000in}}%
\pgfpathlineto{\pgfqpoint{1.901480in}{2.450000in}}%
\pgfpathlineto{\pgfqpoint{1.902720in}{2.520000in}}%
\pgfpathlineto{\pgfqpoint{1.906440in}{2.100000in}}%
\pgfpathlineto{\pgfqpoint{1.907680in}{2.415000in}}%
\pgfpathlineto{\pgfqpoint{1.908920in}{2.205000in}}%
\pgfpathlineto{\pgfqpoint{1.910160in}{2.625000in}}%
\pgfpathlineto{\pgfqpoint{1.911400in}{2.345000in}}%
\pgfpathlineto{\pgfqpoint{1.912640in}{2.380000in}}%
\pgfpathlineto{\pgfqpoint{1.913880in}{2.660000in}}%
\pgfpathlineto{\pgfqpoint{1.915120in}{2.555000in}}%
\pgfpathlineto{\pgfqpoint{1.916360in}{2.345000in}}%
\pgfpathlineto{\pgfqpoint{1.917600in}{2.485000in}}%
\pgfpathlineto{\pgfqpoint{1.920080in}{2.205000in}}%
\pgfpathlineto{\pgfqpoint{1.921320in}{2.275000in}}%
\pgfpathlineto{\pgfqpoint{1.922560in}{2.415000in}}%
\pgfpathlineto{\pgfqpoint{1.925040in}{2.065000in}}%
\pgfpathlineto{\pgfqpoint{1.926280in}{2.030000in}}%
\pgfpathlineto{\pgfqpoint{1.927520in}{1.960000in}}%
\pgfpathlineto{\pgfqpoint{1.930000in}{2.205000in}}%
\pgfpathlineto{\pgfqpoint{1.931240in}{2.730000in}}%
\pgfpathlineto{\pgfqpoint{1.932480in}{2.625000in}}%
\pgfpathlineto{\pgfqpoint{1.933720in}{2.660000in}}%
\pgfpathlineto{\pgfqpoint{1.934960in}{2.625000in}}%
\pgfpathlineto{\pgfqpoint{1.936200in}{2.555000in}}%
\pgfpathlineto{\pgfqpoint{1.937440in}{2.310000in}}%
\pgfpathlineto{\pgfqpoint{1.938680in}{2.660000in}}%
\pgfpathlineto{\pgfqpoint{1.939920in}{2.625000in}}%
\pgfpathlineto{\pgfqpoint{1.941160in}{2.275000in}}%
\pgfpathlineto{\pgfqpoint{1.942400in}{2.275000in}}%
\pgfpathlineto{\pgfqpoint{1.943640in}{2.625000in}}%
\pgfpathlineto{\pgfqpoint{1.944880in}{2.450000in}}%
\pgfpathlineto{\pgfqpoint{1.946120in}{2.520000in}}%
\pgfpathlineto{\pgfqpoint{1.947360in}{2.485000in}}%
\pgfpathlineto{\pgfqpoint{1.948600in}{2.415000in}}%
\pgfpathlineto{\pgfqpoint{1.949840in}{2.065000in}}%
\pgfpathlineto{\pgfqpoint{1.951080in}{2.450000in}}%
\pgfpathlineto{\pgfqpoint{1.952320in}{2.415000in}}%
\pgfpathlineto{\pgfqpoint{1.953560in}{2.310000in}}%
\pgfpathlineto{\pgfqpoint{1.956040in}{2.555000in}}%
\pgfpathlineto{\pgfqpoint{1.958520in}{2.135000in}}%
\pgfpathlineto{\pgfqpoint{1.959760in}{2.135000in}}%
\pgfpathlineto{\pgfqpoint{1.961000in}{2.415000in}}%
\pgfpathlineto{\pgfqpoint{1.962240in}{2.030000in}}%
\pgfpathlineto{\pgfqpoint{1.963480in}{2.205000in}}%
\pgfpathlineto{\pgfqpoint{1.964720in}{2.870000in}}%
\pgfpathlineto{\pgfqpoint{1.967200in}{2.380000in}}%
\pgfpathlineto{\pgfqpoint{1.968440in}{2.415000in}}%
\pgfpathlineto{\pgfqpoint{1.969680in}{2.170000in}}%
\pgfpathlineto{\pgfqpoint{1.970920in}{2.380000in}}%
\pgfpathlineto{\pgfqpoint{1.973400in}{2.170000in}}%
\pgfpathlineto{\pgfqpoint{1.975880in}{2.625000in}}%
\pgfpathlineto{\pgfqpoint{1.977120in}{2.380000in}}%
\pgfpathlineto{\pgfqpoint{1.978360in}{2.590000in}}%
\pgfpathlineto{\pgfqpoint{1.979600in}{2.380000in}}%
\pgfpathlineto{\pgfqpoint{1.982080in}{2.660000in}}%
\pgfpathlineto{\pgfqpoint{1.983320in}{2.660000in}}%
\pgfpathlineto{\pgfqpoint{1.984560in}{2.590000in}}%
\pgfpathlineto{\pgfqpoint{1.985800in}{2.450000in}}%
\pgfpathlineto{\pgfqpoint{1.990760in}{2.450000in}}%
\pgfpathlineto{\pgfqpoint{1.992000in}{2.380000in}}%
\pgfpathlineto{\pgfqpoint{1.994480in}{2.625000in}}%
\pgfpathlineto{\pgfqpoint{1.995720in}{2.695000in}}%
\pgfpathlineto{\pgfqpoint{1.996960in}{2.310000in}}%
\pgfpathlineto{\pgfqpoint{1.998200in}{2.310000in}}%
\pgfpathlineto{\pgfqpoint{1.999440in}{2.030000in}}%
\pgfpathlineto{\pgfqpoint{2.000680in}{2.660000in}}%
\pgfpathlineto{\pgfqpoint{2.001920in}{2.450000in}}%
\pgfpathlineto{\pgfqpoint{2.003160in}{2.765000in}}%
\pgfpathlineto{\pgfqpoint{2.004400in}{2.450000in}}%
\pgfpathlineto{\pgfqpoint{2.005640in}{2.660000in}}%
\pgfpathlineto{\pgfqpoint{2.006880in}{2.170000in}}%
\pgfpathlineto{\pgfqpoint{2.008120in}{2.170000in}}%
\pgfpathlineto{\pgfqpoint{2.009360in}{2.310000in}}%
\pgfpathlineto{\pgfqpoint{2.010600in}{2.170000in}}%
\pgfpathlineto{\pgfqpoint{2.011840in}{2.485000in}}%
\pgfpathlineto{\pgfqpoint{2.013080in}{2.310000in}}%
\pgfpathlineto{\pgfqpoint{2.014320in}{2.555000in}}%
\pgfpathlineto{\pgfqpoint{2.016800in}{2.100000in}}%
\pgfpathlineto{\pgfqpoint{2.018040in}{2.520000in}}%
\pgfpathlineto{\pgfqpoint{2.019280in}{2.135000in}}%
\pgfpathlineto{\pgfqpoint{2.020520in}{2.205000in}}%
\pgfpathlineto{\pgfqpoint{2.021760in}{2.345000in}}%
\pgfpathlineto{\pgfqpoint{2.023000in}{2.345000in}}%
\pgfpathlineto{\pgfqpoint{2.024240in}{2.205000in}}%
\pgfpathlineto{\pgfqpoint{2.025480in}{2.590000in}}%
\pgfpathlineto{\pgfqpoint{2.026720in}{2.415000in}}%
\pgfpathlineto{\pgfqpoint{2.027960in}{2.450000in}}%
\pgfpathlineto{\pgfqpoint{2.029200in}{2.555000in}}%
\pgfpathlineto{\pgfqpoint{2.030440in}{2.415000in}}%
\pgfpathlineto{\pgfqpoint{2.031680in}{2.520000in}}%
\pgfpathlineto{\pgfqpoint{2.032920in}{2.345000in}}%
\pgfpathlineto{\pgfqpoint{2.034160in}{2.555000in}}%
\pgfpathlineto{\pgfqpoint{2.035400in}{2.240000in}}%
\pgfpathlineto{\pgfqpoint{2.036640in}{2.450000in}}%
\pgfpathlineto{\pgfqpoint{2.039120in}{2.205000in}}%
\pgfpathlineto{\pgfqpoint{2.040360in}{2.310000in}}%
\pgfpathlineto{\pgfqpoint{2.041600in}{2.520000in}}%
\pgfpathlineto{\pgfqpoint{2.042840in}{2.345000in}}%
\pgfpathlineto{\pgfqpoint{2.044080in}{2.380000in}}%
\pgfpathlineto{\pgfqpoint{2.045320in}{2.380000in}}%
\pgfpathlineto{\pgfqpoint{2.046560in}{2.345000in}}%
\pgfpathlineto{\pgfqpoint{2.047800in}{2.555000in}}%
\pgfpathlineto{\pgfqpoint{2.049040in}{2.415000in}}%
\pgfpathlineto{\pgfqpoint{2.050280in}{2.555000in}}%
\pgfpathlineto{\pgfqpoint{2.051520in}{2.030000in}}%
\pgfpathlineto{\pgfqpoint{2.055240in}{2.555000in}}%
\pgfpathlineto{\pgfqpoint{2.057720in}{2.135000in}}%
\pgfpathlineto{\pgfqpoint{2.061440in}{2.520000in}}%
\pgfpathlineto{\pgfqpoint{2.062680in}{2.835000in}}%
\pgfpathlineto{\pgfqpoint{2.063920in}{2.275000in}}%
\pgfpathlineto{\pgfqpoint{2.066400in}{2.450000in}}%
\pgfpathlineto{\pgfqpoint{2.067640in}{1.995000in}}%
\pgfpathlineto{\pgfqpoint{2.070120in}{2.485000in}}%
\pgfpathlineto{\pgfqpoint{2.071360in}{2.065000in}}%
\pgfpathlineto{\pgfqpoint{2.073840in}{2.555000in}}%
\pgfpathlineto{\pgfqpoint{2.075080in}{2.590000in}}%
\pgfpathlineto{\pgfqpoint{2.076320in}{2.415000in}}%
\pgfpathlineto{\pgfqpoint{2.077560in}{2.415000in}}%
\pgfpathlineto{\pgfqpoint{2.078800in}{2.590000in}}%
\pgfpathlineto{\pgfqpoint{2.080040in}{2.520000in}}%
\pgfpathlineto{\pgfqpoint{2.081280in}{2.275000in}}%
\pgfpathlineto{\pgfqpoint{2.082520in}{2.380000in}}%
\pgfpathlineto{\pgfqpoint{2.083760in}{2.695000in}}%
\pgfpathlineto{\pgfqpoint{2.085000in}{2.240000in}}%
\pgfpathlineto{\pgfqpoint{2.086240in}{2.800000in}}%
\pgfpathlineto{\pgfqpoint{2.087480in}{2.660000in}}%
\pgfpathlineto{\pgfqpoint{2.088720in}{2.205000in}}%
\pgfpathlineto{\pgfqpoint{2.089960in}{2.240000in}}%
\pgfpathlineto{\pgfqpoint{2.091200in}{2.590000in}}%
\pgfpathlineto{\pgfqpoint{2.092440in}{2.555000in}}%
\pgfpathlineto{\pgfqpoint{2.093680in}{2.695000in}}%
\pgfpathlineto{\pgfqpoint{2.094920in}{2.695000in}}%
\pgfpathlineto{\pgfqpoint{2.096160in}{2.135000in}}%
\pgfpathlineto{\pgfqpoint{2.098640in}{2.450000in}}%
\pgfpathlineto{\pgfqpoint{2.099880in}{2.555000in}}%
\pgfpathlineto{\pgfqpoint{2.101120in}{2.275000in}}%
\pgfpathlineto{\pgfqpoint{2.102360in}{2.660000in}}%
\pgfpathlineto{\pgfqpoint{2.103600in}{2.415000in}}%
\pgfpathlineto{\pgfqpoint{2.104840in}{2.590000in}}%
\pgfpathlineto{\pgfqpoint{2.106080in}{2.415000in}}%
\pgfpathlineto{\pgfqpoint{2.107320in}{2.415000in}}%
\pgfpathlineto{\pgfqpoint{2.108560in}{2.275000in}}%
\pgfpathlineto{\pgfqpoint{2.109800in}{2.520000in}}%
\pgfpathlineto{\pgfqpoint{2.111040in}{2.415000in}}%
\pgfpathlineto{\pgfqpoint{2.112280in}{1.995000in}}%
\pgfpathlineto{\pgfqpoint{2.114760in}{2.520000in}}%
\pgfpathlineto{\pgfqpoint{2.118480in}{2.345000in}}%
\pgfpathlineto{\pgfqpoint{2.119720in}{2.415000in}}%
\pgfpathlineto{\pgfqpoint{2.120960in}{2.415000in}}%
\pgfpathlineto{\pgfqpoint{2.122200in}{2.660000in}}%
\pgfpathlineto{\pgfqpoint{2.123440in}{2.555000in}}%
\pgfpathlineto{\pgfqpoint{2.124680in}{2.345000in}}%
\pgfpathlineto{\pgfqpoint{2.125920in}{2.590000in}}%
\pgfpathlineto{\pgfqpoint{2.127160in}{2.240000in}}%
\pgfpathlineto{\pgfqpoint{2.128400in}{2.275000in}}%
\pgfpathlineto{\pgfqpoint{2.129640in}{2.240000in}}%
\pgfpathlineto{\pgfqpoint{2.130880in}{2.030000in}}%
\pgfpathlineto{\pgfqpoint{2.133360in}{2.555000in}}%
\pgfpathlineto{\pgfqpoint{2.134600in}{2.310000in}}%
\pgfpathlineto{\pgfqpoint{2.135840in}{2.625000in}}%
\pgfpathlineto{\pgfqpoint{2.138320in}{2.275000in}}%
\pgfpathlineto{\pgfqpoint{2.140800in}{2.800000in}}%
\pgfpathlineto{\pgfqpoint{2.142040in}{2.065000in}}%
\pgfpathlineto{\pgfqpoint{2.143280in}{2.030000in}}%
\pgfpathlineto{\pgfqpoint{2.144520in}{2.415000in}}%
\pgfpathlineto{\pgfqpoint{2.145760in}{2.345000in}}%
\pgfpathlineto{\pgfqpoint{2.147000in}{2.415000in}}%
\pgfpathlineto{\pgfqpoint{2.148240in}{2.590000in}}%
\pgfpathlineto{\pgfqpoint{2.149480in}{2.520000in}}%
\pgfpathlineto{\pgfqpoint{2.150720in}{2.275000in}}%
\pgfpathlineto{\pgfqpoint{2.153200in}{2.590000in}}%
\pgfpathlineto{\pgfqpoint{2.154440in}{2.800000in}}%
\pgfpathlineto{\pgfqpoint{2.155680in}{2.555000in}}%
\pgfpathlineto{\pgfqpoint{2.156920in}{2.660000in}}%
\pgfpathlineto{\pgfqpoint{2.158160in}{2.450000in}}%
\pgfpathlineto{\pgfqpoint{2.159400in}{2.555000in}}%
\pgfpathlineto{\pgfqpoint{2.160640in}{2.310000in}}%
\pgfpathlineto{\pgfqpoint{2.161880in}{2.415000in}}%
\pgfpathlineto{\pgfqpoint{2.163120in}{2.415000in}}%
\pgfpathlineto{\pgfqpoint{2.164360in}{2.800000in}}%
\pgfpathlineto{\pgfqpoint{2.165600in}{2.695000in}}%
\pgfpathlineto{\pgfqpoint{2.166840in}{2.695000in}}%
\pgfpathlineto{\pgfqpoint{2.170560in}{2.310000in}}%
\pgfpathlineto{\pgfqpoint{2.171800in}{2.870000in}}%
\pgfpathlineto{\pgfqpoint{2.173040in}{2.800000in}}%
\pgfpathlineto{\pgfqpoint{2.174280in}{2.555000in}}%
\pgfpathlineto{\pgfqpoint{2.175520in}{2.870000in}}%
\pgfpathlineto{\pgfqpoint{2.176760in}{2.765000in}}%
\pgfpathlineto{\pgfqpoint{2.178000in}{2.765000in}}%
\pgfpathlineto{\pgfqpoint{2.179240in}{2.660000in}}%
\pgfpathlineto{\pgfqpoint{2.180480in}{2.730000in}}%
\pgfpathlineto{\pgfqpoint{2.181720in}{2.660000in}}%
\pgfpathlineto{\pgfqpoint{2.184200in}{2.030000in}}%
\pgfpathlineto{\pgfqpoint{2.185440in}{2.555000in}}%
\pgfpathlineto{\pgfqpoint{2.186680in}{2.555000in}}%
\pgfpathlineto{\pgfqpoint{2.187920in}{2.240000in}}%
\pgfpathlineto{\pgfqpoint{2.189160in}{2.310000in}}%
\pgfpathlineto{\pgfqpoint{2.190400in}{2.730000in}}%
\pgfpathlineto{\pgfqpoint{2.192880in}{2.275000in}}%
\pgfpathlineto{\pgfqpoint{2.196600in}{2.730000in}}%
\pgfpathlineto{\pgfqpoint{2.199080in}{2.205000in}}%
\pgfpathlineto{\pgfqpoint{2.201560in}{2.695000in}}%
\pgfpathlineto{\pgfqpoint{2.202800in}{2.660000in}}%
\pgfpathlineto{\pgfqpoint{2.204040in}{2.765000in}}%
\pgfpathlineto{\pgfqpoint{2.206520in}{2.275000in}}%
\pgfpathlineto{\pgfqpoint{2.207760in}{2.520000in}}%
\pgfpathlineto{\pgfqpoint{2.209000in}{2.415000in}}%
\pgfpathlineto{\pgfqpoint{2.210240in}{2.135000in}}%
\pgfpathlineto{\pgfqpoint{2.212720in}{2.345000in}}%
\pgfpathlineto{\pgfqpoint{2.213960in}{2.240000in}}%
\pgfpathlineto{\pgfqpoint{2.217680in}{2.625000in}}%
\pgfpathlineto{\pgfqpoint{2.218920in}{2.345000in}}%
\pgfpathlineto{\pgfqpoint{2.220160in}{2.555000in}}%
\pgfpathlineto{\pgfqpoint{2.222640in}{2.275000in}}%
\pgfpathlineto{\pgfqpoint{2.223880in}{2.380000in}}%
\pgfpathlineto{\pgfqpoint{2.225120in}{2.345000in}}%
\pgfpathlineto{\pgfqpoint{2.226360in}{2.380000in}}%
\pgfpathlineto{\pgfqpoint{2.227600in}{2.100000in}}%
\pgfpathlineto{\pgfqpoint{2.228840in}{2.240000in}}%
\pgfpathlineto{\pgfqpoint{2.230080in}{2.905000in}}%
\pgfpathlineto{\pgfqpoint{2.232560in}{2.310000in}}%
\pgfpathlineto{\pgfqpoint{2.233800in}{2.485000in}}%
\pgfpathlineto{\pgfqpoint{2.235040in}{2.975000in}}%
\pgfpathlineto{\pgfqpoint{2.237520in}{2.450000in}}%
\pgfpathlineto{\pgfqpoint{2.238760in}{2.765000in}}%
\pgfpathlineto{\pgfqpoint{2.240000in}{1.960000in}}%
\pgfpathlineto{\pgfqpoint{2.242480in}{2.520000in}}%
\pgfpathlineto{\pgfqpoint{2.244960in}{2.765000in}}%
\pgfpathlineto{\pgfqpoint{2.247440in}{2.345000in}}%
\pgfpathlineto{\pgfqpoint{2.248680in}{2.345000in}}%
\pgfpathlineto{\pgfqpoint{2.249920in}{2.275000in}}%
\pgfpathlineto{\pgfqpoint{2.251160in}{2.765000in}}%
\pgfpathlineto{\pgfqpoint{2.252400in}{2.485000in}}%
\pgfpathlineto{\pgfqpoint{2.253640in}{2.485000in}}%
\pgfpathlineto{\pgfqpoint{2.254880in}{2.030000in}}%
\pgfpathlineto{\pgfqpoint{2.257360in}{2.765000in}}%
\pgfpathlineto{\pgfqpoint{2.259840in}{2.170000in}}%
\pgfpathlineto{\pgfqpoint{2.261080in}{2.800000in}}%
\pgfpathlineto{\pgfqpoint{2.263560in}{2.415000in}}%
\pgfpathlineto{\pgfqpoint{2.264800in}{2.800000in}}%
\pgfpathlineto{\pgfqpoint{2.266040in}{2.345000in}}%
\pgfpathlineto{\pgfqpoint{2.269760in}{2.590000in}}%
\pgfpathlineto{\pgfqpoint{2.271000in}{2.380000in}}%
\pgfpathlineto{\pgfqpoint{2.272240in}{2.520000in}}%
\pgfpathlineto{\pgfqpoint{2.273480in}{2.520000in}}%
\pgfpathlineto{\pgfqpoint{2.274720in}{2.695000in}}%
\pgfpathlineto{\pgfqpoint{2.275960in}{2.520000in}}%
\pgfpathlineto{\pgfqpoint{2.277200in}{2.590000in}}%
\pgfpathlineto{\pgfqpoint{2.279680in}{2.345000in}}%
\pgfpathlineto{\pgfqpoint{2.280920in}{2.590000in}}%
\pgfpathlineto{\pgfqpoint{2.282160in}{2.590000in}}%
\pgfpathlineto{\pgfqpoint{2.283400in}{2.100000in}}%
\pgfpathlineto{\pgfqpoint{2.285880in}{2.765000in}}%
\pgfpathlineto{\pgfqpoint{2.287120in}{2.555000in}}%
\pgfpathlineto{\pgfqpoint{2.289600in}{2.555000in}}%
\pgfpathlineto{\pgfqpoint{2.290840in}{2.800000in}}%
\pgfpathlineto{\pgfqpoint{2.292080in}{2.100000in}}%
\pgfpathlineto{\pgfqpoint{2.294560in}{2.835000in}}%
\pgfpathlineto{\pgfqpoint{2.295800in}{2.380000in}}%
\pgfpathlineto{\pgfqpoint{2.297040in}{2.380000in}}%
\pgfpathlineto{\pgfqpoint{2.298280in}{2.415000in}}%
\pgfpathlineto{\pgfqpoint{2.299520in}{2.205000in}}%
\pgfpathlineto{\pgfqpoint{2.300760in}{2.625000in}}%
\pgfpathlineto{\pgfqpoint{2.302000in}{2.170000in}}%
\pgfpathlineto{\pgfqpoint{2.304480in}{3.045000in}}%
\pgfpathlineto{\pgfqpoint{2.306960in}{2.415000in}}%
\pgfpathlineto{\pgfqpoint{2.308200in}{2.345000in}}%
\pgfpathlineto{\pgfqpoint{2.309440in}{2.345000in}}%
\pgfpathlineto{\pgfqpoint{2.310680in}{2.380000in}}%
\pgfpathlineto{\pgfqpoint{2.311920in}{2.380000in}}%
\pgfpathlineto{\pgfqpoint{2.313160in}{2.450000in}}%
\pgfpathlineto{\pgfqpoint{2.314400in}{1.960000in}}%
\pgfpathlineto{\pgfqpoint{2.315640in}{2.310000in}}%
\pgfpathlineto{\pgfqpoint{2.316880in}{2.310000in}}%
\pgfpathlineto{\pgfqpoint{2.318120in}{2.485000in}}%
\pgfpathlineto{\pgfqpoint{2.319360in}{2.415000in}}%
\pgfpathlineto{\pgfqpoint{2.320600in}{2.135000in}}%
\pgfpathlineto{\pgfqpoint{2.321840in}{2.625000in}}%
\pgfpathlineto{\pgfqpoint{2.323080in}{2.590000in}}%
\pgfpathlineto{\pgfqpoint{2.324320in}{2.380000in}}%
\pgfpathlineto{\pgfqpoint{2.326800in}{2.590000in}}%
\pgfpathlineto{\pgfqpoint{2.329280in}{2.310000in}}%
\pgfpathlineto{\pgfqpoint{2.330520in}{2.485000in}}%
\pgfpathlineto{\pgfqpoint{2.331760in}{2.240000in}}%
\pgfpathlineto{\pgfqpoint{2.333000in}{2.485000in}}%
\pgfpathlineto{\pgfqpoint{2.334240in}{2.415000in}}%
\pgfpathlineto{\pgfqpoint{2.335480in}{2.485000in}}%
\pgfpathlineto{\pgfqpoint{2.336720in}{2.450000in}}%
\pgfpathlineto{\pgfqpoint{2.337960in}{2.135000in}}%
\pgfpathlineto{\pgfqpoint{2.339200in}{2.380000in}}%
\pgfpathlineto{\pgfqpoint{2.340440in}{2.345000in}}%
\pgfpathlineto{\pgfqpoint{2.342920in}{2.065000in}}%
\pgfpathlineto{\pgfqpoint{2.344160in}{2.450000in}}%
\pgfpathlineto{\pgfqpoint{2.345400in}{2.205000in}}%
\pgfpathlineto{\pgfqpoint{2.346640in}{2.555000in}}%
\pgfpathlineto{\pgfqpoint{2.347880in}{2.450000in}}%
\pgfpathlineto{\pgfqpoint{2.349120in}{2.240000in}}%
\pgfpathlineto{\pgfqpoint{2.351600in}{2.590000in}}%
\pgfpathlineto{\pgfqpoint{2.352840in}{2.450000in}}%
\pgfpathlineto{\pgfqpoint{2.354080in}{2.695000in}}%
\pgfpathlineto{\pgfqpoint{2.356560in}{2.275000in}}%
\pgfpathlineto{\pgfqpoint{2.359040in}{2.450000in}}%
\pgfpathlineto{\pgfqpoint{2.360280in}{2.275000in}}%
\pgfpathlineto{\pgfqpoint{2.361520in}{2.275000in}}%
\pgfpathlineto{\pgfqpoint{2.362760in}{2.625000in}}%
\pgfpathlineto{\pgfqpoint{2.364000in}{2.275000in}}%
\pgfpathlineto{\pgfqpoint{2.365240in}{2.275000in}}%
\pgfpathlineto{\pgfqpoint{2.366480in}{2.660000in}}%
\pgfpathlineto{\pgfqpoint{2.368960in}{2.310000in}}%
\pgfpathlineto{\pgfqpoint{2.370200in}{2.660000in}}%
\pgfpathlineto{\pgfqpoint{2.371440in}{2.345000in}}%
\pgfpathlineto{\pgfqpoint{2.372680in}{2.450000in}}%
\pgfpathlineto{\pgfqpoint{2.375160in}{2.275000in}}%
\pgfpathlineto{\pgfqpoint{2.377640in}{2.555000in}}%
\pgfpathlineto{\pgfqpoint{2.380120in}{2.310000in}}%
\pgfpathlineto{\pgfqpoint{2.381360in}{2.275000in}}%
\pgfpathlineto{\pgfqpoint{2.382600in}{2.170000in}}%
\pgfpathlineto{\pgfqpoint{2.385080in}{2.555000in}}%
\pgfpathlineto{\pgfqpoint{2.386320in}{2.590000in}}%
\pgfpathlineto{\pgfqpoint{2.388800in}{2.170000in}}%
\pgfpathlineto{\pgfqpoint{2.392520in}{2.590000in}}%
\pgfpathlineto{\pgfqpoint{2.395000in}{2.240000in}}%
\pgfpathlineto{\pgfqpoint{2.396240in}{2.415000in}}%
\pgfpathlineto{\pgfqpoint{2.397480in}{2.205000in}}%
\pgfpathlineto{\pgfqpoint{2.398720in}{2.520000in}}%
\pgfpathlineto{\pgfqpoint{2.399960in}{2.520000in}}%
\pgfpathlineto{\pgfqpoint{2.401200in}{2.345000in}}%
\pgfpathlineto{\pgfqpoint{2.402440in}{2.415000in}}%
\pgfpathlineto{\pgfqpoint{2.403680in}{2.345000in}}%
\pgfpathlineto{\pgfqpoint{2.404920in}{2.415000in}}%
\pgfpathlineto{\pgfqpoint{2.406160in}{2.800000in}}%
\pgfpathlineto{\pgfqpoint{2.408640in}{2.275000in}}%
\pgfpathlineto{\pgfqpoint{2.411120in}{2.380000in}}%
\pgfpathlineto{\pgfqpoint{2.412360in}{2.310000in}}%
\pgfpathlineto{\pgfqpoint{2.413600in}{2.310000in}}%
\pgfpathlineto{\pgfqpoint{2.414840in}{2.555000in}}%
\pgfpathlineto{\pgfqpoint{2.416080in}{2.520000in}}%
\pgfpathlineto{\pgfqpoint{2.417320in}{2.170000in}}%
\pgfpathlineto{\pgfqpoint{2.418560in}{2.730000in}}%
\pgfpathlineto{\pgfqpoint{2.419800in}{2.730000in}}%
\pgfpathlineto{\pgfqpoint{2.421040in}{2.450000in}}%
\pgfpathlineto{\pgfqpoint{2.422280in}{2.555000in}}%
\pgfpathlineto{\pgfqpoint{2.424760in}{2.240000in}}%
\pgfpathlineto{\pgfqpoint{2.426000in}{2.345000in}}%
\pgfpathlineto{\pgfqpoint{2.427240in}{2.660000in}}%
\pgfpathlineto{\pgfqpoint{2.429720in}{2.380000in}}%
\pgfpathlineto{\pgfqpoint{2.430960in}{2.695000in}}%
\pgfpathlineto{\pgfqpoint{2.432200in}{2.135000in}}%
\pgfpathlineto{\pgfqpoint{2.433440in}{2.625000in}}%
\pgfpathlineto{\pgfqpoint{2.434680in}{2.100000in}}%
\pgfpathlineto{\pgfqpoint{2.435920in}{2.380000in}}%
\pgfpathlineto{\pgfqpoint{2.440880in}{1.890000in}}%
\pgfpathlineto{\pgfqpoint{2.443360in}{2.275000in}}%
\pgfpathlineto{\pgfqpoint{2.444600in}{2.100000in}}%
\pgfpathlineto{\pgfqpoint{2.445840in}{2.450000in}}%
\pgfpathlineto{\pgfqpoint{2.447080in}{2.450000in}}%
\pgfpathlineto{\pgfqpoint{2.449560in}{1.995000in}}%
\pgfpathlineto{\pgfqpoint{2.450800in}{2.730000in}}%
\pgfpathlineto{\pgfqpoint{2.452040in}{2.345000in}}%
\pgfpathlineto{\pgfqpoint{2.453280in}{2.380000in}}%
\pgfpathlineto{\pgfqpoint{2.454520in}{2.275000in}}%
\pgfpathlineto{\pgfqpoint{2.457000in}{2.800000in}}%
\pgfpathlineto{\pgfqpoint{2.459480in}{2.310000in}}%
\pgfpathlineto{\pgfqpoint{2.461960in}{2.660000in}}%
\pgfpathlineto{\pgfqpoint{2.463200in}{2.660000in}}%
\pgfpathlineto{\pgfqpoint{2.464440in}{2.765000in}}%
\pgfpathlineto{\pgfqpoint{2.465680in}{2.485000in}}%
\pgfpathlineto{\pgfqpoint{2.466920in}{2.695000in}}%
\pgfpathlineto{\pgfqpoint{2.468160in}{2.240000in}}%
\pgfpathlineto{\pgfqpoint{2.470640in}{2.695000in}}%
\pgfpathlineto{\pgfqpoint{2.473120in}{2.240000in}}%
\pgfpathlineto{\pgfqpoint{2.474360in}{2.765000in}}%
\pgfpathlineto{\pgfqpoint{2.476840in}{2.030000in}}%
\pgfpathlineto{\pgfqpoint{2.478080in}{2.170000in}}%
\pgfpathlineto{\pgfqpoint{2.479320in}{2.485000in}}%
\pgfpathlineto{\pgfqpoint{2.480560in}{2.275000in}}%
\pgfpathlineto{\pgfqpoint{2.483040in}{2.485000in}}%
\pgfpathlineto{\pgfqpoint{2.484280in}{2.240000in}}%
\pgfpathlineto{\pgfqpoint{2.485520in}{2.345000in}}%
\pgfpathlineto{\pgfqpoint{2.486760in}{2.625000in}}%
\pgfpathlineto{\pgfqpoint{2.488000in}{2.625000in}}%
\pgfpathlineto{\pgfqpoint{2.489240in}{2.450000in}}%
\pgfpathlineto{\pgfqpoint{2.490480in}{2.590000in}}%
\pgfpathlineto{\pgfqpoint{2.491720in}{2.555000in}}%
\pgfpathlineto{\pgfqpoint{2.495440in}{1.925000in}}%
\pgfpathlineto{\pgfqpoint{2.496680in}{2.590000in}}%
\pgfpathlineto{\pgfqpoint{2.497920in}{2.205000in}}%
\pgfpathlineto{\pgfqpoint{2.499160in}{2.660000in}}%
\pgfpathlineto{\pgfqpoint{2.502880in}{1.925000in}}%
\pgfpathlineto{\pgfqpoint{2.505360in}{2.520000in}}%
\pgfpathlineto{\pgfqpoint{2.506600in}{2.415000in}}%
\pgfpathlineto{\pgfqpoint{2.507840in}{2.205000in}}%
\pgfpathlineto{\pgfqpoint{2.509080in}{2.730000in}}%
\pgfpathlineto{\pgfqpoint{2.510320in}{2.625000in}}%
\pgfpathlineto{\pgfqpoint{2.511560in}{2.310000in}}%
\pgfpathlineto{\pgfqpoint{2.512800in}{2.485000in}}%
\pgfpathlineto{\pgfqpoint{2.514040in}{1.890000in}}%
\pgfpathlineto{\pgfqpoint{2.516520in}{2.415000in}}%
\pgfpathlineto{\pgfqpoint{2.517760in}{2.450000in}}%
\pgfpathlineto{\pgfqpoint{2.519000in}{2.240000in}}%
\pgfpathlineto{\pgfqpoint{2.521480in}{2.765000in}}%
\pgfpathlineto{\pgfqpoint{2.522720in}{2.450000in}}%
\pgfpathlineto{\pgfqpoint{2.523960in}{2.450000in}}%
\pgfpathlineto{\pgfqpoint{2.527680in}{1.995000in}}%
\pgfpathlineto{\pgfqpoint{2.528920in}{2.625000in}}%
\pgfpathlineto{\pgfqpoint{2.530160in}{2.485000in}}%
\pgfpathlineto{\pgfqpoint{2.531400in}{2.520000in}}%
\pgfpathlineto{\pgfqpoint{2.532640in}{2.310000in}}%
\pgfpathlineto{\pgfqpoint{2.535120in}{2.520000in}}%
\pgfpathlineto{\pgfqpoint{2.536360in}{2.310000in}}%
\pgfpathlineto{\pgfqpoint{2.537600in}{2.590000in}}%
\pgfpathlineto{\pgfqpoint{2.540080in}{2.240000in}}%
\pgfpathlineto{\pgfqpoint{2.541320in}{2.450000in}}%
\pgfpathlineto{\pgfqpoint{2.542560in}{2.240000in}}%
\pgfpathlineto{\pgfqpoint{2.543800in}{2.450000in}}%
\pgfpathlineto{\pgfqpoint{2.545040in}{2.345000in}}%
\pgfpathlineto{\pgfqpoint{2.546280in}{2.555000in}}%
\pgfpathlineto{\pgfqpoint{2.547520in}{2.485000in}}%
\pgfpathlineto{\pgfqpoint{2.548760in}{2.940000in}}%
\pgfpathlineto{\pgfqpoint{2.550000in}{2.695000in}}%
\pgfpathlineto{\pgfqpoint{2.551240in}{2.695000in}}%
\pgfpathlineto{\pgfqpoint{2.552480in}{2.345000in}}%
\pgfpathlineto{\pgfqpoint{2.553720in}{2.450000in}}%
\pgfpathlineto{\pgfqpoint{2.554960in}{2.660000in}}%
\pgfpathlineto{\pgfqpoint{2.556200in}{2.135000in}}%
\pgfpathlineto{\pgfqpoint{2.557440in}{2.170000in}}%
\pgfpathlineto{\pgfqpoint{2.558680in}{2.450000in}}%
\pgfpathlineto{\pgfqpoint{2.559920in}{2.065000in}}%
\pgfpathlineto{\pgfqpoint{2.562400in}{2.555000in}}%
\pgfpathlineto{\pgfqpoint{2.563640in}{2.100000in}}%
\pgfpathlineto{\pgfqpoint{2.564880in}{2.135000in}}%
\pgfpathlineto{\pgfqpoint{2.567360in}{2.695000in}}%
\pgfpathlineto{\pgfqpoint{2.568600in}{2.030000in}}%
\pgfpathlineto{\pgfqpoint{2.571080in}{2.660000in}}%
\pgfpathlineto{\pgfqpoint{2.573560in}{2.240000in}}%
\pgfpathlineto{\pgfqpoint{2.576040in}{2.625000in}}%
\pgfpathlineto{\pgfqpoint{2.578520in}{2.765000in}}%
\pgfpathlineto{\pgfqpoint{2.579760in}{2.450000in}}%
\pgfpathlineto{\pgfqpoint{2.581000in}{2.555000in}}%
\pgfpathlineto{\pgfqpoint{2.582240in}{2.065000in}}%
\pgfpathlineto{\pgfqpoint{2.584720in}{2.625000in}}%
\pgfpathlineto{\pgfqpoint{2.585960in}{2.590000in}}%
\pgfpathlineto{\pgfqpoint{2.587200in}{2.660000in}}%
\pgfpathlineto{\pgfqpoint{2.589680in}{2.450000in}}%
\pgfpathlineto{\pgfqpoint{2.590920in}{2.555000in}}%
\pgfpathlineto{\pgfqpoint{2.592160in}{2.380000in}}%
\pgfpathlineto{\pgfqpoint{2.593400in}{3.115000in}}%
\pgfpathlineto{\pgfqpoint{2.595880in}{2.555000in}}%
\pgfpathlineto{\pgfqpoint{2.597120in}{2.765000in}}%
\pgfpathlineto{\pgfqpoint{2.598360in}{2.765000in}}%
\pgfpathlineto{\pgfqpoint{2.599600in}{2.660000in}}%
\pgfpathlineto{\pgfqpoint{2.602080in}{2.275000in}}%
\pgfpathlineto{\pgfqpoint{2.605800in}{2.590000in}}%
\pgfpathlineto{\pgfqpoint{2.607040in}{2.485000in}}%
\pgfpathlineto{\pgfqpoint{2.608280in}{2.625000in}}%
\pgfpathlineto{\pgfqpoint{2.609520in}{2.275000in}}%
\pgfpathlineto{\pgfqpoint{2.610760in}{2.625000in}}%
\pgfpathlineto{\pgfqpoint{2.613240in}{2.310000in}}%
\pgfpathlineto{\pgfqpoint{2.614480in}{2.555000in}}%
\pgfpathlineto{\pgfqpoint{2.615720in}{2.415000in}}%
\pgfpathlineto{\pgfqpoint{2.616960in}{2.485000in}}%
\pgfpathlineto{\pgfqpoint{2.618200in}{2.240000in}}%
\pgfpathlineto{\pgfqpoint{2.621920in}{2.555000in}}%
\pgfpathlineto{\pgfqpoint{2.623160in}{2.450000in}}%
\pgfpathlineto{\pgfqpoint{2.624400in}{2.450000in}}%
\pgfpathlineto{\pgfqpoint{2.625640in}{2.800000in}}%
\pgfpathlineto{\pgfqpoint{2.626880in}{2.415000in}}%
\pgfpathlineto{\pgfqpoint{2.628120in}{2.695000in}}%
\pgfpathlineto{\pgfqpoint{2.629360in}{2.135000in}}%
\pgfpathlineto{\pgfqpoint{2.631840in}{2.765000in}}%
\pgfpathlineto{\pgfqpoint{2.634320in}{2.380000in}}%
\pgfpathlineto{\pgfqpoint{2.635560in}{2.800000in}}%
\pgfpathlineto{\pgfqpoint{2.638040in}{2.380000in}}%
\pgfpathlineto{\pgfqpoint{2.639280in}{2.835000in}}%
\pgfpathlineto{\pgfqpoint{2.640520in}{2.625000in}}%
\pgfpathlineto{\pgfqpoint{2.641760in}{1.960000in}}%
\pgfpathlineto{\pgfqpoint{2.643000in}{2.065000in}}%
\pgfpathlineto{\pgfqpoint{2.645480in}{2.835000in}}%
\pgfpathlineto{\pgfqpoint{2.646720in}{2.485000in}}%
\pgfpathlineto{\pgfqpoint{2.647960in}{2.590000in}}%
\pgfpathlineto{\pgfqpoint{2.649200in}{2.170000in}}%
\pgfpathlineto{\pgfqpoint{2.650440in}{2.345000in}}%
\pgfpathlineto{\pgfqpoint{2.651680in}{2.345000in}}%
\pgfpathlineto{\pgfqpoint{2.654160in}{2.730000in}}%
\pgfpathlineto{\pgfqpoint{2.655400in}{2.660000in}}%
\pgfpathlineto{\pgfqpoint{2.657880in}{2.100000in}}%
\pgfpathlineto{\pgfqpoint{2.659120in}{2.625000in}}%
\pgfpathlineto{\pgfqpoint{2.661600in}{2.380000in}}%
\pgfpathlineto{\pgfqpoint{2.664080in}{2.555000in}}%
\pgfpathlineto{\pgfqpoint{2.665320in}{2.240000in}}%
\pgfpathlineto{\pgfqpoint{2.666560in}{2.380000in}}%
\pgfpathlineto{\pgfqpoint{2.667800in}{2.310000in}}%
\pgfpathlineto{\pgfqpoint{2.669040in}{2.870000in}}%
\pgfpathlineto{\pgfqpoint{2.670280in}{2.345000in}}%
\pgfpathlineto{\pgfqpoint{2.672760in}{2.590000in}}%
\pgfpathlineto{\pgfqpoint{2.674000in}{2.520000in}}%
\pgfpathlineto{\pgfqpoint{2.675240in}{2.625000in}}%
\pgfpathlineto{\pgfqpoint{2.676480in}{2.240000in}}%
\pgfpathlineto{\pgfqpoint{2.677720in}{2.555000in}}%
\pgfpathlineto{\pgfqpoint{2.680200in}{2.205000in}}%
\pgfpathlineto{\pgfqpoint{2.681440in}{2.765000in}}%
\pgfpathlineto{\pgfqpoint{2.683920in}{2.275000in}}%
\pgfpathlineto{\pgfqpoint{2.685160in}{2.625000in}}%
\pgfpathlineto{\pgfqpoint{2.686400in}{2.485000in}}%
\pgfpathlineto{\pgfqpoint{2.687640in}{2.100000in}}%
\pgfpathlineto{\pgfqpoint{2.688880in}{2.100000in}}%
\pgfpathlineto{\pgfqpoint{2.691360in}{2.555000in}}%
\pgfpathlineto{\pgfqpoint{2.692600in}{2.135000in}}%
\pgfpathlineto{\pgfqpoint{2.693840in}{2.170000in}}%
\pgfpathlineto{\pgfqpoint{2.695080in}{2.555000in}}%
\pgfpathlineto{\pgfqpoint{2.696320in}{2.520000in}}%
\pgfpathlineto{\pgfqpoint{2.697560in}{2.345000in}}%
\pgfpathlineto{\pgfqpoint{2.698800in}{2.695000in}}%
\pgfpathlineto{\pgfqpoint{2.701280in}{2.380000in}}%
\pgfpathlineto{\pgfqpoint{2.702520in}{2.380000in}}%
\pgfpathlineto{\pgfqpoint{2.705000in}{2.800000in}}%
\pgfpathlineto{\pgfqpoint{2.708720in}{2.310000in}}%
\pgfpathlineto{\pgfqpoint{2.709960in}{2.835000in}}%
\pgfpathlineto{\pgfqpoint{2.713680in}{2.170000in}}%
\pgfpathlineto{\pgfqpoint{2.714920in}{2.450000in}}%
\pgfpathlineto{\pgfqpoint{2.716160in}{2.345000in}}%
\pgfpathlineto{\pgfqpoint{2.717400in}{1.995000in}}%
\pgfpathlineto{\pgfqpoint{2.719880in}{2.800000in}}%
\pgfpathlineto{\pgfqpoint{2.721120in}{2.205000in}}%
\pgfpathlineto{\pgfqpoint{2.723600in}{2.625000in}}%
\pgfpathlineto{\pgfqpoint{2.724840in}{2.485000in}}%
\pgfpathlineto{\pgfqpoint{2.726080in}{2.485000in}}%
\pgfpathlineto{\pgfqpoint{2.727320in}{2.415000in}}%
\pgfpathlineto{\pgfqpoint{2.728560in}{2.170000in}}%
\pgfpathlineto{\pgfqpoint{2.731040in}{2.625000in}}%
\pgfpathlineto{\pgfqpoint{2.732280in}{2.415000in}}%
\pgfpathlineto{\pgfqpoint{2.733520in}{2.450000in}}%
\pgfpathlineto{\pgfqpoint{2.734760in}{2.205000in}}%
\pgfpathlineto{\pgfqpoint{2.736000in}{2.275000in}}%
\pgfpathlineto{\pgfqpoint{2.737240in}{2.730000in}}%
\pgfpathlineto{\pgfqpoint{2.738480in}{2.205000in}}%
\pgfpathlineto{\pgfqpoint{2.739720in}{2.485000in}}%
\pgfpathlineto{\pgfqpoint{2.740960in}{2.240000in}}%
\pgfpathlineto{\pgfqpoint{2.742200in}{2.345000in}}%
\pgfpathlineto{\pgfqpoint{2.743440in}{2.660000in}}%
\pgfpathlineto{\pgfqpoint{2.744680in}{2.520000in}}%
\pgfpathlineto{\pgfqpoint{2.745920in}{2.520000in}}%
\pgfpathlineto{\pgfqpoint{2.747160in}{2.590000in}}%
\pgfpathlineto{\pgfqpoint{2.749640in}{2.240000in}}%
\pgfpathlineto{\pgfqpoint{2.750880in}{1.960000in}}%
\pgfpathlineto{\pgfqpoint{2.752120in}{2.485000in}}%
\pgfpathlineto{\pgfqpoint{2.753360in}{2.030000in}}%
\pgfpathlineto{\pgfqpoint{2.754600in}{2.275000in}}%
\pgfpathlineto{\pgfqpoint{2.755840in}{2.275000in}}%
\pgfpathlineto{\pgfqpoint{2.757080in}{2.590000in}}%
\pgfpathlineto{\pgfqpoint{2.758320in}{2.135000in}}%
\pgfpathlineto{\pgfqpoint{2.759560in}{2.555000in}}%
\pgfpathlineto{\pgfqpoint{2.760800in}{2.345000in}}%
\pgfpathlineto{\pgfqpoint{2.762040in}{2.415000in}}%
\pgfpathlineto{\pgfqpoint{2.763280in}{2.345000in}}%
\pgfpathlineto{\pgfqpoint{2.764520in}{2.415000in}}%
\pgfpathlineto{\pgfqpoint{2.765760in}{2.275000in}}%
\pgfpathlineto{\pgfqpoint{2.767000in}{2.520000in}}%
\pgfpathlineto{\pgfqpoint{2.769480in}{2.240000in}}%
\pgfpathlineto{\pgfqpoint{2.770720in}{2.135000in}}%
\pgfpathlineto{\pgfqpoint{2.771960in}{2.205000in}}%
\pgfpathlineto{\pgfqpoint{2.773200in}{2.625000in}}%
\pgfpathlineto{\pgfqpoint{2.774440in}{2.555000in}}%
\pgfpathlineto{\pgfqpoint{2.776920in}{2.240000in}}%
\pgfpathlineto{\pgfqpoint{2.778160in}{2.660000in}}%
\pgfpathlineto{\pgfqpoint{2.779400in}{2.240000in}}%
\pgfpathlineto{\pgfqpoint{2.781880in}{2.975000in}}%
\pgfpathlineto{\pgfqpoint{2.785600in}{1.890000in}}%
\pgfpathlineto{\pgfqpoint{2.786840in}{2.520000in}}%
\pgfpathlineto{\pgfqpoint{2.790560in}{2.100000in}}%
\pgfpathlineto{\pgfqpoint{2.793040in}{2.520000in}}%
\pgfpathlineto{\pgfqpoint{2.794280in}{2.240000in}}%
\pgfpathlineto{\pgfqpoint{2.795520in}{2.240000in}}%
\pgfpathlineto{\pgfqpoint{2.798000in}{2.590000in}}%
\pgfpathlineto{\pgfqpoint{2.799240in}{2.380000in}}%
\pgfpathlineto{\pgfqpoint{2.800480in}{2.415000in}}%
\pgfpathlineto{\pgfqpoint{2.801720in}{2.555000in}}%
\pgfpathlineto{\pgfqpoint{2.802960in}{2.170000in}}%
\pgfpathlineto{\pgfqpoint{2.804200in}{2.765000in}}%
\pgfpathlineto{\pgfqpoint{2.805440in}{2.660000in}}%
\pgfpathlineto{\pgfqpoint{2.807920in}{2.660000in}}%
\pgfpathlineto{\pgfqpoint{2.809160in}{2.590000in}}%
\pgfpathlineto{\pgfqpoint{2.811640in}{2.380000in}}%
\pgfpathlineto{\pgfqpoint{2.812880in}{2.695000in}}%
\pgfpathlineto{\pgfqpoint{2.816600in}{2.065000in}}%
\pgfpathlineto{\pgfqpoint{2.817840in}{2.765000in}}%
\pgfpathlineto{\pgfqpoint{2.819080in}{2.520000in}}%
\pgfpathlineto{\pgfqpoint{2.820320in}{2.520000in}}%
\pgfpathlineto{\pgfqpoint{2.821560in}{2.415000in}}%
\pgfpathlineto{\pgfqpoint{2.824040in}{2.590000in}}%
\pgfpathlineto{\pgfqpoint{2.825280in}{2.135000in}}%
\pgfpathlineto{\pgfqpoint{2.827760in}{2.520000in}}%
\pgfpathlineto{\pgfqpoint{2.830240in}{2.275000in}}%
\pgfpathlineto{\pgfqpoint{2.831480in}{2.590000in}}%
\pgfpathlineto{\pgfqpoint{2.832720in}{2.450000in}}%
\pgfpathlineto{\pgfqpoint{2.833960in}{2.660000in}}%
\pgfpathlineto{\pgfqpoint{2.835200in}{2.275000in}}%
\pgfpathlineto{\pgfqpoint{2.836440in}{2.765000in}}%
\pgfpathlineto{\pgfqpoint{2.837680in}{2.275000in}}%
\pgfpathlineto{\pgfqpoint{2.838920in}{2.275000in}}%
\pgfpathlineto{\pgfqpoint{2.840160in}{1.820000in}}%
\pgfpathlineto{\pgfqpoint{2.842640in}{2.590000in}}%
\pgfpathlineto{\pgfqpoint{2.845120in}{2.310000in}}%
\pgfpathlineto{\pgfqpoint{2.846360in}{2.555000in}}%
\pgfpathlineto{\pgfqpoint{2.847600in}{1.995000in}}%
\pgfpathlineto{\pgfqpoint{2.848840in}{2.205000in}}%
\pgfpathlineto{\pgfqpoint{2.851320in}{2.765000in}}%
\pgfpathlineto{\pgfqpoint{2.852560in}{2.590000in}}%
\pgfpathlineto{\pgfqpoint{2.853800in}{2.905000in}}%
\pgfpathlineto{\pgfqpoint{2.855040in}{2.135000in}}%
\pgfpathlineto{\pgfqpoint{2.856280in}{2.940000in}}%
\pgfpathlineto{\pgfqpoint{2.857520in}{2.310000in}}%
\pgfpathlineto{\pgfqpoint{2.860000in}{2.695000in}}%
\pgfpathlineto{\pgfqpoint{2.861240in}{2.835000in}}%
\pgfpathlineto{\pgfqpoint{2.862480in}{2.555000in}}%
\pgfpathlineto{\pgfqpoint{2.863720in}{2.730000in}}%
\pgfpathlineto{\pgfqpoint{2.864960in}{2.450000in}}%
\pgfpathlineto{\pgfqpoint{2.866200in}{2.730000in}}%
\pgfpathlineto{\pgfqpoint{2.867440in}{2.660000in}}%
\pgfpathlineto{\pgfqpoint{2.868680in}{2.310000in}}%
\pgfpathlineto{\pgfqpoint{2.871160in}{2.695000in}}%
\pgfpathlineto{\pgfqpoint{2.872400in}{2.520000in}}%
\pgfpathlineto{\pgfqpoint{2.873640in}{2.765000in}}%
\pgfpathlineto{\pgfqpoint{2.874880in}{2.485000in}}%
\pgfpathlineto{\pgfqpoint{2.876120in}{2.555000in}}%
\pgfpathlineto{\pgfqpoint{2.877360in}{2.695000in}}%
\pgfpathlineto{\pgfqpoint{2.878600in}{2.030000in}}%
\pgfpathlineto{\pgfqpoint{2.879840in}{2.590000in}}%
\pgfpathlineto{\pgfqpoint{2.881080in}{2.485000in}}%
\pgfpathlineto{\pgfqpoint{2.882320in}{2.100000in}}%
\pgfpathlineto{\pgfqpoint{2.883560in}{2.625000in}}%
\pgfpathlineto{\pgfqpoint{2.886040in}{2.135000in}}%
\pgfpathlineto{\pgfqpoint{2.888520in}{2.310000in}}%
\pgfpathlineto{\pgfqpoint{2.891000in}{2.485000in}}%
\pgfpathlineto{\pgfqpoint{2.892240in}{2.415000in}}%
\pgfpathlineto{\pgfqpoint{2.893480in}{2.415000in}}%
\pgfpathlineto{\pgfqpoint{2.894720in}{2.625000in}}%
\pgfpathlineto{\pgfqpoint{2.897200in}{1.995000in}}%
\pgfpathlineto{\pgfqpoint{2.902160in}{2.835000in}}%
\pgfpathlineto{\pgfqpoint{2.904640in}{2.345000in}}%
\pgfpathlineto{\pgfqpoint{2.905880in}{2.310000in}}%
\pgfpathlineto{\pgfqpoint{2.907120in}{2.030000in}}%
\pgfpathlineto{\pgfqpoint{2.908360in}{2.415000in}}%
\pgfpathlineto{\pgfqpoint{2.909600in}{2.415000in}}%
\pgfpathlineto{\pgfqpoint{2.910840in}{2.310000in}}%
\pgfpathlineto{\pgfqpoint{2.912080in}{2.345000in}}%
\pgfpathlineto{\pgfqpoint{2.913320in}{2.590000in}}%
\pgfpathlineto{\pgfqpoint{2.914560in}{2.590000in}}%
\pgfpathlineto{\pgfqpoint{2.915800in}{2.555000in}}%
\pgfpathlineto{\pgfqpoint{2.917040in}{2.590000in}}%
\pgfpathlineto{\pgfqpoint{2.918280in}{2.415000in}}%
\pgfpathlineto{\pgfqpoint{2.919520in}{2.520000in}}%
\pgfpathlineto{\pgfqpoint{2.920760in}{2.345000in}}%
\pgfpathlineto{\pgfqpoint{2.922000in}{2.520000in}}%
\pgfpathlineto{\pgfqpoint{2.923240in}{2.870000in}}%
\pgfpathlineto{\pgfqpoint{2.925720in}{2.275000in}}%
\pgfpathlineto{\pgfqpoint{2.926960in}{2.275000in}}%
\pgfpathlineto{\pgfqpoint{2.928200in}{2.590000in}}%
\pgfpathlineto{\pgfqpoint{2.929440in}{2.205000in}}%
\pgfpathlineto{\pgfqpoint{2.930680in}{2.205000in}}%
\pgfpathlineto{\pgfqpoint{2.931920in}{2.310000in}}%
\pgfpathlineto{\pgfqpoint{2.933160in}{2.625000in}}%
\pgfpathlineto{\pgfqpoint{2.934400in}{2.590000in}}%
\pgfpathlineto{\pgfqpoint{2.935640in}{2.590000in}}%
\pgfpathlineto{\pgfqpoint{2.936880in}{2.835000in}}%
\pgfpathlineto{\pgfqpoint{2.938120in}{2.590000in}}%
\pgfpathlineto{\pgfqpoint{2.939360in}{2.590000in}}%
\pgfpathlineto{\pgfqpoint{2.940600in}{2.485000in}}%
\pgfpathlineto{\pgfqpoint{2.941840in}{2.520000in}}%
\pgfpathlineto{\pgfqpoint{2.943080in}{2.275000in}}%
\pgfpathlineto{\pgfqpoint{2.944320in}{2.660000in}}%
\pgfpathlineto{\pgfqpoint{2.945560in}{2.100000in}}%
\pgfpathlineto{\pgfqpoint{2.948040in}{2.520000in}}%
\pgfpathlineto{\pgfqpoint{2.949280in}{2.520000in}}%
\pgfpathlineto{\pgfqpoint{2.950520in}{2.485000in}}%
\pgfpathlineto{\pgfqpoint{2.951760in}{2.170000in}}%
\pgfpathlineto{\pgfqpoint{2.953000in}{2.590000in}}%
\pgfpathlineto{\pgfqpoint{2.954240in}{2.555000in}}%
\pgfpathlineto{\pgfqpoint{2.955480in}{2.625000in}}%
\pgfpathlineto{\pgfqpoint{2.957960in}{2.345000in}}%
\pgfpathlineto{\pgfqpoint{2.959200in}{2.275000in}}%
\pgfpathlineto{\pgfqpoint{2.960440in}{2.380000in}}%
\pgfpathlineto{\pgfqpoint{2.961680in}{2.240000in}}%
\pgfpathlineto{\pgfqpoint{2.962920in}{2.625000in}}%
\pgfpathlineto{\pgfqpoint{2.965400in}{2.485000in}}%
\pgfpathlineto{\pgfqpoint{2.966640in}{2.450000in}}%
\pgfpathlineto{\pgfqpoint{2.969120in}{1.890000in}}%
\pgfpathlineto{\pgfqpoint{2.970360in}{2.135000in}}%
\pgfpathlineto{\pgfqpoint{2.971600in}{2.625000in}}%
\pgfpathlineto{\pgfqpoint{2.972840in}{2.065000in}}%
\pgfpathlineto{\pgfqpoint{2.974080in}{2.625000in}}%
\pgfpathlineto{\pgfqpoint{2.975320in}{2.030000in}}%
\pgfpathlineto{\pgfqpoint{2.976560in}{2.660000in}}%
\pgfpathlineto{\pgfqpoint{2.977800in}{2.485000in}}%
\pgfpathlineto{\pgfqpoint{2.979040in}{1.995000in}}%
\pgfpathlineto{\pgfqpoint{2.981520in}{2.555000in}}%
\pgfpathlineto{\pgfqpoint{2.984000in}{2.765000in}}%
\pgfpathlineto{\pgfqpoint{2.985240in}{2.695000in}}%
\pgfpathlineto{\pgfqpoint{2.986480in}{2.485000in}}%
\pgfpathlineto{\pgfqpoint{2.987720in}{2.660000in}}%
\pgfpathlineto{\pgfqpoint{2.988960in}{2.590000in}}%
\pgfpathlineto{\pgfqpoint{2.990200in}{2.415000in}}%
\pgfpathlineto{\pgfqpoint{2.991440in}{2.590000in}}%
\pgfpathlineto{\pgfqpoint{2.992680in}{2.450000in}}%
\pgfpathlineto{\pgfqpoint{2.995160in}{2.800000in}}%
\pgfpathlineto{\pgfqpoint{2.996400in}{2.555000in}}%
\pgfpathlineto{\pgfqpoint{2.997640in}{2.555000in}}%
\pgfpathlineto{\pgfqpoint{2.998880in}{2.380000in}}%
\pgfpathlineto{\pgfqpoint{3.000120in}{2.660000in}}%
\pgfpathlineto{\pgfqpoint{3.001360in}{2.310000in}}%
\pgfpathlineto{\pgfqpoint{3.002600in}{2.345000in}}%
\pgfpathlineto{\pgfqpoint{3.003840in}{2.660000in}}%
\pgfpathlineto{\pgfqpoint{3.005080in}{2.275000in}}%
\pgfpathlineto{\pgfqpoint{3.006320in}{2.240000in}}%
\pgfpathlineto{\pgfqpoint{3.008800in}{2.695000in}}%
\pgfpathlineto{\pgfqpoint{3.011280in}{2.240000in}}%
\pgfpathlineto{\pgfqpoint{3.012520in}{2.205000in}}%
\pgfpathlineto{\pgfqpoint{3.013760in}{2.450000in}}%
\pgfpathlineto{\pgfqpoint{3.015000in}{2.345000in}}%
\pgfpathlineto{\pgfqpoint{3.016240in}{1.890000in}}%
\pgfpathlineto{\pgfqpoint{3.017480in}{2.485000in}}%
\pgfpathlineto{\pgfqpoint{3.018720in}{2.450000in}}%
\pgfpathlineto{\pgfqpoint{3.019960in}{1.645000in}}%
\pgfpathlineto{\pgfqpoint{3.021200in}{1.995000in}}%
\pgfpathlineto{\pgfqpoint{3.022440in}{1.855000in}}%
\pgfpathlineto{\pgfqpoint{3.024920in}{2.555000in}}%
\pgfpathlineto{\pgfqpoint{3.029880in}{1.995000in}}%
\pgfpathlineto{\pgfqpoint{3.032360in}{2.450000in}}%
\pgfpathlineto{\pgfqpoint{3.034840in}{2.135000in}}%
\pgfpathlineto{\pgfqpoint{3.036080in}{2.590000in}}%
\pgfpathlineto{\pgfqpoint{3.037320in}{2.485000in}}%
\pgfpathlineto{\pgfqpoint{3.038560in}{2.275000in}}%
\pgfpathlineto{\pgfqpoint{3.039800in}{1.855000in}}%
\pgfpathlineto{\pgfqpoint{3.041040in}{2.555000in}}%
\pgfpathlineto{\pgfqpoint{3.042280in}{2.520000in}}%
\pgfpathlineto{\pgfqpoint{3.043520in}{2.660000in}}%
\pgfpathlineto{\pgfqpoint{3.044760in}{2.380000in}}%
\pgfpathlineto{\pgfqpoint{3.046000in}{2.835000in}}%
\pgfpathlineto{\pgfqpoint{3.048480in}{2.450000in}}%
\pgfpathlineto{\pgfqpoint{3.053440in}{2.835000in}}%
\pgfpathlineto{\pgfqpoint{3.054680in}{2.275000in}}%
\pgfpathlineto{\pgfqpoint{3.055920in}{2.380000in}}%
\pgfpathlineto{\pgfqpoint{3.057160in}{2.205000in}}%
\pgfpathlineto{\pgfqpoint{3.059640in}{2.660000in}}%
\pgfpathlineto{\pgfqpoint{3.062120in}{2.275000in}}%
\pgfpathlineto{\pgfqpoint{3.063360in}{2.310000in}}%
\pgfpathlineto{\pgfqpoint{3.064600in}{2.485000in}}%
\pgfpathlineto{\pgfqpoint{3.067080in}{2.205000in}}%
\pgfpathlineto{\pgfqpoint{3.069560in}{2.380000in}}%
\pgfpathlineto{\pgfqpoint{3.070800in}{2.275000in}}%
\pgfpathlineto{\pgfqpoint{3.073280in}{2.485000in}}%
\pgfpathlineto{\pgfqpoint{3.075760in}{2.275000in}}%
\pgfpathlineto{\pgfqpoint{3.077000in}{2.555000in}}%
\pgfpathlineto{\pgfqpoint{3.078240in}{2.205000in}}%
\pgfpathlineto{\pgfqpoint{3.079480in}{2.625000in}}%
\pgfpathlineto{\pgfqpoint{3.080720in}{2.485000in}}%
\pgfpathlineto{\pgfqpoint{3.081960in}{2.100000in}}%
\pgfpathlineto{\pgfqpoint{3.083200in}{2.065000in}}%
\pgfpathlineto{\pgfqpoint{3.084440in}{2.450000in}}%
\pgfpathlineto{\pgfqpoint{3.088160in}{2.065000in}}%
\pgfpathlineto{\pgfqpoint{3.089400in}{2.135000in}}%
\pgfpathlineto{\pgfqpoint{3.090640in}{2.135000in}}%
\pgfpathlineto{\pgfqpoint{3.091880in}{2.170000in}}%
\pgfpathlineto{\pgfqpoint{3.093120in}{1.960000in}}%
\pgfpathlineto{\pgfqpoint{3.095600in}{2.590000in}}%
\pgfpathlineto{\pgfqpoint{3.096840in}{2.450000in}}%
\pgfpathlineto{\pgfqpoint{3.098080in}{1.960000in}}%
\pgfpathlineto{\pgfqpoint{3.099320in}{2.555000in}}%
\pgfpathlineto{\pgfqpoint{3.100560in}{2.450000in}}%
\pgfpathlineto{\pgfqpoint{3.101800in}{1.785000in}}%
\pgfpathlineto{\pgfqpoint{3.104280in}{2.345000in}}%
\pgfpathlineto{\pgfqpoint{3.105520in}{2.240000in}}%
\pgfpathlineto{\pgfqpoint{3.106760in}{2.310000in}}%
\pgfpathlineto{\pgfqpoint{3.108000in}{2.310000in}}%
\pgfpathlineto{\pgfqpoint{3.109240in}{2.275000in}}%
\pgfpathlineto{\pgfqpoint{3.110480in}{2.065000in}}%
\pgfpathlineto{\pgfqpoint{3.111720in}{2.240000in}}%
\pgfpathlineto{\pgfqpoint{3.112960in}{2.205000in}}%
\pgfpathlineto{\pgfqpoint{3.114200in}{2.555000in}}%
\pgfpathlineto{\pgfqpoint{3.115440in}{2.450000in}}%
\pgfpathlineto{\pgfqpoint{3.116680in}{2.520000in}}%
\pgfpathlineto{\pgfqpoint{3.117920in}{2.380000in}}%
\pgfpathlineto{\pgfqpoint{3.119160in}{2.520000in}}%
\pgfpathlineto{\pgfqpoint{3.120400in}{2.240000in}}%
\pgfpathlineto{\pgfqpoint{3.121640in}{2.310000in}}%
\pgfpathlineto{\pgfqpoint{3.122880in}{2.065000in}}%
\pgfpathlineto{\pgfqpoint{3.124120in}{2.380000in}}%
\pgfpathlineto{\pgfqpoint{3.125360in}{2.275000in}}%
\pgfpathlineto{\pgfqpoint{3.126600in}{2.625000in}}%
\pgfpathlineto{\pgfqpoint{3.127840in}{2.625000in}}%
\pgfpathlineto{\pgfqpoint{3.129080in}{2.730000in}}%
\pgfpathlineto{\pgfqpoint{3.130320in}{2.205000in}}%
\pgfpathlineto{\pgfqpoint{3.131560in}{2.415000in}}%
\pgfpathlineto{\pgfqpoint{3.132800in}{2.380000in}}%
\pgfpathlineto{\pgfqpoint{3.134040in}{2.205000in}}%
\pgfpathlineto{\pgfqpoint{3.135280in}{2.870000in}}%
\pgfpathlineto{\pgfqpoint{3.136520in}{2.170000in}}%
\pgfpathlineto{\pgfqpoint{3.137760in}{2.170000in}}%
\pgfpathlineto{\pgfqpoint{3.139000in}{2.205000in}}%
\pgfpathlineto{\pgfqpoint{3.140240in}{2.345000in}}%
\pgfpathlineto{\pgfqpoint{3.141480in}{2.765000in}}%
\pgfpathlineto{\pgfqpoint{3.142720in}{2.240000in}}%
\pgfpathlineto{\pgfqpoint{3.143960in}{2.205000in}}%
\pgfpathlineto{\pgfqpoint{3.145200in}{2.275000in}}%
\pgfpathlineto{\pgfqpoint{3.146440in}{2.625000in}}%
\pgfpathlineto{\pgfqpoint{3.147680in}{2.135000in}}%
\pgfpathlineto{\pgfqpoint{3.148920in}{2.800000in}}%
\pgfpathlineto{\pgfqpoint{3.151400in}{2.310000in}}%
\pgfpathlineto{\pgfqpoint{3.153880in}{2.485000in}}%
\pgfpathlineto{\pgfqpoint{3.155120in}{2.765000in}}%
\pgfpathlineto{\pgfqpoint{3.156360in}{2.275000in}}%
\pgfpathlineto{\pgfqpoint{3.157600in}{2.240000in}}%
\pgfpathlineto{\pgfqpoint{3.158840in}{2.345000in}}%
\pgfpathlineto{\pgfqpoint{3.160080in}{2.590000in}}%
\pgfpathlineto{\pgfqpoint{3.161320in}{2.450000in}}%
\pgfpathlineto{\pgfqpoint{3.162560in}{2.660000in}}%
\pgfpathlineto{\pgfqpoint{3.163800in}{2.240000in}}%
\pgfpathlineto{\pgfqpoint{3.165040in}{2.590000in}}%
\pgfpathlineto{\pgfqpoint{3.166280in}{2.555000in}}%
\pgfpathlineto{\pgfqpoint{3.170000in}{1.855000in}}%
\pgfpathlineto{\pgfqpoint{3.171240in}{2.450000in}}%
\pgfpathlineto{\pgfqpoint{3.172480in}{2.450000in}}%
\pgfpathlineto{\pgfqpoint{3.173720in}{2.345000in}}%
\pgfpathlineto{\pgfqpoint{3.174960in}{2.695000in}}%
\pgfpathlineto{\pgfqpoint{3.176200in}{2.380000in}}%
\pgfpathlineto{\pgfqpoint{3.177440in}{2.695000in}}%
\pgfpathlineto{\pgfqpoint{3.178680in}{2.275000in}}%
\pgfpathlineto{\pgfqpoint{3.179920in}{2.310000in}}%
\pgfpathlineto{\pgfqpoint{3.182400in}{2.170000in}}%
\pgfpathlineto{\pgfqpoint{3.184880in}{2.555000in}}%
\pgfpathlineto{\pgfqpoint{3.186120in}{2.485000in}}%
\pgfpathlineto{\pgfqpoint{3.187360in}{2.485000in}}%
\pgfpathlineto{\pgfqpoint{3.188600in}{2.275000in}}%
\pgfpathlineto{\pgfqpoint{3.189840in}{2.485000in}}%
\pgfpathlineto{\pgfqpoint{3.191080in}{2.240000in}}%
\pgfpathlineto{\pgfqpoint{3.192320in}{2.415000in}}%
\pgfpathlineto{\pgfqpoint{3.193560in}{2.065000in}}%
\pgfpathlineto{\pgfqpoint{3.194800in}{2.135000in}}%
\pgfpathlineto{\pgfqpoint{3.196040in}{2.100000in}}%
\pgfpathlineto{\pgfqpoint{3.198520in}{2.555000in}}%
\pgfpathlineto{\pgfqpoint{3.199760in}{2.380000in}}%
\pgfpathlineto{\pgfqpoint{3.201000in}{2.030000in}}%
\pgfpathlineto{\pgfqpoint{3.202240in}{2.205000in}}%
\pgfpathlineto{\pgfqpoint{3.204720in}{1.925000in}}%
\pgfpathlineto{\pgfqpoint{3.205960in}{1.925000in}}%
\pgfpathlineto{\pgfqpoint{3.207200in}{2.065000in}}%
\pgfpathlineto{\pgfqpoint{3.208440in}{2.415000in}}%
\pgfpathlineto{\pgfqpoint{3.209680in}{2.450000in}}%
\pgfpathlineto{\pgfqpoint{3.210920in}{2.590000in}}%
\pgfpathlineto{\pgfqpoint{3.212160in}{2.135000in}}%
\pgfpathlineto{\pgfqpoint{3.213400in}{2.380000in}}%
\pgfpathlineto{\pgfqpoint{3.214640in}{2.310000in}}%
\pgfpathlineto{\pgfqpoint{3.215880in}{1.960000in}}%
\pgfpathlineto{\pgfqpoint{3.217120in}{2.170000in}}%
\pgfpathlineto{\pgfqpoint{3.219600in}{1.960000in}}%
\pgfpathlineto{\pgfqpoint{3.220840in}{2.170000in}}%
\pgfpathlineto{\pgfqpoint{3.222080in}{1.785000in}}%
\pgfpathlineto{\pgfqpoint{3.224560in}{2.450000in}}%
\pgfpathlineto{\pgfqpoint{3.225800in}{2.485000in}}%
\pgfpathlineto{\pgfqpoint{3.228280in}{2.135000in}}%
\pgfpathlineto{\pgfqpoint{3.230760in}{2.660000in}}%
\pgfpathlineto{\pgfqpoint{3.232000in}{2.590000in}}%
\pgfpathlineto{\pgfqpoint{3.233240in}{2.240000in}}%
\pgfpathlineto{\pgfqpoint{3.234480in}{2.345000in}}%
\pgfpathlineto{\pgfqpoint{3.235720in}{2.135000in}}%
\pgfpathlineto{\pgfqpoint{3.238200in}{2.520000in}}%
\pgfpathlineto{\pgfqpoint{3.239440in}{2.030000in}}%
\pgfpathlineto{\pgfqpoint{3.240680in}{2.625000in}}%
\pgfpathlineto{\pgfqpoint{3.243160in}{1.995000in}}%
\pgfpathlineto{\pgfqpoint{3.244400in}{2.345000in}}%
\pgfpathlineto{\pgfqpoint{3.245640in}{2.135000in}}%
\pgfpathlineto{\pgfqpoint{3.246880in}{2.520000in}}%
\pgfpathlineto{\pgfqpoint{3.248120in}{2.520000in}}%
\pgfpathlineto{\pgfqpoint{3.249360in}{2.205000in}}%
\pgfpathlineto{\pgfqpoint{3.250600in}{2.275000in}}%
\pgfpathlineto{\pgfqpoint{3.251840in}{2.415000in}}%
\pgfpathlineto{\pgfqpoint{3.253080in}{2.205000in}}%
\pgfpathlineto{\pgfqpoint{3.254320in}{2.555000in}}%
\pgfpathlineto{\pgfqpoint{3.255560in}{2.520000in}}%
\pgfpathlineto{\pgfqpoint{3.256800in}{2.275000in}}%
\pgfpathlineto{\pgfqpoint{3.258040in}{2.485000in}}%
\pgfpathlineto{\pgfqpoint{3.259280in}{2.485000in}}%
\pgfpathlineto{\pgfqpoint{3.260520in}{2.590000in}}%
\pgfpathlineto{\pgfqpoint{3.263000in}{2.345000in}}%
\pgfpathlineto{\pgfqpoint{3.264240in}{2.450000in}}%
\pgfpathlineto{\pgfqpoint{3.265480in}{2.135000in}}%
\pgfpathlineto{\pgfqpoint{3.267960in}{2.450000in}}%
\pgfpathlineto{\pgfqpoint{3.269200in}{2.380000in}}%
\pgfpathlineto{\pgfqpoint{3.270440in}{2.380000in}}%
\pgfpathlineto{\pgfqpoint{3.271680in}{2.485000in}}%
\pgfpathlineto{\pgfqpoint{3.272920in}{2.415000in}}%
\pgfpathlineto{\pgfqpoint{3.274160in}{2.205000in}}%
\pgfpathlineto{\pgfqpoint{3.275400in}{2.275000in}}%
\pgfpathlineto{\pgfqpoint{3.276640in}{1.890000in}}%
\pgfpathlineto{\pgfqpoint{3.279120in}{2.625000in}}%
\pgfpathlineto{\pgfqpoint{3.280360in}{2.380000in}}%
\pgfpathlineto{\pgfqpoint{3.284080in}{2.695000in}}%
\pgfpathlineto{\pgfqpoint{3.285320in}{2.555000in}}%
\pgfpathlineto{\pgfqpoint{3.286560in}{2.205000in}}%
\pgfpathlineto{\pgfqpoint{3.287800in}{2.380000in}}%
\pgfpathlineto{\pgfqpoint{3.289040in}{2.135000in}}%
\pgfpathlineto{\pgfqpoint{3.290280in}{2.415000in}}%
\pgfpathlineto{\pgfqpoint{3.291520in}{2.135000in}}%
\pgfpathlineto{\pgfqpoint{3.292760in}{2.205000in}}%
\pgfpathlineto{\pgfqpoint{3.294000in}{2.625000in}}%
\pgfpathlineto{\pgfqpoint{3.295240in}{2.625000in}}%
\pgfpathlineto{\pgfqpoint{3.296480in}{2.380000in}}%
\pgfpathlineto{\pgfqpoint{3.298960in}{2.625000in}}%
\pgfpathlineto{\pgfqpoint{3.300200in}{2.415000in}}%
\pgfpathlineto{\pgfqpoint{3.301440in}{2.485000in}}%
\pgfpathlineto{\pgfqpoint{3.303920in}{2.170000in}}%
\pgfpathlineto{\pgfqpoint{3.305160in}{2.450000in}}%
\pgfpathlineto{\pgfqpoint{3.306400in}{2.310000in}}%
\pgfpathlineto{\pgfqpoint{3.307640in}{2.730000in}}%
\pgfpathlineto{\pgfqpoint{3.308880in}{2.065000in}}%
\pgfpathlineto{\pgfqpoint{3.310120in}{2.380000in}}%
\pgfpathlineto{\pgfqpoint{3.311360in}{2.310000in}}%
\pgfpathlineto{\pgfqpoint{3.312600in}{2.170000in}}%
\pgfpathlineto{\pgfqpoint{3.313840in}{2.345000in}}%
\pgfpathlineto{\pgfqpoint{3.315080in}{2.275000in}}%
\pgfpathlineto{\pgfqpoint{3.317560in}{2.520000in}}%
\pgfpathlineto{\pgfqpoint{3.318800in}{2.520000in}}%
\pgfpathlineto{\pgfqpoint{3.320040in}{2.485000in}}%
\pgfpathlineto{\pgfqpoint{3.321280in}{2.555000in}}%
\pgfpathlineto{\pgfqpoint{3.322520in}{2.450000in}}%
\pgfpathlineto{\pgfqpoint{3.323760in}{2.730000in}}%
\pgfpathlineto{\pgfqpoint{3.326240in}{2.520000in}}%
\pgfpathlineto{\pgfqpoint{3.327480in}{2.555000in}}%
\pgfpathlineto{\pgfqpoint{3.328720in}{2.275000in}}%
\pgfpathlineto{\pgfqpoint{3.329960in}{2.310000in}}%
\pgfpathlineto{\pgfqpoint{3.331200in}{2.100000in}}%
\pgfpathlineto{\pgfqpoint{3.333680in}{2.590000in}}%
\pgfpathlineto{\pgfqpoint{3.334920in}{2.170000in}}%
\pgfpathlineto{\pgfqpoint{3.336160in}{2.660000in}}%
\pgfpathlineto{\pgfqpoint{3.337400in}{2.380000in}}%
\pgfpathlineto{\pgfqpoint{3.338640in}{2.695000in}}%
\pgfpathlineto{\pgfqpoint{3.342360in}{2.205000in}}%
\pgfpathlineto{\pgfqpoint{3.343600in}{2.275000in}}%
\pgfpathlineto{\pgfqpoint{3.344840in}{2.695000in}}%
\pgfpathlineto{\pgfqpoint{3.348560in}{2.135000in}}%
\pgfpathlineto{\pgfqpoint{3.349800in}{2.380000in}}%
\pgfpathlineto{\pgfqpoint{3.352280in}{2.380000in}}%
\pgfpathlineto{\pgfqpoint{3.353520in}{2.310000in}}%
\pgfpathlineto{\pgfqpoint{3.354760in}{2.520000in}}%
\pgfpathlineto{\pgfqpoint{3.356000in}{2.345000in}}%
\pgfpathlineto{\pgfqpoint{3.357240in}{2.765000in}}%
\pgfpathlineto{\pgfqpoint{3.358480in}{2.240000in}}%
\pgfpathlineto{\pgfqpoint{3.360960in}{2.485000in}}%
\pgfpathlineto{\pgfqpoint{3.362200in}{2.205000in}}%
\pgfpathlineto{\pgfqpoint{3.363440in}{2.660000in}}%
\pgfpathlineto{\pgfqpoint{3.364680in}{2.170000in}}%
\pgfpathlineto{\pgfqpoint{3.368400in}{2.590000in}}%
\pgfpathlineto{\pgfqpoint{3.369640in}{2.275000in}}%
\pgfpathlineto{\pgfqpoint{3.370880in}{2.415000in}}%
\pgfpathlineto{\pgfqpoint{3.372120in}{2.135000in}}%
\pgfpathlineto{\pgfqpoint{3.373360in}{2.485000in}}%
\pgfpathlineto{\pgfqpoint{3.374600in}{2.170000in}}%
\pgfpathlineto{\pgfqpoint{3.375840in}{2.275000in}}%
\pgfpathlineto{\pgfqpoint{3.377080in}{2.135000in}}%
\pgfpathlineto{\pgfqpoint{3.378320in}{2.345000in}}%
\pgfpathlineto{\pgfqpoint{3.379560in}{2.100000in}}%
\pgfpathlineto{\pgfqpoint{3.380800in}{2.100000in}}%
\pgfpathlineto{\pgfqpoint{3.382040in}{2.065000in}}%
\pgfpathlineto{\pgfqpoint{3.383280in}{1.925000in}}%
\pgfpathlineto{\pgfqpoint{3.384520in}{2.555000in}}%
\pgfpathlineto{\pgfqpoint{3.385760in}{2.310000in}}%
\pgfpathlineto{\pgfqpoint{3.387000in}{2.310000in}}%
\pgfpathlineto{\pgfqpoint{3.388240in}{2.170000in}}%
\pgfpathlineto{\pgfqpoint{3.389480in}{2.590000in}}%
\pgfpathlineto{\pgfqpoint{3.390720in}{2.415000in}}%
\pgfpathlineto{\pgfqpoint{3.391960in}{2.485000in}}%
\pgfpathlineto{\pgfqpoint{3.393200in}{2.625000in}}%
\pgfpathlineto{\pgfqpoint{3.396920in}{2.170000in}}%
\pgfpathlineto{\pgfqpoint{3.398160in}{2.240000in}}%
\pgfpathlineto{\pgfqpoint{3.399400in}{2.205000in}}%
\pgfpathlineto{\pgfqpoint{3.400640in}{2.625000in}}%
\pgfpathlineto{\pgfqpoint{3.401880in}{2.415000in}}%
\pgfpathlineto{\pgfqpoint{3.403120in}{2.450000in}}%
\pgfpathlineto{\pgfqpoint{3.404360in}{2.450000in}}%
\pgfpathlineto{\pgfqpoint{3.405600in}{2.380000in}}%
\pgfpathlineto{\pgfqpoint{3.406840in}{2.450000in}}%
\pgfpathlineto{\pgfqpoint{3.408080in}{2.415000in}}%
\pgfpathlineto{\pgfqpoint{3.409320in}{2.485000in}}%
\pgfpathlineto{\pgfqpoint{3.410560in}{2.240000in}}%
\pgfpathlineto{\pgfqpoint{3.411800in}{2.660000in}}%
\pgfpathlineto{\pgfqpoint{3.413040in}{2.240000in}}%
\pgfpathlineto{\pgfqpoint{3.414280in}{2.275000in}}%
\pgfpathlineto{\pgfqpoint{3.415520in}{2.730000in}}%
\pgfpathlineto{\pgfqpoint{3.416760in}{2.415000in}}%
\pgfpathlineto{\pgfqpoint{3.418000in}{2.765000in}}%
\pgfpathlineto{\pgfqpoint{3.419240in}{2.590000in}}%
\pgfpathlineto{\pgfqpoint{3.421720in}{2.695000in}}%
\pgfpathlineto{\pgfqpoint{3.424200in}{2.380000in}}%
\pgfpathlineto{\pgfqpoint{3.425440in}{2.135000in}}%
\pgfpathlineto{\pgfqpoint{3.426680in}{2.135000in}}%
\pgfpathlineto{\pgfqpoint{3.427920in}{2.485000in}}%
\pgfpathlineto{\pgfqpoint{3.429160in}{2.240000in}}%
\pgfpathlineto{\pgfqpoint{3.431640in}{2.380000in}}%
\pgfpathlineto{\pgfqpoint{3.432880in}{2.310000in}}%
\pgfpathlineto{\pgfqpoint{3.434120in}{2.555000in}}%
\pgfpathlineto{\pgfqpoint{3.436600in}{2.205000in}}%
\pgfpathlineto{\pgfqpoint{3.437840in}{1.925000in}}%
\pgfpathlineto{\pgfqpoint{3.439080in}{2.415000in}}%
\pgfpathlineto{\pgfqpoint{3.440320in}{2.345000in}}%
\pgfpathlineto{\pgfqpoint{3.441560in}{2.695000in}}%
\pgfpathlineto{\pgfqpoint{3.442800in}{2.415000in}}%
\pgfpathlineto{\pgfqpoint{3.444040in}{2.730000in}}%
\pgfpathlineto{\pgfqpoint{3.446520in}{2.730000in}}%
\pgfpathlineto{\pgfqpoint{3.447760in}{2.520000in}}%
\pgfpathlineto{\pgfqpoint{3.449000in}{2.905000in}}%
\pgfpathlineto{\pgfqpoint{3.451480in}{2.415000in}}%
\pgfpathlineto{\pgfqpoint{3.453960in}{2.625000in}}%
\pgfpathlineto{\pgfqpoint{3.455200in}{2.275000in}}%
\pgfpathlineto{\pgfqpoint{3.457680in}{2.870000in}}%
\pgfpathlineto{\pgfqpoint{3.458920in}{2.765000in}}%
\pgfpathlineto{\pgfqpoint{3.461400in}{2.345000in}}%
\pgfpathlineto{\pgfqpoint{3.462640in}{2.520000in}}%
\pgfpathlineto{\pgfqpoint{3.465120in}{2.240000in}}%
\pgfpathlineto{\pgfqpoint{3.466360in}{2.520000in}}%
\pgfpathlineto{\pgfqpoint{3.467600in}{2.240000in}}%
\pgfpathlineto{\pgfqpoint{3.468840in}{2.380000in}}%
\pgfpathlineto{\pgfqpoint{3.470080in}{2.905000in}}%
\pgfpathlineto{\pgfqpoint{3.473800in}{2.345000in}}%
\pgfpathlineto{\pgfqpoint{3.475040in}{2.590000in}}%
\pgfpathlineto{\pgfqpoint{3.478760in}{2.030000in}}%
\pgfpathlineto{\pgfqpoint{3.480000in}{2.590000in}}%
\pgfpathlineto{\pgfqpoint{3.481240in}{2.380000in}}%
\pgfpathlineto{\pgfqpoint{3.482480in}{2.380000in}}%
\pgfpathlineto{\pgfqpoint{3.483720in}{2.835000in}}%
\pgfpathlineto{\pgfqpoint{3.484960in}{2.660000in}}%
\pgfpathlineto{\pgfqpoint{3.486200in}{2.905000in}}%
\pgfpathlineto{\pgfqpoint{3.487440in}{2.450000in}}%
\pgfpathlineto{\pgfqpoint{3.488680in}{2.450000in}}%
\pgfpathlineto{\pgfqpoint{3.489920in}{2.275000in}}%
\pgfpathlineto{\pgfqpoint{3.491160in}{2.555000in}}%
\pgfpathlineto{\pgfqpoint{3.493640in}{2.310000in}}%
\pgfpathlineto{\pgfqpoint{3.494880in}{2.380000in}}%
\pgfpathlineto{\pgfqpoint{3.496120in}{2.590000in}}%
\pgfpathlineto{\pgfqpoint{3.497360in}{2.310000in}}%
\pgfpathlineto{\pgfqpoint{3.498600in}{2.625000in}}%
\pgfpathlineto{\pgfqpoint{3.499840in}{2.380000in}}%
\pgfpathlineto{\pgfqpoint{3.502320in}{2.695000in}}%
\pgfpathlineto{\pgfqpoint{3.503560in}{2.520000in}}%
\pgfpathlineto{\pgfqpoint{3.504800in}{2.520000in}}%
\pgfpathlineto{\pgfqpoint{3.506040in}{2.415000in}}%
\pgfpathlineto{\pgfqpoint{3.507280in}{2.555000in}}%
\pgfpathlineto{\pgfqpoint{3.508520in}{2.310000in}}%
\pgfpathlineto{\pgfqpoint{3.509760in}{2.310000in}}%
\pgfpathlineto{\pgfqpoint{3.511000in}{2.625000in}}%
\pgfpathlineto{\pgfqpoint{3.513480in}{2.170000in}}%
\pgfpathlineto{\pgfqpoint{3.514720in}{2.905000in}}%
\pgfpathlineto{\pgfqpoint{3.515960in}{2.100000in}}%
\pgfpathlineto{\pgfqpoint{3.517200in}{2.555000in}}%
\pgfpathlineto{\pgfqpoint{3.518440in}{2.170000in}}%
\pgfpathlineto{\pgfqpoint{3.520920in}{2.590000in}}%
\pgfpathlineto{\pgfqpoint{3.522160in}{2.240000in}}%
\pgfpathlineto{\pgfqpoint{3.523400in}{2.275000in}}%
\pgfpathlineto{\pgfqpoint{3.524640in}{2.555000in}}%
\pgfpathlineto{\pgfqpoint{3.525880in}{2.520000in}}%
\pgfpathlineto{\pgfqpoint{3.527120in}{2.695000in}}%
\pgfpathlineto{\pgfqpoint{3.528360in}{2.625000in}}%
\pgfpathlineto{\pgfqpoint{3.529600in}{2.345000in}}%
\pgfpathlineto{\pgfqpoint{3.532080in}{2.590000in}}%
\pgfpathlineto{\pgfqpoint{3.533320in}{2.590000in}}%
\pgfpathlineto{\pgfqpoint{3.534560in}{2.415000in}}%
\pgfpathlineto{\pgfqpoint{3.535800in}{2.660000in}}%
\pgfpathlineto{\pgfqpoint{3.537040in}{2.205000in}}%
\pgfpathlineto{\pgfqpoint{3.538280in}{2.730000in}}%
\pgfpathlineto{\pgfqpoint{3.539520in}{2.240000in}}%
\pgfpathlineto{\pgfqpoint{3.540760in}{2.555000in}}%
\pgfpathlineto{\pgfqpoint{3.542000in}{2.450000in}}%
\pgfpathlineto{\pgfqpoint{3.543240in}{2.205000in}}%
\pgfpathlineto{\pgfqpoint{3.545720in}{2.625000in}}%
\pgfpathlineto{\pgfqpoint{3.548200in}{2.205000in}}%
\pgfpathlineto{\pgfqpoint{3.549440in}{2.275000in}}%
\pgfpathlineto{\pgfqpoint{3.550680in}{2.590000in}}%
\pgfpathlineto{\pgfqpoint{3.553160in}{2.100000in}}%
\pgfpathlineto{\pgfqpoint{3.554400in}{2.905000in}}%
\pgfpathlineto{\pgfqpoint{3.555640in}{2.240000in}}%
\pgfpathlineto{\pgfqpoint{3.556880in}{2.485000in}}%
\pgfpathlineto{\pgfqpoint{3.558120in}{2.135000in}}%
\pgfpathlineto{\pgfqpoint{3.559360in}{2.100000in}}%
\pgfpathlineto{\pgfqpoint{3.560600in}{2.135000in}}%
\pgfpathlineto{\pgfqpoint{3.561840in}{2.625000in}}%
\pgfpathlineto{\pgfqpoint{3.564320in}{2.520000in}}%
\pgfpathlineto{\pgfqpoint{3.566800in}{2.275000in}}%
\pgfpathlineto{\pgfqpoint{3.568040in}{2.170000in}}%
\pgfpathlineto{\pgfqpoint{3.569280in}{2.485000in}}%
\pgfpathlineto{\pgfqpoint{3.571760in}{2.310000in}}%
\pgfpathlineto{\pgfqpoint{3.573000in}{2.170000in}}%
\pgfpathlineto{\pgfqpoint{3.575480in}{2.625000in}}%
\pgfpathlineto{\pgfqpoint{3.576720in}{2.065000in}}%
\pgfpathlineto{\pgfqpoint{3.577960in}{2.625000in}}%
\pgfpathlineto{\pgfqpoint{3.579200in}{2.660000in}}%
\pgfpathlineto{\pgfqpoint{3.580440in}{2.625000in}}%
\pgfpathlineto{\pgfqpoint{3.581680in}{2.415000in}}%
\pgfpathlineto{\pgfqpoint{3.582920in}{2.590000in}}%
\pgfpathlineto{\pgfqpoint{3.584160in}{2.450000in}}%
\pgfpathlineto{\pgfqpoint{3.585400in}{3.010000in}}%
\pgfpathlineto{\pgfqpoint{3.586640in}{2.345000in}}%
\pgfpathlineto{\pgfqpoint{3.587880in}{2.520000in}}%
\pgfpathlineto{\pgfqpoint{3.589120in}{2.485000in}}%
\pgfpathlineto{\pgfqpoint{3.590360in}{2.695000in}}%
\pgfpathlineto{\pgfqpoint{3.594080in}{1.855000in}}%
\pgfpathlineto{\pgfqpoint{3.595320in}{2.205000in}}%
\pgfpathlineto{\pgfqpoint{3.596560in}{2.170000in}}%
\pgfpathlineto{\pgfqpoint{3.597800in}{2.380000in}}%
\pgfpathlineto{\pgfqpoint{3.599040in}{2.170000in}}%
\pgfpathlineto{\pgfqpoint{3.600280in}{2.590000in}}%
\pgfpathlineto{\pgfqpoint{3.601520in}{2.555000in}}%
\pgfpathlineto{\pgfqpoint{3.602760in}{2.100000in}}%
\pgfpathlineto{\pgfqpoint{3.605240in}{2.275000in}}%
\pgfpathlineto{\pgfqpoint{3.607720in}{1.960000in}}%
\pgfpathlineto{\pgfqpoint{3.608960in}{2.310000in}}%
\pgfpathlineto{\pgfqpoint{3.610200in}{2.345000in}}%
\pgfpathlineto{\pgfqpoint{3.611440in}{2.275000in}}%
\pgfpathlineto{\pgfqpoint{3.612680in}{2.275000in}}%
\pgfpathlineto{\pgfqpoint{3.613920in}{2.380000in}}%
\pgfpathlineto{\pgfqpoint{3.615160in}{2.205000in}}%
\pgfpathlineto{\pgfqpoint{3.616400in}{2.345000in}}%
\pgfpathlineto{\pgfqpoint{3.617640in}{2.345000in}}%
\pgfpathlineto{\pgfqpoint{3.618880in}{2.415000in}}%
\pgfpathlineto{\pgfqpoint{3.620120in}{1.925000in}}%
\pgfpathlineto{\pgfqpoint{3.622600in}{2.415000in}}%
\pgfpathlineto{\pgfqpoint{3.623840in}{2.240000in}}%
\pgfpathlineto{\pgfqpoint{3.625080in}{2.835000in}}%
\pgfpathlineto{\pgfqpoint{3.626320in}{2.555000in}}%
\pgfpathlineto{\pgfqpoint{3.627560in}{1.960000in}}%
\pgfpathlineto{\pgfqpoint{3.628800in}{2.520000in}}%
\pgfpathlineto{\pgfqpoint{3.631280in}{1.995000in}}%
\pgfpathlineto{\pgfqpoint{3.632520in}{2.275000in}}%
\pgfpathlineto{\pgfqpoint{3.633760in}{2.275000in}}%
\pgfpathlineto{\pgfqpoint{3.635000in}{2.485000in}}%
\pgfpathlineto{\pgfqpoint{3.637480in}{2.310000in}}%
\pgfpathlineto{\pgfqpoint{3.638720in}{2.170000in}}%
\pgfpathlineto{\pgfqpoint{3.639960in}{2.660000in}}%
\pgfpathlineto{\pgfqpoint{3.641200in}{2.625000in}}%
\pgfpathlineto{\pgfqpoint{3.643680in}{2.345000in}}%
\pgfpathlineto{\pgfqpoint{3.644920in}{2.450000in}}%
\pgfpathlineto{\pgfqpoint{3.646160in}{2.835000in}}%
\pgfpathlineto{\pgfqpoint{3.647400in}{2.345000in}}%
\pgfpathlineto{\pgfqpoint{3.648640in}{2.660000in}}%
\pgfpathlineto{\pgfqpoint{3.649880in}{2.625000in}}%
\pgfpathlineto{\pgfqpoint{3.651120in}{2.310000in}}%
\pgfpathlineto{\pgfqpoint{3.652360in}{2.345000in}}%
\pgfpathlineto{\pgfqpoint{3.653600in}{2.450000in}}%
\pgfpathlineto{\pgfqpoint{3.654840in}{2.730000in}}%
\pgfpathlineto{\pgfqpoint{3.656080in}{2.135000in}}%
\pgfpathlineto{\pgfqpoint{3.657320in}{2.660000in}}%
\pgfpathlineto{\pgfqpoint{3.658560in}{2.660000in}}%
\pgfpathlineto{\pgfqpoint{3.659800in}{2.205000in}}%
\pgfpathlineto{\pgfqpoint{3.661040in}{2.485000in}}%
\pgfpathlineto{\pgfqpoint{3.662280in}{2.450000in}}%
\pgfpathlineto{\pgfqpoint{3.663520in}{2.485000in}}%
\pgfpathlineto{\pgfqpoint{3.666000in}{2.275000in}}%
\pgfpathlineto{\pgfqpoint{3.669720in}{2.695000in}}%
\pgfpathlineto{\pgfqpoint{3.670960in}{2.170000in}}%
\pgfpathlineto{\pgfqpoint{3.673440in}{2.520000in}}%
\pgfpathlineto{\pgfqpoint{3.674680in}{2.625000in}}%
\pgfpathlineto{\pgfqpoint{3.675920in}{2.310000in}}%
\pgfpathlineto{\pgfqpoint{3.677160in}{2.345000in}}%
\pgfpathlineto{\pgfqpoint{3.678400in}{2.100000in}}%
\pgfpathlineto{\pgfqpoint{3.680880in}{2.625000in}}%
\pgfpathlineto{\pgfqpoint{3.682120in}{2.415000in}}%
\pgfpathlineto{\pgfqpoint{3.683360in}{2.625000in}}%
\pgfpathlineto{\pgfqpoint{3.684600in}{1.925000in}}%
\pgfpathlineto{\pgfqpoint{3.685840in}{2.625000in}}%
\pgfpathlineto{\pgfqpoint{3.687080in}{2.695000in}}%
\pgfpathlineto{\pgfqpoint{3.689560in}{2.520000in}}%
\pgfpathlineto{\pgfqpoint{3.690800in}{2.590000in}}%
\pgfpathlineto{\pgfqpoint{3.692040in}{2.450000in}}%
\pgfpathlineto{\pgfqpoint{3.693280in}{2.450000in}}%
\pgfpathlineto{\pgfqpoint{3.694520in}{2.415000in}}%
\pgfpathlineto{\pgfqpoint{3.698240in}{2.835000in}}%
\pgfpathlineto{\pgfqpoint{3.699480in}{2.590000in}}%
\pgfpathlineto{\pgfqpoint{3.700720in}{2.905000in}}%
\pgfpathlineto{\pgfqpoint{3.701960in}{2.345000in}}%
\pgfpathlineto{\pgfqpoint{3.703200in}{2.625000in}}%
\pgfpathlineto{\pgfqpoint{3.704440in}{2.555000in}}%
\pgfpathlineto{\pgfqpoint{3.705680in}{2.310000in}}%
\pgfpathlineto{\pgfqpoint{3.706920in}{2.310000in}}%
\pgfpathlineto{\pgfqpoint{3.708160in}{2.485000in}}%
\pgfpathlineto{\pgfqpoint{3.709400in}{2.380000in}}%
\pgfpathlineto{\pgfqpoint{3.710640in}{2.730000in}}%
\pgfpathlineto{\pgfqpoint{3.711880in}{2.380000in}}%
\pgfpathlineto{\pgfqpoint{3.713120in}{2.625000in}}%
\pgfpathlineto{\pgfqpoint{3.714360in}{2.590000in}}%
\pgfpathlineto{\pgfqpoint{3.715600in}{2.030000in}}%
\pgfpathlineto{\pgfqpoint{3.718080in}{2.485000in}}%
\pgfpathlineto{\pgfqpoint{3.719320in}{2.450000in}}%
\pgfpathlineto{\pgfqpoint{3.720560in}{2.485000in}}%
\pgfpathlineto{\pgfqpoint{3.721800in}{2.590000in}}%
\pgfpathlineto{\pgfqpoint{3.724280in}{2.520000in}}%
\pgfpathlineto{\pgfqpoint{3.726760in}{2.660000in}}%
\pgfpathlineto{\pgfqpoint{3.728000in}{2.345000in}}%
\pgfpathlineto{\pgfqpoint{3.729240in}{2.625000in}}%
\pgfpathlineto{\pgfqpoint{3.731720in}{2.205000in}}%
\pgfpathlineto{\pgfqpoint{3.732960in}{2.555000in}}%
\pgfpathlineto{\pgfqpoint{3.734200in}{2.135000in}}%
\pgfpathlineto{\pgfqpoint{3.736680in}{2.835000in}}%
\pgfpathlineto{\pgfqpoint{3.739160in}{2.030000in}}%
\pgfpathlineto{\pgfqpoint{3.740400in}{2.555000in}}%
\pgfpathlineto{\pgfqpoint{3.742880in}{2.205000in}}%
\pgfpathlineto{\pgfqpoint{3.744120in}{2.030000in}}%
\pgfpathlineto{\pgfqpoint{3.745360in}{2.450000in}}%
\pgfpathlineto{\pgfqpoint{3.746600in}{2.205000in}}%
\pgfpathlineto{\pgfqpoint{3.747840in}{2.520000in}}%
\pgfpathlineto{\pgfqpoint{3.749080in}{2.380000in}}%
\pgfpathlineto{\pgfqpoint{3.750320in}{2.520000in}}%
\pgfpathlineto{\pgfqpoint{3.751560in}{2.345000in}}%
\pgfpathlineto{\pgfqpoint{3.752800in}{2.660000in}}%
\pgfpathlineto{\pgfqpoint{3.755280in}{2.240000in}}%
\pgfpathlineto{\pgfqpoint{3.756520in}{2.590000in}}%
\pgfpathlineto{\pgfqpoint{3.757760in}{2.520000in}}%
\pgfpathlineto{\pgfqpoint{3.759000in}{2.590000in}}%
\pgfpathlineto{\pgfqpoint{3.761480in}{2.240000in}}%
\pgfpathlineto{\pgfqpoint{3.762720in}{2.240000in}}%
\pgfpathlineto{\pgfqpoint{3.765200in}{2.520000in}}%
\pgfpathlineto{\pgfqpoint{3.767680in}{2.240000in}}%
\pgfpathlineto{\pgfqpoint{3.768920in}{2.555000in}}%
\pgfpathlineto{\pgfqpoint{3.770160in}{2.485000in}}%
\pgfpathlineto{\pgfqpoint{3.771400in}{2.520000in}}%
\pgfpathlineto{\pgfqpoint{3.772640in}{2.100000in}}%
\pgfpathlineto{\pgfqpoint{3.773880in}{2.485000in}}%
\pgfpathlineto{\pgfqpoint{3.775120in}{2.170000in}}%
\pgfpathlineto{\pgfqpoint{3.776360in}{2.415000in}}%
\pgfpathlineto{\pgfqpoint{3.777600in}{2.345000in}}%
\pgfpathlineto{\pgfqpoint{3.778840in}{2.555000in}}%
\pgfpathlineto{\pgfqpoint{3.780080in}{2.240000in}}%
\pgfpathlineto{\pgfqpoint{3.781320in}{2.520000in}}%
\pgfpathlineto{\pgfqpoint{3.783800in}{2.345000in}}%
\pgfpathlineto{\pgfqpoint{3.786280in}{2.205000in}}%
\pgfpathlineto{\pgfqpoint{3.788760in}{2.485000in}}%
\pgfpathlineto{\pgfqpoint{3.790000in}{2.205000in}}%
\pgfpathlineto{\pgfqpoint{3.791240in}{2.520000in}}%
\pgfpathlineto{\pgfqpoint{3.793720in}{2.310000in}}%
\pgfpathlineto{\pgfqpoint{3.796200in}{2.625000in}}%
\pgfpathlineto{\pgfqpoint{3.797440in}{2.625000in}}%
\pgfpathlineto{\pgfqpoint{3.798680in}{2.275000in}}%
\pgfpathlineto{\pgfqpoint{3.799920in}{2.415000in}}%
\pgfpathlineto{\pgfqpoint{3.801160in}{2.275000in}}%
\pgfpathlineto{\pgfqpoint{3.802400in}{2.275000in}}%
\pgfpathlineto{\pgfqpoint{3.803640in}{2.205000in}}%
\pgfpathlineto{\pgfqpoint{3.804880in}{2.240000in}}%
\pgfpathlineto{\pgfqpoint{3.806120in}{2.310000in}}%
\pgfpathlineto{\pgfqpoint{3.808600in}{2.555000in}}%
\pgfpathlineto{\pgfqpoint{3.809840in}{2.520000in}}%
\pgfpathlineto{\pgfqpoint{3.811080in}{2.660000in}}%
\pgfpathlineto{\pgfqpoint{3.813560in}{2.415000in}}%
\pgfpathlineto{\pgfqpoint{3.814800in}{2.345000in}}%
\pgfpathlineto{\pgfqpoint{3.816040in}{2.590000in}}%
\pgfpathlineto{\pgfqpoint{3.818520in}{2.345000in}}%
\pgfpathlineto{\pgfqpoint{3.819760in}{2.310000in}}%
\pgfpathlineto{\pgfqpoint{3.822240in}{2.450000in}}%
\pgfpathlineto{\pgfqpoint{3.823480in}{2.520000in}}%
\pgfpathlineto{\pgfqpoint{3.824720in}{2.450000in}}%
\pgfpathlineto{\pgfqpoint{3.825960in}{2.765000in}}%
\pgfpathlineto{\pgfqpoint{3.827200in}{2.590000in}}%
\pgfpathlineto{\pgfqpoint{3.828440in}{2.205000in}}%
\pgfpathlineto{\pgfqpoint{3.829680in}{2.660000in}}%
\pgfpathlineto{\pgfqpoint{3.830920in}{2.590000in}}%
\pgfpathlineto{\pgfqpoint{3.832160in}{2.310000in}}%
\pgfpathlineto{\pgfqpoint{3.834640in}{2.625000in}}%
\pgfpathlineto{\pgfqpoint{3.835880in}{2.905000in}}%
\pgfpathlineto{\pgfqpoint{3.837120in}{2.555000in}}%
\pgfpathlineto{\pgfqpoint{3.838360in}{2.835000in}}%
\pgfpathlineto{\pgfqpoint{3.840840in}{2.310000in}}%
\pgfpathlineto{\pgfqpoint{3.842080in}{2.275000in}}%
\pgfpathlineto{\pgfqpoint{3.843320in}{2.135000in}}%
\pgfpathlineto{\pgfqpoint{3.844560in}{2.555000in}}%
\pgfpathlineto{\pgfqpoint{3.845800in}{2.240000in}}%
\pgfpathlineto{\pgfqpoint{3.847040in}{2.310000in}}%
\pgfpathlineto{\pgfqpoint{3.848280in}{2.800000in}}%
\pgfpathlineto{\pgfqpoint{3.850760in}{2.380000in}}%
\pgfpathlineto{\pgfqpoint{3.852000in}{2.415000in}}%
\pgfpathlineto{\pgfqpoint{3.853240in}{2.625000in}}%
\pgfpathlineto{\pgfqpoint{3.854480in}{2.345000in}}%
\pgfpathlineto{\pgfqpoint{3.855720in}{2.625000in}}%
\pgfpathlineto{\pgfqpoint{3.859440in}{2.625000in}}%
\pgfpathlineto{\pgfqpoint{3.861920in}{2.450000in}}%
\pgfpathlineto{\pgfqpoint{3.863160in}{2.590000in}}%
\pgfpathlineto{\pgfqpoint{3.864400in}{2.520000in}}%
\pgfpathlineto{\pgfqpoint{3.865640in}{2.800000in}}%
\pgfpathlineto{\pgfqpoint{3.866880in}{2.695000in}}%
\pgfpathlineto{\pgfqpoint{3.868120in}{2.240000in}}%
\pgfpathlineto{\pgfqpoint{3.870600in}{2.485000in}}%
\pgfpathlineto{\pgfqpoint{3.871840in}{2.205000in}}%
\pgfpathlineto{\pgfqpoint{3.873080in}{2.555000in}}%
\pgfpathlineto{\pgfqpoint{3.874320in}{2.590000in}}%
\pgfpathlineto{\pgfqpoint{3.875560in}{2.590000in}}%
\pgfpathlineto{\pgfqpoint{3.876800in}{2.205000in}}%
\pgfpathlineto{\pgfqpoint{3.878040in}{2.170000in}}%
\pgfpathlineto{\pgfqpoint{3.880520in}{2.590000in}}%
\pgfpathlineto{\pgfqpoint{3.881760in}{2.415000in}}%
\pgfpathlineto{\pgfqpoint{3.883000in}{2.415000in}}%
\pgfpathlineto{\pgfqpoint{3.884240in}{2.590000in}}%
\pgfpathlineto{\pgfqpoint{3.885480in}{2.380000in}}%
\pgfpathlineto{\pgfqpoint{3.886720in}{2.555000in}}%
\pgfpathlineto{\pgfqpoint{3.887960in}{2.485000in}}%
\pgfpathlineto{\pgfqpoint{3.890440in}{2.660000in}}%
\pgfpathlineto{\pgfqpoint{3.892920in}{2.240000in}}%
\pgfpathlineto{\pgfqpoint{3.894160in}{2.940000in}}%
\pgfpathlineto{\pgfqpoint{3.895400in}{2.345000in}}%
\pgfpathlineto{\pgfqpoint{3.896640in}{2.345000in}}%
\pgfpathlineto{\pgfqpoint{3.897880in}{2.485000in}}%
\pgfpathlineto{\pgfqpoint{3.899120in}{2.030000in}}%
\pgfpathlineto{\pgfqpoint{3.900360in}{2.450000in}}%
\pgfpathlineto{\pgfqpoint{3.901600in}{2.380000in}}%
\pgfpathlineto{\pgfqpoint{3.902840in}{2.625000in}}%
\pgfpathlineto{\pgfqpoint{3.904080in}{2.345000in}}%
\pgfpathlineto{\pgfqpoint{3.905320in}{2.415000in}}%
\pgfpathlineto{\pgfqpoint{3.907800in}{1.995000in}}%
\pgfpathlineto{\pgfqpoint{3.910280in}{2.485000in}}%
\pgfpathlineto{\pgfqpoint{3.911520in}{2.450000in}}%
\pgfpathlineto{\pgfqpoint{3.912760in}{2.485000in}}%
\pgfpathlineto{\pgfqpoint{3.914000in}{2.345000in}}%
\pgfpathlineto{\pgfqpoint{3.915240in}{2.520000in}}%
\pgfpathlineto{\pgfqpoint{3.917720in}{2.275000in}}%
\pgfpathlineto{\pgfqpoint{3.918960in}{2.835000in}}%
\pgfpathlineto{\pgfqpoint{3.921440in}{2.240000in}}%
\pgfpathlineto{\pgfqpoint{3.922680in}{2.380000in}}%
\pgfpathlineto{\pgfqpoint{3.923920in}{2.345000in}}%
\pgfpathlineto{\pgfqpoint{3.926400in}{2.870000in}}%
\pgfpathlineto{\pgfqpoint{3.928880in}{2.450000in}}%
\pgfpathlineto{\pgfqpoint{3.930120in}{2.450000in}}%
\pgfpathlineto{\pgfqpoint{3.931360in}{2.380000in}}%
\pgfpathlineto{\pgfqpoint{3.932600in}{2.730000in}}%
\pgfpathlineto{\pgfqpoint{3.933840in}{2.695000in}}%
\pgfpathlineto{\pgfqpoint{3.935080in}{2.695000in}}%
\pgfpathlineto{\pgfqpoint{3.936320in}{2.380000in}}%
\pgfpathlineto{\pgfqpoint{3.938800in}{2.520000in}}%
\pgfpathlineto{\pgfqpoint{3.940040in}{2.240000in}}%
\pgfpathlineto{\pgfqpoint{3.941280in}{2.730000in}}%
\pgfpathlineto{\pgfqpoint{3.943760in}{2.730000in}}%
\pgfpathlineto{\pgfqpoint{3.945000in}{2.835000in}}%
\pgfpathlineto{\pgfqpoint{3.949960in}{2.170000in}}%
\pgfpathlineto{\pgfqpoint{3.951200in}{2.660000in}}%
\pgfpathlineto{\pgfqpoint{3.952440in}{2.520000in}}%
\pgfpathlineto{\pgfqpoint{3.953680in}{2.730000in}}%
\pgfpathlineto{\pgfqpoint{3.956160in}{2.310000in}}%
\pgfpathlineto{\pgfqpoint{3.959880in}{2.800000in}}%
\pgfpathlineto{\pgfqpoint{3.962360in}{2.450000in}}%
\pgfpathlineto{\pgfqpoint{3.963600in}{2.485000in}}%
\pgfpathlineto{\pgfqpoint{3.966080in}{2.065000in}}%
\pgfpathlineto{\pgfqpoint{3.967320in}{1.995000in}}%
\pgfpathlineto{\pgfqpoint{3.968560in}{2.730000in}}%
\pgfpathlineto{\pgfqpoint{3.969800in}{2.135000in}}%
\pgfpathlineto{\pgfqpoint{3.971040in}{2.170000in}}%
\pgfpathlineto{\pgfqpoint{3.972280in}{2.275000in}}%
\pgfpathlineto{\pgfqpoint{3.973520in}{2.660000in}}%
\pgfpathlineto{\pgfqpoint{3.974760in}{2.520000in}}%
\pgfpathlineto{\pgfqpoint{3.976000in}{2.240000in}}%
\pgfpathlineto{\pgfqpoint{3.977240in}{2.310000in}}%
\pgfpathlineto{\pgfqpoint{3.978480in}{2.240000in}}%
\pgfpathlineto{\pgfqpoint{3.979720in}{2.590000in}}%
\pgfpathlineto{\pgfqpoint{3.982200in}{2.415000in}}%
\pgfpathlineto{\pgfqpoint{3.983440in}{2.555000in}}%
\pgfpathlineto{\pgfqpoint{3.984680in}{2.555000in}}%
\pgfpathlineto{\pgfqpoint{3.985920in}{2.135000in}}%
\pgfpathlineto{\pgfqpoint{3.987160in}{2.380000in}}%
\pgfpathlineto{\pgfqpoint{3.989640in}{2.135000in}}%
\pgfpathlineto{\pgfqpoint{3.990880in}{1.995000in}}%
\pgfpathlineto{\pgfqpoint{3.992120in}{2.590000in}}%
\pgfpathlineto{\pgfqpoint{3.993360in}{2.625000in}}%
\pgfpathlineto{\pgfqpoint{3.994600in}{1.925000in}}%
\pgfpathlineto{\pgfqpoint{3.997080in}{2.275000in}}%
\pgfpathlineto{\pgfqpoint{3.998320in}{2.065000in}}%
\pgfpathlineto{\pgfqpoint{4.000800in}{2.520000in}}%
\pgfpathlineto{\pgfqpoint{4.002040in}{2.100000in}}%
\pgfpathlineto{\pgfqpoint{4.003280in}{2.135000in}}%
\pgfpathlineto{\pgfqpoint{4.004520in}{2.205000in}}%
\pgfpathlineto{\pgfqpoint{4.007000in}{2.555000in}}%
\pgfpathlineto{\pgfqpoint{4.009480in}{2.380000in}}%
\pgfpathlineto{\pgfqpoint{4.010720in}{2.310000in}}%
\pgfpathlineto{\pgfqpoint{4.011960in}{2.450000in}}%
\pgfpathlineto{\pgfqpoint{4.013200in}{1.855000in}}%
\pgfpathlineto{\pgfqpoint{4.014440in}{1.855000in}}%
\pgfpathlineto{\pgfqpoint{4.015680in}{1.925000in}}%
\pgfpathlineto{\pgfqpoint{4.016920in}{1.785000in}}%
\pgfpathlineto{\pgfqpoint{4.019400in}{2.065000in}}%
\pgfpathlineto{\pgfqpoint{4.020640in}{1.960000in}}%
\pgfpathlineto{\pgfqpoint{4.021880in}{1.995000in}}%
\pgfpathlineto{\pgfqpoint{4.023120in}{2.065000in}}%
\pgfpathlineto{\pgfqpoint{4.024360in}{2.450000in}}%
\pgfpathlineto{\pgfqpoint{4.025600in}{1.820000in}}%
\pgfpathlineto{\pgfqpoint{4.026840in}{2.100000in}}%
\pgfpathlineto{\pgfqpoint{4.028080in}{2.065000in}}%
\pgfpathlineto{\pgfqpoint{4.030560in}{2.485000in}}%
\pgfpathlineto{\pgfqpoint{4.031800in}{2.450000in}}%
\pgfpathlineto{\pgfqpoint{4.033040in}{2.450000in}}%
\pgfpathlineto{\pgfqpoint{4.034280in}{2.240000in}}%
\pgfpathlineto{\pgfqpoint{4.035520in}{2.485000in}}%
\pgfpathlineto{\pgfqpoint{4.036760in}{1.890000in}}%
\pgfpathlineto{\pgfqpoint{4.038000in}{2.380000in}}%
\pgfpathlineto{\pgfqpoint{4.039240in}{1.995000in}}%
\pgfpathlineto{\pgfqpoint{4.040480in}{2.555000in}}%
\pgfpathlineto{\pgfqpoint{4.041720in}{2.485000in}}%
\pgfpathlineto{\pgfqpoint{4.042960in}{1.855000in}}%
\pgfpathlineto{\pgfqpoint{4.044200in}{2.415000in}}%
\pgfpathlineto{\pgfqpoint{4.045440in}{1.820000in}}%
\pgfpathlineto{\pgfqpoint{4.046680in}{2.555000in}}%
\pgfpathlineto{\pgfqpoint{4.047920in}{2.100000in}}%
\pgfpathlineto{\pgfqpoint{4.049160in}{2.310000in}}%
\pgfpathlineto{\pgfqpoint{4.051640in}{2.135000in}}%
\pgfpathlineto{\pgfqpoint{4.052880in}{2.590000in}}%
\pgfpathlineto{\pgfqpoint{4.054120in}{2.310000in}}%
\pgfpathlineto{\pgfqpoint{4.055360in}{2.450000in}}%
\pgfpathlineto{\pgfqpoint{4.056600in}{2.765000in}}%
\pgfpathlineto{\pgfqpoint{4.057840in}{2.485000in}}%
\pgfpathlineto{\pgfqpoint{4.059080in}{2.695000in}}%
\pgfpathlineto{\pgfqpoint{4.060320in}{2.380000in}}%
\pgfpathlineto{\pgfqpoint{4.061560in}{2.590000in}}%
\pgfpathlineto{\pgfqpoint{4.062800in}{2.415000in}}%
\pgfpathlineto{\pgfqpoint{4.064040in}{2.485000in}}%
\pgfpathlineto{\pgfqpoint{4.065280in}{2.415000in}}%
\pgfpathlineto{\pgfqpoint{4.066520in}{2.555000in}}%
\pgfpathlineto{\pgfqpoint{4.067760in}{2.380000in}}%
\pgfpathlineto{\pgfqpoint{4.069000in}{2.590000in}}%
\pgfpathlineto{\pgfqpoint{4.071480in}{2.415000in}}%
\pgfpathlineto{\pgfqpoint{4.072720in}{2.205000in}}%
\pgfpathlineto{\pgfqpoint{4.075200in}{2.310000in}}%
\pgfpathlineto{\pgfqpoint{4.076440in}{2.800000in}}%
\pgfpathlineto{\pgfqpoint{4.077680in}{2.695000in}}%
\pgfpathlineto{\pgfqpoint{4.078920in}{2.275000in}}%
\pgfpathlineto{\pgfqpoint{4.080160in}{2.240000in}}%
\pgfpathlineto{\pgfqpoint{4.081400in}{2.520000in}}%
\pgfpathlineto{\pgfqpoint{4.082640in}{1.925000in}}%
\pgfpathlineto{\pgfqpoint{4.083880in}{2.555000in}}%
\pgfpathlineto{\pgfqpoint{4.085120in}{2.520000in}}%
\pgfpathlineto{\pgfqpoint{4.087600in}{2.380000in}}%
\pgfpathlineto{\pgfqpoint{4.088840in}{2.450000in}}%
\pgfpathlineto{\pgfqpoint{4.090080in}{2.170000in}}%
\pgfpathlineto{\pgfqpoint{4.091320in}{2.590000in}}%
\pgfpathlineto{\pgfqpoint{4.093800in}{1.820000in}}%
\pgfpathlineto{\pgfqpoint{4.096280in}{2.555000in}}%
\pgfpathlineto{\pgfqpoint{4.097520in}{2.310000in}}%
\pgfpathlineto{\pgfqpoint{4.100000in}{2.415000in}}%
\pgfpathlineto{\pgfqpoint{4.101240in}{2.170000in}}%
\pgfpathlineto{\pgfqpoint{4.102480in}{2.275000in}}%
\pgfpathlineto{\pgfqpoint{4.103720in}{2.275000in}}%
\pgfpathlineto{\pgfqpoint{4.106200in}{2.415000in}}%
\pgfpathlineto{\pgfqpoint{4.108680in}{2.345000in}}%
\pgfpathlineto{\pgfqpoint{4.109920in}{2.275000in}}%
\pgfpathlineto{\pgfqpoint{4.111160in}{2.345000in}}%
\pgfpathlineto{\pgfqpoint{4.112400in}{2.520000in}}%
\pgfpathlineto{\pgfqpoint{4.113640in}{2.065000in}}%
\pgfpathlineto{\pgfqpoint{4.114880in}{2.345000in}}%
\pgfpathlineto{\pgfqpoint{4.116120in}{2.310000in}}%
\pgfpathlineto{\pgfqpoint{4.117360in}{2.450000in}}%
\pgfpathlineto{\pgfqpoint{4.118600in}{2.450000in}}%
\pgfpathlineto{\pgfqpoint{4.119840in}{2.485000in}}%
\pgfpathlineto{\pgfqpoint{4.121080in}{2.310000in}}%
\pgfpathlineto{\pgfqpoint{4.124800in}{2.555000in}}%
\pgfpathlineto{\pgfqpoint{4.126040in}{2.205000in}}%
\pgfpathlineto{\pgfqpoint{4.128520in}{2.520000in}}%
\pgfpathlineto{\pgfqpoint{4.129760in}{2.240000in}}%
\pgfpathlineto{\pgfqpoint{4.132240in}{2.555000in}}%
\pgfpathlineto{\pgfqpoint{4.133480in}{2.415000in}}%
\pgfpathlineto{\pgfqpoint{4.134720in}{2.065000in}}%
\pgfpathlineto{\pgfqpoint{4.137200in}{2.625000in}}%
\pgfpathlineto{\pgfqpoint{4.138440in}{2.485000in}}%
\pgfpathlineto{\pgfqpoint{4.139680in}{2.800000in}}%
\pgfpathlineto{\pgfqpoint{4.140920in}{2.380000in}}%
\pgfpathlineto{\pgfqpoint{4.142160in}{2.345000in}}%
\pgfpathlineto{\pgfqpoint{4.143400in}{2.485000in}}%
\pgfpathlineto{\pgfqpoint{4.144640in}{2.485000in}}%
\pgfpathlineto{\pgfqpoint{4.145880in}{2.450000in}}%
\pgfpathlineto{\pgfqpoint{4.147120in}{2.695000in}}%
\pgfpathlineto{\pgfqpoint{4.148360in}{2.625000in}}%
\pgfpathlineto{\pgfqpoint{4.150840in}{2.765000in}}%
\pgfpathlineto{\pgfqpoint{4.152080in}{2.415000in}}%
\pgfpathlineto{\pgfqpoint{4.153320in}{2.555000in}}%
\pgfpathlineto{\pgfqpoint{4.154560in}{2.170000in}}%
\pgfpathlineto{\pgfqpoint{4.157040in}{2.415000in}}%
\pgfpathlineto{\pgfqpoint{4.158280in}{2.380000in}}%
\pgfpathlineto{\pgfqpoint{4.159520in}{1.925000in}}%
\pgfpathlineto{\pgfqpoint{4.160760in}{2.275000in}}%
\pgfpathlineto{\pgfqpoint{4.163240in}{2.275000in}}%
\pgfpathlineto{\pgfqpoint{4.164480in}{2.450000in}}%
\pgfpathlineto{\pgfqpoint{4.165720in}{2.205000in}}%
\pgfpathlineto{\pgfqpoint{4.166960in}{2.345000in}}%
\pgfpathlineto{\pgfqpoint{4.168200in}{2.310000in}}%
\pgfpathlineto{\pgfqpoint{4.170680in}{2.415000in}}%
\pgfpathlineto{\pgfqpoint{4.171920in}{2.590000in}}%
\pgfpathlineto{\pgfqpoint{4.174400in}{2.030000in}}%
\pgfpathlineto{\pgfqpoint{4.175640in}{2.170000in}}%
\pgfpathlineto{\pgfqpoint{4.176880in}{2.765000in}}%
\pgfpathlineto{\pgfqpoint{4.179360in}{2.030000in}}%
\pgfpathlineto{\pgfqpoint{4.180600in}{2.730000in}}%
\pgfpathlineto{\pgfqpoint{4.181840in}{2.205000in}}%
\pgfpathlineto{\pgfqpoint{4.183080in}{2.380000in}}%
\pgfpathlineto{\pgfqpoint{4.184320in}{2.100000in}}%
\pgfpathlineto{\pgfqpoint{4.185560in}{2.555000in}}%
\pgfpathlineto{\pgfqpoint{4.186800in}{2.485000in}}%
\pgfpathlineto{\pgfqpoint{4.188040in}{2.135000in}}%
\pgfpathlineto{\pgfqpoint{4.189280in}{2.660000in}}%
\pgfpathlineto{\pgfqpoint{4.190520in}{2.520000in}}%
\pgfpathlineto{\pgfqpoint{4.191760in}{2.835000in}}%
\pgfpathlineto{\pgfqpoint{4.193000in}{2.520000in}}%
\pgfpathlineto{\pgfqpoint{4.194240in}{2.870000in}}%
\pgfpathlineto{\pgfqpoint{4.195480in}{2.450000in}}%
\pgfpathlineto{\pgfqpoint{4.196720in}{2.660000in}}%
\pgfpathlineto{\pgfqpoint{4.197960in}{2.590000in}}%
\pgfpathlineto{\pgfqpoint{4.199200in}{2.835000in}}%
\pgfpathlineto{\pgfqpoint{4.200440in}{2.520000in}}%
\pgfpathlineto{\pgfqpoint{4.201680in}{2.555000in}}%
\pgfpathlineto{\pgfqpoint{4.202920in}{2.205000in}}%
\pgfpathlineto{\pgfqpoint{4.204160in}{2.625000in}}%
\pgfpathlineto{\pgfqpoint{4.205400in}{2.205000in}}%
\pgfpathlineto{\pgfqpoint{4.207880in}{2.800000in}}%
\pgfpathlineto{\pgfqpoint{4.209120in}{2.730000in}}%
\pgfpathlineto{\pgfqpoint{4.210360in}{2.905000in}}%
\pgfpathlineto{\pgfqpoint{4.212840in}{2.625000in}}%
\pgfpathlineto{\pgfqpoint{4.214080in}{2.660000in}}%
\pgfpathlineto{\pgfqpoint{4.217800in}{2.240000in}}%
\pgfpathlineto{\pgfqpoint{4.219040in}{2.275000in}}%
\pgfpathlineto{\pgfqpoint{4.220280in}{2.345000in}}%
\pgfpathlineto{\pgfqpoint{4.221520in}{2.345000in}}%
\pgfpathlineto{\pgfqpoint{4.222760in}{2.380000in}}%
\pgfpathlineto{\pgfqpoint{4.224000in}{1.960000in}}%
\pgfpathlineto{\pgfqpoint{4.226480in}{2.415000in}}%
\pgfpathlineto{\pgfqpoint{4.227720in}{2.275000in}}%
\pgfpathlineto{\pgfqpoint{4.232680in}{2.695000in}}%
\pgfpathlineto{\pgfqpoint{4.233920in}{2.345000in}}%
\pgfpathlineto{\pgfqpoint{4.235160in}{2.870000in}}%
\pgfpathlineto{\pgfqpoint{4.237640in}{2.415000in}}%
\pgfpathlineto{\pgfqpoint{4.238880in}{2.590000in}}%
\pgfpathlineto{\pgfqpoint{4.240120in}{1.925000in}}%
\pgfpathlineto{\pgfqpoint{4.242600in}{2.520000in}}%
\pgfpathlineto{\pgfqpoint{4.243840in}{2.520000in}}%
\pgfpathlineto{\pgfqpoint{4.245080in}{2.135000in}}%
\pgfpathlineto{\pgfqpoint{4.248800in}{2.940000in}}%
\pgfpathlineto{\pgfqpoint{4.250040in}{2.835000in}}%
\pgfpathlineto{\pgfqpoint{4.251280in}{2.835000in}}%
\pgfpathlineto{\pgfqpoint{4.252520in}{2.345000in}}%
\pgfpathlineto{\pgfqpoint{4.255000in}{2.765000in}}%
\pgfpathlineto{\pgfqpoint{4.257480in}{2.065000in}}%
\pgfpathlineto{\pgfqpoint{4.258720in}{2.485000in}}%
\pgfpathlineto{\pgfqpoint{4.259960in}{1.995000in}}%
\pgfpathlineto{\pgfqpoint{4.262440in}{2.625000in}}%
\pgfpathlineto{\pgfqpoint{4.263680in}{2.695000in}}%
\pgfpathlineto{\pgfqpoint{4.264920in}{2.695000in}}%
\pgfpathlineto{\pgfqpoint{4.267400in}{2.415000in}}%
\pgfpathlineto{\pgfqpoint{4.268640in}{2.555000in}}%
\pgfpathlineto{\pgfqpoint{4.269880in}{2.485000in}}%
\pgfpathlineto{\pgfqpoint{4.271120in}{2.170000in}}%
\pgfpathlineto{\pgfqpoint{4.272360in}{2.520000in}}%
\pgfpathlineto{\pgfqpoint{4.273600in}{2.485000in}}%
\pgfpathlineto{\pgfqpoint{4.276080in}{1.750000in}}%
\pgfpathlineto{\pgfqpoint{4.278560in}{2.695000in}}%
\pgfpathlineto{\pgfqpoint{4.281040in}{2.380000in}}%
\pgfpathlineto{\pgfqpoint{4.282280in}{2.450000in}}%
\pgfpathlineto{\pgfqpoint{4.283520in}{2.590000in}}%
\pgfpathlineto{\pgfqpoint{4.284760in}{2.380000in}}%
\pgfpathlineto{\pgfqpoint{4.286000in}{2.450000in}}%
\pgfpathlineto{\pgfqpoint{4.287240in}{2.380000in}}%
\pgfpathlineto{\pgfqpoint{4.288480in}{2.555000in}}%
\pgfpathlineto{\pgfqpoint{4.289720in}{2.205000in}}%
\pgfpathlineto{\pgfqpoint{4.292200in}{2.730000in}}%
\pgfpathlineto{\pgfqpoint{4.293440in}{2.415000in}}%
\pgfpathlineto{\pgfqpoint{4.294680in}{2.555000in}}%
\pgfpathlineto{\pgfqpoint{4.295920in}{2.240000in}}%
\pgfpathlineto{\pgfqpoint{4.297160in}{2.345000in}}%
\pgfpathlineto{\pgfqpoint{4.298400in}{2.590000in}}%
\pgfpathlineto{\pgfqpoint{4.300880in}{2.415000in}}%
\pgfpathlineto{\pgfqpoint{4.303360in}{2.695000in}}%
\pgfpathlineto{\pgfqpoint{4.304600in}{2.765000in}}%
\pgfpathlineto{\pgfqpoint{4.305840in}{2.380000in}}%
\pgfpathlineto{\pgfqpoint{4.307080in}{2.345000in}}%
\pgfpathlineto{\pgfqpoint{4.308320in}{2.205000in}}%
\pgfpathlineto{\pgfqpoint{4.309560in}{2.520000in}}%
\pgfpathlineto{\pgfqpoint{4.314520in}{1.715000in}}%
\pgfpathlineto{\pgfqpoint{4.317000in}{2.450000in}}%
\pgfpathlineto{\pgfqpoint{4.318240in}{2.625000in}}%
\pgfpathlineto{\pgfqpoint{4.321960in}{1.925000in}}%
\pgfpathlineto{\pgfqpoint{4.324440in}{2.240000in}}%
\pgfpathlineto{\pgfqpoint{4.325680in}{2.380000in}}%
\pgfpathlineto{\pgfqpoint{4.326920in}{2.380000in}}%
\pgfpathlineto{\pgfqpoint{4.328160in}{2.590000in}}%
\pgfpathlineto{\pgfqpoint{4.329400in}{2.415000in}}%
\pgfpathlineto{\pgfqpoint{4.330640in}{2.065000in}}%
\pgfpathlineto{\pgfqpoint{4.333120in}{2.625000in}}%
\pgfpathlineto{\pgfqpoint{4.334360in}{2.415000in}}%
\pgfpathlineto{\pgfqpoint{4.335600in}{2.450000in}}%
\pgfpathlineto{\pgfqpoint{4.338080in}{2.660000in}}%
\pgfpathlineto{\pgfqpoint{4.339320in}{2.625000in}}%
\pgfpathlineto{\pgfqpoint{4.340560in}{2.520000in}}%
\pgfpathlineto{\pgfqpoint{4.341800in}{2.695000in}}%
\pgfpathlineto{\pgfqpoint{4.343040in}{2.205000in}}%
\pgfpathlineto{\pgfqpoint{4.344280in}{2.555000in}}%
\pgfpathlineto{\pgfqpoint{4.345520in}{2.590000in}}%
\pgfpathlineto{\pgfqpoint{4.346760in}{2.730000in}}%
\pgfpathlineto{\pgfqpoint{4.348000in}{2.695000in}}%
\pgfpathlineto{\pgfqpoint{4.349240in}{2.590000in}}%
\pgfpathlineto{\pgfqpoint{4.350480in}{2.730000in}}%
\pgfpathlineto{\pgfqpoint{4.354200in}{2.240000in}}%
\pgfpathlineto{\pgfqpoint{4.355440in}{2.520000in}}%
\pgfpathlineto{\pgfqpoint{4.357920in}{2.415000in}}%
\pgfpathlineto{\pgfqpoint{4.359160in}{2.835000in}}%
\pgfpathlineto{\pgfqpoint{4.360400in}{2.555000in}}%
\pgfpathlineto{\pgfqpoint{4.361640in}{2.765000in}}%
\pgfpathlineto{\pgfqpoint{4.362880in}{2.415000in}}%
\pgfpathlineto{\pgfqpoint{4.364120in}{2.625000in}}%
\pgfpathlineto{\pgfqpoint{4.365360in}{2.100000in}}%
\pgfpathlineto{\pgfqpoint{4.366600in}{2.625000in}}%
\pgfpathlineto{\pgfqpoint{4.367840in}{2.590000in}}%
\pgfpathlineto{\pgfqpoint{4.370320in}{2.135000in}}%
\pgfpathlineto{\pgfqpoint{4.371560in}{2.065000in}}%
\pgfpathlineto{\pgfqpoint{4.374040in}{2.170000in}}%
\pgfpathlineto{\pgfqpoint{4.375280in}{2.380000in}}%
\pgfpathlineto{\pgfqpoint{4.376520in}{2.380000in}}%
\pgfpathlineto{\pgfqpoint{4.377760in}{2.275000in}}%
\pgfpathlineto{\pgfqpoint{4.379000in}{1.995000in}}%
\pgfpathlineto{\pgfqpoint{4.381480in}{2.520000in}}%
\pgfpathlineto{\pgfqpoint{4.382720in}{1.890000in}}%
\pgfpathlineto{\pgfqpoint{4.383960in}{2.695000in}}%
\pgfpathlineto{\pgfqpoint{4.386440in}{2.205000in}}%
\pgfpathlineto{\pgfqpoint{4.388920in}{2.100000in}}%
\pgfpathlineto{\pgfqpoint{4.390160in}{2.170000in}}%
\pgfpathlineto{\pgfqpoint{4.391400in}{1.925000in}}%
\pgfpathlineto{\pgfqpoint{4.392640in}{2.380000in}}%
\pgfpathlineto{\pgfqpoint{4.393880in}{1.855000in}}%
\pgfpathlineto{\pgfqpoint{4.396360in}{2.065000in}}%
\pgfpathlineto{\pgfqpoint{4.397600in}{2.065000in}}%
\pgfpathlineto{\pgfqpoint{4.400080in}{1.855000in}}%
\pgfpathlineto{\pgfqpoint{4.401320in}{2.485000in}}%
\pgfpathlineto{\pgfqpoint{4.402560in}{2.205000in}}%
\pgfpathlineto{\pgfqpoint{4.403800in}{2.310000in}}%
\pgfpathlineto{\pgfqpoint{4.406280in}{2.170000in}}%
\pgfpathlineto{\pgfqpoint{4.407520in}{2.520000in}}%
\pgfpathlineto{\pgfqpoint{4.408760in}{2.170000in}}%
\pgfpathlineto{\pgfqpoint{4.411240in}{2.450000in}}%
\pgfpathlineto{\pgfqpoint{4.412480in}{2.170000in}}%
\pgfpathlineto{\pgfqpoint{4.414960in}{2.765000in}}%
\pgfpathlineto{\pgfqpoint{4.418680in}{2.275000in}}%
\pgfpathlineto{\pgfqpoint{4.419920in}{2.310000in}}%
\pgfpathlineto{\pgfqpoint{4.421160in}{2.275000in}}%
\pgfpathlineto{\pgfqpoint{4.422400in}{2.275000in}}%
\pgfpathlineto{\pgfqpoint{4.423640in}{2.555000in}}%
\pgfpathlineto{\pgfqpoint{4.424880in}{2.170000in}}%
\pgfpathlineto{\pgfqpoint{4.426120in}{2.240000in}}%
\pgfpathlineto{\pgfqpoint{4.427360in}{2.555000in}}%
\pgfpathlineto{\pgfqpoint{4.428600in}{2.485000in}}%
\pgfpathlineto{\pgfqpoint{4.429840in}{2.625000in}}%
\pgfpathlineto{\pgfqpoint{4.431080in}{2.555000in}}%
\pgfpathlineto{\pgfqpoint{4.432320in}{2.275000in}}%
\pgfpathlineto{\pgfqpoint{4.436040in}{2.660000in}}%
\pgfpathlineto{\pgfqpoint{4.437280in}{2.345000in}}%
\pgfpathlineto{\pgfqpoint{4.438520in}{2.625000in}}%
\pgfpathlineto{\pgfqpoint{4.439760in}{2.590000in}}%
\pgfpathlineto{\pgfqpoint{4.441000in}{2.905000in}}%
\pgfpathlineto{\pgfqpoint{4.442240in}{2.450000in}}%
\pgfpathlineto{\pgfqpoint{4.444720in}{2.730000in}}%
\pgfpathlineto{\pgfqpoint{4.445960in}{2.520000in}}%
\pgfpathlineto{\pgfqpoint{4.447200in}{2.555000in}}%
\pgfpathlineto{\pgfqpoint{4.448440in}{2.520000in}}%
\pgfpathlineto{\pgfqpoint{4.449680in}{2.450000in}}%
\pgfpathlineto{\pgfqpoint{4.450920in}{2.485000in}}%
\pgfpathlineto{\pgfqpoint{4.452160in}{2.765000in}}%
\pgfpathlineto{\pgfqpoint{4.453400in}{2.730000in}}%
\pgfpathlineto{\pgfqpoint{4.454640in}{2.485000in}}%
\pgfpathlineto{\pgfqpoint{4.455880in}{2.485000in}}%
\pgfpathlineto{\pgfqpoint{4.457120in}{2.135000in}}%
\pgfpathlineto{\pgfqpoint{4.458360in}{2.660000in}}%
\pgfpathlineto{\pgfqpoint{4.459600in}{2.590000in}}%
\pgfpathlineto{\pgfqpoint{4.462080in}{2.590000in}}%
\pgfpathlineto{\pgfqpoint{4.464560in}{2.345000in}}%
\pgfpathlineto{\pgfqpoint{4.465800in}{2.730000in}}%
\pgfpathlineto{\pgfqpoint{4.467040in}{2.170000in}}%
\pgfpathlineto{\pgfqpoint{4.469520in}{2.660000in}}%
\pgfpathlineto{\pgfqpoint{4.470760in}{2.135000in}}%
\pgfpathlineto{\pgfqpoint{4.473240in}{2.730000in}}%
\pgfpathlineto{\pgfqpoint{4.474480in}{2.380000in}}%
\pgfpathlineto{\pgfqpoint{4.475720in}{2.345000in}}%
\pgfpathlineto{\pgfqpoint{4.476960in}{2.590000in}}%
\pgfpathlineto{\pgfqpoint{4.478200in}{2.170000in}}%
\pgfpathlineto{\pgfqpoint{4.481920in}{2.520000in}}%
\pgfpathlineto{\pgfqpoint{4.483160in}{2.275000in}}%
\pgfpathlineto{\pgfqpoint{4.485640in}{2.450000in}}%
\pgfpathlineto{\pgfqpoint{4.486880in}{2.275000in}}%
\pgfpathlineto{\pgfqpoint{4.488120in}{2.765000in}}%
\pgfpathlineto{\pgfqpoint{4.489360in}{2.695000in}}%
\pgfpathlineto{\pgfqpoint{4.490600in}{2.800000in}}%
\pgfpathlineto{\pgfqpoint{4.491840in}{2.555000in}}%
\pgfpathlineto{\pgfqpoint{4.493080in}{2.695000in}}%
\pgfpathlineto{\pgfqpoint{4.494320in}{2.520000in}}%
\pgfpathlineto{\pgfqpoint{4.495560in}{2.030000in}}%
\pgfpathlineto{\pgfqpoint{4.498040in}{2.170000in}}%
\pgfpathlineto{\pgfqpoint{4.499280in}{2.170000in}}%
\pgfpathlineto{\pgfqpoint{4.501760in}{2.380000in}}%
\pgfpathlineto{\pgfqpoint{4.503000in}{2.205000in}}%
\pgfpathlineto{\pgfqpoint{4.504240in}{2.450000in}}%
\pgfpathlineto{\pgfqpoint{4.505480in}{2.380000in}}%
\pgfpathlineto{\pgfqpoint{4.506720in}{2.100000in}}%
\pgfpathlineto{\pgfqpoint{4.507960in}{2.135000in}}%
\pgfpathlineto{\pgfqpoint{4.509200in}{2.730000in}}%
\pgfpathlineto{\pgfqpoint{4.511680in}{2.345000in}}%
\pgfpathlineto{\pgfqpoint{4.512920in}{2.555000in}}%
\pgfpathlineto{\pgfqpoint{4.514160in}{2.520000in}}%
\pgfpathlineto{\pgfqpoint{4.515400in}{2.450000in}}%
\pgfpathlineto{\pgfqpoint{4.516640in}{2.590000in}}%
\pgfpathlineto{\pgfqpoint{4.517880in}{2.415000in}}%
\pgfpathlineto{\pgfqpoint{4.519120in}{2.625000in}}%
\pgfpathlineto{\pgfqpoint{4.520360in}{2.100000in}}%
\pgfpathlineto{\pgfqpoint{4.524080in}{2.485000in}}%
\pgfpathlineto{\pgfqpoint{4.525320in}{2.345000in}}%
\pgfpathlineto{\pgfqpoint{4.526560in}{2.625000in}}%
\pgfpathlineto{\pgfqpoint{4.529040in}{2.240000in}}%
\pgfpathlineto{\pgfqpoint{4.530280in}{2.625000in}}%
\pgfpathlineto{\pgfqpoint{4.531520in}{2.625000in}}%
\pgfpathlineto{\pgfqpoint{4.534000in}{2.310000in}}%
\pgfpathlineto{\pgfqpoint{4.535240in}{2.345000in}}%
\pgfpathlineto{\pgfqpoint{4.536480in}{2.275000in}}%
\pgfpathlineto{\pgfqpoint{4.537720in}{2.345000in}}%
\pgfpathlineto{\pgfqpoint{4.538960in}{2.205000in}}%
\pgfpathlineto{\pgfqpoint{4.542680in}{2.555000in}}%
\pgfpathlineto{\pgfqpoint{4.543920in}{1.995000in}}%
\pgfpathlineto{\pgfqpoint{4.545160in}{2.520000in}}%
\pgfpathlineto{\pgfqpoint{4.546400in}{2.310000in}}%
\pgfpathlineto{\pgfqpoint{4.547640in}{2.695000in}}%
\pgfpathlineto{\pgfqpoint{4.550120in}{2.380000in}}%
\pgfpathlineto{\pgfqpoint{4.551360in}{2.485000in}}%
\pgfpathlineto{\pgfqpoint{4.552600in}{2.240000in}}%
\pgfpathlineto{\pgfqpoint{4.553840in}{2.765000in}}%
\pgfpathlineto{\pgfqpoint{4.555080in}{2.485000in}}%
\pgfpathlineto{\pgfqpoint{4.556320in}{2.485000in}}%
\pgfpathlineto{\pgfqpoint{4.557560in}{2.275000in}}%
\pgfpathlineto{\pgfqpoint{4.558800in}{2.940000in}}%
\pgfpathlineto{\pgfqpoint{4.561280in}{2.450000in}}%
\pgfpathlineto{\pgfqpoint{4.562520in}{2.520000in}}%
\pgfpathlineto{\pgfqpoint{4.565000in}{2.100000in}}%
\pgfpathlineto{\pgfqpoint{4.566240in}{2.170000in}}%
\pgfpathlineto{\pgfqpoint{4.567480in}{2.450000in}}%
\pgfpathlineto{\pgfqpoint{4.568720in}{2.415000in}}%
\pgfpathlineto{\pgfqpoint{4.569960in}{1.960000in}}%
\pgfpathlineto{\pgfqpoint{4.571200in}{2.275000in}}%
\pgfpathlineto{\pgfqpoint{4.572440in}{2.205000in}}%
\pgfpathlineto{\pgfqpoint{4.573680in}{1.925000in}}%
\pgfpathlineto{\pgfqpoint{4.574920in}{2.240000in}}%
\pgfpathlineto{\pgfqpoint{4.577400in}{1.960000in}}%
\pgfpathlineto{\pgfqpoint{4.579880in}{2.380000in}}%
\pgfpathlineto{\pgfqpoint{4.581120in}{2.170000in}}%
\pgfpathlineto{\pgfqpoint{4.582360in}{2.520000in}}%
\pgfpathlineto{\pgfqpoint{4.584840in}{2.030000in}}%
\pgfpathlineto{\pgfqpoint{4.586080in}{2.065000in}}%
\pgfpathlineto{\pgfqpoint{4.587320in}{2.065000in}}%
\pgfpathlineto{\pgfqpoint{4.588560in}{1.925000in}}%
\pgfpathlineto{\pgfqpoint{4.589800in}{2.310000in}}%
\pgfpathlineto{\pgfqpoint{4.591040in}{2.240000in}}%
\pgfpathlineto{\pgfqpoint{4.592280in}{1.995000in}}%
\pgfpathlineto{\pgfqpoint{4.597240in}{2.485000in}}%
\pgfpathlineto{\pgfqpoint{4.599720in}{2.135000in}}%
\pgfpathlineto{\pgfqpoint{4.600960in}{2.415000in}}%
\pgfpathlineto{\pgfqpoint{4.602200in}{2.205000in}}%
\pgfpathlineto{\pgfqpoint{4.603440in}{2.590000in}}%
\pgfpathlineto{\pgfqpoint{4.605920in}{2.450000in}}%
\pgfpathlineto{\pgfqpoint{4.607160in}{2.485000in}}%
\pgfpathlineto{\pgfqpoint{4.608400in}{2.415000in}}%
\pgfpathlineto{\pgfqpoint{4.609640in}{2.555000in}}%
\pgfpathlineto{\pgfqpoint{4.610880in}{2.170000in}}%
\pgfpathlineto{\pgfqpoint{4.612120in}{2.310000in}}%
\pgfpathlineto{\pgfqpoint{4.613360in}{2.135000in}}%
\pgfpathlineto{\pgfqpoint{4.614600in}{2.380000in}}%
\pgfpathlineto{\pgfqpoint{4.615840in}{2.275000in}}%
\pgfpathlineto{\pgfqpoint{4.617080in}{2.065000in}}%
\pgfpathlineto{\pgfqpoint{4.618320in}{2.415000in}}%
\pgfpathlineto{\pgfqpoint{4.620800in}{1.855000in}}%
\pgfpathlineto{\pgfqpoint{4.622040in}{2.275000in}}%
\pgfpathlineto{\pgfqpoint{4.623280in}{2.100000in}}%
\pgfpathlineto{\pgfqpoint{4.624520in}{2.170000in}}%
\pgfpathlineto{\pgfqpoint{4.625760in}{2.555000in}}%
\pgfpathlineto{\pgfqpoint{4.627000in}{2.485000in}}%
\pgfpathlineto{\pgfqpoint{4.629480in}{2.205000in}}%
\pgfpathlineto{\pgfqpoint{4.630720in}{2.520000in}}%
\pgfpathlineto{\pgfqpoint{4.631960in}{2.415000in}}%
\pgfpathlineto{\pgfqpoint{4.634440in}{2.555000in}}%
\pgfpathlineto{\pgfqpoint{4.635680in}{2.555000in}}%
\pgfpathlineto{\pgfqpoint{4.636920in}{2.205000in}}%
\pgfpathlineto{\pgfqpoint{4.638160in}{2.485000in}}%
\pgfpathlineto{\pgfqpoint{4.639400in}{2.275000in}}%
\pgfpathlineto{\pgfqpoint{4.640640in}{2.345000in}}%
\pgfpathlineto{\pgfqpoint{4.641880in}{2.345000in}}%
\pgfpathlineto{\pgfqpoint{4.644360in}{2.485000in}}%
\pgfpathlineto{\pgfqpoint{4.645600in}{2.625000in}}%
\pgfpathlineto{\pgfqpoint{4.648080in}{2.030000in}}%
\pgfpathlineto{\pgfqpoint{4.650560in}{2.625000in}}%
\pgfpathlineto{\pgfqpoint{4.651800in}{2.520000in}}%
\pgfpathlineto{\pgfqpoint{4.653040in}{2.765000in}}%
\pgfpathlineto{\pgfqpoint{4.655520in}{2.065000in}}%
\pgfpathlineto{\pgfqpoint{4.656760in}{2.380000in}}%
\pgfpathlineto{\pgfqpoint{4.659240in}{2.030000in}}%
\pgfpathlineto{\pgfqpoint{4.660480in}{2.485000in}}%
\pgfpathlineto{\pgfqpoint{4.661720in}{2.205000in}}%
\pgfpathlineto{\pgfqpoint{4.665440in}{2.870000in}}%
\pgfpathlineto{\pgfqpoint{4.666680in}{2.275000in}}%
\pgfpathlineto{\pgfqpoint{4.667920in}{2.590000in}}%
\pgfpathlineto{\pgfqpoint{4.669160in}{2.450000in}}%
\pgfpathlineto{\pgfqpoint{4.670400in}{2.695000in}}%
\pgfpathlineto{\pgfqpoint{4.671640in}{2.240000in}}%
\pgfpathlineto{\pgfqpoint{4.674120in}{2.170000in}}%
\pgfpathlineto{\pgfqpoint{4.676600in}{2.450000in}}%
\pgfpathlineto{\pgfqpoint{4.677840in}{2.065000in}}%
\pgfpathlineto{\pgfqpoint{4.680320in}{2.485000in}}%
\pgfpathlineto{\pgfqpoint{4.681560in}{2.625000in}}%
\pgfpathlineto{\pgfqpoint{4.682800in}{2.625000in}}%
\pgfpathlineto{\pgfqpoint{4.684040in}{2.275000in}}%
\pgfpathlineto{\pgfqpoint{4.685280in}{2.555000in}}%
\pgfpathlineto{\pgfqpoint{4.686520in}{2.450000in}}%
\pgfpathlineto{\pgfqpoint{4.687760in}{1.960000in}}%
\pgfpathlineto{\pgfqpoint{4.689000in}{2.065000in}}%
\pgfpathlineto{\pgfqpoint{4.690240in}{2.450000in}}%
\pgfpathlineto{\pgfqpoint{4.691480in}{2.310000in}}%
\pgfpathlineto{\pgfqpoint{4.695200in}{2.625000in}}%
\pgfpathlineto{\pgfqpoint{4.696440in}{2.275000in}}%
\pgfpathlineto{\pgfqpoint{4.697680in}{2.625000in}}%
\pgfpathlineto{\pgfqpoint{4.698920in}{2.240000in}}%
\pgfpathlineto{\pgfqpoint{4.701400in}{2.730000in}}%
\pgfpathlineto{\pgfqpoint{4.703880in}{2.135000in}}%
\pgfpathlineto{\pgfqpoint{4.705120in}{2.555000in}}%
\pgfpathlineto{\pgfqpoint{4.707600in}{2.380000in}}%
\pgfpathlineto{\pgfqpoint{4.708840in}{2.485000in}}%
\pgfpathlineto{\pgfqpoint{4.710080in}{2.310000in}}%
\pgfpathlineto{\pgfqpoint{4.711320in}{2.310000in}}%
\pgfpathlineto{\pgfqpoint{4.712560in}{2.590000in}}%
\pgfpathlineto{\pgfqpoint{4.713800in}{2.240000in}}%
\pgfpathlineto{\pgfqpoint{4.716280in}{2.590000in}}%
\pgfpathlineto{\pgfqpoint{4.717520in}{2.275000in}}%
\pgfpathlineto{\pgfqpoint{4.720000in}{2.275000in}}%
\pgfpathlineto{\pgfqpoint{4.721240in}{2.695000in}}%
\pgfpathlineto{\pgfqpoint{4.722480in}{2.240000in}}%
\pgfpathlineto{\pgfqpoint{4.723720in}{2.205000in}}%
\pgfpathlineto{\pgfqpoint{4.724960in}{2.485000in}}%
\pgfpathlineto{\pgfqpoint{4.726200in}{2.485000in}}%
\pgfpathlineto{\pgfqpoint{4.727440in}{2.065000in}}%
\pgfpathlineto{\pgfqpoint{4.728680in}{2.730000in}}%
\pgfpathlineto{\pgfqpoint{4.731160in}{2.275000in}}%
\pgfpathlineto{\pgfqpoint{4.732400in}{2.275000in}}%
\pgfpathlineto{\pgfqpoint{4.733640in}{2.415000in}}%
\pgfpathlineto{\pgfqpoint{4.734880in}{2.170000in}}%
\pgfpathlineto{\pgfqpoint{4.736120in}{2.240000in}}%
\pgfpathlineto{\pgfqpoint{4.737360in}{2.100000in}}%
\pgfpathlineto{\pgfqpoint{4.739840in}{2.730000in}}%
\pgfpathlineto{\pgfqpoint{4.741080in}{2.100000in}}%
\pgfpathlineto{\pgfqpoint{4.742320in}{2.730000in}}%
\pgfpathlineto{\pgfqpoint{4.743560in}{2.625000in}}%
\pgfpathlineto{\pgfqpoint{4.746040in}{2.100000in}}%
\pgfpathlineto{\pgfqpoint{4.747280in}{2.415000in}}%
\pgfpathlineto{\pgfqpoint{4.748520in}{2.275000in}}%
\pgfpathlineto{\pgfqpoint{4.751000in}{2.660000in}}%
\pgfpathlineto{\pgfqpoint{4.754720in}{2.030000in}}%
\pgfpathlineto{\pgfqpoint{4.755960in}{2.240000in}}%
\pgfpathlineto{\pgfqpoint{4.757200in}{2.695000in}}%
\pgfpathlineto{\pgfqpoint{4.760920in}{2.100000in}}%
\pgfpathlineto{\pgfqpoint{4.763400in}{2.450000in}}%
\pgfpathlineto{\pgfqpoint{4.764640in}{2.415000in}}%
\pgfpathlineto{\pgfqpoint{4.765880in}{2.520000in}}%
\pgfpathlineto{\pgfqpoint{4.767120in}{2.100000in}}%
\pgfpathlineto{\pgfqpoint{4.769600in}{2.485000in}}%
\pgfpathlineto{\pgfqpoint{4.770840in}{2.135000in}}%
\pgfpathlineto{\pgfqpoint{4.775800in}{2.730000in}}%
\pgfpathlineto{\pgfqpoint{4.777040in}{2.205000in}}%
\pgfpathlineto{\pgfqpoint{4.778280in}{2.625000in}}%
\pgfpathlineto{\pgfqpoint{4.782000in}{2.415000in}}%
\pgfpathlineto{\pgfqpoint{4.783240in}{2.450000in}}%
\pgfpathlineto{\pgfqpoint{4.784480in}{2.730000in}}%
\pgfpathlineto{\pgfqpoint{4.785720in}{2.695000in}}%
\pgfpathlineto{\pgfqpoint{4.788200in}{2.485000in}}%
\pgfpathlineto{\pgfqpoint{4.789440in}{2.520000in}}%
\pgfpathlineto{\pgfqpoint{4.790680in}{2.485000in}}%
\pgfpathlineto{\pgfqpoint{4.791920in}{2.485000in}}%
\pgfpathlineto{\pgfqpoint{4.794400in}{2.205000in}}%
\pgfpathlineto{\pgfqpoint{4.796880in}{2.800000in}}%
\pgfpathlineto{\pgfqpoint{4.798120in}{2.415000in}}%
\pgfpathlineto{\pgfqpoint{4.799360in}{2.555000in}}%
\pgfpathlineto{\pgfqpoint{4.800600in}{2.275000in}}%
\pgfpathlineto{\pgfqpoint{4.803080in}{2.555000in}}%
\pgfpathlineto{\pgfqpoint{4.805560in}{1.995000in}}%
\pgfpathlineto{\pgfqpoint{4.806800in}{2.205000in}}%
\pgfpathlineto{\pgfqpoint{4.808040in}{2.100000in}}%
\pgfpathlineto{\pgfqpoint{4.809280in}{2.695000in}}%
\pgfpathlineto{\pgfqpoint{4.811760in}{2.415000in}}%
\pgfpathlineto{\pgfqpoint{4.814240in}{2.660000in}}%
\pgfpathlineto{\pgfqpoint{4.815480in}{2.275000in}}%
\pgfpathlineto{\pgfqpoint{4.816720in}{2.800000in}}%
\pgfpathlineto{\pgfqpoint{4.819200in}{2.380000in}}%
\pgfpathlineto{\pgfqpoint{4.820440in}{2.765000in}}%
\pgfpathlineto{\pgfqpoint{4.822920in}{2.415000in}}%
\pgfpathlineto{\pgfqpoint{4.824160in}{2.415000in}}%
\pgfpathlineto{\pgfqpoint{4.825400in}{1.925000in}}%
\pgfpathlineto{\pgfqpoint{4.827880in}{2.555000in}}%
\pgfpathlineto{\pgfqpoint{4.829120in}{2.345000in}}%
\pgfpathlineto{\pgfqpoint{4.830360in}{2.380000in}}%
\pgfpathlineto{\pgfqpoint{4.831600in}{2.450000in}}%
\pgfpathlineto{\pgfqpoint{4.832840in}{2.450000in}}%
\pgfpathlineto{\pgfqpoint{4.834080in}{2.625000in}}%
\pgfpathlineto{\pgfqpoint{4.836560in}{2.170000in}}%
\pgfpathlineto{\pgfqpoint{4.837800in}{2.380000in}}%
\pgfpathlineto{\pgfqpoint{4.839040in}{2.345000in}}%
\pgfpathlineto{\pgfqpoint{4.840280in}{2.380000in}}%
\pgfpathlineto{\pgfqpoint{4.841520in}{2.450000in}}%
\pgfpathlineto{\pgfqpoint{4.844000in}{2.905000in}}%
\pgfpathlineto{\pgfqpoint{4.846480in}{2.450000in}}%
\pgfpathlineto{\pgfqpoint{4.847720in}{2.415000in}}%
\pgfpathlineto{\pgfqpoint{4.848960in}{2.450000in}}%
\pgfpathlineto{\pgfqpoint{4.850200in}{2.765000in}}%
\pgfpathlineto{\pgfqpoint{4.852680in}{2.520000in}}%
\pgfpathlineto{\pgfqpoint{4.853920in}{2.800000in}}%
\pgfpathlineto{\pgfqpoint{4.855160in}{2.800000in}}%
\pgfpathlineto{\pgfqpoint{4.857640in}{2.590000in}}%
\pgfpathlineto{\pgfqpoint{4.858880in}{2.695000in}}%
\pgfpathlineto{\pgfqpoint{4.860120in}{2.660000in}}%
\pgfpathlineto{\pgfqpoint{4.861360in}{2.310000in}}%
\pgfpathlineto{\pgfqpoint{4.862600in}{2.870000in}}%
\pgfpathlineto{\pgfqpoint{4.865080in}{2.415000in}}%
\pgfpathlineto{\pgfqpoint{4.866320in}{2.555000in}}%
\pgfpathlineto{\pgfqpoint{4.867560in}{2.870000in}}%
\pgfpathlineto{\pgfqpoint{4.868800in}{2.170000in}}%
\pgfpathlineto{\pgfqpoint{4.870040in}{2.100000in}}%
\pgfpathlineto{\pgfqpoint{4.872520in}{2.765000in}}%
\pgfpathlineto{\pgfqpoint{4.873760in}{2.310000in}}%
\pgfpathlineto{\pgfqpoint{4.875000in}{2.555000in}}%
\pgfpathlineto{\pgfqpoint{4.876240in}{2.485000in}}%
\pgfpathlineto{\pgfqpoint{4.878720in}{2.590000in}}%
\pgfpathlineto{\pgfqpoint{4.879960in}{2.590000in}}%
\pgfpathlineto{\pgfqpoint{4.881200in}{2.625000in}}%
\pgfpathlineto{\pgfqpoint{4.882440in}{2.730000in}}%
\pgfpathlineto{\pgfqpoint{4.883680in}{2.310000in}}%
\pgfpathlineto{\pgfqpoint{4.884920in}{2.415000in}}%
\pgfpathlineto{\pgfqpoint{4.886160in}{2.100000in}}%
\pgfpathlineto{\pgfqpoint{4.887400in}{2.380000in}}%
\pgfpathlineto{\pgfqpoint{4.888640in}{2.345000in}}%
\pgfpathlineto{\pgfqpoint{4.891120in}{2.205000in}}%
\pgfpathlineto{\pgfqpoint{4.892360in}{2.030000in}}%
\pgfpathlineto{\pgfqpoint{4.894840in}{2.380000in}}%
\pgfpathlineto{\pgfqpoint{4.897320in}{2.695000in}}%
\pgfpathlineto{\pgfqpoint{4.899800in}{2.345000in}}%
\pgfpathlineto{\pgfqpoint{4.901040in}{2.555000in}}%
\pgfpathlineto{\pgfqpoint{4.902280in}{2.275000in}}%
\pgfpathlineto{\pgfqpoint{4.903520in}{2.555000in}}%
\pgfpathlineto{\pgfqpoint{4.904760in}{2.520000in}}%
\pgfpathlineto{\pgfqpoint{4.907240in}{2.065000in}}%
\pgfpathlineto{\pgfqpoint{4.910960in}{2.415000in}}%
\pgfpathlineto{\pgfqpoint{4.912200in}{2.345000in}}%
\pgfpathlineto{\pgfqpoint{4.913440in}{2.345000in}}%
\pgfpathlineto{\pgfqpoint{4.914680in}{2.590000in}}%
\pgfpathlineto{\pgfqpoint{4.915920in}{2.100000in}}%
\pgfpathlineto{\pgfqpoint{4.917160in}{2.170000in}}%
\pgfpathlineto{\pgfqpoint{4.918400in}{2.100000in}}%
\pgfpathlineto{\pgfqpoint{4.919640in}{2.135000in}}%
\pgfpathlineto{\pgfqpoint{4.920880in}{2.240000in}}%
\pgfpathlineto{\pgfqpoint{4.922120in}{2.065000in}}%
\pgfpathlineto{\pgfqpoint{4.925840in}{2.555000in}}%
\pgfpathlineto{\pgfqpoint{4.927080in}{2.170000in}}%
\pgfpathlineto{\pgfqpoint{4.928320in}{2.240000in}}%
\pgfpathlineto{\pgfqpoint{4.929560in}{2.240000in}}%
\pgfpathlineto{\pgfqpoint{4.930800in}{2.625000in}}%
\pgfpathlineto{\pgfqpoint{4.932040in}{2.450000in}}%
\pgfpathlineto{\pgfqpoint{4.933280in}{2.450000in}}%
\pgfpathlineto{\pgfqpoint{4.934520in}{2.520000in}}%
\pgfpathlineto{\pgfqpoint{4.935760in}{2.415000in}}%
\pgfpathlineto{\pgfqpoint{4.937000in}{2.975000in}}%
\pgfpathlineto{\pgfqpoint{4.938240in}{2.625000in}}%
\pgfpathlineto{\pgfqpoint{4.939480in}{2.835000in}}%
\pgfpathlineto{\pgfqpoint{4.941960in}{2.555000in}}%
\pgfpathlineto{\pgfqpoint{4.943200in}{2.590000in}}%
\pgfpathlineto{\pgfqpoint{4.944440in}{2.660000in}}%
\pgfpathlineto{\pgfqpoint{4.946920in}{2.275000in}}%
\pgfpathlineto{\pgfqpoint{4.948160in}{2.520000in}}%
\pgfpathlineto{\pgfqpoint{4.949400in}{2.485000in}}%
\pgfpathlineto{\pgfqpoint{4.950640in}{2.100000in}}%
\pgfpathlineto{\pgfqpoint{4.951880in}{2.590000in}}%
\pgfpathlineto{\pgfqpoint{4.953120in}{2.240000in}}%
\pgfpathlineto{\pgfqpoint{4.954360in}{2.730000in}}%
\pgfpathlineto{\pgfqpoint{4.955600in}{2.380000in}}%
\pgfpathlineto{\pgfqpoint{4.956840in}{2.905000in}}%
\pgfpathlineto{\pgfqpoint{4.958080in}{2.660000in}}%
\pgfpathlineto{\pgfqpoint{4.959320in}{2.765000in}}%
\pgfpathlineto{\pgfqpoint{4.960560in}{2.275000in}}%
\pgfpathlineto{\pgfqpoint{4.961800in}{2.310000in}}%
\pgfpathlineto{\pgfqpoint{4.963040in}{2.170000in}}%
\pgfpathlineto{\pgfqpoint{4.965520in}{2.590000in}}%
\pgfpathlineto{\pgfqpoint{4.968000in}{2.205000in}}%
\pgfpathlineto{\pgfqpoint{4.969240in}{2.205000in}}%
\pgfpathlineto{\pgfqpoint{4.971720in}{2.450000in}}%
\pgfpathlineto{\pgfqpoint{4.972960in}{2.415000in}}%
\pgfpathlineto{\pgfqpoint{4.975440in}{2.555000in}}%
\pgfpathlineto{\pgfqpoint{4.976680in}{2.590000in}}%
\pgfpathlineto{\pgfqpoint{4.980400in}{1.995000in}}%
\pgfpathlineto{\pgfqpoint{4.982880in}{2.555000in}}%
\pgfpathlineto{\pgfqpoint{4.984120in}{2.100000in}}%
\pgfpathlineto{\pgfqpoint{4.989080in}{2.625000in}}%
\pgfpathlineto{\pgfqpoint{4.991560in}{2.135000in}}%
\pgfpathlineto{\pgfqpoint{4.994040in}{2.555000in}}%
\pgfpathlineto{\pgfqpoint{4.995280in}{2.415000in}}%
\pgfpathlineto{\pgfqpoint{4.996520in}{2.485000in}}%
\pgfpathlineto{\pgfqpoint{4.997760in}{2.450000in}}%
\pgfpathlineto{\pgfqpoint{4.999000in}{2.730000in}}%
\pgfpathlineto{\pgfqpoint{5.000240in}{2.625000in}}%
\pgfpathlineto{\pgfqpoint{5.002720in}{2.695000in}}%
\pgfpathlineto{\pgfqpoint{5.003960in}{2.380000in}}%
\pgfpathlineto{\pgfqpoint{5.005200in}{2.485000in}}%
\pgfpathlineto{\pgfqpoint{5.006440in}{2.345000in}}%
\pgfpathlineto{\pgfqpoint{5.007680in}{2.485000in}}%
\pgfpathlineto{\pgfqpoint{5.008920in}{2.800000in}}%
\pgfpathlineto{\pgfqpoint{5.010160in}{2.485000in}}%
\pgfpathlineto{\pgfqpoint{5.011400in}{2.485000in}}%
\pgfpathlineto{\pgfqpoint{5.012640in}{2.450000in}}%
\pgfpathlineto{\pgfqpoint{5.013880in}{2.310000in}}%
\pgfpathlineto{\pgfqpoint{5.016360in}{2.555000in}}%
\pgfpathlineto{\pgfqpoint{5.017600in}{2.310000in}}%
\pgfpathlineto{\pgfqpoint{5.018840in}{2.485000in}}%
\pgfpathlineto{\pgfqpoint{5.020080in}{2.940000in}}%
\pgfpathlineto{\pgfqpoint{5.021320in}{2.940000in}}%
\pgfpathlineto{\pgfqpoint{5.022560in}{3.010000in}}%
\pgfpathlineto{\pgfqpoint{5.023800in}{2.555000in}}%
\pgfpathlineto{\pgfqpoint{5.025040in}{2.765000in}}%
\pgfpathlineto{\pgfqpoint{5.026280in}{2.345000in}}%
\pgfpathlineto{\pgfqpoint{5.027520in}{2.450000in}}%
\pgfpathlineto{\pgfqpoint{5.028760in}{2.415000in}}%
\pgfpathlineto{\pgfqpoint{5.030000in}{2.485000in}}%
\pgfpathlineto{\pgfqpoint{5.031240in}{2.765000in}}%
\pgfpathlineto{\pgfqpoint{5.032480in}{2.275000in}}%
\pgfpathlineto{\pgfqpoint{5.033720in}{2.800000in}}%
\pgfpathlineto{\pgfqpoint{5.034960in}{2.730000in}}%
\pgfpathlineto{\pgfqpoint{5.036200in}{2.415000in}}%
\pgfpathlineto{\pgfqpoint{5.038680in}{2.520000in}}%
\pgfpathlineto{\pgfqpoint{5.039920in}{2.380000in}}%
\pgfpathlineto{\pgfqpoint{5.041160in}{2.625000in}}%
\pgfpathlineto{\pgfqpoint{5.042400in}{2.555000in}}%
\pgfpathlineto{\pgfqpoint{5.043640in}{2.310000in}}%
\pgfpathlineto{\pgfqpoint{5.044880in}{2.485000in}}%
\pgfpathlineto{\pgfqpoint{5.047360in}{2.485000in}}%
\pgfpathlineto{\pgfqpoint{5.048600in}{2.275000in}}%
\pgfpathlineto{\pgfqpoint{5.049840in}{2.345000in}}%
\pgfpathlineto{\pgfqpoint{5.051080in}{2.730000in}}%
\pgfpathlineto{\pgfqpoint{5.054800in}{2.135000in}}%
\pgfpathlineto{\pgfqpoint{5.056040in}{2.310000in}}%
\pgfpathlineto{\pgfqpoint{5.058520in}{1.715000in}}%
\pgfpathlineto{\pgfqpoint{5.061000in}{2.555000in}}%
\pgfpathlineto{\pgfqpoint{5.062240in}{2.380000in}}%
\pgfpathlineto{\pgfqpoint{5.064720in}{1.855000in}}%
\pgfpathlineto{\pgfqpoint{5.065960in}{2.380000in}}%
\pgfpathlineto{\pgfqpoint{5.067200in}{1.680000in}}%
\pgfpathlineto{\pgfqpoint{5.068440in}{2.485000in}}%
\pgfpathlineto{\pgfqpoint{5.069680in}{2.170000in}}%
\pgfpathlineto{\pgfqpoint{5.070920in}{2.555000in}}%
\pgfpathlineto{\pgfqpoint{5.074640in}{2.135000in}}%
\pgfpathlineto{\pgfqpoint{5.075880in}{2.205000in}}%
\pgfpathlineto{\pgfqpoint{5.077120in}{2.520000in}}%
\pgfpathlineto{\pgfqpoint{5.078360in}{2.275000in}}%
\pgfpathlineto{\pgfqpoint{5.080840in}{2.485000in}}%
\pgfpathlineto{\pgfqpoint{5.083320in}{2.135000in}}%
\pgfpathlineto{\pgfqpoint{5.084560in}{2.450000in}}%
\pgfpathlineto{\pgfqpoint{5.085800in}{2.135000in}}%
\pgfpathlineto{\pgfqpoint{5.087040in}{2.660000in}}%
\pgfpathlineto{\pgfqpoint{5.088280in}{2.170000in}}%
\pgfpathlineto{\pgfqpoint{5.089520in}{2.240000in}}%
\pgfpathlineto{\pgfqpoint{5.092000in}{2.170000in}}%
\pgfpathlineto{\pgfqpoint{5.093240in}{2.345000in}}%
\pgfpathlineto{\pgfqpoint{5.094480in}{2.240000in}}%
\pgfpathlineto{\pgfqpoint{5.096960in}{2.590000in}}%
\pgfpathlineto{\pgfqpoint{5.098200in}{2.100000in}}%
\pgfpathlineto{\pgfqpoint{5.099440in}{2.135000in}}%
\pgfpathlineto{\pgfqpoint{5.100680in}{2.415000in}}%
\pgfpathlineto{\pgfqpoint{5.101920in}{2.345000in}}%
\pgfpathlineto{\pgfqpoint{5.103160in}{2.485000in}}%
\pgfpathlineto{\pgfqpoint{5.105640in}{2.380000in}}%
\pgfpathlineto{\pgfqpoint{5.108120in}{2.310000in}}%
\pgfpathlineto{\pgfqpoint{5.110600in}{2.730000in}}%
\pgfpathlineto{\pgfqpoint{5.111840in}{2.310000in}}%
\pgfpathlineto{\pgfqpoint{5.113080in}{2.345000in}}%
\pgfpathlineto{\pgfqpoint{5.114320in}{2.275000in}}%
\pgfpathlineto{\pgfqpoint{5.115560in}{1.995000in}}%
\pgfpathlineto{\pgfqpoint{5.116800in}{2.625000in}}%
\pgfpathlineto{\pgfqpoint{5.118040in}{2.310000in}}%
\pgfpathlineto{\pgfqpoint{5.119280in}{2.660000in}}%
\pgfpathlineto{\pgfqpoint{5.120520in}{2.240000in}}%
\pgfpathlineto{\pgfqpoint{5.121760in}{2.450000in}}%
\pgfpathlineto{\pgfqpoint{5.123000in}{2.205000in}}%
\pgfpathlineto{\pgfqpoint{5.124240in}{2.695000in}}%
\pgfpathlineto{\pgfqpoint{5.125480in}{2.520000in}}%
\pgfpathlineto{\pgfqpoint{5.126720in}{2.590000in}}%
\pgfpathlineto{\pgfqpoint{5.127960in}{2.765000in}}%
\pgfpathlineto{\pgfqpoint{5.129200in}{2.555000in}}%
\pgfpathlineto{\pgfqpoint{5.130440in}{2.625000in}}%
\pgfpathlineto{\pgfqpoint{5.132920in}{2.135000in}}%
\pgfpathlineto{\pgfqpoint{5.134160in}{2.450000in}}%
\pgfpathlineto{\pgfqpoint{5.135400in}{2.135000in}}%
\pgfpathlineto{\pgfqpoint{5.136640in}{2.135000in}}%
\pgfpathlineto{\pgfqpoint{5.137880in}{2.065000in}}%
\pgfpathlineto{\pgfqpoint{5.139120in}{2.555000in}}%
\pgfpathlineto{\pgfqpoint{5.140360in}{2.380000in}}%
\pgfpathlineto{\pgfqpoint{5.141600in}{2.730000in}}%
\pgfpathlineto{\pgfqpoint{5.144080in}{2.415000in}}%
\pgfpathlineto{\pgfqpoint{5.145320in}{2.415000in}}%
\pgfpathlineto{\pgfqpoint{5.147800in}{2.730000in}}%
\pgfpathlineto{\pgfqpoint{5.149040in}{2.555000in}}%
\pgfpathlineto{\pgfqpoint{5.150280in}{2.170000in}}%
\pgfpathlineto{\pgfqpoint{5.151520in}{2.485000in}}%
\pgfpathlineto{\pgfqpoint{5.152760in}{2.485000in}}%
\pgfpathlineto{\pgfqpoint{5.154000in}{2.590000in}}%
\pgfpathlineto{\pgfqpoint{5.155240in}{2.590000in}}%
\pgfpathlineto{\pgfqpoint{5.156480in}{2.520000in}}%
\pgfpathlineto{\pgfqpoint{5.157720in}{2.205000in}}%
\pgfpathlineto{\pgfqpoint{5.158960in}{2.450000in}}%
\pgfpathlineto{\pgfqpoint{5.160200in}{2.415000in}}%
\pgfpathlineto{\pgfqpoint{5.161440in}{2.730000in}}%
\pgfpathlineto{\pgfqpoint{5.162680in}{2.415000in}}%
\pgfpathlineto{\pgfqpoint{5.163920in}{2.450000in}}%
\pgfpathlineto{\pgfqpoint{5.165160in}{2.345000in}}%
\pgfpathlineto{\pgfqpoint{5.166400in}{2.345000in}}%
\pgfpathlineto{\pgfqpoint{5.167640in}{2.380000in}}%
\pgfpathlineto{\pgfqpoint{5.168880in}{2.520000in}}%
\pgfpathlineto{\pgfqpoint{5.170120in}{2.380000in}}%
\pgfpathlineto{\pgfqpoint{5.171360in}{2.030000in}}%
\pgfpathlineto{\pgfqpoint{5.172600in}{2.485000in}}%
\pgfpathlineto{\pgfqpoint{5.173840in}{2.065000in}}%
\pgfpathlineto{\pgfqpoint{5.175080in}{2.275000in}}%
\pgfpathlineto{\pgfqpoint{5.176320in}{2.695000in}}%
\pgfpathlineto{\pgfqpoint{5.178800in}{2.345000in}}%
\pgfpathlineto{\pgfqpoint{5.180040in}{2.100000in}}%
\pgfpathlineto{\pgfqpoint{5.181280in}{2.590000in}}%
\pgfpathlineto{\pgfqpoint{5.182520in}{2.205000in}}%
\pgfpathlineto{\pgfqpoint{5.183760in}{2.380000in}}%
\pgfpathlineto{\pgfqpoint{5.185000in}{2.345000in}}%
\pgfpathlineto{\pgfqpoint{5.186240in}{2.275000in}}%
\pgfpathlineto{\pgfqpoint{5.187480in}{2.905000in}}%
\pgfpathlineto{\pgfqpoint{5.188720in}{2.380000in}}%
\pgfpathlineto{\pgfqpoint{5.189960in}{2.590000in}}%
\pgfpathlineto{\pgfqpoint{5.191200in}{2.485000in}}%
\pgfpathlineto{\pgfqpoint{5.192440in}{2.660000in}}%
\pgfpathlineto{\pgfqpoint{5.193680in}{2.450000in}}%
\pgfpathlineto{\pgfqpoint{5.194920in}{2.450000in}}%
\pgfpathlineto{\pgfqpoint{5.196160in}{2.310000in}}%
\pgfpathlineto{\pgfqpoint{5.197400in}{2.485000in}}%
\pgfpathlineto{\pgfqpoint{5.198640in}{2.100000in}}%
\pgfpathlineto{\pgfqpoint{5.199880in}{2.380000in}}%
\pgfpathlineto{\pgfqpoint{5.201120in}{1.995000in}}%
\pgfpathlineto{\pgfqpoint{5.202360in}{2.030000in}}%
\pgfpathlineto{\pgfqpoint{5.203600in}{2.310000in}}%
\pgfpathlineto{\pgfqpoint{5.206080in}{2.100000in}}%
\pgfpathlineto{\pgfqpoint{5.208560in}{2.450000in}}%
\pgfpathlineto{\pgfqpoint{5.209800in}{1.995000in}}%
\pgfpathlineto{\pgfqpoint{5.212280in}{2.415000in}}%
\pgfpathlineto{\pgfqpoint{5.213520in}{2.170000in}}%
\pgfpathlineto{\pgfqpoint{5.216000in}{2.380000in}}%
\pgfpathlineto{\pgfqpoint{5.217240in}{2.590000in}}%
\pgfpathlineto{\pgfqpoint{5.218480in}{2.380000in}}%
\pgfpathlineto{\pgfqpoint{5.219720in}{2.555000in}}%
\pgfpathlineto{\pgfqpoint{5.220960in}{2.100000in}}%
\pgfpathlineto{\pgfqpoint{5.223440in}{2.520000in}}%
\pgfpathlineto{\pgfqpoint{5.224680in}{2.485000in}}%
\pgfpathlineto{\pgfqpoint{5.225920in}{2.380000in}}%
\pgfpathlineto{\pgfqpoint{5.227160in}{2.380000in}}%
\pgfpathlineto{\pgfqpoint{5.228400in}{2.485000in}}%
\pgfpathlineto{\pgfqpoint{5.230880in}{2.030000in}}%
\pgfpathlineto{\pgfqpoint{5.233360in}{2.450000in}}%
\pgfpathlineto{\pgfqpoint{5.234600in}{2.345000in}}%
\pgfpathlineto{\pgfqpoint{5.235840in}{2.520000in}}%
\pgfpathlineto{\pgfqpoint{5.237080in}{2.485000in}}%
\pgfpathlineto{\pgfqpoint{5.238320in}{2.590000in}}%
\pgfpathlineto{\pgfqpoint{5.239560in}{2.240000in}}%
\pgfpathlineto{\pgfqpoint{5.242040in}{2.730000in}}%
\pgfpathlineto{\pgfqpoint{5.243280in}{2.240000in}}%
\pgfpathlineto{\pgfqpoint{5.245760in}{2.660000in}}%
\pgfpathlineto{\pgfqpoint{5.248240in}{2.555000in}}%
\pgfpathlineto{\pgfqpoint{5.249480in}{2.205000in}}%
\pgfpathlineto{\pgfqpoint{5.250720in}{2.275000in}}%
\pgfpathlineto{\pgfqpoint{5.251960in}{2.240000in}}%
\pgfpathlineto{\pgfqpoint{5.253200in}{1.995000in}}%
\pgfpathlineto{\pgfqpoint{5.254440in}{2.275000in}}%
\pgfpathlineto{\pgfqpoint{5.255680in}{2.205000in}}%
\pgfpathlineto{\pgfqpoint{5.258160in}{2.520000in}}%
\pgfpathlineto{\pgfqpoint{5.260640in}{1.925000in}}%
\pgfpathlineto{\pgfqpoint{5.263120in}{2.345000in}}%
\pgfpathlineto{\pgfqpoint{5.264360in}{2.345000in}}%
\pgfpathlineto{\pgfqpoint{5.265600in}{2.310000in}}%
\pgfpathlineto{\pgfqpoint{5.268080in}{1.750000in}}%
\pgfpathlineto{\pgfqpoint{5.269320in}{2.065000in}}%
\pgfpathlineto{\pgfqpoint{5.270560in}{1.995000in}}%
\pgfpathlineto{\pgfqpoint{5.273040in}{2.520000in}}%
\pgfpathlineto{\pgfqpoint{5.275520in}{2.100000in}}%
\pgfpathlineto{\pgfqpoint{5.276760in}{2.205000in}}%
\pgfpathlineto{\pgfqpoint{5.279240in}{2.450000in}}%
\pgfpathlineto{\pgfqpoint{5.280480in}{2.310000in}}%
\pgfpathlineto{\pgfqpoint{5.281720in}{2.310000in}}%
\pgfpathlineto{\pgfqpoint{5.284200in}{2.135000in}}%
\pgfpathlineto{\pgfqpoint{5.285440in}{2.205000in}}%
\pgfpathlineto{\pgfqpoint{5.286680in}{2.380000in}}%
\pgfpathlineto{\pgfqpoint{5.287920in}{2.240000in}}%
\pgfpathlineto{\pgfqpoint{5.289160in}{2.310000in}}%
\pgfpathlineto{\pgfqpoint{5.290400in}{2.275000in}}%
\pgfpathlineto{\pgfqpoint{5.291640in}{2.275000in}}%
\pgfpathlineto{\pgfqpoint{5.292880in}{2.135000in}}%
\pgfpathlineto{\pgfqpoint{5.295360in}{2.590000in}}%
\pgfpathlineto{\pgfqpoint{5.296600in}{2.380000in}}%
\pgfpathlineto{\pgfqpoint{5.297840in}{2.485000in}}%
\pgfpathlineto{\pgfqpoint{5.299080in}{2.345000in}}%
\pgfpathlineto{\pgfqpoint{5.300320in}{2.345000in}}%
\pgfpathlineto{\pgfqpoint{5.301560in}{2.275000in}}%
\pgfpathlineto{\pgfqpoint{5.302800in}{2.345000in}}%
\pgfpathlineto{\pgfqpoint{5.304040in}{2.345000in}}%
\pgfpathlineto{\pgfqpoint{5.305280in}{2.625000in}}%
\pgfpathlineto{\pgfqpoint{5.307760in}{2.240000in}}%
\pgfpathlineto{\pgfqpoint{5.310240in}{2.625000in}}%
\pgfpathlineto{\pgfqpoint{5.311480in}{2.415000in}}%
\pgfpathlineto{\pgfqpoint{5.312720in}{2.450000in}}%
\pgfpathlineto{\pgfqpoint{5.315200in}{2.205000in}}%
\pgfpathlineto{\pgfqpoint{5.316440in}{2.380000in}}%
\pgfpathlineto{\pgfqpoint{5.317680in}{2.205000in}}%
\pgfpathlineto{\pgfqpoint{5.318920in}{2.485000in}}%
\pgfpathlineto{\pgfqpoint{5.322640in}{2.100000in}}%
\pgfpathlineto{\pgfqpoint{5.323880in}{2.450000in}}%
\pgfpathlineto{\pgfqpoint{5.325120in}{2.170000in}}%
\pgfpathlineto{\pgfqpoint{5.326360in}{2.345000in}}%
\pgfpathlineto{\pgfqpoint{5.328840in}{2.135000in}}%
\pgfpathlineto{\pgfqpoint{5.330080in}{2.380000in}}%
\pgfpathlineto{\pgfqpoint{5.331320in}{2.345000in}}%
\pgfpathlineto{\pgfqpoint{5.332560in}{2.450000in}}%
\pgfpathlineto{\pgfqpoint{5.333800in}{2.275000in}}%
\pgfpathlineto{\pgfqpoint{5.336280in}{2.415000in}}%
\pgfpathlineto{\pgfqpoint{5.337520in}{2.135000in}}%
\pgfpathlineto{\pgfqpoint{5.338760in}{2.485000in}}%
\pgfpathlineto{\pgfqpoint{5.340000in}{2.520000in}}%
\pgfpathlineto{\pgfqpoint{5.342480in}{2.275000in}}%
\pgfpathlineto{\pgfqpoint{5.343720in}{2.485000in}}%
\pgfpathlineto{\pgfqpoint{5.344960in}{2.100000in}}%
\pgfpathlineto{\pgfqpoint{5.346200in}{2.240000in}}%
\pgfpathlineto{\pgfqpoint{5.347440in}{2.555000in}}%
\pgfpathlineto{\pgfqpoint{5.348680in}{2.520000in}}%
\pgfpathlineto{\pgfqpoint{5.349920in}{2.380000in}}%
\pgfpathlineto{\pgfqpoint{5.351160in}{2.625000in}}%
\pgfpathlineto{\pgfqpoint{5.352400in}{2.415000in}}%
\pgfpathlineto{\pgfqpoint{5.353640in}{1.820000in}}%
\pgfpathlineto{\pgfqpoint{5.356120in}{2.520000in}}%
\pgfpathlineto{\pgfqpoint{5.357360in}{2.730000in}}%
\pgfpathlineto{\pgfqpoint{5.358600in}{2.660000in}}%
\pgfpathlineto{\pgfqpoint{5.362320in}{2.030000in}}%
\pgfpathlineto{\pgfqpoint{5.363560in}{2.520000in}}%
\pgfpathlineto{\pgfqpoint{5.366040in}{2.205000in}}%
\pgfpathlineto{\pgfqpoint{5.367280in}{2.240000in}}%
\pgfpathlineto{\pgfqpoint{5.368520in}{2.135000in}}%
\pgfpathlineto{\pgfqpoint{5.371000in}{2.520000in}}%
\pgfpathlineto{\pgfqpoint{5.372240in}{1.750000in}}%
\pgfpathlineto{\pgfqpoint{5.373480in}{1.890000in}}%
\pgfpathlineto{\pgfqpoint{5.374720in}{2.590000in}}%
\pgfpathlineto{\pgfqpoint{5.375960in}{2.555000in}}%
\pgfpathlineto{\pgfqpoint{5.377200in}{2.380000in}}%
\pgfpathlineto{\pgfqpoint{5.379680in}{2.590000in}}%
\pgfpathlineto{\pgfqpoint{5.382160in}{2.345000in}}%
\pgfpathlineto{\pgfqpoint{5.383400in}{2.415000in}}%
\pgfpathlineto{\pgfqpoint{5.384640in}{2.555000in}}%
\pgfpathlineto{\pgfqpoint{5.385880in}{2.555000in}}%
\pgfpathlineto{\pgfqpoint{5.387120in}{2.030000in}}%
\pgfpathlineto{\pgfqpoint{5.389600in}{2.730000in}}%
\pgfpathlineto{\pgfqpoint{5.390840in}{2.660000in}}%
\pgfpathlineto{\pgfqpoint{5.392080in}{2.660000in}}%
\pgfpathlineto{\pgfqpoint{5.393320in}{2.415000in}}%
\pgfpathlineto{\pgfqpoint{5.394560in}{2.415000in}}%
\pgfpathlineto{\pgfqpoint{5.395800in}{2.135000in}}%
\pgfpathlineto{\pgfqpoint{5.397040in}{2.345000in}}%
\pgfpathlineto{\pgfqpoint{5.398280in}{2.870000in}}%
\pgfpathlineto{\pgfqpoint{5.400760in}{1.890000in}}%
\pgfpathlineto{\pgfqpoint{5.402000in}{2.100000in}}%
\pgfpathlineto{\pgfqpoint{5.403240in}{2.065000in}}%
\pgfpathlineto{\pgfqpoint{5.404480in}{1.995000in}}%
\pgfpathlineto{\pgfqpoint{5.405720in}{2.205000in}}%
\pgfpathlineto{\pgfqpoint{5.406960in}{2.625000in}}%
\pgfpathlineto{\pgfqpoint{5.408200in}{2.310000in}}%
\pgfpathlineto{\pgfqpoint{5.409440in}{2.450000in}}%
\pgfpathlineto{\pgfqpoint{5.410680in}{2.065000in}}%
\pgfpathlineto{\pgfqpoint{5.411920in}{2.590000in}}%
\pgfpathlineto{\pgfqpoint{5.413160in}{2.275000in}}%
\pgfpathlineto{\pgfqpoint{5.414400in}{2.485000in}}%
\pgfpathlineto{\pgfqpoint{5.415640in}{2.345000in}}%
\pgfpathlineto{\pgfqpoint{5.416880in}{2.555000in}}%
\pgfpathlineto{\pgfqpoint{5.418120in}{2.520000in}}%
\pgfpathlineto{\pgfqpoint{5.420600in}{1.995000in}}%
\pgfpathlineto{\pgfqpoint{5.421840in}{2.380000in}}%
\pgfpathlineto{\pgfqpoint{5.423080in}{2.170000in}}%
\pgfpathlineto{\pgfqpoint{5.424320in}{2.275000in}}%
\pgfpathlineto{\pgfqpoint{5.426800in}{2.030000in}}%
\pgfpathlineto{\pgfqpoint{5.429280in}{2.310000in}}%
\pgfpathlineto{\pgfqpoint{5.430520in}{1.890000in}}%
\pgfpathlineto{\pgfqpoint{5.431760in}{2.275000in}}%
\pgfpathlineto{\pgfqpoint{5.433000in}{2.135000in}}%
\pgfpathlineto{\pgfqpoint{5.435480in}{2.765000in}}%
\pgfpathlineto{\pgfqpoint{5.436720in}{2.310000in}}%
\pgfpathlineto{\pgfqpoint{5.437960in}{2.555000in}}%
\pgfpathlineto{\pgfqpoint{5.439200in}{2.240000in}}%
\pgfpathlineto{\pgfqpoint{5.440440in}{2.730000in}}%
\pgfpathlineto{\pgfqpoint{5.441680in}{2.660000in}}%
\pgfpathlineto{\pgfqpoint{5.442920in}{2.485000in}}%
\pgfpathlineto{\pgfqpoint{5.444160in}{2.765000in}}%
\pgfpathlineto{\pgfqpoint{5.446640in}{2.310000in}}%
\pgfpathlineto{\pgfqpoint{5.447880in}{2.275000in}}%
\pgfpathlineto{\pgfqpoint{5.450360in}{2.520000in}}%
\pgfpathlineto{\pgfqpoint{5.451600in}{2.275000in}}%
\pgfpathlineto{\pgfqpoint{5.452840in}{2.275000in}}%
\pgfpathlineto{\pgfqpoint{5.455320in}{2.485000in}}%
\pgfpathlineto{\pgfqpoint{5.456560in}{2.135000in}}%
\pgfpathlineto{\pgfqpoint{5.459040in}{2.765000in}}%
\pgfpathlineto{\pgfqpoint{5.460280in}{2.520000in}}%
\pgfpathlineto{\pgfqpoint{5.461520in}{2.555000in}}%
\pgfpathlineto{\pgfqpoint{5.462760in}{2.310000in}}%
\pgfpathlineto{\pgfqpoint{5.464000in}{2.345000in}}%
\pgfpathlineto{\pgfqpoint{5.465240in}{2.310000in}}%
\pgfpathlineto{\pgfqpoint{5.466480in}{2.415000in}}%
\pgfpathlineto{\pgfqpoint{5.467720in}{2.800000in}}%
\pgfpathlineto{\pgfqpoint{5.470200in}{2.345000in}}%
\pgfpathlineto{\pgfqpoint{5.471440in}{2.625000in}}%
\pgfpathlineto{\pgfqpoint{5.473920in}{2.240000in}}%
\pgfpathlineto{\pgfqpoint{5.475160in}{2.590000in}}%
\pgfpathlineto{\pgfqpoint{5.476400in}{2.485000in}}%
\pgfpathlineto{\pgfqpoint{5.477640in}{2.660000in}}%
\pgfpathlineto{\pgfqpoint{5.478880in}{2.380000in}}%
\pgfpathlineto{\pgfqpoint{5.480120in}{2.730000in}}%
\pgfpathlineto{\pgfqpoint{5.481360in}{2.660000in}}%
\pgfpathlineto{\pgfqpoint{5.483840in}{2.100000in}}%
\pgfpathlineto{\pgfqpoint{5.485080in}{2.520000in}}%
\pgfpathlineto{\pgfqpoint{5.486320in}{2.520000in}}%
\pgfpathlineto{\pgfqpoint{5.490040in}{2.135000in}}%
\pgfpathlineto{\pgfqpoint{5.491280in}{2.240000in}}%
\pgfpathlineto{\pgfqpoint{5.493760in}{2.065000in}}%
\pgfpathlineto{\pgfqpoint{5.495000in}{2.205000in}}%
\pgfpathlineto{\pgfqpoint{5.496240in}{2.135000in}}%
\pgfpathlineto{\pgfqpoint{5.497480in}{2.380000in}}%
\pgfpathlineto{\pgfqpoint{5.498720in}{1.995000in}}%
\pgfpathlineto{\pgfqpoint{5.502440in}{2.555000in}}%
\pgfpathlineto{\pgfqpoint{5.503680in}{2.205000in}}%
\pgfpathlineto{\pgfqpoint{5.506160in}{2.555000in}}%
\pgfpathlineto{\pgfqpoint{5.507400in}{2.275000in}}%
\pgfpathlineto{\pgfqpoint{5.508640in}{2.380000in}}%
\pgfpathlineto{\pgfqpoint{5.509880in}{2.100000in}}%
\pgfpathlineto{\pgfqpoint{5.511120in}{2.205000in}}%
\pgfpathlineto{\pgfqpoint{5.512360in}{2.520000in}}%
\pgfpathlineto{\pgfqpoint{5.514840in}{2.100000in}}%
\pgfpathlineto{\pgfqpoint{5.516080in}{2.485000in}}%
\pgfpathlineto{\pgfqpoint{5.518560in}{2.135000in}}%
\pgfpathlineto{\pgfqpoint{5.519800in}{2.205000in}}%
\pgfpathlineto{\pgfqpoint{5.521040in}{2.030000in}}%
\pgfpathlineto{\pgfqpoint{5.522280in}{2.485000in}}%
\pgfpathlineto{\pgfqpoint{5.523520in}{2.135000in}}%
\pgfpathlineto{\pgfqpoint{5.524760in}{2.240000in}}%
\pgfpathlineto{\pgfqpoint{5.526000in}{2.205000in}}%
\pgfpathlineto{\pgfqpoint{5.527240in}{2.310000in}}%
\pgfpathlineto{\pgfqpoint{5.528480in}{2.275000in}}%
\pgfpathlineto{\pgfqpoint{5.529720in}{2.345000in}}%
\pgfpathlineto{\pgfqpoint{5.530960in}{2.590000in}}%
\pgfpathlineto{\pgfqpoint{5.532200in}{2.555000in}}%
\pgfpathlineto{\pgfqpoint{5.533440in}{2.170000in}}%
\pgfpathlineto{\pgfqpoint{5.534680in}{2.275000in}}%
\pgfpathlineto{\pgfqpoint{5.535920in}{2.100000in}}%
\pgfpathlineto{\pgfqpoint{5.538400in}{2.450000in}}%
\pgfpathlineto{\pgfqpoint{5.540880in}{2.625000in}}%
\pgfpathlineto{\pgfqpoint{5.542120in}{2.450000in}}%
\pgfpathlineto{\pgfqpoint{5.543360in}{2.625000in}}%
\pgfpathlineto{\pgfqpoint{5.544600in}{2.415000in}}%
\pgfpathlineto{\pgfqpoint{5.548320in}{2.765000in}}%
\pgfpathlineto{\pgfqpoint{5.549560in}{2.275000in}}%
\pgfpathlineto{\pgfqpoint{5.552040in}{2.835000in}}%
\pgfpathlineto{\pgfqpoint{5.553280in}{2.450000in}}%
\pgfpathlineto{\pgfqpoint{5.554520in}{2.730000in}}%
\pgfpathlineto{\pgfqpoint{5.555760in}{2.065000in}}%
\pgfpathlineto{\pgfqpoint{5.557000in}{2.275000in}}%
\pgfpathlineto{\pgfqpoint{5.558240in}{2.100000in}}%
\pgfpathlineto{\pgfqpoint{5.560720in}{2.450000in}}%
\pgfpathlineto{\pgfqpoint{5.561960in}{2.170000in}}%
\pgfpathlineto{\pgfqpoint{5.564440in}{2.310000in}}%
\pgfpathlineto{\pgfqpoint{5.565680in}{2.520000in}}%
\pgfpathlineto{\pgfqpoint{5.566920in}{2.240000in}}%
\pgfpathlineto{\pgfqpoint{5.568160in}{2.625000in}}%
\pgfpathlineto{\pgfqpoint{5.569400in}{2.485000in}}%
\pgfpathlineto{\pgfqpoint{5.571880in}{2.765000in}}%
\pgfpathlineto{\pgfqpoint{5.573120in}{2.450000in}}%
\pgfpathlineto{\pgfqpoint{5.575600in}{2.800000in}}%
\pgfpathlineto{\pgfqpoint{5.576840in}{2.240000in}}%
\pgfpathlineto{\pgfqpoint{5.579320in}{2.555000in}}%
\pgfpathlineto{\pgfqpoint{5.580560in}{2.345000in}}%
\pgfpathlineto{\pgfqpoint{5.583040in}{2.450000in}}%
\pgfpathlineto{\pgfqpoint{5.584280in}{2.380000in}}%
\pgfpathlineto{\pgfqpoint{5.585520in}{2.240000in}}%
\pgfpathlineto{\pgfqpoint{5.586760in}{2.240000in}}%
\pgfpathlineto{\pgfqpoint{5.589240in}{2.695000in}}%
\pgfpathlineto{\pgfqpoint{5.591720in}{2.135000in}}%
\pgfpathlineto{\pgfqpoint{5.592960in}{2.380000in}}%
\pgfpathlineto{\pgfqpoint{5.594200in}{2.345000in}}%
\pgfpathlineto{\pgfqpoint{5.595440in}{2.730000in}}%
\pgfpathlineto{\pgfqpoint{5.599160in}{2.205000in}}%
\pgfpathlineto{\pgfqpoint{5.601640in}{2.555000in}}%
\pgfpathlineto{\pgfqpoint{5.602880in}{2.205000in}}%
\pgfpathlineto{\pgfqpoint{5.604120in}{2.205000in}}%
\pgfpathlineto{\pgfqpoint{5.605360in}{2.625000in}}%
\pgfpathlineto{\pgfqpoint{5.606600in}{2.240000in}}%
\pgfpathlineto{\pgfqpoint{5.607840in}{2.240000in}}%
\pgfpathlineto{\pgfqpoint{5.610320in}{2.485000in}}%
\pgfpathlineto{\pgfqpoint{5.612800in}{2.660000in}}%
\pgfpathlineto{\pgfqpoint{5.615280in}{2.380000in}}%
\pgfpathlineto{\pgfqpoint{5.616520in}{2.380000in}}%
\pgfpathlineto{\pgfqpoint{5.617760in}{2.345000in}}%
\pgfpathlineto{\pgfqpoint{5.619000in}{2.205000in}}%
\pgfpathlineto{\pgfqpoint{5.620240in}{2.345000in}}%
\pgfpathlineto{\pgfqpoint{5.621480in}{2.345000in}}%
\pgfpathlineto{\pgfqpoint{5.623960in}{2.100000in}}%
\pgfpathlineto{\pgfqpoint{5.627680in}{2.485000in}}%
\pgfpathlineto{\pgfqpoint{5.630160in}{2.345000in}}%
\pgfpathlineto{\pgfqpoint{5.631400in}{2.345000in}}%
\pgfpathlineto{\pgfqpoint{5.632640in}{2.275000in}}%
\pgfpathlineto{\pgfqpoint{5.633880in}{2.310000in}}%
\pgfpathlineto{\pgfqpoint{5.635120in}{2.240000in}}%
\pgfpathlineto{\pgfqpoint{5.636360in}{2.275000in}}%
\pgfpathlineto{\pgfqpoint{5.637600in}{2.555000in}}%
\pgfpathlineto{\pgfqpoint{5.638840in}{2.205000in}}%
\pgfpathlineto{\pgfqpoint{5.640080in}{2.415000in}}%
\pgfpathlineto{\pgfqpoint{5.641320in}{2.345000in}}%
\pgfpathlineto{\pgfqpoint{5.642560in}{2.555000in}}%
\pgfpathlineto{\pgfqpoint{5.643800in}{2.135000in}}%
\pgfpathlineto{\pgfqpoint{5.646280in}{2.800000in}}%
\pgfpathlineto{\pgfqpoint{5.647520in}{2.240000in}}%
\pgfpathlineto{\pgfqpoint{5.648760in}{2.240000in}}%
\pgfpathlineto{\pgfqpoint{5.650000in}{1.995000in}}%
\pgfpathlineto{\pgfqpoint{5.651240in}{2.520000in}}%
\pgfpathlineto{\pgfqpoint{5.653720in}{2.170000in}}%
\pgfpathlineto{\pgfqpoint{5.654960in}{2.555000in}}%
\pgfpathlineto{\pgfqpoint{5.656200in}{2.240000in}}%
\pgfpathlineto{\pgfqpoint{5.657440in}{2.415000in}}%
\pgfpathlineto{\pgfqpoint{5.658680in}{2.415000in}}%
\pgfpathlineto{\pgfqpoint{5.661160in}{2.275000in}}%
\pgfpathlineto{\pgfqpoint{5.662400in}{2.520000in}}%
\pgfpathlineto{\pgfqpoint{5.664880in}{2.345000in}}%
\pgfpathlineto{\pgfqpoint{5.666120in}{2.590000in}}%
\pgfpathlineto{\pgfqpoint{5.668600in}{2.590000in}}%
\pgfpathlineto{\pgfqpoint{5.669840in}{2.485000in}}%
\pgfpathlineto{\pgfqpoint{5.671080in}{2.905000in}}%
\pgfpathlineto{\pgfqpoint{5.673560in}{2.205000in}}%
\pgfpathlineto{\pgfqpoint{5.674800in}{2.450000in}}%
\pgfpathlineto{\pgfqpoint{5.676040in}{2.415000in}}%
\pgfpathlineto{\pgfqpoint{5.677280in}{2.415000in}}%
\pgfpathlineto{\pgfqpoint{5.678520in}{2.380000in}}%
\pgfpathlineto{\pgfqpoint{5.679760in}{2.205000in}}%
\pgfpathlineto{\pgfqpoint{5.681000in}{2.380000in}}%
\pgfpathlineto{\pgfqpoint{5.683480in}{2.275000in}}%
\pgfpathlineto{\pgfqpoint{5.684720in}{2.590000in}}%
\pgfpathlineto{\pgfqpoint{5.685960in}{2.205000in}}%
\pgfpathlineto{\pgfqpoint{5.687200in}{2.310000in}}%
\pgfpathlineto{\pgfqpoint{5.688440in}{2.065000in}}%
\pgfpathlineto{\pgfqpoint{5.689680in}{2.100000in}}%
\pgfpathlineto{\pgfqpoint{5.692160in}{2.485000in}}%
\pgfpathlineto{\pgfqpoint{5.693400in}{2.555000in}}%
\pgfpathlineto{\pgfqpoint{5.694640in}{2.555000in}}%
\pgfpathlineto{\pgfqpoint{5.695880in}{2.590000in}}%
\pgfpathlineto{\pgfqpoint{5.698360in}{2.520000in}}%
\pgfpathlineto{\pgfqpoint{5.699600in}{2.625000in}}%
\pgfpathlineto{\pgfqpoint{5.702080in}{2.170000in}}%
\pgfpathlineto{\pgfqpoint{5.703320in}{2.310000in}}%
\pgfpathlineto{\pgfqpoint{5.704560in}{2.065000in}}%
\pgfpathlineto{\pgfqpoint{5.707040in}{2.590000in}}%
\pgfpathlineto{\pgfqpoint{5.708280in}{2.660000in}}%
\pgfpathlineto{\pgfqpoint{5.709520in}{2.310000in}}%
\pgfpathlineto{\pgfqpoint{5.710760in}{2.345000in}}%
\pgfpathlineto{\pgfqpoint{5.712000in}{2.590000in}}%
\pgfpathlineto{\pgfqpoint{5.713240in}{2.380000in}}%
\pgfpathlineto{\pgfqpoint{5.714480in}{2.590000in}}%
\pgfpathlineto{\pgfqpoint{5.716960in}{2.170000in}}%
\pgfpathlineto{\pgfqpoint{5.718200in}{2.345000in}}%
\pgfpathlineto{\pgfqpoint{5.719440in}{2.205000in}}%
\pgfpathlineto{\pgfqpoint{5.720680in}{2.310000in}}%
\pgfpathlineto{\pgfqpoint{5.721920in}{2.625000in}}%
\pgfpathlineto{\pgfqpoint{5.723160in}{2.450000in}}%
\pgfpathlineto{\pgfqpoint{5.724400in}{2.870000in}}%
\pgfpathlineto{\pgfqpoint{5.725640in}{2.730000in}}%
\pgfpathlineto{\pgfqpoint{5.726880in}{2.205000in}}%
\pgfpathlineto{\pgfqpoint{5.728120in}{2.345000in}}%
\pgfpathlineto{\pgfqpoint{5.730600in}{2.240000in}}%
\pgfpathlineto{\pgfqpoint{5.731840in}{2.450000in}}%
\pgfpathlineto{\pgfqpoint{5.734320in}{2.275000in}}%
\pgfpathlineto{\pgfqpoint{5.736800in}{2.555000in}}%
\pgfpathlineto{\pgfqpoint{5.738040in}{2.275000in}}%
\pgfpathlineto{\pgfqpoint{5.739280in}{2.310000in}}%
\pgfpathlineto{\pgfqpoint{5.740520in}{1.925000in}}%
\pgfpathlineto{\pgfqpoint{5.743000in}{2.240000in}}%
\pgfpathlineto{\pgfqpoint{5.744240in}{2.205000in}}%
\pgfpathlineto{\pgfqpoint{5.745480in}{2.380000in}}%
\pgfpathlineto{\pgfqpoint{5.746720in}{2.345000in}}%
\pgfpathlineto{\pgfqpoint{5.747960in}{2.275000in}}%
\pgfpathlineto{\pgfqpoint{5.749200in}{2.380000in}}%
\pgfpathlineto{\pgfqpoint{5.750440in}{2.835000in}}%
\pgfpathlineto{\pgfqpoint{5.752920in}{2.380000in}}%
\pgfpathlineto{\pgfqpoint{5.754160in}{2.590000in}}%
\pgfpathlineto{\pgfqpoint{5.755400in}{2.590000in}}%
\pgfpathlineto{\pgfqpoint{5.756640in}{2.625000in}}%
\pgfpathlineto{\pgfqpoint{5.757880in}{2.520000in}}%
\pgfpathlineto{\pgfqpoint{5.759120in}{2.240000in}}%
\pgfpathlineto{\pgfqpoint{5.760360in}{2.450000in}}%
\pgfpathlineto{\pgfqpoint{5.761600in}{2.205000in}}%
\pgfpathlineto{\pgfqpoint{5.762840in}{2.625000in}}%
\pgfpathlineto{\pgfqpoint{5.764080in}{2.625000in}}%
\pgfpathlineto{\pgfqpoint{5.765320in}{2.170000in}}%
\pgfpathlineto{\pgfqpoint{5.766560in}{2.275000in}}%
\pgfpathlineto{\pgfqpoint{5.767800in}{1.960000in}}%
\pgfpathlineto{\pgfqpoint{5.769040in}{2.415000in}}%
\pgfpathlineto{\pgfqpoint{5.770280in}{2.415000in}}%
\pgfpathlineto{\pgfqpoint{5.771520in}{2.625000in}}%
\pgfpathlineto{\pgfqpoint{5.774000in}{2.345000in}}%
\pgfpathlineto{\pgfqpoint{5.775240in}{2.555000in}}%
\pgfpathlineto{\pgfqpoint{5.776480in}{2.345000in}}%
\pgfpathlineto{\pgfqpoint{5.780200in}{2.660000in}}%
\pgfpathlineto{\pgfqpoint{5.782680in}{2.275000in}}%
\pgfpathlineto{\pgfqpoint{5.785160in}{2.520000in}}%
\pgfpathlineto{\pgfqpoint{5.786400in}{2.485000in}}%
\pgfpathlineto{\pgfqpoint{5.787640in}{2.135000in}}%
\pgfpathlineto{\pgfqpoint{5.788880in}{2.485000in}}%
\pgfpathlineto{\pgfqpoint{5.790120in}{2.520000in}}%
\pgfpathlineto{\pgfqpoint{5.791360in}{2.660000in}}%
\pgfpathlineto{\pgfqpoint{5.795080in}{2.205000in}}%
\pgfpathlineto{\pgfqpoint{5.796320in}{2.135000in}}%
\pgfpathlineto{\pgfqpoint{5.797560in}{2.520000in}}%
\pgfpathlineto{\pgfqpoint{5.798800in}{2.485000in}}%
\pgfpathlineto{\pgfqpoint{5.800040in}{2.625000in}}%
\pgfpathlineto{\pgfqpoint{5.801280in}{2.345000in}}%
\pgfpathlineto{\pgfqpoint{5.802520in}{2.520000in}}%
\pgfpathlineto{\pgfqpoint{5.803760in}{2.415000in}}%
\pgfpathlineto{\pgfqpoint{5.805000in}{2.485000in}}%
\pgfpathlineto{\pgfqpoint{5.806240in}{2.205000in}}%
\pgfpathlineto{\pgfqpoint{5.807480in}{2.240000in}}%
\pgfpathlineto{\pgfqpoint{5.808720in}{2.135000in}}%
\pgfpathlineto{\pgfqpoint{5.809960in}{2.170000in}}%
\pgfpathlineto{\pgfqpoint{5.811200in}{1.960000in}}%
\pgfpathlineto{\pgfqpoint{5.812440in}{2.240000in}}%
\pgfpathlineto{\pgfqpoint{5.813680in}{2.065000in}}%
\pgfpathlineto{\pgfqpoint{5.814920in}{2.555000in}}%
\pgfpathlineto{\pgfqpoint{5.816160in}{2.415000in}}%
\pgfpathlineto{\pgfqpoint{5.817400in}{2.485000in}}%
\pgfpathlineto{\pgfqpoint{5.819880in}{2.170000in}}%
\pgfpathlineto{\pgfqpoint{5.821120in}{2.240000in}}%
\pgfpathlineto{\pgfqpoint{5.823600in}{2.625000in}}%
\pgfpathlineto{\pgfqpoint{5.824840in}{2.555000in}}%
\pgfpathlineto{\pgfqpoint{5.826080in}{2.065000in}}%
\pgfpathlineto{\pgfqpoint{5.828560in}{2.520000in}}%
\pgfpathlineto{\pgfqpoint{5.829800in}{2.765000in}}%
\pgfpathlineto{\pgfqpoint{5.832280in}{2.065000in}}%
\pgfpathlineto{\pgfqpoint{5.833520in}{2.275000in}}%
\pgfpathlineto{\pgfqpoint{5.834760in}{2.240000in}}%
\pgfpathlineto{\pgfqpoint{5.836000in}{2.275000in}}%
\pgfpathlineto{\pgfqpoint{5.837240in}{2.450000in}}%
\pgfpathlineto{\pgfqpoint{5.838480in}{2.170000in}}%
\pgfpathlineto{\pgfqpoint{5.840960in}{2.695000in}}%
\pgfpathlineto{\pgfqpoint{5.842200in}{2.800000in}}%
\pgfpathlineto{\pgfqpoint{5.843440in}{2.800000in}}%
\pgfpathlineto{\pgfqpoint{5.844680in}{2.730000in}}%
\pgfpathlineto{\pgfqpoint{5.845920in}{2.275000in}}%
\pgfpathlineto{\pgfqpoint{5.847160in}{2.450000in}}%
\pgfpathlineto{\pgfqpoint{5.848400in}{2.240000in}}%
\pgfpathlineto{\pgfqpoint{5.849640in}{2.625000in}}%
\pgfpathlineto{\pgfqpoint{5.850880in}{2.030000in}}%
\pgfpathlineto{\pgfqpoint{5.853360in}{2.590000in}}%
\pgfpathlineto{\pgfqpoint{5.854600in}{2.660000in}}%
\pgfpathlineto{\pgfqpoint{5.855840in}{2.975000in}}%
\pgfpathlineto{\pgfqpoint{5.857080in}{2.520000in}}%
\pgfpathlineto{\pgfqpoint{5.858320in}{2.695000in}}%
\pgfpathlineto{\pgfqpoint{5.859560in}{2.065000in}}%
\pgfpathlineto{\pgfqpoint{5.860800in}{2.205000in}}%
\pgfpathlineto{\pgfqpoint{5.862040in}{2.625000in}}%
\pgfpathlineto{\pgfqpoint{5.863280in}{2.345000in}}%
\pgfpathlineto{\pgfqpoint{5.864520in}{2.380000in}}%
\pgfpathlineto{\pgfqpoint{5.865760in}{2.275000in}}%
\pgfpathlineto{\pgfqpoint{5.867000in}{2.450000in}}%
\pgfpathlineto{\pgfqpoint{5.868240in}{2.450000in}}%
\pgfpathlineto{\pgfqpoint{5.869480in}{2.240000in}}%
\pgfpathlineto{\pgfqpoint{5.870720in}{2.625000in}}%
\pgfpathlineto{\pgfqpoint{5.871960in}{2.240000in}}%
\pgfpathlineto{\pgfqpoint{5.875680in}{2.730000in}}%
\pgfpathlineto{\pgfqpoint{5.876920in}{2.625000in}}%
\pgfpathlineto{\pgfqpoint{5.878160in}{2.660000in}}%
\pgfpathlineto{\pgfqpoint{5.879400in}{2.590000in}}%
\pgfpathlineto{\pgfqpoint{5.880640in}{2.450000in}}%
\pgfpathlineto{\pgfqpoint{5.881880in}{2.590000in}}%
\pgfpathlineto{\pgfqpoint{5.883120in}{2.555000in}}%
\pgfpathlineto{\pgfqpoint{5.884360in}{2.485000in}}%
\pgfpathlineto{\pgfqpoint{5.886840in}{2.205000in}}%
\pgfpathlineto{\pgfqpoint{5.889320in}{2.415000in}}%
\pgfpathlineto{\pgfqpoint{5.890560in}{2.065000in}}%
\pgfpathlineto{\pgfqpoint{5.891800in}{2.275000in}}%
\pgfpathlineto{\pgfqpoint{5.893040in}{2.275000in}}%
\pgfpathlineto{\pgfqpoint{5.894280in}{2.135000in}}%
\pgfpathlineto{\pgfqpoint{5.895520in}{2.485000in}}%
\pgfpathlineto{\pgfqpoint{5.896760in}{2.345000in}}%
\pgfpathlineto{\pgfqpoint{5.898000in}{2.555000in}}%
\pgfpathlineto{\pgfqpoint{5.899240in}{2.100000in}}%
\pgfpathlineto{\pgfqpoint{5.901720in}{2.485000in}}%
\pgfpathlineto{\pgfqpoint{5.902960in}{2.170000in}}%
\pgfpathlineto{\pgfqpoint{5.904200in}{2.240000in}}%
\pgfpathlineto{\pgfqpoint{5.905440in}{2.100000in}}%
\pgfpathlineto{\pgfqpoint{5.906680in}{2.520000in}}%
\pgfpathlineto{\pgfqpoint{5.907920in}{2.030000in}}%
\pgfpathlineto{\pgfqpoint{5.909160in}{1.995000in}}%
\pgfpathlineto{\pgfqpoint{5.910400in}{2.695000in}}%
\pgfpathlineto{\pgfqpoint{5.912880in}{2.345000in}}%
\pgfpathlineto{\pgfqpoint{5.914120in}{2.450000in}}%
\pgfpathlineto{\pgfqpoint{5.916600in}{2.870000in}}%
\pgfpathlineto{\pgfqpoint{5.917840in}{2.590000in}}%
\pgfpathlineto{\pgfqpoint{5.919080in}{2.800000in}}%
\pgfpathlineto{\pgfqpoint{5.920320in}{2.415000in}}%
\pgfpathlineto{\pgfqpoint{5.921560in}{2.555000in}}%
\pgfpathlineto{\pgfqpoint{5.922800in}{2.870000in}}%
\pgfpathlineto{\pgfqpoint{5.924040in}{2.625000in}}%
\pgfpathlineto{\pgfqpoint{5.925280in}{2.730000in}}%
\pgfpathlineto{\pgfqpoint{5.926520in}{2.660000in}}%
\pgfpathlineto{\pgfqpoint{5.927760in}{2.380000in}}%
\pgfpathlineto{\pgfqpoint{5.929000in}{3.045000in}}%
\pgfpathlineto{\pgfqpoint{5.931480in}{2.450000in}}%
\pgfpathlineto{\pgfqpoint{5.933960in}{2.100000in}}%
\pgfpathlineto{\pgfqpoint{5.935200in}{2.240000in}}%
\pgfpathlineto{\pgfqpoint{5.936440in}{1.995000in}}%
\pgfpathlineto{\pgfqpoint{5.938920in}{2.800000in}}%
\pgfpathlineto{\pgfqpoint{5.940160in}{2.730000in}}%
\pgfpathlineto{\pgfqpoint{5.941400in}{2.555000in}}%
\pgfpathlineto{\pgfqpoint{5.942640in}{2.205000in}}%
\pgfpathlineto{\pgfqpoint{5.943880in}{2.310000in}}%
\pgfpathlineto{\pgfqpoint{5.945120in}{2.205000in}}%
\pgfpathlineto{\pgfqpoint{5.946360in}{2.555000in}}%
\pgfpathlineto{\pgfqpoint{5.948840in}{2.205000in}}%
\pgfpathlineto{\pgfqpoint{5.950080in}{2.555000in}}%
\pgfpathlineto{\pgfqpoint{5.951320in}{2.345000in}}%
\pgfpathlineto{\pgfqpoint{5.953800in}{2.625000in}}%
\pgfpathlineto{\pgfqpoint{5.956280in}{2.520000in}}%
\pgfpathlineto{\pgfqpoint{5.957520in}{2.345000in}}%
\pgfpathlineto{\pgfqpoint{5.958760in}{2.660000in}}%
\pgfpathlineto{\pgfqpoint{5.960000in}{1.995000in}}%
\pgfpathlineto{\pgfqpoint{5.961240in}{2.415000in}}%
\pgfpathlineto{\pgfqpoint{5.962480in}{2.345000in}}%
\pgfpathlineto{\pgfqpoint{5.963720in}{2.590000in}}%
\pgfpathlineto{\pgfqpoint{5.964960in}{2.345000in}}%
\pgfpathlineto{\pgfqpoint{5.966200in}{2.660000in}}%
\pgfpathlineto{\pgfqpoint{5.967440in}{2.590000in}}%
\pgfpathlineto{\pgfqpoint{5.968680in}{2.345000in}}%
\pgfpathlineto{\pgfqpoint{5.969920in}{2.520000in}}%
\pgfpathlineto{\pgfqpoint{5.972400in}{2.240000in}}%
\pgfpathlineto{\pgfqpoint{5.973640in}{2.100000in}}%
\pgfpathlineto{\pgfqpoint{5.974880in}{2.380000in}}%
\pgfpathlineto{\pgfqpoint{5.976120in}{2.345000in}}%
\pgfpathlineto{\pgfqpoint{5.977360in}{2.240000in}}%
\pgfpathlineto{\pgfqpoint{5.978600in}{2.625000in}}%
\pgfpathlineto{\pgfqpoint{5.979840in}{2.275000in}}%
\pgfpathlineto{\pgfqpoint{5.981080in}{2.625000in}}%
\pgfpathlineto{\pgfqpoint{5.983560in}{2.345000in}}%
\pgfpathlineto{\pgfqpoint{5.984800in}{2.380000in}}%
\pgfpathlineto{\pgfqpoint{5.986040in}{2.485000in}}%
\pgfpathlineto{\pgfqpoint{5.987280in}{2.415000in}}%
\pgfpathlineto{\pgfqpoint{5.988520in}{2.030000in}}%
\pgfpathlineto{\pgfqpoint{5.989760in}{2.415000in}}%
\pgfpathlineto{\pgfqpoint{5.991000in}{2.240000in}}%
\pgfpathlineto{\pgfqpoint{5.992240in}{2.835000in}}%
\pgfpathlineto{\pgfqpoint{5.994720in}{1.855000in}}%
\pgfpathlineto{\pgfqpoint{5.995960in}{2.100000in}}%
\pgfpathlineto{\pgfqpoint{5.997200in}{1.960000in}}%
\pgfpathlineto{\pgfqpoint{5.998440in}{2.100000in}}%
\pgfpathlineto{\pgfqpoint{5.999680in}{2.485000in}}%
\pgfpathlineto{\pgfqpoint{6.000920in}{2.310000in}}%
\pgfpathlineto{\pgfqpoint{6.002160in}{2.555000in}}%
\pgfpathlineto{\pgfqpoint{6.003400in}{2.450000in}}%
\pgfpathlineto{\pgfqpoint{6.004640in}{2.520000in}}%
\pgfpathlineto{\pgfqpoint{6.007120in}{1.890000in}}%
\pgfpathlineto{\pgfqpoint{6.008360in}{1.995000in}}%
\pgfpathlineto{\pgfqpoint{6.010840in}{2.275000in}}%
\pgfpathlineto{\pgfqpoint{6.012080in}{2.240000in}}%
\pgfpathlineto{\pgfqpoint{6.014560in}{1.855000in}}%
\pgfpathlineto{\pgfqpoint{6.017040in}{2.765000in}}%
\pgfpathlineto{\pgfqpoint{6.018280in}{1.995000in}}%
\pgfpathlineto{\pgfqpoint{6.020760in}{2.485000in}}%
\pgfpathlineto{\pgfqpoint{6.022000in}{2.310000in}}%
\pgfpathlineto{\pgfqpoint{6.023240in}{2.345000in}}%
\pgfpathlineto{\pgfqpoint{6.024480in}{1.995000in}}%
\pgfpathlineto{\pgfqpoint{6.026960in}{2.485000in}}%
\pgfpathlineto{\pgfqpoint{6.029440in}{2.310000in}}%
\pgfpathlineto{\pgfqpoint{6.030680in}{2.520000in}}%
\pgfpathlineto{\pgfqpoint{6.033160in}{2.100000in}}%
\pgfpathlineto{\pgfqpoint{6.036880in}{2.730000in}}%
\pgfpathlineto{\pgfqpoint{6.039360in}{2.135000in}}%
\pgfpathlineto{\pgfqpoint{6.041840in}{2.940000in}}%
\pgfpathlineto{\pgfqpoint{6.043080in}{2.100000in}}%
\pgfpathlineto{\pgfqpoint{6.045560in}{2.555000in}}%
\pgfpathlineto{\pgfqpoint{6.048040in}{2.240000in}}%
\pgfpathlineto{\pgfqpoint{6.049280in}{2.310000in}}%
\pgfpathlineto{\pgfqpoint{6.050520in}{2.310000in}}%
\pgfpathlineto{\pgfqpoint{6.053000in}{2.065000in}}%
\pgfpathlineto{\pgfqpoint{6.055480in}{2.345000in}}%
\pgfpathlineto{\pgfqpoint{6.056720in}{2.625000in}}%
\pgfpathlineto{\pgfqpoint{6.059200in}{2.275000in}}%
\pgfpathlineto{\pgfqpoint{6.060440in}{2.170000in}}%
\pgfpathlineto{\pgfqpoint{6.061680in}{2.485000in}}%
\pgfpathlineto{\pgfqpoint{6.062920in}{2.275000in}}%
\pgfpathlineto{\pgfqpoint{6.064160in}{2.730000in}}%
\pgfpathlineto{\pgfqpoint{6.065400in}{2.100000in}}%
\pgfpathlineto{\pgfqpoint{6.066640in}{2.135000in}}%
\pgfpathlineto{\pgfqpoint{6.067880in}{2.555000in}}%
\pgfpathlineto{\pgfqpoint{6.069120in}{2.205000in}}%
\pgfpathlineto{\pgfqpoint{6.071600in}{2.205000in}}%
\pgfpathlineto{\pgfqpoint{6.072840in}{2.485000in}}%
\pgfpathlineto{\pgfqpoint{6.076560in}{2.135000in}}%
\pgfpathlineto{\pgfqpoint{6.077800in}{2.275000in}}%
\pgfpathlineto{\pgfqpoint{6.079040in}{2.275000in}}%
\pgfpathlineto{\pgfqpoint{6.080280in}{2.135000in}}%
\pgfpathlineto{\pgfqpoint{6.081520in}{2.625000in}}%
\pgfpathlineto{\pgfqpoint{6.082760in}{2.555000in}}%
\pgfpathlineto{\pgfqpoint{6.084000in}{2.660000in}}%
\pgfpathlineto{\pgfqpoint{6.086480in}{2.170000in}}%
\pgfpathlineto{\pgfqpoint{6.087720in}{2.520000in}}%
\pgfpathlineto{\pgfqpoint{6.088960in}{2.310000in}}%
\pgfpathlineto{\pgfqpoint{6.090200in}{2.310000in}}%
\pgfpathlineto{\pgfqpoint{6.091440in}{2.240000in}}%
\pgfpathlineto{\pgfqpoint{6.092680in}{2.625000in}}%
\pgfpathlineto{\pgfqpoint{6.093920in}{2.660000in}}%
\pgfpathlineto{\pgfqpoint{6.095160in}{2.240000in}}%
\pgfpathlineto{\pgfqpoint{6.096400in}{2.730000in}}%
\pgfpathlineto{\pgfqpoint{6.097640in}{2.415000in}}%
\pgfpathlineto{\pgfqpoint{6.098880in}{2.940000in}}%
\pgfpathlineto{\pgfqpoint{6.101360in}{2.415000in}}%
\pgfpathlineto{\pgfqpoint{6.102600in}{2.695000in}}%
\pgfpathlineto{\pgfqpoint{6.105080in}{2.450000in}}%
\pgfpathlineto{\pgfqpoint{6.106320in}{2.625000in}}%
\pgfpathlineto{\pgfqpoint{6.110040in}{2.240000in}}%
\pgfpathlineto{\pgfqpoint{6.111280in}{2.590000in}}%
\pgfpathlineto{\pgfqpoint{6.112520in}{2.345000in}}%
\pgfpathlineto{\pgfqpoint{6.113760in}{2.520000in}}%
\pgfpathlineto{\pgfqpoint{6.115000in}{2.240000in}}%
\pgfpathlineto{\pgfqpoint{6.116240in}{2.485000in}}%
\pgfpathlineto{\pgfqpoint{6.117480in}{2.170000in}}%
\pgfpathlineto{\pgfqpoint{6.118720in}{2.415000in}}%
\pgfpathlineto{\pgfqpoint{6.121200in}{2.240000in}}%
\pgfpathlineto{\pgfqpoint{6.123680in}{2.625000in}}%
\pgfpathlineto{\pgfqpoint{6.124920in}{2.485000in}}%
\pgfpathlineto{\pgfqpoint{6.126160in}{2.625000in}}%
\pgfpathlineto{\pgfqpoint{6.127400in}{2.905000in}}%
\pgfpathlineto{\pgfqpoint{6.129880in}{2.380000in}}%
\pgfpathlineto{\pgfqpoint{6.131120in}{2.380000in}}%
\pgfpathlineto{\pgfqpoint{6.132360in}{2.275000in}}%
\pgfpathlineto{\pgfqpoint{6.133600in}{2.590000in}}%
\pgfpathlineto{\pgfqpoint{6.134840in}{2.520000in}}%
\pgfpathlineto{\pgfqpoint{6.136080in}{2.730000in}}%
\pgfpathlineto{\pgfqpoint{6.138560in}{2.240000in}}%
\pgfpathlineto{\pgfqpoint{6.141040in}{2.590000in}}%
\pgfpathlineto{\pgfqpoint{6.142280in}{2.275000in}}%
\pgfpathlineto{\pgfqpoint{6.143520in}{2.380000in}}%
\pgfpathlineto{\pgfqpoint{6.146000in}{2.240000in}}%
\pgfpathlineto{\pgfqpoint{6.147240in}{2.485000in}}%
\pgfpathlineto{\pgfqpoint{6.149720in}{2.205000in}}%
\pgfpathlineto{\pgfqpoint{6.150960in}{2.205000in}}%
\pgfpathlineto{\pgfqpoint{6.153440in}{2.415000in}}%
\pgfpathlineto{\pgfqpoint{6.154680in}{2.415000in}}%
\pgfpathlineto{\pgfqpoint{6.155920in}{2.450000in}}%
\pgfpathlineto{\pgfqpoint{6.157160in}{2.555000in}}%
\pgfpathlineto{\pgfqpoint{6.158400in}{2.765000in}}%
\pgfpathlineto{\pgfqpoint{6.159640in}{2.730000in}}%
\pgfpathlineto{\pgfqpoint{6.160880in}{2.660000in}}%
\pgfpathlineto{\pgfqpoint{6.162120in}{2.240000in}}%
\pgfpathlineto{\pgfqpoint{6.164600in}{2.555000in}}%
\pgfpathlineto{\pgfqpoint{6.165840in}{2.555000in}}%
\pgfpathlineto{\pgfqpoint{6.167080in}{2.450000in}}%
\pgfpathlineto{\pgfqpoint{6.168320in}{2.625000in}}%
\pgfpathlineto{\pgfqpoint{6.169560in}{2.555000in}}%
\pgfpathlineto{\pgfqpoint{6.170800in}{2.590000in}}%
\pgfpathlineto{\pgfqpoint{6.172040in}{2.380000in}}%
\pgfpathlineto{\pgfqpoint{6.173280in}{2.485000in}}%
\pgfpathlineto{\pgfqpoint{6.174520in}{2.450000in}}%
\pgfpathlineto{\pgfqpoint{6.177000in}{2.520000in}}%
\pgfpathlineto{\pgfqpoint{6.178240in}{2.240000in}}%
\pgfpathlineto{\pgfqpoint{6.179480in}{2.625000in}}%
\pgfpathlineto{\pgfqpoint{6.180720in}{2.485000in}}%
\pgfpathlineto{\pgfqpoint{6.181960in}{2.135000in}}%
\pgfpathlineto{\pgfqpoint{6.183200in}{2.695000in}}%
\pgfpathlineto{\pgfqpoint{6.184440in}{2.590000in}}%
\pgfpathlineto{\pgfqpoint{6.185680in}{2.380000in}}%
\pgfpathlineto{\pgfqpoint{6.186920in}{2.695000in}}%
\pgfpathlineto{\pgfqpoint{6.189400in}{2.415000in}}%
\pgfpathlineto{\pgfqpoint{6.190640in}{2.450000in}}%
\pgfpathlineto{\pgfqpoint{6.191880in}{2.450000in}}%
\pgfpathlineto{\pgfqpoint{6.194360in}{2.240000in}}%
\pgfpathlineto{\pgfqpoint{6.195600in}{2.555000in}}%
\pgfpathlineto{\pgfqpoint{6.196840in}{2.415000in}}%
\pgfpathlineto{\pgfqpoint{6.199320in}{2.625000in}}%
\pgfpathlineto{\pgfqpoint{6.200560in}{2.310000in}}%
\pgfpathlineto{\pgfqpoint{6.201800in}{2.310000in}}%
\pgfpathlineto{\pgfqpoint{6.203040in}{2.205000in}}%
\pgfpathlineto{\pgfqpoint{6.205520in}{2.905000in}}%
\pgfpathlineto{\pgfqpoint{6.206760in}{2.555000in}}%
\pgfpathlineto{\pgfqpoint{6.208000in}{2.590000in}}%
\pgfpathlineto{\pgfqpoint{6.209240in}{2.240000in}}%
\pgfpathlineto{\pgfqpoint{6.210480in}{2.450000in}}%
\pgfpathlineto{\pgfqpoint{6.211720in}{2.135000in}}%
\pgfpathlineto{\pgfqpoint{6.212960in}{2.380000in}}%
\pgfpathlineto{\pgfqpoint{6.214200in}{2.135000in}}%
\pgfpathlineto{\pgfqpoint{6.216680in}{2.555000in}}%
\pgfpathlineto{\pgfqpoint{6.217920in}{2.415000in}}%
\pgfpathlineto{\pgfqpoint{6.219160in}{2.555000in}}%
\pgfpathlineto{\pgfqpoint{6.220400in}{2.205000in}}%
\pgfpathlineto{\pgfqpoint{6.221640in}{2.520000in}}%
\pgfpathlineto{\pgfqpoint{6.224120in}{2.240000in}}%
\pgfpathlineto{\pgfqpoint{6.226600in}{2.450000in}}%
\pgfpathlineto{\pgfqpoint{6.227840in}{2.205000in}}%
\pgfpathlineto{\pgfqpoint{6.229080in}{2.310000in}}%
\pgfpathlineto{\pgfqpoint{6.231560in}{2.625000in}}%
\pgfpathlineto{\pgfqpoint{6.232800in}{2.205000in}}%
\pgfpathlineto{\pgfqpoint{6.235280in}{2.310000in}}%
\pgfpathlineto{\pgfqpoint{6.236520in}{2.240000in}}%
\pgfpathlineto{\pgfqpoint{6.237760in}{2.415000in}}%
\pgfpathlineto{\pgfqpoint{6.239000in}{2.345000in}}%
\pgfpathlineto{\pgfqpoint{6.240240in}{2.450000in}}%
\pgfpathlineto{\pgfqpoint{6.241480in}{2.450000in}}%
\pgfpathlineto{\pgfqpoint{6.242720in}{2.170000in}}%
\pgfpathlineto{\pgfqpoint{6.245200in}{2.660000in}}%
\pgfpathlineto{\pgfqpoint{6.246440in}{2.520000in}}%
\pgfpathlineto{\pgfqpoint{6.247680in}{2.205000in}}%
\pgfpathlineto{\pgfqpoint{6.248920in}{2.345000in}}%
\pgfpathlineto{\pgfqpoint{6.250160in}{2.135000in}}%
\pgfpathlineto{\pgfqpoint{6.251400in}{2.310000in}}%
\pgfpathlineto{\pgfqpoint{6.252640in}{2.660000in}}%
\pgfpathlineto{\pgfqpoint{6.255120in}{2.135000in}}%
\pgfpathlineto{\pgfqpoint{6.256360in}{2.450000in}}%
\pgfpathlineto{\pgfqpoint{6.258840in}{2.170000in}}%
\pgfpathlineto{\pgfqpoint{6.261320in}{2.695000in}}%
\pgfpathlineto{\pgfqpoint{6.265040in}{2.135000in}}%
\pgfpathlineto{\pgfqpoint{6.267520in}{2.485000in}}%
\pgfpathlineto{\pgfqpoint{6.270000in}{2.415000in}}%
\pgfpathlineto{\pgfqpoint{6.271240in}{2.345000in}}%
\pgfpathlineto{\pgfqpoint{6.272480in}{2.625000in}}%
\pgfpathlineto{\pgfqpoint{6.273720in}{2.205000in}}%
\pgfpathlineto{\pgfqpoint{6.274960in}{2.695000in}}%
\pgfpathlineto{\pgfqpoint{6.276200in}{2.485000in}}%
\pgfpathlineto{\pgfqpoint{6.277440in}{2.765000in}}%
\pgfpathlineto{\pgfqpoint{6.278680in}{2.310000in}}%
\pgfpathlineto{\pgfqpoint{6.282400in}{2.695000in}}%
\pgfpathlineto{\pgfqpoint{6.283640in}{2.765000in}}%
\pgfpathlineto{\pgfqpoint{6.284880in}{2.240000in}}%
\pgfpathlineto{\pgfqpoint{6.286120in}{2.205000in}}%
\pgfpathlineto{\pgfqpoint{6.288600in}{2.380000in}}%
\pgfpathlineto{\pgfqpoint{6.289840in}{1.960000in}}%
\pgfpathlineto{\pgfqpoint{6.291080in}{2.415000in}}%
\pgfpathlineto{\pgfqpoint{6.292320in}{2.275000in}}%
\pgfpathlineto{\pgfqpoint{6.293560in}{2.275000in}}%
\pgfpathlineto{\pgfqpoint{6.294800in}{2.450000in}}%
\pgfpathlineto{\pgfqpoint{6.297280in}{2.275000in}}%
\pgfpathlineto{\pgfqpoint{6.298520in}{2.450000in}}%
\pgfpathlineto{\pgfqpoint{6.299760in}{1.785000in}}%
\pgfpathlineto{\pgfqpoint{6.301000in}{2.485000in}}%
\pgfpathlineto{\pgfqpoint{6.302240in}{2.170000in}}%
\pgfpathlineto{\pgfqpoint{6.303480in}{2.310000in}}%
\pgfpathlineto{\pgfqpoint{6.304720in}{2.135000in}}%
\pgfpathlineto{\pgfqpoint{6.305960in}{2.555000in}}%
\pgfpathlineto{\pgfqpoint{6.307200in}{2.520000in}}%
\pgfpathlineto{\pgfqpoint{6.308440in}{2.625000in}}%
\pgfpathlineto{\pgfqpoint{6.309680in}{2.555000in}}%
\pgfpathlineto{\pgfqpoint{6.310920in}{2.415000in}}%
\pgfpathlineto{\pgfqpoint{6.312160in}{2.625000in}}%
\pgfpathlineto{\pgfqpoint{6.313400in}{2.625000in}}%
\pgfpathlineto{\pgfqpoint{6.315880in}{2.240000in}}%
\pgfpathlineto{\pgfqpoint{6.320840in}{2.730000in}}%
\pgfpathlineto{\pgfqpoint{6.322080in}{2.695000in}}%
\pgfpathlineto{\pgfqpoint{6.323320in}{2.345000in}}%
\pgfpathlineto{\pgfqpoint{6.324560in}{2.765000in}}%
\pgfpathlineto{\pgfqpoint{6.325800in}{2.800000in}}%
\pgfpathlineto{\pgfqpoint{6.328280in}{2.345000in}}%
\pgfpathlineto{\pgfqpoint{6.330760in}{2.170000in}}%
\pgfpathlineto{\pgfqpoint{6.332000in}{2.485000in}}%
\pgfpathlineto{\pgfqpoint{6.333240in}{2.485000in}}%
\pgfpathlineto{\pgfqpoint{6.334480in}{2.205000in}}%
\pgfpathlineto{\pgfqpoint{6.335720in}{2.380000in}}%
\pgfpathlineto{\pgfqpoint{6.336960in}{1.960000in}}%
\pgfpathlineto{\pgfqpoint{6.339440in}{2.590000in}}%
\pgfpathlineto{\pgfqpoint{6.340680in}{2.590000in}}%
\pgfpathlineto{\pgfqpoint{6.343160in}{2.170000in}}%
\pgfpathlineto{\pgfqpoint{6.344400in}{2.240000in}}%
\pgfpathlineto{\pgfqpoint{6.345640in}{2.450000in}}%
\pgfpathlineto{\pgfqpoint{6.346880in}{2.275000in}}%
\pgfpathlineto{\pgfqpoint{6.348120in}{2.380000in}}%
\pgfpathlineto{\pgfqpoint{6.349360in}{2.590000in}}%
\pgfpathlineto{\pgfqpoint{6.351840in}{2.170000in}}%
\pgfpathlineto{\pgfqpoint{6.353080in}{2.555000in}}%
\pgfpathlineto{\pgfqpoint{6.358040in}{2.135000in}}%
\pgfpathlineto{\pgfqpoint{6.360520in}{2.520000in}}%
\pgfpathlineto{\pgfqpoint{6.361760in}{2.345000in}}%
\pgfpathlineto{\pgfqpoint{6.363000in}{2.485000in}}%
\pgfpathlineto{\pgfqpoint{6.364240in}{2.800000in}}%
\pgfpathlineto{\pgfqpoint{6.367960in}{2.100000in}}%
\pgfpathlineto{\pgfqpoint{6.369200in}{2.415000in}}%
\pgfpathlineto{\pgfqpoint{6.370440in}{2.380000in}}%
\pgfpathlineto{\pgfqpoint{6.371680in}{2.135000in}}%
\pgfpathlineto{\pgfqpoint{6.372920in}{2.345000in}}%
\pgfpathlineto{\pgfqpoint{6.374160in}{2.205000in}}%
\pgfpathlineto{\pgfqpoint{6.375400in}{2.660000in}}%
\pgfpathlineto{\pgfqpoint{6.376640in}{2.310000in}}%
\pgfpathlineto{\pgfqpoint{6.377880in}{2.450000in}}%
\pgfpathlineto{\pgfqpoint{6.379120in}{2.030000in}}%
\pgfpathlineto{\pgfqpoint{6.380360in}{2.695000in}}%
\pgfpathlineto{\pgfqpoint{6.381600in}{2.380000in}}%
\pgfpathlineto{\pgfqpoint{6.385320in}{2.730000in}}%
\pgfpathlineto{\pgfqpoint{6.387800in}{2.380000in}}%
\pgfpathlineto{\pgfqpoint{6.389040in}{2.555000in}}%
\pgfpathlineto{\pgfqpoint{6.390280in}{2.275000in}}%
\pgfpathlineto{\pgfqpoint{6.394000in}{2.660000in}}%
\pgfpathlineto{\pgfqpoint{6.395240in}{2.520000in}}%
\pgfpathlineto{\pgfqpoint{6.396480in}{2.555000in}}%
\pgfpathlineto{\pgfqpoint{6.397720in}{2.555000in}}%
\pgfpathlineto{\pgfqpoint{6.398960in}{2.905000in}}%
\pgfpathlineto{\pgfqpoint{6.402680in}{2.380000in}}%
\pgfpathlineto{\pgfqpoint{6.403920in}{2.695000in}}%
\pgfpathlineto{\pgfqpoint{6.406400in}{2.345000in}}%
\pgfpathlineto{\pgfqpoint{6.407640in}{2.415000in}}%
\pgfpathlineto{\pgfqpoint{6.408880in}{2.310000in}}%
\pgfpathlineto{\pgfqpoint{6.410120in}{2.380000in}}%
\pgfpathlineto{\pgfqpoint{6.411360in}{2.730000in}}%
\pgfpathlineto{\pgfqpoint{6.412600in}{2.590000in}}%
\pgfpathlineto{\pgfqpoint{6.413840in}{2.870000in}}%
\pgfpathlineto{\pgfqpoint{6.416320in}{2.450000in}}%
\pgfpathlineto{\pgfqpoint{6.417560in}{2.275000in}}%
\pgfpathlineto{\pgfqpoint{6.418800in}{2.275000in}}%
\pgfpathlineto{\pgfqpoint{6.420040in}{2.415000in}}%
\pgfpathlineto{\pgfqpoint{6.421280in}{2.205000in}}%
\pgfpathlineto{\pgfqpoint{6.422520in}{2.660000in}}%
\pgfpathlineto{\pgfqpoint{6.423760in}{2.660000in}}%
\pgfpathlineto{\pgfqpoint{6.425000in}{2.590000in}}%
\pgfpathlineto{\pgfqpoint{6.426240in}{2.240000in}}%
\pgfpathlineto{\pgfqpoint{6.427480in}{2.625000in}}%
\pgfpathlineto{\pgfqpoint{6.428720in}{2.485000in}}%
\pgfpathlineto{\pgfqpoint{6.429960in}{2.695000in}}%
\pgfpathlineto{\pgfqpoint{6.431200in}{2.380000in}}%
\pgfpathlineto{\pgfqpoint{6.432440in}{2.380000in}}%
\pgfpathlineto{\pgfqpoint{6.433680in}{2.450000in}}%
\pgfpathlineto{\pgfqpoint{6.434920in}{2.240000in}}%
\pgfpathlineto{\pgfqpoint{6.436160in}{2.450000in}}%
\pgfpathlineto{\pgfqpoint{6.437400in}{2.205000in}}%
\pgfpathlineto{\pgfqpoint{6.438640in}{2.590000in}}%
\pgfpathlineto{\pgfqpoint{6.439880in}{2.555000in}}%
\pgfpathlineto{\pgfqpoint{6.441120in}{2.275000in}}%
\pgfpathlineto{\pgfqpoint{6.442360in}{2.415000in}}%
\pgfpathlineto{\pgfqpoint{6.443600in}{2.135000in}}%
\pgfpathlineto{\pgfqpoint{6.444840in}{2.485000in}}%
\pgfpathlineto{\pgfqpoint{6.446080in}{2.310000in}}%
\pgfpathlineto{\pgfqpoint{6.447320in}{2.660000in}}%
\pgfpathlineto{\pgfqpoint{6.448560in}{2.625000in}}%
\pgfpathlineto{\pgfqpoint{6.449800in}{2.450000in}}%
\pgfpathlineto{\pgfqpoint{6.451040in}{2.695000in}}%
\pgfpathlineto{\pgfqpoint{6.453520in}{2.380000in}}%
\pgfpathlineto{\pgfqpoint{6.454760in}{2.345000in}}%
\pgfpathlineto{\pgfqpoint{6.456000in}{2.520000in}}%
\pgfpathlineto{\pgfqpoint{6.457240in}{2.065000in}}%
\pgfpathlineto{\pgfqpoint{6.458480in}{2.485000in}}%
\pgfpathlineto{\pgfqpoint{6.460960in}{2.100000in}}%
\pgfpathlineto{\pgfqpoint{6.462200in}{2.170000in}}%
\pgfpathlineto{\pgfqpoint{6.463440in}{2.380000in}}%
\pgfpathlineto{\pgfqpoint{6.464680in}{2.240000in}}%
\pgfpathlineto{\pgfqpoint{6.465920in}{2.520000in}}%
\pgfpathlineto{\pgfqpoint{6.467160in}{2.450000in}}%
\pgfpathlineto{\pgfqpoint{6.468400in}{2.660000in}}%
\pgfpathlineto{\pgfqpoint{6.469640in}{2.310000in}}%
\pgfpathlineto{\pgfqpoint{6.470880in}{2.450000in}}%
\pgfpathlineto{\pgfqpoint{6.472120in}{2.345000in}}%
\pgfpathlineto{\pgfqpoint{6.474600in}{2.485000in}}%
\pgfpathlineto{\pgfqpoint{6.475840in}{2.485000in}}%
\pgfpathlineto{\pgfqpoint{6.477080in}{2.730000in}}%
\pgfpathlineto{\pgfqpoint{6.478320in}{2.695000in}}%
\pgfpathlineto{\pgfqpoint{6.479560in}{2.240000in}}%
\pgfpathlineto{\pgfqpoint{6.480800in}{2.415000in}}%
\pgfpathlineto{\pgfqpoint{6.482040in}{2.415000in}}%
\pgfpathlineto{\pgfqpoint{6.483280in}{2.240000in}}%
\pgfpathlineto{\pgfqpoint{6.484520in}{2.590000in}}%
\pgfpathlineto{\pgfqpoint{6.485760in}{2.520000in}}%
\pgfpathlineto{\pgfqpoint{6.487000in}{2.555000in}}%
\pgfpathlineto{\pgfqpoint{6.488240in}{2.240000in}}%
\pgfpathlineto{\pgfqpoint{6.489480in}{2.450000in}}%
\pgfpathlineto{\pgfqpoint{6.490720in}{2.450000in}}%
\pgfpathlineto{\pgfqpoint{6.491960in}{2.520000in}}%
\pgfpathlineto{\pgfqpoint{6.493200in}{2.170000in}}%
\pgfpathlineto{\pgfqpoint{6.494440in}{2.275000in}}%
\pgfpathlineto{\pgfqpoint{6.495680in}{1.995000in}}%
\pgfpathlineto{\pgfqpoint{6.496920in}{2.345000in}}%
\pgfpathlineto{\pgfqpoint{6.498160in}{2.310000in}}%
\pgfpathlineto{\pgfqpoint{6.499400in}{2.450000in}}%
\pgfpathlineto{\pgfqpoint{6.500640in}{2.205000in}}%
\pgfpathlineto{\pgfqpoint{6.501880in}{2.240000in}}%
\pgfpathlineto{\pgfqpoint{6.503120in}{2.030000in}}%
\pgfpathlineto{\pgfqpoint{6.505600in}{2.205000in}}%
\pgfpathlineto{\pgfqpoint{6.506840in}{2.205000in}}%
\pgfpathlineto{\pgfqpoint{6.508080in}{2.415000in}}%
\pgfpathlineto{\pgfqpoint{6.510560in}{2.170000in}}%
\pgfpathlineto{\pgfqpoint{6.513040in}{2.520000in}}%
\pgfpathlineto{\pgfqpoint{6.514280in}{2.135000in}}%
\pgfpathlineto{\pgfqpoint{6.515520in}{2.380000in}}%
\pgfpathlineto{\pgfqpoint{6.516760in}{1.925000in}}%
\pgfpathlineto{\pgfqpoint{6.518000in}{2.310000in}}%
\pgfpathlineto{\pgfqpoint{6.519240in}{2.240000in}}%
\pgfpathlineto{\pgfqpoint{6.520480in}{2.275000in}}%
\pgfpathlineto{\pgfqpoint{6.521720in}{2.485000in}}%
\pgfpathlineto{\pgfqpoint{6.522960in}{2.275000in}}%
\pgfpathlineto{\pgfqpoint{6.525440in}{2.555000in}}%
\pgfpathlineto{\pgfqpoint{6.526680in}{2.135000in}}%
\pgfpathlineto{\pgfqpoint{6.527920in}{2.275000in}}%
\pgfpathlineto{\pgfqpoint{6.529160in}{2.205000in}}%
\pgfpathlineto{\pgfqpoint{6.530400in}{2.555000in}}%
\pgfpathlineto{\pgfqpoint{6.531640in}{2.380000in}}%
\pgfpathlineto{\pgfqpoint{6.532880in}{2.555000in}}%
\pgfpathlineto{\pgfqpoint{6.534120in}{2.310000in}}%
\pgfpathlineto{\pgfqpoint{6.535360in}{2.380000in}}%
\pgfpathlineto{\pgfqpoint{6.537840in}{2.100000in}}%
\pgfpathlineto{\pgfqpoint{6.539080in}{2.275000in}}%
\pgfpathlineto{\pgfqpoint{6.540320in}{2.275000in}}%
\pgfpathlineto{\pgfqpoint{6.541560in}{2.135000in}}%
\pgfpathlineto{\pgfqpoint{6.544040in}{2.415000in}}%
\pgfpathlineto{\pgfqpoint{6.545280in}{2.415000in}}%
\pgfpathlineto{\pgfqpoint{6.546520in}{2.275000in}}%
\pgfpathlineto{\pgfqpoint{6.547760in}{2.345000in}}%
\pgfpathlineto{\pgfqpoint{6.549000in}{2.275000in}}%
\pgfpathlineto{\pgfqpoint{6.550240in}{2.660000in}}%
\pgfpathlineto{\pgfqpoint{6.551480in}{2.170000in}}%
\pgfpathlineto{\pgfqpoint{6.552720in}{2.240000in}}%
\pgfpathlineto{\pgfqpoint{6.553960in}{2.170000in}}%
\pgfpathlineto{\pgfqpoint{6.556440in}{2.555000in}}%
\pgfpathlineto{\pgfqpoint{6.557680in}{1.890000in}}%
\pgfpathlineto{\pgfqpoint{6.558920in}{2.380000in}}%
\pgfpathlineto{\pgfqpoint{6.560160in}{2.310000in}}%
\pgfpathlineto{\pgfqpoint{6.561400in}{2.380000in}}%
\pgfpathlineto{\pgfqpoint{6.562640in}{2.555000in}}%
\pgfpathlineto{\pgfqpoint{6.563880in}{2.415000in}}%
\pgfpathlineto{\pgfqpoint{6.565120in}{2.415000in}}%
\pgfpathlineto{\pgfqpoint{6.566360in}{2.380000in}}%
\pgfpathlineto{\pgfqpoint{6.567600in}{2.310000in}}%
\pgfpathlineto{\pgfqpoint{6.568840in}{2.485000in}}%
\pgfpathlineto{\pgfqpoint{6.570080in}{2.835000in}}%
\pgfpathlineto{\pgfqpoint{6.571320in}{2.450000in}}%
\pgfpathlineto{\pgfqpoint{6.572560in}{2.520000in}}%
\pgfpathlineto{\pgfqpoint{6.573800in}{2.170000in}}%
\pgfpathlineto{\pgfqpoint{6.576280in}{2.345000in}}%
\pgfpathlineto{\pgfqpoint{6.577520in}{2.135000in}}%
\pgfpathlineto{\pgfqpoint{6.578760in}{2.240000in}}%
\pgfpathlineto{\pgfqpoint{6.580000in}{2.170000in}}%
\pgfpathlineto{\pgfqpoint{6.581240in}{1.995000in}}%
\pgfpathlineto{\pgfqpoint{6.582480in}{2.310000in}}%
\pgfpathlineto{\pgfqpoint{6.583720in}{2.205000in}}%
\pgfpathlineto{\pgfqpoint{6.584960in}{2.380000in}}%
\pgfpathlineto{\pgfqpoint{6.586200in}{2.310000in}}%
\pgfpathlineto{\pgfqpoint{6.588680in}{2.730000in}}%
\pgfpathlineto{\pgfqpoint{6.591160in}{2.380000in}}%
\pgfpathlineto{\pgfqpoint{6.592400in}{2.590000in}}%
\pgfpathlineto{\pgfqpoint{6.594880in}{2.240000in}}%
\pgfpathlineto{\pgfqpoint{6.596120in}{2.555000in}}%
\pgfpathlineto{\pgfqpoint{6.597360in}{2.520000in}}%
\pgfpathlineto{\pgfqpoint{6.598600in}{2.275000in}}%
\pgfpathlineto{\pgfqpoint{6.599840in}{2.310000in}}%
\pgfpathlineto{\pgfqpoint{6.601080in}{2.660000in}}%
\pgfpathlineto{\pgfqpoint{6.602320in}{2.555000in}}%
\pgfpathlineto{\pgfqpoint{6.603560in}{2.555000in}}%
\pgfpathlineto{\pgfqpoint{6.604800in}{2.380000in}}%
\pgfpathlineto{\pgfqpoint{6.606040in}{2.485000in}}%
\pgfpathlineto{\pgfqpoint{6.608520in}{2.135000in}}%
\pgfpathlineto{\pgfqpoint{6.609760in}{2.135000in}}%
\pgfpathlineto{\pgfqpoint{6.611000in}{2.240000in}}%
\pgfpathlineto{\pgfqpoint{6.612240in}{2.625000in}}%
\pgfpathlineto{\pgfqpoint{6.614720in}{2.205000in}}%
\pgfpathlineto{\pgfqpoint{6.615960in}{2.520000in}}%
\pgfpathlineto{\pgfqpoint{6.617200in}{1.995000in}}%
\pgfpathlineto{\pgfqpoint{6.619680in}{2.310000in}}%
\pgfpathlineto{\pgfqpoint{6.620920in}{2.310000in}}%
\pgfpathlineto{\pgfqpoint{6.622160in}{2.485000in}}%
\pgfpathlineto{\pgfqpoint{6.623400in}{2.450000in}}%
\pgfpathlineto{\pgfqpoint{6.624640in}{2.660000in}}%
\pgfpathlineto{\pgfqpoint{6.625880in}{2.310000in}}%
\pgfpathlineto{\pgfqpoint{6.627120in}{2.380000in}}%
\pgfpathlineto{\pgfqpoint{6.628360in}{2.625000in}}%
\pgfpathlineto{\pgfqpoint{6.629600in}{2.205000in}}%
\pgfpathlineto{\pgfqpoint{6.630840in}{2.345000in}}%
\pgfpathlineto{\pgfqpoint{6.632080in}{2.310000in}}%
\pgfpathlineto{\pgfqpoint{6.633320in}{2.520000in}}%
\pgfpathlineto{\pgfqpoint{6.634560in}{2.170000in}}%
\pgfpathlineto{\pgfqpoint{6.635800in}{2.170000in}}%
\pgfpathlineto{\pgfqpoint{6.637040in}{2.520000in}}%
\pgfpathlineto{\pgfqpoint{6.639520in}{2.240000in}}%
\pgfpathlineto{\pgfqpoint{6.640760in}{2.485000in}}%
\pgfpathlineto{\pgfqpoint{6.642000in}{2.275000in}}%
\pgfpathlineto{\pgfqpoint{6.643240in}{2.380000in}}%
\pgfpathlineto{\pgfqpoint{6.644480in}{2.310000in}}%
\pgfpathlineto{\pgfqpoint{6.645720in}{2.660000in}}%
\pgfpathlineto{\pgfqpoint{6.646960in}{2.695000in}}%
\pgfpathlineto{\pgfqpoint{6.649440in}{2.310000in}}%
\pgfpathlineto{\pgfqpoint{6.650680in}{2.205000in}}%
\pgfpathlineto{\pgfqpoint{6.653160in}{2.380000in}}%
\pgfpathlineto{\pgfqpoint{6.655640in}{2.100000in}}%
\pgfpathlineto{\pgfqpoint{6.658120in}{2.555000in}}%
\pgfpathlineto{\pgfqpoint{6.659360in}{2.555000in}}%
\pgfpathlineto{\pgfqpoint{6.660600in}{2.240000in}}%
\pgfpathlineto{\pgfqpoint{6.661840in}{2.660000in}}%
\pgfpathlineto{\pgfqpoint{6.663080in}{2.380000in}}%
\pgfpathlineto{\pgfqpoint{6.664320in}{2.415000in}}%
\pgfpathlineto{\pgfqpoint{6.665560in}{2.660000in}}%
\pgfpathlineto{\pgfqpoint{6.666800in}{2.310000in}}%
\pgfpathlineto{\pgfqpoint{6.669280in}{2.765000in}}%
\pgfpathlineto{\pgfqpoint{6.671760in}{2.170000in}}%
\pgfpathlineto{\pgfqpoint{6.673000in}{2.520000in}}%
\pgfpathlineto{\pgfqpoint{6.674240in}{2.520000in}}%
\pgfpathlineto{\pgfqpoint{6.676720in}{2.135000in}}%
\pgfpathlineto{\pgfqpoint{6.677960in}{2.345000in}}%
\pgfpathlineto{\pgfqpoint{6.680440in}{2.170000in}}%
\pgfpathlineto{\pgfqpoint{6.681680in}{2.660000in}}%
\pgfpathlineto{\pgfqpoint{6.682920in}{2.380000in}}%
\pgfpathlineto{\pgfqpoint{6.684160in}{2.590000in}}%
\pgfpathlineto{\pgfqpoint{6.685400in}{2.590000in}}%
\pgfpathlineto{\pgfqpoint{6.687880in}{2.415000in}}%
\pgfpathlineto{\pgfqpoint{6.689120in}{2.380000in}}%
\pgfpathlineto{\pgfqpoint{6.690360in}{2.450000in}}%
\pgfpathlineto{\pgfqpoint{6.691600in}{2.065000in}}%
\pgfpathlineto{\pgfqpoint{6.694080in}{2.625000in}}%
\pgfpathlineto{\pgfqpoint{6.696560in}{1.960000in}}%
\pgfpathlineto{\pgfqpoint{6.697800in}{2.380000in}}%
\pgfpathlineto{\pgfqpoint{6.699040in}{2.275000in}}%
\pgfpathlineto{\pgfqpoint{6.700280in}{2.310000in}}%
\pgfpathlineto{\pgfqpoint{6.701520in}{2.450000in}}%
\pgfpathlineto{\pgfqpoint{6.702760in}{2.380000in}}%
\pgfpathlineto{\pgfqpoint{6.704000in}{2.135000in}}%
\pgfpathlineto{\pgfqpoint{6.705240in}{2.205000in}}%
\pgfpathlineto{\pgfqpoint{6.707720in}{2.660000in}}%
\pgfpathlineto{\pgfqpoint{6.708960in}{2.450000in}}%
\pgfpathlineto{\pgfqpoint{6.710200in}{2.730000in}}%
\pgfpathlineto{\pgfqpoint{6.711440in}{2.415000in}}%
\pgfpathlineto{\pgfqpoint{6.712680in}{2.555000in}}%
\pgfpathlineto{\pgfqpoint{6.713920in}{2.380000in}}%
\pgfpathlineto{\pgfqpoint{6.715160in}{2.590000in}}%
\pgfpathlineto{\pgfqpoint{6.718880in}{2.345000in}}%
\pgfpathlineto{\pgfqpoint{6.720120in}{2.485000in}}%
\pgfpathlineto{\pgfqpoint{6.721360in}{2.485000in}}%
\pgfpathlineto{\pgfqpoint{6.722600in}{2.555000in}}%
\pgfpathlineto{\pgfqpoint{6.723840in}{2.765000in}}%
\pgfpathlineto{\pgfqpoint{6.725080in}{2.765000in}}%
\pgfpathlineto{\pgfqpoint{6.727560in}{2.345000in}}%
\pgfpathlineto{\pgfqpoint{6.728800in}{2.555000in}}%
\pgfpathlineto{\pgfqpoint{6.731280in}{1.960000in}}%
\pgfpathlineto{\pgfqpoint{6.733760in}{1.610000in}}%
\pgfpathlineto{\pgfqpoint{6.735000in}{2.275000in}}%
\pgfpathlineto{\pgfqpoint{6.736240in}{2.275000in}}%
\pgfpathlineto{\pgfqpoint{6.737480in}{2.065000in}}%
\pgfpathlineto{\pgfqpoint{6.738720in}{2.065000in}}%
\pgfpathlineto{\pgfqpoint{6.739960in}{2.135000in}}%
\pgfpathlineto{\pgfqpoint{6.742440in}{2.555000in}}%
\pgfpathlineto{\pgfqpoint{6.746160in}{2.240000in}}%
\pgfpathlineto{\pgfqpoint{6.747400in}{2.590000in}}%
\pgfpathlineto{\pgfqpoint{6.749880in}{2.660000in}}%
\pgfpathlineto{\pgfqpoint{6.751120in}{2.800000in}}%
\pgfpathlineto{\pgfqpoint{6.753600in}{2.345000in}}%
\pgfpathlineto{\pgfqpoint{6.754840in}{2.905000in}}%
\pgfpathlineto{\pgfqpoint{6.756080in}{2.450000in}}%
\pgfpathlineto{\pgfqpoint{6.757320in}{2.730000in}}%
\pgfpathlineto{\pgfqpoint{6.758560in}{2.660000in}}%
\pgfpathlineto{\pgfqpoint{6.759800in}{2.660000in}}%
\pgfpathlineto{\pgfqpoint{6.761040in}{2.625000in}}%
\pgfpathlineto{\pgfqpoint{6.762280in}{2.625000in}}%
\pgfpathlineto{\pgfqpoint{6.763520in}{2.905000in}}%
\pgfpathlineto{\pgfqpoint{6.764760in}{2.345000in}}%
\pgfpathlineto{\pgfqpoint{6.766000in}{2.555000in}}%
\pgfpathlineto{\pgfqpoint{6.767240in}{2.065000in}}%
\pgfpathlineto{\pgfqpoint{6.769720in}{2.660000in}}%
\pgfpathlineto{\pgfqpoint{6.770960in}{2.800000in}}%
\pgfpathlineto{\pgfqpoint{6.772200in}{2.415000in}}%
\pgfpathlineto{\pgfqpoint{6.773440in}{2.625000in}}%
\pgfpathlineto{\pgfqpoint{6.774680in}{2.240000in}}%
\pgfpathlineto{\pgfqpoint{6.775920in}{2.485000in}}%
\pgfpathlineto{\pgfqpoint{6.778400in}{2.170000in}}%
\pgfpathlineto{\pgfqpoint{6.779640in}{2.695000in}}%
\pgfpathlineto{\pgfqpoint{6.780880in}{2.730000in}}%
\pgfpathlineto{\pgfqpoint{6.782120in}{2.345000in}}%
\pgfpathlineto{\pgfqpoint{6.783360in}{2.590000in}}%
\pgfpathlineto{\pgfqpoint{6.784600in}{2.310000in}}%
\pgfpathlineto{\pgfqpoint{6.785840in}{2.415000in}}%
\pgfpathlineto{\pgfqpoint{6.787080in}{2.415000in}}%
\pgfpathlineto{\pgfqpoint{6.788320in}{2.905000in}}%
\pgfpathlineto{\pgfqpoint{6.792040in}{2.275000in}}%
\pgfpathlineto{\pgfqpoint{6.793280in}{2.870000in}}%
\pgfpathlineto{\pgfqpoint{6.794520in}{2.660000in}}%
\pgfpathlineto{\pgfqpoint{6.795760in}{2.800000in}}%
\pgfpathlineto{\pgfqpoint{6.797000in}{2.695000in}}%
\pgfpathlineto{\pgfqpoint{6.798240in}{2.310000in}}%
\pgfpathlineto{\pgfqpoint{6.799480in}{2.380000in}}%
\pgfpathlineto{\pgfqpoint{6.800720in}{2.765000in}}%
\pgfpathlineto{\pgfqpoint{6.801960in}{2.380000in}}%
\pgfpathlineto{\pgfqpoint{6.803200in}{2.415000in}}%
\pgfpathlineto{\pgfqpoint{6.804440in}{2.730000in}}%
\pgfpathlineto{\pgfqpoint{6.805680in}{2.695000in}}%
\pgfpathlineto{\pgfqpoint{6.806920in}{2.345000in}}%
\pgfpathlineto{\pgfqpoint{6.809400in}{2.940000in}}%
\pgfpathlineto{\pgfqpoint{6.810640in}{2.800000in}}%
\pgfpathlineto{\pgfqpoint{6.813120in}{2.170000in}}%
\pgfpathlineto{\pgfqpoint{6.814360in}{2.485000in}}%
\pgfpathlineto{\pgfqpoint{6.815600in}{2.450000in}}%
\pgfpathlineto{\pgfqpoint{6.816840in}{2.310000in}}%
\pgfpathlineto{\pgfqpoint{6.818080in}{2.450000in}}%
\pgfpathlineto{\pgfqpoint{6.819320in}{2.380000in}}%
\pgfpathlineto{\pgfqpoint{6.820560in}{2.415000in}}%
\pgfpathlineto{\pgfqpoint{6.821800in}{2.485000in}}%
\pgfpathlineto{\pgfqpoint{6.823040in}{2.170000in}}%
\pgfpathlineto{\pgfqpoint{6.824280in}{2.485000in}}%
\pgfpathlineto{\pgfqpoint{6.825520in}{2.100000in}}%
\pgfpathlineto{\pgfqpoint{6.828000in}{2.730000in}}%
\pgfpathlineto{\pgfqpoint{6.829240in}{2.275000in}}%
\pgfpathlineto{\pgfqpoint{6.830480in}{2.450000in}}%
\pgfpathlineto{\pgfqpoint{6.831720in}{2.240000in}}%
\pgfpathlineto{\pgfqpoint{6.832960in}{2.695000in}}%
\pgfpathlineto{\pgfqpoint{6.834200in}{1.925000in}}%
\pgfpathlineto{\pgfqpoint{6.835440in}{2.485000in}}%
\pgfpathlineto{\pgfqpoint{6.836680in}{2.135000in}}%
\pgfpathlineto{\pgfqpoint{6.837920in}{2.485000in}}%
\pgfpathlineto{\pgfqpoint{6.839160in}{2.310000in}}%
\pgfpathlineto{\pgfqpoint{6.840400in}{2.345000in}}%
\pgfpathlineto{\pgfqpoint{6.842880in}{2.835000in}}%
\pgfpathlineto{\pgfqpoint{6.846600in}{2.030000in}}%
\pgfpathlineto{\pgfqpoint{6.850320in}{2.800000in}}%
\pgfpathlineto{\pgfqpoint{6.851560in}{2.275000in}}%
\pgfpathlineto{\pgfqpoint{6.852800in}{2.310000in}}%
\pgfpathlineto{\pgfqpoint{6.854040in}{2.800000in}}%
\pgfpathlineto{\pgfqpoint{6.855280in}{2.310000in}}%
\pgfpathlineto{\pgfqpoint{6.857760in}{2.590000in}}%
\pgfpathlineto{\pgfqpoint{6.859000in}{2.380000in}}%
\pgfpathlineto{\pgfqpoint{6.860240in}{2.555000in}}%
\pgfpathlineto{\pgfqpoint{6.861480in}{2.415000in}}%
\pgfpathlineto{\pgfqpoint{6.862720in}{2.485000in}}%
\pgfpathlineto{\pgfqpoint{6.863960in}{2.205000in}}%
\pgfpathlineto{\pgfqpoint{6.866440in}{2.730000in}}%
\pgfpathlineto{\pgfqpoint{6.867680in}{2.485000in}}%
\pgfpathlineto{\pgfqpoint{6.868920in}{2.520000in}}%
\pgfpathlineto{\pgfqpoint{6.870160in}{2.205000in}}%
\pgfpathlineto{\pgfqpoint{6.871400in}{2.205000in}}%
\pgfpathlineto{\pgfqpoint{6.872640in}{2.415000in}}%
\pgfpathlineto{\pgfqpoint{6.873880in}{2.415000in}}%
\pgfpathlineto{\pgfqpoint{6.875120in}{2.275000in}}%
\pgfpathlineto{\pgfqpoint{6.877600in}{2.275000in}}%
\pgfpathlineto{\pgfqpoint{6.881320in}{2.870000in}}%
\pgfpathlineto{\pgfqpoint{6.885040in}{2.135000in}}%
\pgfpathlineto{\pgfqpoint{6.887520in}{2.590000in}}%
\pgfpathlineto{\pgfqpoint{6.891240in}{2.485000in}}%
\pgfpathlineto{\pgfqpoint{6.892480in}{2.765000in}}%
\pgfpathlineto{\pgfqpoint{6.893720in}{2.520000in}}%
\pgfpathlineto{\pgfqpoint{6.894960in}{1.995000in}}%
\pgfpathlineto{\pgfqpoint{6.897440in}{2.380000in}}%
\pgfpathlineto{\pgfqpoint{6.898680in}{2.135000in}}%
\pgfpathlineto{\pgfqpoint{6.899920in}{2.555000in}}%
\pgfpathlineto{\pgfqpoint{6.901160in}{2.450000in}}%
\pgfpathlineto{\pgfqpoint{6.902400in}{2.730000in}}%
\pgfpathlineto{\pgfqpoint{6.903640in}{2.555000in}}%
\pgfpathlineto{\pgfqpoint{6.904880in}{2.660000in}}%
\pgfpathlineto{\pgfqpoint{6.906120in}{2.625000in}}%
\pgfpathlineto{\pgfqpoint{6.908600in}{2.240000in}}%
\pgfpathlineto{\pgfqpoint{6.909840in}{2.625000in}}%
\pgfpathlineto{\pgfqpoint{6.912320in}{2.520000in}}%
\pgfpathlineto{\pgfqpoint{6.913560in}{2.800000in}}%
\pgfpathlineto{\pgfqpoint{6.914800in}{2.485000in}}%
\pgfpathlineto{\pgfqpoint{6.916040in}{2.625000in}}%
\pgfpathlineto{\pgfqpoint{6.917280in}{2.275000in}}%
\pgfpathlineto{\pgfqpoint{6.918520in}{2.905000in}}%
\pgfpathlineto{\pgfqpoint{6.919760in}{2.205000in}}%
\pgfpathlineto{\pgfqpoint{6.922240in}{2.660000in}}%
\pgfpathlineto{\pgfqpoint{6.923480in}{2.555000in}}%
\pgfpathlineto{\pgfqpoint{6.924720in}{2.940000in}}%
\pgfpathlineto{\pgfqpoint{6.925960in}{2.625000in}}%
\pgfpathlineto{\pgfqpoint{6.928440in}{3.080000in}}%
\pgfpathlineto{\pgfqpoint{6.929680in}{2.520000in}}%
\pgfpathlineto{\pgfqpoint{6.930920in}{2.520000in}}%
\pgfpathlineto{\pgfqpoint{6.932160in}{2.310000in}}%
\pgfpathlineto{\pgfqpoint{6.933400in}{2.590000in}}%
\pgfpathlineto{\pgfqpoint{6.934640in}{2.275000in}}%
\pgfpathlineto{\pgfqpoint{6.935880in}{2.310000in}}%
\pgfpathlineto{\pgfqpoint{6.937120in}{2.205000in}}%
\pgfpathlineto{\pgfqpoint{6.940840in}{2.765000in}}%
\pgfpathlineto{\pgfqpoint{6.943320in}{2.520000in}}%
\pgfpathlineto{\pgfqpoint{6.944560in}{2.520000in}}%
\pgfpathlineto{\pgfqpoint{6.945800in}{2.485000in}}%
\pgfpathlineto{\pgfqpoint{6.947040in}{2.835000in}}%
\pgfpathlineto{\pgfqpoint{6.948280in}{2.415000in}}%
\pgfpathlineto{\pgfqpoint{6.949520in}{2.380000in}}%
\pgfpathlineto{\pgfqpoint{6.950760in}{2.695000in}}%
\pgfpathlineto{\pgfqpoint{6.952000in}{2.590000in}}%
\pgfpathlineto{\pgfqpoint{6.953240in}{2.205000in}}%
\pgfpathlineto{\pgfqpoint{6.954480in}{2.625000in}}%
\pgfpathlineto{\pgfqpoint{6.956960in}{2.415000in}}%
\pgfpathlineto{\pgfqpoint{6.959440in}{2.660000in}}%
\pgfpathlineto{\pgfqpoint{6.961920in}{2.135000in}}%
\pgfpathlineto{\pgfqpoint{6.963160in}{2.415000in}}%
\pgfpathlineto{\pgfqpoint{6.964400in}{2.415000in}}%
\pgfpathlineto{\pgfqpoint{6.965640in}{2.625000in}}%
\pgfpathlineto{\pgfqpoint{6.966880in}{2.415000in}}%
\pgfpathlineto{\pgfqpoint{6.968120in}{2.520000in}}%
\pgfpathlineto{\pgfqpoint{6.969360in}{2.345000in}}%
\pgfpathlineto{\pgfqpoint{6.970600in}{2.660000in}}%
\pgfpathlineto{\pgfqpoint{6.971840in}{2.415000in}}%
\pgfpathlineto{\pgfqpoint{6.973080in}{2.975000in}}%
\pgfpathlineto{\pgfqpoint{6.974320in}{2.555000in}}%
\pgfpathlineto{\pgfqpoint{6.975560in}{2.625000in}}%
\pgfpathlineto{\pgfqpoint{6.976800in}{2.870000in}}%
\pgfpathlineto{\pgfqpoint{6.978040in}{2.660000in}}%
\pgfpathlineto{\pgfqpoint{6.979280in}{2.800000in}}%
\pgfpathlineto{\pgfqpoint{6.981760in}{2.485000in}}%
\pgfpathlineto{\pgfqpoint{6.983000in}{2.485000in}}%
\pgfpathlineto{\pgfqpoint{6.984240in}{2.205000in}}%
\pgfpathlineto{\pgfqpoint{6.986720in}{2.730000in}}%
\pgfpathlineto{\pgfqpoint{6.987960in}{2.345000in}}%
\pgfpathlineto{\pgfqpoint{6.990440in}{2.695000in}}%
\pgfpathlineto{\pgfqpoint{6.992920in}{2.415000in}}%
\pgfpathlineto{\pgfqpoint{6.994160in}{2.380000in}}%
\pgfpathlineto{\pgfqpoint{6.995400in}{2.205000in}}%
\pgfpathlineto{\pgfqpoint{6.999120in}{2.625000in}}%
\pgfpathlineto{\pgfqpoint{7.000360in}{2.625000in}}%
\pgfpathlineto{\pgfqpoint{7.002840in}{2.345000in}}%
\pgfpathlineto{\pgfqpoint{7.004080in}{2.625000in}}%
\pgfpathlineto{\pgfqpoint{7.005320in}{2.240000in}}%
\pgfpathlineto{\pgfqpoint{7.006560in}{2.520000in}}%
\pgfpathlineto{\pgfqpoint{7.007800in}{2.415000in}}%
\pgfpathlineto{\pgfqpoint{7.009040in}{2.625000in}}%
\pgfpathlineto{\pgfqpoint{7.010280in}{2.345000in}}%
\pgfpathlineto{\pgfqpoint{7.011520in}{2.380000in}}%
\pgfpathlineto{\pgfqpoint{7.014000in}{2.555000in}}%
\pgfpathlineto{\pgfqpoint{7.015240in}{2.590000in}}%
\pgfpathlineto{\pgfqpoint{7.016480in}{2.205000in}}%
\pgfpathlineto{\pgfqpoint{7.017720in}{2.520000in}}%
\pgfpathlineto{\pgfqpoint{7.018960in}{2.450000in}}%
\pgfpathlineto{\pgfqpoint{7.020200in}{2.240000in}}%
\pgfpathlineto{\pgfqpoint{7.022680in}{2.450000in}}%
\pgfpathlineto{\pgfqpoint{7.023920in}{2.555000in}}%
\pgfpathlineto{\pgfqpoint{7.025160in}{2.415000in}}%
\pgfpathlineto{\pgfqpoint{7.027640in}{2.835000in}}%
\pgfpathlineto{\pgfqpoint{7.030120in}{2.695000in}}%
\pgfpathlineto{\pgfqpoint{7.031360in}{2.695000in}}%
\pgfpathlineto{\pgfqpoint{7.032600in}{2.520000in}}%
\pgfpathlineto{\pgfqpoint{7.033840in}{2.765000in}}%
\pgfpathlineto{\pgfqpoint{7.035080in}{2.695000in}}%
\pgfpathlineto{\pgfqpoint{7.036320in}{2.940000in}}%
\pgfpathlineto{\pgfqpoint{7.038800in}{2.310000in}}%
\pgfpathlineto{\pgfqpoint{7.041280in}{2.625000in}}%
\pgfpathlineto{\pgfqpoint{7.042520in}{2.590000in}}%
\pgfpathlineto{\pgfqpoint{7.043760in}{2.275000in}}%
\pgfpathlineto{\pgfqpoint{7.045000in}{2.660000in}}%
\pgfpathlineto{\pgfqpoint{7.046240in}{2.520000in}}%
\pgfpathlineto{\pgfqpoint{7.047480in}{2.625000in}}%
\pgfpathlineto{\pgfqpoint{7.049960in}{2.275000in}}%
\pgfpathlineto{\pgfqpoint{7.051200in}{2.520000in}}%
\pgfpathlineto{\pgfqpoint{7.052440in}{2.485000in}}%
\pgfpathlineto{\pgfqpoint{7.053680in}{2.415000in}}%
\pgfpathlineto{\pgfqpoint{7.054920in}{2.590000in}}%
\pgfpathlineto{\pgfqpoint{7.056160in}{2.380000in}}%
\pgfpathlineto{\pgfqpoint{7.057400in}{2.520000in}}%
\pgfpathlineto{\pgfqpoint{7.058640in}{2.450000in}}%
\pgfpathlineto{\pgfqpoint{7.059880in}{2.310000in}}%
\pgfpathlineto{\pgfqpoint{7.061120in}{2.310000in}}%
\pgfpathlineto{\pgfqpoint{7.062360in}{2.205000in}}%
\pgfpathlineto{\pgfqpoint{7.063600in}{2.380000in}}%
\pgfpathlineto{\pgfqpoint{7.064840in}{2.240000in}}%
\pgfpathlineto{\pgfqpoint{7.066080in}{2.485000in}}%
\pgfpathlineto{\pgfqpoint{7.067320in}{1.995000in}}%
\pgfpathlineto{\pgfqpoint{7.069800in}{2.345000in}}%
\pgfpathlineto{\pgfqpoint{7.071040in}{2.450000in}}%
\pgfpathlineto{\pgfqpoint{7.073520in}{2.030000in}}%
\pgfpathlineto{\pgfqpoint{7.074760in}{1.995000in}}%
\pgfpathlineto{\pgfqpoint{7.076000in}{2.520000in}}%
\pgfpathlineto{\pgfqpoint{7.077240in}{2.520000in}}%
\pgfpathlineto{\pgfqpoint{7.078480in}{2.835000in}}%
\pgfpathlineto{\pgfqpoint{7.079720in}{2.380000in}}%
\pgfpathlineto{\pgfqpoint{7.080960in}{2.345000in}}%
\pgfpathlineto{\pgfqpoint{7.082200in}{2.765000in}}%
\pgfpathlineto{\pgfqpoint{7.083440in}{2.450000in}}%
\pgfpathlineto{\pgfqpoint{7.084680in}{2.660000in}}%
\pgfpathlineto{\pgfqpoint{7.085920in}{1.925000in}}%
\pgfpathlineto{\pgfqpoint{7.087160in}{2.695000in}}%
\pgfpathlineto{\pgfqpoint{7.088400in}{2.275000in}}%
\pgfpathlineto{\pgfqpoint{7.089640in}{2.485000in}}%
\pgfpathlineto{\pgfqpoint{7.090880in}{2.240000in}}%
\pgfpathlineto{\pgfqpoint{7.092120in}{2.590000in}}%
\pgfpathlineto{\pgfqpoint{7.093360in}{2.555000in}}%
\pgfpathlineto{\pgfqpoint{7.094600in}{2.240000in}}%
\pgfpathlineto{\pgfqpoint{7.097080in}{2.415000in}}%
\pgfpathlineto{\pgfqpoint{7.099560in}{2.170000in}}%
\pgfpathlineto{\pgfqpoint{7.100800in}{2.240000in}}%
\pgfpathlineto{\pgfqpoint{7.102040in}{2.030000in}}%
\pgfpathlineto{\pgfqpoint{7.103280in}{2.380000in}}%
\pgfpathlineto{\pgfqpoint{7.104520in}{2.345000in}}%
\pgfpathlineto{\pgfqpoint{7.107000in}{2.065000in}}%
\pgfpathlineto{\pgfqpoint{7.109480in}{2.590000in}}%
\pgfpathlineto{\pgfqpoint{7.110720in}{2.310000in}}%
\pgfpathlineto{\pgfqpoint{7.111960in}{2.380000in}}%
\pgfpathlineto{\pgfqpoint{7.113200in}{2.100000in}}%
\pgfpathlineto{\pgfqpoint{7.114440in}{2.800000in}}%
\pgfpathlineto{\pgfqpoint{7.116920in}{2.380000in}}%
\pgfpathlineto{\pgfqpoint{7.118160in}{2.730000in}}%
\pgfpathlineto{\pgfqpoint{7.119400in}{2.240000in}}%
\pgfpathlineto{\pgfqpoint{7.120640in}{2.450000in}}%
\pgfpathlineto{\pgfqpoint{7.121880in}{2.310000in}}%
\pgfpathlineto{\pgfqpoint{7.124360in}{2.835000in}}%
\pgfpathlineto{\pgfqpoint{7.125600in}{2.380000in}}%
\pgfpathlineto{\pgfqpoint{7.126840in}{2.765000in}}%
\pgfpathlineto{\pgfqpoint{7.128080in}{2.660000in}}%
\pgfpathlineto{\pgfqpoint{7.129320in}{2.380000in}}%
\pgfpathlineto{\pgfqpoint{7.131800in}{2.450000in}}%
\pgfpathlineto{\pgfqpoint{7.134280in}{2.275000in}}%
\pgfpathlineto{\pgfqpoint{7.135520in}{2.485000in}}%
\pgfpathlineto{\pgfqpoint{7.136760in}{2.275000in}}%
\pgfpathlineto{\pgfqpoint{7.138000in}{2.380000in}}%
\pgfpathlineto{\pgfqpoint{7.139240in}{2.625000in}}%
\pgfpathlineto{\pgfqpoint{7.140480in}{2.380000in}}%
\pgfpathlineto{\pgfqpoint{7.141720in}{2.415000in}}%
\pgfpathlineto{\pgfqpoint{7.142960in}{1.995000in}}%
\pgfpathlineto{\pgfqpoint{7.146680in}{2.765000in}}%
\pgfpathlineto{\pgfqpoint{7.149160in}{2.555000in}}%
\pgfpathlineto{\pgfqpoint{7.150400in}{2.660000in}}%
\pgfpathlineto{\pgfqpoint{7.152880in}{2.170000in}}%
\pgfpathlineto{\pgfqpoint{7.155360in}{2.590000in}}%
\pgfpathlineto{\pgfqpoint{7.156600in}{2.275000in}}%
\pgfpathlineto{\pgfqpoint{7.160320in}{2.660000in}}%
\pgfpathlineto{\pgfqpoint{7.161560in}{2.485000in}}%
\pgfpathlineto{\pgfqpoint{7.164040in}{2.730000in}}%
\pgfpathlineto{\pgfqpoint{7.167760in}{2.135000in}}%
\pgfpathlineto{\pgfqpoint{7.170240in}{2.450000in}}%
\pgfpathlineto{\pgfqpoint{7.171480in}{2.275000in}}%
\pgfpathlineto{\pgfqpoint{7.172720in}{2.380000in}}%
\pgfpathlineto{\pgfqpoint{7.173960in}{1.890000in}}%
\pgfpathlineto{\pgfqpoint{7.176440in}{2.485000in}}%
\pgfpathlineto{\pgfqpoint{7.177680in}{2.450000in}}%
\pgfpathlineto{\pgfqpoint{7.178920in}{2.170000in}}%
\pgfpathlineto{\pgfqpoint{7.181400in}{2.380000in}}%
\pgfpathlineto{\pgfqpoint{7.182640in}{2.695000in}}%
\pgfpathlineto{\pgfqpoint{7.183880in}{2.520000in}}%
\pgfpathlineto{\pgfqpoint{7.185120in}{2.625000in}}%
\pgfpathlineto{\pgfqpoint{7.186360in}{2.590000in}}%
\pgfpathlineto{\pgfqpoint{7.187600in}{2.135000in}}%
\pgfpathlineto{\pgfqpoint{7.190080in}{2.485000in}}%
\pgfpathlineto{\pgfqpoint{7.191320in}{2.625000in}}%
\pgfpathlineto{\pgfqpoint{7.192560in}{2.240000in}}%
\pgfpathlineto{\pgfqpoint{7.193800in}{2.275000in}}%
\pgfpathlineto{\pgfqpoint{7.195040in}{1.960000in}}%
\pgfpathlineto{\pgfqpoint{7.196280in}{2.415000in}}%
\pgfpathlineto{\pgfqpoint{7.198760in}{2.415000in}}%
\pgfpathlineto{\pgfqpoint{7.200000in}{2.240000in}}%
\pgfpathlineto{\pgfqpoint{7.200000in}{2.240000in}}%
\pgfusepath{stroke}%
\end{pgfscope}%
\begin{pgfscope}%
\pgfpathrectangle{\pgfqpoint{1.000000in}{0.350000in}}{\pgfqpoint{6.200000in}{2.800000in}} %
\pgfusepath{clip}%
\pgfsetrectcap%
\pgfsetroundjoin%
\pgfsetlinewidth{1.003750pt}%
\definecolor{currentstroke}{rgb}{1.000000,0.000000,0.000000}%
\pgfsetstrokecolor{currentstroke}%
\pgfsetdash{}{0pt}%
\pgfpathmoveto{\pgfqpoint{1.000000in}{2.555000in}}%
\pgfpathlineto{\pgfqpoint{1.001240in}{2.065000in}}%
\pgfpathlineto{\pgfqpoint{1.003720in}{2.135000in}}%
\pgfpathlineto{\pgfqpoint{1.006200in}{1.645000in}}%
\pgfpathlineto{\pgfqpoint{1.007440in}{1.750000in}}%
\pgfpathlineto{\pgfqpoint{1.009920in}{1.400000in}}%
\pgfpathlineto{\pgfqpoint{1.013640in}{1.715000in}}%
\pgfpathlineto{\pgfqpoint{1.014880in}{1.645000in}}%
\pgfpathlineto{\pgfqpoint{1.016120in}{1.855000in}}%
\pgfpathlineto{\pgfqpoint{1.019840in}{1.120000in}}%
\pgfpathlineto{\pgfqpoint{1.021080in}{1.540000in}}%
\pgfpathlineto{\pgfqpoint{1.022320in}{1.400000in}}%
\pgfpathlineto{\pgfqpoint{1.023560in}{1.400000in}}%
\pgfpathlineto{\pgfqpoint{1.024800in}{1.190000in}}%
\pgfpathlineto{\pgfqpoint{1.026040in}{1.295000in}}%
\pgfpathlineto{\pgfqpoint{1.027280in}{1.260000in}}%
\pgfpathlineto{\pgfqpoint{1.028520in}{1.365000in}}%
\pgfpathlineto{\pgfqpoint{1.029760in}{1.295000in}}%
\pgfpathlineto{\pgfqpoint{1.031000in}{1.155000in}}%
\pgfpathlineto{\pgfqpoint{1.032240in}{1.715000in}}%
\pgfpathlineto{\pgfqpoint{1.033480in}{1.715000in}}%
\pgfpathlineto{\pgfqpoint{1.034720in}{1.330000in}}%
\pgfpathlineto{\pgfqpoint{1.037200in}{1.540000in}}%
\pgfpathlineto{\pgfqpoint{1.038440in}{1.505000in}}%
\pgfpathlineto{\pgfqpoint{1.039680in}{1.680000in}}%
\pgfpathlineto{\pgfqpoint{1.040920in}{1.435000in}}%
\pgfpathlineto{\pgfqpoint{1.042160in}{1.645000in}}%
\pgfpathlineto{\pgfqpoint{1.043400in}{1.645000in}}%
\pgfpathlineto{\pgfqpoint{1.044640in}{1.505000in}}%
\pgfpathlineto{\pgfqpoint{1.045880in}{1.715000in}}%
\pgfpathlineto{\pgfqpoint{1.047120in}{1.400000in}}%
\pgfpathlineto{\pgfqpoint{1.048360in}{1.680000in}}%
\pgfpathlineto{\pgfqpoint{1.049600in}{1.645000in}}%
\pgfpathlineto{\pgfqpoint{1.052080in}{1.050000in}}%
\pgfpathlineto{\pgfqpoint{1.054560in}{1.610000in}}%
\pgfpathlineto{\pgfqpoint{1.057040in}{0.945000in}}%
\pgfpathlineto{\pgfqpoint{1.058280in}{1.540000in}}%
\pgfpathlineto{\pgfqpoint{1.059520in}{1.400000in}}%
\pgfpathlineto{\pgfqpoint{1.060760in}{2.065000in}}%
\pgfpathlineto{\pgfqpoint{1.062000in}{1.365000in}}%
\pgfpathlineto{\pgfqpoint{1.063240in}{1.575000in}}%
\pgfpathlineto{\pgfqpoint{1.064480in}{1.190000in}}%
\pgfpathlineto{\pgfqpoint{1.065720in}{1.575000in}}%
\pgfpathlineto{\pgfqpoint{1.066960in}{1.085000in}}%
\pgfpathlineto{\pgfqpoint{1.070680in}{1.750000in}}%
\pgfpathlineto{\pgfqpoint{1.071920in}{1.540000in}}%
\pgfpathlineto{\pgfqpoint{1.073160in}{1.610000in}}%
\pgfpathlineto{\pgfqpoint{1.074400in}{1.190000in}}%
\pgfpathlineto{\pgfqpoint{1.075640in}{1.890000in}}%
\pgfpathlineto{\pgfqpoint{1.076880in}{1.295000in}}%
\pgfpathlineto{\pgfqpoint{1.078120in}{1.365000in}}%
\pgfpathlineto{\pgfqpoint{1.079360in}{1.155000in}}%
\pgfpathlineto{\pgfqpoint{1.080600in}{1.295000in}}%
\pgfpathlineto{\pgfqpoint{1.081840in}{1.890000in}}%
\pgfpathlineto{\pgfqpoint{1.083080in}{1.435000in}}%
\pgfpathlineto{\pgfqpoint{1.084320in}{1.400000in}}%
\pgfpathlineto{\pgfqpoint{1.085560in}{1.400000in}}%
\pgfpathlineto{\pgfqpoint{1.086800in}{1.435000in}}%
\pgfpathlineto{\pgfqpoint{1.088040in}{1.750000in}}%
\pgfpathlineto{\pgfqpoint{1.089280in}{1.400000in}}%
\pgfpathlineto{\pgfqpoint{1.091760in}{1.715000in}}%
\pgfpathlineto{\pgfqpoint{1.094240in}{0.805000in}}%
\pgfpathlineto{\pgfqpoint{1.096720in}{1.785000in}}%
\pgfpathlineto{\pgfqpoint{1.097960in}{0.980000in}}%
\pgfpathlineto{\pgfqpoint{1.100440in}{1.400000in}}%
\pgfpathlineto{\pgfqpoint{1.101680in}{0.560000in}}%
\pgfpathlineto{\pgfqpoint{1.102920in}{1.470000in}}%
\pgfpathlineto{\pgfqpoint{1.105400in}{1.260000in}}%
\pgfpathlineto{\pgfqpoint{1.106640in}{1.435000in}}%
\pgfpathlineto{\pgfqpoint{1.107880in}{1.050000in}}%
\pgfpathlineto{\pgfqpoint{1.109120in}{1.365000in}}%
\pgfpathlineto{\pgfqpoint{1.110360in}{1.365000in}}%
\pgfpathlineto{\pgfqpoint{1.111600in}{1.050000in}}%
\pgfpathlineto{\pgfqpoint{1.112840in}{1.540000in}}%
\pgfpathlineto{\pgfqpoint{1.114080in}{1.505000in}}%
\pgfpathlineto{\pgfqpoint{1.115320in}{1.820000in}}%
\pgfpathlineto{\pgfqpoint{1.119040in}{1.050000in}}%
\pgfpathlineto{\pgfqpoint{1.120280in}{1.260000in}}%
\pgfpathlineto{\pgfqpoint{1.121520in}{1.120000in}}%
\pgfpathlineto{\pgfqpoint{1.122760in}{1.225000in}}%
\pgfpathlineto{\pgfqpoint{1.125240in}{0.840000in}}%
\pgfpathlineto{\pgfqpoint{1.127720in}{1.295000in}}%
\pgfpathlineto{\pgfqpoint{1.128960in}{1.575000in}}%
\pgfpathlineto{\pgfqpoint{1.131440in}{0.945000in}}%
\pgfpathlineto{\pgfqpoint{1.132680in}{1.540000in}}%
\pgfpathlineto{\pgfqpoint{1.133920in}{1.295000in}}%
\pgfpathlineto{\pgfqpoint{1.135160in}{1.645000in}}%
\pgfpathlineto{\pgfqpoint{1.137640in}{1.435000in}}%
\pgfpathlineto{\pgfqpoint{1.138880in}{1.400000in}}%
\pgfpathlineto{\pgfqpoint{1.140120in}{1.610000in}}%
\pgfpathlineto{\pgfqpoint{1.141360in}{1.470000in}}%
\pgfpathlineto{\pgfqpoint{1.142600in}{1.680000in}}%
\pgfpathlineto{\pgfqpoint{1.143840in}{1.190000in}}%
\pgfpathlineto{\pgfqpoint{1.146320in}{1.365000in}}%
\pgfpathlineto{\pgfqpoint{1.147560in}{1.155000in}}%
\pgfpathlineto{\pgfqpoint{1.148800in}{1.785000in}}%
\pgfpathlineto{\pgfqpoint{1.150040in}{1.225000in}}%
\pgfpathlineto{\pgfqpoint{1.151280in}{1.470000in}}%
\pgfpathlineto{\pgfqpoint{1.153760in}{1.050000in}}%
\pgfpathlineto{\pgfqpoint{1.155000in}{1.400000in}}%
\pgfpathlineto{\pgfqpoint{1.156240in}{1.400000in}}%
\pgfpathlineto{\pgfqpoint{1.157480in}{1.365000in}}%
\pgfpathlineto{\pgfqpoint{1.158720in}{1.050000in}}%
\pgfpathlineto{\pgfqpoint{1.159960in}{1.365000in}}%
\pgfpathlineto{\pgfqpoint{1.162440in}{1.365000in}}%
\pgfpathlineto{\pgfqpoint{1.163680in}{1.820000in}}%
\pgfpathlineto{\pgfqpoint{1.164920in}{1.540000in}}%
\pgfpathlineto{\pgfqpoint{1.166160in}{2.170000in}}%
\pgfpathlineto{\pgfqpoint{1.167400in}{0.945000in}}%
\pgfpathlineto{\pgfqpoint{1.168640in}{1.750000in}}%
\pgfpathlineto{\pgfqpoint{1.169880in}{1.400000in}}%
\pgfpathlineto{\pgfqpoint{1.171120in}{1.470000in}}%
\pgfpathlineto{\pgfqpoint{1.173600in}{1.785000in}}%
\pgfpathlineto{\pgfqpoint{1.176080in}{1.330000in}}%
\pgfpathlineto{\pgfqpoint{1.177320in}{1.505000in}}%
\pgfpathlineto{\pgfqpoint{1.178560in}{0.805000in}}%
\pgfpathlineto{\pgfqpoint{1.179800in}{1.120000in}}%
\pgfpathlineto{\pgfqpoint{1.181040in}{1.050000in}}%
\pgfpathlineto{\pgfqpoint{1.182280in}{1.470000in}}%
\pgfpathlineto{\pgfqpoint{1.183520in}{1.470000in}}%
\pgfpathlineto{\pgfqpoint{1.184760in}{1.890000in}}%
\pgfpathlineto{\pgfqpoint{1.187240in}{1.295000in}}%
\pgfpathlineto{\pgfqpoint{1.188480in}{1.435000in}}%
\pgfpathlineto{\pgfqpoint{1.189720in}{1.260000in}}%
\pgfpathlineto{\pgfqpoint{1.192200in}{1.435000in}}%
\pgfpathlineto{\pgfqpoint{1.193440in}{1.225000in}}%
\pgfpathlineto{\pgfqpoint{1.194680in}{2.030000in}}%
\pgfpathlineto{\pgfqpoint{1.197160in}{1.470000in}}%
\pgfpathlineto{\pgfqpoint{1.198400in}{1.365000in}}%
\pgfpathlineto{\pgfqpoint{1.199640in}{1.155000in}}%
\pgfpathlineto{\pgfqpoint{1.200880in}{1.225000in}}%
\pgfpathlineto{\pgfqpoint{1.202120in}{1.610000in}}%
\pgfpathlineto{\pgfqpoint{1.204600in}{1.155000in}}%
\pgfpathlineto{\pgfqpoint{1.207080in}{1.505000in}}%
\pgfpathlineto{\pgfqpoint{1.209560in}{1.785000in}}%
\pgfpathlineto{\pgfqpoint{1.210800in}{1.435000in}}%
\pgfpathlineto{\pgfqpoint{1.213280in}{1.925000in}}%
\pgfpathlineto{\pgfqpoint{1.214520in}{1.400000in}}%
\pgfpathlineto{\pgfqpoint{1.215760in}{1.890000in}}%
\pgfpathlineto{\pgfqpoint{1.217000in}{1.295000in}}%
\pgfpathlineto{\pgfqpoint{1.219480in}{2.030000in}}%
\pgfpathlineto{\pgfqpoint{1.220720in}{1.365000in}}%
\pgfpathlineto{\pgfqpoint{1.221960in}{1.715000in}}%
\pgfpathlineto{\pgfqpoint{1.224440in}{1.505000in}}%
\pgfpathlineto{\pgfqpoint{1.225680in}{1.645000in}}%
\pgfpathlineto{\pgfqpoint{1.226920in}{1.400000in}}%
\pgfpathlineto{\pgfqpoint{1.228160in}{1.750000in}}%
\pgfpathlineto{\pgfqpoint{1.229400in}{1.610000in}}%
\pgfpathlineto{\pgfqpoint{1.230640in}{1.785000in}}%
\pgfpathlineto{\pgfqpoint{1.231880in}{1.750000in}}%
\pgfpathlineto{\pgfqpoint{1.233120in}{1.680000in}}%
\pgfpathlineto{\pgfqpoint{1.234360in}{1.365000in}}%
\pgfpathlineto{\pgfqpoint{1.235600in}{1.750000in}}%
\pgfpathlineto{\pgfqpoint{1.236840in}{1.155000in}}%
\pgfpathlineto{\pgfqpoint{1.239320in}{1.435000in}}%
\pgfpathlineto{\pgfqpoint{1.240560in}{1.225000in}}%
\pgfpathlineto{\pgfqpoint{1.241800in}{1.225000in}}%
\pgfpathlineto{\pgfqpoint{1.243040in}{1.155000in}}%
\pgfpathlineto{\pgfqpoint{1.244280in}{1.260000in}}%
\pgfpathlineto{\pgfqpoint{1.245520in}{1.505000in}}%
\pgfpathlineto{\pgfqpoint{1.246760in}{1.470000in}}%
\pgfpathlineto{\pgfqpoint{1.248000in}{1.750000in}}%
\pgfpathlineto{\pgfqpoint{1.249240in}{1.260000in}}%
\pgfpathlineto{\pgfqpoint{1.250480in}{1.505000in}}%
\pgfpathlineto{\pgfqpoint{1.251720in}{1.400000in}}%
\pgfpathlineto{\pgfqpoint{1.252960in}{1.400000in}}%
\pgfpathlineto{\pgfqpoint{1.254200in}{1.680000in}}%
\pgfpathlineto{\pgfqpoint{1.256680in}{1.435000in}}%
\pgfpathlineto{\pgfqpoint{1.257920in}{1.575000in}}%
\pgfpathlineto{\pgfqpoint{1.259160in}{1.400000in}}%
\pgfpathlineto{\pgfqpoint{1.260400in}{1.540000in}}%
\pgfpathlineto{\pgfqpoint{1.261640in}{2.065000in}}%
\pgfpathlineto{\pgfqpoint{1.262880in}{1.365000in}}%
\pgfpathlineto{\pgfqpoint{1.264120in}{1.330000in}}%
\pgfpathlineto{\pgfqpoint{1.265360in}{1.680000in}}%
\pgfpathlineto{\pgfqpoint{1.266600in}{1.435000in}}%
\pgfpathlineto{\pgfqpoint{1.267840in}{1.540000in}}%
\pgfpathlineto{\pgfqpoint{1.269080in}{1.260000in}}%
\pgfpathlineto{\pgfqpoint{1.270320in}{1.435000in}}%
\pgfpathlineto{\pgfqpoint{1.272800in}{0.875000in}}%
\pgfpathlineto{\pgfqpoint{1.275280in}{1.470000in}}%
\pgfpathlineto{\pgfqpoint{1.276520in}{0.840000in}}%
\pgfpathlineto{\pgfqpoint{1.277760in}{1.540000in}}%
\pgfpathlineto{\pgfqpoint{1.279000in}{1.260000in}}%
\pgfpathlineto{\pgfqpoint{1.281480in}{1.540000in}}%
\pgfpathlineto{\pgfqpoint{1.283960in}{1.225000in}}%
\pgfpathlineto{\pgfqpoint{1.285200in}{1.575000in}}%
\pgfpathlineto{\pgfqpoint{1.286440in}{1.470000in}}%
\pgfpathlineto{\pgfqpoint{1.287680in}{1.680000in}}%
\pgfpathlineto{\pgfqpoint{1.288920in}{1.470000in}}%
\pgfpathlineto{\pgfqpoint{1.290160in}{1.785000in}}%
\pgfpathlineto{\pgfqpoint{1.291400in}{1.295000in}}%
\pgfpathlineto{\pgfqpoint{1.292640in}{1.680000in}}%
\pgfpathlineto{\pgfqpoint{1.297600in}{1.050000in}}%
\pgfpathlineto{\pgfqpoint{1.298840in}{1.330000in}}%
\pgfpathlineto{\pgfqpoint{1.300080in}{1.015000in}}%
\pgfpathlineto{\pgfqpoint{1.302560in}{1.260000in}}%
\pgfpathlineto{\pgfqpoint{1.305040in}{1.680000in}}%
\pgfpathlineto{\pgfqpoint{1.306280in}{1.400000in}}%
\pgfpathlineto{\pgfqpoint{1.307520in}{1.680000in}}%
\pgfpathlineto{\pgfqpoint{1.308760in}{1.085000in}}%
\pgfpathlineto{\pgfqpoint{1.311240in}{1.575000in}}%
\pgfpathlineto{\pgfqpoint{1.312480in}{1.295000in}}%
\pgfpathlineto{\pgfqpoint{1.313720in}{1.400000in}}%
\pgfpathlineto{\pgfqpoint{1.314960in}{1.400000in}}%
\pgfpathlineto{\pgfqpoint{1.316200in}{1.435000in}}%
\pgfpathlineto{\pgfqpoint{1.317440in}{1.050000in}}%
\pgfpathlineto{\pgfqpoint{1.318680in}{1.295000in}}%
\pgfpathlineto{\pgfqpoint{1.319920in}{0.980000in}}%
\pgfpathlineto{\pgfqpoint{1.321160in}{1.190000in}}%
\pgfpathlineto{\pgfqpoint{1.322400in}{1.015000in}}%
\pgfpathlineto{\pgfqpoint{1.323640in}{1.190000in}}%
\pgfpathlineto{\pgfqpoint{1.324880in}{1.645000in}}%
\pgfpathlineto{\pgfqpoint{1.327360in}{1.190000in}}%
\pgfpathlineto{\pgfqpoint{1.329840in}{1.995000in}}%
\pgfpathlineto{\pgfqpoint{1.331080in}{1.750000in}}%
\pgfpathlineto{\pgfqpoint{1.332320in}{1.190000in}}%
\pgfpathlineto{\pgfqpoint{1.333560in}{1.225000in}}%
\pgfpathlineto{\pgfqpoint{1.334800in}{1.855000in}}%
\pgfpathlineto{\pgfqpoint{1.336040in}{1.645000in}}%
\pgfpathlineto{\pgfqpoint{1.337280in}{1.715000in}}%
\pgfpathlineto{\pgfqpoint{1.339760in}{1.400000in}}%
\pgfpathlineto{\pgfqpoint{1.341000in}{1.715000in}}%
\pgfpathlineto{\pgfqpoint{1.343480in}{1.400000in}}%
\pgfpathlineto{\pgfqpoint{1.344720in}{1.400000in}}%
\pgfpathlineto{\pgfqpoint{1.345960in}{1.435000in}}%
\pgfpathlineto{\pgfqpoint{1.347200in}{1.575000in}}%
\pgfpathlineto{\pgfqpoint{1.349680in}{1.225000in}}%
\pgfpathlineto{\pgfqpoint{1.350920in}{1.610000in}}%
\pgfpathlineto{\pgfqpoint{1.352160in}{1.540000in}}%
\pgfpathlineto{\pgfqpoint{1.354640in}{2.100000in}}%
\pgfpathlineto{\pgfqpoint{1.357120in}{1.365000in}}%
\pgfpathlineto{\pgfqpoint{1.358360in}{1.960000in}}%
\pgfpathlineto{\pgfqpoint{1.359600in}{1.645000in}}%
\pgfpathlineto{\pgfqpoint{1.360840in}{1.925000in}}%
\pgfpathlineto{\pgfqpoint{1.362080in}{1.890000in}}%
\pgfpathlineto{\pgfqpoint{1.363320in}{0.910000in}}%
\pgfpathlineto{\pgfqpoint{1.364560in}{1.680000in}}%
\pgfpathlineto{\pgfqpoint{1.368280in}{1.260000in}}%
\pgfpathlineto{\pgfqpoint{1.369520in}{1.260000in}}%
\pgfpathlineto{\pgfqpoint{1.372000in}{1.680000in}}%
\pgfpathlineto{\pgfqpoint{1.373240in}{1.330000in}}%
\pgfpathlineto{\pgfqpoint{1.376960in}{1.890000in}}%
\pgfpathlineto{\pgfqpoint{1.379440in}{1.295000in}}%
\pgfpathlineto{\pgfqpoint{1.380680in}{1.190000in}}%
\pgfpathlineto{\pgfqpoint{1.381920in}{1.610000in}}%
\pgfpathlineto{\pgfqpoint{1.383160in}{0.805000in}}%
\pgfpathlineto{\pgfqpoint{1.385640in}{1.645000in}}%
\pgfpathlineto{\pgfqpoint{1.386880in}{1.540000in}}%
\pgfpathlineto{\pgfqpoint{1.389360in}{1.050000in}}%
\pgfpathlineto{\pgfqpoint{1.390600in}{1.330000in}}%
\pgfpathlineto{\pgfqpoint{1.391840in}{1.330000in}}%
\pgfpathlineto{\pgfqpoint{1.394320in}{1.435000in}}%
\pgfpathlineto{\pgfqpoint{1.395560in}{1.260000in}}%
\pgfpathlineto{\pgfqpoint{1.396800in}{1.785000in}}%
\pgfpathlineto{\pgfqpoint{1.399280in}{1.295000in}}%
\pgfpathlineto{\pgfqpoint{1.400520in}{1.050000in}}%
\pgfpathlineto{\pgfqpoint{1.403000in}{1.505000in}}%
\pgfpathlineto{\pgfqpoint{1.404240in}{1.050000in}}%
\pgfpathlineto{\pgfqpoint{1.406720in}{1.505000in}}%
\pgfpathlineto{\pgfqpoint{1.407960in}{0.980000in}}%
\pgfpathlineto{\pgfqpoint{1.410440in}{1.505000in}}%
\pgfpathlineto{\pgfqpoint{1.412920in}{1.365000in}}%
\pgfpathlineto{\pgfqpoint{1.415400in}{1.995000in}}%
\pgfpathlineto{\pgfqpoint{1.416640in}{1.470000in}}%
\pgfpathlineto{\pgfqpoint{1.417880in}{1.610000in}}%
\pgfpathlineto{\pgfqpoint{1.419120in}{1.190000in}}%
\pgfpathlineto{\pgfqpoint{1.420360in}{1.575000in}}%
\pgfpathlineto{\pgfqpoint{1.421600in}{1.610000in}}%
\pgfpathlineto{\pgfqpoint{1.422840in}{1.295000in}}%
\pgfpathlineto{\pgfqpoint{1.424080in}{1.645000in}}%
\pgfpathlineto{\pgfqpoint{1.425320in}{1.365000in}}%
\pgfpathlineto{\pgfqpoint{1.426560in}{1.400000in}}%
\pgfpathlineto{\pgfqpoint{1.427800in}{1.470000in}}%
\pgfpathlineto{\pgfqpoint{1.429040in}{1.435000in}}%
\pgfpathlineto{\pgfqpoint{1.430280in}{1.540000in}}%
\pgfpathlineto{\pgfqpoint{1.431520in}{1.505000in}}%
\pgfpathlineto{\pgfqpoint{1.432760in}{1.435000in}}%
\pgfpathlineto{\pgfqpoint{1.435240in}{1.750000in}}%
\pgfpathlineto{\pgfqpoint{1.437720in}{1.365000in}}%
\pgfpathlineto{\pgfqpoint{1.438960in}{1.785000in}}%
\pgfpathlineto{\pgfqpoint{1.440200in}{1.015000in}}%
\pgfpathlineto{\pgfqpoint{1.441440in}{1.540000in}}%
\pgfpathlineto{\pgfqpoint{1.442680in}{1.435000in}}%
\pgfpathlineto{\pgfqpoint{1.445160in}{1.015000in}}%
\pgfpathlineto{\pgfqpoint{1.447640in}{1.715000in}}%
\pgfpathlineto{\pgfqpoint{1.448880in}{1.190000in}}%
\pgfpathlineto{\pgfqpoint{1.450120in}{1.330000in}}%
\pgfpathlineto{\pgfqpoint{1.451360in}{1.330000in}}%
\pgfpathlineto{\pgfqpoint{1.455080in}{1.050000in}}%
\pgfpathlineto{\pgfqpoint{1.456320in}{1.540000in}}%
\pgfpathlineto{\pgfqpoint{1.457560in}{1.225000in}}%
\pgfpathlineto{\pgfqpoint{1.458800in}{1.750000in}}%
\pgfpathlineto{\pgfqpoint{1.461280in}{1.365000in}}%
\pgfpathlineto{\pgfqpoint{1.462520in}{1.295000in}}%
\pgfpathlineto{\pgfqpoint{1.463760in}{1.330000in}}%
\pgfpathlineto{\pgfqpoint{1.465000in}{1.120000in}}%
\pgfpathlineto{\pgfqpoint{1.466240in}{1.260000in}}%
\pgfpathlineto{\pgfqpoint{1.467480in}{1.260000in}}%
\pgfpathlineto{\pgfqpoint{1.468720in}{1.645000in}}%
\pgfpathlineto{\pgfqpoint{1.471200in}{0.875000in}}%
\pgfpathlineto{\pgfqpoint{1.474920in}{1.645000in}}%
\pgfpathlineto{\pgfqpoint{1.476160in}{1.295000in}}%
\pgfpathlineto{\pgfqpoint{1.477400in}{1.470000in}}%
\pgfpathlineto{\pgfqpoint{1.478640in}{1.855000in}}%
\pgfpathlineto{\pgfqpoint{1.479880in}{1.400000in}}%
\pgfpathlineto{\pgfqpoint{1.481120in}{1.890000in}}%
\pgfpathlineto{\pgfqpoint{1.482360in}{1.295000in}}%
\pgfpathlineto{\pgfqpoint{1.483600in}{1.610000in}}%
\pgfpathlineto{\pgfqpoint{1.484840in}{1.540000in}}%
\pgfpathlineto{\pgfqpoint{1.486080in}{1.540000in}}%
\pgfpathlineto{\pgfqpoint{1.487320in}{1.645000in}}%
\pgfpathlineto{\pgfqpoint{1.488560in}{1.610000in}}%
\pgfpathlineto{\pgfqpoint{1.489800in}{1.470000in}}%
\pgfpathlineto{\pgfqpoint{1.491040in}{1.470000in}}%
\pgfpathlineto{\pgfqpoint{1.493520in}{1.715000in}}%
\pgfpathlineto{\pgfqpoint{1.494760in}{1.715000in}}%
\pgfpathlineto{\pgfqpoint{1.496000in}{1.470000in}}%
\pgfpathlineto{\pgfqpoint{1.498480in}{1.610000in}}%
\pgfpathlineto{\pgfqpoint{1.499720in}{1.925000in}}%
\pgfpathlineto{\pgfqpoint{1.500960in}{1.435000in}}%
\pgfpathlineto{\pgfqpoint{1.502200in}{1.400000in}}%
\pgfpathlineto{\pgfqpoint{1.503440in}{1.400000in}}%
\pgfpathlineto{\pgfqpoint{1.504680in}{1.575000in}}%
\pgfpathlineto{\pgfqpoint{1.505920in}{2.415000in}}%
\pgfpathlineto{\pgfqpoint{1.507160in}{1.365000in}}%
\pgfpathlineto{\pgfqpoint{1.509640in}{1.645000in}}%
\pgfpathlineto{\pgfqpoint{1.510880in}{1.680000in}}%
\pgfpathlineto{\pgfqpoint{1.512120in}{1.365000in}}%
\pgfpathlineto{\pgfqpoint{1.514600in}{1.505000in}}%
\pgfpathlineto{\pgfqpoint{1.515840in}{1.295000in}}%
\pgfpathlineto{\pgfqpoint{1.517080in}{1.680000in}}%
\pgfpathlineto{\pgfqpoint{1.518320in}{1.470000in}}%
\pgfpathlineto{\pgfqpoint{1.519560in}{1.855000in}}%
\pgfpathlineto{\pgfqpoint{1.522040in}{1.505000in}}%
\pgfpathlineto{\pgfqpoint{1.523280in}{1.575000in}}%
\pgfpathlineto{\pgfqpoint{1.524520in}{1.575000in}}%
\pgfpathlineto{\pgfqpoint{1.527000in}{1.155000in}}%
\pgfpathlineto{\pgfqpoint{1.528240in}{1.540000in}}%
\pgfpathlineto{\pgfqpoint{1.529480in}{1.260000in}}%
\pgfpathlineto{\pgfqpoint{1.530720in}{1.295000in}}%
\pgfpathlineto{\pgfqpoint{1.531960in}{1.470000in}}%
\pgfpathlineto{\pgfqpoint{1.534440in}{1.155000in}}%
\pgfpathlineto{\pgfqpoint{1.535680in}{1.225000in}}%
\pgfpathlineto{\pgfqpoint{1.536920in}{1.400000in}}%
\pgfpathlineto{\pgfqpoint{1.538160in}{1.295000in}}%
\pgfpathlineto{\pgfqpoint{1.539400in}{1.400000in}}%
\pgfpathlineto{\pgfqpoint{1.540640in}{1.365000in}}%
\pgfpathlineto{\pgfqpoint{1.541880in}{1.295000in}}%
\pgfpathlineto{\pgfqpoint{1.543120in}{1.085000in}}%
\pgfpathlineto{\pgfqpoint{1.544360in}{1.645000in}}%
\pgfpathlineto{\pgfqpoint{1.546840in}{1.120000in}}%
\pgfpathlineto{\pgfqpoint{1.548080in}{1.295000in}}%
\pgfpathlineto{\pgfqpoint{1.549320in}{2.100000in}}%
\pgfpathlineto{\pgfqpoint{1.550560in}{1.610000in}}%
\pgfpathlineto{\pgfqpoint{1.551800in}{1.645000in}}%
\pgfpathlineto{\pgfqpoint{1.553040in}{1.575000in}}%
\pgfpathlineto{\pgfqpoint{1.554280in}{1.365000in}}%
\pgfpathlineto{\pgfqpoint{1.556760in}{1.785000in}}%
\pgfpathlineto{\pgfqpoint{1.558000in}{1.050000in}}%
\pgfpathlineto{\pgfqpoint{1.559240in}{1.260000in}}%
\pgfpathlineto{\pgfqpoint{1.560480in}{0.805000in}}%
\pgfpathlineto{\pgfqpoint{1.562960in}{1.190000in}}%
\pgfpathlineto{\pgfqpoint{1.564200in}{1.190000in}}%
\pgfpathlineto{\pgfqpoint{1.565440in}{1.715000in}}%
\pgfpathlineto{\pgfqpoint{1.566680in}{1.155000in}}%
\pgfpathlineto{\pgfqpoint{1.570400in}{1.680000in}}%
\pgfpathlineto{\pgfqpoint{1.571640in}{1.610000in}}%
\pgfpathlineto{\pgfqpoint{1.572880in}{1.750000in}}%
\pgfpathlineto{\pgfqpoint{1.574120in}{1.155000in}}%
\pgfpathlineto{\pgfqpoint{1.575360in}{1.750000in}}%
\pgfpathlineto{\pgfqpoint{1.576600in}{1.505000in}}%
\pgfpathlineto{\pgfqpoint{1.577840in}{1.680000in}}%
\pgfpathlineto{\pgfqpoint{1.579080in}{1.645000in}}%
\pgfpathlineto{\pgfqpoint{1.580320in}{1.540000in}}%
\pgfpathlineto{\pgfqpoint{1.581560in}{1.575000in}}%
\pgfpathlineto{\pgfqpoint{1.582800in}{1.365000in}}%
\pgfpathlineto{\pgfqpoint{1.584040in}{1.680000in}}%
\pgfpathlineto{\pgfqpoint{1.585280in}{1.470000in}}%
\pgfpathlineto{\pgfqpoint{1.586520in}{1.785000in}}%
\pgfpathlineto{\pgfqpoint{1.587760in}{1.190000in}}%
\pgfpathlineto{\pgfqpoint{1.589000in}{1.435000in}}%
\pgfpathlineto{\pgfqpoint{1.591480in}{1.225000in}}%
\pgfpathlineto{\pgfqpoint{1.593960in}{1.890000in}}%
\pgfpathlineto{\pgfqpoint{1.595200in}{1.680000in}}%
\pgfpathlineto{\pgfqpoint{1.596440in}{1.680000in}}%
\pgfpathlineto{\pgfqpoint{1.597680in}{0.980000in}}%
\pgfpathlineto{\pgfqpoint{1.598920in}{1.400000in}}%
\pgfpathlineto{\pgfqpoint{1.600160in}{1.260000in}}%
\pgfpathlineto{\pgfqpoint{1.602640in}{1.820000in}}%
\pgfpathlineto{\pgfqpoint{1.603880in}{1.330000in}}%
\pgfpathlineto{\pgfqpoint{1.605120in}{1.295000in}}%
\pgfpathlineto{\pgfqpoint{1.606360in}{1.190000in}}%
\pgfpathlineto{\pgfqpoint{1.607600in}{1.260000in}}%
\pgfpathlineto{\pgfqpoint{1.608840in}{1.575000in}}%
\pgfpathlineto{\pgfqpoint{1.610080in}{1.225000in}}%
\pgfpathlineto{\pgfqpoint{1.611320in}{1.190000in}}%
\pgfpathlineto{\pgfqpoint{1.612560in}{1.295000in}}%
\pgfpathlineto{\pgfqpoint{1.613800in}{0.875000in}}%
\pgfpathlineto{\pgfqpoint{1.615040in}{1.925000in}}%
\pgfpathlineto{\pgfqpoint{1.616280in}{1.120000in}}%
\pgfpathlineto{\pgfqpoint{1.617520in}{1.470000in}}%
\pgfpathlineto{\pgfqpoint{1.618760in}{1.085000in}}%
\pgfpathlineto{\pgfqpoint{1.620000in}{1.645000in}}%
\pgfpathlineto{\pgfqpoint{1.621240in}{1.155000in}}%
\pgfpathlineto{\pgfqpoint{1.623720in}{1.505000in}}%
\pgfpathlineto{\pgfqpoint{1.624960in}{1.225000in}}%
\pgfpathlineto{\pgfqpoint{1.626200in}{1.470000in}}%
\pgfpathlineto{\pgfqpoint{1.627440in}{1.365000in}}%
\pgfpathlineto{\pgfqpoint{1.628680in}{1.505000in}}%
\pgfpathlineto{\pgfqpoint{1.629920in}{1.190000in}}%
\pgfpathlineto{\pgfqpoint{1.632400in}{1.330000in}}%
\pgfpathlineto{\pgfqpoint{1.633640in}{0.945000in}}%
\pgfpathlineto{\pgfqpoint{1.636120in}{1.820000in}}%
\pgfpathlineto{\pgfqpoint{1.637360in}{1.225000in}}%
\pgfpathlineto{\pgfqpoint{1.638600in}{1.505000in}}%
\pgfpathlineto{\pgfqpoint{1.639840in}{1.400000in}}%
\pgfpathlineto{\pgfqpoint{1.641080in}{1.610000in}}%
\pgfpathlineto{\pgfqpoint{1.642320in}{1.085000in}}%
\pgfpathlineto{\pgfqpoint{1.643560in}{1.365000in}}%
\pgfpathlineto{\pgfqpoint{1.644800in}{1.225000in}}%
\pgfpathlineto{\pgfqpoint{1.646040in}{1.540000in}}%
\pgfpathlineto{\pgfqpoint{1.647280in}{1.435000in}}%
\pgfpathlineto{\pgfqpoint{1.648520in}{1.435000in}}%
\pgfpathlineto{\pgfqpoint{1.649760in}{1.260000in}}%
\pgfpathlineto{\pgfqpoint{1.651000in}{1.540000in}}%
\pgfpathlineto{\pgfqpoint{1.652240in}{0.910000in}}%
\pgfpathlineto{\pgfqpoint{1.654720in}{1.785000in}}%
\pgfpathlineto{\pgfqpoint{1.658440in}{1.120000in}}%
\pgfpathlineto{\pgfqpoint{1.660920in}{1.890000in}}%
\pgfpathlineto{\pgfqpoint{1.662160in}{1.610000in}}%
\pgfpathlineto{\pgfqpoint{1.663400in}{1.715000in}}%
\pgfpathlineto{\pgfqpoint{1.664640in}{1.190000in}}%
\pgfpathlineto{\pgfqpoint{1.665880in}{1.505000in}}%
\pgfpathlineto{\pgfqpoint{1.667120in}{1.400000in}}%
\pgfpathlineto{\pgfqpoint{1.668360in}{1.435000in}}%
\pgfpathlineto{\pgfqpoint{1.669600in}{1.505000in}}%
\pgfpathlineto{\pgfqpoint{1.670840in}{1.400000in}}%
\pgfpathlineto{\pgfqpoint{1.672080in}{1.540000in}}%
\pgfpathlineto{\pgfqpoint{1.673320in}{1.085000in}}%
\pgfpathlineto{\pgfqpoint{1.674560in}{1.260000in}}%
\pgfpathlineto{\pgfqpoint{1.675800in}{1.085000in}}%
\pgfpathlineto{\pgfqpoint{1.677040in}{1.155000in}}%
\pgfpathlineto{\pgfqpoint{1.678280in}{2.030000in}}%
\pgfpathlineto{\pgfqpoint{1.680760in}{1.575000in}}%
\pgfpathlineto{\pgfqpoint{1.682000in}{1.925000in}}%
\pgfpathlineto{\pgfqpoint{1.683240in}{1.785000in}}%
\pgfpathlineto{\pgfqpoint{1.684480in}{1.890000in}}%
\pgfpathlineto{\pgfqpoint{1.686960in}{1.330000in}}%
\pgfpathlineto{\pgfqpoint{1.688200in}{1.890000in}}%
\pgfpathlineto{\pgfqpoint{1.690680in}{1.400000in}}%
\pgfpathlineto{\pgfqpoint{1.691920in}{1.470000in}}%
\pgfpathlineto{\pgfqpoint{1.693160in}{1.435000in}}%
\pgfpathlineto{\pgfqpoint{1.695640in}{1.050000in}}%
\pgfpathlineto{\pgfqpoint{1.696880in}{1.050000in}}%
\pgfpathlineto{\pgfqpoint{1.699360in}{1.225000in}}%
\pgfpathlineto{\pgfqpoint{1.700600in}{1.715000in}}%
\pgfpathlineto{\pgfqpoint{1.703080in}{1.225000in}}%
\pgfpathlineto{\pgfqpoint{1.704320in}{1.155000in}}%
\pgfpathlineto{\pgfqpoint{1.705560in}{1.645000in}}%
\pgfpathlineto{\pgfqpoint{1.706800in}{1.260000in}}%
\pgfpathlineto{\pgfqpoint{1.708040in}{1.855000in}}%
\pgfpathlineto{\pgfqpoint{1.710520in}{1.400000in}}%
\pgfpathlineto{\pgfqpoint{1.711760in}{1.820000in}}%
\pgfpathlineto{\pgfqpoint{1.714240in}{0.840000in}}%
\pgfpathlineto{\pgfqpoint{1.715480in}{1.330000in}}%
\pgfpathlineto{\pgfqpoint{1.716720in}{1.365000in}}%
\pgfpathlineto{\pgfqpoint{1.717960in}{1.155000in}}%
\pgfpathlineto{\pgfqpoint{1.720440in}{1.470000in}}%
\pgfpathlineto{\pgfqpoint{1.721680in}{1.225000in}}%
\pgfpathlineto{\pgfqpoint{1.722920in}{1.750000in}}%
\pgfpathlineto{\pgfqpoint{1.724160in}{1.225000in}}%
\pgfpathlineto{\pgfqpoint{1.725400in}{1.260000in}}%
\pgfpathlineto{\pgfqpoint{1.727880in}{0.875000in}}%
\pgfpathlineto{\pgfqpoint{1.729120in}{1.190000in}}%
\pgfpathlineto{\pgfqpoint{1.730360in}{1.120000in}}%
\pgfpathlineto{\pgfqpoint{1.732840in}{1.890000in}}%
\pgfpathlineto{\pgfqpoint{1.734080in}{1.365000in}}%
\pgfpathlineto{\pgfqpoint{1.735320in}{1.330000in}}%
\pgfpathlineto{\pgfqpoint{1.736560in}{1.610000in}}%
\pgfpathlineto{\pgfqpoint{1.739040in}{1.260000in}}%
\pgfpathlineto{\pgfqpoint{1.742760in}{1.505000in}}%
\pgfpathlineto{\pgfqpoint{1.744000in}{1.470000in}}%
\pgfpathlineto{\pgfqpoint{1.745240in}{1.680000in}}%
\pgfpathlineto{\pgfqpoint{1.746480in}{1.575000in}}%
\pgfpathlineto{\pgfqpoint{1.747720in}{1.365000in}}%
\pgfpathlineto{\pgfqpoint{1.748960in}{1.400000in}}%
\pgfpathlineto{\pgfqpoint{1.750200in}{1.190000in}}%
\pgfpathlineto{\pgfqpoint{1.751440in}{1.715000in}}%
\pgfpathlineto{\pgfqpoint{1.753920in}{1.505000in}}%
\pgfpathlineto{\pgfqpoint{1.755160in}{1.680000in}}%
\pgfpathlineto{\pgfqpoint{1.757640in}{1.400000in}}%
\pgfpathlineto{\pgfqpoint{1.758880in}{1.575000in}}%
\pgfpathlineto{\pgfqpoint{1.760120in}{1.505000in}}%
\pgfpathlineto{\pgfqpoint{1.761360in}{1.050000in}}%
\pgfpathlineto{\pgfqpoint{1.762600in}{1.925000in}}%
\pgfpathlineto{\pgfqpoint{1.763840in}{1.260000in}}%
\pgfpathlineto{\pgfqpoint{1.765080in}{1.365000in}}%
\pgfpathlineto{\pgfqpoint{1.766320in}{1.295000in}}%
\pgfpathlineto{\pgfqpoint{1.767560in}{2.240000in}}%
\pgfpathlineto{\pgfqpoint{1.770040in}{1.505000in}}%
\pgfpathlineto{\pgfqpoint{1.771280in}{1.995000in}}%
\pgfpathlineto{\pgfqpoint{1.772520in}{1.925000in}}%
\pgfpathlineto{\pgfqpoint{1.773760in}{1.435000in}}%
\pgfpathlineto{\pgfqpoint{1.776240in}{1.715000in}}%
\pgfpathlineto{\pgfqpoint{1.777480in}{0.980000in}}%
\pgfpathlineto{\pgfqpoint{1.778720in}{0.980000in}}%
\pgfpathlineto{\pgfqpoint{1.781200in}{1.575000in}}%
\pgfpathlineto{\pgfqpoint{1.782440in}{1.295000in}}%
\pgfpathlineto{\pgfqpoint{1.784920in}{1.505000in}}%
\pgfpathlineto{\pgfqpoint{1.786160in}{1.505000in}}%
\pgfpathlineto{\pgfqpoint{1.788640in}{1.750000in}}%
\pgfpathlineto{\pgfqpoint{1.789880in}{1.855000in}}%
\pgfpathlineto{\pgfqpoint{1.791120in}{1.260000in}}%
\pgfpathlineto{\pgfqpoint{1.792360in}{1.260000in}}%
\pgfpathlineto{\pgfqpoint{1.793600in}{1.890000in}}%
\pgfpathlineto{\pgfqpoint{1.794840in}{1.715000in}}%
\pgfpathlineto{\pgfqpoint{1.796080in}{1.715000in}}%
\pgfpathlineto{\pgfqpoint{1.798560in}{1.190000in}}%
\pgfpathlineto{\pgfqpoint{1.799800in}{1.225000in}}%
\pgfpathlineto{\pgfqpoint{1.801040in}{1.015000in}}%
\pgfpathlineto{\pgfqpoint{1.803520in}{1.750000in}}%
\pgfpathlineto{\pgfqpoint{1.804760in}{1.435000in}}%
\pgfpathlineto{\pgfqpoint{1.806000in}{1.960000in}}%
\pgfpathlineto{\pgfqpoint{1.808480in}{1.330000in}}%
\pgfpathlineto{\pgfqpoint{1.809720in}{1.155000in}}%
\pgfpathlineto{\pgfqpoint{1.810960in}{1.365000in}}%
\pgfpathlineto{\pgfqpoint{1.812200in}{1.260000in}}%
\pgfpathlineto{\pgfqpoint{1.815920in}{1.785000in}}%
\pgfpathlineto{\pgfqpoint{1.817160in}{1.750000in}}%
\pgfpathlineto{\pgfqpoint{1.818400in}{1.715000in}}%
\pgfpathlineto{\pgfqpoint{1.819640in}{1.295000in}}%
\pgfpathlineto{\pgfqpoint{1.820880in}{1.645000in}}%
\pgfpathlineto{\pgfqpoint{1.822120in}{1.610000in}}%
\pgfpathlineto{\pgfqpoint{1.823360in}{1.610000in}}%
\pgfpathlineto{\pgfqpoint{1.824600in}{1.680000in}}%
\pgfpathlineto{\pgfqpoint{1.825840in}{1.435000in}}%
\pgfpathlineto{\pgfqpoint{1.827080in}{1.435000in}}%
\pgfpathlineto{\pgfqpoint{1.828320in}{1.855000in}}%
\pgfpathlineto{\pgfqpoint{1.829560in}{1.365000in}}%
\pgfpathlineto{\pgfqpoint{1.832040in}{1.785000in}}%
\pgfpathlineto{\pgfqpoint{1.835760in}{1.295000in}}%
\pgfpathlineto{\pgfqpoint{1.837000in}{1.680000in}}%
\pgfpathlineto{\pgfqpoint{1.838240in}{1.260000in}}%
\pgfpathlineto{\pgfqpoint{1.839480in}{1.540000in}}%
\pgfpathlineto{\pgfqpoint{1.840720in}{1.400000in}}%
\pgfpathlineto{\pgfqpoint{1.841960in}{1.120000in}}%
\pgfpathlineto{\pgfqpoint{1.843200in}{1.365000in}}%
\pgfpathlineto{\pgfqpoint{1.844440in}{1.330000in}}%
\pgfpathlineto{\pgfqpoint{1.846920in}{1.330000in}}%
\pgfpathlineto{\pgfqpoint{1.848160in}{1.400000in}}%
\pgfpathlineto{\pgfqpoint{1.849400in}{1.680000in}}%
\pgfpathlineto{\pgfqpoint{1.850640in}{1.435000in}}%
\pgfpathlineto{\pgfqpoint{1.851880in}{1.540000in}}%
\pgfpathlineto{\pgfqpoint{1.853120in}{1.785000in}}%
\pgfpathlineto{\pgfqpoint{1.854360in}{1.295000in}}%
\pgfpathlineto{\pgfqpoint{1.855600in}{1.260000in}}%
\pgfpathlineto{\pgfqpoint{1.856840in}{1.120000in}}%
\pgfpathlineto{\pgfqpoint{1.858080in}{1.260000in}}%
\pgfpathlineto{\pgfqpoint{1.859320in}{1.225000in}}%
\pgfpathlineto{\pgfqpoint{1.860560in}{1.155000in}}%
\pgfpathlineto{\pgfqpoint{1.861800in}{0.840000in}}%
\pgfpathlineto{\pgfqpoint{1.863040in}{1.225000in}}%
\pgfpathlineto{\pgfqpoint{1.864280in}{0.840000in}}%
\pgfpathlineto{\pgfqpoint{1.866760in}{1.610000in}}%
\pgfpathlineto{\pgfqpoint{1.868000in}{1.190000in}}%
\pgfpathlineto{\pgfqpoint{1.869240in}{1.190000in}}%
\pgfpathlineto{\pgfqpoint{1.870480in}{1.645000in}}%
\pgfpathlineto{\pgfqpoint{1.871720in}{1.365000in}}%
\pgfpathlineto{\pgfqpoint{1.872960in}{1.505000in}}%
\pgfpathlineto{\pgfqpoint{1.874200in}{0.945000in}}%
\pgfpathlineto{\pgfqpoint{1.875440in}{1.400000in}}%
\pgfpathlineto{\pgfqpoint{1.876680in}{1.330000in}}%
\pgfpathlineto{\pgfqpoint{1.877920in}{1.575000in}}%
\pgfpathlineto{\pgfqpoint{1.879160in}{1.540000in}}%
\pgfpathlineto{\pgfqpoint{1.881640in}{1.155000in}}%
\pgfpathlineto{\pgfqpoint{1.882880in}{1.225000in}}%
\pgfpathlineto{\pgfqpoint{1.884120in}{1.715000in}}%
\pgfpathlineto{\pgfqpoint{1.889080in}{1.120000in}}%
\pgfpathlineto{\pgfqpoint{1.890320in}{1.470000in}}%
\pgfpathlineto{\pgfqpoint{1.891560in}{1.435000in}}%
\pgfpathlineto{\pgfqpoint{1.892800in}{1.820000in}}%
\pgfpathlineto{\pgfqpoint{1.894040in}{1.260000in}}%
\pgfpathlineto{\pgfqpoint{1.895280in}{1.820000in}}%
\pgfpathlineto{\pgfqpoint{1.896520in}{1.260000in}}%
\pgfpathlineto{\pgfqpoint{1.897760in}{1.680000in}}%
\pgfpathlineto{\pgfqpoint{1.900240in}{1.050000in}}%
\pgfpathlineto{\pgfqpoint{1.901480in}{1.505000in}}%
\pgfpathlineto{\pgfqpoint{1.903960in}{1.575000in}}%
\pgfpathlineto{\pgfqpoint{1.906440in}{1.120000in}}%
\pgfpathlineto{\pgfqpoint{1.908920in}{1.505000in}}%
\pgfpathlineto{\pgfqpoint{1.910160in}{1.470000in}}%
\pgfpathlineto{\pgfqpoint{1.911400in}{1.645000in}}%
\pgfpathlineto{\pgfqpoint{1.912640in}{1.295000in}}%
\pgfpathlineto{\pgfqpoint{1.913880in}{1.435000in}}%
\pgfpathlineto{\pgfqpoint{1.915120in}{1.400000in}}%
\pgfpathlineto{\pgfqpoint{1.916360in}{1.750000in}}%
\pgfpathlineto{\pgfqpoint{1.917600in}{1.260000in}}%
\pgfpathlineto{\pgfqpoint{1.918840in}{1.505000in}}%
\pgfpathlineto{\pgfqpoint{1.920080in}{1.435000in}}%
\pgfpathlineto{\pgfqpoint{1.921320in}{1.435000in}}%
\pgfpathlineto{\pgfqpoint{1.922560in}{1.505000in}}%
\pgfpathlineto{\pgfqpoint{1.923800in}{1.330000in}}%
\pgfpathlineto{\pgfqpoint{1.925040in}{1.680000in}}%
\pgfpathlineto{\pgfqpoint{1.926280in}{1.190000in}}%
\pgfpathlineto{\pgfqpoint{1.927520in}{1.470000in}}%
\pgfpathlineto{\pgfqpoint{1.930000in}{1.085000in}}%
\pgfpathlineto{\pgfqpoint{1.931240in}{1.435000in}}%
\pgfpathlineto{\pgfqpoint{1.932480in}{1.470000in}}%
\pgfpathlineto{\pgfqpoint{1.934960in}{1.190000in}}%
\pgfpathlineto{\pgfqpoint{1.936200in}{1.575000in}}%
\pgfpathlineto{\pgfqpoint{1.937440in}{1.295000in}}%
\pgfpathlineto{\pgfqpoint{1.938680in}{1.680000in}}%
\pgfpathlineto{\pgfqpoint{1.941160in}{1.260000in}}%
\pgfpathlineto{\pgfqpoint{1.942400in}{1.295000in}}%
\pgfpathlineto{\pgfqpoint{1.943640in}{0.875000in}}%
\pgfpathlineto{\pgfqpoint{1.944880in}{1.925000in}}%
\pgfpathlineto{\pgfqpoint{1.946120in}{1.015000in}}%
\pgfpathlineto{\pgfqpoint{1.947360in}{1.470000in}}%
\pgfpathlineto{\pgfqpoint{1.948600in}{1.400000in}}%
\pgfpathlineto{\pgfqpoint{1.949840in}{1.925000in}}%
\pgfpathlineto{\pgfqpoint{1.952320in}{1.400000in}}%
\pgfpathlineto{\pgfqpoint{1.953560in}{0.945000in}}%
\pgfpathlineto{\pgfqpoint{1.954800in}{1.015000in}}%
\pgfpathlineto{\pgfqpoint{1.956040in}{1.750000in}}%
\pgfpathlineto{\pgfqpoint{1.957280in}{1.085000in}}%
\pgfpathlineto{\pgfqpoint{1.958520in}{1.540000in}}%
\pgfpathlineto{\pgfqpoint{1.959760in}{1.435000in}}%
\pgfpathlineto{\pgfqpoint{1.961000in}{1.680000in}}%
\pgfpathlineto{\pgfqpoint{1.962240in}{1.540000in}}%
\pgfpathlineto{\pgfqpoint{1.963480in}{1.610000in}}%
\pgfpathlineto{\pgfqpoint{1.964720in}{1.435000in}}%
\pgfpathlineto{\pgfqpoint{1.965960in}{1.680000in}}%
\pgfpathlineto{\pgfqpoint{1.967200in}{1.610000in}}%
\pgfpathlineto{\pgfqpoint{1.969680in}{1.015000in}}%
\pgfpathlineto{\pgfqpoint{1.970920in}{1.610000in}}%
\pgfpathlineto{\pgfqpoint{1.972160in}{1.295000in}}%
\pgfpathlineto{\pgfqpoint{1.973400in}{1.400000in}}%
\pgfpathlineto{\pgfqpoint{1.974640in}{2.100000in}}%
\pgfpathlineto{\pgfqpoint{1.977120in}{1.435000in}}%
\pgfpathlineto{\pgfqpoint{1.978360in}{1.645000in}}%
\pgfpathlineto{\pgfqpoint{1.979600in}{1.260000in}}%
\pgfpathlineto{\pgfqpoint{1.982080in}{1.890000in}}%
\pgfpathlineto{\pgfqpoint{1.983320in}{1.680000in}}%
\pgfpathlineto{\pgfqpoint{1.984560in}{1.680000in}}%
\pgfpathlineto{\pgfqpoint{1.987040in}{1.925000in}}%
\pgfpathlineto{\pgfqpoint{1.988280in}{1.540000in}}%
\pgfpathlineto{\pgfqpoint{1.989520in}{1.890000in}}%
\pgfpathlineto{\pgfqpoint{1.990760in}{1.260000in}}%
\pgfpathlineto{\pgfqpoint{1.992000in}{1.785000in}}%
\pgfpathlineto{\pgfqpoint{1.993240in}{1.050000in}}%
\pgfpathlineto{\pgfqpoint{1.995720in}{1.365000in}}%
\pgfpathlineto{\pgfqpoint{1.996960in}{1.260000in}}%
\pgfpathlineto{\pgfqpoint{1.998200in}{1.505000in}}%
\pgfpathlineto{\pgfqpoint{1.999440in}{0.805000in}}%
\pgfpathlineto{\pgfqpoint{2.000680in}{1.575000in}}%
\pgfpathlineto{\pgfqpoint{2.003160in}{1.435000in}}%
\pgfpathlineto{\pgfqpoint{2.004400in}{1.785000in}}%
\pgfpathlineto{\pgfqpoint{2.005640in}{1.435000in}}%
\pgfpathlineto{\pgfqpoint{2.006880in}{1.820000in}}%
\pgfpathlineto{\pgfqpoint{2.008120in}{1.400000in}}%
\pgfpathlineto{\pgfqpoint{2.009360in}{1.505000in}}%
\pgfpathlineto{\pgfqpoint{2.010600in}{1.505000in}}%
\pgfpathlineto{\pgfqpoint{2.011840in}{1.575000in}}%
\pgfpathlineto{\pgfqpoint{2.014320in}{1.260000in}}%
\pgfpathlineto{\pgfqpoint{2.015560in}{1.435000in}}%
\pgfpathlineto{\pgfqpoint{2.016800in}{1.225000in}}%
\pgfpathlineto{\pgfqpoint{2.019280in}{1.750000in}}%
\pgfpathlineto{\pgfqpoint{2.021760in}{1.225000in}}%
\pgfpathlineto{\pgfqpoint{2.023000in}{1.785000in}}%
\pgfpathlineto{\pgfqpoint{2.024240in}{1.680000in}}%
\pgfpathlineto{\pgfqpoint{2.025480in}{1.680000in}}%
\pgfpathlineto{\pgfqpoint{2.027960in}{1.470000in}}%
\pgfpathlineto{\pgfqpoint{2.029200in}{1.505000in}}%
\pgfpathlineto{\pgfqpoint{2.030440in}{1.680000in}}%
\pgfpathlineto{\pgfqpoint{2.031680in}{1.295000in}}%
\pgfpathlineto{\pgfqpoint{2.032920in}{1.645000in}}%
\pgfpathlineto{\pgfqpoint{2.034160in}{1.575000in}}%
\pgfpathlineto{\pgfqpoint{2.036640in}{1.120000in}}%
\pgfpathlineto{\pgfqpoint{2.037880in}{1.295000in}}%
\pgfpathlineto{\pgfqpoint{2.039120in}{1.225000in}}%
\pgfpathlineto{\pgfqpoint{2.040360in}{1.540000in}}%
\pgfpathlineto{\pgfqpoint{2.041600in}{0.840000in}}%
\pgfpathlineto{\pgfqpoint{2.044080in}{1.330000in}}%
\pgfpathlineto{\pgfqpoint{2.047800in}{1.435000in}}%
\pgfpathlineto{\pgfqpoint{2.050280in}{1.575000in}}%
\pgfpathlineto{\pgfqpoint{2.051520in}{1.645000in}}%
\pgfpathlineto{\pgfqpoint{2.052760in}{1.155000in}}%
\pgfpathlineto{\pgfqpoint{2.055240in}{1.680000in}}%
\pgfpathlineto{\pgfqpoint{2.056480in}{1.680000in}}%
\pgfpathlineto{\pgfqpoint{2.057720in}{1.960000in}}%
\pgfpathlineto{\pgfqpoint{2.058960in}{1.890000in}}%
\pgfpathlineto{\pgfqpoint{2.060200in}{2.205000in}}%
\pgfpathlineto{\pgfqpoint{2.061440in}{1.610000in}}%
\pgfpathlineto{\pgfqpoint{2.062680in}{1.785000in}}%
\pgfpathlineto{\pgfqpoint{2.063920in}{1.715000in}}%
\pgfpathlineto{\pgfqpoint{2.065160in}{1.470000in}}%
\pgfpathlineto{\pgfqpoint{2.066400in}{1.470000in}}%
\pgfpathlineto{\pgfqpoint{2.067640in}{1.295000in}}%
\pgfpathlineto{\pgfqpoint{2.068880in}{1.435000in}}%
\pgfpathlineto{\pgfqpoint{2.070120in}{1.365000in}}%
\pgfpathlineto{\pgfqpoint{2.071360in}{1.365000in}}%
\pgfpathlineto{\pgfqpoint{2.073840in}{1.750000in}}%
\pgfpathlineto{\pgfqpoint{2.075080in}{1.260000in}}%
\pgfpathlineto{\pgfqpoint{2.077560in}{1.400000in}}%
\pgfpathlineto{\pgfqpoint{2.078800in}{1.785000in}}%
\pgfpathlineto{\pgfqpoint{2.081280in}{1.505000in}}%
\pgfpathlineto{\pgfqpoint{2.082520in}{1.435000in}}%
\pgfpathlineto{\pgfqpoint{2.083760in}{1.680000in}}%
\pgfpathlineto{\pgfqpoint{2.086240in}{1.680000in}}%
\pgfpathlineto{\pgfqpoint{2.089960in}{1.190000in}}%
\pgfpathlineto{\pgfqpoint{2.091200in}{1.400000in}}%
\pgfpathlineto{\pgfqpoint{2.092440in}{1.225000in}}%
\pgfpathlineto{\pgfqpoint{2.093680in}{1.400000in}}%
\pgfpathlineto{\pgfqpoint{2.096160in}{1.155000in}}%
\pgfpathlineto{\pgfqpoint{2.097400in}{1.330000in}}%
\pgfpathlineto{\pgfqpoint{2.098640in}{1.260000in}}%
\pgfpathlineto{\pgfqpoint{2.099880in}{1.435000in}}%
\pgfpathlineto{\pgfqpoint{2.102360in}{1.330000in}}%
\pgfpathlineto{\pgfqpoint{2.103600in}{1.400000in}}%
\pgfpathlineto{\pgfqpoint{2.106080in}{1.820000in}}%
\pgfpathlineto{\pgfqpoint{2.107320in}{1.295000in}}%
\pgfpathlineto{\pgfqpoint{2.108560in}{1.645000in}}%
\pgfpathlineto{\pgfqpoint{2.109800in}{1.085000in}}%
\pgfpathlineto{\pgfqpoint{2.113520in}{1.750000in}}%
\pgfpathlineto{\pgfqpoint{2.116000in}{1.890000in}}%
\pgfpathlineto{\pgfqpoint{2.117240in}{1.330000in}}%
\pgfpathlineto{\pgfqpoint{2.118480in}{1.645000in}}%
\pgfpathlineto{\pgfqpoint{2.119720in}{1.085000in}}%
\pgfpathlineto{\pgfqpoint{2.122200in}{1.365000in}}%
\pgfpathlineto{\pgfqpoint{2.123440in}{1.995000in}}%
\pgfpathlineto{\pgfqpoint{2.124680in}{1.925000in}}%
\pgfpathlineto{\pgfqpoint{2.125920in}{1.925000in}}%
\pgfpathlineto{\pgfqpoint{2.128400in}{1.120000in}}%
\pgfpathlineto{\pgfqpoint{2.129640in}{1.365000in}}%
\pgfpathlineto{\pgfqpoint{2.130880in}{1.050000in}}%
\pgfpathlineto{\pgfqpoint{2.132120in}{1.715000in}}%
\pgfpathlineto{\pgfqpoint{2.133360in}{1.750000in}}%
\pgfpathlineto{\pgfqpoint{2.134600in}{1.505000in}}%
\pgfpathlineto{\pgfqpoint{2.135840in}{1.505000in}}%
\pgfpathlineto{\pgfqpoint{2.137080in}{1.575000in}}%
\pgfpathlineto{\pgfqpoint{2.138320in}{1.435000in}}%
\pgfpathlineto{\pgfqpoint{2.139560in}{1.120000in}}%
\pgfpathlineto{\pgfqpoint{2.140800in}{1.400000in}}%
\pgfpathlineto{\pgfqpoint{2.142040in}{1.330000in}}%
\pgfpathlineto{\pgfqpoint{2.144520in}{1.855000in}}%
\pgfpathlineto{\pgfqpoint{2.145760in}{1.505000in}}%
\pgfpathlineto{\pgfqpoint{2.147000in}{1.505000in}}%
\pgfpathlineto{\pgfqpoint{2.148240in}{1.540000in}}%
\pgfpathlineto{\pgfqpoint{2.149480in}{1.680000in}}%
\pgfpathlineto{\pgfqpoint{2.150720in}{1.330000in}}%
\pgfpathlineto{\pgfqpoint{2.151960in}{1.435000in}}%
\pgfpathlineto{\pgfqpoint{2.153200in}{1.820000in}}%
\pgfpathlineto{\pgfqpoint{2.154440in}{1.225000in}}%
\pgfpathlineto{\pgfqpoint{2.155680in}{1.295000in}}%
\pgfpathlineto{\pgfqpoint{2.156920in}{1.260000in}}%
\pgfpathlineto{\pgfqpoint{2.158160in}{1.470000in}}%
\pgfpathlineto{\pgfqpoint{2.159400in}{1.435000in}}%
\pgfpathlineto{\pgfqpoint{2.161880in}{1.225000in}}%
\pgfpathlineto{\pgfqpoint{2.164360in}{1.680000in}}%
\pgfpathlineto{\pgfqpoint{2.165600in}{1.645000in}}%
\pgfpathlineto{\pgfqpoint{2.166840in}{1.680000in}}%
\pgfpathlineto{\pgfqpoint{2.168080in}{1.470000in}}%
\pgfpathlineto{\pgfqpoint{2.169320in}{1.680000in}}%
\pgfpathlineto{\pgfqpoint{2.170560in}{1.505000in}}%
\pgfpathlineto{\pgfqpoint{2.173040in}{1.715000in}}%
\pgfpathlineto{\pgfqpoint{2.174280in}{1.260000in}}%
\pgfpathlineto{\pgfqpoint{2.175520in}{1.260000in}}%
\pgfpathlineto{\pgfqpoint{2.176760in}{1.785000in}}%
\pgfpathlineto{\pgfqpoint{2.178000in}{1.225000in}}%
\pgfpathlineto{\pgfqpoint{2.179240in}{1.820000in}}%
\pgfpathlineto{\pgfqpoint{2.181720in}{1.120000in}}%
\pgfpathlineto{\pgfqpoint{2.182960in}{1.505000in}}%
\pgfpathlineto{\pgfqpoint{2.184200in}{1.225000in}}%
\pgfpathlineto{\pgfqpoint{2.185440in}{1.610000in}}%
\pgfpathlineto{\pgfqpoint{2.186680in}{1.400000in}}%
\pgfpathlineto{\pgfqpoint{2.189160in}{1.820000in}}%
\pgfpathlineto{\pgfqpoint{2.191640in}{1.225000in}}%
\pgfpathlineto{\pgfqpoint{2.192880in}{1.365000in}}%
\pgfpathlineto{\pgfqpoint{2.194120in}{1.190000in}}%
\pgfpathlineto{\pgfqpoint{2.195360in}{1.365000in}}%
\pgfpathlineto{\pgfqpoint{2.197840in}{0.910000in}}%
\pgfpathlineto{\pgfqpoint{2.199080in}{0.980000in}}%
\pgfpathlineto{\pgfqpoint{2.200320in}{1.295000in}}%
\pgfpathlineto{\pgfqpoint{2.201560in}{1.015000in}}%
\pgfpathlineto{\pgfqpoint{2.204040in}{1.365000in}}%
\pgfpathlineto{\pgfqpoint{2.205280in}{1.365000in}}%
\pgfpathlineto{\pgfqpoint{2.206520in}{1.015000in}}%
\pgfpathlineto{\pgfqpoint{2.207760in}{1.400000in}}%
\pgfpathlineto{\pgfqpoint{2.209000in}{1.085000in}}%
\pgfpathlineto{\pgfqpoint{2.210240in}{1.785000in}}%
\pgfpathlineto{\pgfqpoint{2.212720in}{1.225000in}}%
\pgfpathlineto{\pgfqpoint{2.213960in}{1.925000in}}%
\pgfpathlineto{\pgfqpoint{2.216440in}{0.700000in}}%
\pgfpathlineto{\pgfqpoint{2.217680in}{1.610000in}}%
\pgfpathlineto{\pgfqpoint{2.218920in}{1.330000in}}%
\pgfpathlineto{\pgfqpoint{2.220160in}{1.505000in}}%
\pgfpathlineto{\pgfqpoint{2.221400in}{1.330000in}}%
\pgfpathlineto{\pgfqpoint{2.223880in}{1.575000in}}%
\pgfpathlineto{\pgfqpoint{2.226360in}{1.085000in}}%
\pgfpathlineto{\pgfqpoint{2.228840in}{1.960000in}}%
\pgfpathlineto{\pgfqpoint{2.230080in}{1.715000in}}%
\pgfpathlineto{\pgfqpoint{2.231320in}{1.925000in}}%
\pgfpathlineto{\pgfqpoint{2.232560in}{1.750000in}}%
\pgfpathlineto{\pgfqpoint{2.233800in}{1.365000in}}%
\pgfpathlineto{\pgfqpoint{2.235040in}{1.645000in}}%
\pgfpathlineto{\pgfqpoint{2.236280in}{1.400000in}}%
\pgfpathlineto{\pgfqpoint{2.237520in}{1.750000in}}%
\pgfpathlineto{\pgfqpoint{2.238760in}{1.680000in}}%
\pgfpathlineto{\pgfqpoint{2.240000in}{1.855000in}}%
\pgfpathlineto{\pgfqpoint{2.241240in}{1.575000in}}%
\pgfpathlineto{\pgfqpoint{2.242480in}{1.575000in}}%
\pgfpathlineto{\pgfqpoint{2.243720in}{1.680000in}}%
\pgfpathlineto{\pgfqpoint{2.244960in}{1.330000in}}%
\pgfpathlineto{\pgfqpoint{2.246200in}{1.295000in}}%
\pgfpathlineto{\pgfqpoint{2.247440in}{1.295000in}}%
\pgfpathlineto{\pgfqpoint{2.248680in}{1.400000in}}%
\pgfpathlineto{\pgfqpoint{2.249920in}{1.190000in}}%
\pgfpathlineto{\pgfqpoint{2.252400in}{1.365000in}}%
\pgfpathlineto{\pgfqpoint{2.253640in}{1.610000in}}%
\pgfpathlineto{\pgfqpoint{2.254880in}{1.365000in}}%
\pgfpathlineto{\pgfqpoint{2.256120in}{1.715000in}}%
\pgfpathlineto{\pgfqpoint{2.257360in}{1.190000in}}%
\pgfpathlineto{\pgfqpoint{2.258600in}{1.610000in}}%
\pgfpathlineto{\pgfqpoint{2.259840in}{1.190000in}}%
\pgfpathlineto{\pgfqpoint{2.261080in}{1.155000in}}%
\pgfpathlineto{\pgfqpoint{2.262320in}{1.645000in}}%
\pgfpathlineto{\pgfqpoint{2.263560in}{1.225000in}}%
\pgfpathlineto{\pgfqpoint{2.264800in}{1.400000in}}%
\pgfpathlineto{\pgfqpoint{2.266040in}{1.120000in}}%
\pgfpathlineto{\pgfqpoint{2.267280in}{1.155000in}}%
\pgfpathlineto{\pgfqpoint{2.268520in}{1.855000in}}%
\pgfpathlineto{\pgfqpoint{2.269760in}{1.785000in}}%
\pgfpathlineto{\pgfqpoint{2.271000in}{1.330000in}}%
\pgfpathlineto{\pgfqpoint{2.272240in}{1.715000in}}%
\pgfpathlineto{\pgfqpoint{2.273480in}{1.575000in}}%
\pgfpathlineto{\pgfqpoint{2.274720in}{1.575000in}}%
\pgfpathlineto{\pgfqpoint{2.275960in}{2.065000in}}%
\pgfpathlineto{\pgfqpoint{2.278440in}{0.980000in}}%
\pgfpathlineto{\pgfqpoint{2.279680in}{1.785000in}}%
\pgfpathlineto{\pgfqpoint{2.280920in}{1.540000in}}%
\pgfpathlineto{\pgfqpoint{2.282160in}{1.015000in}}%
\pgfpathlineto{\pgfqpoint{2.283400in}{1.190000in}}%
\pgfpathlineto{\pgfqpoint{2.284640in}{1.820000in}}%
\pgfpathlineto{\pgfqpoint{2.285880in}{1.365000in}}%
\pgfpathlineto{\pgfqpoint{2.287120in}{1.715000in}}%
\pgfpathlineto{\pgfqpoint{2.288360in}{1.400000in}}%
\pgfpathlineto{\pgfqpoint{2.289600in}{1.680000in}}%
\pgfpathlineto{\pgfqpoint{2.292080in}{1.680000in}}%
\pgfpathlineto{\pgfqpoint{2.293320in}{1.645000in}}%
\pgfpathlineto{\pgfqpoint{2.294560in}{1.820000in}}%
\pgfpathlineto{\pgfqpoint{2.297040in}{1.435000in}}%
\pgfpathlineto{\pgfqpoint{2.299520in}{1.960000in}}%
\pgfpathlineto{\pgfqpoint{2.300760in}{1.820000in}}%
\pgfpathlineto{\pgfqpoint{2.302000in}{1.995000in}}%
\pgfpathlineto{\pgfqpoint{2.303240in}{1.330000in}}%
\pgfpathlineto{\pgfqpoint{2.304480in}{1.400000in}}%
\pgfpathlineto{\pgfqpoint{2.306960in}{1.750000in}}%
\pgfpathlineto{\pgfqpoint{2.310680in}{1.120000in}}%
\pgfpathlineto{\pgfqpoint{2.311920in}{1.715000in}}%
\pgfpathlineto{\pgfqpoint{2.313160in}{1.575000in}}%
\pgfpathlineto{\pgfqpoint{2.314400in}{1.575000in}}%
\pgfpathlineto{\pgfqpoint{2.315640in}{0.980000in}}%
\pgfpathlineto{\pgfqpoint{2.318120in}{1.715000in}}%
\pgfpathlineto{\pgfqpoint{2.319360in}{1.330000in}}%
\pgfpathlineto{\pgfqpoint{2.321840in}{1.820000in}}%
\pgfpathlineto{\pgfqpoint{2.325560in}{1.085000in}}%
\pgfpathlineto{\pgfqpoint{2.326800in}{1.715000in}}%
\pgfpathlineto{\pgfqpoint{2.328040in}{1.260000in}}%
\pgfpathlineto{\pgfqpoint{2.329280in}{1.505000in}}%
\pgfpathlineto{\pgfqpoint{2.330520in}{1.470000in}}%
\pgfpathlineto{\pgfqpoint{2.331760in}{1.680000in}}%
\pgfpathlineto{\pgfqpoint{2.333000in}{1.295000in}}%
\pgfpathlineto{\pgfqpoint{2.337960in}{1.925000in}}%
\pgfpathlineto{\pgfqpoint{2.339200in}{1.435000in}}%
\pgfpathlineto{\pgfqpoint{2.340440in}{2.275000in}}%
\pgfpathlineto{\pgfqpoint{2.341680in}{1.470000in}}%
\pgfpathlineto{\pgfqpoint{2.342920in}{1.575000in}}%
\pgfpathlineto{\pgfqpoint{2.344160in}{1.295000in}}%
\pgfpathlineto{\pgfqpoint{2.346640in}{1.715000in}}%
\pgfpathlineto{\pgfqpoint{2.347880in}{1.645000in}}%
\pgfpathlineto{\pgfqpoint{2.349120in}{1.435000in}}%
\pgfpathlineto{\pgfqpoint{2.350360in}{1.470000in}}%
\pgfpathlineto{\pgfqpoint{2.351600in}{1.645000in}}%
\pgfpathlineto{\pgfqpoint{2.352840in}{1.645000in}}%
\pgfpathlineto{\pgfqpoint{2.354080in}{1.435000in}}%
\pgfpathlineto{\pgfqpoint{2.355320in}{1.610000in}}%
\pgfpathlineto{\pgfqpoint{2.357800in}{1.155000in}}%
\pgfpathlineto{\pgfqpoint{2.359040in}{1.120000in}}%
\pgfpathlineto{\pgfqpoint{2.361520in}{1.645000in}}%
\pgfpathlineto{\pgfqpoint{2.362760in}{1.575000in}}%
\pgfpathlineto{\pgfqpoint{2.364000in}{1.365000in}}%
\pgfpathlineto{\pgfqpoint{2.365240in}{1.715000in}}%
\pgfpathlineto{\pgfqpoint{2.366480in}{1.295000in}}%
\pgfpathlineto{\pgfqpoint{2.367720in}{1.540000in}}%
\pgfpathlineto{\pgfqpoint{2.368960in}{1.470000in}}%
\pgfpathlineto{\pgfqpoint{2.370200in}{1.820000in}}%
\pgfpathlineto{\pgfqpoint{2.371440in}{1.470000in}}%
\pgfpathlineto{\pgfqpoint{2.372680in}{1.435000in}}%
\pgfpathlineto{\pgfqpoint{2.373920in}{1.855000in}}%
\pgfpathlineto{\pgfqpoint{2.376400in}{1.505000in}}%
\pgfpathlineto{\pgfqpoint{2.377640in}{1.610000in}}%
\pgfpathlineto{\pgfqpoint{2.378880in}{2.065000in}}%
\pgfpathlineto{\pgfqpoint{2.380120in}{0.945000in}}%
\pgfpathlineto{\pgfqpoint{2.381360in}{1.260000in}}%
\pgfpathlineto{\pgfqpoint{2.382600in}{1.120000in}}%
\pgfpathlineto{\pgfqpoint{2.383840in}{1.435000in}}%
\pgfpathlineto{\pgfqpoint{2.385080in}{1.435000in}}%
\pgfpathlineto{\pgfqpoint{2.386320in}{1.400000in}}%
\pgfpathlineto{\pgfqpoint{2.387560in}{1.400000in}}%
\pgfpathlineto{\pgfqpoint{2.388800in}{1.260000in}}%
\pgfpathlineto{\pgfqpoint{2.390040in}{1.400000in}}%
\pgfpathlineto{\pgfqpoint{2.391280in}{1.820000in}}%
\pgfpathlineto{\pgfqpoint{2.392520in}{1.365000in}}%
\pgfpathlineto{\pgfqpoint{2.393760in}{1.540000in}}%
\pgfpathlineto{\pgfqpoint{2.395000in}{1.470000in}}%
\pgfpathlineto{\pgfqpoint{2.396240in}{1.505000in}}%
\pgfpathlineto{\pgfqpoint{2.398720in}{1.225000in}}%
\pgfpathlineto{\pgfqpoint{2.401200in}{1.680000in}}%
\pgfpathlineto{\pgfqpoint{2.402440in}{0.910000in}}%
\pgfpathlineto{\pgfqpoint{2.403680in}{1.435000in}}%
\pgfpathlineto{\pgfqpoint{2.404920in}{1.400000in}}%
\pgfpathlineto{\pgfqpoint{2.406160in}{1.225000in}}%
\pgfpathlineto{\pgfqpoint{2.407400in}{1.575000in}}%
\pgfpathlineto{\pgfqpoint{2.408640in}{1.225000in}}%
\pgfpathlineto{\pgfqpoint{2.411120in}{1.715000in}}%
\pgfpathlineto{\pgfqpoint{2.412360in}{1.715000in}}%
\pgfpathlineto{\pgfqpoint{2.413600in}{1.785000in}}%
\pgfpathlineto{\pgfqpoint{2.414840in}{1.680000in}}%
\pgfpathlineto{\pgfqpoint{2.417320in}{1.015000in}}%
\pgfpathlineto{\pgfqpoint{2.418560in}{1.575000in}}%
\pgfpathlineto{\pgfqpoint{2.419800in}{1.575000in}}%
\pgfpathlineto{\pgfqpoint{2.421040in}{1.085000in}}%
\pgfpathlineto{\pgfqpoint{2.422280in}{1.470000in}}%
\pgfpathlineto{\pgfqpoint{2.423520in}{1.365000in}}%
\pgfpathlineto{\pgfqpoint{2.424760in}{1.015000in}}%
\pgfpathlineto{\pgfqpoint{2.426000in}{1.680000in}}%
\pgfpathlineto{\pgfqpoint{2.427240in}{1.330000in}}%
\pgfpathlineto{\pgfqpoint{2.428480in}{1.575000in}}%
\pgfpathlineto{\pgfqpoint{2.429720in}{1.330000in}}%
\pgfpathlineto{\pgfqpoint{2.432200in}{1.575000in}}%
\pgfpathlineto{\pgfqpoint{2.434680in}{1.295000in}}%
\pgfpathlineto{\pgfqpoint{2.435920in}{1.400000in}}%
\pgfpathlineto{\pgfqpoint{2.437160in}{1.890000in}}%
\pgfpathlineto{\pgfqpoint{2.438400in}{1.505000in}}%
\pgfpathlineto{\pgfqpoint{2.440880in}{1.435000in}}%
\pgfpathlineto{\pgfqpoint{2.442120in}{1.540000in}}%
\pgfpathlineto{\pgfqpoint{2.443360in}{1.120000in}}%
\pgfpathlineto{\pgfqpoint{2.444600in}{1.260000in}}%
\pgfpathlineto{\pgfqpoint{2.445840in}{1.260000in}}%
\pgfpathlineto{\pgfqpoint{2.447080in}{1.785000in}}%
\pgfpathlineto{\pgfqpoint{2.449560in}{1.435000in}}%
\pgfpathlineto{\pgfqpoint{2.450800in}{1.190000in}}%
\pgfpathlineto{\pgfqpoint{2.452040in}{1.330000in}}%
\pgfpathlineto{\pgfqpoint{2.453280in}{1.155000in}}%
\pgfpathlineto{\pgfqpoint{2.454520in}{1.820000in}}%
\pgfpathlineto{\pgfqpoint{2.455760in}{1.645000in}}%
\pgfpathlineto{\pgfqpoint{2.457000in}{1.785000in}}%
\pgfpathlineto{\pgfqpoint{2.458240in}{1.575000in}}%
\pgfpathlineto{\pgfqpoint{2.459480in}{1.120000in}}%
\pgfpathlineto{\pgfqpoint{2.460720in}{1.750000in}}%
\pgfpathlineto{\pgfqpoint{2.463200in}{1.575000in}}%
\pgfpathlineto{\pgfqpoint{2.464440in}{1.610000in}}%
\pgfpathlineto{\pgfqpoint{2.465680in}{1.295000in}}%
\pgfpathlineto{\pgfqpoint{2.466920in}{1.330000in}}%
\pgfpathlineto{\pgfqpoint{2.469400in}{1.470000in}}%
\pgfpathlineto{\pgfqpoint{2.470640in}{1.120000in}}%
\pgfpathlineto{\pgfqpoint{2.471880in}{1.575000in}}%
\pgfpathlineto{\pgfqpoint{2.475600in}{1.050000in}}%
\pgfpathlineto{\pgfqpoint{2.478080in}{1.540000in}}%
\pgfpathlineto{\pgfqpoint{2.479320in}{1.400000in}}%
\pgfpathlineto{\pgfqpoint{2.480560in}{1.855000in}}%
\pgfpathlineto{\pgfqpoint{2.483040in}{1.050000in}}%
\pgfpathlineto{\pgfqpoint{2.484280in}{1.330000in}}%
\pgfpathlineto{\pgfqpoint{2.485520in}{0.980000in}}%
\pgfpathlineto{\pgfqpoint{2.486760in}{1.540000in}}%
\pgfpathlineto{\pgfqpoint{2.488000in}{1.190000in}}%
\pgfpathlineto{\pgfqpoint{2.489240in}{1.330000in}}%
\pgfpathlineto{\pgfqpoint{2.490480in}{1.820000in}}%
\pgfpathlineto{\pgfqpoint{2.491720in}{1.785000in}}%
\pgfpathlineto{\pgfqpoint{2.492960in}{1.365000in}}%
\pgfpathlineto{\pgfqpoint{2.494200in}{1.505000in}}%
\pgfpathlineto{\pgfqpoint{2.495440in}{1.260000in}}%
\pgfpathlineto{\pgfqpoint{2.496680in}{1.645000in}}%
\pgfpathlineto{\pgfqpoint{2.497920in}{1.050000in}}%
\pgfpathlineto{\pgfqpoint{2.499160in}{1.680000in}}%
\pgfpathlineto{\pgfqpoint{2.501640in}{1.190000in}}%
\pgfpathlineto{\pgfqpoint{2.502880in}{1.610000in}}%
\pgfpathlineto{\pgfqpoint{2.504120in}{1.435000in}}%
\pgfpathlineto{\pgfqpoint{2.505360in}{1.645000in}}%
\pgfpathlineto{\pgfqpoint{2.506600in}{1.295000in}}%
\pgfpathlineto{\pgfqpoint{2.507840in}{1.575000in}}%
\pgfpathlineto{\pgfqpoint{2.509080in}{1.400000in}}%
\pgfpathlineto{\pgfqpoint{2.510320in}{1.750000in}}%
\pgfpathlineto{\pgfqpoint{2.512800in}{1.365000in}}%
\pgfpathlineto{\pgfqpoint{2.514040in}{1.400000in}}%
\pgfpathlineto{\pgfqpoint{2.515280in}{2.135000in}}%
\pgfpathlineto{\pgfqpoint{2.520240in}{1.050000in}}%
\pgfpathlineto{\pgfqpoint{2.521480in}{1.295000in}}%
\pgfpathlineto{\pgfqpoint{2.522720in}{1.260000in}}%
\pgfpathlineto{\pgfqpoint{2.523960in}{1.295000in}}%
\pgfpathlineto{\pgfqpoint{2.525200in}{1.505000in}}%
\pgfpathlineto{\pgfqpoint{2.526440in}{1.400000in}}%
\pgfpathlineto{\pgfqpoint{2.527680in}{1.505000in}}%
\pgfpathlineto{\pgfqpoint{2.528920in}{1.505000in}}%
\pgfpathlineto{\pgfqpoint{2.530160in}{0.910000in}}%
\pgfpathlineto{\pgfqpoint{2.531400in}{1.085000in}}%
\pgfpathlineto{\pgfqpoint{2.532640in}{1.540000in}}%
\pgfpathlineto{\pgfqpoint{2.533880in}{1.330000in}}%
\pgfpathlineto{\pgfqpoint{2.535120in}{1.365000in}}%
\pgfpathlineto{\pgfqpoint{2.536360in}{1.295000in}}%
\pgfpathlineto{\pgfqpoint{2.537600in}{1.365000in}}%
\pgfpathlineto{\pgfqpoint{2.538840in}{0.805000in}}%
\pgfpathlineto{\pgfqpoint{2.540080in}{1.435000in}}%
\pgfpathlineto{\pgfqpoint{2.541320in}{1.330000in}}%
\pgfpathlineto{\pgfqpoint{2.542560in}{1.505000in}}%
\pgfpathlineto{\pgfqpoint{2.543800in}{0.945000in}}%
\pgfpathlineto{\pgfqpoint{2.545040in}{1.435000in}}%
\pgfpathlineto{\pgfqpoint{2.546280in}{1.155000in}}%
\pgfpathlineto{\pgfqpoint{2.547520in}{1.785000in}}%
\pgfpathlineto{\pgfqpoint{2.548760in}{1.295000in}}%
\pgfpathlineto{\pgfqpoint{2.550000in}{1.680000in}}%
\pgfpathlineto{\pgfqpoint{2.551240in}{1.365000in}}%
\pgfpathlineto{\pgfqpoint{2.552480in}{1.435000in}}%
\pgfpathlineto{\pgfqpoint{2.553720in}{1.330000in}}%
\pgfpathlineto{\pgfqpoint{2.554960in}{1.400000in}}%
\pgfpathlineto{\pgfqpoint{2.556200in}{0.980000in}}%
\pgfpathlineto{\pgfqpoint{2.557440in}{1.610000in}}%
\pgfpathlineto{\pgfqpoint{2.558680in}{1.470000in}}%
\pgfpathlineto{\pgfqpoint{2.559920in}{1.680000in}}%
\pgfpathlineto{\pgfqpoint{2.562400in}{1.085000in}}%
\pgfpathlineto{\pgfqpoint{2.564880in}{1.540000in}}%
\pgfpathlineto{\pgfqpoint{2.566120in}{1.260000in}}%
\pgfpathlineto{\pgfqpoint{2.567360in}{1.610000in}}%
\pgfpathlineto{\pgfqpoint{2.568600in}{1.260000in}}%
\pgfpathlineto{\pgfqpoint{2.569840in}{1.260000in}}%
\pgfpathlineto{\pgfqpoint{2.571080in}{1.505000in}}%
\pgfpathlineto{\pgfqpoint{2.572320in}{1.260000in}}%
\pgfpathlineto{\pgfqpoint{2.574800in}{1.610000in}}%
\pgfpathlineto{\pgfqpoint{2.576040in}{0.770000in}}%
\pgfpathlineto{\pgfqpoint{2.577280in}{0.875000in}}%
\pgfpathlineto{\pgfqpoint{2.578520in}{1.470000in}}%
\pgfpathlineto{\pgfqpoint{2.579760in}{1.435000in}}%
\pgfpathlineto{\pgfqpoint{2.582240in}{1.855000in}}%
\pgfpathlineto{\pgfqpoint{2.584720in}{1.540000in}}%
\pgfpathlineto{\pgfqpoint{2.585960in}{1.785000in}}%
\pgfpathlineto{\pgfqpoint{2.587200in}{1.715000in}}%
\pgfpathlineto{\pgfqpoint{2.588440in}{1.260000in}}%
\pgfpathlineto{\pgfqpoint{2.589680in}{1.645000in}}%
\pgfpathlineto{\pgfqpoint{2.590920in}{1.295000in}}%
\pgfpathlineto{\pgfqpoint{2.592160in}{1.330000in}}%
\pgfpathlineto{\pgfqpoint{2.593400in}{1.295000in}}%
\pgfpathlineto{\pgfqpoint{2.594640in}{1.785000in}}%
\pgfpathlineto{\pgfqpoint{2.597120in}{1.610000in}}%
\pgfpathlineto{\pgfqpoint{2.598360in}{1.785000in}}%
\pgfpathlineto{\pgfqpoint{2.599600in}{1.330000in}}%
\pgfpathlineto{\pgfqpoint{2.600840in}{1.400000in}}%
\pgfpathlineto{\pgfqpoint{2.602080in}{1.365000in}}%
\pgfpathlineto{\pgfqpoint{2.603320in}{1.120000in}}%
\pgfpathlineto{\pgfqpoint{2.604560in}{1.400000in}}%
\pgfpathlineto{\pgfqpoint{2.605800in}{0.735000in}}%
\pgfpathlineto{\pgfqpoint{2.607040in}{1.470000in}}%
\pgfpathlineto{\pgfqpoint{2.608280in}{1.505000in}}%
\pgfpathlineto{\pgfqpoint{2.609520in}{1.190000in}}%
\pgfpathlineto{\pgfqpoint{2.612000in}{1.400000in}}%
\pgfpathlineto{\pgfqpoint{2.613240in}{1.365000in}}%
\pgfpathlineto{\pgfqpoint{2.614480in}{1.540000in}}%
\pgfpathlineto{\pgfqpoint{2.615720in}{1.190000in}}%
\pgfpathlineto{\pgfqpoint{2.616960in}{1.610000in}}%
\pgfpathlineto{\pgfqpoint{2.619440in}{1.120000in}}%
\pgfpathlineto{\pgfqpoint{2.620680in}{1.505000in}}%
\pgfpathlineto{\pgfqpoint{2.621920in}{1.505000in}}%
\pgfpathlineto{\pgfqpoint{2.623160in}{1.225000in}}%
\pgfpathlineto{\pgfqpoint{2.624400in}{1.435000in}}%
\pgfpathlineto{\pgfqpoint{2.625640in}{1.120000in}}%
\pgfpathlineto{\pgfqpoint{2.628120in}{1.365000in}}%
\pgfpathlineto{\pgfqpoint{2.629360in}{1.505000in}}%
\pgfpathlineto{\pgfqpoint{2.630600in}{1.365000in}}%
\pgfpathlineto{\pgfqpoint{2.631840in}{1.680000in}}%
\pgfpathlineto{\pgfqpoint{2.633080in}{1.225000in}}%
\pgfpathlineto{\pgfqpoint{2.635560in}{1.715000in}}%
\pgfpathlineto{\pgfqpoint{2.636800in}{1.750000in}}%
\pgfpathlineto{\pgfqpoint{2.638040in}{1.645000in}}%
\pgfpathlineto{\pgfqpoint{2.639280in}{1.820000in}}%
\pgfpathlineto{\pgfqpoint{2.640520in}{1.750000in}}%
\pgfpathlineto{\pgfqpoint{2.641760in}{1.365000in}}%
\pgfpathlineto{\pgfqpoint{2.643000in}{1.610000in}}%
\pgfpathlineto{\pgfqpoint{2.645480in}{1.365000in}}%
\pgfpathlineto{\pgfqpoint{2.646720in}{1.505000in}}%
\pgfpathlineto{\pgfqpoint{2.647960in}{1.470000in}}%
\pgfpathlineto{\pgfqpoint{2.649200in}{2.240000in}}%
\pgfpathlineto{\pgfqpoint{2.651680in}{1.610000in}}%
\pgfpathlineto{\pgfqpoint{2.652920in}{1.610000in}}%
\pgfpathlineto{\pgfqpoint{2.654160in}{1.295000in}}%
\pgfpathlineto{\pgfqpoint{2.655400in}{1.435000in}}%
\pgfpathlineto{\pgfqpoint{2.656640in}{1.400000in}}%
\pgfpathlineto{\pgfqpoint{2.659120in}{1.610000in}}%
\pgfpathlineto{\pgfqpoint{2.660360in}{1.295000in}}%
\pgfpathlineto{\pgfqpoint{2.662840in}{1.890000in}}%
\pgfpathlineto{\pgfqpoint{2.664080in}{1.785000in}}%
\pgfpathlineto{\pgfqpoint{2.665320in}{1.855000in}}%
\pgfpathlineto{\pgfqpoint{2.667800in}{1.295000in}}%
\pgfpathlineto{\pgfqpoint{2.670280in}{1.680000in}}%
\pgfpathlineto{\pgfqpoint{2.671520in}{1.050000in}}%
\pgfpathlineto{\pgfqpoint{2.674000in}{1.330000in}}%
\pgfpathlineto{\pgfqpoint{2.675240in}{1.400000in}}%
\pgfpathlineto{\pgfqpoint{2.676480in}{1.890000in}}%
\pgfpathlineto{\pgfqpoint{2.677720in}{1.330000in}}%
\pgfpathlineto{\pgfqpoint{2.678960in}{1.330000in}}%
\pgfpathlineto{\pgfqpoint{2.681440in}{0.945000in}}%
\pgfpathlineto{\pgfqpoint{2.682680in}{1.645000in}}%
\pgfpathlineto{\pgfqpoint{2.683920in}{1.575000in}}%
\pgfpathlineto{\pgfqpoint{2.685160in}{1.925000in}}%
\pgfpathlineto{\pgfqpoint{2.687640in}{1.750000in}}%
\pgfpathlineto{\pgfqpoint{2.688880in}{1.225000in}}%
\pgfpathlineto{\pgfqpoint{2.691360in}{1.645000in}}%
\pgfpathlineto{\pgfqpoint{2.692600in}{1.855000in}}%
\pgfpathlineto{\pgfqpoint{2.693840in}{1.470000in}}%
\pgfpathlineto{\pgfqpoint{2.695080in}{1.715000in}}%
\pgfpathlineto{\pgfqpoint{2.696320in}{1.680000in}}%
\pgfpathlineto{\pgfqpoint{2.697560in}{1.855000in}}%
\pgfpathlineto{\pgfqpoint{2.700040in}{1.575000in}}%
\pgfpathlineto{\pgfqpoint{2.701280in}{1.505000in}}%
\pgfpathlineto{\pgfqpoint{2.702520in}{2.100000in}}%
\pgfpathlineto{\pgfqpoint{2.703760in}{1.575000in}}%
\pgfpathlineto{\pgfqpoint{2.705000in}{1.575000in}}%
\pgfpathlineto{\pgfqpoint{2.706240in}{1.435000in}}%
\pgfpathlineto{\pgfqpoint{2.707480in}{1.715000in}}%
\pgfpathlineto{\pgfqpoint{2.708720in}{1.575000in}}%
\pgfpathlineto{\pgfqpoint{2.709960in}{1.680000in}}%
\pgfpathlineto{\pgfqpoint{2.712440in}{1.505000in}}%
\pgfpathlineto{\pgfqpoint{2.713680in}{1.610000in}}%
\pgfpathlineto{\pgfqpoint{2.714920in}{1.610000in}}%
\pgfpathlineto{\pgfqpoint{2.716160in}{1.925000in}}%
\pgfpathlineto{\pgfqpoint{2.718640in}{1.120000in}}%
\pgfpathlineto{\pgfqpoint{2.719880in}{1.890000in}}%
\pgfpathlineto{\pgfqpoint{2.721120in}{1.540000in}}%
\pgfpathlineto{\pgfqpoint{2.722360in}{1.715000in}}%
\pgfpathlineto{\pgfqpoint{2.723600in}{1.715000in}}%
\pgfpathlineto{\pgfqpoint{2.724840in}{1.330000in}}%
\pgfpathlineto{\pgfqpoint{2.726080in}{1.820000in}}%
\pgfpathlineto{\pgfqpoint{2.728560in}{1.295000in}}%
\pgfpathlineto{\pgfqpoint{2.729800in}{1.435000in}}%
\pgfpathlineto{\pgfqpoint{2.731040in}{1.260000in}}%
\pgfpathlineto{\pgfqpoint{2.734760in}{1.645000in}}%
\pgfpathlineto{\pgfqpoint{2.738480in}{0.980000in}}%
\pgfpathlineto{\pgfqpoint{2.739720in}{1.680000in}}%
\pgfpathlineto{\pgfqpoint{2.740960in}{1.750000in}}%
\pgfpathlineto{\pgfqpoint{2.743440in}{1.365000in}}%
\pgfpathlineto{\pgfqpoint{2.744680in}{1.085000in}}%
\pgfpathlineto{\pgfqpoint{2.745920in}{1.435000in}}%
\pgfpathlineto{\pgfqpoint{2.748400in}{1.155000in}}%
\pgfpathlineto{\pgfqpoint{2.750880in}{1.645000in}}%
\pgfpathlineto{\pgfqpoint{2.752120in}{1.225000in}}%
\pgfpathlineto{\pgfqpoint{2.754600in}{1.505000in}}%
\pgfpathlineto{\pgfqpoint{2.755840in}{1.400000in}}%
\pgfpathlineto{\pgfqpoint{2.757080in}{1.435000in}}%
\pgfpathlineto{\pgfqpoint{2.758320in}{1.540000in}}%
\pgfpathlineto{\pgfqpoint{2.759560in}{1.540000in}}%
\pgfpathlineto{\pgfqpoint{2.760800in}{1.680000in}}%
\pgfpathlineto{\pgfqpoint{2.763280in}{1.295000in}}%
\pgfpathlineto{\pgfqpoint{2.764520in}{1.260000in}}%
\pgfpathlineto{\pgfqpoint{2.765760in}{1.540000in}}%
\pgfpathlineto{\pgfqpoint{2.767000in}{1.050000in}}%
\pgfpathlineto{\pgfqpoint{2.768240in}{1.155000in}}%
\pgfpathlineto{\pgfqpoint{2.769480in}{1.505000in}}%
\pgfpathlineto{\pgfqpoint{2.770720in}{1.225000in}}%
\pgfpathlineto{\pgfqpoint{2.771960in}{1.750000in}}%
\pgfpathlineto{\pgfqpoint{2.773200in}{1.225000in}}%
\pgfpathlineto{\pgfqpoint{2.774440in}{1.260000in}}%
\pgfpathlineto{\pgfqpoint{2.775680in}{1.470000in}}%
\pgfpathlineto{\pgfqpoint{2.776920in}{1.330000in}}%
\pgfpathlineto{\pgfqpoint{2.778160in}{1.715000in}}%
\pgfpathlineto{\pgfqpoint{2.779400in}{1.470000in}}%
\pgfpathlineto{\pgfqpoint{2.781880in}{1.890000in}}%
\pgfpathlineto{\pgfqpoint{2.784360in}{1.260000in}}%
\pgfpathlineto{\pgfqpoint{2.785600in}{1.190000in}}%
\pgfpathlineto{\pgfqpoint{2.786840in}{1.505000in}}%
\pgfpathlineto{\pgfqpoint{2.790560in}{1.050000in}}%
\pgfpathlineto{\pgfqpoint{2.791800in}{1.505000in}}%
\pgfpathlineto{\pgfqpoint{2.793040in}{0.875000in}}%
\pgfpathlineto{\pgfqpoint{2.794280in}{1.330000in}}%
\pgfpathlineto{\pgfqpoint{2.795520in}{0.980000in}}%
\pgfpathlineto{\pgfqpoint{2.798000in}{1.715000in}}%
\pgfpathlineto{\pgfqpoint{2.800480in}{0.910000in}}%
\pgfpathlineto{\pgfqpoint{2.801720in}{1.330000in}}%
\pgfpathlineto{\pgfqpoint{2.802960in}{1.120000in}}%
\pgfpathlineto{\pgfqpoint{2.804200in}{1.645000in}}%
\pgfpathlineto{\pgfqpoint{2.806680in}{1.190000in}}%
\pgfpathlineto{\pgfqpoint{2.807920in}{1.470000in}}%
\pgfpathlineto{\pgfqpoint{2.809160in}{1.330000in}}%
\pgfpathlineto{\pgfqpoint{2.810400in}{1.750000in}}%
\pgfpathlineto{\pgfqpoint{2.811640in}{1.120000in}}%
\pgfpathlineto{\pgfqpoint{2.812880in}{1.540000in}}%
\pgfpathlineto{\pgfqpoint{2.814120in}{1.120000in}}%
\pgfpathlineto{\pgfqpoint{2.815360in}{1.785000in}}%
\pgfpathlineto{\pgfqpoint{2.816600in}{1.610000in}}%
\pgfpathlineto{\pgfqpoint{2.817840in}{1.785000in}}%
\pgfpathlineto{\pgfqpoint{2.819080in}{1.330000in}}%
\pgfpathlineto{\pgfqpoint{2.820320in}{1.575000in}}%
\pgfpathlineto{\pgfqpoint{2.822800in}{1.050000in}}%
\pgfpathlineto{\pgfqpoint{2.824040in}{1.715000in}}%
\pgfpathlineto{\pgfqpoint{2.825280in}{0.805000in}}%
\pgfpathlineto{\pgfqpoint{2.826520in}{1.260000in}}%
\pgfpathlineto{\pgfqpoint{2.827760in}{1.050000in}}%
\pgfpathlineto{\pgfqpoint{2.830240in}{1.575000in}}%
\pgfpathlineto{\pgfqpoint{2.831480in}{1.575000in}}%
\pgfpathlineto{\pgfqpoint{2.832720in}{1.225000in}}%
\pgfpathlineto{\pgfqpoint{2.833960in}{1.855000in}}%
\pgfpathlineto{\pgfqpoint{2.835200in}{1.050000in}}%
\pgfpathlineto{\pgfqpoint{2.837680in}{1.400000in}}%
\pgfpathlineto{\pgfqpoint{2.838920in}{0.980000in}}%
\pgfpathlineto{\pgfqpoint{2.840160in}{1.435000in}}%
\pgfpathlineto{\pgfqpoint{2.841400in}{1.400000in}}%
\pgfpathlineto{\pgfqpoint{2.843880in}{1.610000in}}%
\pgfpathlineto{\pgfqpoint{2.845120in}{1.400000in}}%
\pgfpathlineto{\pgfqpoint{2.846360in}{0.840000in}}%
\pgfpathlineto{\pgfqpoint{2.848840in}{1.750000in}}%
\pgfpathlineto{\pgfqpoint{2.850080in}{0.980000in}}%
\pgfpathlineto{\pgfqpoint{2.851320in}{1.715000in}}%
\pgfpathlineto{\pgfqpoint{2.852560in}{1.190000in}}%
\pgfpathlineto{\pgfqpoint{2.855040in}{1.330000in}}%
\pgfpathlineto{\pgfqpoint{2.856280in}{1.085000in}}%
\pgfpathlineto{\pgfqpoint{2.857520in}{1.540000in}}%
\pgfpathlineto{\pgfqpoint{2.858760in}{1.225000in}}%
\pgfpathlineto{\pgfqpoint{2.860000in}{1.330000in}}%
\pgfpathlineto{\pgfqpoint{2.861240in}{1.540000in}}%
\pgfpathlineto{\pgfqpoint{2.863720in}{1.260000in}}%
\pgfpathlineto{\pgfqpoint{2.866200in}{1.085000in}}%
\pgfpathlineto{\pgfqpoint{2.867440in}{1.575000in}}%
\pgfpathlineto{\pgfqpoint{2.868680in}{1.190000in}}%
\pgfpathlineto{\pgfqpoint{2.869920in}{1.155000in}}%
\pgfpathlineto{\pgfqpoint{2.871160in}{1.470000in}}%
\pgfpathlineto{\pgfqpoint{2.872400in}{1.365000in}}%
\pgfpathlineto{\pgfqpoint{2.874880in}{1.960000in}}%
\pgfpathlineto{\pgfqpoint{2.876120in}{1.820000in}}%
\pgfpathlineto{\pgfqpoint{2.877360in}{1.120000in}}%
\pgfpathlineto{\pgfqpoint{2.879840in}{1.785000in}}%
\pgfpathlineto{\pgfqpoint{2.881080in}{1.820000in}}%
\pgfpathlineto{\pgfqpoint{2.882320in}{1.715000in}}%
\pgfpathlineto{\pgfqpoint{2.884800in}{1.400000in}}%
\pgfpathlineto{\pgfqpoint{2.886040in}{1.645000in}}%
\pgfpathlineto{\pgfqpoint{2.887280in}{1.365000in}}%
\pgfpathlineto{\pgfqpoint{2.888520in}{1.575000in}}%
\pgfpathlineto{\pgfqpoint{2.889760in}{1.400000in}}%
\pgfpathlineto{\pgfqpoint{2.891000in}{1.400000in}}%
\pgfpathlineto{\pgfqpoint{2.892240in}{1.085000in}}%
\pgfpathlineto{\pgfqpoint{2.893480in}{1.260000in}}%
\pgfpathlineto{\pgfqpoint{2.894720in}{1.155000in}}%
\pgfpathlineto{\pgfqpoint{2.898440in}{1.890000in}}%
\pgfpathlineto{\pgfqpoint{2.899680in}{1.295000in}}%
\pgfpathlineto{\pgfqpoint{2.900920in}{1.365000in}}%
\pgfpathlineto{\pgfqpoint{2.903400in}{1.085000in}}%
\pgfpathlineto{\pgfqpoint{2.904640in}{1.680000in}}%
\pgfpathlineto{\pgfqpoint{2.905880in}{1.295000in}}%
\pgfpathlineto{\pgfqpoint{2.907120in}{1.645000in}}%
\pgfpathlineto{\pgfqpoint{2.908360in}{1.190000in}}%
\pgfpathlineto{\pgfqpoint{2.909600in}{1.680000in}}%
\pgfpathlineto{\pgfqpoint{2.910840in}{1.575000in}}%
\pgfpathlineto{\pgfqpoint{2.912080in}{1.680000in}}%
\pgfpathlineto{\pgfqpoint{2.913320in}{1.260000in}}%
\pgfpathlineto{\pgfqpoint{2.914560in}{1.225000in}}%
\pgfpathlineto{\pgfqpoint{2.917040in}{1.540000in}}%
\pgfpathlineto{\pgfqpoint{2.918280in}{1.015000in}}%
\pgfpathlineto{\pgfqpoint{2.920760in}{1.540000in}}%
\pgfpathlineto{\pgfqpoint{2.922000in}{1.820000in}}%
\pgfpathlineto{\pgfqpoint{2.923240in}{1.610000in}}%
\pgfpathlineto{\pgfqpoint{2.924480in}{1.855000in}}%
\pgfpathlineto{\pgfqpoint{2.925720in}{1.260000in}}%
\pgfpathlineto{\pgfqpoint{2.928200in}{1.645000in}}%
\pgfpathlineto{\pgfqpoint{2.929440in}{1.575000in}}%
\pgfpathlineto{\pgfqpoint{2.930680in}{1.295000in}}%
\pgfpathlineto{\pgfqpoint{2.931920in}{1.680000in}}%
\pgfpathlineto{\pgfqpoint{2.933160in}{1.365000in}}%
\pgfpathlineto{\pgfqpoint{2.934400in}{1.925000in}}%
\pgfpathlineto{\pgfqpoint{2.935640in}{1.610000in}}%
\pgfpathlineto{\pgfqpoint{2.936880in}{1.820000in}}%
\pgfpathlineto{\pgfqpoint{2.939360in}{0.910000in}}%
\pgfpathlineto{\pgfqpoint{2.941840in}{1.505000in}}%
\pgfpathlineto{\pgfqpoint{2.943080in}{1.645000in}}%
\pgfpathlineto{\pgfqpoint{2.944320in}{1.295000in}}%
\pgfpathlineto{\pgfqpoint{2.946800in}{1.680000in}}%
\pgfpathlineto{\pgfqpoint{2.949280in}{1.015000in}}%
\pgfpathlineto{\pgfqpoint{2.950520in}{1.750000in}}%
\pgfpathlineto{\pgfqpoint{2.951760in}{1.750000in}}%
\pgfpathlineto{\pgfqpoint{2.954240in}{1.400000in}}%
\pgfpathlineto{\pgfqpoint{2.955480in}{1.190000in}}%
\pgfpathlineto{\pgfqpoint{2.957960in}{1.680000in}}%
\pgfpathlineto{\pgfqpoint{2.960440in}{1.190000in}}%
\pgfpathlineto{\pgfqpoint{2.964160in}{1.820000in}}%
\pgfpathlineto{\pgfqpoint{2.965400in}{1.120000in}}%
\pgfpathlineto{\pgfqpoint{2.966640in}{1.575000in}}%
\pgfpathlineto{\pgfqpoint{2.969120in}{1.225000in}}%
\pgfpathlineto{\pgfqpoint{2.971600in}{1.365000in}}%
\pgfpathlineto{\pgfqpoint{2.972840in}{1.610000in}}%
\pgfpathlineto{\pgfqpoint{2.974080in}{1.505000in}}%
\pgfpathlineto{\pgfqpoint{2.975320in}{1.505000in}}%
\pgfpathlineto{\pgfqpoint{2.976560in}{1.120000in}}%
\pgfpathlineto{\pgfqpoint{2.977800in}{1.470000in}}%
\pgfpathlineto{\pgfqpoint{2.979040in}{1.155000in}}%
\pgfpathlineto{\pgfqpoint{2.980280in}{1.260000in}}%
\pgfpathlineto{\pgfqpoint{2.982760in}{1.540000in}}%
\pgfpathlineto{\pgfqpoint{2.984000in}{1.155000in}}%
\pgfpathlineto{\pgfqpoint{2.985240in}{1.435000in}}%
\pgfpathlineto{\pgfqpoint{2.986480in}{1.225000in}}%
\pgfpathlineto{\pgfqpoint{2.987720in}{1.260000in}}%
\pgfpathlineto{\pgfqpoint{2.988960in}{0.875000in}}%
\pgfpathlineto{\pgfqpoint{2.990200in}{1.400000in}}%
\pgfpathlineto{\pgfqpoint{2.991440in}{1.400000in}}%
\pgfpathlineto{\pgfqpoint{2.992680in}{1.715000in}}%
\pgfpathlineto{\pgfqpoint{2.993920in}{1.330000in}}%
\pgfpathlineto{\pgfqpoint{2.995160in}{1.505000in}}%
\pgfpathlineto{\pgfqpoint{2.996400in}{1.400000in}}%
\pgfpathlineto{\pgfqpoint{2.997640in}{1.435000in}}%
\pgfpathlineto{\pgfqpoint{2.998880in}{1.610000in}}%
\pgfpathlineto{\pgfqpoint{3.000120in}{1.400000in}}%
\pgfpathlineto{\pgfqpoint{3.001360in}{1.470000in}}%
\pgfpathlineto{\pgfqpoint{3.002600in}{1.435000in}}%
\pgfpathlineto{\pgfqpoint{3.003840in}{1.890000in}}%
\pgfpathlineto{\pgfqpoint{3.006320in}{1.610000in}}%
\pgfpathlineto{\pgfqpoint{3.007560in}{1.925000in}}%
\pgfpathlineto{\pgfqpoint{3.010040in}{1.225000in}}%
\pgfpathlineto{\pgfqpoint{3.011280in}{1.295000in}}%
\pgfpathlineto{\pgfqpoint{3.012520in}{1.295000in}}%
\pgfpathlineto{\pgfqpoint{3.013760in}{1.260000in}}%
\pgfpathlineto{\pgfqpoint{3.015000in}{1.400000in}}%
\pgfpathlineto{\pgfqpoint{3.016240in}{1.680000in}}%
\pgfpathlineto{\pgfqpoint{3.017480in}{1.330000in}}%
\pgfpathlineto{\pgfqpoint{3.018720in}{1.330000in}}%
\pgfpathlineto{\pgfqpoint{3.019960in}{1.505000in}}%
\pgfpathlineto{\pgfqpoint{3.021200in}{1.295000in}}%
\pgfpathlineto{\pgfqpoint{3.022440in}{1.505000in}}%
\pgfpathlineto{\pgfqpoint{3.023680in}{1.330000in}}%
\pgfpathlineto{\pgfqpoint{3.024920in}{1.365000in}}%
\pgfpathlineto{\pgfqpoint{3.026160in}{1.470000in}}%
\pgfpathlineto{\pgfqpoint{3.027400in}{1.085000in}}%
\pgfpathlineto{\pgfqpoint{3.029880in}{1.680000in}}%
\pgfpathlineto{\pgfqpoint{3.031120in}{1.575000in}}%
\pgfpathlineto{\pgfqpoint{3.032360in}{1.085000in}}%
\pgfpathlineto{\pgfqpoint{3.033600in}{1.505000in}}%
\pgfpathlineto{\pgfqpoint{3.034840in}{1.225000in}}%
\pgfpathlineto{\pgfqpoint{3.036080in}{1.470000in}}%
\pgfpathlineto{\pgfqpoint{3.037320in}{1.120000in}}%
\pgfpathlineto{\pgfqpoint{3.038560in}{1.365000in}}%
\pgfpathlineto{\pgfqpoint{3.039800in}{1.330000in}}%
\pgfpathlineto{\pgfqpoint{3.041040in}{1.015000in}}%
\pgfpathlineto{\pgfqpoint{3.042280in}{1.295000in}}%
\pgfpathlineto{\pgfqpoint{3.043520in}{1.260000in}}%
\pgfpathlineto{\pgfqpoint{3.044760in}{1.610000in}}%
\pgfpathlineto{\pgfqpoint{3.047240in}{1.225000in}}%
\pgfpathlineto{\pgfqpoint{3.048480in}{1.925000in}}%
\pgfpathlineto{\pgfqpoint{3.049720in}{1.155000in}}%
\pgfpathlineto{\pgfqpoint{3.052200in}{1.470000in}}%
\pgfpathlineto{\pgfqpoint{3.053440in}{1.540000in}}%
\pgfpathlineto{\pgfqpoint{3.054680in}{1.540000in}}%
\pgfpathlineto{\pgfqpoint{3.055920in}{1.330000in}}%
\pgfpathlineto{\pgfqpoint{3.057160in}{1.575000in}}%
\pgfpathlineto{\pgfqpoint{3.058400in}{1.225000in}}%
\pgfpathlineto{\pgfqpoint{3.059640in}{1.190000in}}%
\pgfpathlineto{\pgfqpoint{3.060880in}{1.610000in}}%
\pgfpathlineto{\pgfqpoint{3.062120in}{1.260000in}}%
\pgfpathlineto{\pgfqpoint{3.063360in}{1.365000in}}%
\pgfpathlineto{\pgfqpoint{3.064600in}{1.260000in}}%
\pgfpathlineto{\pgfqpoint{3.065840in}{1.610000in}}%
\pgfpathlineto{\pgfqpoint{3.067080in}{1.330000in}}%
\pgfpathlineto{\pgfqpoint{3.068320in}{1.400000in}}%
\pgfpathlineto{\pgfqpoint{3.069560in}{1.295000in}}%
\pgfpathlineto{\pgfqpoint{3.070800in}{1.400000in}}%
\pgfpathlineto{\pgfqpoint{3.072040in}{1.715000in}}%
\pgfpathlineto{\pgfqpoint{3.073280in}{1.190000in}}%
\pgfpathlineto{\pgfqpoint{3.074520in}{1.820000in}}%
\pgfpathlineto{\pgfqpoint{3.075760in}{1.330000in}}%
\pgfpathlineto{\pgfqpoint{3.077000in}{1.505000in}}%
\pgfpathlineto{\pgfqpoint{3.078240in}{1.330000in}}%
\pgfpathlineto{\pgfqpoint{3.079480in}{1.610000in}}%
\pgfpathlineto{\pgfqpoint{3.080720in}{1.120000in}}%
\pgfpathlineto{\pgfqpoint{3.083200in}{1.470000in}}%
\pgfpathlineto{\pgfqpoint{3.084440in}{1.085000in}}%
\pgfpathlineto{\pgfqpoint{3.085680in}{1.400000in}}%
\pgfpathlineto{\pgfqpoint{3.086920in}{1.365000in}}%
\pgfpathlineto{\pgfqpoint{3.088160in}{1.190000in}}%
\pgfpathlineto{\pgfqpoint{3.089400in}{1.225000in}}%
\pgfpathlineto{\pgfqpoint{3.090640in}{1.050000in}}%
\pgfpathlineto{\pgfqpoint{3.091880in}{0.700000in}}%
\pgfpathlineto{\pgfqpoint{3.095600in}{1.610000in}}%
\pgfpathlineto{\pgfqpoint{3.096840in}{1.470000in}}%
\pgfpathlineto{\pgfqpoint{3.099320in}{1.855000in}}%
\pgfpathlineto{\pgfqpoint{3.100560in}{1.715000in}}%
\pgfpathlineto{\pgfqpoint{3.101800in}{1.295000in}}%
\pgfpathlineto{\pgfqpoint{3.103040in}{1.855000in}}%
\pgfpathlineto{\pgfqpoint{3.104280in}{1.085000in}}%
\pgfpathlineto{\pgfqpoint{3.105520in}{1.715000in}}%
\pgfpathlineto{\pgfqpoint{3.106760in}{1.645000in}}%
\pgfpathlineto{\pgfqpoint{3.108000in}{1.295000in}}%
\pgfpathlineto{\pgfqpoint{3.110480in}{1.540000in}}%
\pgfpathlineto{\pgfqpoint{3.111720in}{1.470000in}}%
\pgfpathlineto{\pgfqpoint{3.112960in}{1.260000in}}%
\pgfpathlineto{\pgfqpoint{3.114200in}{1.540000in}}%
\pgfpathlineto{\pgfqpoint{3.115440in}{1.015000in}}%
\pgfpathlineto{\pgfqpoint{3.116680in}{1.365000in}}%
\pgfpathlineto{\pgfqpoint{3.117920in}{1.260000in}}%
\pgfpathlineto{\pgfqpoint{3.119160in}{1.610000in}}%
\pgfpathlineto{\pgfqpoint{3.122880in}{1.085000in}}%
\pgfpathlineto{\pgfqpoint{3.124120in}{1.190000in}}%
\pgfpathlineto{\pgfqpoint{3.126600in}{1.715000in}}%
\pgfpathlineto{\pgfqpoint{3.129080in}{1.995000in}}%
\pgfpathlineto{\pgfqpoint{3.130320in}{1.750000in}}%
\pgfpathlineto{\pgfqpoint{3.131560in}{2.100000in}}%
\pgfpathlineto{\pgfqpoint{3.132800in}{1.680000in}}%
\pgfpathlineto{\pgfqpoint{3.134040in}{1.750000in}}%
\pgfpathlineto{\pgfqpoint{3.135280in}{1.645000in}}%
\pgfpathlineto{\pgfqpoint{3.136520in}{1.295000in}}%
\pgfpathlineto{\pgfqpoint{3.137760in}{1.435000in}}%
\pgfpathlineto{\pgfqpoint{3.139000in}{1.015000in}}%
\pgfpathlineto{\pgfqpoint{3.141480in}{1.820000in}}%
\pgfpathlineto{\pgfqpoint{3.143960in}{1.155000in}}%
\pgfpathlineto{\pgfqpoint{3.145200in}{1.470000in}}%
\pgfpathlineto{\pgfqpoint{3.146440in}{1.155000in}}%
\pgfpathlineto{\pgfqpoint{3.147680in}{1.785000in}}%
\pgfpathlineto{\pgfqpoint{3.148920in}{1.190000in}}%
\pgfpathlineto{\pgfqpoint{3.151400in}{1.470000in}}%
\pgfpathlineto{\pgfqpoint{3.152640in}{1.260000in}}%
\pgfpathlineto{\pgfqpoint{3.153880in}{1.505000in}}%
\pgfpathlineto{\pgfqpoint{3.156360in}{1.400000in}}%
\pgfpathlineto{\pgfqpoint{3.158840in}{0.980000in}}%
\pgfpathlineto{\pgfqpoint{3.160080in}{1.750000in}}%
\pgfpathlineto{\pgfqpoint{3.161320in}{1.365000in}}%
\pgfpathlineto{\pgfqpoint{3.162560in}{1.610000in}}%
\pgfpathlineto{\pgfqpoint{3.163800in}{1.260000in}}%
\pgfpathlineto{\pgfqpoint{3.165040in}{1.295000in}}%
\pgfpathlineto{\pgfqpoint{3.166280in}{1.645000in}}%
\pgfpathlineto{\pgfqpoint{3.167520in}{1.505000in}}%
\pgfpathlineto{\pgfqpoint{3.168760in}{1.750000in}}%
\pgfpathlineto{\pgfqpoint{3.171240in}{1.225000in}}%
\pgfpathlineto{\pgfqpoint{3.173720in}{1.645000in}}%
\pgfpathlineto{\pgfqpoint{3.174960in}{1.295000in}}%
\pgfpathlineto{\pgfqpoint{3.176200in}{1.295000in}}%
\pgfpathlineto{\pgfqpoint{3.177440in}{1.540000in}}%
\pgfpathlineto{\pgfqpoint{3.178680in}{1.085000in}}%
\pgfpathlineto{\pgfqpoint{3.179920in}{1.435000in}}%
\pgfpathlineto{\pgfqpoint{3.181160in}{1.400000in}}%
\pgfpathlineto{\pgfqpoint{3.182400in}{1.330000in}}%
\pgfpathlineto{\pgfqpoint{3.183640in}{1.155000in}}%
\pgfpathlineto{\pgfqpoint{3.184880in}{1.855000in}}%
\pgfpathlineto{\pgfqpoint{3.187360in}{1.260000in}}%
\pgfpathlineto{\pgfqpoint{3.188600in}{1.785000in}}%
\pgfpathlineto{\pgfqpoint{3.189840in}{1.715000in}}%
\pgfpathlineto{\pgfqpoint{3.191080in}{1.330000in}}%
\pgfpathlineto{\pgfqpoint{3.193560in}{1.715000in}}%
\pgfpathlineto{\pgfqpoint{3.194800in}{1.365000in}}%
\pgfpathlineto{\pgfqpoint{3.196040in}{1.505000in}}%
\pgfpathlineto{\pgfqpoint{3.197280in}{1.435000in}}%
\pgfpathlineto{\pgfqpoint{3.198520in}{1.435000in}}%
\pgfpathlineto{\pgfqpoint{3.199760in}{1.540000in}}%
\pgfpathlineto{\pgfqpoint{3.201000in}{1.120000in}}%
\pgfpathlineto{\pgfqpoint{3.203480in}{1.995000in}}%
\pgfpathlineto{\pgfqpoint{3.204720in}{1.155000in}}%
\pgfpathlineto{\pgfqpoint{3.205960in}{1.785000in}}%
\pgfpathlineto{\pgfqpoint{3.207200in}{1.365000in}}%
\pgfpathlineto{\pgfqpoint{3.208440in}{1.505000in}}%
\pgfpathlineto{\pgfqpoint{3.209680in}{1.155000in}}%
\pgfpathlineto{\pgfqpoint{3.210920in}{1.750000in}}%
\pgfpathlineto{\pgfqpoint{3.212160in}{1.295000in}}%
\pgfpathlineto{\pgfqpoint{3.214640in}{1.995000in}}%
\pgfpathlineto{\pgfqpoint{3.215880in}{1.400000in}}%
\pgfpathlineto{\pgfqpoint{3.217120in}{1.820000in}}%
\pgfpathlineto{\pgfqpoint{3.218360in}{1.225000in}}%
\pgfpathlineto{\pgfqpoint{3.219600in}{1.330000in}}%
\pgfpathlineto{\pgfqpoint{3.220840in}{1.540000in}}%
\pgfpathlineto{\pgfqpoint{3.222080in}{1.540000in}}%
\pgfpathlineto{\pgfqpoint{3.224560in}{1.400000in}}%
\pgfpathlineto{\pgfqpoint{3.225800in}{1.645000in}}%
\pgfpathlineto{\pgfqpoint{3.228280in}{1.085000in}}%
\pgfpathlineto{\pgfqpoint{3.229520in}{1.575000in}}%
\pgfpathlineto{\pgfqpoint{3.230760in}{1.540000in}}%
\pgfpathlineto{\pgfqpoint{3.233240in}{1.085000in}}%
\pgfpathlineto{\pgfqpoint{3.235720in}{1.540000in}}%
\pgfpathlineto{\pgfqpoint{3.236960in}{1.260000in}}%
\pgfpathlineto{\pgfqpoint{3.239440in}{1.610000in}}%
\pgfpathlineto{\pgfqpoint{3.240680in}{1.225000in}}%
\pgfpathlineto{\pgfqpoint{3.241920in}{1.225000in}}%
\pgfpathlineto{\pgfqpoint{3.243160in}{1.330000in}}%
\pgfpathlineto{\pgfqpoint{3.245640in}{1.225000in}}%
\pgfpathlineto{\pgfqpoint{3.246880in}{1.295000in}}%
\pgfpathlineto{\pgfqpoint{3.248120in}{1.785000in}}%
\pgfpathlineto{\pgfqpoint{3.250600in}{1.120000in}}%
\pgfpathlineto{\pgfqpoint{3.251840in}{1.890000in}}%
\pgfpathlineto{\pgfqpoint{3.254320in}{1.540000in}}%
\pgfpathlineto{\pgfqpoint{3.259280in}{0.945000in}}%
\pgfpathlineto{\pgfqpoint{3.263000in}{1.540000in}}%
\pgfpathlineto{\pgfqpoint{3.264240in}{1.610000in}}%
\pgfpathlineto{\pgfqpoint{3.265480in}{1.540000in}}%
\pgfpathlineto{\pgfqpoint{3.266720in}{1.365000in}}%
\pgfpathlineto{\pgfqpoint{3.267960in}{1.610000in}}%
\pgfpathlineto{\pgfqpoint{3.270440in}{1.365000in}}%
\pgfpathlineto{\pgfqpoint{3.272920in}{1.505000in}}%
\pgfpathlineto{\pgfqpoint{3.274160in}{1.645000in}}%
\pgfpathlineto{\pgfqpoint{3.275400in}{1.225000in}}%
\pgfpathlineto{\pgfqpoint{3.276640in}{1.295000in}}%
\pgfpathlineto{\pgfqpoint{3.277880in}{1.540000in}}%
\pgfpathlineto{\pgfqpoint{3.279120in}{1.155000in}}%
\pgfpathlineto{\pgfqpoint{3.281600in}{1.540000in}}%
\pgfpathlineto{\pgfqpoint{3.282840in}{1.750000in}}%
\pgfpathlineto{\pgfqpoint{3.284080in}{1.750000in}}%
\pgfpathlineto{\pgfqpoint{3.285320in}{1.330000in}}%
\pgfpathlineto{\pgfqpoint{3.286560in}{1.400000in}}%
\pgfpathlineto{\pgfqpoint{3.289040in}{1.750000in}}%
\pgfpathlineto{\pgfqpoint{3.290280in}{1.400000in}}%
\pgfpathlineto{\pgfqpoint{3.291520in}{1.680000in}}%
\pgfpathlineto{\pgfqpoint{3.292760in}{1.470000in}}%
\pgfpathlineto{\pgfqpoint{3.295240in}{1.960000in}}%
\pgfpathlineto{\pgfqpoint{3.297720in}{1.400000in}}%
\pgfpathlineto{\pgfqpoint{3.300200in}{1.820000in}}%
\pgfpathlineto{\pgfqpoint{3.302680in}{1.085000in}}%
\pgfpathlineto{\pgfqpoint{3.303920in}{1.295000in}}%
\pgfpathlineto{\pgfqpoint{3.305160in}{1.295000in}}%
\pgfpathlineto{\pgfqpoint{3.306400in}{1.575000in}}%
\pgfpathlineto{\pgfqpoint{3.308880in}{1.120000in}}%
\pgfpathlineto{\pgfqpoint{3.310120in}{1.470000in}}%
\pgfpathlineto{\pgfqpoint{3.311360in}{1.155000in}}%
\pgfpathlineto{\pgfqpoint{3.312600in}{1.715000in}}%
\pgfpathlineto{\pgfqpoint{3.313840in}{1.680000in}}%
\pgfpathlineto{\pgfqpoint{3.315080in}{1.540000in}}%
\pgfpathlineto{\pgfqpoint{3.317560in}{1.225000in}}%
\pgfpathlineto{\pgfqpoint{3.318800in}{1.155000in}}%
\pgfpathlineto{\pgfqpoint{3.320040in}{1.295000in}}%
\pgfpathlineto{\pgfqpoint{3.321280in}{1.190000in}}%
\pgfpathlineto{\pgfqpoint{3.322520in}{1.225000in}}%
\pgfpathlineto{\pgfqpoint{3.325000in}{1.890000in}}%
\pgfpathlineto{\pgfqpoint{3.327480in}{1.295000in}}%
\pgfpathlineto{\pgfqpoint{3.328720in}{1.330000in}}%
\pgfpathlineto{\pgfqpoint{3.329960in}{1.400000in}}%
\pgfpathlineto{\pgfqpoint{3.331200in}{1.750000in}}%
\pgfpathlineto{\pgfqpoint{3.332440in}{1.330000in}}%
\pgfpathlineto{\pgfqpoint{3.333680in}{1.575000in}}%
\pgfpathlineto{\pgfqpoint{3.334920in}{1.260000in}}%
\pgfpathlineto{\pgfqpoint{3.337400in}{1.610000in}}%
\pgfpathlineto{\pgfqpoint{3.338640in}{1.330000in}}%
\pgfpathlineto{\pgfqpoint{3.339880in}{1.855000in}}%
\pgfpathlineto{\pgfqpoint{3.341120in}{1.365000in}}%
\pgfpathlineto{\pgfqpoint{3.342360in}{1.890000in}}%
\pgfpathlineto{\pgfqpoint{3.343600in}{1.540000in}}%
\pgfpathlineto{\pgfqpoint{3.344840in}{1.575000in}}%
\pgfpathlineto{\pgfqpoint{3.346080in}{1.785000in}}%
\pgfpathlineto{\pgfqpoint{3.348560in}{1.540000in}}%
\pgfpathlineto{\pgfqpoint{3.351040in}{1.820000in}}%
\pgfpathlineto{\pgfqpoint{3.353520in}{2.100000in}}%
\pgfpathlineto{\pgfqpoint{3.354760in}{1.470000in}}%
\pgfpathlineto{\pgfqpoint{3.356000in}{1.785000in}}%
\pgfpathlineto{\pgfqpoint{3.358480in}{1.365000in}}%
\pgfpathlineto{\pgfqpoint{3.359720in}{1.610000in}}%
\pgfpathlineto{\pgfqpoint{3.360960in}{1.610000in}}%
\pgfpathlineto{\pgfqpoint{3.362200in}{1.505000in}}%
\pgfpathlineto{\pgfqpoint{3.364680in}{1.155000in}}%
\pgfpathlineto{\pgfqpoint{3.368400in}{1.540000in}}%
\pgfpathlineto{\pgfqpoint{3.369640in}{1.120000in}}%
\pgfpathlineto{\pgfqpoint{3.370880in}{1.330000in}}%
\pgfpathlineto{\pgfqpoint{3.372120in}{1.785000in}}%
\pgfpathlineto{\pgfqpoint{3.375840in}{0.945000in}}%
\pgfpathlineto{\pgfqpoint{3.377080in}{1.505000in}}%
\pgfpathlineto{\pgfqpoint{3.378320in}{1.365000in}}%
\pgfpathlineto{\pgfqpoint{3.379560in}{1.680000in}}%
\pgfpathlineto{\pgfqpoint{3.380800in}{1.260000in}}%
\pgfpathlineto{\pgfqpoint{3.383280in}{1.505000in}}%
\pgfpathlineto{\pgfqpoint{3.384520in}{2.135000in}}%
\pgfpathlineto{\pgfqpoint{3.385760in}{1.995000in}}%
\pgfpathlineto{\pgfqpoint{3.387000in}{1.645000in}}%
\pgfpathlineto{\pgfqpoint{3.388240in}{1.715000in}}%
\pgfpathlineto{\pgfqpoint{3.389480in}{1.890000in}}%
\pgfpathlineto{\pgfqpoint{3.390720in}{1.680000in}}%
\pgfpathlineto{\pgfqpoint{3.393200in}{1.050000in}}%
\pgfpathlineto{\pgfqpoint{3.394440in}{1.120000in}}%
\pgfpathlineto{\pgfqpoint{3.395680in}{1.085000in}}%
\pgfpathlineto{\pgfqpoint{3.396920in}{1.610000in}}%
\pgfpathlineto{\pgfqpoint{3.399400in}{1.295000in}}%
\pgfpathlineto{\pgfqpoint{3.400640in}{1.680000in}}%
\pgfpathlineto{\pgfqpoint{3.401880in}{1.575000in}}%
\pgfpathlineto{\pgfqpoint{3.403120in}{1.295000in}}%
\pgfpathlineto{\pgfqpoint{3.404360in}{1.645000in}}%
\pgfpathlineto{\pgfqpoint{3.405600in}{1.295000in}}%
\pgfpathlineto{\pgfqpoint{3.406840in}{1.400000in}}%
\pgfpathlineto{\pgfqpoint{3.408080in}{1.190000in}}%
\pgfpathlineto{\pgfqpoint{3.409320in}{1.435000in}}%
\pgfpathlineto{\pgfqpoint{3.410560in}{1.260000in}}%
\pgfpathlineto{\pgfqpoint{3.411800in}{1.540000in}}%
\pgfpathlineto{\pgfqpoint{3.413040in}{1.435000in}}%
\pgfpathlineto{\pgfqpoint{3.414280in}{1.155000in}}%
\pgfpathlineto{\pgfqpoint{3.415520in}{1.330000in}}%
\pgfpathlineto{\pgfqpoint{3.416760in}{1.260000in}}%
\pgfpathlineto{\pgfqpoint{3.418000in}{1.330000in}}%
\pgfpathlineto{\pgfqpoint{3.419240in}{1.715000in}}%
\pgfpathlineto{\pgfqpoint{3.420480in}{1.155000in}}%
\pgfpathlineto{\pgfqpoint{3.422960in}{1.435000in}}%
\pgfpathlineto{\pgfqpoint{3.424200in}{2.240000in}}%
\pgfpathlineto{\pgfqpoint{3.425440in}{1.155000in}}%
\pgfpathlineto{\pgfqpoint{3.426680in}{1.575000in}}%
\pgfpathlineto{\pgfqpoint{3.427920in}{1.540000in}}%
\pgfpathlineto{\pgfqpoint{3.430400in}{1.610000in}}%
\pgfpathlineto{\pgfqpoint{3.431640in}{1.365000in}}%
\pgfpathlineto{\pgfqpoint{3.432880in}{1.365000in}}%
\pgfpathlineto{\pgfqpoint{3.434120in}{1.680000in}}%
\pgfpathlineto{\pgfqpoint{3.435360in}{1.435000in}}%
\pgfpathlineto{\pgfqpoint{3.436600in}{1.435000in}}%
\pgfpathlineto{\pgfqpoint{3.437840in}{1.015000in}}%
\pgfpathlineto{\pgfqpoint{3.439080in}{1.715000in}}%
\pgfpathlineto{\pgfqpoint{3.440320in}{1.225000in}}%
\pgfpathlineto{\pgfqpoint{3.441560in}{1.505000in}}%
\pgfpathlineto{\pgfqpoint{3.442800in}{1.470000in}}%
\pgfpathlineto{\pgfqpoint{3.444040in}{1.750000in}}%
\pgfpathlineto{\pgfqpoint{3.445280in}{1.225000in}}%
\pgfpathlineto{\pgfqpoint{3.446520in}{1.295000in}}%
\pgfpathlineto{\pgfqpoint{3.447760in}{1.155000in}}%
\pgfpathlineto{\pgfqpoint{3.449000in}{1.610000in}}%
\pgfpathlineto{\pgfqpoint{3.450240in}{1.260000in}}%
\pgfpathlineto{\pgfqpoint{3.451480in}{1.540000in}}%
\pgfpathlineto{\pgfqpoint{3.452720in}{1.540000in}}%
\pgfpathlineto{\pgfqpoint{3.453960in}{1.610000in}}%
\pgfpathlineto{\pgfqpoint{3.455200in}{1.750000in}}%
\pgfpathlineto{\pgfqpoint{3.456440in}{1.750000in}}%
\pgfpathlineto{\pgfqpoint{3.457680in}{1.470000in}}%
\pgfpathlineto{\pgfqpoint{3.458920in}{1.470000in}}%
\pgfpathlineto{\pgfqpoint{3.460160in}{1.645000in}}%
\pgfpathlineto{\pgfqpoint{3.462640in}{1.575000in}}%
\pgfpathlineto{\pgfqpoint{3.463880in}{1.575000in}}%
\pgfpathlineto{\pgfqpoint{3.465120in}{1.855000in}}%
\pgfpathlineto{\pgfqpoint{3.466360in}{1.435000in}}%
\pgfpathlineto{\pgfqpoint{3.467600in}{1.400000in}}%
\pgfpathlineto{\pgfqpoint{3.468840in}{1.225000in}}%
\pgfpathlineto{\pgfqpoint{3.470080in}{1.260000in}}%
\pgfpathlineto{\pgfqpoint{3.472560in}{1.050000in}}%
\pgfpathlineto{\pgfqpoint{3.473800in}{1.400000in}}%
\pgfpathlineto{\pgfqpoint{3.475040in}{0.980000in}}%
\pgfpathlineto{\pgfqpoint{3.477520in}{1.435000in}}%
\pgfpathlineto{\pgfqpoint{3.478760in}{1.365000in}}%
\pgfpathlineto{\pgfqpoint{3.480000in}{1.190000in}}%
\pgfpathlineto{\pgfqpoint{3.481240in}{1.260000in}}%
\pgfpathlineto{\pgfqpoint{3.482480in}{1.225000in}}%
\pgfpathlineto{\pgfqpoint{3.483720in}{1.610000in}}%
\pgfpathlineto{\pgfqpoint{3.486200in}{1.260000in}}%
\pgfpathlineto{\pgfqpoint{3.489920in}{1.750000in}}%
\pgfpathlineto{\pgfqpoint{3.491160in}{1.610000in}}%
\pgfpathlineto{\pgfqpoint{3.492400in}{1.015000in}}%
\pgfpathlineto{\pgfqpoint{3.493640in}{1.400000in}}%
\pgfpathlineto{\pgfqpoint{3.494880in}{1.190000in}}%
\pgfpathlineto{\pgfqpoint{3.496120in}{1.610000in}}%
\pgfpathlineto{\pgfqpoint{3.497360in}{1.575000in}}%
\pgfpathlineto{\pgfqpoint{3.499840in}{1.085000in}}%
\pgfpathlineto{\pgfqpoint{3.501080in}{1.330000in}}%
\pgfpathlineto{\pgfqpoint{3.502320in}{1.330000in}}%
\pgfpathlineto{\pgfqpoint{3.503560in}{1.435000in}}%
\pgfpathlineto{\pgfqpoint{3.506040in}{1.785000in}}%
\pgfpathlineto{\pgfqpoint{3.507280in}{1.645000in}}%
\pgfpathlineto{\pgfqpoint{3.508520in}{1.960000in}}%
\pgfpathlineto{\pgfqpoint{3.509760in}{1.820000in}}%
\pgfpathlineto{\pgfqpoint{3.511000in}{1.925000in}}%
\pgfpathlineto{\pgfqpoint{3.512240in}{1.120000in}}%
\pgfpathlineto{\pgfqpoint{3.514720in}{1.680000in}}%
\pgfpathlineto{\pgfqpoint{3.515960in}{1.540000in}}%
\pgfpathlineto{\pgfqpoint{3.517200in}{1.575000in}}%
\pgfpathlineto{\pgfqpoint{3.518440in}{1.085000in}}%
\pgfpathlineto{\pgfqpoint{3.519680in}{1.645000in}}%
\pgfpathlineto{\pgfqpoint{3.520920in}{1.540000in}}%
\pgfpathlineto{\pgfqpoint{3.522160in}{1.190000in}}%
\pgfpathlineto{\pgfqpoint{3.523400in}{1.330000in}}%
\pgfpathlineto{\pgfqpoint{3.524640in}{1.750000in}}%
\pgfpathlineto{\pgfqpoint{3.525880in}{1.260000in}}%
\pgfpathlineto{\pgfqpoint{3.527120in}{1.645000in}}%
\pgfpathlineto{\pgfqpoint{3.528360in}{1.575000in}}%
\pgfpathlineto{\pgfqpoint{3.529600in}{1.015000in}}%
\pgfpathlineto{\pgfqpoint{3.530840in}{1.085000in}}%
\pgfpathlineto{\pgfqpoint{3.532080in}{1.505000in}}%
\pgfpathlineto{\pgfqpoint{3.533320in}{1.540000in}}%
\pgfpathlineto{\pgfqpoint{3.534560in}{1.435000in}}%
\pgfpathlineto{\pgfqpoint{3.535800in}{1.610000in}}%
\pgfpathlineto{\pgfqpoint{3.538280in}{1.015000in}}%
\pgfpathlineto{\pgfqpoint{3.540760in}{1.750000in}}%
\pgfpathlineto{\pgfqpoint{3.543240in}{1.330000in}}%
\pgfpathlineto{\pgfqpoint{3.544480in}{1.295000in}}%
\pgfpathlineto{\pgfqpoint{3.545720in}{1.680000in}}%
\pgfpathlineto{\pgfqpoint{3.546960in}{1.365000in}}%
\pgfpathlineto{\pgfqpoint{3.548200in}{1.925000in}}%
\pgfpathlineto{\pgfqpoint{3.550680in}{1.505000in}}%
\pgfpathlineto{\pgfqpoint{3.553160in}{1.295000in}}%
\pgfpathlineto{\pgfqpoint{3.554400in}{1.365000in}}%
\pgfpathlineto{\pgfqpoint{3.555640in}{1.820000in}}%
\pgfpathlineto{\pgfqpoint{3.558120in}{1.190000in}}%
\pgfpathlineto{\pgfqpoint{3.559360in}{1.400000in}}%
\pgfpathlineto{\pgfqpoint{3.560600in}{1.120000in}}%
\pgfpathlineto{\pgfqpoint{3.561840in}{1.645000in}}%
\pgfpathlineto{\pgfqpoint{3.563080in}{1.470000in}}%
\pgfpathlineto{\pgfqpoint{3.564320in}{1.120000in}}%
\pgfpathlineto{\pgfqpoint{3.566800in}{1.715000in}}%
\pgfpathlineto{\pgfqpoint{3.568040in}{1.715000in}}%
\pgfpathlineto{\pgfqpoint{3.569280in}{1.960000in}}%
\pgfpathlineto{\pgfqpoint{3.570520in}{1.225000in}}%
\pgfpathlineto{\pgfqpoint{3.571760in}{1.190000in}}%
\pgfpathlineto{\pgfqpoint{3.573000in}{1.365000in}}%
\pgfpathlineto{\pgfqpoint{3.574240in}{1.120000in}}%
\pgfpathlineto{\pgfqpoint{3.576720in}{1.400000in}}%
\pgfpathlineto{\pgfqpoint{3.577960in}{1.260000in}}%
\pgfpathlineto{\pgfqpoint{3.579200in}{1.295000in}}%
\pgfpathlineto{\pgfqpoint{3.580440in}{1.260000in}}%
\pgfpathlineto{\pgfqpoint{3.581680in}{1.190000in}}%
\pgfpathlineto{\pgfqpoint{3.584160in}{1.750000in}}%
\pgfpathlineto{\pgfqpoint{3.585400in}{1.365000in}}%
\pgfpathlineto{\pgfqpoint{3.587880in}{2.065000in}}%
\pgfpathlineto{\pgfqpoint{3.589120in}{1.400000in}}%
\pgfpathlineto{\pgfqpoint{3.590360in}{1.435000in}}%
\pgfpathlineto{\pgfqpoint{3.592840in}{1.610000in}}%
\pgfpathlineto{\pgfqpoint{3.595320in}{1.330000in}}%
\pgfpathlineto{\pgfqpoint{3.596560in}{1.505000in}}%
\pgfpathlineto{\pgfqpoint{3.597800in}{1.505000in}}%
\pgfpathlineto{\pgfqpoint{3.599040in}{1.715000in}}%
\pgfpathlineto{\pgfqpoint{3.600280in}{1.505000in}}%
\pgfpathlineto{\pgfqpoint{3.601520in}{1.785000in}}%
\pgfpathlineto{\pgfqpoint{3.602760in}{1.190000in}}%
\pgfpathlineto{\pgfqpoint{3.605240in}{1.610000in}}%
\pgfpathlineto{\pgfqpoint{3.606480in}{1.050000in}}%
\pgfpathlineto{\pgfqpoint{3.607720in}{1.890000in}}%
\pgfpathlineto{\pgfqpoint{3.610200in}{1.295000in}}%
\pgfpathlineto{\pgfqpoint{3.611440in}{1.645000in}}%
\pgfpathlineto{\pgfqpoint{3.612680in}{1.540000in}}%
\pgfpathlineto{\pgfqpoint{3.613920in}{1.295000in}}%
\pgfpathlineto{\pgfqpoint{3.616400in}{1.435000in}}%
\pgfpathlineto{\pgfqpoint{3.618880in}{1.750000in}}%
\pgfpathlineto{\pgfqpoint{3.620120in}{1.540000in}}%
\pgfpathlineto{\pgfqpoint{3.621360in}{1.540000in}}%
\pgfpathlineto{\pgfqpoint{3.623840in}{1.820000in}}%
\pgfpathlineto{\pgfqpoint{3.625080in}{1.715000in}}%
\pgfpathlineto{\pgfqpoint{3.626320in}{1.260000in}}%
\pgfpathlineto{\pgfqpoint{3.627560in}{1.295000in}}%
\pgfpathlineto{\pgfqpoint{3.628800in}{1.505000in}}%
\pgfpathlineto{\pgfqpoint{3.630040in}{1.435000in}}%
\pgfpathlineto{\pgfqpoint{3.631280in}{1.505000in}}%
\pgfpathlineto{\pgfqpoint{3.632520in}{1.155000in}}%
\pgfpathlineto{\pgfqpoint{3.633760in}{1.715000in}}%
\pgfpathlineto{\pgfqpoint{3.635000in}{1.750000in}}%
\pgfpathlineto{\pgfqpoint{3.637480in}{1.365000in}}%
\pgfpathlineto{\pgfqpoint{3.638720in}{1.435000in}}%
\pgfpathlineto{\pgfqpoint{3.641200in}{1.155000in}}%
\pgfpathlineto{\pgfqpoint{3.642440in}{1.785000in}}%
\pgfpathlineto{\pgfqpoint{3.643680in}{1.470000in}}%
\pgfpathlineto{\pgfqpoint{3.644920in}{1.470000in}}%
\pgfpathlineto{\pgfqpoint{3.646160in}{1.330000in}}%
\pgfpathlineto{\pgfqpoint{3.647400in}{1.470000in}}%
\pgfpathlineto{\pgfqpoint{3.648640in}{1.470000in}}%
\pgfpathlineto{\pgfqpoint{3.649880in}{0.735000in}}%
\pgfpathlineto{\pgfqpoint{3.651120in}{1.855000in}}%
\pgfpathlineto{\pgfqpoint{3.653600in}{1.225000in}}%
\pgfpathlineto{\pgfqpoint{3.654840in}{1.540000in}}%
\pgfpathlineto{\pgfqpoint{3.657320in}{1.400000in}}%
\pgfpathlineto{\pgfqpoint{3.658560in}{1.785000in}}%
\pgfpathlineto{\pgfqpoint{3.659800in}{1.785000in}}%
\pgfpathlineto{\pgfqpoint{3.661040in}{1.330000in}}%
\pgfpathlineto{\pgfqpoint{3.662280in}{1.295000in}}%
\pgfpathlineto{\pgfqpoint{3.663520in}{1.295000in}}%
\pgfpathlineto{\pgfqpoint{3.664760in}{1.365000in}}%
\pgfpathlineto{\pgfqpoint{3.666000in}{1.225000in}}%
\pgfpathlineto{\pgfqpoint{3.667240in}{1.610000in}}%
\pgfpathlineto{\pgfqpoint{3.668480in}{0.980000in}}%
\pgfpathlineto{\pgfqpoint{3.669720in}{1.330000in}}%
\pgfpathlineto{\pgfqpoint{3.670960in}{0.840000in}}%
\pgfpathlineto{\pgfqpoint{3.672200in}{1.225000in}}%
\pgfpathlineto{\pgfqpoint{3.673440in}{1.190000in}}%
\pgfpathlineto{\pgfqpoint{3.674680in}{1.470000in}}%
\pgfpathlineto{\pgfqpoint{3.675920in}{1.330000in}}%
\pgfpathlineto{\pgfqpoint{3.677160in}{1.540000in}}%
\pgfpathlineto{\pgfqpoint{3.679640in}{1.190000in}}%
\pgfpathlineto{\pgfqpoint{3.680880in}{1.190000in}}%
\pgfpathlineto{\pgfqpoint{3.682120in}{1.575000in}}%
\pgfpathlineto{\pgfqpoint{3.683360in}{1.330000in}}%
\pgfpathlineto{\pgfqpoint{3.684600in}{1.575000in}}%
\pgfpathlineto{\pgfqpoint{3.687080in}{1.050000in}}%
\pgfpathlineto{\pgfqpoint{3.689560in}{1.575000in}}%
\pgfpathlineto{\pgfqpoint{3.690800in}{1.505000in}}%
\pgfpathlineto{\pgfqpoint{3.692040in}{1.540000in}}%
\pgfpathlineto{\pgfqpoint{3.693280in}{0.875000in}}%
\pgfpathlineto{\pgfqpoint{3.697000in}{1.715000in}}%
\pgfpathlineto{\pgfqpoint{3.698240in}{1.540000in}}%
\pgfpathlineto{\pgfqpoint{3.699480in}{1.050000in}}%
\pgfpathlineto{\pgfqpoint{3.700720in}{1.645000in}}%
\pgfpathlineto{\pgfqpoint{3.701960in}{1.435000in}}%
\pgfpathlineto{\pgfqpoint{3.703200in}{1.505000in}}%
\pgfpathlineto{\pgfqpoint{3.705680in}{1.260000in}}%
\pgfpathlineto{\pgfqpoint{3.706920in}{1.085000in}}%
\pgfpathlineto{\pgfqpoint{3.708160in}{1.680000in}}%
\pgfpathlineto{\pgfqpoint{3.709400in}{1.680000in}}%
\pgfpathlineto{\pgfqpoint{3.710640in}{1.400000in}}%
\pgfpathlineto{\pgfqpoint{3.711880in}{1.505000in}}%
\pgfpathlineto{\pgfqpoint{3.713120in}{1.225000in}}%
\pgfpathlineto{\pgfqpoint{3.715600in}{1.750000in}}%
\pgfpathlineto{\pgfqpoint{3.718080in}{1.120000in}}%
\pgfpathlineto{\pgfqpoint{3.719320in}{1.085000in}}%
\pgfpathlineto{\pgfqpoint{3.720560in}{1.190000in}}%
\pgfpathlineto{\pgfqpoint{3.721800in}{1.435000in}}%
\pgfpathlineto{\pgfqpoint{3.723040in}{1.190000in}}%
\pgfpathlineto{\pgfqpoint{3.724280in}{1.610000in}}%
\pgfpathlineto{\pgfqpoint{3.725520in}{1.575000in}}%
\pgfpathlineto{\pgfqpoint{3.726760in}{1.365000in}}%
\pgfpathlineto{\pgfqpoint{3.728000in}{1.680000in}}%
\pgfpathlineto{\pgfqpoint{3.729240in}{1.470000in}}%
\pgfpathlineto{\pgfqpoint{3.730480in}{1.820000in}}%
\pgfpathlineto{\pgfqpoint{3.732960in}{1.120000in}}%
\pgfpathlineto{\pgfqpoint{3.734200in}{1.540000in}}%
\pgfpathlineto{\pgfqpoint{3.735440in}{1.470000in}}%
\pgfpathlineto{\pgfqpoint{3.736680in}{1.750000in}}%
\pgfpathlineto{\pgfqpoint{3.737920in}{1.680000in}}%
\pgfpathlineto{\pgfqpoint{3.739160in}{1.925000in}}%
\pgfpathlineto{\pgfqpoint{3.741640in}{1.260000in}}%
\pgfpathlineto{\pgfqpoint{3.742880in}{1.330000in}}%
\pgfpathlineto{\pgfqpoint{3.744120in}{1.155000in}}%
\pgfpathlineto{\pgfqpoint{3.745360in}{1.575000in}}%
\pgfpathlineto{\pgfqpoint{3.746600in}{1.575000in}}%
\pgfpathlineto{\pgfqpoint{3.747840in}{2.030000in}}%
\pgfpathlineto{\pgfqpoint{3.749080in}{1.365000in}}%
\pgfpathlineto{\pgfqpoint{3.750320in}{1.645000in}}%
\pgfpathlineto{\pgfqpoint{3.752800in}{1.470000in}}%
\pgfpathlineto{\pgfqpoint{3.754040in}{1.715000in}}%
\pgfpathlineto{\pgfqpoint{3.756520in}{1.260000in}}%
\pgfpathlineto{\pgfqpoint{3.757760in}{1.750000in}}%
\pgfpathlineto{\pgfqpoint{3.759000in}{1.540000in}}%
\pgfpathlineto{\pgfqpoint{3.760240in}{1.540000in}}%
\pgfpathlineto{\pgfqpoint{3.761480in}{1.645000in}}%
\pgfpathlineto{\pgfqpoint{3.762720in}{1.610000in}}%
\pgfpathlineto{\pgfqpoint{3.763960in}{1.540000in}}%
\pgfpathlineto{\pgfqpoint{3.765200in}{1.400000in}}%
\pgfpathlineto{\pgfqpoint{3.766440in}{1.085000in}}%
\pgfpathlineto{\pgfqpoint{3.767680in}{1.715000in}}%
\pgfpathlineto{\pgfqpoint{3.768920in}{1.365000in}}%
\pgfpathlineto{\pgfqpoint{3.770160in}{1.785000in}}%
\pgfpathlineto{\pgfqpoint{3.771400in}{1.155000in}}%
\pgfpathlineto{\pgfqpoint{3.772640in}{1.365000in}}%
\pgfpathlineto{\pgfqpoint{3.773880in}{1.295000in}}%
\pgfpathlineto{\pgfqpoint{3.776360in}{1.575000in}}%
\pgfpathlineto{\pgfqpoint{3.777600in}{1.645000in}}%
\pgfpathlineto{\pgfqpoint{3.778840in}{1.190000in}}%
\pgfpathlineto{\pgfqpoint{3.780080in}{1.540000in}}%
\pgfpathlineto{\pgfqpoint{3.782560in}{1.190000in}}%
\pgfpathlineto{\pgfqpoint{3.786280in}{1.715000in}}%
\pgfpathlineto{\pgfqpoint{3.787520in}{1.890000in}}%
\pgfpathlineto{\pgfqpoint{3.791240in}{1.085000in}}%
\pgfpathlineto{\pgfqpoint{3.792480in}{1.750000in}}%
\pgfpathlineto{\pgfqpoint{3.793720in}{1.365000in}}%
\pgfpathlineto{\pgfqpoint{3.794960in}{1.330000in}}%
\pgfpathlineto{\pgfqpoint{3.796200in}{0.910000in}}%
\pgfpathlineto{\pgfqpoint{3.797440in}{1.820000in}}%
\pgfpathlineto{\pgfqpoint{3.801160in}{1.260000in}}%
\pgfpathlineto{\pgfqpoint{3.802400in}{1.785000in}}%
\pgfpathlineto{\pgfqpoint{3.803640in}{1.680000in}}%
\pgfpathlineto{\pgfqpoint{3.804880in}{1.085000in}}%
\pgfpathlineto{\pgfqpoint{3.806120in}{1.260000in}}%
\pgfpathlineto{\pgfqpoint{3.808600in}{2.030000in}}%
\pgfpathlineto{\pgfqpoint{3.809840in}{1.470000in}}%
\pgfpathlineto{\pgfqpoint{3.811080in}{1.505000in}}%
\pgfpathlineto{\pgfqpoint{3.812320in}{1.785000in}}%
\pgfpathlineto{\pgfqpoint{3.813560in}{1.225000in}}%
\pgfpathlineto{\pgfqpoint{3.814800in}{1.260000in}}%
\pgfpathlineto{\pgfqpoint{3.816040in}{1.505000in}}%
\pgfpathlineto{\pgfqpoint{3.817280in}{0.980000in}}%
\pgfpathlineto{\pgfqpoint{3.819760in}{1.610000in}}%
\pgfpathlineto{\pgfqpoint{3.821000in}{1.435000in}}%
\pgfpathlineto{\pgfqpoint{3.822240in}{0.665000in}}%
\pgfpathlineto{\pgfqpoint{3.824720in}{1.225000in}}%
\pgfpathlineto{\pgfqpoint{3.827200in}{1.540000in}}%
\pgfpathlineto{\pgfqpoint{3.828440in}{2.065000in}}%
\pgfpathlineto{\pgfqpoint{3.829680in}{1.365000in}}%
\pgfpathlineto{\pgfqpoint{3.832160in}{1.575000in}}%
\pgfpathlineto{\pgfqpoint{3.833400in}{1.295000in}}%
\pgfpathlineto{\pgfqpoint{3.834640in}{1.365000in}}%
\pgfpathlineto{\pgfqpoint{3.835880in}{1.225000in}}%
\pgfpathlineto{\pgfqpoint{3.837120in}{1.435000in}}%
\pgfpathlineto{\pgfqpoint{3.838360in}{1.400000in}}%
\pgfpathlineto{\pgfqpoint{3.839600in}{1.680000in}}%
\pgfpathlineto{\pgfqpoint{3.840840in}{1.330000in}}%
\pgfpathlineto{\pgfqpoint{3.842080in}{1.540000in}}%
\pgfpathlineto{\pgfqpoint{3.843320in}{1.190000in}}%
\pgfpathlineto{\pgfqpoint{3.844560in}{1.505000in}}%
\pgfpathlineto{\pgfqpoint{3.845800in}{1.435000in}}%
\pgfpathlineto{\pgfqpoint{3.847040in}{1.680000in}}%
\pgfpathlineto{\pgfqpoint{3.848280in}{1.155000in}}%
\pgfpathlineto{\pgfqpoint{3.850760in}{1.645000in}}%
\pgfpathlineto{\pgfqpoint{3.852000in}{1.470000in}}%
\pgfpathlineto{\pgfqpoint{3.853240in}{1.085000in}}%
\pgfpathlineto{\pgfqpoint{3.854480in}{1.750000in}}%
\pgfpathlineto{\pgfqpoint{3.855720in}{1.400000in}}%
\pgfpathlineto{\pgfqpoint{3.858200in}{1.645000in}}%
\pgfpathlineto{\pgfqpoint{3.859440in}{1.155000in}}%
\pgfpathlineto{\pgfqpoint{3.860680in}{1.715000in}}%
\pgfpathlineto{\pgfqpoint{3.861920in}{1.295000in}}%
\pgfpathlineto{\pgfqpoint{3.863160in}{1.435000in}}%
\pgfpathlineto{\pgfqpoint{3.864400in}{1.365000in}}%
\pgfpathlineto{\pgfqpoint{3.865640in}{1.190000in}}%
\pgfpathlineto{\pgfqpoint{3.866880in}{1.505000in}}%
\pgfpathlineto{\pgfqpoint{3.868120in}{1.435000in}}%
\pgfpathlineto{\pgfqpoint{3.869360in}{0.805000in}}%
\pgfpathlineto{\pgfqpoint{3.870600in}{1.365000in}}%
\pgfpathlineto{\pgfqpoint{3.871840in}{0.945000in}}%
\pgfpathlineto{\pgfqpoint{3.873080in}{1.155000in}}%
\pgfpathlineto{\pgfqpoint{3.874320in}{1.575000in}}%
\pgfpathlineto{\pgfqpoint{3.875560in}{1.505000in}}%
\pgfpathlineto{\pgfqpoint{3.876800in}{1.365000in}}%
\pgfpathlineto{\pgfqpoint{3.878040in}{0.875000in}}%
\pgfpathlineto{\pgfqpoint{3.879280in}{1.120000in}}%
\pgfpathlineto{\pgfqpoint{3.880520in}{1.085000in}}%
\pgfpathlineto{\pgfqpoint{3.881760in}{0.770000in}}%
\pgfpathlineto{\pgfqpoint{3.883000in}{1.295000in}}%
\pgfpathlineto{\pgfqpoint{3.884240in}{1.085000in}}%
\pgfpathlineto{\pgfqpoint{3.885480in}{1.120000in}}%
\pgfpathlineto{\pgfqpoint{3.887960in}{1.470000in}}%
\pgfpathlineto{\pgfqpoint{3.889200in}{0.980000in}}%
\pgfpathlineto{\pgfqpoint{3.891680in}{1.540000in}}%
\pgfpathlineto{\pgfqpoint{3.892920in}{1.470000in}}%
\pgfpathlineto{\pgfqpoint{3.894160in}{1.470000in}}%
\pgfpathlineto{\pgfqpoint{3.895400in}{1.645000in}}%
\pgfpathlineto{\pgfqpoint{3.897880in}{1.050000in}}%
\pgfpathlineto{\pgfqpoint{3.899120in}{1.435000in}}%
\pgfpathlineto{\pgfqpoint{3.900360in}{1.435000in}}%
\pgfpathlineto{\pgfqpoint{3.901600in}{1.645000in}}%
\pgfpathlineto{\pgfqpoint{3.902840in}{1.400000in}}%
\pgfpathlineto{\pgfqpoint{3.905320in}{1.820000in}}%
\pgfpathlineto{\pgfqpoint{3.906560in}{1.785000in}}%
\pgfpathlineto{\pgfqpoint{3.907800in}{1.120000in}}%
\pgfpathlineto{\pgfqpoint{3.910280in}{1.470000in}}%
\pgfpathlineto{\pgfqpoint{3.912760in}{1.295000in}}%
\pgfpathlineto{\pgfqpoint{3.915240in}{1.680000in}}%
\pgfpathlineto{\pgfqpoint{3.916480in}{1.715000in}}%
\pgfpathlineto{\pgfqpoint{3.917720in}{1.820000in}}%
\pgfpathlineto{\pgfqpoint{3.918960in}{1.435000in}}%
\pgfpathlineto{\pgfqpoint{3.920200in}{1.715000in}}%
\pgfpathlineto{\pgfqpoint{3.923920in}{1.365000in}}%
\pgfpathlineto{\pgfqpoint{3.925160in}{1.400000in}}%
\pgfpathlineto{\pgfqpoint{3.926400in}{1.260000in}}%
\pgfpathlineto{\pgfqpoint{3.927640in}{1.470000in}}%
\pgfpathlineto{\pgfqpoint{3.928880in}{1.120000in}}%
\pgfpathlineto{\pgfqpoint{3.930120in}{1.190000in}}%
\pgfpathlineto{\pgfqpoint{3.931360in}{1.610000in}}%
\pgfpathlineto{\pgfqpoint{3.932600in}{1.645000in}}%
\pgfpathlineto{\pgfqpoint{3.933840in}{1.365000in}}%
\pgfpathlineto{\pgfqpoint{3.936320in}{1.820000in}}%
\pgfpathlineto{\pgfqpoint{3.937560in}{1.785000in}}%
\pgfpathlineto{\pgfqpoint{3.940040in}{1.295000in}}%
\pgfpathlineto{\pgfqpoint{3.941280in}{1.330000in}}%
\pgfpathlineto{\pgfqpoint{3.942520in}{1.120000in}}%
\pgfpathlineto{\pgfqpoint{3.943760in}{1.575000in}}%
\pgfpathlineto{\pgfqpoint{3.945000in}{1.260000in}}%
\pgfpathlineto{\pgfqpoint{3.947480in}{1.715000in}}%
\pgfpathlineto{\pgfqpoint{3.951200in}{1.190000in}}%
\pgfpathlineto{\pgfqpoint{3.952440in}{1.505000in}}%
\pgfpathlineto{\pgfqpoint{3.953680in}{1.400000in}}%
\pgfpathlineto{\pgfqpoint{3.954920in}{1.435000in}}%
\pgfpathlineto{\pgfqpoint{3.957400in}{1.050000in}}%
\pgfpathlineto{\pgfqpoint{3.958640in}{1.155000in}}%
\pgfpathlineto{\pgfqpoint{3.959880in}{1.575000in}}%
\pgfpathlineto{\pgfqpoint{3.961120in}{1.575000in}}%
\pgfpathlineto{\pgfqpoint{3.962360in}{1.540000in}}%
\pgfpathlineto{\pgfqpoint{3.963600in}{1.540000in}}%
\pgfpathlineto{\pgfqpoint{3.964840in}{1.470000in}}%
\pgfpathlineto{\pgfqpoint{3.966080in}{1.610000in}}%
\pgfpathlineto{\pgfqpoint{3.967320in}{1.470000in}}%
\pgfpathlineto{\pgfqpoint{3.971040in}{1.995000in}}%
\pgfpathlineto{\pgfqpoint{3.972280in}{1.470000in}}%
\pgfpathlineto{\pgfqpoint{3.974760in}{1.820000in}}%
\pgfpathlineto{\pgfqpoint{3.977240in}{1.225000in}}%
\pgfpathlineto{\pgfqpoint{3.979720in}{1.645000in}}%
\pgfpathlineto{\pgfqpoint{3.980960in}{1.365000in}}%
\pgfpathlineto{\pgfqpoint{3.982200in}{1.645000in}}%
\pgfpathlineto{\pgfqpoint{3.983440in}{1.540000in}}%
\pgfpathlineto{\pgfqpoint{3.984680in}{1.645000in}}%
\pgfpathlineto{\pgfqpoint{3.985920in}{0.910000in}}%
\pgfpathlineto{\pgfqpoint{3.987160in}{1.470000in}}%
\pgfpathlineto{\pgfqpoint{3.989640in}{1.225000in}}%
\pgfpathlineto{\pgfqpoint{3.992120in}{1.435000in}}%
\pgfpathlineto{\pgfqpoint{3.993360in}{1.680000in}}%
\pgfpathlineto{\pgfqpoint{3.995840in}{0.980000in}}%
\pgfpathlineto{\pgfqpoint{3.997080in}{1.225000in}}%
\pgfpathlineto{\pgfqpoint{3.998320in}{1.225000in}}%
\pgfpathlineto{\pgfqpoint{3.999560in}{1.960000in}}%
\pgfpathlineto{\pgfqpoint{4.002040in}{1.295000in}}%
\pgfpathlineto{\pgfqpoint{4.003280in}{1.400000in}}%
\pgfpathlineto{\pgfqpoint{4.004520in}{1.365000in}}%
\pgfpathlineto{\pgfqpoint{4.005760in}{1.645000in}}%
\pgfpathlineto{\pgfqpoint{4.007000in}{1.120000in}}%
\pgfpathlineto{\pgfqpoint{4.009480in}{1.820000in}}%
\pgfpathlineto{\pgfqpoint{4.010720in}{1.190000in}}%
\pgfpathlineto{\pgfqpoint{4.011960in}{1.785000in}}%
\pgfpathlineto{\pgfqpoint{4.013200in}{1.330000in}}%
\pgfpathlineto{\pgfqpoint{4.014440in}{2.030000in}}%
\pgfpathlineto{\pgfqpoint{4.016920in}{1.400000in}}%
\pgfpathlineto{\pgfqpoint{4.018160in}{1.015000in}}%
\pgfpathlineto{\pgfqpoint{4.019400in}{1.155000in}}%
\pgfpathlineto{\pgfqpoint{4.020640in}{1.575000in}}%
\pgfpathlineto{\pgfqpoint{4.021880in}{1.225000in}}%
\pgfpathlineto{\pgfqpoint{4.023120in}{1.715000in}}%
\pgfpathlineto{\pgfqpoint{4.024360in}{1.400000in}}%
\pgfpathlineto{\pgfqpoint{4.025600in}{1.750000in}}%
\pgfpathlineto{\pgfqpoint{4.026840in}{1.575000in}}%
\pgfpathlineto{\pgfqpoint{4.028080in}{1.715000in}}%
\pgfpathlineto{\pgfqpoint{4.029320in}{1.505000in}}%
\pgfpathlineto{\pgfqpoint{4.030560in}{1.750000in}}%
\pgfpathlineto{\pgfqpoint{4.033040in}{1.260000in}}%
\pgfpathlineto{\pgfqpoint{4.034280in}{1.505000in}}%
\pgfpathlineto{\pgfqpoint{4.035520in}{0.770000in}}%
\pgfpathlineto{\pgfqpoint{4.038000in}{1.400000in}}%
\pgfpathlineto{\pgfqpoint{4.039240in}{1.540000in}}%
\pgfpathlineto{\pgfqpoint{4.040480in}{1.295000in}}%
\pgfpathlineto{\pgfqpoint{4.041720in}{1.505000in}}%
\pgfpathlineto{\pgfqpoint{4.042960in}{1.435000in}}%
\pgfpathlineto{\pgfqpoint{4.044200in}{1.260000in}}%
\pgfpathlineto{\pgfqpoint{4.045440in}{1.260000in}}%
\pgfpathlineto{\pgfqpoint{4.046680in}{1.540000in}}%
\pgfpathlineto{\pgfqpoint{4.047920in}{1.330000in}}%
\pgfpathlineto{\pgfqpoint{4.049160in}{1.470000in}}%
\pgfpathlineto{\pgfqpoint{4.050400in}{1.120000in}}%
\pgfpathlineto{\pgfqpoint{4.052880in}{1.995000in}}%
\pgfpathlineto{\pgfqpoint{4.054120in}{1.540000in}}%
\pgfpathlineto{\pgfqpoint{4.055360in}{1.680000in}}%
\pgfpathlineto{\pgfqpoint{4.056600in}{1.260000in}}%
\pgfpathlineto{\pgfqpoint{4.057840in}{1.575000in}}%
\pgfpathlineto{\pgfqpoint{4.059080in}{1.155000in}}%
\pgfpathlineto{\pgfqpoint{4.060320in}{1.120000in}}%
\pgfpathlineto{\pgfqpoint{4.061560in}{1.225000in}}%
\pgfpathlineto{\pgfqpoint{4.062800in}{1.505000in}}%
\pgfpathlineto{\pgfqpoint{4.064040in}{1.050000in}}%
\pgfpathlineto{\pgfqpoint{4.065280in}{1.260000in}}%
\pgfpathlineto{\pgfqpoint{4.066520in}{1.120000in}}%
\pgfpathlineto{\pgfqpoint{4.067760in}{1.610000in}}%
\pgfpathlineto{\pgfqpoint{4.070240in}{1.365000in}}%
\pgfpathlineto{\pgfqpoint{4.072720in}{1.645000in}}%
\pgfpathlineto{\pgfqpoint{4.073960in}{1.540000in}}%
\pgfpathlineto{\pgfqpoint{4.075200in}{1.295000in}}%
\pgfpathlineto{\pgfqpoint{4.077680in}{1.785000in}}%
\pgfpathlineto{\pgfqpoint{4.080160in}{1.050000in}}%
\pgfpathlineto{\pgfqpoint{4.081400in}{1.155000in}}%
\pgfpathlineto{\pgfqpoint{4.083880in}{1.575000in}}%
\pgfpathlineto{\pgfqpoint{4.085120in}{1.330000in}}%
\pgfpathlineto{\pgfqpoint{4.086360in}{1.680000in}}%
\pgfpathlineto{\pgfqpoint{4.087600in}{1.610000in}}%
\pgfpathlineto{\pgfqpoint{4.088840in}{1.680000in}}%
\pgfpathlineto{\pgfqpoint{4.090080in}{1.260000in}}%
\pgfpathlineto{\pgfqpoint{4.091320in}{1.750000in}}%
\pgfpathlineto{\pgfqpoint{4.092560in}{1.365000in}}%
\pgfpathlineto{\pgfqpoint{4.095040in}{1.890000in}}%
\pgfpathlineto{\pgfqpoint{4.096280in}{0.980000in}}%
\pgfpathlineto{\pgfqpoint{4.098760in}{1.680000in}}%
\pgfpathlineto{\pgfqpoint{4.100000in}{1.050000in}}%
\pgfpathlineto{\pgfqpoint{4.101240in}{1.155000in}}%
\pgfpathlineto{\pgfqpoint{4.102480in}{1.050000in}}%
\pgfpathlineto{\pgfqpoint{4.104960in}{1.715000in}}%
\pgfpathlineto{\pgfqpoint{4.106200in}{1.505000in}}%
\pgfpathlineto{\pgfqpoint{4.107440in}{1.610000in}}%
\pgfpathlineto{\pgfqpoint{4.108680in}{1.855000in}}%
\pgfpathlineto{\pgfqpoint{4.109920in}{1.540000in}}%
\pgfpathlineto{\pgfqpoint{4.111160in}{1.575000in}}%
\pgfpathlineto{\pgfqpoint{4.112400in}{1.540000in}}%
\pgfpathlineto{\pgfqpoint{4.116120in}{1.365000in}}%
\pgfpathlineto{\pgfqpoint{4.117360in}{1.050000in}}%
\pgfpathlineto{\pgfqpoint{4.121080in}{1.820000in}}%
\pgfpathlineto{\pgfqpoint{4.122320in}{1.015000in}}%
\pgfpathlineto{\pgfqpoint{4.123560in}{1.505000in}}%
\pgfpathlineto{\pgfqpoint{4.124800in}{1.435000in}}%
\pgfpathlineto{\pgfqpoint{4.127280in}{1.820000in}}%
\pgfpathlineto{\pgfqpoint{4.129760in}{1.260000in}}%
\pgfpathlineto{\pgfqpoint{4.132240in}{1.400000in}}%
\pgfpathlineto{\pgfqpoint{4.133480in}{1.050000in}}%
\pgfpathlineto{\pgfqpoint{4.135960in}{1.295000in}}%
\pgfpathlineto{\pgfqpoint{4.137200in}{1.225000in}}%
\pgfpathlineto{\pgfqpoint{4.138440in}{1.925000in}}%
\pgfpathlineto{\pgfqpoint{4.139680in}{1.715000in}}%
\pgfpathlineto{\pgfqpoint{4.142160in}{0.735000in}}%
\pgfpathlineto{\pgfqpoint{4.143400in}{1.400000in}}%
\pgfpathlineto{\pgfqpoint{4.144640in}{1.050000in}}%
\pgfpathlineto{\pgfqpoint{4.147120in}{1.470000in}}%
\pgfpathlineto{\pgfqpoint{4.148360in}{1.155000in}}%
\pgfpathlineto{\pgfqpoint{4.149600in}{1.505000in}}%
\pgfpathlineto{\pgfqpoint{4.150840in}{1.015000in}}%
\pgfpathlineto{\pgfqpoint{4.152080in}{1.715000in}}%
\pgfpathlineto{\pgfqpoint{4.155800in}{0.945000in}}%
\pgfpathlineto{\pgfqpoint{4.157040in}{1.575000in}}%
\pgfpathlineto{\pgfqpoint{4.159520in}{1.190000in}}%
\pgfpathlineto{\pgfqpoint{4.160760in}{1.330000in}}%
\pgfpathlineto{\pgfqpoint{4.162000in}{1.330000in}}%
\pgfpathlineto{\pgfqpoint{4.163240in}{2.100000in}}%
\pgfpathlineto{\pgfqpoint{4.166960in}{1.190000in}}%
\pgfpathlineto{\pgfqpoint{4.169440in}{1.680000in}}%
\pgfpathlineto{\pgfqpoint{4.170680in}{1.295000in}}%
\pgfpathlineto{\pgfqpoint{4.171920in}{1.785000in}}%
\pgfpathlineto{\pgfqpoint{4.173160in}{1.330000in}}%
\pgfpathlineto{\pgfqpoint{4.174400in}{1.435000in}}%
\pgfpathlineto{\pgfqpoint{4.175640in}{1.400000in}}%
\pgfpathlineto{\pgfqpoint{4.178120in}{2.135000in}}%
\pgfpathlineto{\pgfqpoint{4.179360in}{1.155000in}}%
\pgfpathlineto{\pgfqpoint{4.181840in}{1.680000in}}%
\pgfpathlineto{\pgfqpoint{4.183080in}{1.680000in}}%
\pgfpathlineto{\pgfqpoint{4.184320in}{1.365000in}}%
\pgfpathlineto{\pgfqpoint{4.185560in}{1.680000in}}%
\pgfpathlineto{\pgfqpoint{4.186800in}{1.120000in}}%
\pgfpathlineto{\pgfqpoint{4.188040in}{1.120000in}}%
\pgfpathlineto{\pgfqpoint{4.189280in}{1.890000in}}%
\pgfpathlineto{\pgfqpoint{4.191760in}{1.085000in}}%
\pgfpathlineto{\pgfqpoint{4.193000in}{1.575000in}}%
\pgfpathlineto{\pgfqpoint{4.194240in}{1.120000in}}%
\pgfpathlineto{\pgfqpoint{4.195480in}{1.470000in}}%
\pgfpathlineto{\pgfqpoint{4.197960in}{1.540000in}}%
\pgfpathlineto{\pgfqpoint{4.199200in}{1.610000in}}%
\pgfpathlineto{\pgfqpoint{4.201680in}{0.910000in}}%
\pgfpathlineto{\pgfqpoint{4.204160in}{1.540000in}}%
\pgfpathlineto{\pgfqpoint{4.205400in}{1.820000in}}%
\pgfpathlineto{\pgfqpoint{4.207880in}{1.295000in}}%
\pgfpathlineto{\pgfqpoint{4.209120in}{1.365000in}}%
\pgfpathlineto{\pgfqpoint{4.210360in}{1.365000in}}%
\pgfpathlineto{\pgfqpoint{4.211600in}{1.155000in}}%
\pgfpathlineto{\pgfqpoint{4.212840in}{1.155000in}}%
\pgfpathlineto{\pgfqpoint{4.214080in}{1.120000in}}%
\pgfpathlineto{\pgfqpoint{4.216560in}{1.435000in}}%
\pgfpathlineto{\pgfqpoint{4.217800in}{1.435000in}}%
\pgfpathlineto{\pgfqpoint{4.219040in}{1.085000in}}%
\pgfpathlineto{\pgfqpoint{4.220280in}{1.295000in}}%
\pgfpathlineto{\pgfqpoint{4.221520in}{1.015000in}}%
\pgfpathlineto{\pgfqpoint{4.222760in}{1.610000in}}%
\pgfpathlineto{\pgfqpoint{4.224000in}{1.645000in}}%
\pgfpathlineto{\pgfqpoint{4.225240in}{1.610000in}}%
\pgfpathlineto{\pgfqpoint{4.226480in}{1.610000in}}%
\pgfpathlineto{\pgfqpoint{4.228960in}{1.015000in}}%
\pgfpathlineto{\pgfqpoint{4.231440in}{1.575000in}}%
\pgfpathlineto{\pgfqpoint{4.233920in}{1.330000in}}%
\pgfpathlineto{\pgfqpoint{4.235160in}{1.610000in}}%
\pgfpathlineto{\pgfqpoint{4.237640in}{1.295000in}}%
\pgfpathlineto{\pgfqpoint{4.238880in}{1.540000in}}%
\pgfpathlineto{\pgfqpoint{4.240120in}{1.540000in}}%
\pgfpathlineto{\pgfqpoint{4.241360in}{0.700000in}}%
\pgfpathlineto{\pgfqpoint{4.242600in}{1.330000in}}%
\pgfpathlineto{\pgfqpoint{4.243840in}{1.295000in}}%
\pgfpathlineto{\pgfqpoint{4.245080in}{1.400000in}}%
\pgfpathlineto{\pgfqpoint{4.246320in}{1.190000in}}%
\pgfpathlineto{\pgfqpoint{4.247560in}{1.505000in}}%
\pgfpathlineto{\pgfqpoint{4.251280in}{1.120000in}}%
\pgfpathlineto{\pgfqpoint{4.252520in}{1.260000in}}%
\pgfpathlineto{\pgfqpoint{4.253760in}{1.750000in}}%
\pgfpathlineto{\pgfqpoint{4.256240in}{0.910000in}}%
\pgfpathlineto{\pgfqpoint{4.259960in}{1.575000in}}%
\pgfpathlineto{\pgfqpoint{4.262440in}{1.190000in}}%
\pgfpathlineto{\pgfqpoint{4.264920in}{1.715000in}}%
\pgfpathlineto{\pgfqpoint{4.267400in}{1.295000in}}%
\pgfpathlineto{\pgfqpoint{4.268640in}{1.505000in}}%
\pgfpathlineto{\pgfqpoint{4.269880in}{1.225000in}}%
\pgfpathlineto{\pgfqpoint{4.272360in}{1.505000in}}%
\pgfpathlineto{\pgfqpoint{4.273600in}{1.155000in}}%
\pgfpathlineto{\pgfqpoint{4.277320in}{1.715000in}}%
\pgfpathlineto{\pgfqpoint{4.278560in}{1.645000in}}%
\pgfpathlineto{\pgfqpoint{4.279800in}{1.925000in}}%
\pgfpathlineto{\pgfqpoint{4.282280in}{1.050000in}}%
\pgfpathlineto{\pgfqpoint{4.283520in}{1.400000in}}%
\pgfpathlineto{\pgfqpoint{4.284760in}{1.330000in}}%
\pgfpathlineto{\pgfqpoint{4.286000in}{1.855000in}}%
\pgfpathlineto{\pgfqpoint{4.288480in}{1.400000in}}%
\pgfpathlineto{\pgfqpoint{4.292200in}{1.925000in}}%
\pgfpathlineto{\pgfqpoint{4.294680in}{1.225000in}}%
\pgfpathlineto{\pgfqpoint{4.295920in}{1.820000in}}%
\pgfpathlineto{\pgfqpoint{4.297160in}{1.470000in}}%
\pgfpathlineto{\pgfqpoint{4.298400in}{1.820000in}}%
\pgfpathlineto{\pgfqpoint{4.300880in}{1.190000in}}%
\pgfpathlineto{\pgfqpoint{4.302120in}{1.295000in}}%
\pgfpathlineto{\pgfqpoint{4.303360in}{1.225000in}}%
\pgfpathlineto{\pgfqpoint{4.304600in}{1.085000in}}%
\pgfpathlineto{\pgfqpoint{4.305840in}{1.645000in}}%
\pgfpathlineto{\pgfqpoint{4.307080in}{1.330000in}}%
\pgfpathlineto{\pgfqpoint{4.308320in}{1.715000in}}%
\pgfpathlineto{\pgfqpoint{4.310800in}{1.365000in}}%
\pgfpathlineto{\pgfqpoint{4.312040in}{1.365000in}}%
\pgfpathlineto{\pgfqpoint{4.313280in}{1.890000in}}%
\pgfpathlineto{\pgfqpoint{4.314520in}{1.155000in}}%
\pgfpathlineto{\pgfqpoint{4.315760in}{1.680000in}}%
\pgfpathlineto{\pgfqpoint{4.317000in}{1.575000in}}%
\pgfpathlineto{\pgfqpoint{4.319480in}{1.785000in}}%
\pgfpathlineto{\pgfqpoint{4.320720in}{1.645000in}}%
\pgfpathlineto{\pgfqpoint{4.321960in}{1.715000in}}%
\pgfpathlineto{\pgfqpoint{4.323200in}{1.365000in}}%
\pgfpathlineto{\pgfqpoint{4.325680in}{1.610000in}}%
\pgfpathlineto{\pgfqpoint{4.326920in}{1.645000in}}%
\pgfpathlineto{\pgfqpoint{4.328160in}{1.120000in}}%
\pgfpathlineto{\pgfqpoint{4.329400in}{1.435000in}}%
\pgfpathlineto{\pgfqpoint{4.330640in}{1.260000in}}%
\pgfpathlineto{\pgfqpoint{4.333120in}{1.400000in}}%
\pgfpathlineto{\pgfqpoint{4.334360in}{0.945000in}}%
\pgfpathlineto{\pgfqpoint{4.335600in}{1.295000in}}%
\pgfpathlineto{\pgfqpoint{4.336840in}{1.225000in}}%
\pgfpathlineto{\pgfqpoint{4.338080in}{1.680000in}}%
\pgfpathlineto{\pgfqpoint{4.339320in}{1.330000in}}%
\pgfpathlineto{\pgfqpoint{4.340560in}{1.435000in}}%
\pgfpathlineto{\pgfqpoint{4.341800in}{1.120000in}}%
\pgfpathlineto{\pgfqpoint{4.343040in}{1.190000in}}%
\pgfpathlineto{\pgfqpoint{4.344280in}{0.665000in}}%
\pgfpathlineto{\pgfqpoint{4.345520in}{1.260000in}}%
\pgfpathlineto{\pgfqpoint{4.346760in}{1.050000in}}%
\pgfpathlineto{\pgfqpoint{4.350480in}{1.575000in}}%
\pgfpathlineto{\pgfqpoint{4.352960in}{1.225000in}}%
\pgfpathlineto{\pgfqpoint{4.354200in}{1.890000in}}%
\pgfpathlineto{\pgfqpoint{4.357920in}{1.400000in}}%
\pgfpathlineto{\pgfqpoint{4.359160in}{1.890000in}}%
\pgfpathlineto{\pgfqpoint{4.360400in}{1.505000in}}%
\pgfpathlineto{\pgfqpoint{4.361640in}{1.575000in}}%
\pgfpathlineto{\pgfqpoint{4.362880in}{1.085000in}}%
\pgfpathlineto{\pgfqpoint{4.364120in}{1.155000in}}%
\pgfpathlineto{\pgfqpoint{4.365360in}{1.330000in}}%
\pgfpathlineto{\pgfqpoint{4.366600in}{0.980000in}}%
\pgfpathlineto{\pgfqpoint{4.370320in}{1.575000in}}%
\pgfpathlineto{\pgfqpoint{4.371560in}{1.505000in}}%
\pgfpathlineto{\pgfqpoint{4.372800in}{1.365000in}}%
\pgfpathlineto{\pgfqpoint{4.374040in}{1.085000in}}%
\pgfpathlineto{\pgfqpoint{4.375280in}{1.260000in}}%
\pgfpathlineto{\pgfqpoint{4.376520in}{1.610000in}}%
\pgfpathlineto{\pgfqpoint{4.377760in}{1.610000in}}%
\pgfpathlineto{\pgfqpoint{4.380240in}{1.330000in}}%
\pgfpathlineto{\pgfqpoint{4.381480in}{1.435000in}}%
\pgfpathlineto{\pgfqpoint{4.382720in}{1.400000in}}%
\pgfpathlineto{\pgfqpoint{4.383960in}{1.400000in}}%
\pgfpathlineto{\pgfqpoint{4.385200in}{1.610000in}}%
\pgfpathlineto{\pgfqpoint{4.386440in}{1.260000in}}%
\pgfpathlineto{\pgfqpoint{4.387680in}{1.715000in}}%
\pgfpathlineto{\pgfqpoint{4.391400in}{1.225000in}}%
\pgfpathlineto{\pgfqpoint{4.392640in}{1.470000in}}%
\pgfpathlineto{\pgfqpoint{4.393880in}{1.365000in}}%
\pgfpathlineto{\pgfqpoint{4.395120in}{1.820000in}}%
\pgfpathlineto{\pgfqpoint{4.397600in}{1.365000in}}%
\pgfpathlineto{\pgfqpoint{4.398840in}{1.295000in}}%
\pgfpathlineto{\pgfqpoint{4.400080in}{1.575000in}}%
\pgfpathlineto{\pgfqpoint{4.402560in}{1.330000in}}%
\pgfpathlineto{\pgfqpoint{4.403800in}{1.925000in}}%
\pgfpathlineto{\pgfqpoint{4.406280in}{1.225000in}}%
\pgfpathlineto{\pgfqpoint{4.407520in}{1.435000in}}%
\pgfpathlineto{\pgfqpoint{4.408760in}{1.400000in}}%
\pgfpathlineto{\pgfqpoint{4.410000in}{1.680000in}}%
\pgfpathlineto{\pgfqpoint{4.411240in}{1.470000in}}%
\pgfpathlineto{\pgfqpoint{4.413720in}{2.205000in}}%
\pgfpathlineto{\pgfqpoint{4.414960in}{1.645000in}}%
\pgfpathlineto{\pgfqpoint{4.416200in}{1.680000in}}%
\pgfpathlineto{\pgfqpoint{4.418680in}{1.190000in}}%
\pgfpathlineto{\pgfqpoint{4.421160in}{2.170000in}}%
\pgfpathlineto{\pgfqpoint{4.423640in}{1.820000in}}%
\pgfpathlineto{\pgfqpoint{4.424880in}{1.715000in}}%
\pgfpathlineto{\pgfqpoint{4.426120in}{1.155000in}}%
\pgfpathlineto{\pgfqpoint{4.428600in}{1.890000in}}%
\pgfpathlineto{\pgfqpoint{4.429840in}{1.470000in}}%
\pgfpathlineto{\pgfqpoint{4.432320in}{1.995000in}}%
\pgfpathlineto{\pgfqpoint{4.434800in}{1.890000in}}%
\pgfpathlineto{\pgfqpoint{4.436040in}{1.925000in}}%
\pgfpathlineto{\pgfqpoint{4.437280in}{1.820000in}}%
\pgfpathlineto{\pgfqpoint{4.439760in}{1.190000in}}%
\pgfpathlineto{\pgfqpoint{4.442240in}{1.680000in}}%
\pgfpathlineto{\pgfqpoint{4.443480in}{1.540000in}}%
\pgfpathlineto{\pgfqpoint{4.444720in}{1.190000in}}%
\pgfpathlineto{\pgfqpoint{4.445960in}{1.330000in}}%
\pgfpathlineto{\pgfqpoint{4.447200in}{1.085000in}}%
\pgfpathlineto{\pgfqpoint{4.448440in}{1.260000in}}%
\pgfpathlineto{\pgfqpoint{4.449680in}{1.610000in}}%
\pgfpathlineto{\pgfqpoint{4.452160in}{1.015000in}}%
\pgfpathlineto{\pgfqpoint{4.453400in}{2.030000in}}%
\pgfpathlineto{\pgfqpoint{4.454640in}{1.225000in}}%
\pgfpathlineto{\pgfqpoint{4.455880in}{1.505000in}}%
\pgfpathlineto{\pgfqpoint{4.457120in}{1.015000in}}%
\pgfpathlineto{\pgfqpoint{4.459600in}{1.610000in}}%
\pgfpathlineto{\pgfqpoint{4.460840in}{1.050000in}}%
\pgfpathlineto{\pgfqpoint{4.462080in}{1.225000in}}%
\pgfpathlineto{\pgfqpoint{4.464560in}{1.750000in}}%
\pgfpathlineto{\pgfqpoint{4.465800in}{1.435000in}}%
\pgfpathlineto{\pgfqpoint{4.467040in}{1.645000in}}%
\pgfpathlineto{\pgfqpoint{4.468280in}{1.295000in}}%
\pgfpathlineto{\pgfqpoint{4.469520in}{1.435000in}}%
\pgfpathlineto{\pgfqpoint{4.470760in}{1.400000in}}%
\pgfpathlineto{\pgfqpoint{4.472000in}{1.680000in}}%
\pgfpathlineto{\pgfqpoint{4.474480in}{0.945000in}}%
\pgfpathlineto{\pgfqpoint{4.475720in}{1.050000in}}%
\pgfpathlineto{\pgfqpoint{4.476960in}{1.400000in}}%
\pgfpathlineto{\pgfqpoint{4.478200in}{1.365000in}}%
\pgfpathlineto{\pgfqpoint{4.479440in}{1.365000in}}%
\pgfpathlineto{\pgfqpoint{4.480680in}{1.190000in}}%
\pgfpathlineto{\pgfqpoint{4.481920in}{1.785000in}}%
\pgfpathlineto{\pgfqpoint{4.484400in}{1.295000in}}%
\pgfpathlineto{\pgfqpoint{4.485640in}{1.365000in}}%
\pgfpathlineto{\pgfqpoint{4.488120in}{0.980000in}}%
\pgfpathlineto{\pgfqpoint{4.490600in}{1.330000in}}%
\pgfpathlineto{\pgfqpoint{4.491840in}{1.400000in}}%
\pgfpathlineto{\pgfqpoint{4.494320in}{1.050000in}}%
\pgfpathlineto{\pgfqpoint{4.496800in}{1.610000in}}%
\pgfpathlineto{\pgfqpoint{4.499280in}{0.980000in}}%
\pgfpathlineto{\pgfqpoint{4.500520in}{1.190000in}}%
\pgfpathlineto{\pgfqpoint{4.501760in}{1.085000in}}%
\pgfpathlineto{\pgfqpoint{4.505480in}{1.715000in}}%
\pgfpathlineto{\pgfqpoint{4.509200in}{1.190000in}}%
\pgfpathlineto{\pgfqpoint{4.510440in}{1.575000in}}%
\pgfpathlineto{\pgfqpoint{4.512920in}{1.015000in}}%
\pgfpathlineto{\pgfqpoint{4.514160in}{0.980000in}}%
\pgfpathlineto{\pgfqpoint{4.516640in}{1.540000in}}%
\pgfpathlineto{\pgfqpoint{4.517880in}{1.155000in}}%
\pgfpathlineto{\pgfqpoint{4.520360in}{1.995000in}}%
\pgfpathlineto{\pgfqpoint{4.521600in}{1.575000in}}%
\pgfpathlineto{\pgfqpoint{4.522840in}{1.575000in}}%
\pgfpathlineto{\pgfqpoint{4.524080in}{1.645000in}}%
\pgfpathlineto{\pgfqpoint{4.525320in}{0.875000in}}%
\pgfpathlineto{\pgfqpoint{4.526560in}{1.785000in}}%
\pgfpathlineto{\pgfqpoint{4.527800in}{1.470000in}}%
\pgfpathlineto{\pgfqpoint{4.529040in}{1.610000in}}%
\pgfpathlineto{\pgfqpoint{4.531520in}{1.085000in}}%
\pgfpathlineto{\pgfqpoint{4.532760in}{1.365000in}}%
\pgfpathlineto{\pgfqpoint{4.534000in}{1.155000in}}%
\pgfpathlineto{\pgfqpoint{4.535240in}{1.750000in}}%
\pgfpathlineto{\pgfqpoint{4.536480in}{1.295000in}}%
\pgfpathlineto{\pgfqpoint{4.537720in}{1.400000in}}%
\pgfpathlineto{\pgfqpoint{4.538960in}{1.365000in}}%
\pgfpathlineto{\pgfqpoint{4.540200in}{1.365000in}}%
\pgfpathlineto{\pgfqpoint{4.542680in}{1.085000in}}%
\pgfpathlineto{\pgfqpoint{4.545160in}{1.470000in}}%
\pgfpathlineto{\pgfqpoint{4.546400in}{1.400000in}}%
\pgfpathlineto{\pgfqpoint{4.547640in}{0.980000in}}%
\pgfpathlineto{\pgfqpoint{4.548880in}{1.330000in}}%
\pgfpathlineto{\pgfqpoint{4.550120in}{1.330000in}}%
\pgfpathlineto{\pgfqpoint{4.551360in}{1.295000in}}%
\pgfpathlineto{\pgfqpoint{4.552600in}{1.015000in}}%
\pgfpathlineto{\pgfqpoint{4.555080in}{1.540000in}}%
\pgfpathlineto{\pgfqpoint{4.557560in}{1.435000in}}%
\pgfpathlineto{\pgfqpoint{4.558800in}{1.575000in}}%
\pgfpathlineto{\pgfqpoint{4.560040in}{1.330000in}}%
\pgfpathlineto{\pgfqpoint{4.561280in}{1.330000in}}%
\pgfpathlineto{\pgfqpoint{4.562520in}{1.715000in}}%
\pgfpathlineto{\pgfqpoint{4.563760in}{1.610000in}}%
\pgfpathlineto{\pgfqpoint{4.565000in}{1.645000in}}%
\pgfpathlineto{\pgfqpoint{4.566240in}{1.050000in}}%
\pgfpathlineto{\pgfqpoint{4.567480in}{1.610000in}}%
\pgfpathlineto{\pgfqpoint{4.568720in}{1.575000in}}%
\pgfpathlineto{\pgfqpoint{4.569960in}{1.645000in}}%
\pgfpathlineto{\pgfqpoint{4.571200in}{1.785000in}}%
\pgfpathlineto{\pgfqpoint{4.573680in}{1.295000in}}%
\pgfpathlineto{\pgfqpoint{4.576160in}{1.610000in}}%
\pgfpathlineto{\pgfqpoint{4.577400in}{1.505000in}}%
\pgfpathlineto{\pgfqpoint{4.578640in}{1.085000in}}%
\pgfpathlineto{\pgfqpoint{4.579880in}{1.540000in}}%
\pgfpathlineto{\pgfqpoint{4.582360in}{1.050000in}}%
\pgfpathlineto{\pgfqpoint{4.583600in}{1.120000in}}%
\pgfpathlineto{\pgfqpoint{4.586080in}{1.750000in}}%
\pgfpathlineto{\pgfqpoint{4.587320in}{1.330000in}}%
\pgfpathlineto{\pgfqpoint{4.589800in}{1.785000in}}%
\pgfpathlineto{\pgfqpoint{4.591040in}{1.750000in}}%
\pgfpathlineto{\pgfqpoint{4.592280in}{1.540000in}}%
\pgfpathlineto{\pgfqpoint{4.593520in}{1.540000in}}%
\pgfpathlineto{\pgfqpoint{4.594760in}{1.435000in}}%
\pgfpathlineto{\pgfqpoint{4.596000in}{1.645000in}}%
\pgfpathlineto{\pgfqpoint{4.597240in}{1.295000in}}%
\pgfpathlineto{\pgfqpoint{4.598480in}{1.365000in}}%
\pgfpathlineto{\pgfqpoint{4.599720in}{1.015000in}}%
\pgfpathlineto{\pgfqpoint{4.602200in}{1.470000in}}%
\pgfpathlineto{\pgfqpoint{4.603440in}{1.365000in}}%
\pgfpathlineto{\pgfqpoint{4.604680in}{1.400000in}}%
\pgfpathlineto{\pgfqpoint{4.607160in}{1.190000in}}%
\pgfpathlineto{\pgfqpoint{4.608400in}{1.190000in}}%
\pgfpathlineto{\pgfqpoint{4.610880in}{1.645000in}}%
\pgfpathlineto{\pgfqpoint{4.612120in}{1.085000in}}%
\pgfpathlineto{\pgfqpoint{4.613360in}{1.085000in}}%
\pgfpathlineto{\pgfqpoint{4.614600in}{1.505000in}}%
\pgfpathlineto{\pgfqpoint{4.615840in}{1.260000in}}%
\pgfpathlineto{\pgfqpoint{4.618320in}{1.575000in}}%
\pgfpathlineto{\pgfqpoint{4.619560in}{1.400000in}}%
\pgfpathlineto{\pgfqpoint{4.620800in}{1.715000in}}%
\pgfpathlineto{\pgfqpoint{4.622040in}{1.400000in}}%
\pgfpathlineto{\pgfqpoint{4.623280in}{1.750000in}}%
\pgfpathlineto{\pgfqpoint{4.625760in}{1.435000in}}%
\pgfpathlineto{\pgfqpoint{4.627000in}{1.260000in}}%
\pgfpathlineto{\pgfqpoint{4.628240in}{1.540000in}}%
\pgfpathlineto{\pgfqpoint{4.629480in}{1.295000in}}%
\pgfpathlineto{\pgfqpoint{4.630720in}{0.805000in}}%
\pgfpathlineto{\pgfqpoint{4.631960in}{1.645000in}}%
\pgfpathlineto{\pgfqpoint{4.633200in}{1.610000in}}%
\pgfpathlineto{\pgfqpoint{4.634440in}{1.330000in}}%
\pgfpathlineto{\pgfqpoint{4.635680in}{1.365000in}}%
\pgfpathlineto{\pgfqpoint{4.636920in}{0.980000in}}%
\pgfpathlineto{\pgfqpoint{4.638160in}{1.120000in}}%
\pgfpathlineto{\pgfqpoint{4.639400in}{1.505000in}}%
\pgfpathlineto{\pgfqpoint{4.641880in}{0.980000in}}%
\pgfpathlineto{\pgfqpoint{4.643120in}{1.120000in}}%
\pgfpathlineto{\pgfqpoint{4.644360in}{0.665000in}}%
\pgfpathlineto{\pgfqpoint{4.645600in}{1.400000in}}%
\pgfpathlineto{\pgfqpoint{4.646840in}{1.400000in}}%
\pgfpathlineto{\pgfqpoint{4.649320in}{1.890000in}}%
\pgfpathlineto{\pgfqpoint{4.650560in}{1.120000in}}%
\pgfpathlineto{\pgfqpoint{4.651800in}{1.365000in}}%
\pgfpathlineto{\pgfqpoint{4.653040in}{1.050000in}}%
\pgfpathlineto{\pgfqpoint{4.654280in}{1.715000in}}%
\pgfpathlineto{\pgfqpoint{4.655520in}{1.750000in}}%
\pgfpathlineto{\pgfqpoint{4.656760in}{1.715000in}}%
\pgfpathlineto{\pgfqpoint{4.658000in}{1.820000in}}%
\pgfpathlineto{\pgfqpoint{4.659240in}{1.435000in}}%
\pgfpathlineto{\pgfqpoint{4.660480in}{1.505000in}}%
\pgfpathlineto{\pgfqpoint{4.661720in}{1.400000in}}%
\pgfpathlineto{\pgfqpoint{4.662960in}{1.855000in}}%
\pgfpathlineto{\pgfqpoint{4.664200in}{1.470000in}}%
\pgfpathlineto{\pgfqpoint{4.665440in}{1.645000in}}%
\pgfpathlineto{\pgfqpoint{4.666680in}{1.610000in}}%
\pgfpathlineto{\pgfqpoint{4.667920in}{1.295000in}}%
\pgfpathlineto{\pgfqpoint{4.669160in}{1.645000in}}%
\pgfpathlineto{\pgfqpoint{4.670400in}{1.505000in}}%
\pgfpathlineto{\pgfqpoint{4.671640in}{0.945000in}}%
\pgfpathlineto{\pgfqpoint{4.672880in}{1.680000in}}%
\pgfpathlineto{\pgfqpoint{4.674120in}{1.260000in}}%
\pgfpathlineto{\pgfqpoint{4.676600in}{1.855000in}}%
\pgfpathlineto{\pgfqpoint{4.679080in}{1.575000in}}%
\pgfpathlineto{\pgfqpoint{4.680320in}{1.050000in}}%
\pgfpathlineto{\pgfqpoint{4.682800in}{1.890000in}}%
\pgfpathlineto{\pgfqpoint{4.684040in}{1.925000in}}%
\pgfpathlineto{\pgfqpoint{4.685280in}{1.295000in}}%
\pgfpathlineto{\pgfqpoint{4.686520in}{1.295000in}}%
\pgfpathlineto{\pgfqpoint{4.689000in}{1.575000in}}%
\pgfpathlineto{\pgfqpoint{4.690240in}{1.190000in}}%
\pgfpathlineto{\pgfqpoint{4.691480in}{1.435000in}}%
\pgfpathlineto{\pgfqpoint{4.692720in}{1.120000in}}%
\pgfpathlineto{\pgfqpoint{4.695200in}{1.855000in}}%
\pgfpathlineto{\pgfqpoint{4.696440in}{1.820000in}}%
\pgfpathlineto{\pgfqpoint{4.697680in}{1.750000in}}%
\pgfpathlineto{\pgfqpoint{4.698920in}{1.470000in}}%
\pgfpathlineto{\pgfqpoint{4.700160in}{1.610000in}}%
\pgfpathlineto{\pgfqpoint{4.701400in}{1.365000in}}%
\pgfpathlineto{\pgfqpoint{4.702640in}{1.540000in}}%
\pgfpathlineto{\pgfqpoint{4.703880in}{1.225000in}}%
\pgfpathlineto{\pgfqpoint{4.705120in}{1.505000in}}%
\pgfpathlineto{\pgfqpoint{4.706360in}{0.980000in}}%
\pgfpathlineto{\pgfqpoint{4.707600in}{1.680000in}}%
\pgfpathlineto{\pgfqpoint{4.708840in}{1.365000in}}%
\pgfpathlineto{\pgfqpoint{4.710080in}{1.435000in}}%
\pgfpathlineto{\pgfqpoint{4.711320in}{1.925000in}}%
\pgfpathlineto{\pgfqpoint{4.712560in}{1.435000in}}%
\pgfpathlineto{\pgfqpoint{4.713800in}{1.540000in}}%
\pgfpathlineto{\pgfqpoint{4.715040in}{1.155000in}}%
\pgfpathlineto{\pgfqpoint{4.716280in}{1.995000in}}%
\pgfpathlineto{\pgfqpoint{4.718760in}{1.400000in}}%
\pgfpathlineto{\pgfqpoint{4.720000in}{1.155000in}}%
\pgfpathlineto{\pgfqpoint{4.721240in}{1.750000in}}%
\pgfpathlineto{\pgfqpoint{4.723720in}{1.330000in}}%
\pgfpathlineto{\pgfqpoint{4.724960in}{1.365000in}}%
\pgfpathlineto{\pgfqpoint{4.726200in}{1.190000in}}%
\pgfpathlineto{\pgfqpoint{4.727440in}{1.260000in}}%
\pgfpathlineto{\pgfqpoint{4.728680in}{1.890000in}}%
\pgfpathlineto{\pgfqpoint{4.729920in}{1.365000in}}%
\pgfpathlineto{\pgfqpoint{4.731160in}{1.960000in}}%
\pgfpathlineto{\pgfqpoint{4.733640in}{1.610000in}}%
\pgfpathlineto{\pgfqpoint{4.736120in}{1.120000in}}%
\pgfpathlineto{\pgfqpoint{4.737360in}{1.050000in}}%
\pgfpathlineto{\pgfqpoint{4.738600in}{1.365000in}}%
\pgfpathlineto{\pgfqpoint{4.739840in}{1.330000in}}%
\pgfpathlineto{\pgfqpoint{4.741080in}{1.365000in}}%
\pgfpathlineto{\pgfqpoint{4.742320in}{1.330000in}}%
\pgfpathlineto{\pgfqpoint{4.743560in}{1.120000in}}%
\pgfpathlineto{\pgfqpoint{4.746040in}{1.785000in}}%
\pgfpathlineto{\pgfqpoint{4.747280in}{1.855000in}}%
\pgfpathlineto{\pgfqpoint{4.748520in}{1.435000in}}%
\pgfpathlineto{\pgfqpoint{4.749760in}{1.820000in}}%
\pgfpathlineto{\pgfqpoint{4.751000in}{1.610000in}}%
\pgfpathlineto{\pgfqpoint{4.752240in}{1.785000in}}%
\pgfpathlineto{\pgfqpoint{4.755960in}{1.435000in}}%
\pgfpathlineto{\pgfqpoint{4.757200in}{1.155000in}}%
\pgfpathlineto{\pgfqpoint{4.758440in}{1.645000in}}%
\pgfpathlineto{\pgfqpoint{4.760920in}{1.085000in}}%
\pgfpathlineto{\pgfqpoint{4.763400in}{1.645000in}}%
\pgfpathlineto{\pgfqpoint{4.764640in}{1.645000in}}%
\pgfpathlineto{\pgfqpoint{4.767120in}{1.085000in}}%
\pgfpathlineto{\pgfqpoint{4.769600in}{1.610000in}}%
\pgfpathlineto{\pgfqpoint{4.770840in}{1.330000in}}%
\pgfpathlineto{\pgfqpoint{4.773320in}{1.610000in}}%
\pgfpathlineto{\pgfqpoint{4.775800in}{1.295000in}}%
\pgfpathlineto{\pgfqpoint{4.779520in}{1.540000in}}%
\pgfpathlineto{\pgfqpoint{4.780760in}{1.365000in}}%
\pgfpathlineto{\pgfqpoint{4.782000in}{1.645000in}}%
\pgfpathlineto{\pgfqpoint{4.783240in}{1.015000in}}%
\pgfpathlineto{\pgfqpoint{4.784480in}{1.540000in}}%
\pgfpathlineto{\pgfqpoint{4.785720in}{1.295000in}}%
\pgfpathlineto{\pgfqpoint{4.786960in}{1.610000in}}%
\pgfpathlineto{\pgfqpoint{4.788200in}{1.365000in}}%
\pgfpathlineto{\pgfqpoint{4.790680in}{0.665000in}}%
\pgfpathlineto{\pgfqpoint{4.791920in}{1.400000in}}%
\pgfpathlineto{\pgfqpoint{4.794400in}{1.575000in}}%
\pgfpathlineto{\pgfqpoint{4.795640in}{1.890000in}}%
\pgfpathlineto{\pgfqpoint{4.796880in}{1.715000in}}%
\pgfpathlineto{\pgfqpoint{4.798120in}{1.365000in}}%
\pgfpathlineto{\pgfqpoint{4.799360in}{1.540000in}}%
\pgfpathlineto{\pgfqpoint{4.800600in}{0.980000in}}%
\pgfpathlineto{\pgfqpoint{4.801840in}{1.575000in}}%
\pgfpathlineto{\pgfqpoint{4.803080in}{1.435000in}}%
\pgfpathlineto{\pgfqpoint{4.804320in}{1.610000in}}%
\pgfpathlineto{\pgfqpoint{4.805560in}{1.015000in}}%
\pgfpathlineto{\pgfqpoint{4.806800in}{1.400000in}}%
\pgfpathlineto{\pgfqpoint{4.808040in}{1.295000in}}%
\pgfpathlineto{\pgfqpoint{4.810520in}{0.875000in}}%
\pgfpathlineto{\pgfqpoint{4.811760in}{1.680000in}}%
\pgfpathlineto{\pgfqpoint{4.813000in}{1.680000in}}%
\pgfpathlineto{\pgfqpoint{4.815480in}{1.330000in}}%
\pgfpathlineto{\pgfqpoint{4.817960in}{0.910000in}}%
\pgfpathlineto{\pgfqpoint{4.820440in}{1.750000in}}%
\pgfpathlineto{\pgfqpoint{4.822920in}{1.155000in}}%
\pgfpathlineto{\pgfqpoint{4.824160in}{0.875000in}}%
\pgfpathlineto{\pgfqpoint{4.826640in}{1.540000in}}%
\pgfpathlineto{\pgfqpoint{4.827880in}{1.190000in}}%
\pgfpathlineto{\pgfqpoint{4.830360in}{1.785000in}}%
\pgfpathlineto{\pgfqpoint{4.831600in}{1.575000in}}%
\pgfpathlineto{\pgfqpoint{4.832840in}{1.890000in}}%
\pgfpathlineto{\pgfqpoint{4.834080in}{1.470000in}}%
\pgfpathlineto{\pgfqpoint{4.835320in}{1.610000in}}%
\pgfpathlineto{\pgfqpoint{4.836560in}{1.225000in}}%
\pgfpathlineto{\pgfqpoint{4.837800in}{1.610000in}}%
\pgfpathlineto{\pgfqpoint{4.839040in}{1.365000in}}%
\pgfpathlineto{\pgfqpoint{4.841520in}{1.575000in}}%
\pgfpathlineto{\pgfqpoint{4.842760in}{1.540000in}}%
\pgfpathlineto{\pgfqpoint{4.844000in}{1.155000in}}%
\pgfpathlineto{\pgfqpoint{4.846480in}{1.785000in}}%
\pgfpathlineto{\pgfqpoint{4.847720in}{1.400000in}}%
\pgfpathlineto{\pgfqpoint{4.851440in}{1.960000in}}%
\pgfpathlineto{\pgfqpoint{4.852680in}{1.890000in}}%
\pgfpathlineto{\pgfqpoint{4.855160in}{1.015000in}}%
\pgfpathlineto{\pgfqpoint{4.856400in}{1.470000in}}%
\pgfpathlineto{\pgfqpoint{4.857640in}{1.050000in}}%
\pgfpathlineto{\pgfqpoint{4.858880in}{1.085000in}}%
\pgfpathlineto{\pgfqpoint{4.861360in}{1.505000in}}%
\pgfpathlineto{\pgfqpoint{4.862600in}{1.470000in}}%
\pgfpathlineto{\pgfqpoint{4.863840in}{1.890000in}}%
\pgfpathlineto{\pgfqpoint{4.865080in}{1.330000in}}%
\pgfpathlineto{\pgfqpoint{4.866320in}{1.785000in}}%
\pgfpathlineto{\pgfqpoint{4.867560in}{1.820000in}}%
\pgfpathlineto{\pgfqpoint{4.868800in}{1.400000in}}%
\pgfpathlineto{\pgfqpoint{4.870040in}{1.610000in}}%
\pgfpathlineto{\pgfqpoint{4.871280in}{1.610000in}}%
\pgfpathlineto{\pgfqpoint{4.872520in}{1.890000in}}%
\pgfpathlineto{\pgfqpoint{4.873760in}{1.890000in}}%
\pgfpathlineto{\pgfqpoint{4.875000in}{1.610000in}}%
\pgfpathlineto{\pgfqpoint{4.876240in}{1.785000in}}%
\pgfpathlineto{\pgfqpoint{4.878720in}{1.225000in}}%
\pgfpathlineto{\pgfqpoint{4.879960in}{1.435000in}}%
\pgfpathlineto{\pgfqpoint{4.881200in}{1.960000in}}%
\pgfpathlineto{\pgfqpoint{4.882440in}{1.365000in}}%
\pgfpathlineto{\pgfqpoint{4.884920in}{1.855000in}}%
\pgfpathlineto{\pgfqpoint{4.886160in}{1.680000in}}%
\pgfpathlineto{\pgfqpoint{4.887400in}{1.960000in}}%
\pgfpathlineto{\pgfqpoint{4.888640in}{1.890000in}}%
\pgfpathlineto{\pgfqpoint{4.889880in}{2.030000in}}%
\pgfpathlineto{\pgfqpoint{4.891120in}{1.470000in}}%
\pgfpathlineto{\pgfqpoint{4.892360in}{1.680000in}}%
\pgfpathlineto{\pgfqpoint{4.893600in}{1.505000in}}%
\pgfpathlineto{\pgfqpoint{4.894840in}{1.715000in}}%
\pgfpathlineto{\pgfqpoint{4.896080in}{1.470000in}}%
\pgfpathlineto{\pgfqpoint{4.897320in}{1.645000in}}%
\pgfpathlineto{\pgfqpoint{4.898560in}{1.610000in}}%
\pgfpathlineto{\pgfqpoint{4.899800in}{1.610000in}}%
\pgfpathlineto{\pgfqpoint{4.902280in}{1.400000in}}%
\pgfpathlineto{\pgfqpoint{4.903520in}{1.435000in}}%
\pgfpathlineto{\pgfqpoint{4.904760in}{1.540000in}}%
\pgfpathlineto{\pgfqpoint{4.906000in}{1.260000in}}%
\pgfpathlineto{\pgfqpoint{4.908480in}{1.610000in}}%
\pgfpathlineto{\pgfqpoint{4.912200in}{1.085000in}}%
\pgfpathlineto{\pgfqpoint{4.914680in}{1.505000in}}%
\pgfpathlineto{\pgfqpoint{4.915920in}{1.330000in}}%
\pgfpathlineto{\pgfqpoint{4.917160in}{1.575000in}}%
\pgfpathlineto{\pgfqpoint{4.918400in}{1.575000in}}%
\pgfpathlineto{\pgfqpoint{4.919640in}{1.330000in}}%
\pgfpathlineto{\pgfqpoint{4.920880in}{1.505000in}}%
\pgfpathlineto{\pgfqpoint{4.922120in}{1.855000in}}%
\pgfpathlineto{\pgfqpoint{4.923360in}{1.785000in}}%
\pgfpathlineto{\pgfqpoint{4.924600in}{1.995000in}}%
\pgfpathlineto{\pgfqpoint{4.925840in}{1.330000in}}%
\pgfpathlineto{\pgfqpoint{4.927080in}{1.400000in}}%
\pgfpathlineto{\pgfqpoint{4.928320in}{1.330000in}}%
\pgfpathlineto{\pgfqpoint{4.929560in}{1.015000in}}%
\pgfpathlineto{\pgfqpoint{4.930800in}{1.470000in}}%
\pgfpathlineto{\pgfqpoint{4.932040in}{0.875000in}}%
\pgfpathlineto{\pgfqpoint{4.933280in}{1.505000in}}%
\pgfpathlineto{\pgfqpoint{4.934520in}{1.050000in}}%
\pgfpathlineto{\pgfqpoint{4.937000in}{2.100000in}}%
\pgfpathlineto{\pgfqpoint{4.938240in}{1.225000in}}%
\pgfpathlineto{\pgfqpoint{4.939480in}{1.610000in}}%
\pgfpathlineto{\pgfqpoint{4.941960in}{1.225000in}}%
\pgfpathlineto{\pgfqpoint{4.944440in}{1.575000in}}%
\pgfpathlineto{\pgfqpoint{4.945680in}{2.065000in}}%
\pgfpathlineto{\pgfqpoint{4.946920in}{1.400000in}}%
\pgfpathlineto{\pgfqpoint{4.948160in}{1.400000in}}%
\pgfpathlineto{\pgfqpoint{4.949400in}{1.715000in}}%
\pgfpathlineto{\pgfqpoint{4.950640in}{1.575000in}}%
\pgfpathlineto{\pgfqpoint{4.951880in}{1.995000in}}%
\pgfpathlineto{\pgfqpoint{4.955600in}{1.435000in}}%
\pgfpathlineto{\pgfqpoint{4.958080in}{1.610000in}}%
\pgfpathlineto{\pgfqpoint{4.959320in}{1.015000in}}%
\pgfpathlineto{\pgfqpoint{4.961800in}{1.785000in}}%
\pgfpathlineto{\pgfqpoint{4.963040in}{1.435000in}}%
\pgfpathlineto{\pgfqpoint{4.964280in}{1.680000in}}%
\pgfpathlineto{\pgfqpoint{4.965520in}{1.645000in}}%
\pgfpathlineto{\pgfqpoint{4.968000in}{1.820000in}}%
\pgfpathlineto{\pgfqpoint{4.969240in}{1.890000in}}%
\pgfpathlineto{\pgfqpoint{4.971720in}{1.400000in}}%
\pgfpathlineto{\pgfqpoint{4.974200in}{1.645000in}}%
\pgfpathlineto{\pgfqpoint{4.976680in}{1.330000in}}%
\pgfpathlineto{\pgfqpoint{4.977920in}{1.890000in}}%
\pgfpathlineto{\pgfqpoint{4.980400in}{1.295000in}}%
\pgfpathlineto{\pgfqpoint{4.981640in}{1.785000in}}%
\pgfpathlineto{\pgfqpoint{4.984120in}{1.085000in}}%
\pgfpathlineto{\pgfqpoint{4.985360in}{1.295000in}}%
\pgfpathlineto{\pgfqpoint{4.986600in}{1.260000in}}%
\pgfpathlineto{\pgfqpoint{4.987840in}{1.085000in}}%
\pgfpathlineto{\pgfqpoint{4.989080in}{1.995000in}}%
\pgfpathlineto{\pgfqpoint{4.990320in}{0.980000in}}%
\pgfpathlineto{\pgfqpoint{4.994040in}{1.680000in}}%
\pgfpathlineto{\pgfqpoint{4.995280in}{1.225000in}}%
\pgfpathlineto{\pgfqpoint{4.996520in}{1.260000in}}%
\pgfpathlineto{\pgfqpoint{4.997760in}{1.715000in}}%
\pgfpathlineto{\pgfqpoint{4.999000in}{1.540000in}}%
\pgfpathlineto{\pgfqpoint{5.000240in}{1.540000in}}%
\pgfpathlineto{\pgfqpoint{5.001480in}{1.190000in}}%
\pgfpathlineto{\pgfqpoint{5.002720in}{1.435000in}}%
\pgfpathlineto{\pgfqpoint{5.006440in}{1.015000in}}%
\pgfpathlineto{\pgfqpoint{5.007680in}{1.645000in}}%
\pgfpathlineto{\pgfqpoint{5.008920in}{1.365000in}}%
\pgfpathlineto{\pgfqpoint{5.010160in}{1.575000in}}%
\pgfpathlineto{\pgfqpoint{5.011400in}{1.365000in}}%
\pgfpathlineto{\pgfqpoint{5.012640in}{1.715000in}}%
\pgfpathlineto{\pgfqpoint{5.015120in}{1.050000in}}%
\pgfpathlineto{\pgfqpoint{5.016360in}{1.190000in}}%
\pgfpathlineto{\pgfqpoint{5.017600in}{0.910000in}}%
\pgfpathlineto{\pgfqpoint{5.018840in}{1.330000in}}%
\pgfpathlineto{\pgfqpoint{5.020080in}{1.120000in}}%
\pgfpathlineto{\pgfqpoint{5.021320in}{1.120000in}}%
\pgfpathlineto{\pgfqpoint{5.023800in}{1.575000in}}%
\pgfpathlineto{\pgfqpoint{5.025040in}{1.120000in}}%
\pgfpathlineto{\pgfqpoint{5.026280in}{1.295000in}}%
\pgfpathlineto{\pgfqpoint{5.027520in}{1.155000in}}%
\pgfpathlineto{\pgfqpoint{5.028760in}{0.735000in}}%
\pgfpathlineto{\pgfqpoint{5.031240in}{1.680000in}}%
\pgfpathlineto{\pgfqpoint{5.032480in}{1.295000in}}%
\pgfpathlineto{\pgfqpoint{5.033720in}{1.435000in}}%
\pgfpathlineto{\pgfqpoint{5.036200in}{1.995000in}}%
\pgfpathlineto{\pgfqpoint{5.037440in}{1.890000in}}%
\pgfpathlineto{\pgfqpoint{5.038680in}{1.645000in}}%
\pgfpathlineto{\pgfqpoint{5.039920in}{1.715000in}}%
\pgfpathlineto{\pgfqpoint{5.041160in}{1.680000in}}%
\pgfpathlineto{\pgfqpoint{5.042400in}{1.820000in}}%
\pgfpathlineto{\pgfqpoint{5.044880in}{1.330000in}}%
\pgfpathlineto{\pgfqpoint{5.046120in}{1.400000in}}%
\pgfpathlineto{\pgfqpoint{5.047360in}{0.910000in}}%
\pgfpathlineto{\pgfqpoint{5.048600in}{1.855000in}}%
\pgfpathlineto{\pgfqpoint{5.049840in}{1.505000in}}%
\pgfpathlineto{\pgfqpoint{5.051080in}{1.470000in}}%
\pgfpathlineto{\pgfqpoint{5.052320in}{1.295000in}}%
\pgfpathlineto{\pgfqpoint{5.053560in}{1.820000in}}%
\pgfpathlineto{\pgfqpoint{5.054800in}{1.260000in}}%
\pgfpathlineto{\pgfqpoint{5.056040in}{1.470000in}}%
\pgfpathlineto{\pgfqpoint{5.057280in}{1.890000in}}%
\pgfpathlineto{\pgfqpoint{5.058520in}{1.260000in}}%
\pgfpathlineto{\pgfqpoint{5.059760in}{1.365000in}}%
\pgfpathlineto{\pgfqpoint{5.061000in}{1.155000in}}%
\pgfpathlineto{\pgfqpoint{5.062240in}{1.680000in}}%
\pgfpathlineto{\pgfqpoint{5.063480in}{1.225000in}}%
\pgfpathlineto{\pgfqpoint{5.065960in}{1.470000in}}%
\pgfpathlineto{\pgfqpoint{5.067200in}{1.400000in}}%
\pgfpathlineto{\pgfqpoint{5.068440in}{1.505000in}}%
\pgfpathlineto{\pgfqpoint{5.070920in}{1.050000in}}%
\pgfpathlineto{\pgfqpoint{5.072160in}{1.680000in}}%
\pgfpathlineto{\pgfqpoint{5.078360in}{1.050000in}}%
\pgfpathlineto{\pgfqpoint{5.080840in}{1.610000in}}%
\pgfpathlineto{\pgfqpoint{5.082080in}{1.610000in}}%
\pgfpathlineto{\pgfqpoint{5.083320in}{1.155000in}}%
\pgfpathlineto{\pgfqpoint{5.084560in}{1.785000in}}%
\pgfpathlineto{\pgfqpoint{5.088280in}{1.260000in}}%
\pgfpathlineto{\pgfqpoint{5.089520in}{1.680000in}}%
\pgfpathlineto{\pgfqpoint{5.092000in}{1.050000in}}%
\pgfpathlineto{\pgfqpoint{5.093240in}{1.715000in}}%
\pgfpathlineto{\pgfqpoint{5.094480in}{1.540000in}}%
\pgfpathlineto{\pgfqpoint{5.095720in}{0.945000in}}%
\pgfpathlineto{\pgfqpoint{5.096960in}{1.540000in}}%
\pgfpathlineto{\pgfqpoint{5.098200in}{1.575000in}}%
\pgfpathlineto{\pgfqpoint{5.099440in}{1.680000in}}%
\pgfpathlineto{\pgfqpoint{5.100680in}{1.435000in}}%
\pgfpathlineto{\pgfqpoint{5.101920in}{0.875000in}}%
\pgfpathlineto{\pgfqpoint{5.103160in}{0.875000in}}%
\pgfpathlineto{\pgfqpoint{5.104400in}{1.365000in}}%
\pgfpathlineto{\pgfqpoint{5.105640in}{1.260000in}}%
\pgfpathlineto{\pgfqpoint{5.106880in}{1.365000in}}%
\pgfpathlineto{\pgfqpoint{5.108120in}{1.295000in}}%
\pgfpathlineto{\pgfqpoint{5.110600in}{1.295000in}}%
\pgfpathlineto{\pgfqpoint{5.111840in}{1.750000in}}%
\pgfpathlineto{\pgfqpoint{5.113080in}{0.980000in}}%
\pgfpathlineto{\pgfqpoint{5.114320in}{1.540000in}}%
\pgfpathlineto{\pgfqpoint{5.116800in}{0.665000in}}%
\pgfpathlineto{\pgfqpoint{5.119280in}{1.470000in}}%
\pgfpathlineto{\pgfqpoint{5.120520in}{0.980000in}}%
\pgfpathlineto{\pgfqpoint{5.123000in}{1.645000in}}%
\pgfpathlineto{\pgfqpoint{5.124240in}{1.960000in}}%
\pgfpathlineto{\pgfqpoint{5.125480in}{1.435000in}}%
\pgfpathlineto{\pgfqpoint{5.126720in}{1.470000in}}%
\pgfpathlineto{\pgfqpoint{5.127960in}{1.715000in}}%
\pgfpathlineto{\pgfqpoint{5.130440in}{1.400000in}}%
\pgfpathlineto{\pgfqpoint{5.131680in}{1.645000in}}%
\pgfpathlineto{\pgfqpoint{5.132920in}{1.645000in}}%
\pgfpathlineto{\pgfqpoint{5.134160in}{1.610000in}}%
\pgfpathlineto{\pgfqpoint{5.135400in}{0.980000in}}%
\pgfpathlineto{\pgfqpoint{5.140360in}{1.855000in}}%
\pgfpathlineto{\pgfqpoint{5.141600in}{1.505000in}}%
\pgfpathlineto{\pgfqpoint{5.144080in}{2.065000in}}%
\pgfpathlineto{\pgfqpoint{5.145320in}{1.120000in}}%
\pgfpathlineto{\pgfqpoint{5.149040in}{1.785000in}}%
\pgfpathlineto{\pgfqpoint{5.150280in}{1.645000in}}%
\pgfpathlineto{\pgfqpoint{5.151520in}{1.260000in}}%
\pgfpathlineto{\pgfqpoint{5.152760in}{1.505000in}}%
\pgfpathlineto{\pgfqpoint{5.154000in}{1.260000in}}%
\pgfpathlineto{\pgfqpoint{5.156480in}{2.100000in}}%
\pgfpathlineto{\pgfqpoint{5.157720in}{1.365000in}}%
\pgfpathlineto{\pgfqpoint{5.158960in}{1.365000in}}%
\pgfpathlineto{\pgfqpoint{5.160200in}{1.715000in}}%
\pgfpathlineto{\pgfqpoint{5.162680in}{1.225000in}}%
\pgfpathlineto{\pgfqpoint{5.163920in}{1.715000in}}%
\pgfpathlineto{\pgfqpoint{5.165160in}{1.680000in}}%
\pgfpathlineto{\pgfqpoint{5.166400in}{1.435000in}}%
\pgfpathlineto{\pgfqpoint{5.167640in}{1.505000in}}%
\pgfpathlineto{\pgfqpoint{5.168880in}{1.750000in}}%
\pgfpathlineto{\pgfqpoint{5.171360in}{1.470000in}}%
\pgfpathlineto{\pgfqpoint{5.172600in}{1.365000in}}%
\pgfpathlineto{\pgfqpoint{5.173840in}{1.820000in}}%
\pgfpathlineto{\pgfqpoint{5.175080in}{1.155000in}}%
\pgfpathlineto{\pgfqpoint{5.176320in}{1.715000in}}%
\pgfpathlineto{\pgfqpoint{5.177560in}{1.645000in}}%
\pgfpathlineto{\pgfqpoint{5.178800in}{1.505000in}}%
\pgfpathlineto{\pgfqpoint{5.180040in}{1.645000in}}%
\pgfpathlineto{\pgfqpoint{5.181280in}{1.540000in}}%
\pgfpathlineto{\pgfqpoint{5.182520in}{1.575000in}}%
\pgfpathlineto{\pgfqpoint{5.183760in}{1.190000in}}%
\pgfpathlineto{\pgfqpoint{5.185000in}{1.680000in}}%
\pgfpathlineto{\pgfqpoint{5.186240in}{1.470000in}}%
\pgfpathlineto{\pgfqpoint{5.187480in}{1.050000in}}%
\pgfpathlineto{\pgfqpoint{5.188720in}{1.540000in}}%
\pgfpathlineto{\pgfqpoint{5.189960in}{1.540000in}}%
\pgfpathlineto{\pgfqpoint{5.191200in}{1.435000in}}%
\pgfpathlineto{\pgfqpoint{5.192440in}{1.225000in}}%
\pgfpathlineto{\pgfqpoint{5.193680in}{1.365000in}}%
\pgfpathlineto{\pgfqpoint{5.194920in}{1.820000in}}%
\pgfpathlineto{\pgfqpoint{5.196160in}{1.400000in}}%
\pgfpathlineto{\pgfqpoint{5.197400in}{1.610000in}}%
\pgfpathlineto{\pgfqpoint{5.198640in}{1.470000in}}%
\pgfpathlineto{\pgfqpoint{5.201120in}{1.715000in}}%
\pgfpathlineto{\pgfqpoint{5.203600in}{1.365000in}}%
\pgfpathlineto{\pgfqpoint{5.206080in}{1.715000in}}%
\pgfpathlineto{\pgfqpoint{5.207320in}{1.435000in}}%
\pgfpathlineto{\pgfqpoint{5.211040in}{1.890000in}}%
\pgfpathlineto{\pgfqpoint{5.212280in}{1.435000in}}%
\pgfpathlineto{\pgfqpoint{5.213520in}{1.505000in}}%
\pgfpathlineto{\pgfqpoint{5.216000in}{1.785000in}}%
\pgfpathlineto{\pgfqpoint{5.218480in}{1.120000in}}%
\pgfpathlineto{\pgfqpoint{5.219720in}{1.610000in}}%
\pgfpathlineto{\pgfqpoint{5.220960in}{1.015000in}}%
\pgfpathlineto{\pgfqpoint{5.222200in}{1.365000in}}%
\pgfpathlineto{\pgfqpoint{5.223440in}{1.085000in}}%
\pgfpathlineto{\pgfqpoint{5.224680in}{1.925000in}}%
\pgfpathlineto{\pgfqpoint{5.225920in}{1.785000in}}%
\pgfpathlineto{\pgfqpoint{5.227160in}{1.225000in}}%
\pgfpathlineto{\pgfqpoint{5.229640in}{1.715000in}}%
\pgfpathlineto{\pgfqpoint{5.230880in}{1.505000in}}%
\pgfpathlineto{\pgfqpoint{5.232120in}{1.890000in}}%
\pgfpathlineto{\pgfqpoint{5.233360in}{1.505000in}}%
\pgfpathlineto{\pgfqpoint{5.234600in}{1.785000in}}%
\pgfpathlineto{\pgfqpoint{5.235840in}{1.470000in}}%
\pgfpathlineto{\pgfqpoint{5.238320in}{1.680000in}}%
\pgfpathlineto{\pgfqpoint{5.239560in}{1.190000in}}%
\pgfpathlineto{\pgfqpoint{5.240800in}{1.575000in}}%
\pgfpathlineto{\pgfqpoint{5.242040in}{1.540000in}}%
\pgfpathlineto{\pgfqpoint{5.243280in}{1.260000in}}%
\pgfpathlineto{\pgfqpoint{5.245760in}{1.365000in}}%
\pgfpathlineto{\pgfqpoint{5.247000in}{1.085000in}}%
\pgfpathlineto{\pgfqpoint{5.248240in}{1.435000in}}%
\pgfpathlineto{\pgfqpoint{5.249480in}{1.295000in}}%
\pgfpathlineto{\pgfqpoint{5.250720in}{1.540000in}}%
\pgfpathlineto{\pgfqpoint{5.251960in}{1.050000in}}%
\pgfpathlineto{\pgfqpoint{5.253200in}{1.295000in}}%
\pgfpathlineto{\pgfqpoint{5.254440in}{1.785000in}}%
\pgfpathlineto{\pgfqpoint{5.255680in}{1.155000in}}%
\pgfpathlineto{\pgfqpoint{5.256920in}{1.155000in}}%
\pgfpathlineto{\pgfqpoint{5.260640in}{1.855000in}}%
\pgfpathlineto{\pgfqpoint{5.263120in}{1.155000in}}%
\pgfpathlineto{\pgfqpoint{5.264360in}{1.540000in}}%
\pgfpathlineto{\pgfqpoint{5.265600in}{1.225000in}}%
\pgfpathlineto{\pgfqpoint{5.268080in}{1.715000in}}%
\pgfpathlineto{\pgfqpoint{5.270560in}{1.435000in}}%
\pgfpathlineto{\pgfqpoint{5.271800in}{1.400000in}}%
\pgfpathlineto{\pgfqpoint{5.273040in}{1.085000in}}%
\pgfpathlineto{\pgfqpoint{5.275520in}{1.400000in}}%
\pgfpathlineto{\pgfqpoint{5.276760in}{1.330000in}}%
\pgfpathlineto{\pgfqpoint{5.279240in}{1.785000in}}%
\pgfpathlineto{\pgfqpoint{5.281720in}{1.435000in}}%
\pgfpathlineto{\pgfqpoint{5.282960in}{1.505000in}}%
\pgfpathlineto{\pgfqpoint{5.284200in}{1.645000in}}%
\pgfpathlineto{\pgfqpoint{5.286680in}{1.295000in}}%
\pgfpathlineto{\pgfqpoint{5.287920in}{1.610000in}}%
\pgfpathlineto{\pgfqpoint{5.289160in}{1.400000in}}%
\pgfpathlineto{\pgfqpoint{5.290400in}{1.400000in}}%
\pgfpathlineto{\pgfqpoint{5.291640in}{1.155000in}}%
\pgfpathlineto{\pgfqpoint{5.292880in}{1.365000in}}%
\pgfpathlineto{\pgfqpoint{5.294120in}{1.295000in}}%
\pgfpathlineto{\pgfqpoint{5.295360in}{1.470000in}}%
\pgfpathlineto{\pgfqpoint{5.296600in}{0.980000in}}%
\pgfpathlineto{\pgfqpoint{5.297840in}{1.820000in}}%
\pgfpathlineto{\pgfqpoint{5.299080in}{1.330000in}}%
\pgfpathlineto{\pgfqpoint{5.300320in}{1.505000in}}%
\pgfpathlineto{\pgfqpoint{5.302800in}{1.190000in}}%
\pgfpathlineto{\pgfqpoint{5.305280in}{1.575000in}}%
\pgfpathlineto{\pgfqpoint{5.306520in}{1.470000in}}%
\pgfpathlineto{\pgfqpoint{5.307760in}{1.680000in}}%
\pgfpathlineto{\pgfqpoint{5.310240in}{1.155000in}}%
\pgfpathlineto{\pgfqpoint{5.311480in}{1.610000in}}%
\pgfpathlineto{\pgfqpoint{5.313960in}{1.015000in}}%
\pgfpathlineto{\pgfqpoint{5.315200in}{1.050000in}}%
\pgfpathlineto{\pgfqpoint{5.316440in}{1.820000in}}%
\pgfpathlineto{\pgfqpoint{5.317680in}{1.435000in}}%
\pgfpathlineto{\pgfqpoint{5.318920in}{1.470000in}}%
\pgfpathlineto{\pgfqpoint{5.320160in}{1.435000in}}%
\pgfpathlineto{\pgfqpoint{5.322640in}{1.295000in}}%
\pgfpathlineto{\pgfqpoint{5.323880in}{1.470000in}}%
\pgfpathlineto{\pgfqpoint{5.325120in}{1.260000in}}%
\pgfpathlineto{\pgfqpoint{5.326360in}{1.750000in}}%
\pgfpathlineto{\pgfqpoint{5.328840in}{0.875000in}}%
\pgfpathlineto{\pgfqpoint{5.330080in}{0.980000in}}%
\pgfpathlineto{\pgfqpoint{5.331320in}{1.680000in}}%
\pgfpathlineto{\pgfqpoint{5.332560in}{1.715000in}}%
\pgfpathlineto{\pgfqpoint{5.333800in}{1.645000in}}%
\pgfpathlineto{\pgfqpoint{5.336280in}{1.260000in}}%
\pgfpathlineto{\pgfqpoint{5.337520in}{1.190000in}}%
\pgfpathlineto{\pgfqpoint{5.338760in}{1.225000in}}%
\pgfpathlineto{\pgfqpoint{5.340000in}{1.085000in}}%
\pgfpathlineto{\pgfqpoint{5.341240in}{1.680000in}}%
\pgfpathlineto{\pgfqpoint{5.342480in}{1.295000in}}%
\pgfpathlineto{\pgfqpoint{5.344960in}{1.540000in}}%
\pgfpathlineto{\pgfqpoint{5.346200in}{1.505000in}}%
\pgfpathlineto{\pgfqpoint{5.347440in}{2.030000in}}%
\pgfpathlineto{\pgfqpoint{5.348680in}{1.575000in}}%
\pgfpathlineto{\pgfqpoint{5.349920in}{1.820000in}}%
\pgfpathlineto{\pgfqpoint{5.351160in}{1.750000in}}%
\pgfpathlineto{\pgfqpoint{5.352400in}{1.365000in}}%
\pgfpathlineto{\pgfqpoint{5.353640in}{1.540000in}}%
\pgfpathlineto{\pgfqpoint{5.354880in}{1.260000in}}%
\pgfpathlineto{\pgfqpoint{5.356120in}{1.540000in}}%
\pgfpathlineto{\pgfqpoint{5.357360in}{1.015000in}}%
\pgfpathlineto{\pgfqpoint{5.358600in}{1.190000in}}%
\pgfpathlineto{\pgfqpoint{5.359840in}{1.680000in}}%
\pgfpathlineto{\pgfqpoint{5.362320in}{1.050000in}}%
\pgfpathlineto{\pgfqpoint{5.363560in}{1.085000in}}%
\pgfpathlineto{\pgfqpoint{5.364800in}{1.015000in}}%
\pgfpathlineto{\pgfqpoint{5.366040in}{1.540000in}}%
\pgfpathlineto{\pgfqpoint{5.367280in}{1.400000in}}%
\pgfpathlineto{\pgfqpoint{5.368520in}{1.085000in}}%
\pgfpathlineto{\pgfqpoint{5.372240in}{1.540000in}}%
\pgfpathlineto{\pgfqpoint{5.373480in}{1.400000in}}%
\pgfpathlineto{\pgfqpoint{5.374720in}{1.470000in}}%
\pgfpathlineto{\pgfqpoint{5.375960in}{1.225000in}}%
\pgfpathlineto{\pgfqpoint{5.378440in}{1.995000in}}%
\pgfpathlineto{\pgfqpoint{5.379680in}{1.400000in}}%
\pgfpathlineto{\pgfqpoint{5.380920in}{1.540000in}}%
\pgfpathlineto{\pgfqpoint{5.382160in}{1.295000in}}%
\pgfpathlineto{\pgfqpoint{5.383400in}{0.735000in}}%
\pgfpathlineto{\pgfqpoint{5.385880in}{1.435000in}}%
\pgfpathlineto{\pgfqpoint{5.387120in}{1.015000in}}%
\pgfpathlineto{\pgfqpoint{5.389600in}{1.575000in}}%
\pgfpathlineto{\pgfqpoint{5.390840in}{0.770000in}}%
\pgfpathlineto{\pgfqpoint{5.393320in}{1.785000in}}%
\pgfpathlineto{\pgfqpoint{5.394560in}{1.505000in}}%
\pgfpathlineto{\pgfqpoint{5.395800in}{0.945000in}}%
\pgfpathlineto{\pgfqpoint{5.398280in}{1.960000in}}%
\pgfpathlineto{\pgfqpoint{5.399520in}{1.400000in}}%
\pgfpathlineto{\pgfqpoint{5.400760in}{1.505000in}}%
\pgfpathlineto{\pgfqpoint{5.402000in}{1.365000in}}%
\pgfpathlineto{\pgfqpoint{5.404480in}{1.575000in}}%
\pgfpathlineto{\pgfqpoint{5.405720in}{1.610000in}}%
\pgfpathlineto{\pgfqpoint{5.408200in}{1.295000in}}%
\pgfpathlineto{\pgfqpoint{5.409440in}{1.225000in}}%
\pgfpathlineto{\pgfqpoint{5.410680in}{1.785000in}}%
\pgfpathlineto{\pgfqpoint{5.413160in}{1.680000in}}%
\pgfpathlineto{\pgfqpoint{5.414400in}{2.100000in}}%
\pgfpathlineto{\pgfqpoint{5.415640in}{1.575000in}}%
\pgfpathlineto{\pgfqpoint{5.416880in}{1.890000in}}%
\pgfpathlineto{\pgfqpoint{5.418120in}{1.295000in}}%
\pgfpathlineto{\pgfqpoint{5.419360in}{1.260000in}}%
\pgfpathlineto{\pgfqpoint{5.420600in}{1.470000in}}%
\pgfpathlineto{\pgfqpoint{5.421840in}{1.260000in}}%
\pgfpathlineto{\pgfqpoint{5.424320in}{1.890000in}}%
\pgfpathlineto{\pgfqpoint{5.425560in}{1.260000in}}%
\pgfpathlineto{\pgfqpoint{5.426800in}{1.750000in}}%
\pgfpathlineto{\pgfqpoint{5.429280in}{1.820000in}}%
\pgfpathlineto{\pgfqpoint{5.431760in}{1.645000in}}%
\pgfpathlineto{\pgfqpoint{5.433000in}{1.960000in}}%
\pgfpathlineto{\pgfqpoint{5.435480in}{1.050000in}}%
\pgfpathlineto{\pgfqpoint{5.436720in}{1.155000in}}%
\pgfpathlineto{\pgfqpoint{5.437960in}{1.750000in}}%
\pgfpathlineto{\pgfqpoint{5.440440in}{1.050000in}}%
\pgfpathlineto{\pgfqpoint{5.444160in}{1.575000in}}%
\pgfpathlineto{\pgfqpoint{5.447880in}{1.120000in}}%
\pgfpathlineto{\pgfqpoint{5.451600in}{1.540000in}}%
\pgfpathlineto{\pgfqpoint{5.454080in}{1.225000in}}%
\pgfpathlineto{\pgfqpoint{5.455320in}{1.505000in}}%
\pgfpathlineto{\pgfqpoint{5.456560in}{1.085000in}}%
\pgfpathlineto{\pgfqpoint{5.457800in}{1.610000in}}%
\pgfpathlineto{\pgfqpoint{5.459040in}{1.400000in}}%
\pgfpathlineto{\pgfqpoint{5.460280in}{1.540000in}}%
\pgfpathlineto{\pgfqpoint{5.461520in}{1.190000in}}%
\pgfpathlineto{\pgfqpoint{5.462760in}{1.645000in}}%
\pgfpathlineto{\pgfqpoint{5.464000in}{1.645000in}}%
\pgfpathlineto{\pgfqpoint{5.467720in}{1.120000in}}%
\pgfpathlineto{\pgfqpoint{5.468960in}{1.575000in}}%
\pgfpathlineto{\pgfqpoint{5.471440in}{1.330000in}}%
\pgfpathlineto{\pgfqpoint{5.472680in}{1.470000in}}%
\pgfpathlineto{\pgfqpoint{5.473920in}{1.785000in}}%
\pgfpathlineto{\pgfqpoint{5.477640in}{1.435000in}}%
\pgfpathlineto{\pgfqpoint{5.478880in}{1.925000in}}%
\pgfpathlineto{\pgfqpoint{5.481360in}{1.085000in}}%
\pgfpathlineto{\pgfqpoint{5.483840in}{1.295000in}}%
\pgfpathlineto{\pgfqpoint{5.485080in}{1.190000in}}%
\pgfpathlineto{\pgfqpoint{5.486320in}{1.400000in}}%
\pgfpathlineto{\pgfqpoint{5.487560in}{1.400000in}}%
\pgfpathlineto{\pgfqpoint{5.488800in}{1.155000in}}%
\pgfpathlineto{\pgfqpoint{5.492520in}{1.610000in}}%
\pgfpathlineto{\pgfqpoint{5.493760in}{1.715000in}}%
\pgfpathlineto{\pgfqpoint{5.495000in}{2.065000in}}%
\pgfpathlineto{\pgfqpoint{5.496240in}{1.330000in}}%
\pgfpathlineto{\pgfqpoint{5.497480in}{1.435000in}}%
\pgfpathlineto{\pgfqpoint{5.498720in}{1.680000in}}%
\pgfpathlineto{\pgfqpoint{5.499960in}{1.225000in}}%
\pgfpathlineto{\pgfqpoint{5.501200in}{1.610000in}}%
\pgfpathlineto{\pgfqpoint{5.502440in}{1.505000in}}%
\pgfpathlineto{\pgfqpoint{5.503680in}{0.770000in}}%
\pgfpathlineto{\pgfqpoint{5.504920in}{1.190000in}}%
\pgfpathlineto{\pgfqpoint{5.506160in}{1.155000in}}%
\pgfpathlineto{\pgfqpoint{5.508640in}{1.890000in}}%
\pgfpathlineto{\pgfqpoint{5.511120in}{1.435000in}}%
\pgfpathlineto{\pgfqpoint{5.513600in}{1.505000in}}%
\pgfpathlineto{\pgfqpoint{5.514840in}{1.400000in}}%
\pgfpathlineto{\pgfqpoint{5.516080in}{1.470000in}}%
\pgfpathlineto{\pgfqpoint{5.517320in}{1.470000in}}%
\pgfpathlineto{\pgfqpoint{5.518560in}{1.295000in}}%
\pgfpathlineto{\pgfqpoint{5.519800in}{1.435000in}}%
\pgfpathlineto{\pgfqpoint{5.521040in}{1.050000in}}%
\pgfpathlineto{\pgfqpoint{5.522280in}{1.715000in}}%
\pgfpathlineto{\pgfqpoint{5.523520in}{1.295000in}}%
\pgfpathlineto{\pgfqpoint{5.526000in}{1.610000in}}%
\pgfpathlineto{\pgfqpoint{5.527240in}{1.225000in}}%
\pgfpathlineto{\pgfqpoint{5.528480in}{1.260000in}}%
\pgfpathlineto{\pgfqpoint{5.530960in}{1.505000in}}%
\pgfpathlineto{\pgfqpoint{5.532200in}{1.435000in}}%
\pgfpathlineto{\pgfqpoint{5.534680in}{1.085000in}}%
\pgfpathlineto{\pgfqpoint{5.535920in}{1.715000in}}%
\pgfpathlineto{\pgfqpoint{5.537160in}{1.155000in}}%
\pgfpathlineto{\pgfqpoint{5.538400in}{1.295000in}}%
\pgfpathlineto{\pgfqpoint{5.539640in}{1.225000in}}%
\pgfpathlineto{\pgfqpoint{5.540880in}{1.365000in}}%
\pgfpathlineto{\pgfqpoint{5.542120in}{1.225000in}}%
\pgfpathlineto{\pgfqpoint{5.543360in}{1.680000in}}%
\pgfpathlineto{\pgfqpoint{5.544600in}{1.330000in}}%
\pgfpathlineto{\pgfqpoint{5.545840in}{1.295000in}}%
\pgfpathlineto{\pgfqpoint{5.547080in}{1.435000in}}%
\pgfpathlineto{\pgfqpoint{5.548320in}{1.435000in}}%
\pgfpathlineto{\pgfqpoint{5.549560in}{1.470000in}}%
\pgfpathlineto{\pgfqpoint{5.550800in}{1.470000in}}%
\pgfpathlineto{\pgfqpoint{5.552040in}{1.120000in}}%
\pgfpathlineto{\pgfqpoint{5.554520in}{1.715000in}}%
\pgfpathlineto{\pgfqpoint{5.555760in}{1.400000in}}%
\pgfpathlineto{\pgfqpoint{5.557000in}{1.540000in}}%
\pgfpathlineto{\pgfqpoint{5.558240in}{1.540000in}}%
\pgfpathlineto{\pgfqpoint{5.559480in}{1.505000in}}%
\pgfpathlineto{\pgfqpoint{5.560720in}{1.365000in}}%
\pgfpathlineto{\pgfqpoint{5.561960in}{1.855000in}}%
\pgfpathlineto{\pgfqpoint{5.563200in}{1.505000in}}%
\pgfpathlineto{\pgfqpoint{5.564440in}{1.645000in}}%
\pgfpathlineto{\pgfqpoint{5.565680in}{1.015000in}}%
\pgfpathlineto{\pgfqpoint{5.568160in}{1.435000in}}%
\pgfpathlineto{\pgfqpoint{5.569400in}{1.260000in}}%
\pgfpathlineto{\pgfqpoint{5.570640in}{1.540000in}}%
\pgfpathlineto{\pgfqpoint{5.571880in}{1.225000in}}%
\pgfpathlineto{\pgfqpoint{5.573120in}{1.680000in}}%
\pgfpathlineto{\pgfqpoint{5.574360in}{0.980000in}}%
\pgfpathlineto{\pgfqpoint{5.575600in}{1.610000in}}%
\pgfpathlineto{\pgfqpoint{5.576840in}{1.085000in}}%
\pgfpathlineto{\pgfqpoint{5.579320in}{1.575000in}}%
\pgfpathlineto{\pgfqpoint{5.580560in}{1.400000in}}%
\pgfpathlineto{\pgfqpoint{5.581800in}{1.785000in}}%
\pgfpathlineto{\pgfqpoint{5.583040in}{1.295000in}}%
\pgfpathlineto{\pgfqpoint{5.584280in}{1.820000in}}%
\pgfpathlineto{\pgfqpoint{5.585520in}{1.400000in}}%
\pgfpathlineto{\pgfqpoint{5.586760in}{1.505000in}}%
\pgfpathlineto{\pgfqpoint{5.588000in}{1.785000in}}%
\pgfpathlineto{\pgfqpoint{5.589240in}{1.365000in}}%
\pgfpathlineto{\pgfqpoint{5.590480in}{2.065000in}}%
\pgfpathlineto{\pgfqpoint{5.591720in}{2.030000in}}%
\pgfpathlineto{\pgfqpoint{5.592960in}{1.365000in}}%
\pgfpathlineto{\pgfqpoint{5.594200in}{1.435000in}}%
\pgfpathlineto{\pgfqpoint{5.595440in}{1.785000in}}%
\pgfpathlineto{\pgfqpoint{5.596680in}{1.295000in}}%
\pgfpathlineto{\pgfqpoint{5.597920in}{1.680000in}}%
\pgfpathlineto{\pgfqpoint{5.599160in}{1.540000in}}%
\pgfpathlineto{\pgfqpoint{5.600400in}{1.785000in}}%
\pgfpathlineto{\pgfqpoint{5.601640in}{1.330000in}}%
\pgfpathlineto{\pgfqpoint{5.604120in}{1.960000in}}%
\pgfpathlineto{\pgfqpoint{5.606600in}{1.190000in}}%
\pgfpathlineto{\pgfqpoint{5.607840in}{1.190000in}}%
\pgfpathlineto{\pgfqpoint{5.610320in}{1.680000in}}%
\pgfpathlineto{\pgfqpoint{5.611560in}{1.680000in}}%
\pgfpathlineto{\pgfqpoint{5.612800in}{0.910000in}}%
\pgfpathlineto{\pgfqpoint{5.614040in}{1.680000in}}%
\pgfpathlineto{\pgfqpoint{5.615280in}{1.610000in}}%
\pgfpathlineto{\pgfqpoint{5.617760in}{1.155000in}}%
\pgfpathlineto{\pgfqpoint{5.619000in}{1.260000in}}%
\pgfpathlineto{\pgfqpoint{5.621480in}{1.715000in}}%
\pgfpathlineto{\pgfqpoint{5.623960in}{1.295000in}}%
\pgfpathlineto{\pgfqpoint{5.625200in}{1.820000in}}%
\pgfpathlineto{\pgfqpoint{5.627680in}{1.295000in}}%
\pgfpathlineto{\pgfqpoint{5.630160in}{2.100000in}}%
\pgfpathlineto{\pgfqpoint{5.632640in}{1.330000in}}%
\pgfpathlineto{\pgfqpoint{5.633880in}{1.505000in}}%
\pgfpathlineto{\pgfqpoint{5.635120in}{1.505000in}}%
\pgfpathlineto{\pgfqpoint{5.636360in}{1.085000in}}%
\pgfpathlineto{\pgfqpoint{5.638840in}{1.610000in}}%
\pgfpathlineto{\pgfqpoint{5.641320in}{1.225000in}}%
\pgfpathlineto{\pgfqpoint{5.643800in}{1.575000in}}%
\pgfpathlineto{\pgfqpoint{5.645040in}{1.505000in}}%
\pgfpathlineto{\pgfqpoint{5.646280in}{1.505000in}}%
\pgfpathlineto{\pgfqpoint{5.647520in}{1.435000in}}%
\pgfpathlineto{\pgfqpoint{5.648760in}{1.785000in}}%
\pgfpathlineto{\pgfqpoint{5.650000in}{1.540000in}}%
\pgfpathlineto{\pgfqpoint{5.652480in}{1.890000in}}%
\pgfpathlineto{\pgfqpoint{5.653720in}{1.785000in}}%
\pgfpathlineto{\pgfqpoint{5.656200in}{1.330000in}}%
\pgfpathlineto{\pgfqpoint{5.657440in}{0.735000in}}%
\pgfpathlineto{\pgfqpoint{5.659920in}{1.960000in}}%
\pgfpathlineto{\pgfqpoint{5.662400in}{1.575000in}}%
\pgfpathlineto{\pgfqpoint{5.663640in}{1.855000in}}%
\pgfpathlineto{\pgfqpoint{5.664880in}{1.365000in}}%
\pgfpathlineto{\pgfqpoint{5.666120in}{1.400000in}}%
\pgfpathlineto{\pgfqpoint{5.667360in}{1.365000in}}%
\pgfpathlineto{\pgfqpoint{5.669840in}{1.540000in}}%
\pgfpathlineto{\pgfqpoint{5.671080in}{1.295000in}}%
\pgfpathlineto{\pgfqpoint{5.673560in}{1.575000in}}%
\pgfpathlineto{\pgfqpoint{5.674800in}{1.505000in}}%
\pgfpathlineto{\pgfqpoint{5.676040in}{1.680000in}}%
\pgfpathlineto{\pgfqpoint{5.677280in}{1.645000in}}%
\pgfpathlineto{\pgfqpoint{5.678520in}{1.680000in}}%
\pgfpathlineto{\pgfqpoint{5.681000in}{1.225000in}}%
\pgfpathlineto{\pgfqpoint{5.683480in}{1.960000in}}%
\pgfpathlineto{\pgfqpoint{5.684720in}{0.910000in}}%
\pgfpathlineto{\pgfqpoint{5.685960in}{1.400000in}}%
\pgfpathlineto{\pgfqpoint{5.687200in}{1.295000in}}%
\pgfpathlineto{\pgfqpoint{5.688440in}{1.575000in}}%
\pgfpathlineto{\pgfqpoint{5.689680in}{1.330000in}}%
\pgfpathlineto{\pgfqpoint{5.690920in}{1.540000in}}%
\pgfpathlineto{\pgfqpoint{5.692160in}{1.330000in}}%
\pgfpathlineto{\pgfqpoint{5.693400in}{1.785000in}}%
\pgfpathlineto{\pgfqpoint{5.694640in}{1.435000in}}%
\pgfpathlineto{\pgfqpoint{5.695880in}{1.470000in}}%
\pgfpathlineto{\pgfqpoint{5.697120in}{1.330000in}}%
\pgfpathlineto{\pgfqpoint{5.698360in}{1.435000in}}%
\pgfpathlineto{\pgfqpoint{5.699600in}{1.435000in}}%
\pgfpathlineto{\pgfqpoint{5.700840in}{1.925000in}}%
\pgfpathlineto{\pgfqpoint{5.702080in}{1.225000in}}%
\pgfpathlineto{\pgfqpoint{5.704560in}{1.715000in}}%
\pgfpathlineto{\pgfqpoint{5.705800in}{1.260000in}}%
\pgfpathlineto{\pgfqpoint{5.708280in}{1.505000in}}%
\pgfpathlineto{\pgfqpoint{5.709520in}{1.400000in}}%
\pgfpathlineto{\pgfqpoint{5.710760in}{1.820000in}}%
\pgfpathlineto{\pgfqpoint{5.712000in}{1.470000in}}%
\pgfpathlineto{\pgfqpoint{5.713240in}{1.645000in}}%
\pgfpathlineto{\pgfqpoint{5.714480in}{1.575000in}}%
\pgfpathlineto{\pgfqpoint{5.715720in}{1.890000in}}%
\pgfpathlineto{\pgfqpoint{5.716960in}{1.610000in}}%
\pgfpathlineto{\pgfqpoint{5.718200in}{1.610000in}}%
\pgfpathlineto{\pgfqpoint{5.719440in}{1.645000in}}%
\pgfpathlineto{\pgfqpoint{5.721920in}{1.435000in}}%
\pgfpathlineto{\pgfqpoint{5.723160in}{1.645000in}}%
\pgfpathlineto{\pgfqpoint{5.725640in}{1.330000in}}%
\pgfpathlineto{\pgfqpoint{5.726880in}{1.750000in}}%
\pgfpathlineto{\pgfqpoint{5.728120in}{1.435000in}}%
\pgfpathlineto{\pgfqpoint{5.729360in}{1.575000in}}%
\pgfpathlineto{\pgfqpoint{5.730600in}{1.330000in}}%
\pgfpathlineto{\pgfqpoint{5.731840in}{1.365000in}}%
\pgfpathlineto{\pgfqpoint{5.733080in}{1.330000in}}%
\pgfpathlineto{\pgfqpoint{5.734320in}{1.365000in}}%
\pgfpathlineto{\pgfqpoint{5.736800in}{1.295000in}}%
\pgfpathlineto{\pgfqpoint{5.738040in}{1.015000in}}%
\pgfpathlineto{\pgfqpoint{5.739280in}{1.400000in}}%
\pgfpathlineto{\pgfqpoint{5.741760in}{1.120000in}}%
\pgfpathlineto{\pgfqpoint{5.743000in}{1.610000in}}%
\pgfpathlineto{\pgfqpoint{5.745480in}{0.980000in}}%
\pgfpathlineto{\pgfqpoint{5.746720in}{1.645000in}}%
\pgfpathlineto{\pgfqpoint{5.747960in}{1.400000in}}%
\pgfpathlineto{\pgfqpoint{5.749200in}{1.645000in}}%
\pgfpathlineto{\pgfqpoint{5.750440in}{1.575000in}}%
\pgfpathlineto{\pgfqpoint{5.751680in}{1.890000in}}%
\pgfpathlineto{\pgfqpoint{5.752920in}{0.945000in}}%
\pgfpathlineto{\pgfqpoint{5.754160in}{1.645000in}}%
\pgfpathlineto{\pgfqpoint{5.755400in}{1.645000in}}%
\pgfpathlineto{\pgfqpoint{5.757880in}{0.875000in}}%
\pgfpathlineto{\pgfqpoint{5.760360in}{1.750000in}}%
\pgfpathlineto{\pgfqpoint{5.761600in}{1.155000in}}%
\pgfpathlineto{\pgfqpoint{5.762840in}{1.260000in}}%
\pgfpathlineto{\pgfqpoint{5.764080in}{0.980000in}}%
\pgfpathlineto{\pgfqpoint{5.765320in}{1.435000in}}%
\pgfpathlineto{\pgfqpoint{5.766560in}{1.435000in}}%
\pgfpathlineto{\pgfqpoint{5.769040in}{1.680000in}}%
\pgfpathlineto{\pgfqpoint{5.770280in}{0.980000in}}%
\pgfpathlineto{\pgfqpoint{5.772760in}{1.505000in}}%
\pgfpathlineto{\pgfqpoint{5.774000in}{0.980000in}}%
\pgfpathlineto{\pgfqpoint{5.776480in}{1.400000in}}%
\pgfpathlineto{\pgfqpoint{5.777720in}{1.435000in}}%
\pgfpathlineto{\pgfqpoint{5.780200in}{1.820000in}}%
\pgfpathlineto{\pgfqpoint{5.783920in}{1.120000in}}%
\pgfpathlineto{\pgfqpoint{5.785160in}{1.295000in}}%
\pgfpathlineto{\pgfqpoint{5.786400in}{1.190000in}}%
\pgfpathlineto{\pgfqpoint{5.787640in}{1.505000in}}%
\pgfpathlineto{\pgfqpoint{5.788880in}{1.365000in}}%
\pgfpathlineto{\pgfqpoint{5.790120in}{1.750000in}}%
\pgfpathlineto{\pgfqpoint{5.791360in}{1.365000in}}%
\pgfpathlineto{\pgfqpoint{5.792600in}{1.680000in}}%
\pgfpathlineto{\pgfqpoint{5.793840in}{1.295000in}}%
\pgfpathlineto{\pgfqpoint{5.795080in}{1.365000in}}%
\pgfpathlineto{\pgfqpoint{5.797560in}{1.680000in}}%
\pgfpathlineto{\pgfqpoint{5.800040in}{1.155000in}}%
\pgfpathlineto{\pgfqpoint{5.801280in}{1.610000in}}%
\pgfpathlineto{\pgfqpoint{5.802520in}{1.120000in}}%
\pgfpathlineto{\pgfqpoint{5.805000in}{1.680000in}}%
\pgfpathlineto{\pgfqpoint{5.806240in}{1.785000in}}%
\pgfpathlineto{\pgfqpoint{5.809960in}{1.225000in}}%
\pgfpathlineto{\pgfqpoint{5.811200in}{1.645000in}}%
\pgfpathlineto{\pgfqpoint{5.812440in}{1.435000in}}%
\pgfpathlineto{\pgfqpoint{5.813680in}{1.505000in}}%
\pgfpathlineto{\pgfqpoint{5.816160in}{1.820000in}}%
\pgfpathlineto{\pgfqpoint{5.817400in}{1.120000in}}%
\pgfpathlineto{\pgfqpoint{5.818640in}{1.575000in}}%
\pgfpathlineto{\pgfqpoint{5.819880in}{1.330000in}}%
\pgfpathlineto{\pgfqpoint{5.823600in}{2.030000in}}%
\pgfpathlineto{\pgfqpoint{5.824840in}{1.715000in}}%
\pgfpathlineto{\pgfqpoint{5.826080in}{0.875000in}}%
\pgfpathlineto{\pgfqpoint{5.827320in}{1.680000in}}%
\pgfpathlineto{\pgfqpoint{5.828560in}{1.750000in}}%
\pgfpathlineto{\pgfqpoint{5.829800in}{1.330000in}}%
\pgfpathlineto{\pgfqpoint{5.831040in}{1.330000in}}%
\pgfpathlineto{\pgfqpoint{5.832280in}{1.260000in}}%
\pgfpathlineto{\pgfqpoint{5.833520in}{1.610000in}}%
\pgfpathlineto{\pgfqpoint{5.834760in}{1.190000in}}%
\pgfpathlineto{\pgfqpoint{5.836000in}{1.225000in}}%
\pgfpathlineto{\pgfqpoint{5.839720in}{1.435000in}}%
\pgfpathlineto{\pgfqpoint{5.840960in}{1.225000in}}%
\pgfpathlineto{\pgfqpoint{5.842200in}{1.715000in}}%
\pgfpathlineto{\pgfqpoint{5.843440in}{1.120000in}}%
\pgfpathlineto{\pgfqpoint{5.845920in}{1.925000in}}%
\pgfpathlineto{\pgfqpoint{5.850880in}{1.470000in}}%
\pgfpathlineto{\pgfqpoint{5.852120in}{1.610000in}}%
\pgfpathlineto{\pgfqpoint{5.853360in}{1.505000in}}%
\pgfpathlineto{\pgfqpoint{5.854600in}{1.190000in}}%
\pgfpathlineto{\pgfqpoint{5.855840in}{1.295000in}}%
\pgfpathlineto{\pgfqpoint{5.857080in}{1.155000in}}%
\pgfpathlineto{\pgfqpoint{5.858320in}{1.190000in}}%
\pgfpathlineto{\pgfqpoint{5.859560in}{1.330000in}}%
\pgfpathlineto{\pgfqpoint{5.860800in}{1.225000in}}%
\pgfpathlineto{\pgfqpoint{5.862040in}{1.365000in}}%
\pgfpathlineto{\pgfqpoint{5.863280in}{1.855000in}}%
\pgfpathlineto{\pgfqpoint{5.864520in}{1.435000in}}%
\pgfpathlineto{\pgfqpoint{5.865760in}{1.750000in}}%
\pgfpathlineto{\pgfqpoint{5.867000in}{1.470000in}}%
\pgfpathlineto{\pgfqpoint{5.868240in}{1.470000in}}%
\pgfpathlineto{\pgfqpoint{5.869480in}{1.155000in}}%
\pgfpathlineto{\pgfqpoint{5.870720in}{1.260000in}}%
\pgfpathlineto{\pgfqpoint{5.871960in}{1.260000in}}%
\pgfpathlineto{\pgfqpoint{5.874440in}{1.435000in}}%
\pgfpathlineto{\pgfqpoint{5.875680in}{1.820000in}}%
\pgfpathlineto{\pgfqpoint{5.876920in}{1.365000in}}%
\pgfpathlineto{\pgfqpoint{5.878160in}{1.400000in}}%
\pgfpathlineto{\pgfqpoint{5.879400in}{1.330000in}}%
\pgfpathlineto{\pgfqpoint{5.880640in}{1.050000in}}%
\pgfpathlineto{\pgfqpoint{5.883120in}{1.610000in}}%
\pgfpathlineto{\pgfqpoint{5.884360in}{1.050000in}}%
\pgfpathlineto{\pgfqpoint{5.886840in}{1.505000in}}%
\pgfpathlineto{\pgfqpoint{5.888080in}{1.365000in}}%
\pgfpathlineto{\pgfqpoint{5.889320in}{1.400000in}}%
\pgfpathlineto{\pgfqpoint{5.890560in}{1.400000in}}%
\pgfpathlineto{\pgfqpoint{5.893040in}{1.750000in}}%
\pgfpathlineto{\pgfqpoint{5.894280in}{1.785000in}}%
\pgfpathlineto{\pgfqpoint{5.895520in}{1.260000in}}%
\pgfpathlineto{\pgfqpoint{5.896760in}{1.575000in}}%
\pgfpathlineto{\pgfqpoint{5.898000in}{1.435000in}}%
\pgfpathlineto{\pgfqpoint{5.899240in}{1.540000in}}%
\pgfpathlineto{\pgfqpoint{5.900480in}{1.435000in}}%
\pgfpathlineto{\pgfqpoint{5.901720in}{1.715000in}}%
\pgfpathlineto{\pgfqpoint{5.902960in}{1.435000in}}%
\pgfpathlineto{\pgfqpoint{5.904200in}{1.505000in}}%
\pgfpathlineto{\pgfqpoint{5.905440in}{1.365000in}}%
\pgfpathlineto{\pgfqpoint{5.906680in}{1.785000in}}%
\pgfpathlineto{\pgfqpoint{5.907920in}{1.785000in}}%
\pgfpathlineto{\pgfqpoint{5.910400in}{1.260000in}}%
\pgfpathlineto{\pgfqpoint{5.911640in}{1.260000in}}%
\pgfpathlineto{\pgfqpoint{5.912880in}{1.610000in}}%
\pgfpathlineto{\pgfqpoint{5.914120in}{1.610000in}}%
\pgfpathlineto{\pgfqpoint{5.915360in}{1.995000in}}%
\pgfpathlineto{\pgfqpoint{5.916600in}{1.540000in}}%
\pgfpathlineto{\pgfqpoint{5.917840in}{1.540000in}}%
\pgfpathlineto{\pgfqpoint{5.919080in}{1.330000in}}%
\pgfpathlineto{\pgfqpoint{5.920320in}{1.540000in}}%
\pgfpathlineto{\pgfqpoint{5.921560in}{0.980000in}}%
\pgfpathlineto{\pgfqpoint{5.922800in}{1.645000in}}%
\pgfpathlineto{\pgfqpoint{5.925280in}{1.120000in}}%
\pgfpathlineto{\pgfqpoint{5.926520in}{1.540000in}}%
\pgfpathlineto{\pgfqpoint{5.929000in}{1.295000in}}%
\pgfpathlineto{\pgfqpoint{5.930240in}{1.365000in}}%
\pgfpathlineto{\pgfqpoint{5.931480in}{1.645000in}}%
\pgfpathlineto{\pgfqpoint{5.932720in}{1.505000in}}%
\pgfpathlineto{\pgfqpoint{5.933960in}{1.855000in}}%
\pgfpathlineto{\pgfqpoint{5.936440in}{1.190000in}}%
\pgfpathlineto{\pgfqpoint{5.937680in}{1.330000in}}%
\pgfpathlineto{\pgfqpoint{5.938920in}{1.260000in}}%
\pgfpathlineto{\pgfqpoint{5.940160in}{1.400000in}}%
\pgfpathlineto{\pgfqpoint{5.941400in}{1.365000in}}%
\pgfpathlineto{\pgfqpoint{5.942640in}{1.890000in}}%
\pgfpathlineto{\pgfqpoint{5.945120in}{1.505000in}}%
\pgfpathlineto{\pgfqpoint{5.946360in}{1.680000in}}%
\pgfpathlineto{\pgfqpoint{5.947600in}{1.645000in}}%
\pgfpathlineto{\pgfqpoint{5.948840in}{1.120000in}}%
\pgfpathlineto{\pgfqpoint{5.950080in}{1.715000in}}%
\pgfpathlineto{\pgfqpoint{5.951320in}{1.365000in}}%
\pgfpathlineto{\pgfqpoint{5.953800in}{1.540000in}}%
\pgfpathlineto{\pgfqpoint{5.956280in}{1.120000in}}%
\pgfpathlineto{\pgfqpoint{5.957520in}{1.260000in}}%
\pgfpathlineto{\pgfqpoint{5.958760in}{1.085000in}}%
\pgfpathlineto{\pgfqpoint{5.962480in}{1.925000in}}%
\pgfpathlineto{\pgfqpoint{5.963720in}{1.575000in}}%
\pgfpathlineto{\pgfqpoint{5.964960in}{1.540000in}}%
\pgfpathlineto{\pgfqpoint{5.966200in}{1.435000in}}%
\pgfpathlineto{\pgfqpoint{5.967440in}{1.820000in}}%
\pgfpathlineto{\pgfqpoint{5.968680in}{0.980000in}}%
\pgfpathlineto{\pgfqpoint{5.969920in}{1.575000in}}%
\pgfpathlineto{\pgfqpoint{5.971160in}{1.540000in}}%
\pgfpathlineto{\pgfqpoint{5.972400in}{1.960000in}}%
\pgfpathlineto{\pgfqpoint{5.976120in}{1.260000in}}%
\pgfpathlineto{\pgfqpoint{5.977360in}{1.330000in}}%
\pgfpathlineto{\pgfqpoint{5.978600in}{1.295000in}}%
\pgfpathlineto{\pgfqpoint{5.979840in}{1.435000in}}%
\pgfpathlineto{\pgfqpoint{5.981080in}{1.260000in}}%
\pgfpathlineto{\pgfqpoint{5.982320in}{1.260000in}}%
\pgfpathlineto{\pgfqpoint{5.983560in}{1.470000in}}%
\pgfpathlineto{\pgfqpoint{5.986040in}{1.155000in}}%
\pgfpathlineto{\pgfqpoint{5.988520in}{1.610000in}}%
\pgfpathlineto{\pgfqpoint{5.991000in}{1.295000in}}%
\pgfpathlineto{\pgfqpoint{5.992240in}{1.260000in}}%
\pgfpathlineto{\pgfqpoint{5.993480in}{1.575000in}}%
\pgfpathlineto{\pgfqpoint{5.995960in}{1.155000in}}%
\pgfpathlineto{\pgfqpoint{5.997200in}{1.155000in}}%
\pgfpathlineto{\pgfqpoint{5.998440in}{1.470000in}}%
\pgfpathlineto{\pgfqpoint{5.999680in}{1.085000in}}%
\pgfpathlineto{\pgfqpoint{6.002160in}{1.505000in}}%
\pgfpathlineto{\pgfqpoint{6.003400in}{1.365000in}}%
\pgfpathlineto{\pgfqpoint{6.004640in}{1.505000in}}%
\pgfpathlineto{\pgfqpoint{6.005880in}{1.085000in}}%
\pgfpathlineto{\pgfqpoint{6.007120in}{1.610000in}}%
\pgfpathlineto{\pgfqpoint{6.008360in}{1.505000in}}%
\pgfpathlineto{\pgfqpoint{6.009600in}{1.540000in}}%
\pgfpathlineto{\pgfqpoint{6.010840in}{2.030000in}}%
\pgfpathlineto{\pgfqpoint{6.012080in}{1.435000in}}%
\pgfpathlineto{\pgfqpoint{6.013320in}{1.995000in}}%
\pgfpathlineto{\pgfqpoint{6.015800in}{1.680000in}}%
\pgfpathlineto{\pgfqpoint{6.017040in}{1.785000in}}%
\pgfpathlineto{\pgfqpoint{6.018280in}{1.575000in}}%
\pgfpathlineto{\pgfqpoint{6.019520in}{1.750000in}}%
\pgfpathlineto{\pgfqpoint{6.020760in}{1.715000in}}%
\pgfpathlineto{\pgfqpoint{6.022000in}{1.610000in}}%
\pgfpathlineto{\pgfqpoint{6.023240in}{1.260000in}}%
\pgfpathlineto{\pgfqpoint{6.024480in}{1.295000in}}%
\pgfpathlineto{\pgfqpoint{6.025720in}{1.610000in}}%
\pgfpathlineto{\pgfqpoint{6.026960in}{0.980000in}}%
\pgfpathlineto{\pgfqpoint{6.028200in}{1.190000in}}%
\pgfpathlineto{\pgfqpoint{6.029440in}{1.820000in}}%
\pgfpathlineto{\pgfqpoint{6.030680in}{1.750000in}}%
\pgfpathlineto{\pgfqpoint{6.031920in}{1.505000in}}%
\pgfpathlineto{\pgfqpoint{6.033160in}{1.785000in}}%
\pgfpathlineto{\pgfqpoint{6.034400in}{1.750000in}}%
\pgfpathlineto{\pgfqpoint{6.035640in}{1.540000in}}%
\pgfpathlineto{\pgfqpoint{6.036880in}{1.680000in}}%
\pgfpathlineto{\pgfqpoint{6.038120in}{1.960000in}}%
\pgfpathlineto{\pgfqpoint{6.039360in}{1.435000in}}%
\pgfpathlineto{\pgfqpoint{6.040600in}{1.505000in}}%
\pgfpathlineto{\pgfqpoint{6.041840in}{1.925000in}}%
\pgfpathlineto{\pgfqpoint{6.043080in}{1.330000in}}%
\pgfpathlineto{\pgfqpoint{6.045560in}{1.785000in}}%
\pgfpathlineto{\pgfqpoint{6.049280in}{0.980000in}}%
\pgfpathlineto{\pgfqpoint{6.050520in}{1.785000in}}%
\pgfpathlineto{\pgfqpoint{6.051760in}{1.400000in}}%
\pgfpathlineto{\pgfqpoint{6.054240in}{1.680000in}}%
\pgfpathlineto{\pgfqpoint{6.055480in}{1.435000in}}%
\pgfpathlineto{\pgfqpoint{6.056720in}{1.715000in}}%
\pgfpathlineto{\pgfqpoint{6.057960in}{1.680000in}}%
\pgfpathlineto{\pgfqpoint{6.060440in}{1.295000in}}%
\pgfpathlineto{\pgfqpoint{6.061680in}{1.575000in}}%
\pgfpathlineto{\pgfqpoint{6.062920in}{1.155000in}}%
\pgfpathlineto{\pgfqpoint{6.064160in}{1.575000in}}%
\pgfpathlineto{\pgfqpoint{6.065400in}{1.470000in}}%
\pgfpathlineto{\pgfqpoint{6.066640in}{1.645000in}}%
\pgfpathlineto{\pgfqpoint{6.067880in}{1.470000in}}%
\pgfpathlineto{\pgfqpoint{6.071600in}{1.715000in}}%
\pgfpathlineto{\pgfqpoint{6.072840in}{1.610000in}}%
\pgfpathlineto{\pgfqpoint{6.075320in}{1.855000in}}%
\pgfpathlineto{\pgfqpoint{6.076560in}{1.575000in}}%
\pgfpathlineto{\pgfqpoint{6.077800in}{1.575000in}}%
\pgfpathlineto{\pgfqpoint{6.079040in}{1.190000in}}%
\pgfpathlineto{\pgfqpoint{6.080280in}{1.365000in}}%
\pgfpathlineto{\pgfqpoint{6.081520in}{1.365000in}}%
\pgfpathlineto{\pgfqpoint{6.082760in}{1.505000in}}%
\pgfpathlineto{\pgfqpoint{6.084000in}{1.470000in}}%
\pgfpathlineto{\pgfqpoint{6.085240in}{1.260000in}}%
\pgfpathlineto{\pgfqpoint{6.088960in}{1.715000in}}%
\pgfpathlineto{\pgfqpoint{6.090200in}{1.505000in}}%
\pgfpathlineto{\pgfqpoint{6.091440in}{1.505000in}}%
\pgfpathlineto{\pgfqpoint{6.092680in}{1.400000in}}%
\pgfpathlineto{\pgfqpoint{6.093920in}{1.505000in}}%
\pgfpathlineto{\pgfqpoint{6.095160in}{1.365000in}}%
\pgfpathlineto{\pgfqpoint{6.096400in}{0.805000in}}%
\pgfpathlineto{\pgfqpoint{6.097640in}{1.960000in}}%
\pgfpathlineto{\pgfqpoint{6.098880in}{1.680000in}}%
\pgfpathlineto{\pgfqpoint{6.100120in}{1.855000in}}%
\pgfpathlineto{\pgfqpoint{6.101360in}{1.190000in}}%
\pgfpathlineto{\pgfqpoint{6.102600in}{1.435000in}}%
\pgfpathlineto{\pgfqpoint{6.103840in}{1.155000in}}%
\pgfpathlineto{\pgfqpoint{6.105080in}{1.400000in}}%
\pgfpathlineto{\pgfqpoint{6.106320in}{1.050000in}}%
\pgfpathlineto{\pgfqpoint{6.108800in}{1.680000in}}%
\pgfpathlineto{\pgfqpoint{6.111280in}{1.295000in}}%
\pgfpathlineto{\pgfqpoint{6.112520in}{1.610000in}}%
\pgfpathlineto{\pgfqpoint{6.113760in}{1.085000in}}%
\pgfpathlineto{\pgfqpoint{6.115000in}{1.470000in}}%
\pgfpathlineto{\pgfqpoint{6.116240in}{1.365000in}}%
\pgfpathlineto{\pgfqpoint{6.117480in}{1.540000in}}%
\pgfpathlineto{\pgfqpoint{6.118720in}{1.470000in}}%
\pgfpathlineto{\pgfqpoint{6.119960in}{1.295000in}}%
\pgfpathlineto{\pgfqpoint{6.121200in}{1.645000in}}%
\pgfpathlineto{\pgfqpoint{6.122440in}{1.225000in}}%
\pgfpathlineto{\pgfqpoint{6.124920in}{1.855000in}}%
\pgfpathlineto{\pgfqpoint{6.126160in}{1.435000in}}%
\pgfpathlineto{\pgfqpoint{6.127400in}{1.785000in}}%
\pgfpathlineto{\pgfqpoint{6.128640in}{1.295000in}}%
\pgfpathlineto{\pgfqpoint{6.129880in}{1.330000in}}%
\pgfpathlineto{\pgfqpoint{6.131120in}{1.400000in}}%
\pgfpathlineto{\pgfqpoint{6.132360in}{1.365000in}}%
\pgfpathlineto{\pgfqpoint{6.133600in}{1.470000in}}%
\pgfpathlineto{\pgfqpoint{6.134840in}{1.365000in}}%
\pgfpathlineto{\pgfqpoint{6.136080in}{1.120000in}}%
\pgfpathlineto{\pgfqpoint{6.137320in}{1.890000in}}%
\pgfpathlineto{\pgfqpoint{6.138560in}{1.925000in}}%
\pgfpathlineto{\pgfqpoint{6.139800in}{1.330000in}}%
\pgfpathlineto{\pgfqpoint{6.141040in}{1.715000in}}%
\pgfpathlineto{\pgfqpoint{6.142280in}{1.680000in}}%
\pgfpathlineto{\pgfqpoint{6.143520in}{1.960000in}}%
\pgfpathlineto{\pgfqpoint{6.144760in}{1.400000in}}%
\pgfpathlineto{\pgfqpoint{6.146000in}{1.540000in}}%
\pgfpathlineto{\pgfqpoint{6.147240in}{1.470000in}}%
\pgfpathlineto{\pgfqpoint{6.148480in}{1.540000in}}%
\pgfpathlineto{\pgfqpoint{6.149720in}{1.260000in}}%
\pgfpathlineto{\pgfqpoint{6.152200in}{1.330000in}}%
\pgfpathlineto{\pgfqpoint{6.153440in}{1.260000in}}%
\pgfpathlineto{\pgfqpoint{6.155920in}{1.540000in}}%
\pgfpathlineto{\pgfqpoint{6.157160in}{1.435000in}}%
\pgfpathlineto{\pgfqpoint{6.158400in}{0.875000in}}%
\pgfpathlineto{\pgfqpoint{6.159640in}{1.470000in}}%
\pgfpathlineto{\pgfqpoint{6.160880in}{1.505000in}}%
\pgfpathlineto{\pgfqpoint{6.162120in}{1.330000in}}%
\pgfpathlineto{\pgfqpoint{6.164600in}{1.330000in}}%
\pgfpathlineto{\pgfqpoint{6.165840in}{1.260000in}}%
\pgfpathlineto{\pgfqpoint{6.167080in}{0.980000in}}%
\pgfpathlineto{\pgfqpoint{6.168320in}{1.540000in}}%
\pgfpathlineto{\pgfqpoint{6.169560in}{1.540000in}}%
\pgfpathlineto{\pgfqpoint{6.172040in}{1.155000in}}%
\pgfpathlineto{\pgfqpoint{6.173280in}{1.330000in}}%
\pgfpathlineto{\pgfqpoint{6.174520in}{1.190000in}}%
\pgfpathlineto{\pgfqpoint{6.175760in}{1.400000in}}%
\pgfpathlineto{\pgfqpoint{6.177000in}{1.120000in}}%
\pgfpathlineto{\pgfqpoint{6.180720in}{1.925000in}}%
\pgfpathlineto{\pgfqpoint{6.181960in}{1.505000in}}%
\pgfpathlineto{\pgfqpoint{6.183200in}{1.610000in}}%
\pgfpathlineto{\pgfqpoint{6.185680in}{1.015000in}}%
\pgfpathlineto{\pgfqpoint{6.186920in}{1.470000in}}%
\pgfpathlineto{\pgfqpoint{6.189400in}{1.295000in}}%
\pgfpathlineto{\pgfqpoint{6.190640in}{0.980000in}}%
\pgfpathlineto{\pgfqpoint{6.191880in}{1.715000in}}%
\pgfpathlineto{\pgfqpoint{6.193120in}{1.260000in}}%
\pgfpathlineto{\pgfqpoint{6.194360in}{1.575000in}}%
\pgfpathlineto{\pgfqpoint{6.195600in}{1.470000in}}%
\pgfpathlineto{\pgfqpoint{6.196840in}{1.575000in}}%
\pgfpathlineto{\pgfqpoint{6.198080in}{1.260000in}}%
\pgfpathlineto{\pgfqpoint{6.199320in}{1.260000in}}%
\pgfpathlineto{\pgfqpoint{6.203040in}{1.645000in}}%
\pgfpathlineto{\pgfqpoint{6.205520in}{1.225000in}}%
\pgfpathlineto{\pgfqpoint{6.206760in}{1.190000in}}%
\pgfpathlineto{\pgfqpoint{6.208000in}{1.785000in}}%
\pgfpathlineto{\pgfqpoint{6.209240in}{1.120000in}}%
\pgfpathlineto{\pgfqpoint{6.212960in}{1.820000in}}%
\pgfpathlineto{\pgfqpoint{6.214200in}{1.855000in}}%
\pgfpathlineto{\pgfqpoint{6.215440in}{1.400000in}}%
\pgfpathlineto{\pgfqpoint{6.216680in}{1.750000in}}%
\pgfpathlineto{\pgfqpoint{6.219160in}{1.085000in}}%
\pgfpathlineto{\pgfqpoint{6.220400in}{1.645000in}}%
\pgfpathlineto{\pgfqpoint{6.221640in}{1.575000in}}%
\pgfpathlineto{\pgfqpoint{6.222880in}{1.365000in}}%
\pgfpathlineto{\pgfqpoint{6.224120in}{1.470000in}}%
\pgfpathlineto{\pgfqpoint{6.225360in}{1.680000in}}%
\pgfpathlineto{\pgfqpoint{6.226600in}{1.610000in}}%
\pgfpathlineto{\pgfqpoint{6.227840in}{1.680000in}}%
\pgfpathlineto{\pgfqpoint{6.230320in}{1.155000in}}%
\pgfpathlineto{\pgfqpoint{6.232800in}{1.645000in}}%
\pgfpathlineto{\pgfqpoint{6.234040in}{1.015000in}}%
\pgfpathlineto{\pgfqpoint{6.236520in}{1.610000in}}%
\pgfpathlineto{\pgfqpoint{6.237760in}{0.980000in}}%
\pgfpathlineto{\pgfqpoint{6.239000in}{1.995000in}}%
\pgfpathlineto{\pgfqpoint{6.240240in}{1.260000in}}%
\pgfpathlineto{\pgfqpoint{6.241480in}{1.855000in}}%
\pgfpathlineto{\pgfqpoint{6.242720in}{1.575000in}}%
\pgfpathlineto{\pgfqpoint{6.243960in}{1.680000in}}%
\pgfpathlineto{\pgfqpoint{6.245200in}{1.260000in}}%
\pgfpathlineto{\pgfqpoint{6.247680in}{1.190000in}}%
\pgfpathlineto{\pgfqpoint{6.248920in}{1.505000in}}%
\pgfpathlineto{\pgfqpoint{6.250160in}{1.260000in}}%
\pgfpathlineto{\pgfqpoint{6.251400in}{1.365000in}}%
\pgfpathlineto{\pgfqpoint{6.252640in}{1.085000in}}%
\pgfpathlineto{\pgfqpoint{6.253880in}{1.295000in}}%
\pgfpathlineto{\pgfqpoint{6.255120in}{1.050000in}}%
\pgfpathlineto{\pgfqpoint{6.257600in}{2.030000in}}%
\pgfpathlineto{\pgfqpoint{6.260080in}{1.540000in}}%
\pgfpathlineto{\pgfqpoint{6.261320in}{1.365000in}}%
\pgfpathlineto{\pgfqpoint{6.262560in}{1.505000in}}%
\pgfpathlineto{\pgfqpoint{6.263800in}{1.225000in}}%
\pgfpathlineto{\pgfqpoint{6.265040in}{1.715000in}}%
\pgfpathlineto{\pgfqpoint{6.266280in}{1.225000in}}%
\pgfpathlineto{\pgfqpoint{6.268760in}{1.470000in}}%
\pgfpathlineto{\pgfqpoint{6.270000in}{1.190000in}}%
\pgfpathlineto{\pgfqpoint{6.271240in}{1.225000in}}%
\pgfpathlineto{\pgfqpoint{6.272480in}{1.575000in}}%
\pgfpathlineto{\pgfqpoint{6.273720in}{1.575000in}}%
\pgfpathlineto{\pgfqpoint{6.274960in}{1.330000in}}%
\pgfpathlineto{\pgfqpoint{6.278680in}{1.750000in}}%
\pgfpathlineto{\pgfqpoint{6.279920in}{1.715000in}}%
\pgfpathlineto{\pgfqpoint{6.283640in}{1.295000in}}%
\pgfpathlineto{\pgfqpoint{6.284880in}{1.540000in}}%
\pgfpathlineto{\pgfqpoint{6.287360in}{1.190000in}}%
\pgfpathlineto{\pgfqpoint{6.288600in}{1.890000in}}%
\pgfpathlineto{\pgfqpoint{6.289840in}{1.155000in}}%
\pgfpathlineto{\pgfqpoint{6.291080in}{1.085000in}}%
\pgfpathlineto{\pgfqpoint{6.292320in}{1.575000in}}%
\pgfpathlineto{\pgfqpoint{6.293560in}{0.980000in}}%
\pgfpathlineto{\pgfqpoint{6.294800in}{1.645000in}}%
\pgfpathlineto{\pgfqpoint{6.296040in}{1.470000in}}%
\pgfpathlineto{\pgfqpoint{6.297280in}{1.645000in}}%
\pgfpathlineto{\pgfqpoint{6.298520in}{1.470000in}}%
\pgfpathlineto{\pgfqpoint{6.299760in}{1.085000in}}%
\pgfpathlineto{\pgfqpoint{6.301000in}{1.400000in}}%
\pgfpathlineto{\pgfqpoint{6.302240in}{1.295000in}}%
\pgfpathlineto{\pgfqpoint{6.304720in}{1.715000in}}%
\pgfpathlineto{\pgfqpoint{6.305960in}{1.610000in}}%
\pgfpathlineto{\pgfqpoint{6.307200in}{1.260000in}}%
\pgfpathlineto{\pgfqpoint{6.308440in}{1.400000in}}%
\pgfpathlineto{\pgfqpoint{6.309680in}{1.260000in}}%
\pgfpathlineto{\pgfqpoint{6.312160in}{1.540000in}}%
\pgfpathlineto{\pgfqpoint{6.314640in}{1.050000in}}%
\pgfpathlineto{\pgfqpoint{6.315880in}{1.610000in}}%
\pgfpathlineto{\pgfqpoint{6.317120in}{1.435000in}}%
\pgfpathlineto{\pgfqpoint{6.318360in}{1.540000in}}%
\pgfpathlineto{\pgfqpoint{6.319600in}{0.875000in}}%
\pgfpathlineto{\pgfqpoint{6.320840in}{1.225000in}}%
\pgfpathlineto{\pgfqpoint{6.322080in}{1.225000in}}%
\pgfpathlineto{\pgfqpoint{6.323320in}{1.540000in}}%
\pgfpathlineto{\pgfqpoint{6.327040in}{1.295000in}}%
\pgfpathlineto{\pgfqpoint{6.328280in}{0.735000in}}%
\pgfpathlineto{\pgfqpoint{6.329520in}{1.295000in}}%
\pgfpathlineto{\pgfqpoint{6.330760in}{1.155000in}}%
\pgfpathlineto{\pgfqpoint{6.332000in}{1.505000in}}%
\pgfpathlineto{\pgfqpoint{6.333240in}{1.190000in}}%
\pgfpathlineto{\pgfqpoint{6.334480in}{1.680000in}}%
\pgfpathlineto{\pgfqpoint{6.335720in}{1.540000in}}%
\pgfpathlineto{\pgfqpoint{6.336960in}{1.610000in}}%
\pgfpathlineto{\pgfqpoint{6.338200in}{1.015000in}}%
\pgfpathlineto{\pgfqpoint{6.339440in}{1.575000in}}%
\pgfpathlineto{\pgfqpoint{6.340680in}{1.400000in}}%
\pgfpathlineto{\pgfqpoint{6.341920in}{1.400000in}}%
\pgfpathlineto{\pgfqpoint{6.343160in}{1.120000in}}%
\pgfpathlineto{\pgfqpoint{6.345640in}{1.505000in}}%
\pgfpathlineto{\pgfqpoint{6.346880in}{1.400000in}}%
\pgfpathlineto{\pgfqpoint{6.349360in}{1.680000in}}%
\pgfpathlineto{\pgfqpoint{6.350600in}{1.155000in}}%
\pgfpathlineto{\pgfqpoint{6.351840in}{1.155000in}}%
\pgfpathlineto{\pgfqpoint{6.353080in}{1.575000in}}%
\pgfpathlineto{\pgfqpoint{6.354320in}{1.575000in}}%
\pgfpathlineto{\pgfqpoint{6.355560in}{1.085000in}}%
\pgfpathlineto{\pgfqpoint{6.356800in}{1.295000in}}%
\pgfpathlineto{\pgfqpoint{6.358040in}{1.260000in}}%
\pgfpathlineto{\pgfqpoint{6.359280in}{1.680000in}}%
\pgfpathlineto{\pgfqpoint{6.360520in}{1.400000in}}%
\pgfpathlineto{\pgfqpoint{6.361760in}{1.435000in}}%
\pgfpathlineto{\pgfqpoint{6.363000in}{1.540000in}}%
\pgfpathlineto{\pgfqpoint{6.364240in}{0.980000in}}%
\pgfpathlineto{\pgfqpoint{6.365480in}{1.575000in}}%
\pgfpathlineto{\pgfqpoint{6.366720in}{1.540000in}}%
\pgfpathlineto{\pgfqpoint{6.367960in}{1.855000in}}%
\pgfpathlineto{\pgfqpoint{6.369200in}{1.225000in}}%
\pgfpathlineto{\pgfqpoint{6.370440in}{1.330000in}}%
\pgfpathlineto{\pgfqpoint{6.372920in}{1.260000in}}%
\pgfpathlineto{\pgfqpoint{6.375400in}{1.575000in}}%
\pgfpathlineto{\pgfqpoint{6.376640in}{1.575000in}}%
\pgfpathlineto{\pgfqpoint{6.379120in}{1.715000in}}%
\pgfpathlineto{\pgfqpoint{6.380360in}{1.190000in}}%
\pgfpathlineto{\pgfqpoint{6.381600in}{1.645000in}}%
\pgfpathlineto{\pgfqpoint{6.382840in}{1.365000in}}%
\pgfpathlineto{\pgfqpoint{6.385320in}{1.575000in}}%
\pgfpathlineto{\pgfqpoint{6.386560in}{1.155000in}}%
\pgfpathlineto{\pgfqpoint{6.389040in}{1.505000in}}%
\pgfpathlineto{\pgfqpoint{6.390280in}{1.435000in}}%
\pgfpathlineto{\pgfqpoint{6.391520in}{1.470000in}}%
\pgfpathlineto{\pgfqpoint{6.392760in}{1.645000in}}%
\pgfpathlineto{\pgfqpoint{6.396480in}{1.260000in}}%
\pgfpathlineto{\pgfqpoint{6.397720in}{1.225000in}}%
\pgfpathlineto{\pgfqpoint{6.398960in}{1.260000in}}%
\pgfpathlineto{\pgfqpoint{6.400200in}{1.050000in}}%
\pgfpathlineto{\pgfqpoint{6.401440in}{1.575000in}}%
\pgfpathlineto{\pgfqpoint{6.403920in}{1.435000in}}%
\pgfpathlineto{\pgfqpoint{6.405160in}{1.050000in}}%
\pgfpathlineto{\pgfqpoint{6.406400in}{1.050000in}}%
\pgfpathlineto{\pgfqpoint{6.410120in}{1.540000in}}%
\pgfpathlineto{\pgfqpoint{6.411360in}{1.155000in}}%
\pgfpathlineto{\pgfqpoint{6.412600in}{1.155000in}}%
\pgfpathlineto{\pgfqpoint{6.413840in}{1.365000in}}%
\pgfpathlineto{\pgfqpoint{6.415080in}{1.120000in}}%
\pgfpathlineto{\pgfqpoint{6.416320in}{1.190000in}}%
\pgfpathlineto{\pgfqpoint{6.417560in}{1.365000in}}%
\pgfpathlineto{\pgfqpoint{6.418800in}{1.330000in}}%
\pgfpathlineto{\pgfqpoint{6.420040in}{1.085000in}}%
\pgfpathlineto{\pgfqpoint{6.421280in}{1.120000in}}%
\pgfpathlineto{\pgfqpoint{6.422520in}{1.575000in}}%
\pgfpathlineto{\pgfqpoint{6.426240in}{1.120000in}}%
\pgfpathlineto{\pgfqpoint{6.427480in}{1.505000in}}%
\pgfpathlineto{\pgfqpoint{6.428720in}{1.400000in}}%
\pgfpathlineto{\pgfqpoint{6.429960in}{1.435000in}}%
\pgfpathlineto{\pgfqpoint{6.431200in}{1.435000in}}%
\pgfpathlineto{\pgfqpoint{6.433680in}{1.190000in}}%
\pgfpathlineto{\pgfqpoint{6.434920in}{1.575000in}}%
\pgfpathlineto{\pgfqpoint{6.437400in}{1.330000in}}%
\pgfpathlineto{\pgfqpoint{6.438640in}{1.610000in}}%
\pgfpathlineto{\pgfqpoint{6.439880in}{1.575000in}}%
\pgfpathlineto{\pgfqpoint{6.441120in}{1.680000in}}%
\pgfpathlineto{\pgfqpoint{6.442360in}{1.400000in}}%
\pgfpathlineto{\pgfqpoint{6.443600in}{1.750000in}}%
\pgfpathlineto{\pgfqpoint{6.444840in}{1.260000in}}%
\pgfpathlineto{\pgfqpoint{6.446080in}{1.820000in}}%
\pgfpathlineto{\pgfqpoint{6.451040in}{1.400000in}}%
\pgfpathlineto{\pgfqpoint{6.452280in}{1.645000in}}%
\pgfpathlineto{\pgfqpoint{6.453520in}{1.400000in}}%
\pgfpathlineto{\pgfqpoint{6.454760in}{1.435000in}}%
\pgfpathlineto{\pgfqpoint{6.456000in}{1.435000in}}%
\pgfpathlineto{\pgfqpoint{6.458480in}{1.190000in}}%
\pgfpathlineto{\pgfqpoint{6.459720in}{1.435000in}}%
\pgfpathlineto{\pgfqpoint{6.460960in}{1.330000in}}%
\pgfpathlineto{\pgfqpoint{6.462200in}{1.470000in}}%
\pgfpathlineto{\pgfqpoint{6.463440in}{1.295000in}}%
\pgfpathlineto{\pgfqpoint{6.464680in}{1.505000in}}%
\pgfpathlineto{\pgfqpoint{6.467160in}{1.120000in}}%
\pgfpathlineto{\pgfqpoint{6.468400in}{1.330000in}}%
\pgfpathlineto{\pgfqpoint{6.469640in}{1.120000in}}%
\pgfpathlineto{\pgfqpoint{6.470880in}{1.750000in}}%
\pgfpathlineto{\pgfqpoint{6.472120in}{1.330000in}}%
\pgfpathlineto{\pgfqpoint{6.473360in}{1.575000in}}%
\pgfpathlineto{\pgfqpoint{6.474600in}{1.505000in}}%
\pgfpathlineto{\pgfqpoint{6.475840in}{1.855000in}}%
\pgfpathlineto{\pgfqpoint{6.477080in}{1.715000in}}%
\pgfpathlineto{\pgfqpoint{6.478320in}{1.820000in}}%
\pgfpathlineto{\pgfqpoint{6.482040in}{1.155000in}}%
\pgfpathlineto{\pgfqpoint{6.484520in}{1.365000in}}%
\pgfpathlineto{\pgfqpoint{6.485760in}{1.260000in}}%
\pgfpathlineto{\pgfqpoint{6.488240in}{1.260000in}}%
\pgfpathlineto{\pgfqpoint{6.489480in}{1.400000in}}%
\pgfpathlineto{\pgfqpoint{6.490720in}{1.155000in}}%
\pgfpathlineto{\pgfqpoint{6.493200in}{1.575000in}}%
\pgfpathlineto{\pgfqpoint{6.494440in}{1.120000in}}%
\pgfpathlineto{\pgfqpoint{6.495680in}{1.645000in}}%
\pgfpathlineto{\pgfqpoint{6.498160in}{1.155000in}}%
\pgfpathlineto{\pgfqpoint{6.500640in}{1.575000in}}%
\pgfpathlineto{\pgfqpoint{6.501880in}{1.610000in}}%
\pgfpathlineto{\pgfqpoint{6.503120in}{1.085000in}}%
\pgfpathlineto{\pgfqpoint{6.505600in}{1.750000in}}%
\pgfpathlineto{\pgfqpoint{6.506840in}{1.505000in}}%
\pgfpathlineto{\pgfqpoint{6.508080in}{1.610000in}}%
\pgfpathlineto{\pgfqpoint{6.509320in}{1.435000in}}%
\pgfpathlineto{\pgfqpoint{6.510560in}{1.785000in}}%
\pgfpathlineto{\pgfqpoint{6.511800in}{1.820000in}}%
\pgfpathlineto{\pgfqpoint{6.513040in}{1.295000in}}%
\pgfpathlineto{\pgfqpoint{6.515520in}{1.610000in}}%
\pgfpathlineto{\pgfqpoint{6.516760in}{0.840000in}}%
\pgfpathlineto{\pgfqpoint{6.518000in}{1.400000in}}%
\pgfpathlineto{\pgfqpoint{6.519240in}{1.295000in}}%
\pgfpathlineto{\pgfqpoint{6.520480in}{0.980000in}}%
\pgfpathlineto{\pgfqpoint{6.521720in}{1.330000in}}%
\pgfpathlineto{\pgfqpoint{6.522960in}{1.155000in}}%
\pgfpathlineto{\pgfqpoint{6.524200in}{1.470000in}}%
\pgfpathlineto{\pgfqpoint{6.525440in}{1.435000in}}%
\pgfpathlineto{\pgfqpoint{6.526680in}{1.330000in}}%
\pgfpathlineto{\pgfqpoint{6.527920in}{1.925000in}}%
\pgfpathlineto{\pgfqpoint{6.530400in}{1.330000in}}%
\pgfpathlineto{\pgfqpoint{6.531640in}{1.680000in}}%
\pgfpathlineto{\pgfqpoint{6.532880in}{1.435000in}}%
\pgfpathlineto{\pgfqpoint{6.534120in}{1.750000in}}%
\pgfpathlineto{\pgfqpoint{6.535360in}{1.225000in}}%
\pgfpathlineto{\pgfqpoint{6.537840in}{1.505000in}}%
\pgfpathlineto{\pgfqpoint{6.539080in}{1.120000in}}%
\pgfpathlineto{\pgfqpoint{6.540320in}{1.750000in}}%
\pgfpathlineto{\pgfqpoint{6.541560in}{0.980000in}}%
\pgfpathlineto{\pgfqpoint{6.542800in}{1.855000in}}%
\pgfpathlineto{\pgfqpoint{6.544040in}{1.085000in}}%
\pgfpathlineto{\pgfqpoint{6.545280in}{1.225000in}}%
\pgfpathlineto{\pgfqpoint{6.546520in}{1.785000in}}%
\pgfpathlineto{\pgfqpoint{6.547760in}{1.680000in}}%
\pgfpathlineto{\pgfqpoint{6.549000in}{1.330000in}}%
\pgfpathlineto{\pgfqpoint{6.550240in}{1.680000in}}%
\pgfpathlineto{\pgfqpoint{6.551480in}{1.680000in}}%
\pgfpathlineto{\pgfqpoint{6.552720in}{1.575000in}}%
\pgfpathlineto{\pgfqpoint{6.553960in}{1.295000in}}%
\pgfpathlineto{\pgfqpoint{6.555200in}{1.750000in}}%
\pgfpathlineto{\pgfqpoint{6.556440in}{1.610000in}}%
\pgfpathlineto{\pgfqpoint{6.557680in}{1.680000in}}%
\pgfpathlineto{\pgfqpoint{6.560160in}{1.085000in}}%
\pgfpathlineto{\pgfqpoint{6.561400in}{1.750000in}}%
\pgfpathlineto{\pgfqpoint{6.562640in}{1.085000in}}%
\pgfpathlineto{\pgfqpoint{6.565120in}{1.750000in}}%
\pgfpathlineto{\pgfqpoint{6.570080in}{1.400000in}}%
\pgfpathlineto{\pgfqpoint{6.571320in}{1.505000in}}%
\pgfpathlineto{\pgfqpoint{6.573800in}{1.505000in}}%
\pgfpathlineto{\pgfqpoint{6.575040in}{1.610000in}}%
\pgfpathlineto{\pgfqpoint{6.577520in}{1.540000in}}%
\pgfpathlineto{\pgfqpoint{6.578760in}{1.575000in}}%
\pgfpathlineto{\pgfqpoint{6.580000in}{1.435000in}}%
\pgfpathlineto{\pgfqpoint{6.582480in}{0.805000in}}%
\pgfpathlineto{\pgfqpoint{6.583720in}{1.645000in}}%
\pgfpathlineto{\pgfqpoint{6.584960in}{1.295000in}}%
\pgfpathlineto{\pgfqpoint{6.586200in}{1.610000in}}%
\pgfpathlineto{\pgfqpoint{6.587440in}{1.435000in}}%
\pgfpathlineto{\pgfqpoint{6.589920in}{0.945000in}}%
\pgfpathlineto{\pgfqpoint{6.592400in}{1.540000in}}%
\pgfpathlineto{\pgfqpoint{6.593640in}{0.980000in}}%
\pgfpathlineto{\pgfqpoint{6.594880in}{1.225000in}}%
\pgfpathlineto{\pgfqpoint{6.596120in}{1.225000in}}%
\pgfpathlineto{\pgfqpoint{6.597360in}{1.400000in}}%
\pgfpathlineto{\pgfqpoint{6.598600in}{0.945000in}}%
\pgfpathlineto{\pgfqpoint{6.601080in}{2.065000in}}%
\pgfpathlineto{\pgfqpoint{6.604800in}{1.225000in}}%
\pgfpathlineto{\pgfqpoint{6.606040in}{1.610000in}}%
\pgfpathlineto{\pgfqpoint{6.607280in}{1.645000in}}%
\pgfpathlineto{\pgfqpoint{6.608520in}{1.050000in}}%
\pgfpathlineto{\pgfqpoint{6.609760in}{1.820000in}}%
\pgfpathlineto{\pgfqpoint{6.611000in}{1.155000in}}%
\pgfpathlineto{\pgfqpoint{6.612240in}{1.820000in}}%
\pgfpathlineto{\pgfqpoint{6.613480in}{1.785000in}}%
\pgfpathlineto{\pgfqpoint{6.614720in}{1.365000in}}%
\pgfpathlineto{\pgfqpoint{6.615960in}{1.820000in}}%
\pgfpathlineto{\pgfqpoint{6.618440in}{1.225000in}}%
\pgfpathlineto{\pgfqpoint{6.622160in}{1.400000in}}%
\pgfpathlineto{\pgfqpoint{6.623400in}{1.365000in}}%
\pgfpathlineto{\pgfqpoint{6.624640in}{1.365000in}}%
\pgfpathlineto{\pgfqpoint{6.625880in}{1.575000in}}%
\pgfpathlineto{\pgfqpoint{6.627120in}{1.330000in}}%
\pgfpathlineto{\pgfqpoint{6.629600in}{1.960000in}}%
\pgfpathlineto{\pgfqpoint{6.630840in}{1.855000in}}%
\pgfpathlineto{\pgfqpoint{6.632080in}{1.610000in}}%
\pgfpathlineto{\pgfqpoint{6.633320in}{1.960000in}}%
\pgfpathlineto{\pgfqpoint{6.635800in}{1.750000in}}%
\pgfpathlineto{\pgfqpoint{6.637040in}{1.540000in}}%
\pgfpathlineto{\pgfqpoint{6.638280in}{1.120000in}}%
\pgfpathlineto{\pgfqpoint{6.639520in}{1.610000in}}%
\pgfpathlineto{\pgfqpoint{6.640760in}{1.400000in}}%
\pgfpathlineto{\pgfqpoint{6.642000in}{1.680000in}}%
\pgfpathlineto{\pgfqpoint{6.643240in}{1.575000in}}%
\pgfpathlineto{\pgfqpoint{6.645720in}{1.225000in}}%
\pgfpathlineto{\pgfqpoint{6.646960in}{1.260000in}}%
\pgfpathlineto{\pgfqpoint{6.648200in}{1.435000in}}%
\pgfpathlineto{\pgfqpoint{6.649440in}{1.855000in}}%
\pgfpathlineto{\pgfqpoint{6.653160in}{1.330000in}}%
\pgfpathlineto{\pgfqpoint{6.655640in}{1.505000in}}%
\pgfpathlineto{\pgfqpoint{6.656880in}{1.365000in}}%
\pgfpathlineto{\pgfqpoint{6.658120in}{1.575000in}}%
\pgfpathlineto{\pgfqpoint{6.660600in}{1.260000in}}%
\pgfpathlineto{\pgfqpoint{6.661840in}{1.540000in}}%
\pgfpathlineto{\pgfqpoint{6.663080in}{1.120000in}}%
\pgfpathlineto{\pgfqpoint{6.665560in}{1.610000in}}%
\pgfpathlineto{\pgfqpoint{6.666800in}{1.680000in}}%
\pgfpathlineto{\pgfqpoint{6.668040in}{1.435000in}}%
\pgfpathlineto{\pgfqpoint{6.669280in}{1.540000in}}%
\pgfpathlineto{\pgfqpoint{6.670520in}{1.330000in}}%
\pgfpathlineto{\pgfqpoint{6.673000in}{1.715000in}}%
\pgfpathlineto{\pgfqpoint{6.675480in}{1.435000in}}%
\pgfpathlineto{\pgfqpoint{6.676720in}{1.960000in}}%
\pgfpathlineto{\pgfqpoint{6.677960in}{1.575000in}}%
\pgfpathlineto{\pgfqpoint{6.679200in}{1.645000in}}%
\pgfpathlineto{\pgfqpoint{6.681680in}{1.365000in}}%
\pgfpathlineto{\pgfqpoint{6.682920in}{1.435000in}}%
\pgfpathlineto{\pgfqpoint{6.684160in}{1.295000in}}%
\pgfpathlineto{\pgfqpoint{6.686640in}{1.610000in}}%
\pgfpathlineto{\pgfqpoint{6.689120in}{1.225000in}}%
\pgfpathlineto{\pgfqpoint{6.690360in}{1.085000in}}%
\pgfpathlineto{\pgfqpoint{6.691600in}{1.120000in}}%
\pgfpathlineto{\pgfqpoint{6.692840in}{1.645000in}}%
\pgfpathlineto{\pgfqpoint{6.694080in}{1.470000in}}%
\pgfpathlineto{\pgfqpoint{6.695320in}{1.575000in}}%
\pgfpathlineto{\pgfqpoint{6.696560in}{1.505000in}}%
\pgfpathlineto{\pgfqpoint{6.697800in}{1.330000in}}%
\pgfpathlineto{\pgfqpoint{6.699040in}{1.365000in}}%
\pgfpathlineto{\pgfqpoint{6.700280in}{1.365000in}}%
\pgfpathlineto{\pgfqpoint{6.701520in}{1.470000in}}%
\pgfpathlineto{\pgfqpoint{6.702760in}{1.715000in}}%
\pgfpathlineto{\pgfqpoint{6.705240in}{1.330000in}}%
\pgfpathlineto{\pgfqpoint{6.706480in}{1.295000in}}%
\pgfpathlineto{\pgfqpoint{6.707720in}{1.750000in}}%
\pgfpathlineto{\pgfqpoint{6.708960in}{1.330000in}}%
\pgfpathlineto{\pgfqpoint{6.710200in}{1.295000in}}%
\pgfpathlineto{\pgfqpoint{6.711440in}{1.540000in}}%
\pgfpathlineto{\pgfqpoint{6.712680in}{1.225000in}}%
\pgfpathlineto{\pgfqpoint{6.715160in}{1.610000in}}%
\pgfpathlineto{\pgfqpoint{6.716400in}{1.610000in}}%
\pgfpathlineto{\pgfqpoint{6.717640in}{1.470000in}}%
\pgfpathlineto{\pgfqpoint{6.718880in}{1.470000in}}%
\pgfpathlineto{\pgfqpoint{6.720120in}{1.785000in}}%
\pgfpathlineto{\pgfqpoint{6.722600in}{1.260000in}}%
\pgfpathlineto{\pgfqpoint{6.723840in}{1.155000in}}%
\pgfpathlineto{\pgfqpoint{6.725080in}{1.540000in}}%
\pgfpathlineto{\pgfqpoint{6.726320in}{1.435000in}}%
\pgfpathlineto{\pgfqpoint{6.727560in}{1.435000in}}%
\pgfpathlineto{\pgfqpoint{6.728800in}{1.855000in}}%
\pgfpathlineto{\pgfqpoint{6.730040in}{1.540000in}}%
\pgfpathlineto{\pgfqpoint{6.731280in}{1.715000in}}%
\pgfpathlineto{\pgfqpoint{6.732520in}{1.680000in}}%
\pgfpathlineto{\pgfqpoint{6.733760in}{1.155000in}}%
\pgfpathlineto{\pgfqpoint{6.735000in}{1.330000in}}%
\pgfpathlineto{\pgfqpoint{6.736240in}{1.260000in}}%
\pgfpathlineto{\pgfqpoint{6.737480in}{1.680000in}}%
\pgfpathlineto{\pgfqpoint{6.738720in}{1.295000in}}%
\pgfpathlineto{\pgfqpoint{6.739960in}{1.330000in}}%
\pgfpathlineto{\pgfqpoint{6.741200in}{1.225000in}}%
\pgfpathlineto{\pgfqpoint{6.742440in}{1.645000in}}%
\pgfpathlineto{\pgfqpoint{6.743680in}{1.365000in}}%
\pgfpathlineto{\pgfqpoint{6.744920in}{1.435000in}}%
\pgfpathlineto{\pgfqpoint{6.746160in}{1.925000in}}%
\pgfpathlineto{\pgfqpoint{6.747400in}{1.680000in}}%
\pgfpathlineto{\pgfqpoint{6.748640in}{1.680000in}}%
\pgfpathlineto{\pgfqpoint{6.749880in}{1.470000in}}%
\pgfpathlineto{\pgfqpoint{6.751120in}{1.995000in}}%
\pgfpathlineto{\pgfqpoint{6.752360in}{1.540000in}}%
\pgfpathlineto{\pgfqpoint{6.753600in}{1.540000in}}%
\pgfpathlineto{\pgfqpoint{6.754840in}{1.470000in}}%
\pgfpathlineto{\pgfqpoint{6.756080in}{1.505000in}}%
\pgfpathlineto{\pgfqpoint{6.757320in}{1.155000in}}%
\pgfpathlineto{\pgfqpoint{6.758560in}{1.365000in}}%
\pgfpathlineto{\pgfqpoint{6.759800in}{2.030000in}}%
\pgfpathlineto{\pgfqpoint{6.762280in}{1.505000in}}%
\pgfpathlineto{\pgfqpoint{6.763520in}{1.470000in}}%
\pgfpathlineto{\pgfqpoint{6.766000in}{1.855000in}}%
\pgfpathlineto{\pgfqpoint{6.769720in}{1.435000in}}%
\pgfpathlineto{\pgfqpoint{6.770960in}{1.960000in}}%
\pgfpathlineto{\pgfqpoint{6.772200in}{1.750000in}}%
\pgfpathlineto{\pgfqpoint{6.773440in}{1.260000in}}%
\pgfpathlineto{\pgfqpoint{6.774680in}{1.540000in}}%
\pgfpathlineto{\pgfqpoint{6.775920in}{1.260000in}}%
\pgfpathlineto{\pgfqpoint{6.777160in}{1.575000in}}%
\pgfpathlineto{\pgfqpoint{6.778400in}{1.330000in}}%
\pgfpathlineto{\pgfqpoint{6.779640in}{1.330000in}}%
\pgfpathlineto{\pgfqpoint{6.780880in}{1.750000in}}%
\pgfpathlineto{\pgfqpoint{6.782120in}{1.540000in}}%
\pgfpathlineto{\pgfqpoint{6.783360in}{1.995000in}}%
\pgfpathlineto{\pgfqpoint{6.784600in}{1.820000in}}%
\pgfpathlineto{\pgfqpoint{6.785840in}{1.295000in}}%
\pgfpathlineto{\pgfqpoint{6.788320in}{1.890000in}}%
\pgfpathlineto{\pgfqpoint{6.789560in}{1.890000in}}%
\pgfpathlineto{\pgfqpoint{6.790800in}{1.295000in}}%
\pgfpathlineto{\pgfqpoint{6.793280in}{1.540000in}}%
\pgfpathlineto{\pgfqpoint{6.795760in}{1.295000in}}%
\pgfpathlineto{\pgfqpoint{6.798240in}{1.855000in}}%
\pgfpathlineto{\pgfqpoint{6.799480in}{1.155000in}}%
\pgfpathlineto{\pgfqpoint{6.800720in}{1.505000in}}%
\pgfpathlineto{\pgfqpoint{6.801960in}{1.470000in}}%
\pgfpathlineto{\pgfqpoint{6.803200in}{1.260000in}}%
\pgfpathlineto{\pgfqpoint{6.804440in}{1.540000in}}%
\pgfpathlineto{\pgfqpoint{6.805680in}{0.980000in}}%
\pgfpathlineto{\pgfqpoint{6.808160in}{1.470000in}}%
\pgfpathlineto{\pgfqpoint{6.809400in}{1.365000in}}%
\pgfpathlineto{\pgfqpoint{6.810640in}{1.120000in}}%
\pgfpathlineto{\pgfqpoint{6.811880in}{1.855000in}}%
\pgfpathlineto{\pgfqpoint{6.813120in}{1.470000in}}%
\pgfpathlineto{\pgfqpoint{6.815600in}{1.575000in}}%
\pgfpathlineto{\pgfqpoint{6.816840in}{1.260000in}}%
\pgfpathlineto{\pgfqpoint{6.818080in}{1.610000in}}%
\pgfpathlineto{\pgfqpoint{6.819320in}{1.610000in}}%
\pgfpathlineto{\pgfqpoint{6.820560in}{1.470000in}}%
\pgfpathlineto{\pgfqpoint{6.821800in}{1.505000in}}%
\pgfpathlineto{\pgfqpoint{6.823040in}{1.225000in}}%
\pgfpathlineto{\pgfqpoint{6.824280in}{1.225000in}}%
\pgfpathlineto{\pgfqpoint{6.825520in}{1.365000in}}%
\pgfpathlineto{\pgfqpoint{6.826760in}{1.225000in}}%
\pgfpathlineto{\pgfqpoint{6.828000in}{1.505000in}}%
\pgfpathlineto{\pgfqpoint{6.829240in}{0.875000in}}%
\pgfpathlineto{\pgfqpoint{6.831720in}{1.610000in}}%
\pgfpathlineto{\pgfqpoint{6.834200in}{1.400000in}}%
\pgfpathlineto{\pgfqpoint{6.835440in}{1.715000in}}%
\pgfpathlineto{\pgfqpoint{6.836680in}{1.715000in}}%
\pgfpathlineto{\pgfqpoint{6.837920in}{1.890000in}}%
\pgfpathlineto{\pgfqpoint{6.839160in}{1.505000in}}%
\pgfpathlineto{\pgfqpoint{6.840400in}{1.610000in}}%
\pgfpathlineto{\pgfqpoint{6.841640in}{1.505000in}}%
\pgfpathlineto{\pgfqpoint{6.842880in}{1.505000in}}%
\pgfpathlineto{\pgfqpoint{6.844120in}{1.995000in}}%
\pgfpathlineto{\pgfqpoint{6.845360in}{1.575000in}}%
\pgfpathlineto{\pgfqpoint{6.846600in}{1.855000in}}%
\pgfpathlineto{\pgfqpoint{6.847840in}{1.750000in}}%
\pgfpathlineto{\pgfqpoint{6.849080in}{1.400000in}}%
\pgfpathlineto{\pgfqpoint{6.851560in}{1.785000in}}%
\pgfpathlineto{\pgfqpoint{6.852800in}{1.750000in}}%
\pgfpathlineto{\pgfqpoint{6.854040in}{2.100000in}}%
\pgfpathlineto{\pgfqpoint{6.856520in}{1.680000in}}%
\pgfpathlineto{\pgfqpoint{6.857760in}{1.960000in}}%
\pgfpathlineto{\pgfqpoint{6.859000in}{1.645000in}}%
\pgfpathlineto{\pgfqpoint{6.860240in}{1.680000in}}%
\pgfpathlineto{\pgfqpoint{6.861480in}{1.645000in}}%
\pgfpathlineto{\pgfqpoint{6.862720in}{1.505000in}}%
\pgfpathlineto{\pgfqpoint{6.863960in}{1.715000in}}%
\pgfpathlineto{\pgfqpoint{6.865200in}{1.645000in}}%
\pgfpathlineto{\pgfqpoint{6.866440in}{1.750000in}}%
\pgfpathlineto{\pgfqpoint{6.867680in}{1.680000in}}%
\pgfpathlineto{\pgfqpoint{6.868920in}{1.750000in}}%
\pgfpathlineto{\pgfqpoint{6.870160in}{1.400000in}}%
\pgfpathlineto{\pgfqpoint{6.872640in}{1.785000in}}%
\pgfpathlineto{\pgfqpoint{6.873880in}{1.260000in}}%
\pgfpathlineto{\pgfqpoint{6.875120in}{1.505000in}}%
\pgfpathlineto{\pgfqpoint{6.876360in}{1.470000in}}%
\pgfpathlineto{\pgfqpoint{6.878840in}{1.470000in}}%
\pgfpathlineto{\pgfqpoint{6.880080in}{1.435000in}}%
\pgfpathlineto{\pgfqpoint{6.882560in}{1.120000in}}%
\pgfpathlineto{\pgfqpoint{6.883800in}{1.575000in}}%
\pgfpathlineto{\pgfqpoint{6.885040in}{1.225000in}}%
\pgfpathlineto{\pgfqpoint{6.886280in}{1.435000in}}%
\pgfpathlineto{\pgfqpoint{6.887520in}{1.400000in}}%
\pgfpathlineto{\pgfqpoint{6.888760in}{1.715000in}}%
\pgfpathlineto{\pgfqpoint{6.890000in}{1.295000in}}%
\pgfpathlineto{\pgfqpoint{6.891240in}{1.470000in}}%
\pgfpathlineto{\pgfqpoint{6.894960in}{0.770000in}}%
\pgfpathlineto{\pgfqpoint{6.896200in}{1.540000in}}%
\pgfpathlineto{\pgfqpoint{6.898680in}{1.435000in}}%
\pgfpathlineto{\pgfqpoint{6.899920in}{0.980000in}}%
\pgfpathlineto{\pgfqpoint{6.903640in}{1.575000in}}%
\pgfpathlineto{\pgfqpoint{6.904880in}{1.190000in}}%
\pgfpathlineto{\pgfqpoint{6.907360in}{1.995000in}}%
\pgfpathlineto{\pgfqpoint{6.908600in}{1.435000in}}%
\pgfpathlineto{\pgfqpoint{6.909840in}{1.540000in}}%
\pgfpathlineto{\pgfqpoint{6.911080in}{1.155000in}}%
\pgfpathlineto{\pgfqpoint{6.912320in}{1.680000in}}%
\pgfpathlineto{\pgfqpoint{6.913560in}{1.540000in}}%
\pgfpathlineto{\pgfqpoint{6.914800in}{1.890000in}}%
\pgfpathlineto{\pgfqpoint{6.917280in}{1.505000in}}%
\pgfpathlineto{\pgfqpoint{6.918520in}{1.470000in}}%
\pgfpathlineto{\pgfqpoint{6.919760in}{1.610000in}}%
\pgfpathlineto{\pgfqpoint{6.923480in}{1.225000in}}%
\pgfpathlineto{\pgfqpoint{6.924720in}{1.365000in}}%
\pgfpathlineto{\pgfqpoint{6.925960in}{1.365000in}}%
\pgfpathlineto{\pgfqpoint{6.928440in}{1.610000in}}%
\pgfpathlineto{\pgfqpoint{6.929680in}{1.155000in}}%
\pgfpathlineto{\pgfqpoint{6.930920in}{1.680000in}}%
\pgfpathlineto{\pgfqpoint{6.932160in}{1.645000in}}%
\pgfpathlineto{\pgfqpoint{6.933400in}{1.435000in}}%
\pgfpathlineto{\pgfqpoint{6.934640in}{1.470000in}}%
\pgfpathlineto{\pgfqpoint{6.935880in}{1.435000in}}%
\pgfpathlineto{\pgfqpoint{6.937120in}{1.470000in}}%
\pgfpathlineto{\pgfqpoint{6.939600in}{1.890000in}}%
\pgfpathlineto{\pgfqpoint{6.940840in}{1.365000in}}%
\pgfpathlineto{\pgfqpoint{6.942080in}{1.400000in}}%
\pgfpathlineto{\pgfqpoint{6.943320in}{1.330000in}}%
\pgfpathlineto{\pgfqpoint{6.944560in}{1.155000in}}%
\pgfpathlineto{\pgfqpoint{6.945800in}{1.155000in}}%
\pgfpathlineto{\pgfqpoint{6.948280in}{1.925000in}}%
\pgfpathlineto{\pgfqpoint{6.949520in}{1.575000in}}%
\pgfpathlineto{\pgfqpoint{6.952000in}{1.750000in}}%
\pgfpathlineto{\pgfqpoint{6.953240in}{1.645000in}}%
\pgfpathlineto{\pgfqpoint{6.954480in}{1.855000in}}%
\pgfpathlineto{\pgfqpoint{6.955720in}{1.715000in}}%
\pgfpathlineto{\pgfqpoint{6.956960in}{1.435000in}}%
\pgfpathlineto{\pgfqpoint{6.958200in}{1.540000in}}%
\pgfpathlineto{\pgfqpoint{6.959440in}{1.400000in}}%
\pgfpathlineto{\pgfqpoint{6.960680in}{1.470000in}}%
\pgfpathlineto{\pgfqpoint{6.961920in}{1.225000in}}%
\pgfpathlineto{\pgfqpoint{6.963160in}{1.575000in}}%
\pgfpathlineto{\pgfqpoint{6.964400in}{1.470000in}}%
\pgfpathlineto{\pgfqpoint{6.965640in}{1.855000in}}%
\pgfpathlineto{\pgfqpoint{6.969360in}{0.805000in}}%
\pgfpathlineto{\pgfqpoint{6.970600in}{0.980000in}}%
\pgfpathlineto{\pgfqpoint{6.971840in}{1.505000in}}%
\pgfpathlineto{\pgfqpoint{6.974320in}{1.365000in}}%
\pgfpathlineto{\pgfqpoint{6.975560in}{1.540000in}}%
\pgfpathlineto{\pgfqpoint{6.976800in}{1.190000in}}%
\pgfpathlineto{\pgfqpoint{6.978040in}{1.155000in}}%
\pgfpathlineto{\pgfqpoint{6.979280in}{1.295000in}}%
\pgfpathlineto{\pgfqpoint{6.980520in}{2.065000in}}%
\pgfpathlineto{\pgfqpoint{6.983000in}{1.400000in}}%
\pgfpathlineto{\pgfqpoint{6.984240in}{1.715000in}}%
\pgfpathlineto{\pgfqpoint{6.986720in}{1.190000in}}%
\pgfpathlineto{\pgfqpoint{6.987960in}{1.435000in}}%
\pgfpathlineto{\pgfqpoint{6.989200in}{1.225000in}}%
\pgfpathlineto{\pgfqpoint{6.990440in}{1.750000in}}%
\pgfpathlineto{\pgfqpoint{6.994160in}{1.540000in}}%
\pgfpathlineto{\pgfqpoint{6.995400in}{1.365000in}}%
\pgfpathlineto{\pgfqpoint{6.996640in}{1.890000in}}%
\pgfpathlineto{\pgfqpoint{6.997880in}{1.540000in}}%
\pgfpathlineto{\pgfqpoint{6.999120in}{1.855000in}}%
\pgfpathlineto{\pgfqpoint{7.001600in}{1.505000in}}%
\pgfpathlineto{\pgfqpoint{7.002840in}{1.610000in}}%
\pgfpathlineto{\pgfqpoint{7.004080in}{1.925000in}}%
\pgfpathlineto{\pgfqpoint{7.005320in}{1.575000in}}%
\pgfpathlineto{\pgfqpoint{7.006560in}{2.030000in}}%
\pgfpathlineto{\pgfqpoint{7.009040in}{1.575000in}}%
\pgfpathlineto{\pgfqpoint{7.010280in}{1.820000in}}%
\pgfpathlineto{\pgfqpoint{7.011520in}{1.435000in}}%
\pgfpathlineto{\pgfqpoint{7.012760in}{1.680000in}}%
\pgfpathlineto{\pgfqpoint{7.016480in}{1.225000in}}%
\pgfpathlineto{\pgfqpoint{7.017720in}{1.330000in}}%
\pgfpathlineto{\pgfqpoint{7.018960in}{1.120000in}}%
\pgfpathlineto{\pgfqpoint{7.020200in}{1.365000in}}%
\pgfpathlineto{\pgfqpoint{7.021440in}{1.225000in}}%
\pgfpathlineto{\pgfqpoint{7.022680in}{1.295000in}}%
\pgfpathlineto{\pgfqpoint{7.025160in}{1.505000in}}%
\pgfpathlineto{\pgfqpoint{7.026400in}{1.610000in}}%
\pgfpathlineto{\pgfqpoint{7.028880in}{1.295000in}}%
\pgfpathlineto{\pgfqpoint{7.030120in}{1.435000in}}%
\pgfpathlineto{\pgfqpoint{7.031360in}{1.330000in}}%
\pgfpathlineto{\pgfqpoint{7.032600in}{1.400000in}}%
\pgfpathlineto{\pgfqpoint{7.033840in}{1.260000in}}%
\pgfpathlineto{\pgfqpoint{7.035080in}{1.260000in}}%
\pgfpathlineto{\pgfqpoint{7.036320in}{1.925000in}}%
\pgfpathlineto{\pgfqpoint{7.037560in}{1.155000in}}%
\pgfpathlineto{\pgfqpoint{7.038800in}{1.820000in}}%
\pgfpathlineto{\pgfqpoint{7.040040in}{1.575000in}}%
\pgfpathlineto{\pgfqpoint{7.041280in}{1.890000in}}%
\pgfpathlineto{\pgfqpoint{7.045000in}{1.295000in}}%
\pgfpathlineto{\pgfqpoint{7.048720in}{1.680000in}}%
\pgfpathlineto{\pgfqpoint{7.049960in}{1.610000in}}%
\pgfpathlineto{\pgfqpoint{7.051200in}{1.435000in}}%
\pgfpathlineto{\pgfqpoint{7.052440in}{1.610000in}}%
\pgfpathlineto{\pgfqpoint{7.053680in}{1.470000in}}%
\pgfpathlineto{\pgfqpoint{7.054920in}{1.785000in}}%
\pgfpathlineto{\pgfqpoint{7.056160in}{1.295000in}}%
\pgfpathlineto{\pgfqpoint{7.058640in}{1.715000in}}%
\pgfpathlineto{\pgfqpoint{7.059880in}{1.925000in}}%
\pgfpathlineto{\pgfqpoint{7.062360in}{1.365000in}}%
\pgfpathlineto{\pgfqpoint{7.064840in}{1.680000in}}%
\pgfpathlineto{\pgfqpoint{7.066080in}{1.575000in}}%
\pgfpathlineto{\pgfqpoint{7.067320in}{1.330000in}}%
\pgfpathlineto{\pgfqpoint{7.068560in}{1.470000in}}%
\pgfpathlineto{\pgfqpoint{7.069800in}{1.225000in}}%
\pgfpathlineto{\pgfqpoint{7.072280in}{1.505000in}}%
\pgfpathlineto{\pgfqpoint{7.073520in}{2.065000in}}%
\pgfpathlineto{\pgfqpoint{7.076000in}{1.505000in}}%
\pgfpathlineto{\pgfqpoint{7.077240in}{1.505000in}}%
\pgfpathlineto{\pgfqpoint{7.078480in}{1.470000in}}%
\pgfpathlineto{\pgfqpoint{7.079720in}{1.680000in}}%
\pgfpathlineto{\pgfqpoint{7.080960in}{1.260000in}}%
\pgfpathlineto{\pgfqpoint{7.083440in}{1.715000in}}%
\pgfpathlineto{\pgfqpoint{7.085920in}{1.610000in}}%
\pgfpathlineto{\pgfqpoint{7.087160in}{1.260000in}}%
\pgfpathlineto{\pgfqpoint{7.088400in}{1.715000in}}%
\pgfpathlineto{\pgfqpoint{7.090880in}{1.260000in}}%
\pgfpathlineto{\pgfqpoint{7.092120in}{1.610000in}}%
\pgfpathlineto{\pgfqpoint{7.093360in}{1.295000in}}%
\pgfpathlineto{\pgfqpoint{7.094600in}{1.400000in}}%
\pgfpathlineto{\pgfqpoint{7.095840in}{1.680000in}}%
\pgfpathlineto{\pgfqpoint{7.097080in}{1.645000in}}%
\pgfpathlineto{\pgfqpoint{7.098320in}{1.260000in}}%
\pgfpathlineto{\pgfqpoint{7.099560in}{1.470000in}}%
\pgfpathlineto{\pgfqpoint{7.100800in}{1.365000in}}%
\pgfpathlineto{\pgfqpoint{7.102040in}{1.540000in}}%
\pgfpathlineto{\pgfqpoint{7.104520in}{0.980000in}}%
\pgfpathlineto{\pgfqpoint{7.107000in}{1.820000in}}%
\pgfpathlineto{\pgfqpoint{7.108240in}{1.120000in}}%
\pgfpathlineto{\pgfqpoint{7.109480in}{1.435000in}}%
\pgfpathlineto{\pgfqpoint{7.110720in}{1.050000in}}%
\pgfpathlineto{\pgfqpoint{7.111960in}{1.540000in}}%
\pgfpathlineto{\pgfqpoint{7.113200in}{1.330000in}}%
\pgfpathlineto{\pgfqpoint{7.114440in}{1.505000in}}%
\pgfpathlineto{\pgfqpoint{7.115680in}{1.470000in}}%
\pgfpathlineto{\pgfqpoint{7.116920in}{1.610000in}}%
\pgfpathlineto{\pgfqpoint{7.118160in}{1.575000in}}%
\pgfpathlineto{\pgfqpoint{7.119400in}{1.855000in}}%
\pgfpathlineto{\pgfqpoint{7.120640in}{1.715000in}}%
\pgfpathlineto{\pgfqpoint{7.123120in}{1.155000in}}%
\pgfpathlineto{\pgfqpoint{7.124360in}{2.030000in}}%
\pgfpathlineto{\pgfqpoint{7.126840in}{1.295000in}}%
\pgfpathlineto{\pgfqpoint{7.128080in}{1.890000in}}%
\pgfpathlineto{\pgfqpoint{7.130560in}{0.805000in}}%
\pgfpathlineto{\pgfqpoint{7.133040in}{1.330000in}}%
\pgfpathlineto{\pgfqpoint{7.134280in}{1.330000in}}%
\pgfpathlineto{\pgfqpoint{7.135520in}{1.295000in}}%
\pgfpathlineto{\pgfqpoint{7.136760in}{1.610000in}}%
\pgfpathlineto{\pgfqpoint{7.138000in}{1.610000in}}%
\pgfpathlineto{\pgfqpoint{7.139240in}{1.365000in}}%
\pgfpathlineto{\pgfqpoint{7.140480in}{1.750000in}}%
\pgfpathlineto{\pgfqpoint{7.141720in}{1.085000in}}%
\pgfpathlineto{\pgfqpoint{7.142960in}{1.225000in}}%
\pgfpathlineto{\pgfqpoint{7.145440in}{1.015000in}}%
\pgfpathlineto{\pgfqpoint{7.147920in}{1.715000in}}%
\pgfpathlineto{\pgfqpoint{7.149160in}{1.190000in}}%
\pgfpathlineto{\pgfqpoint{7.150400in}{1.225000in}}%
\pgfpathlineto{\pgfqpoint{7.151640in}{1.610000in}}%
\pgfpathlineto{\pgfqpoint{7.152880in}{1.365000in}}%
\pgfpathlineto{\pgfqpoint{7.154120in}{1.505000in}}%
\pgfpathlineto{\pgfqpoint{7.156600in}{1.505000in}}%
\pgfpathlineto{\pgfqpoint{7.157840in}{1.330000in}}%
\pgfpathlineto{\pgfqpoint{7.159080in}{1.400000in}}%
\pgfpathlineto{\pgfqpoint{7.161560in}{1.610000in}}%
\pgfpathlineto{\pgfqpoint{7.162800in}{1.330000in}}%
\pgfpathlineto{\pgfqpoint{7.164040in}{1.470000in}}%
\pgfpathlineto{\pgfqpoint{7.165280in}{0.945000in}}%
\pgfpathlineto{\pgfqpoint{7.166520in}{1.015000in}}%
\pgfpathlineto{\pgfqpoint{7.167760in}{0.805000in}}%
\pgfpathlineto{\pgfqpoint{7.169000in}{1.295000in}}%
\pgfpathlineto{\pgfqpoint{7.170240in}{1.225000in}}%
\pgfpathlineto{\pgfqpoint{7.171480in}{1.085000in}}%
\pgfpathlineto{\pgfqpoint{7.173960in}{1.295000in}}%
\pgfpathlineto{\pgfqpoint{7.175200in}{1.085000in}}%
\pgfpathlineto{\pgfqpoint{7.176440in}{1.155000in}}%
\pgfpathlineto{\pgfqpoint{7.178920in}{1.610000in}}%
\pgfpathlineto{\pgfqpoint{7.180160in}{1.645000in}}%
\pgfpathlineto{\pgfqpoint{7.182640in}{1.400000in}}%
\pgfpathlineto{\pgfqpoint{7.186360in}{1.645000in}}%
\pgfpathlineto{\pgfqpoint{7.187600in}{1.435000in}}%
\pgfpathlineto{\pgfqpoint{7.188840in}{1.470000in}}%
\pgfpathlineto{\pgfqpoint{7.191320in}{1.470000in}}%
\pgfpathlineto{\pgfqpoint{7.193800in}{1.645000in}}%
\pgfpathlineto{\pgfqpoint{7.198760in}{0.945000in}}%
\pgfpathlineto{\pgfqpoint{7.200000in}{1.120000in}}%
\pgfpathlineto{\pgfqpoint{7.200000in}{1.120000in}}%
\pgfusepath{stroke}%
\end{pgfscope}%
\begin{pgfscope}%
\pgfpathrectangle{\pgfqpoint{1.000000in}{0.350000in}}{\pgfqpoint{6.200000in}{2.800000in}} %
\pgfusepath{clip}%
\pgfsetbuttcap%
\pgfsetroundjoin%
\pgfsetlinewidth{0.501875pt}%
\definecolor{currentstroke}{rgb}{0.000000,0.000000,0.000000}%
\pgfsetstrokecolor{currentstroke}%
\pgfsetdash{{1.000000pt}{3.000000pt}}{0.000000pt}%
\pgfpathmoveto{\pgfqpoint{1.000000in}{0.350000in}}%
\pgfpathlineto{\pgfqpoint{1.000000in}{3.150000in}}%
\pgfusepath{stroke}%
\end{pgfscope}%
\begin{pgfscope}%
\pgfsetbuttcap%
\pgfsetroundjoin%
\definecolor{currentfill}{rgb}{0.000000,0.000000,0.000000}%
\pgfsetfillcolor{currentfill}%
\pgfsetlinewidth{0.501875pt}%
\definecolor{currentstroke}{rgb}{0.000000,0.000000,0.000000}%
\pgfsetstrokecolor{currentstroke}%
\pgfsetdash{}{0pt}%
\pgfsys@defobject{currentmarker}{\pgfqpoint{0.000000in}{0.000000in}}{\pgfqpoint{0.000000in}{0.055556in}}{%
\pgfpathmoveto{\pgfqpoint{0.000000in}{0.000000in}}%
\pgfpathlineto{\pgfqpoint{0.000000in}{0.055556in}}%
\pgfusepath{stroke,fill}%
}%
\begin{pgfscope}%
\pgfsys@transformshift{1.000000in}{0.350000in}%
\pgfsys@useobject{currentmarker}{}%
\end{pgfscope}%
\end{pgfscope}%
\begin{pgfscope}%
\pgfsetbuttcap%
\pgfsetroundjoin%
\definecolor{currentfill}{rgb}{0.000000,0.000000,0.000000}%
\pgfsetfillcolor{currentfill}%
\pgfsetlinewidth{0.501875pt}%
\definecolor{currentstroke}{rgb}{0.000000,0.000000,0.000000}%
\pgfsetstrokecolor{currentstroke}%
\pgfsetdash{}{0pt}%
\pgfsys@defobject{currentmarker}{\pgfqpoint{0.000000in}{-0.055556in}}{\pgfqpoint{0.000000in}{0.000000in}}{%
\pgfpathmoveto{\pgfqpoint{0.000000in}{0.000000in}}%
\pgfpathlineto{\pgfqpoint{0.000000in}{-0.055556in}}%
\pgfusepath{stroke,fill}%
}%
\begin{pgfscope}%
\pgfsys@transformshift{1.000000in}{3.150000in}%
\pgfsys@useobject{currentmarker}{}%
\end{pgfscope}%
\end{pgfscope}%
\begin{pgfscope}%
\pgftext[left,bottom,x=0.946981in,y=0.168387in,rotate=0.000000]{{\sffamily\fontsize{12.000000}{14.400000}\selectfont 0}}
%
\end{pgfscope}%
\begin{pgfscope}%
\pgfpathrectangle{\pgfqpoint{1.000000in}{0.350000in}}{\pgfqpoint{6.200000in}{2.800000in}} %
\pgfusepath{clip}%
\pgfsetbuttcap%
\pgfsetroundjoin%
\pgfsetlinewidth{0.501875pt}%
\definecolor{currentstroke}{rgb}{0.000000,0.000000,0.000000}%
\pgfsetstrokecolor{currentstroke}%
\pgfsetdash{{1.000000pt}{3.000000pt}}{0.000000pt}%
\pgfpathmoveto{\pgfqpoint{2.240000in}{0.350000in}}%
\pgfpathlineto{\pgfqpoint{2.240000in}{3.150000in}}%
\pgfusepath{stroke}%
\end{pgfscope}%
\begin{pgfscope}%
\pgfsetbuttcap%
\pgfsetroundjoin%
\definecolor{currentfill}{rgb}{0.000000,0.000000,0.000000}%
\pgfsetfillcolor{currentfill}%
\pgfsetlinewidth{0.501875pt}%
\definecolor{currentstroke}{rgb}{0.000000,0.000000,0.000000}%
\pgfsetstrokecolor{currentstroke}%
\pgfsetdash{}{0pt}%
\pgfsys@defobject{currentmarker}{\pgfqpoint{0.000000in}{0.000000in}}{\pgfqpoint{0.000000in}{0.055556in}}{%
\pgfpathmoveto{\pgfqpoint{0.000000in}{0.000000in}}%
\pgfpathlineto{\pgfqpoint{0.000000in}{0.055556in}}%
\pgfusepath{stroke,fill}%
}%
\begin{pgfscope}%
\pgfsys@transformshift{2.240000in}{0.350000in}%
\pgfsys@useobject{currentmarker}{}%
\end{pgfscope}%
\end{pgfscope}%
\begin{pgfscope}%
\pgfsetbuttcap%
\pgfsetroundjoin%
\definecolor{currentfill}{rgb}{0.000000,0.000000,0.000000}%
\pgfsetfillcolor{currentfill}%
\pgfsetlinewidth{0.501875pt}%
\definecolor{currentstroke}{rgb}{0.000000,0.000000,0.000000}%
\pgfsetstrokecolor{currentstroke}%
\pgfsetdash{}{0pt}%
\pgfsys@defobject{currentmarker}{\pgfqpoint{0.000000in}{-0.055556in}}{\pgfqpoint{0.000000in}{0.000000in}}{%
\pgfpathmoveto{\pgfqpoint{0.000000in}{0.000000in}}%
\pgfpathlineto{\pgfqpoint{0.000000in}{-0.055556in}}%
\pgfusepath{stroke,fill}%
}%
\begin{pgfscope}%
\pgfsys@transformshift{2.240000in}{3.150000in}%
\pgfsys@useobject{currentmarker}{}%
\end{pgfscope}%
\end{pgfscope}%
\begin{pgfscope}%
\pgftext[left,bottom,x=2.080942in,y=0.168387in,rotate=0.000000]{{\sffamily\fontsize{12.000000}{14.400000}\selectfont 100}}
%
\end{pgfscope}%
\begin{pgfscope}%
\pgfpathrectangle{\pgfqpoint{1.000000in}{0.350000in}}{\pgfqpoint{6.200000in}{2.800000in}} %
\pgfusepath{clip}%
\pgfsetbuttcap%
\pgfsetroundjoin%
\pgfsetlinewidth{0.501875pt}%
\definecolor{currentstroke}{rgb}{0.000000,0.000000,0.000000}%
\pgfsetstrokecolor{currentstroke}%
\pgfsetdash{{1.000000pt}{3.000000pt}}{0.000000pt}%
\pgfpathmoveto{\pgfqpoint{3.480000in}{0.350000in}}%
\pgfpathlineto{\pgfqpoint{3.480000in}{3.150000in}}%
\pgfusepath{stroke}%
\end{pgfscope}%
\begin{pgfscope}%
\pgfsetbuttcap%
\pgfsetroundjoin%
\definecolor{currentfill}{rgb}{0.000000,0.000000,0.000000}%
\pgfsetfillcolor{currentfill}%
\pgfsetlinewidth{0.501875pt}%
\definecolor{currentstroke}{rgb}{0.000000,0.000000,0.000000}%
\pgfsetstrokecolor{currentstroke}%
\pgfsetdash{}{0pt}%
\pgfsys@defobject{currentmarker}{\pgfqpoint{0.000000in}{0.000000in}}{\pgfqpoint{0.000000in}{0.055556in}}{%
\pgfpathmoveto{\pgfqpoint{0.000000in}{0.000000in}}%
\pgfpathlineto{\pgfqpoint{0.000000in}{0.055556in}}%
\pgfusepath{stroke,fill}%
}%
\begin{pgfscope}%
\pgfsys@transformshift{3.480000in}{0.350000in}%
\pgfsys@useobject{currentmarker}{}%
\end{pgfscope}%
\end{pgfscope}%
\begin{pgfscope}%
\pgfsetbuttcap%
\pgfsetroundjoin%
\definecolor{currentfill}{rgb}{0.000000,0.000000,0.000000}%
\pgfsetfillcolor{currentfill}%
\pgfsetlinewidth{0.501875pt}%
\definecolor{currentstroke}{rgb}{0.000000,0.000000,0.000000}%
\pgfsetstrokecolor{currentstroke}%
\pgfsetdash{}{0pt}%
\pgfsys@defobject{currentmarker}{\pgfqpoint{0.000000in}{-0.055556in}}{\pgfqpoint{0.000000in}{0.000000in}}{%
\pgfpathmoveto{\pgfqpoint{0.000000in}{0.000000in}}%
\pgfpathlineto{\pgfqpoint{0.000000in}{-0.055556in}}%
\pgfusepath{stroke,fill}%
}%
\begin{pgfscope}%
\pgfsys@transformshift{3.480000in}{3.150000in}%
\pgfsys@useobject{currentmarker}{}%
\end{pgfscope}%
\end{pgfscope}%
\begin{pgfscope}%
\pgftext[left,bottom,x=3.320942in,y=0.168387in,rotate=0.000000]{{\sffamily\fontsize{12.000000}{14.400000}\selectfont 200}}
%
\end{pgfscope}%
\begin{pgfscope}%
\pgfpathrectangle{\pgfqpoint{1.000000in}{0.350000in}}{\pgfqpoint{6.200000in}{2.800000in}} %
\pgfusepath{clip}%
\pgfsetbuttcap%
\pgfsetroundjoin%
\pgfsetlinewidth{0.501875pt}%
\definecolor{currentstroke}{rgb}{0.000000,0.000000,0.000000}%
\pgfsetstrokecolor{currentstroke}%
\pgfsetdash{{1.000000pt}{3.000000pt}}{0.000000pt}%
\pgfpathmoveto{\pgfqpoint{4.720000in}{0.350000in}}%
\pgfpathlineto{\pgfqpoint{4.720000in}{3.150000in}}%
\pgfusepath{stroke}%
\end{pgfscope}%
\begin{pgfscope}%
\pgfsetbuttcap%
\pgfsetroundjoin%
\definecolor{currentfill}{rgb}{0.000000,0.000000,0.000000}%
\pgfsetfillcolor{currentfill}%
\pgfsetlinewidth{0.501875pt}%
\definecolor{currentstroke}{rgb}{0.000000,0.000000,0.000000}%
\pgfsetstrokecolor{currentstroke}%
\pgfsetdash{}{0pt}%
\pgfsys@defobject{currentmarker}{\pgfqpoint{0.000000in}{0.000000in}}{\pgfqpoint{0.000000in}{0.055556in}}{%
\pgfpathmoveto{\pgfqpoint{0.000000in}{0.000000in}}%
\pgfpathlineto{\pgfqpoint{0.000000in}{0.055556in}}%
\pgfusepath{stroke,fill}%
}%
\begin{pgfscope}%
\pgfsys@transformshift{4.720000in}{0.350000in}%
\pgfsys@useobject{currentmarker}{}%
\end{pgfscope}%
\end{pgfscope}%
\begin{pgfscope}%
\pgfsetbuttcap%
\pgfsetroundjoin%
\definecolor{currentfill}{rgb}{0.000000,0.000000,0.000000}%
\pgfsetfillcolor{currentfill}%
\pgfsetlinewidth{0.501875pt}%
\definecolor{currentstroke}{rgb}{0.000000,0.000000,0.000000}%
\pgfsetstrokecolor{currentstroke}%
\pgfsetdash{}{0pt}%
\pgfsys@defobject{currentmarker}{\pgfqpoint{0.000000in}{-0.055556in}}{\pgfqpoint{0.000000in}{0.000000in}}{%
\pgfpathmoveto{\pgfqpoint{0.000000in}{0.000000in}}%
\pgfpathlineto{\pgfqpoint{0.000000in}{-0.055556in}}%
\pgfusepath{stroke,fill}%
}%
\begin{pgfscope}%
\pgfsys@transformshift{4.720000in}{3.150000in}%
\pgfsys@useobject{currentmarker}{}%
\end{pgfscope}%
\end{pgfscope}%
\begin{pgfscope}%
\pgftext[left,bottom,x=4.560942in,y=0.168387in,rotate=0.000000]{{\sffamily\fontsize{12.000000}{14.400000}\selectfont 300}}
%
\end{pgfscope}%
\begin{pgfscope}%
\pgfpathrectangle{\pgfqpoint{1.000000in}{0.350000in}}{\pgfqpoint{6.200000in}{2.800000in}} %
\pgfusepath{clip}%
\pgfsetbuttcap%
\pgfsetroundjoin%
\pgfsetlinewidth{0.501875pt}%
\definecolor{currentstroke}{rgb}{0.000000,0.000000,0.000000}%
\pgfsetstrokecolor{currentstroke}%
\pgfsetdash{{1.000000pt}{3.000000pt}}{0.000000pt}%
\pgfpathmoveto{\pgfqpoint{5.960000in}{0.350000in}}%
\pgfpathlineto{\pgfqpoint{5.960000in}{3.150000in}}%
\pgfusepath{stroke}%
\end{pgfscope}%
\begin{pgfscope}%
\pgfsetbuttcap%
\pgfsetroundjoin%
\definecolor{currentfill}{rgb}{0.000000,0.000000,0.000000}%
\pgfsetfillcolor{currentfill}%
\pgfsetlinewidth{0.501875pt}%
\definecolor{currentstroke}{rgb}{0.000000,0.000000,0.000000}%
\pgfsetstrokecolor{currentstroke}%
\pgfsetdash{}{0pt}%
\pgfsys@defobject{currentmarker}{\pgfqpoint{0.000000in}{0.000000in}}{\pgfqpoint{0.000000in}{0.055556in}}{%
\pgfpathmoveto{\pgfqpoint{0.000000in}{0.000000in}}%
\pgfpathlineto{\pgfqpoint{0.000000in}{0.055556in}}%
\pgfusepath{stroke,fill}%
}%
\begin{pgfscope}%
\pgfsys@transformshift{5.960000in}{0.350000in}%
\pgfsys@useobject{currentmarker}{}%
\end{pgfscope}%
\end{pgfscope}%
\begin{pgfscope}%
\pgfsetbuttcap%
\pgfsetroundjoin%
\definecolor{currentfill}{rgb}{0.000000,0.000000,0.000000}%
\pgfsetfillcolor{currentfill}%
\pgfsetlinewidth{0.501875pt}%
\definecolor{currentstroke}{rgb}{0.000000,0.000000,0.000000}%
\pgfsetstrokecolor{currentstroke}%
\pgfsetdash{}{0pt}%
\pgfsys@defobject{currentmarker}{\pgfqpoint{0.000000in}{-0.055556in}}{\pgfqpoint{0.000000in}{0.000000in}}{%
\pgfpathmoveto{\pgfqpoint{0.000000in}{0.000000in}}%
\pgfpathlineto{\pgfqpoint{0.000000in}{-0.055556in}}%
\pgfusepath{stroke,fill}%
}%
\begin{pgfscope}%
\pgfsys@transformshift{5.960000in}{3.150000in}%
\pgfsys@useobject{currentmarker}{}%
\end{pgfscope}%
\end{pgfscope}%
\begin{pgfscope}%
\pgftext[left,bottom,x=5.800942in,y=0.168387in,rotate=0.000000]{{\sffamily\fontsize{12.000000}{14.400000}\selectfont 400}}
%
\end{pgfscope}%
\begin{pgfscope}%
\pgfpathrectangle{\pgfqpoint{1.000000in}{0.350000in}}{\pgfqpoint{6.200000in}{2.800000in}} %
\pgfusepath{clip}%
\pgfsetbuttcap%
\pgfsetroundjoin%
\pgfsetlinewidth{0.501875pt}%
\definecolor{currentstroke}{rgb}{0.000000,0.000000,0.000000}%
\pgfsetstrokecolor{currentstroke}%
\pgfsetdash{{1.000000pt}{3.000000pt}}{0.000000pt}%
\pgfpathmoveto{\pgfqpoint{7.200000in}{0.350000in}}%
\pgfpathlineto{\pgfqpoint{7.200000in}{3.150000in}}%
\pgfusepath{stroke}%
\end{pgfscope}%
\begin{pgfscope}%
\pgfsetbuttcap%
\pgfsetroundjoin%
\definecolor{currentfill}{rgb}{0.000000,0.000000,0.000000}%
\pgfsetfillcolor{currentfill}%
\pgfsetlinewidth{0.501875pt}%
\definecolor{currentstroke}{rgb}{0.000000,0.000000,0.000000}%
\pgfsetstrokecolor{currentstroke}%
\pgfsetdash{}{0pt}%
\pgfsys@defobject{currentmarker}{\pgfqpoint{0.000000in}{0.000000in}}{\pgfqpoint{0.000000in}{0.055556in}}{%
\pgfpathmoveto{\pgfqpoint{0.000000in}{0.000000in}}%
\pgfpathlineto{\pgfqpoint{0.000000in}{0.055556in}}%
\pgfusepath{stroke,fill}%
}%
\begin{pgfscope}%
\pgfsys@transformshift{7.200000in}{0.350000in}%
\pgfsys@useobject{currentmarker}{}%
\end{pgfscope}%
\end{pgfscope}%
\begin{pgfscope}%
\pgfsetbuttcap%
\pgfsetroundjoin%
\definecolor{currentfill}{rgb}{0.000000,0.000000,0.000000}%
\pgfsetfillcolor{currentfill}%
\pgfsetlinewidth{0.501875pt}%
\definecolor{currentstroke}{rgb}{0.000000,0.000000,0.000000}%
\pgfsetstrokecolor{currentstroke}%
\pgfsetdash{}{0pt}%
\pgfsys@defobject{currentmarker}{\pgfqpoint{0.000000in}{-0.055556in}}{\pgfqpoint{0.000000in}{0.000000in}}{%
\pgfpathmoveto{\pgfqpoint{0.000000in}{0.000000in}}%
\pgfpathlineto{\pgfqpoint{0.000000in}{-0.055556in}}%
\pgfusepath{stroke,fill}%
}%
\begin{pgfscope}%
\pgfsys@transformshift{7.200000in}{3.150000in}%
\pgfsys@useobject{currentmarker}{}%
\end{pgfscope}%
\end{pgfscope}%
\begin{pgfscope}%
\pgftext[left,bottom,x=7.040942in,y=0.168387in,rotate=0.000000]{{\sffamily\fontsize{12.000000}{14.400000}\selectfont 500}}
%
\end{pgfscope}%
\begin{pgfscope}%
\pgftext[left,bottom,x=3.911727in,y=-0.030045in,rotate=0.000000]{{\sffamily\fontsize{12.000000}{14.400000}\selectfont time}}
%
\end{pgfscope}%
\begin{pgfscope}%
\pgfpathrectangle{\pgfqpoint{1.000000in}{0.350000in}}{\pgfqpoint{6.200000in}{2.800000in}} %
\pgfusepath{clip}%
\pgfsetbuttcap%
\pgfsetroundjoin%
\pgfsetlinewidth{0.501875pt}%
\definecolor{currentstroke}{rgb}{0.000000,0.000000,0.000000}%
\pgfsetstrokecolor{currentstroke}%
\pgfsetdash{{1.000000pt}{3.000000pt}}{0.000000pt}%
\pgfpathmoveto{\pgfqpoint{1.000000in}{0.350000in}}%
\pgfpathlineto{\pgfqpoint{7.200000in}{0.350000in}}%
\pgfusepath{stroke}%
\end{pgfscope}%
\begin{pgfscope}%
\pgfsetbuttcap%
\pgfsetroundjoin%
\definecolor{currentfill}{rgb}{0.000000,0.000000,0.000000}%
\pgfsetfillcolor{currentfill}%
\pgfsetlinewidth{0.501875pt}%
\definecolor{currentstroke}{rgb}{0.000000,0.000000,0.000000}%
\pgfsetstrokecolor{currentstroke}%
\pgfsetdash{}{0pt}%
\pgfsys@defobject{currentmarker}{\pgfqpoint{0.000000in}{0.000000in}}{\pgfqpoint{0.055556in}{0.000000in}}{%
\pgfpathmoveto{\pgfqpoint{0.000000in}{0.000000in}}%
\pgfpathlineto{\pgfqpoint{0.055556in}{0.000000in}}%
\pgfusepath{stroke,fill}%
}%
\begin{pgfscope}%
\pgfsys@transformshift{1.000000in}{0.350000in}%
\pgfsys@useobject{currentmarker}{}%
\end{pgfscope}%
\end{pgfscope}%
\begin{pgfscope}%
\pgfsetbuttcap%
\pgfsetroundjoin%
\definecolor{currentfill}{rgb}{0.000000,0.000000,0.000000}%
\pgfsetfillcolor{currentfill}%
\pgfsetlinewidth{0.501875pt}%
\definecolor{currentstroke}{rgb}{0.000000,0.000000,0.000000}%
\pgfsetstrokecolor{currentstroke}%
\pgfsetdash{}{0pt}%
\pgfsys@defobject{currentmarker}{\pgfqpoint{-0.055556in}{0.000000in}}{\pgfqpoint{0.000000in}{0.000000in}}{%
\pgfpathmoveto{\pgfqpoint{0.000000in}{0.000000in}}%
\pgfpathlineto{\pgfqpoint{-0.055556in}{0.000000in}}%
\pgfusepath{stroke,fill}%
}%
\begin{pgfscope}%
\pgfsys@transformshift{7.200000in}{0.350000in}%
\pgfsys@useobject{currentmarker}{}%
\end{pgfscope}%
\end{pgfscope}%
\begin{pgfscope}%
\pgftext[left,bottom,x=0.626329in,y=0.286971in,rotate=0.000000]{{\sffamily\fontsize{12.000000}{14.400000}\selectfont 330}}
%
\end{pgfscope}%
\begin{pgfscope}%
\pgfpathrectangle{\pgfqpoint{1.000000in}{0.350000in}}{\pgfqpoint{6.200000in}{2.800000in}} %
\pgfusepath{clip}%
\pgfsetbuttcap%
\pgfsetroundjoin%
\pgfsetlinewidth{0.501875pt}%
\definecolor{currentstroke}{rgb}{0.000000,0.000000,0.000000}%
\pgfsetstrokecolor{currentstroke}%
\pgfsetdash{{1.000000pt}{3.000000pt}}{0.000000pt}%
\pgfpathmoveto{\pgfqpoint{1.000000in}{0.700000in}}%
\pgfpathlineto{\pgfqpoint{7.200000in}{0.700000in}}%
\pgfusepath{stroke}%
\end{pgfscope}%
\begin{pgfscope}%
\pgfsetbuttcap%
\pgfsetroundjoin%
\definecolor{currentfill}{rgb}{0.000000,0.000000,0.000000}%
\pgfsetfillcolor{currentfill}%
\pgfsetlinewidth{0.501875pt}%
\definecolor{currentstroke}{rgb}{0.000000,0.000000,0.000000}%
\pgfsetstrokecolor{currentstroke}%
\pgfsetdash{}{0pt}%
\pgfsys@defobject{currentmarker}{\pgfqpoint{0.000000in}{0.000000in}}{\pgfqpoint{0.055556in}{0.000000in}}{%
\pgfpathmoveto{\pgfqpoint{0.000000in}{0.000000in}}%
\pgfpathlineto{\pgfqpoint{0.055556in}{0.000000in}}%
\pgfusepath{stroke,fill}%
}%
\begin{pgfscope}%
\pgfsys@transformshift{1.000000in}{0.700000in}%
\pgfsys@useobject{currentmarker}{}%
\end{pgfscope}%
\end{pgfscope}%
\begin{pgfscope}%
\pgfsetbuttcap%
\pgfsetroundjoin%
\definecolor{currentfill}{rgb}{0.000000,0.000000,0.000000}%
\pgfsetfillcolor{currentfill}%
\pgfsetlinewidth{0.501875pt}%
\definecolor{currentstroke}{rgb}{0.000000,0.000000,0.000000}%
\pgfsetstrokecolor{currentstroke}%
\pgfsetdash{}{0pt}%
\pgfsys@defobject{currentmarker}{\pgfqpoint{-0.055556in}{0.000000in}}{\pgfqpoint{0.000000in}{0.000000in}}{%
\pgfpathmoveto{\pgfqpoint{0.000000in}{0.000000in}}%
\pgfpathlineto{\pgfqpoint{-0.055556in}{0.000000in}}%
\pgfusepath{stroke,fill}%
}%
\begin{pgfscope}%
\pgfsys@transformshift{7.200000in}{0.700000in}%
\pgfsys@useobject{currentmarker}{}%
\end{pgfscope}%
\end{pgfscope}%
\begin{pgfscope}%
\pgftext[left,bottom,x=0.626329in,y=0.636971in,rotate=0.000000]{{\sffamily\fontsize{12.000000}{14.400000}\selectfont 340}}
%
\end{pgfscope}%
\begin{pgfscope}%
\pgfpathrectangle{\pgfqpoint{1.000000in}{0.350000in}}{\pgfqpoint{6.200000in}{2.800000in}} %
\pgfusepath{clip}%
\pgfsetbuttcap%
\pgfsetroundjoin%
\pgfsetlinewidth{0.501875pt}%
\definecolor{currentstroke}{rgb}{0.000000,0.000000,0.000000}%
\pgfsetstrokecolor{currentstroke}%
\pgfsetdash{{1.000000pt}{3.000000pt}}{0.000000pt}%
\pgfpathmoveto{\pgfqpoint{1.000000in}{1.050000in}}%
\pgfpathlineto{\pgfqpoint{7.200000in}{1.050000in}}%
\pgfusepath{stroke}%
\end{pgfscope}%
\begin{pgfscope}%
\pgfsetbuttcap%
\pgfsetroundjoin%
\definecolor{currentfill}{rgb}{0.000000,0.000000,0.000000}%
\pgfsetfillcolor{currentfill}%
\pgfsetlinewidth{0.501875pt}%
\definecolor{currentstroke}{rgb}{0.000000,0.000000,0.000000}%
\pgfsetstrokecolor{currentstroke}%
\pgfsetdash{}{0pt}%
\pgfsys@defobject{currentmarker}{\pgfqpoint{0.000000in}{0.000000in}}{\pgfqpoint{0.055556in}{0.000000in}}{%
\pgfpathmoveto{\pgfqpoint{0.000000in}{0.000000in}}%
\pgfpathlineto{\pgfqpoint{0.055556in}{0.000000in}}%
\pgfusepath{stroke,fill}%
}%
\begin{pgfscope}%
\pgfsys@transformshift{1.000000in}{1.050000in}%
\pgfsys@useobject{currentmarker}{}%
\end{pgfscope}%
\end{pgfscope}%
\begin{pgfscope}%
\pgfsetbuttcap%
\pgfsetroundjoin%
\definecolor{currentfill}{rgb}{0.000000,0.000000,0.000000}%
\pgfsetfillcolor{currentfill}%
\pgfsetlinewidth{0.501875pt}%
\definecolor{currentstroke}{rgb}{0.000000,0.000000,0.000000}%
\pgfsetstrokecolor{currentstroke}%
\pgfsetdash{}{0pt}%
\pgfsys@defobject{currentmarker}{\pgfqpoint{-0.055556in}{0.000000in}}{\pgfqpoint{0.000000in}{0.000000in}}{%
\pgfpathmoveto{\pgfqpoint{0.000000in}{0.000000in}}%
\pgfpathlineto{\pgfqpoint{-0.055556in}{0.000000in}}%
\pgfusepath{stroke,fill}%
}%
\begin{pgfscope}%
\pgfsys@transformshift{7.200000in}{1.050000in}%
\pgfsys@useobject{currentmarker}{}%
\end{pgfscope}%
\end{pgfscope}%
\begin{pgfscope}%
\pgftext[left,bottom,x=0.626329in,y=0.986971in,rotate=0.000000]{{\sffamily\fontsize{12.000000}{14.400000}\selectfont 350}}
%
\end{pgfscope}%
\begin{pgfscope}%
\pgfpathrectangle{\pgfqpoint{1.000000in}{0.350000in}}{\pgfqpoint{6.200000in}{2.800000in}} %
\pgfusepath{clip}%
\pgfsetbuttcap%
\pgfsetroundjoin%
\pgfsetlinewidth{0.501875pt}%
\definecolor{currentstroke}{rgb}{0.000000,0.000000,0.000000}%
\pgfsetstrokecolor{currentstroke}%
\pgfsetdash{{1.000000pt}{3.000000pt}}{0.000000pt}%
\pgfpathmoveto{\pgfqpoint{1.000000in}{1.400000in}}%
\pgfpathlineto{\pgfqpoint{7.200000in}{1.400000in}}%
\pgfusepath{stroke}%
\end{pgfscope}%
\begin{pgfscope}%
\pgfsetbuttcap%
\pgfsetroundjoin%
\definecolor{currentfill}{rgb}{0.000000,0.000000,0.000000}%
\pgfsetfillcolor{currentfill}%
\pgfsetlinewidth{0.501875pt}%
\definecolor{currentstroke}{rgb}{0.000000,0.000000,0.000000}%
\pgfsetstrokecolor{currentstroke}%
\pgfsetdash{}{0pt}%
\pgfsys@defobject{currentmarker}{\pgfqpoint{0.000000in}{0.000000in}}{\pgfqpoint{0.055556in}{0.000000in}}{%
\pgfpathmoveto{\pgfqpoint{0.000000in}{0.000000in}}%
\pgfpathlineto{\pgfqpoint{0.055556in}{0.000000in}}%
\pgfusepath{stroke,fill}%
}%
\begin{pgfscope}%
\pgfsys@transformshift{1.000000in}{1.400000in}%
\pgfsys@useobject{currentmarker}{}%
\end{pgfscope}%
\end{pgfscope}%
\begin{pgfscope}%
\pgfsetbuttcap%
\pgfsetroundjoin%
\definecolor{currentfill}{rgb}{0.000000,0.000000,0.000000}%
\pgfsetfillcolor{currentfill}%
\pgfsetlinewidth{0.501875pt}%
\definecolor{currentstroke}{rgb}{0.000000,0.000000,0.000000}%
\pgfsetstrokecolor{currentstroke}%
\pgfsetdash{}{0pt}%
\pgfsys@defobject{currentmarker}{\pgfqpoint{-0.055556in}{0.000000in}}{\pgfqpoint{0.000000in}{0.000000in}}{%
\pgfpathmoveto{\pgfqpoint{0.000000in}{0.000000in}}%
\pgfpathlineto{\pgfqpoint{-0.055556in}{0.000000in}}%
\pgfusepath{stroke,fill}%
}%
\begin{pgfscope}%
\pgfsys@transformshift{7.200000in}{1.400000in}%
\pgfsys@useobject{currentmarker}{}%
\end{pgfscope}%
\end{pgfscope}%
\begin{pgfscope}%
\pgftext[left,bottom,x=0.626329in,y=1.336971in,rotate=0.000000]{{\sffamily\fontsize{12.000000}{14.400000}\selectfont 360}}
%
\end{pgfscope}%
\begin{pgfscope}%
\pgfpathrectangle{\pgfqpoint{1.000000in}{0.350000in}}{\pgfqpoint{6.200000in}{2.800000in}} %
\pgfusepath{clip}%
\pgfsetbuttcap%
\pgfsetroundjoin%
\pgfsetlinewidth{0.501875pt}%
\definecolor{currentstroke}{rgb}{0.000000,0.000000,0.000000}%
\pgfsetstrokecolor{currentstroke}%
\pgfsetdash{{1.000000pt}{3.000000pt}}{0.000000pt}%
\pgfpathmoveto{\pgfqpoint{1.000000in}{1.750000in}}%
\pgfpathlineto{\pgfqpoint{7.200000in}{1.750000in}}%
\pgfusepath{stroke}%
\end{pgfscope}%
\begin{pgfscope}%
\pgfsetbuttcap%
\pgfsetroundjoin%
\definecolor{currentfill}{rgb}{0.000000,0.000000,0.000000}%
\pgfsetfillcolor{currentfill}%
\pgfsetlinewidth{0.501875pt}%
\definecolor{currentstroke}{rgb}{0.000000,0.000000,0.000000}%
\pgfsetstrokecolor{currentstroke}%
\pgfsetdash{}{0pt}%
\pgfsys@defobject{currentmarker}{\pgfqpoint{0.000000in}{0.000000in}}{\pgfqpoint{0.055556in}{0.000000in}}{%
\pgfpathmoveto{\pgfqpoint{0.000000in}{0.000000in}}%
\pgfpathlineto{\pgfqpoint{0.055556in}{0.000000in}}%
\pgfusepath{stroke,fill}%
}%
\begin{pgfscope}%
\pgfsys@transformshift{1.000000in}{1.750000in}%
\pgfsys@useobject{currentmarker}{}%
\end{pgfscope}%
\end{pgfscope}%
\begin{pgfscope}%
\pgfsetbuttcap%
\pgfsetroundjoin%
\definecolor{currentfill}{rgb}{0.000000,0.000000,0.000000}%
\pgfsetfillcolor{currentfill}%
\pgfsetlinewidth{0.501875pt}%
\definecolor{currentstroke}{rgb}{0.000000,0.000000,0.000000}%
\pgfsetstrokecolor{currentstroke}%
\pgfsetdash{}{0pt}%
\pgfsys@defobject{currentmarker}{\pgfqpoint{-0.055556in}{0.000000in}}{\pgfqpoint{0.000000in}{0.000000in}}{%
\pgfpathmoveto{\pgfqpoint{0.000000in}{0.000000in}}%
\pgfpathlineto{\pgfqpoint{-0.055556in}{0.000000in}}%
\pgfusepath{stroke,fill}%
}%
\begin{pgfscope}%
\pgfsys@transformshift{7.200000in}{1.750000in}%
\pgfsys@useobject{currentmarker}{}%
\end{pgfscope}%
\end{pgfscope}%
\begin{pgfscope}%
\pgftext[left,bottom,x=0.626329in,y=1.686971in,rotate=0.000000]{{\sffamily\fontsize{12.000000}{14.400000}\selectfont 370}}
%
\end{pgfscope}%
\begin{pgfscope}%
\pgfpathrectangle{\pgfqpoint{1.000000in}{0.350000in}}{\pgfqpoint{6.200000in}{2.800000in}} %
\pgfusepath{clip}%
\pgfsetbuttcap%
\pgfsetroundjoin%
\pgfsetlinewidth{0.501875pt}%
\definecolor{currentstroke}{rgb}{0.000000,0.000000,0.000000}%
\pgfsetstrokecolor{currentstroke}%
\pgfsetdash{{1.000000pt}{3.000000pt}}{0.000000pt}%
\pgfpathmoveto{\pgfqpoint{1.000000in}{2.100000in}}%
\pgfpathlineto{\pgfqpoint{7.200000in}{2.100000in}}%
\pgfusepath{stroke}%
\end{pgfscope}%
\begin{pgfscope}%
\pgfsetbuttcap%
\pgfsetroundjoin%
\definecolor{currentfill}{rgb}{0.000000,0.000000,0.000000}%
\pgfsetfillcolor{currentfill}%
\pgfsetlinewidth{0.501875pt}%
\definecolor{currentstroke}{rgb}{0.000000,0.000000,0.000000}%
\pgfsetstrokecolor{currentstroke}%
\pgfsetdash{}{0pt}%
\pgfsys@defobject{currentmarker}{\pgfqpoint{0.000000in}{0.000000in}}{\pgfqpoint{0.055556in}{0.000000in}}{%
\pgfpathmoveto{\pgfqpoint{0.000000in}{0.000000in}}%
\pgfpathlineto{\pgfqpoint{0.055556in}{0.000000in}}%
\pgfusepath{stroke,fill}%
}%
\begin{pgfscope}%
\pgfsys@transformshift{1.000000in}{2.100000in}%
\pgfsys@useobject{currentmarker}{}%
\end{pgfscope}%
\end{pgfscope}%
\begin{pgfscope}%
\pgfsetbuttcap%
\pgfsetroundjoin%
\definecolor{currentfill}{rgb}{0.000000,0.000000,0.000000}%
\pgfsetfillcolor{currentfill}%
\pgfsetlinewidth{0.501875pt}%
\definecolor{currentstroke}{rgb}{0.000000,0.000000,0.000000}%
\pgfsetstrokecolor{currentstroke}%
\pgfsetdash{}{0pt}%
\pgfsys@defobject{currentmarker}{\pgfqpoint{-0.055556in}{0.000000in}}{\pgfqpoint{0.000000in}{0.000000in}}{%
\pgfpathmoveto{\pgfqpoint{0.000000in}{0.000000in}}%
\pgfpathlineto{\pgfqpoint{-0.055556in}{0.000000in}}%
\pgfusepath{stroke,fill}%
}%
\begin{pgfscope}%
\pgfsys@transformshift{7.200000in}{2.100000in}%
\pgfsys@useobject{currentmarker}{}%
\end{pgfscope}%
\end{pgfscope}%
\begin{pgfscope}%
\pgftext[left,bottom,x=0.626329in,y=2.036971in,rotate=0.000000]{{\sffamily\fontsize{12.000000}{14.400000}\selectfont 380}}
%
\end{pgfscope}%
\begin{pgfscope}%
\pgfpathrectangle{\pgfqpoint{1.000000in}{0.350000in}}{\pgfqpoint{6.200000in}{2.800000in}} %
\pgfusepath{clip}%
\pgfsetbuttcap%
\pgfsetroundjoin%
\pgfsetlinewidth{0.501875pt}%
\definecolor{currentstroke}{rgb}{0.000000,0.000000,0.000000}%
\pgfsetstrokecolor{currentstroke}%
\pgfsetdash{{1.000000pt}{3.000000pt}}{0.000000pt}%
\pgfpathmoveto{\pgfqpoint{1.000000in}{2.450000in}}%
\pgfpathlineto{\pgfqpoint{7.200000in}{2.450000in}}%
\pgfusepath{stroke}%
\end{pgfscope}%
\begin{pgfscope}%
\pgfsetbuttcap%
\pgfsetroundjoin%
\definecolor{currentfill}{rgb}{0.000000,0.000000,0.000000}%
\pgfsetfillcolor{currentfill}%
\pgfsetlinewidth{0.501875pt}%
\definecolor{currentstroke}{rgb}{0.000000,0.000000,0.000000}%
\pgfsetstrokecolor{currentstroke}%
\pgfsetdash{}{0pt}%
\pgfsys@defobject{currentmarker}{\pgfqpoint{0.000000in}{0.000000in}}{\pgfqpoint{0.055556in}{0.000000in}}{%
\pgfpathmoveto{\pgfqpoint{0.000000in}{0.000000in}}%
\pgfpathlineto{\pgfqpoint{0.055556in}{0.000000in}}%
\pgfusepath{stroke,fill}%
}%
\begin{pgfscope}%
\pgfsys@transformshift{1.000000in}{2.450000in}%
\pgfsys@useobject{currentmarker}{}%
\end{pgfscope}%
\end{pgfscope}%
\begin{pgfscope}%
\pgfsetbuttcap%
\pgfsetroundjoin%
\definecolor{currentfill}{rgb}{0.000000,0.000000,0.000000}%
\pgfsetfillcolor{currentfill}%
\pgfsetlinewidth{0.501875pt}%
\definecolor{currentstroke}{rgb}{0.000000,0.000000,0.000000}%
\pgfsetstrokecolor{currentstroke}%
\pgfsetdash{}{0pt}%
\pgfsys@defobject{currentmarker}{\pgfqpoint{-0.055556in}{0.000000in}}{\pgfqpoint{0.000000in}{0.000000in}}{%
\pgfpathmoveto{\pgfqpoint{0.000000in}{0.000000in}}%
\pgfpathlineto{\pgfqpoint{-0.055556in}{0.000000in}}%
\pgfusepath{stroke,fill}%
}%
\begin{pgfscope}%
\pgfsys@transformshift{7.200000in}{2.450000in}%
\pgfsys@useobject{currentmarker}{}%
\end{pgfscope}%
\end{pgfscope}%
\begin{pgfscope}%
\pgftext[left,bottom,x=0.626329in,y=2.386971in,rotate=0.000000]{{\sffamily\fontsize{12.000000}{14.400000}\selectfont 390}}
%
\end{pgfscope}%
\begin{pgfscope}%
\pgfpathrectangle{\pgfqpoint{1.000000in}{0.350000in}}{\pgfqpoint{6.200000in}{2.800000in}} %
\pgfusepath{clip}%
\pgfsetbuttcap%
\pgfsetroundjoin%
\pgfsetlinewidth{0.501875pt}%
\definecolor{currentstroke}{rgb}{0.000000,0.000000,0.000000}%
\pgfsetstrokecolor{currentstroke}%
\pgfsetdash{{1.000000pt}{3.000000pt}}{0.000000pt}%
\pgfpathmoveto{\pgfqpoint{1.000000in}{2.800000in}}%
\pgfpathlineto{\pgfqpoint{7.200000in}{2.800000in}}%
\pgfusepath{stroke}%
\end{pgfscope}%
\begin{pgfscope}%
\pgfsetbuttcap%
\pgfsetroundjoin%
\definecolor{currentfill}{rgb}{0.000000,0.000000,0.000000}%
\pgfsetfillcolor{currentfill}%
\pgfsetlinewidth{0.501875pt}%
\definecolor{currentstroke}{rgb}{0.000000,0.000000,0.000000}%
\pgfsetstrokecolor{currentstroke}%
\pgfsetdash{}{0pt}%
\pgfsys@defobject{currentmarker}{\pgfqpoint{0.000000in}{0.000000in}}{\pgfqpoint{0.055556in}{0.000000in}}{%
\pgfpathmoveto{\pgfqpoint{0.000000in}{0.000000in}}%
\pgfpathlineto{\pgfqpoint{0.055556in}{0.000000in}}%
\pgfusepath{stroke,fill}%
}%
\begin{pgfscope}%
\pgfsys@transformshift{1.000000in}{2.800000in}%
\pgfsys@useobject{currentmarker}{}%
\end{pgfscope}%
\end{pgfscope}%
\begin{pgfscope}%
\pgfsetbuttcap%
\pgfsetroundjoin%
\definecolor{currentfill}{rgb}{0.000000,0.000000,0.000000}%
\pgfsetfillcolor{currentfill}%
\pgfsetlinewidth{0.501875pt}%
\definecolor{currentstroke}{rgb}{0.000000,0.000000,0.000000}%
\pgfsetstrokecolor{currentstroke}%
\pgfsetdash{}{0pt}%
\pgfsys@defobject{currentmarker}{\pgfqpoint{-0.055556in}{0.000000in}}{\pgfqpoint{0.000000in}{0.000000in}}{%
\pgfpathmoveto{\pgfqpoint{0.000000in}{0.000000in}}%
\pgfpathlineto{\pgfqpoint{-0.055556in}{0.000000in}}%
\pgfusepath{stroke,fill}%
}%
\begin{pgfscope}%
\pgfsys@transformshift{7.200000in}{2.800000in}%
\pgfsys@useobject{currentmarker}{}%
\end{pgfscope}%
\end{pgfscope}%
\begin{pgfscope}%
\pgftext[left,bottom,x=0.626329in,y=2.736971in,rotate=0.000000]{{\sffamily\fontsize{12.000000}{14.400000}\selectfont 400}}
%
\end{pgfscope}%
\begin{pgfscope}%
\pgfpathrectangle{\pgfqpoint{1.000000in}{0.350000in}}{\pgfqpoint{6.200000in}{2.800000in}} %
\pgfusepath{clip}%
\pgfsetbuttcap%
\pgfsetroundjoin%
\pgfsetlinewidth{0.501875pt}%
\definecolor{currentstroke}{rgb}{0.000000,0.000000,0.000000}%
\pgfsetstrokecolor{currentstroke}%
\pgfsetdash{{1.000000pt}{3.000000pt}}{0.000000pt}%
\pgfpathmoveto{\pgfqpoint{1.000000in}{3.150000in}}%
\pgfpathlineto{\pgfqpoint{7.200000in}{3.150000in}}%
\pgfusepath{stroke}%
\end{pgfscope}%
\begin{pgfscope}%
\pgfsetbuttcap%
\pgfsetroundjoin%
\definecolor{currentfill}{rgb}{0.000000,0.000000,0.000000}%
\pgfsetfillcolor{currentfill}%
\pgfsetlinewidth{0.501875pt}%
\definecolor{currentstroke}{rgb}{0.000000,0.000000,0.000000}%
\pgfsetstrokecolor{currentstroke}%
\pgfsetdash{}{0pt}%
\pgfsys@defobject{currentmarker}{\pgfqpoint{0.000000in}{0.000000in}}{\pgfqpoint{0.055556in}{0.000000in}}{%
\pgfpathmoveto{\pgfqpoint{0.000000in}{0.000000in}}%
\pgfpathlineto{\pgfqpoint{0.055556in}{0.000000in}}%
\pgfusepath{stroke,fill}%
}%
\begin{pgfscope}%
\pgfsys@transformshift{1.000000in}{3.150000in}%
\pgfsys@useobject{currentmarker}{}%
\end{pgfscope}%
\end{pgfscope}%
\begin{pgfscope}%
\pgfsetbuttcap%
\pgfsetroundjoin%
\definecolor{currentfill}{rgb}{0.000000,0.000000,0.000000}%
\pgfsetfillcolor{currentfill}%
\pgfsetlinewidth{0.501875pt}%
\definecolor{currentstroke}{rgb}{0.000000,0.000000,0.000000}%
\pgfsetstrokecolor{currentstroke}%
\pgfsetdash{}{0pt}%
\pgfsys@defobject{currentmarker}{\pgfqpoint{-0.055556in}{0.000000in}}{\pgfqpoint{0.000000in}{0.000000in}}{%
\pgfpathmoveto{\pgfqpoint{0.000000in}{0.000000in}}%
\pgfpathlineto{\pgfqpoint{-0.055556in}{0.000000in}}%
\pgfusepath{stroke,fill}%
}%
\begin{pgfscope}%
\pgfsys@transformshift{7.200000in}{3.150000in}%
\pgfsys@useobject{currentmarker}{}%
\end{pgfscope}%
\end{pgfscope}%
\begin{pgfscope}%
\pgftext[left,bottom,x=0.626329in,y=3.086971in,rotate=0.000000]{{\sffamily\fontsize{12.000000}{14.400000}\selectfont 410}}
%
\end{pgfscope}%
\begin{pgfscope}%
\pgftext[left,bottom,x=0.556885in,y=0.616536in,rotate=90.000000]{{\sffamily\fontsize{12.000000}{14.400000}\selectfont number of hydrogen bonds}}
%
\end{pgfscope}%
\begin{pgfscope}%
\pgfsetrectcap%
\pgfsetroundjoin%
\pgfsetlinewidth{1.003750pt}%
\definecolor{currentstroke}{rgb}{0.000000,0.000000,0.000000}%
\pgfsetstrokecolor{currentstroke}%
\pgfsetdash{}{0pt}%
\pgfpathmoveto{\pgfqpoint{1.000000in}{3.150000in}}%
\pgfpathlineto{\pgfqpoint{7.200000in}{3.150000in}}%
\pgfusepath{stroke}%
\end{pgfscope}%
\begin{pgfscope}%
\pgfsetrectcap%
\pgfsetroundjoin%
\pgfsetlinewidth{1.003750pt}%
\definecolor{currentstroke}{rgb}{0.000000,0.000000,0.000000}%
\pgfsetstrokecolor{currentstroke}%
\pgfsetdash{}{0pt}%
\pgfpathmoveto{\pgfqpoint{7.200000in}{0.350000in}}%
\pgfpathlineto{\pgfqpoint{7.200000in}{3.150000in}}%
\pgfusepath{stroke}%
\end{pgfscope}%
\begin{pgfscope}%
\pgfsetrectcap%
\pgfsetroundjoin%
\pgfsetlinewidth{1.003750pt}%
\definecolor{currentstroke}{rgb}{0.000000,0.000000,0.000000}%
\pgfsetstrokecolor{currentstroke}%
\pgfsetdash{}{0pt}%
\pgfpathmoveto{\pgfqpoint{1.000000in}{0.350000in}}%
\pgfpathlineto{\pgfqpoint{7.200000in}{0.350000in}}%
\pgfusepath{stroke}%
\end{pgfscope}%
\begin{pgfscope}%
\pgfsetrectcap%
\pgfsetroundjoin%
\pgfsetlinewidth{1.003750pt}%
\definecolor{currentstroke}{rgb}{0.000000,0.000000,0.000000}%
\pgfsetstrokecolor{currentstroke}%
\pgfsetdash{}{0pt}%
\pgfpathmoveto{\pgfqpoint{1.000000in}{0.350000in}}%
\pgfpathlineto{\pgfqpoint{1.000000in}{3.150000in}}%
\pgfusepath{stroke}%
\end{pgfscope}%
\begin{pgfscope}%
\pgfsetrectcap%
\pgfsetroundjoin%
\definecolor{currentfill}{rgb}{1.000000,1.000000,1.000000}%
\pgfsetfillcolor{currentfill}%
\pgfsetlinewidth{1.003750pt}%
\definecolor{currentstroke}{rgb}{0.000000,0.000000,0.000000}%
\pgfsetstrokecolor{currentstroke}%
\pgfsetdash{}{0pt}%
\pgfpathmoveto{\pgfqpoint{1.069417in}{2.427606in}}%
\pgfpathlineto{\pgfqpoint{1.926808in}{2.427606in}}%
\pgfpathlineto{\pgfqpoint{1.926808in}{3.080583in}}%
\pgfpathlineto{\pgfqpoint{1.069417in}{3.080583in}}%
\pgfpathlineto{\pgfqpoint{1.069417in}{2.427606in}}%
\pgfpathclose%
\pgfusepath{stroke,fill}%
\end{pgfscope}%
\begin{pgfscope}%
\pgfsetrectcap%
\pgfsetroundjoin%
\pgfsetlinewidth{1.003750pt}%
\definecolor{currentstroke}{rgb}{0.000000,0.000000,1.000000}%
\pgfsetstrokecolor{currentstroke}%
\pgfsetdash{}{0pt}%
\pgfpathmoveto{\pgfqpoint{1.166600in}{2.968161in}}%
\pgfpathlineto{\pgfqpoint{1.360967in}{2.968161in}}%
\pgfusepath{stroke}%
\end{pgfscope}%
\begin{pgfscope}%
\pgftext[left,bottom,x=1.513683in,y=2.890691in,rotate=0.000000]{{\sffamily\fontsize{9.996000}{11.995200}\selectfont spc}}
%
\end{pgfscope}%
\begin{pgfscope}%
\pgfsetrectcap%
\pgfsetroundjoin%
\pgfsetlinewidth{1.003750pt}%
\definecolor{currentstroke}{rgb}{0.000000,0.500000,0.000000}%
\pgfsetstrokecolor{currentstroke}%
\pgfsetdash{}{0pt}%
\pgfpathmoveto{\pgfqpoint{1.166600in}{2.764385in}}%
\pgfpathlineto{\pgfqpoint{1.360967in}{2.764385in}}%
\pgfusepath{stroke}%
\end{pgfscope}%
\begin{pgfscope}%
\pgftext[left,bottom,x=1.513683in,y=2.686915in,rotate=0.000000]{{\sffamily\fontsize{9.996000}{11.995200}\selectfont spce}}
%
\end{pgfscope}%
\begin{pgfscope}%
\pgfsetrectcap%
\pgfsetroundjoin%
\pgfsetlinewidth{1.003750pt}%
\definecolor{currentstroke}{rgb}{1.000000,0.000000,0.000000}%
\pgfsetstrokecolor{currentstroke}%
\pgfsetdash{}{0pt}%
\pgfpathmoveto{\pgfqpoint{1.166600in}{2.560609in}}%
\pgfpathlineto{\pgfqpoint{1.360967in}{2.560609in}}%
\pgfusepath{stroke}%
\end{pgfscope}%
\begin{pgfscope}%
\pgftext[left,bottom,x=1.513683in,y=2.483139in,rotate=0.000000]{{\sffamily\fontsize{9.996000}{11.995200}\selectfont tip3p}}
%
\end{pgfscope}%
\end{pgfpicture}%
\makeatother%
\endgroup%
}
    \caption{Fluctuation of the number of hydrogen bonds.} \label{fig:hbnum}
\end{figure}

The water molecule works as a donor and an acceptor for hydrogen bridges due to its electronegative oxygen and electropositive hydrogen. The number of hydrogen bonds fluctuates. The average was calculated starting at $\unit[30]{ps}$ to ensure equilibrium. \textit{SPC/E} yields the highest and \textit{TIP3P} the lowest value. The difference between the values is up to $\unit[9]{\%}$.

\subsection{Mean-square displacement}
\begin{figure}[H]
	\resizebox{\linewidth}{!}{%% Creator: Matplotlib, PGF backend
%%
%% To include the figure in your LaTeX document, write
%%   \input{<filename>.pgf}
%%
%% Make sure the required packages are loaded in your preamble
%%   \usepackage{pgf}
%%
%% Figures using additional raster images can only be included by \input if
%% they are in the same directory as the main LaTeX file. For loading figures
%% from other directories you can use the `import` package
%%   \usepackage{import}
%% and then include the figures with
%%   \import{<path to file>}{<filename>.pgf}
%%
%% Matplotlib used the following preamble
%%   \usepackage{fontspec}
%%   \setmainfont{DejaVu Serif}
%%   \setsansfont{DejaVu Sans}
%%   \setmonofont{DejaVu Sans Mono}
%%
\begingroup%
\makeatletter%
\begin{pgfpicture}%
\pgfpathrectangle{\pgfpointorigin}{\pgfqpoint{8.000000in}{3.000000in}}%
\pgfusepath{use as bounding box}%
\begin{pgfscope}%
\pgfsetrectcap%
\pgfsetroundjoin%
\definecolor{currentfill}{rgb}{1.000000,1.000000,1.000000}%
\pgfsetfillcolor{currentfill}%
\pgfsetlinewidth{0.000000pt}%
\definecolor{currentstroke}{rgb}{1.000000,1.000000,1.000000}%
\pgfsetstrokecolor{currentstroke}%
\pgfsetdash{}{0pt}%
\pgfpathmoveto{\pgfqpoint{0.000000in}{0.000000in}}%
\pgfpathlineto{\pgfqpoint{8.000000in}{0.000000in}}%
\pgfpathlineto{\pgfqpoint{8.000000in}{3.000000in}}%
\pgfpathlineto{\pgfqpoint{0.000000in}{3.000000in}}%
\pgfpathclose%
\pgfusepath{fill}%
\end{pgfscope}%
\begin{pgfscope}%
\pgfsetrectcap%
\pgfsetroundjoin%
\definecolor{currentfill}{rgb}{1.000000,1.000000,1.000000}%
\pgfsetfillcolor{currentfill}%
\pgfsetlinewidth{0.000000pt}%
\definecolor{currentstroke}{rgb}{0.000000,0.000000,0.000000}%
\pgfsetstrokecolor{currentstroke}%
\pgfsetdash{}{0pt}%
\pgfpathmoveto{\pgfqpoint{1.000000in}{0.300000in}}%
\pgfpathlineto{\pgfqpoint{7.200000in}{0.300000in}}%
\pgfpathlineto{\pgfqpoint{7.200000in}{2.700000in}}%
\pgfpathlineto{\pgfqpoint{1.000000in}{2.700000in}}%
\pgfpathclose%
\pgfusepath{fill}%
\end{pgfscope}%
\begin{pgfscope}%
\pgfpathrectangle{\pgfqpoint{1.000000in}{0.300000in}}{\pgfqpoint{6.200000in}{2.400000in}} %
\pgfusepath{clip}%
\pgfsetrectcap%
\pgfsetroundjoin%
\pgfsetlinewidth{1.003750pt}%
\definecolor{currentstroke}{rgb}{0.000000,0.000000,1.000000}%
\pgfsetstrokecolor{currentstroke}%
\pgfsetdash{}{0pt}%
\pgfpathmoveto{\pgfqpoint{1.000000in}{0.490652in}}%
\pgfpathlineto{\pgfqpoint{1.466596in}{0.678595in}}%
\pgfpathlineto{\pgfqpoint{1.739538in}{0.776805in}}%
\pgfpathlineto{\pgfqpoint{1.933193in}{0.837675in}}%
\pgfpathlineto{\pgfqpoint{2.614159in}{1.018632in}}%
\pgfpathlineto{\pgfqpoint{2.822941in}{1.068905in}}%
\pgfpathlineto{\pgfqpoint{2.945672in}{1.103021in}}%
\pgfpathlineto{\pgfqpoint{3.311611in}{1.206745in}}%
\pgfpathlineto{\pgfqpoint{3.448665in}{1.243931in}}%
\pgfpathlineto{\pgfqpoint{3.562479in}{1.278696in}}%
\pgfpathlineto{\pgfqpoint{3.697562in}{1.318435in}}%
\pgfpathlineto{\pgfqpoint{3.810016in}{1.353491in}}%
\pgfpathlineto{\pgfqpoint{3.982633in}{1.403738in}}%
\pgfpathlineto{\pgfqpoint{4.006255in}{1.409618in}}%
\pgfpathlineto{\pgfqpoint{4.036514in}{1.418703in}}%
\pgfpathlineto{\pgfqpoint{4.188203in}{1.463418in}}%
\pgfpathlineto{\pgfqpoint{4.211417in}{1.470487in}}%
\pgfpathlineto{\pgfqpoint{4.244804in}{1.481321in}}%
\pgfpathlineto{\pgfqpoint{4.331290in}{1.507673in}}%
\pgfpathlineto{\pgfqpoint{4.345462in}{1.513228in}}%
\pgfpathlineto{\pgfqpoint{4.395014in}{1.527894in}}%
\pgfpathlineto{\pgfqpoint{4.453224in}{1.547018in}}%
\pgfpathlineto{\pgfqpoint{4.491922in}{1.557454in}}%
\pgfpathlineto{\pgfqpoint{4.950955in}{1.696155in}}%
\pgfpathlineto{\pgfqpoint{4.956636in}{1.697696in}}%
\pgfpathlineto{\pgfqpoint{4.969707in}{1.702707in}}%
\pgfpathlineto{\pgfqpoint{4.998665in}{1.710671in}}%
\pgfpathlineto{\pgfqpoint{5.017874in}{1.716260in}}%
\pgfpathlineto{\pgfqpoint{5.263288in}{1.791778in}}%
\pgfpathlineto{\pgfqpoint{5.270424in}{1.793495in}}%
\pgfpathlineto{\pgfqpoint{5.283313in}{1.797389in}}%
\pgfpathlineto{\pgfqpoint{5.300501in}{1.802135in}}%
\pgfpathlineto{\pgfqpoint{5.309492in}{1.805282in}}%
\pgfpathlineto{\pgfqpoint{5.347472in}{1.816202in}}%
\pgfpathlineto{\pgfqpoint{5.354818in}{1.818604in}}%
\pgfpathlineto{\pgfqpoint{5.380416in}{1.826285in}}%
\pgfpathlineto{\pgfqpoint{5.385421in}{1.827569in}}%
\pgfpathlineto{\pgfqpoint{5.394337in}{1.830562in}}%
\pgfpathlineto{\pgfqpoint{5.406044in}{1.833712in}}%
\pgfpathlineto{\pgfqpoint{5.432595in}{1.842333in}}%
\pgfpathlineto{\pgfqpoint{5.453647in}{1.847937in}}%
\pgfpathlineto{\pgfqpoint{5.461709in}{1.850274in}}%
\pgfpathlineto{\pgfqpoint{5.501470in}{1.862880in}}%
\pgfpathlineto{\pgfqpoint{5.505654in}{1.864186in}}%
\pgfpathlineto{\pgfqpoint{5.547689in}{1.876256in}}%
\pgfpathlineto{\pgfqpoint{5.553930in}{1.878220in}}%
\pgfpathlineto{\pgfqpoint{5.562416in}{1.880542in}}%
\pgfpathlineto{\pgfqpoint{5.574573in}{1.883765in}}%
\pgfpathlineto{\pgfqpoint{5.630190in}{1.900503in}}%
\pgfpathlineto{\pgfqpoint{5.633647in}{1.901690in}}%
\pgfpathlineto{\pgfqpoint{5.639142in}{1.903373in}}%
\pgfpathlineto{\pgfqpoint{5.642554in}{1.904314in}}%
\pgfpathlineto{\pgfqpoint{5.647977in}{1.905600in}}%
\pgfpathlineto{\pgfqpoint{5.651345in}{1.907215in}}%
\pgfpathlineto{\pgfqpoint{5.654696in}{1.908233in}}%
\pgfpathlineto{\pgfqpoint{5.680274in}{1.915580in}}%
\pgfpathlineto{\pgfqpoint{5.694915in}{1.920019in}}%
\pgfpathlineto{\pgfqpoint{5.698057in}{1.920621in}}%
\pgfpathlineto{\pgfqpoint{5.705536in}{1.922882in}}%
\pgfpathlineto{\pgfqpoint{5.714159in}{1.925454in}}%
\pgfpathlineto{\pgfqpoint{5.716602in}{1.926911in}}%
\pgfpathlineto{\pgfqpoint{5.727489in}{1.929980in}}%
\pgfpathlineto{\pgfqpoint{5.735244in}{1.932092in}}%
\pgfpathlineto{\pgfqpoint{5.741149in}{1.933727in}}%
\pgfpathlineto{\pgfqpoint{5.746419in}{1.935760in}}%
\pgfpathlineto{\pgfqpoint{5.749329in}{1.936639in}}%
\pgfpathlineto{\pgfqpoint{5.755112in}{1.938305in}}%
\pgfpathlineto{\pgfqpoint{5.758559in}{1.939157in}}%
\pgfpathlineto{\pgfqpoint{5.769920in}{1.942704in}}%
\pgfpathlineto{\pgfqpoint{5.784409in}{1.947343in}}%
\pgfpathlineto{\pgfqpoint{5.790447in}{1.949074in}}%
\pgfpathlineto{\pgfqpoint{5.796972in}{1.950721in}}%
\pgfpathlineto{\pgfqpoint{5.810366in}{1.954884in}}%
\pgfpathlineto{\pgfqpoint{5.813541in}{1.955664in}}%
\pgfpathlineto{\pgfqpoint{5.822978in}{1.958735in}}%
\pgfpathlineto{\pgfqpoint{5.829712in}{1.961101in}}%
\pgfpathlineto{\pgfqpoint{5.862408in}{1.970415in}}%
\pgfpathlineto{\pgfqpoint{5.866325in}{1.971851in}}%
\pgfpathlineto{\pgfqpoint{5.909800in}{1.984563in}}%
\pgfpathlineto{\pgfqpoint{5.914360in}{1.986203in}}%
\pgfpathlineto{\pgfqpoint{5.929640in}{1.991107in}}%
\pgfpathlineto{\pgfqpoint{5.932743in}{1.991880in}}%
\pgfpathlineto{\pgfqpoint{5.952786in}{1.997284in}}%
\pgfpathlineto{\pgfqpoint{5.957066in}{1.998419in}}%
\pgfpathlineto{\pgfqpoint{5.972250in}{2.003475in}}%
\pgfpathlineto{\pgfqpoint{5.975579in}{2.004147in}}%
\pgfpathlineto{\pgfqpoint{5.994807in}{2.009010in}}%
\pgfpathlineto{\pgfqpoint{5.999229in}{2.010413in}}%
\pgfpathlineto{\pgfqpoint{6.010356in}{2.013824in}}%
\pgfpathlineto{\pgfqpoint{6.025939in}{2.017674in}}%
\pgfpathlineto{\pgfqpoint{6.033217in}{2.019566in}}%
\pgfpathlineto{\pgfqpoint{6.036257in}{2.020037in}}%
\pgfpathlineto{\pgfqpoint{6.040039in}{2.021398in}}%
\pgfpathlineto{\pgfqpoint{6.041922in}{2.022084in}}%
\pgfpathlineto{\pgfqpoint{6.070274in}{2.029160in}}%
\pgfpathlineto{\pgfqpoint{6.072793in}{2.029274in}}%
\pgfpathlineto{\pgfqpoint{6.079228in}{2.031932in}}%
\pgfpathlineto{\pgfqpoint{6.103340in}{2.038358in}}%
\pgfpathlineto{\pgfqpoint{6.105397in}{2.038453in}}%
\pgfpathlineto{\pgfqpoint{6.108129in}{2.038930in}}%
\pgfpathlineto{\pgfqpoint{6.111868in}{2.040383in}}%
\pgfpathlineto{\pgfqpoint{6.113898in}{2.041364in}}%
\pgfpathlineto{\pgfqpoint{6.131245in}{2.045936in}}%
\pgfpathlineto{\pgfqpoint{6.133547in}{2.046757in}}%
\pgfpathlineto{\pgfqpoint{6.137474in}{2.048063in}}%
\pgfpathlineto{\pgfqpoint{6.142351in}{2.048772in}}%
\pgfpathlineto{\pgfqpoint{6.145260in}{2.049656in}}%
\pgfpathlineto{\pgfqpoint{6.159936in}{2.054396in}}%
\pgfpathlineto{\pgfqpoint{6.164340in}{2.055031in}}%
\pgfpathlineto{\pgfqpoint{6.177381in}{2.058923in}}%
\pgfpathlineto{\pgfqpoint{6.180143in}{2.060203in}}%
\pgfpathlineto{\pgfqpoint{6.187756in}{2.062394in}}%
\pgfpathlineto{\pgfqpoint{6.190174in}{2.063065in}}%
\pgfpathlineto{\pgfqpoint{6.192884in}{2.063385in}}%
\pgfpathlineto{\pgfqpoint{6.195284in}{2.063814in}}%
\pgfpathlineto{\pgfqpoint{6.202729in}{2.065390in}}%
\pgfpathlineto{\pgfqpoint{6.204504in}{2.066239in}}%
\pgfpathlineto{\pgfqpoint{6.218245in}{2.070677in}}%
\pgfpathlineto{\pgfqpoint{6.241008in}{2.077864in}}%
\pgfpathlineto{\pgfqpoint{6.244079in}{2.078182in}}%
\pgfpathlineto{\pgfqpoint{6.263568in}{2.082853in}}%
\pgfpathlineto{\pgfqpoint{6.266538in}{2.084775in}}%
\pgfpathlineto{\pgfqpoint{6.268152in}{2.085261in}}%
\pgfpathlineto{\pgfqpoint{6.271369in}{2.085694in}}%
\pgfpathlineto{\pgfqpoint{6.276962in}{2.087250in}}%
\pgfpathlineto{\pgfqpoint{6.278816in}{2.087385in}}%
\pgfpathlineto{\pgfqpoint{6.283560in}{2.088839in}}%
\pgfpathlineto{\pgfqpoint{6.286705in}{2.088934in}}%
\pgfpathlineto{\pgfqpoint{6.288532in}{2.089970in}}%
\pgfpathlineto{\pgfqpoint{6.290355in}{2.090515in}}%
\pgfpathlineto{\pgfqpoint{6.292950in}{2.091989in}}%
\pgfpathlineto{\pgfqpoint{6.294502in}{2.092319in}}%
\pgfpathlineto{\pgfqpoint{6.301954in}{2.094096in}}%
\pgfpathlineto{\pgfqpoint{6.302976in}{2.094429in}}%
\pgfpathlineto{\pgfqpoint{6.304505in}{2.094523in}}%
\pgfpathlineto{\pgfqpoint{6.308567in}{2.095866in}}%
\pgfpathlineto{\pgfqpoint{6.334871in}{2.102772in}}%
\pgfpathlineto{\pgfqpoint{6.337058in}{2.102965in}}%
\pgfpathlineto{\pgfqpoint{6.339237in}{2.104060in}}%
\pgfpathlineto{\pgfqpoint{6.342854in}{2.105757in}}%
\pgfpathlineto{\pgfqpoint{6.344536in}{2.106093in}}%
\pgfpathlineto{\pgfqpoint{6.347170in}{2.106529in}}%
\pgfpathlineto{\pgfqpoint{6.392002in}{2.119209in}}%
\pgfpathlineto{\pgfqpoint{6.393565in}{2.119220in}}%
\pgfpathlineto{\pgfqpoint{6.396902in}{2.120088in}}%
\pgfpathlineto{\pgfqpoint{6.399560in}{2.120643in}}%
\pgfpathlineto{\pgfqpoint{6.401327in}{2.121254in}}%
\pgfpathlineto{\pgfqpoint{6.410959in}{2.124137in}}%
\pgfpathlineto{\pgfqpoint{6.414860in}{2.125281in}}%
\pgfpathlineto{\pgfqpoint{6.416371in}{2.125686in}}%
\pgfpathlineto{\pgfqpoint{6.421954in}{2.126974in}}%
\pgfpathlineto{\pgfqpoint{6.426004in}{2.127602in}}%
\pgfpathlineto{\pgfqpoint{6.427491in}{2.127653in}}%
\pgfpathlineto{\pgfqpoint{6.430242in}{2.128678in}}%
\pgfpathlineto{\pgfqpoint{6.440929in}{2.132036in}}%
\pgfpathlineto{\pgfqpoint{6.444040in}{2.133084in}}%
\pgfpathlineto{\pgfqpoint{6.451244in}{2.134426in}}%
\pgfpathlineto{\pgfqpoint{6.453697in}{2.135367in}}%
\pgfpathlineto{\pgfqpoint{6.455733in}{2.136042in}}%
\pgfpathlineto{\pgfqpoint{6.458372in}{2.137109in}}%
\pgfpathlineto{\pgfqpoint{6.460395in}{2.137370in}}%
\pgfpathlineto{\pgfqpoint{6.465826in}{2.139498in}}%
\pgfpathlineto{\pgfqpoint{6.467227in}{2.139225in}}%
\pgfpathlineto{\pgfqpoint{6.468425in}{2.139067in}}%
\pgfpathlineto{\pgfqpoint{6.474782in}{2.141288in}}%
\pgfpathlineto{\pgfqpoint{6.476164in}{2.142283in}}%
\pgfpathlineto{\pgfqpoint{6.478724in}{2.142863in}}%
\pgfpathlineto{\pgfqpoint{6.480098in}{2.143210in}}%
\pgfpathlineto{\pgfqpoint{6.485179in}{2.145018in}}%
\pgfpathlineto{\pgfqpoint{6.486928in}{2.144984in}}%
\pgfpathlineto{\pgfqpoint{6.487705in}{2.144757in}}%
\pgfpathlineto{\pgfqpoint{6.488092in}{2.144965in}}%
\pgfpathlineto{\pgfqpoint{6.493305in}{2.147049in}}%
\pgfpathlineto{\pgfqpoint{6.494458in}{2.146772in}}%
\pgfpathlineto{\pgfqpoint{6.498288in}{2.148178in}}%
\pgfpathlineto{\pgfqpoint{6.500004in}{2.148699in}}%
\pgfpathlineto{\pgfqpoint{6.502664in}{2.149678in}}%
\pgfpathlineto{\pgfqpoint{6.503612in}{2.150240in}}%
\pgfpathlineto{\pgfqpoint{6.505881in}{2.150714in}}%
\pgfpathlineto{\pgfqpoint{6.507201in}{2.150395in}}%
\pgfpathlineto{\pgfqpoint{6.512269in}{2.152104in}}%
\pgfpathlineto{\pgfqpoint{6.513950in}{2.152363in}}%
\pgfpathlineto{\pgfqpoint{6.516184in}{2.153438in}}%
\pgfpathlineto{\pgfqpoint{6.539748in}{2.160542in}}%
\pgfpathlineto{\pgfqpoint{6.541362in}{2.161838in}}%
\pgfpathlineto{\pgfqpoint{6.542793in}{2.161362in}}%
\pgfpathlineto{\pgfqpoint{6.544043in}{2.161378in}}%
\pgfpathlineto{\pgfqpoint{6.549019in}{2.162800in}}%
\pgfpathlineto{\pgfqpoint{6.552551in}{2.164128in}}%
\pgfpathlineto{\pgfqpoint{6.555538in}{2.165406in}}%
\pgfpathlineto{\pgfqpoint{6.557114in}{2.165562in}}%
\pgfpathlineto{\pgfqpoint{6.560256in}{2.167340in}}%
\pgfpathlineto{\pgfqpoint{6.561126in}{2.166801in}}%
\pgfpathlineto{\pgfqpoint{6.561474in}{2.166970in}}%
\pgfpathlineto{\pgfqpoint{6.564595in}{2.167948in}}%
\pgfpathlineto{\pgfqpoint{6.565805in}{2.168142in}}%
\pgfpathlineto{\pgfqpoint{6.570966in}{2.169870in}}%
\pgfpathlineto{\pgfqpoint{6.572848in}{2.170470in}}%
\pgfpathlineto{\pgfqpoint{6.573702in}{2.170172in}}%
\pgfpathlineto{\pgfqpoint{6.574043in}{2.170764in}}%
\pgfpathlineto{\pgfqpoint{6.575066in}{2.171476in}}%
\pgfpathlineto{\pgfqpoint{6.576937in}{2.173129in}}%
\pgfpathlineto{\pgfqpoint{6.577107in}{2.172978in}}%
\pgfpathlineto{\pgfqpoint{6.578634in}{2.172310in}}%
\pgfpathlineto{\pgfqpoint{6.582856in}{2.173352in}}%
\pgfpathlineto{\pgfqpoint{6.583866in}{2.174314in}}%
\pgfpathlineto{\pgfqpoint{6.585713in}{2.175049in}}%
\pgfpathlineto{\pgfqpoint{6.586048in}{2.174278in}}%
\pgfpathlineto{\pgfqpoint{6.586885in}{2.174965in}}%
\pgfpathlineto{\pgfqpoint{6.588557in}{2.175851in}}%
\pgfpathlineto{\pgfqpoint{6.590058in}{2.175563in}}%
\pgfpathlineto{\pgfqpoint{6.596028in}{2.177899in}}%
\pgfpathlineto{\pgfqpoint{6.597183in}{2.178310in}}%
\pgfpathlineto{\pgfqpoint{6.598171in}{2.178648in}}%
\pgfpathlineto{\pgfqpoint{6.598336in}{2.178407in}}%
\pgfpathlineto{\pgfqpoint{6.599651in}{2.178726in}}%
\pgfpathlineto{\pgfqpoint{6.600471in}{2.180028in}}%
\pgfpathlineto{\pgfqpoint{6.602273in}{2.180906in}}%
\pgfpathlineto{\pgfqpoint{6.602437in}{2.180676in}}%
\pgfpathlineto{\pgfqpoint{6.602600in}{2.180206in}}%
\pgfpathlineto{\pgfqpoint{6.603417in}{2.180624in}}%
\pgfpathlineto{\pgfqpoint{6.605048in}{2.181798in}}%
\pgfpathlineto{\pgfqpoint{6.605211in}{2.181616in}}%
\pgfpathlineto{\pgfqpoint{6.607975in}{2.182094in}}%
\pgfpathlineto{\pgfqpoint{6.609595in}{2.183868in}}%
\pgfpathlineto{\pgfqpoint{6.612984in}{2.183317in}}%
\pgfpathlineto{\pgfqpoint{6.614432in}{2.184199in}}%
\pgfpathlineto{\pgfqpoint{6.615716in}{2.184056in}}%
\pgfpathlineto{\pgfqpoint{6.617797in}{2.185624in}}%
\pgfpathlineto{\pgfqpoint{6.618277in}{2.185843in}}%
\pgfpathlineto{\pgfqpoint{6.618596in}{2.185276in}}%
\pgfpathlineto{\pgfqpoint{6.619553in}{2.185081in}}%
\pgfpathlineto{\pgfqpoint{6.619713in}{2.185391in}}%
\pgfpathlineto{\pgfqpoint{6.620509in}{2.186252in}}%
\pgfpathlineto{\pgfqpoint{6.622258in}{2.187613in}}%
\pgfpathlineto{\pgfqpoint{6.622417in}{2.187389in}}%
\pgfpathlineto{\pgfqpoint{6.623052in}{2.186987in}}%
\pgfpathlineto{\pgfqpoint{6.623528in}{2.187721in}}%
\pgfpathlineto{\pgfqpoint{6.625111in}{2.188748in}}%
\pgfpathlineto{\pgfqpoint{6.626059in}{2.189038in}}%
\pgfpathlineto{\pgfqpoint{6.626217in}{2.188846in}}%
\pgfpathlineto{\pgfqpoint{6.627478in}{2.187994in}}%
\pgfpathlineto{\pgfqpoint{6.627636in}{2.188084in}}%
\pgfpathlineto{\pgfqpoint{6.629367in}{2.188924in}}%
\pgfpathlineto{\pgfqpoint{6.630779in}{2.189491in}}%
\pgfpathlineto{\pgfqpoint{6.632189in}{2.189305in}}%
\pgfpathlineto{\pgfqpoint{6.633908in}{2.190838in}}%
\pgfpathlineto{\pgfqpoint{6.634064in}{2.190760in}}%
\pgfpathlineto{\pgfqpoint{6.635311in}{2.189722in}}%
\pgfpathlineto{\pgfqpoint{6.635467in}{2.190004in}}%
\pgfpathlineto{\pgfqpoint{6.641667in}{2.193905in}}%
\pgfpathlineto{\pgfqpoint{6.644131in}{2.192785in}}%
\pgfpathlineto{\pgfqpoint{6.651622in}{2.195961in}}%
\pgfpathlineto{\pgfqpoint{6.653292in}{2.197653in}}%
\pgfpathlineto{\pgfqpoint{6.653444in}{2.197419in}}%
\pgfpathlineto{\pgfqpoint{6.654201in}{2.196954in}}%
\pgfpathlineto{\pgfqpoint{6.654655in}{2.197303in}}%
\pgfpathlineto{\pgfqpoint{6.657072in}{2.198880in}}%
\pgfpathlineto{\pgfqpoint{6.657223in}{2.198727in}}%
\pgfpathlineto{\pgfqpoint{6.659030in}{2.198228in}}%
\pgfpathlineto{\pgfqpoint{6.659481in}{2.198140in}}%
\pgfpathlineto{\pgfqpoint{6.659932in}{2.198542in}}%
\pgfpathlineto{\pgfqpoint{6.661581in}{2.198542in}}%
\pgfpathlineto{\pgfqpoint{6.662479in}{2.197958in}}%
\pgfpathlineto{\pgfqpoint{6.662779in}{2.198948in}}%
\pgfpathlineto{\pgfqpoint{6.663675in}{2.199790in}}%
\pgfpathlineto{\pgfqpoint{6.664123in}{2.199634in}}%
\pgfpathlineto{\pgfqpoint{6.664421in}{2.200219in}}%
\pgfpathlineto{\pgfqpoint{6.664869in}{2.199462in}}%
\pgfpathlineto{\pgfqpoint{6.664869in}{2.199462in}}%
\pgfpathlineto{\pgfqpoint{6.665018in}{2.199014in}}%
\pgfpathlineto{\pgfqpoint{6.665316in}{2.199485in}}%
\pgfpathlineto{\pgfqpoint{6.665316in}{2.199485in}}%
\pgfpathlineto{\pgfqpoint{6.667546in}{2.202355in}}%
\pgfpathlineto{\pgfqpoint{6.668881in}{2.203152in}}%
\pgfpathlineto{\pgfqpoint{6.669029in}{2.202673in}}%
\pgfpathlineto{\pgfqpoint{6.670213in}{2.203476in}}%
\pgfpathlineto{\pgfqpoint{6.671986in}{2.204608in}}%
\pgfpathlineto{\pgfqpoint{6.677275in}{2.202886in}}%
\pgfpathlineto{\pgfqpoint{6.678444in}{2.204860in}}%
\pgfpathlineto{\pgfqpoint{6.678590in}{2.204448in}}%
\pgfpathlineto{\pgfqpoint{6.679320in}{2.204816in}}%
\pgfpathlineto{\pgfqpoint{6.679612in}{2.204240in}}%
\pgfpathlineto{\pgfqpoint{6.679758in}{2.203927in}}%
\pgfpathlineto{\pgfqpoint{6.679904in}{2.204343in}}%
\pgfpathlineto{\pgfqpoint{6.679904in}{2.204343in}}%
\pgfpathlineto{\pgfqpoint{6.685998in}{2.209760in}}%
\pgfpathlineto{\pgfqpoint{6.686431in}{2.209102in}}%
\pgfpathlineto{\pgfqpoint{6.688881in}{2.207784in}}%
\pgfpathlineto{\pgfqpoint{6.689600in}{2.209137in}}%
\pgfpathlineto{\pgfqpoint{6.690461in}{2.208545in}}%
\pgfpathlineto{\pgfqpoint{6.691752in}{2.207458in}}%
\pgfpathlineto{\pgfqpoint{6.692038in}{2.208328in}}%
\pgfpathlineto{\pgfqpoint{6.693611in}{2.210186in}}%
\pgfpathlineto{\pgfqpoint{6.694895in}{2.209148in}}%
\pgfpathlineto{\pgfqpoint{6.695038in}{2.209693in}}%
\pgfpathlineto{\pgfqpoint{6.695465in}{2.210773in}}%
\pgfpathlineto{\pgfqpoint{6.700291in}{2.216004in}}%
\pgfpathlineto{\pgfqpoint{6.700432in}{2.215673in}}%
\pgfpathlineto{\pgfqpoint{6.701845in}{2.213630in}}%
\pgfpathlineto{\pgfqpoint{6.702268in}{2.214524in}}%
\pgfpathlineto{\pgfqpoint{6.702972in}{2.214140in}}%
\pgfpathlineto{\pgfqpoint{6.703113in}{2.213787in}}%
\pgfpathlineto{\pgfqpoint{6.703536in}{2.214169in}}%
\pgfpathlineto{\pgfqpoint{6.703536in}{2.214169in}}%
\pgfpathlineto{\pgfqpoint{6.703817in}{2.215381in}}%
\pgfpathlineto{\pgfqpoint{6.704661in}{2.214629in}}%
\pgfpathlineto{\pgfqpoint{6.706905in}{2.215050in}}%
\pgfpathlineto{\pgfqpoint{6.707045in}{2.215327in}}%
\pgfpathlineto{\pgfqpoint{6.707045in}{2.215327in}}%
\pgfpathlineto{\pgfqpoint{6.707045in}{2.215327in}}%
\pgfpathlineto{\pgfqpoint{6.707185in}{2.214886in}}%
\pgfpathlineto{\pgfqpoint{6.707605in}{2.215443in}}%
\pgfpathlineto{\pgfqpoint{6.707605in}{2.215443in}}%
\pgfpathlineto{\pgfqpoint{6.707745in}{2.215585in}}%
\pgfpathlineto{\pgfqpoint{6.707884in}{2.215429in}}%
\pgfpathlineto{\pgfqpoint{6.707884in}{2.215429in}}%
\pgfpathlineto{\pgfqpoint{6.708304in}{2.213859in}}%
\pgfpathlineto{\pgfqpoint{6.709002in}{2.214235in}}%
\pgfpathlineto{\pgfqpoint{6.711093in}{2.218737in}}%
\pgfpathlineto{\pgfqpoint{6.711927in}{2.218740in}}%
\pgfpathlineto{\pgfqpoint{6.712066in}{2.217961in}}%
\pgfpathlineto{\pgfqpoint{6.712205in}{2.218065in}}%
\pgfpathlineto{\pgfqpoint{6.712205in}{2.218065in}}%
\pgfpathlineto{\pgfqpoint{6.713455in}{2.219377in}}%
\pgfpathlineto{\pgfqpoint{6.714286in}{2.221357in}}%
\pgfpathlineto{\pgfqpoint{6.714840in}{2.219542in}}%
\pgfpathlineto{\pgfqpoint{6.717464in}{2.217770in}}%
\pgfpathlineto{\pgfqpoint{6.717602in}{2.218182in}}%
\pgfpathlineto{\pgfqpoint{6.718842in}{2.219703in}}%
\pgfpathlineto{\pgfqpoint{6.718979in}{2.219371in}}%
\pgfpathlineto{\pgfqpoint{6.719117in}{2.219549in}}%
\pgfpathlineto{\pgfqpoint{6.719117in}{2.219549in}}%
\pgfpathlineto{\pgfqpoint{6.719254in}{2.219912in}}%
\pgfpathlineto{\pgfqpoint{6.719254in}{2.219912in}}%
\pgfpathlineto{\pgfqpoint{6.719254in}{2.219912in}}%
\pgfpathlineto{\pgfqpoint{6.719529in}{2.219265in}}%
\pgfpathlineto{\pgfqpoint{6.719804in}{2.219508in}}%
\pgfpathlineto{\pgfqpoint{6.719804in}{2.219508in}}%
\pgfpathlineto{\pgfqpoint{6.720765in}{2.222614in}}%
\pgfpathlineto{\pgfqpoint{6.721039in}{2.221647in}}%
\pgfpathlineto{\pgfqpoint{6.722272in}{2.218901in}}%
\pgfpathlineto{\pgfqpoint{6.722819in}{2.219120in}}%
\pgfpathlineto{\pgfqpoint{6.725140in}{2.224938in}}%
\pgfpathlineto{\pgfqpoint{6.725277in}{2.223782in}}%
\pgfpathlineto{\pgfqpoint{6.725413in}{2.223461in}}%
\pgfpathlineto{\pgfqpoint{6.725413in}{2.223461in}}%
\pgfpathlineto{\pgfqpoint{6.725413in}{2.223461in}}%
\pgfpathlineto{\pgfqpoint{6.726774in}{2.225884in}}%
\pgfpathlineto{\pgfqpoint{6.727997in}{2.227680in}}%
\pgfpathlineto{\pgfqpoint{6.728132in}{2.226928in}}%
\pgfpathlineto{\pgfqpoint{6.729488in}{2.224938in}}%
\pgfpathlineto{\pgfqpoint{6.729623in}{2.224753in}}%
\pgfpathlineto{\pgfqpoint{6.729623in}{2.224753in}}%
\pgfpathlineto{\pgfqpoint{6.729623in}{2.224753in}}%
\pgfpathlineto{\pgfqpoint{6.729894in}{2.226158in}}%
\pgfpathlineto{\pgfqpoint{6.730570in}{2.225070in}}%
\pgfpathlineto{\pgfqpoint{6.730976in}{2.223766in}}%
\pgfpathlineto{\pgfqpoint{6.731246in}{2.224455in}}%
\pgfpathlineto{\pgfqpoint{6.731246in}{2.224455in}}%
\pgfpathlineto{\pgfqpoint{6.732460in}{2.226528in}}%
\pgfpathlineto{\pgfqpoint{6.732999in}{2.225536in}}%
\pgfpathlineto{\pgfqpoint{6.733404in}{2.226469in}}%
\pgfusepath{stroke}%
\end{pgfscope}%
\begin{pgfscope}%
\pgfpathrectangle{\pgfqpoint{1.000000in}{0.300000in}}{\pgfqpoint{6.200000in}{2.400000in}} %
\pgfusepath{clip}%
\pgfsetrectcap%
\pgfsetroundjoin%
\pgfsetlinewidth{1.003750pt}%
\definecolor{currentstroke}{rgb}{0.000000,0.500000,0.000000}%
\pgfsetstrokecolor{currentstroke}%
\pgfsetdash{}{0pt}%
\pgfpathmoveto{\pgfqpoint{1.000000in}{0.479596in}}%
\pgfpathlineto{\pgfqpoint{1.466596in}{0.659706in}}%
\pgfpathlineto{\pgfqpoint{1.739538in}{0.747768in}}%
\pgfpathlineto{\pgfqpoint{1.933193in}{0.797601in}}%
\pgfpathlineto{\pgfqpoint{2.550000in}{0.933805in}}%
\pgfpathlineto{\pgfqpoint{2.672731in}{0.960515in}}%
\pgfpathlineto{\pgfqpoint{2.907196in}{1.014980in}}%
\pgfpathlineto{\pgfqpoint{3.110678in}{1.066823in}}%
\pgfpathlineto{\pgfqpoint{3.193209in}{1.089701in}}%
\pgfpathlineto{\pgfqpoint{3.448665in}{1.163300in}}%
\pgfpathlineto{\pgfqpoint{3.516036in}{1.180984in}}%
\pgfpathlineto{\pgfqpoint{3.633404in}{1.213700in}}%
\pgfpathlineto{\pgfqpoint{3.685210in}{1.226580in}}%
\pgfpathlineto{\pgfqpoint{3.767261in}{1.251933in}}%
\pgfpathlineto{\pgfqpoint{3.820293in}{1.268300in}}%
\pgfpathlineto{\pgfqpoint{3.949789in}{1.302819in}}%
\pgfpathlineto{\pgfqpoint{4.199910in}{1.379119in}}%
\pgfpathlineto{\pgfqpoint{4.255574in}{1.393159in}}%
\pgfpathlineto{\pgfqpoint{4.271414in}{1.399346in}}%
\pgfpathlineto{\pgfqpoint{4.302017in}{1.408146in}}%
\pgfpathlineto{\pgfqpoint{4.331290in}{1.416301in}}%
\pgfpathlineto{\pgfqpoint{4.359342in}{1.425235in}}%
\pgfpathlineto{\pgfqpoint{4.372941in}{1.428946in}}%
\pgfpathlineto{\pgfqpoint{4.395014in}{1.434524in}}%
\pgfpathlineto{\pgfqpoint{4.453224in}{1.452799in}}%
\pgfpathlineto{\pgfqpoint{4.517745in}{1.471600in}}%
\pgfpathlineto{\pgfqpoint{4.528516in}{1.475431in}}%
\pgfpathlineto{\pgfqpoint{4.539117in}{1.479185in}}%
\pgfpathlineto{\pgfqpoint{4.573295in}{1.488764in}}%
\pgfpathlineto{\pgfqpoint{4.596227in}{1.496721in}}%
\pgfpathlineto{\pgfqpoint{4.686517in}{1.523112in}}%
\pgfpathlineto{\pgfqpoint{4.703207in}{1.528736in}}%
\pgfpathlineto{\pgfqpoint{4.722171in}{1.534127in}}%
\pgfpathlineto{\pgfqpoint{4.732772in}{1.536441in}}%
\pgfpathlineto{\pgfqpoint{4.761092in}{1.545700in}}%
\pgfpathlineto{\pgfqpoint{4.783409in}{1.552366in}}%
\pgfpathlineto{\pgfqpoint{4.835035in}{1.568195in}}%
\pgfpathlineto{\pgfqpoint{4.844011in}{1.570761in}}%
\pgfpathlineto{\pgfqpoint{4.885083in}{1.583742in}}%
\pgfpathlineto{\pgfqpoint{4.949051in}{1.603028in}}%
\pgfpathlineto{\pgfqpoint{4.956636in}{1.605310in}}%
\pgfpathlineto{\pgfqpoint{4.978891in}{1.612508in}}%
\pgfpathlineto{\pgfqpoint{4.986149in}{1.614374in}}%
\pgfpathlineto{\pgfqpoint{4.993329in}{1.617488in}}%
\pgfpathlineto{\pgfqpoint{5.000434in}{1.619183in}}%
\pgfpathlineto{\pgfqpoint{5.009210in}{1.622085in}}%
\pgfpathlineto{\pgfqpoint{5.043215in}{1.633270in}}%
\pgfpathlineto{\pgfqpoint{5.051455in}{1.635868in}}%
\pgfpathlineto{\pgfqpoint{5.062823in}{1.639198in}}%
\pgfpathlineto{\pgfqpoint{5.074003in}{1.641954in}}%
\pgfpathlineto{\pgfqpoint{5.088109in}{1.647035in}}%
\pgfpathlineto{\pgfqpoint{5.106469in}{1.652707in}}%
\pgfpathlineto{\pgfqpoint{5.147456in}{1.665077in}}%
\pgfpathlineto{\pgfqpoint{5.153113in}{1.667082in}}%
\pgfpathlineto{\pgfqpoint{5.161510in}{1.669145in}}%
\pgfpathlineto{\pgfqpoint{5.252440in}{1.697068in}}%
\pgfpathlineto{\pgfqpoint{5.259691in}{1.699665in}}%
\pgfpathlineto{\pgfqpoint{5.364146in}{1.731708in}}%
\pgfpathlineto{\pgfqpoint{5.369272in}{1.733604in}}%
\pgfpathlineto{\pgfqpoint{5.379410in}{1.736908in}}%
\pgfpathlineto{\pgfqpoint{5.382422in}{1.737538in}}%
\pgfpathlineto{\pgfqpoint{5.392366in}{1.741402in}}%
\pgfpathlineto{\pgfqpoint{5.398262in}{1.742668in}}%
\pgfpathlineto{\pgfqpoint{5.425115in}{1.751773in}}%
\pgfpathlineto{\pgfqpoint{5.430733in}{1.753310in}}%
\pgfpathlineto{\pgfqpoint{5.438150in}{1.755758in}}%
\pgfpathlineto{\pgfqpoint{5.459032in}{1.761687in}}%
\pgfpathlineto{\pgfqpoint{5.465261in}{1.763580in}}%
\pgfpathlineto{\pgfqpoint{5.485331in}{1.770267in}}%
\pgfpathlineto{\pgfqpoint{5.518870in}{1.780770in}}%
\pgfpathlineto{\pgfqpoint{5.522947in}{1.782075in}}%
\pgfpathlineto{\pgfqpoint{5.537424in}{1.786760in}}%
\pgfpathlineto{\pgfqpoint{5.542971in}{1.787668in}}%
\pgfpathlineto{\pgfqpoint{5.549255in}{1.789657in}}%
\pgfpathlineto{\pgfqpoint{5.553930in}{1.791037in}}%
\pgfpathlineto{\pgfqpoint{5.580570in}{1.799782in}}%
\pgfpathlineto{\pgfqpoint{5.587992in}{1.802328in}}%
\pgfpathlineto{\pgfqpoint{5.615472in}{1.810782in}}%
\pgfpathlineto{\pgfqpoint{5.617594in}{1.810897in}}%
\pgfpathlineto{\pgfqpoint{5.623221in}{1.812565in}}%
\pgfpathlineto{\pgfqpoint{5.627411in}{1.813759in}}%
\pgfpathlineto{\pgfqpoint{5.646626in}{1.819923in}}%
\pgfpathlineto{\pgfqpoint{5.652687in}{1.822527in}}%
\pgfpathlineto{\pgfqpoint{5.658695in}{1.824426in}}%
\pgfpathlineto{\pgfqpoint{5.665307in}{1.826128in}}%
\pgfpathlineto{\pgfqpoint{5.669244in}{1.827519in}}%
\pgfpathlineto{\pgfqpoint{5.678986in}{1.830939in}}%
\pgfpathlineto{\pgfqpoint{5.697430in}{1.835807in}}%
\pgfpathlineto{\pgfqpoint{5.701807in}{1.837529in}}%
\pgfpathlineto{\pgfqpoint{5.722068in}{1.844395in}}%
\pgfpathlineto{\pgfqpoint{5.728688in}{1.845831in}}%
\pgfpathlineto{\pgfqpoint{5.732272in}{1.847337in}}%
\pgfpathlineto{\pgfqpoint{5.738203in}{1.849341in}}%
\pgfpathlineto{\pgfqpoint{5.742910in}{1.850683in}}%
\pgfpathlineto{\pgfqpoint{5.745835in}{1.851165in}}%
\pgfpathlineto{\pgfqpoint{5.749329in}{1.852049in}}%
\pgfpathlineto{\pgfqpoint{5.776646in}{1.860705in}}%
\pgfpathlineto{\pgfqpoint{5.780539in}{1.862226in}}%
\pgfpathlineto{\pgfqpoint{5.792084in}{1.865512in}}%
\pgfpathlineto{\pgfqpoint{5.794804in}{1.866629in}}%
\pgfpathlineto{\pgfqpoint{5.806108in}{1.869476in}}%
\pgfpathlineto{\pgfqpoint{5.809835in}{1.870446in}}%
\pgfpathlineto{\pgfqpoint{5.813013in}{1.871819in}}%
\pgfpathlineto{\pgfqpoint{5.817226in}{1.873008in}}%
\pgfpathlineto{\pgfqpoint{5.820892in}{1.873906in}}%
\pgfpathlineto{\pgfqpoint{5.824018in}{1.874763in}}%
\pgfpathlineto{\pgfqpoint{5.829196in}{1.876692in}}%
\pgfpathlineto{\pgfqpoint{5.841970in}{1.880815in}}%
\pgfpathlineto{\pgfqpoint{5.845504in}{1.882184in}}%
\pgfpathlineto{\pgfqpoint{5.851019in}{1.883818in}}%
\pgfpathlineto{\pgfqpoint{5.853512in}{1.884259in}}%
\pgfpathlineto{\pgfqpoint{5.855995in}{1.884634in}}%
\pgfpathlineto{\pgfqpoint{5.870219in}{1.888887in}}%
\pgfpathlineto{\pgfqpoint{5.873124in}{1.889749in}}%
\pgfpathlineto{\pgfqpoint{5.881767in}{1.893095in}}%
\pgfpathlineto{\pgfqpoint{5.898261in}{1.898248in}}%
\pgfpathlineto{\pgfqpoint{5.900584in}{1.898506in}}%
\pgfpathlineto{\pgfqpoint{5.902438in}{1.898538in}}%
\pgfpathlineto{\pgfqpoint{5.932300in}{1.908398in}}%
\pgfpathlineto{\pgfqpoint{5.941967in}{1.911828in}}%
\pgfpathlineto{\pgfqpoint{5.945882in}{1.912883in}}%
\pgfpathlineto{\pgfqpoint{5.949774in}{1.913403in}}%
\pgfpathlineto{\pgfqpoint{5.953644in}{1.914457in}}%
\pgfpathlineto{\pgfqpoint{5.957492in}{1.916318in}}%
\pgfpathlineto{\pgfqpoint{5.963434in}{1.917881in}}%
\pgfpathlineto{\pgfqpoint{5.968067in}{1.919555in}}%
\pgfpathlineto{\pgfqpoint{5.981364in}{1.924081in}}%
\pgfpathlineto{\pgfqpoint{5.983828in}{1.924877in}}%
\pgfpathlineto{\pgfqpoint{5.987916in}{1.926036in}}%
\pgfpathlineto{\pgfqpoint{5.991573in}{1.926438in}}%
\pgfpathlineto{\pgfqpoint{5.997625in}{1.928497in}}%
\pgfpathlineto{\pgfqpoint{6.015069in}{1.934505in}}%
\pgfpathlineto{\pgfqpoint{6.018193in}{1.935223in}}%
\pgfpathlineto{\pgfqpoint{6.035878in}{1.940450in}}%
\pgfpathlineto{\pgfqpoint{6.038529in}{1.941391in}}%
\pgfpathlineto{\pgfqpoint{6.069913in}{1.950775in}}%
\pgfpathlineto{\pgfqpoint{6.072434in}{1.951633in}}%
\pgfpathlineto{\pgfqpoint{6.075660in}{1.953006in}}%
\pgfpathlineto{\pgfqpoint{6.078160in}{1.953508in}}%
\pgfpathlineto{\pgfqpoint{6.081004in}{1.954721in}}%
\pgfpathlineto{\pgfqpoint{6.084544in}{1.956512in}}%
\pgfpathlineto{\pgfqpoint{6.087713in}{1.957554in}}%
\pgfpathlineto{\pgfqpoint{6.089818in}{1.957861in}}%
\pgfpathlineto{\pgfqpoint{6.094356in}{1.959567in}}%
\pgfpathlineto{\pgfqpoint{6.097479in}{1.960809in}}%
\pgfpathlineto{\pgfqpoint{6.105397in}{1.961883in}}%
\pgfpathlineto{\pgfqpoint{6.112207in}{1.964551in}}%
\pgfpathlineto{\pgfqpoint{6.113560in}{1.965059in}}%
\pgfpathlineto{\pgfqpoint{6.119284in}{1.968015in}}%
\pgfpathlineto{\pgfqpoint{6.126950in}{1.970348in}}%
\pgfpathlineto{\pgfqpoint{6.134531in}{1.972318in}}%
\pgfpathlineto{\pgfqpoint{6.136821in}{1.972738in}}%
\pgfpathlineto{\pgfqpoint{6.169649in}{1.983884in}}%
\pgfpathlineto{\pgfqpoint{6.171202in}{1.984158in}}%
\pgfpathlineto{\pgfqpoint{6.173681in}{1.985355in}}%
\pgfpathlineto{\pgfqpoint{6.176766in}{1.986332in}}%
\pgfpathlineto{\pgfqpoint{6.177996in}{1.986749in}}%
\pgfpathlineto{\pgfqpoint{6.182589in}{1.989081in}}%
\pgfpathlineto{\pgfqpoint{6.186847in}{1.990530in}}%
\pgfpathlineto{\pgfqpoint{6.189873in}{1.991447in}}%
\pgfpathlineto{\pgfqpoint{6.194085in}{1.993247in}}%
\pgfpathlineto{\pgfqpoint{6.211263in}{1.998386in}}%
\pgfpathlineto{\pgfqpoint{6.215345in}{2.000194in}}%
\pgfpathlineto{\pgfqpoint{6.217376in}{2.000495in}}%
\pgfpathlineto{\pgfqpoint{6.219113in}{2.001067in}}%
\pgfpathlineto{\pgfqpoint{6.225155in}{2.003314in}}%
\pgfpathlineto{\pgfqpoint{6.227157in}{2.003341in}}%
\pgfpathlineto{\pgfqpoint{6.231144in}{2.003851in}}%
\pgfpathlineto{\pgfqpoint{6.233694in}{2.005412in}}%
\pgfpathlineto{\pgfqpoint{6.236798in}{2.006291in}}%
\pgfpathlineto{\pgfqpoint{6.240728in}{2.008142in}}%
\pgfpathlineto{\pgfqpoint{6.245748in}{2.009671in}}%
\pgfpathlineto{\pgfqpoint{6.247413in}{2.010169in}}%
\pgfpathlineto{\pgfqpoint{6.252383in}{2.012048in}}%
\pgfpathlineto{\pgfqpoint{6.260585in}{2.014068in}}%
\pgfpathlineto{\pgfqpoint{6.263027in}{2.014847in}}%
\pgfpathlineto{\pgfqpoint{6.265729in}{2.015551in}}%
\pgfpathlineto{\pgfqpoint{6.269226in}{2.017471in}}%
\pgfpathlineto{\pgfqpoint{6.272972in}{2.018469in}}%
\pgfpathlineto{\pgfqpoint{6.274571in}{2.018901in}}%
\pgfpathlineto{\pgfqpoint{6.280929in}{2.020776in}}%
\pgfpathlineto{\pgfqpoint{6.282772in}{2.020628in}}%
\pgfpathlineto{\pgfqpoint{6.307046in}{2.028435in}}%
\pgfpathlineto{\pgfqpoint{6.310588in}{2.028899in}}%
\pgfpathlineto{\pgfqpoint{6.314111in}{2.030325in}}%
\pgfpathlineto{\pgfqpoint{6.315615in}{2.030801in}}%
\pgfpathlineto{\pgfqpoint{6.318364in}{2.032047in}}%
\pgfpathlineto{\pgfqpoint{6.319610in}{2.032656in}}%
\pgfpathlineto{\pgfqpoint{6.322839in}{2.033888in}}%
\pgfpathlineto{\pgfqpoint{6.337785in}{2.037744in}}%
\pgfpathlineto{\pgfqpoint{6.351932in}{2.042716in}}%
\pgfpathlineto{\pgfqpoint{6.353117in}{2.043096in}}%
\pgfpathlineto{\pgfqpoint{6.355482in}{2.044065in}}%
\pgfpathlineto{\pgfqpoint{6.357132in}{2.043649in}}%
\pgfpathlineto{\pgfqpoint{6.368803in}{2.047886in}}%
\pgfpathlineto{\pgfqpoint{6.372035in}{2.049052in}}%
\pgfpathlineto{\pgfqpoint{6.373645in}{2.049603in}}%
\pgfpathlineto{\pgfqpoint{6.379136in}{2.050703in}}%
\pgfpathlineto{\pgfqpoint{6.381184in}{2.051064in}}%
\pgfpathlineto{\pgfqpoint{6.385035in}{2.052453in}}%
\pgfpathlineto{\pgfqpoint{6.388640in}{2.053733in}}%
\pgfpathlineto{\pgfqpoint{6.389987in}{2.053872in}}%
\pgfpathlineto{\pgfqpoint{6.394234in}{2.055730in}}%
\pgfpathlineto{\pgfqpoint{6.396236in}{2.056381in}}%
\pgfpathlineto{\pgfqpoint{6.404187in}{2.057541in}}%
\pgfpathlineto{\pgfqpoint{6.405722in}{2.058381in}}%
\pgfpathlineto{\pgfqpoint{6.408782in}{2.059130in}}%
\pgfpathlineto{\pgfqpoint{6.410741in}{2.060205in}}%
\pgfpathlineto{\pgfqpoint{6.412262in}{2.060617in}}%
\pgfpathlineto{\pgfqpoint{6.416802in}{2.062484in}}%
\pgfpathlineto{\pgfqpoint{6.418524in}{2.062961in}}%
\pgfpathlineto{\pgfqpoint{6.475572in}{2.079222in}}%
\pgfpathlineto{\pgfqpoint{6.477150in}{2.080135in}}%
\pgfpathlineto{\pgfqpoint{6.479510in}{2.080467in}}%
\pgfpathlineto{\pgfqpoint{6.514136in}{2.092148in}}%
\pgfpathlineto{\pgfqpoint{6.515440in}{2.092580in}}%
\pgfpathlineto{\pgfqpoint{6.517299in}{2.092479in}}%
\pgfpathlineto{\pgfqpoint{6.519152in}{2.092870in}}%
\pgfpathlineto{\pgfqpoint{6.520077in}{2.092963in}}%
\pgfpathlineto{\pgfqpoint{6.520262in}{2.093248in}}%
\pgfpathlineto{\pgfqpoint{6.521923in}{2.094245in}}%
\pgfpathlineto{\pgfqpoint{6.522291in}{2.093806in}}%
\pgfpathlineto{\pgfqpoint{6.525049in}{2.094114in}}%
\pgfpathlineto{\pgfqpoint{6.527064in}{2.095345in}}%
\pgfpathlineto{\pgfqpoint{6.530349in}{2.096927in}}%
\pgfpathlineto{\pgfqpoint{6.530713in}{2.096668in}}%
\pgfpathlineto{\pgfqpoint{6.531440in}{2.097211in}}%
\pgfpathlineto{\pgfqpoint{6.535788in}{2.098513in}}%
\pgfpathlineto{\pgfqpoint{6.537050in}{2.098731in}}%
\pgfpathlineto{\pgfqpoint{6.538671in}{2.099513in}}%
\pgfpathlineto{\pgfqpoint{6.539928in}{2.100138in}}%
\pgfpathlineto{\pgfqpoint{6.541362in}{2.100056in}}%
\pgfpathlineto{\pgfqpoint{6.550434in}{2.103519in}}%
\pgfpathlineto{\pgfqpoint{6.550964in}{2.104543in}}%
\pgfpathlineto{\pgfqpoint{6.551669in}{2.104323in}}%
\pgfpathlineto{\pgfqpoint{6.553431in}{2.104227in}}%
\pgfpathlineto{\pgfqpoint{6.558163in}{2.105831in}}%
\pgfpathlineto{\pgfqpoint{6.559559in}{2.105789in}}%
\pgfpathlineto{\pgfqpoint{6.561474in}{2.106198in}}%
\pgfpathlineto{\pgfqpoint{6.563036in}{2.106924in}}%
\pgfpathlineto{\pgfqpoint{6.564422in}{2.107370in}}%
\pgfpathlineto{\pgfqpoint{6.567874in}{2.109399in}}%
\pgfpathlineto{\pgfqpoint{6.568390in}{2.110566in}}%
\pgfpathlineto{\pgfqpoint{6.569078in}{2.110338in}}%
\pgfpathlineto{\pgfqpoint{6.570623in}{2.110155in}}%
\pgfpathlineto{\pgfqpoint{6.572506in}{2.110885in}}%
\pgfpathlineto{\pgfqpoint{6.573873in}{2.111213in}}%
\pgfpathlineto{\pgfqpoint{6.578634in}{2.111872in}}%
\pgfpathlineto{\pgfqpoint{6.580157in}{2.112620in}}%
\pgfpathlineto{\pgfqpoint{6.583361in}{2.113910in}}%
\pgfpathlineto{\pgfqpoint{6.587387in}{2.116166in}}%
\pgfpathlineto{\pgfqpoint{6.588557in}{2.116625in}}%
\pgfpathlineto{\pgfqpoint{6.590890in}{2.117380in}}%
\pgfpathlineto{\pgfqpoint{6.591056in}{2.117061in}}%
\pgfpathlineto{\pgfqpoint{6.591555in}{2.117607in}}%
\pgfpathlineto{\pgfqpoint{6.591555in}{2.117607in}}%
\pgfpathlineto{\pgfqpoint{6.592718in}{2.117558in}}%
\pgfpathlineto{\pgfqpoint{6.594375in}{2.117055in}}%
\pgfpathlineto{\pgfqpoint{6.596523in}{2.118472in}}%
\pgfpathlineto{\pgfqpoint{6.598994in}{2.119406in}}%
\pgfpathlineto{\pgfqpoint{6.601455in}{2.121031in}}%
\pgfpathlineto{\pgfqpoint{6.601946in}{2.122079in}}%
\pgfpathlineto{\pgfqpoint{6.602600in}{2.121474in}}%
\pgfpathlineto{\pgfqpoint{6.604233in}{2.121620in}}%
\pgfpathlineto{\pgfqpoint{6.607001in}{2.122978in}}%
\pgfpathlineto{\pgfqpoint{6.607650in}{2.122634in}}%
\pgfpathlineto{\pgfqpoint{6.608137in}{2.123397in}}%
\pgfpathlineto{\pgfqpoint{6.608299in}{2.123608in}}%
\pgfpathlineto{\pgfqpoint{6.608623in}{2.123156in}}%
\pgfpathlineto{\pgfqpoint{6.608623in}{2.123156in}}%
\pgfpathlineto{\pgfqpoint{6.609756in}{2.122990in}}%
\pgfpathlineto{\pgfqpoint{6.625111in}{2.128013in}}%
\pgfpathlineto{\pgfqpoint{6.627006in}{2.127666in}}%
\pgfpathlineto{\pgfqpoint{6.629681in}{2.129842in}}%
\pgfpathlineto{\pgfqpoint{6.629838in}{2.129599in}}%
\pgfpathlineto{\pgfqpoint{6.630779in}{2.129583in}}%
\pgfpathlineto{\pgfqpoint{6.630936in}{2.129777in}}%
\pgfpathlineto{\pgfqpoint{6.632658in}{2.131126in}}%
\pgfpathlineto{\pgfqpoint{6.633908in}{2.132723in}}%
\pgfpathlineto{\pgfqpoint{6.634376in}{2.131801in}}%
\pgfpathlineto{\pgfqpoint{6.640277in}{2.133117in}}%
\pgfpathlineto{\pgfqpoint{6.641050in}{2.132653in}}%
\pgfpathlineto{\pgfqpoint{6.641358in}{2.133099in}}%
\pgfpathlineto{\pgfqpoint{6.642900in}{2.132895in}}%
\pgfpathlineto{\pgfqpoint{6.643054in}{2.133103in}}%
\pgfpathlineto{\pgfqpoint{6.645206in}{2.134478in}}%
\pgfpathlineto{\pgfqpoint{6.645360in}{2.134181in}}%
\pgfpathlineto{\pgfqpoint{6.645667in}{2.133964in}}%
\pgfpathlineto{\pgfqpoint{6.646433in}{2.134449in}}%
\pgfpathlineto{\pgfqpoint{6.649185in}{2.136144in}}%
\pgfpathlineto{\pgfqpoint{6.650405in}{2.136674in}}%
\pgfpathlineto{\pgfqpoint{6.653292in}{2.136863in}}%
\pgfpathlineto{\pgfqpoint{6.654958in}{2.137517in}}%
\pgfpathlineto{\pgfqpoint{6.655260in}{2.138043in}}%
\pgfpathlineto{\pgfqpoint{6.655412in}{2.137449in}}%
\pgfpathlineto{\pgfqpoint{6.655412in}{2.137449in}}%
\pgfpathlineto{\pgfqpoint{6.656922in}{2.136413in}}%
\pgfpathlineto{\pgfqpoint{6.659781in}{2.138675in}}%
\pgfpathlineto{\pgfqpoint{6.664272in}{2.140814in}}%
\pgfpathlineto{\pgfqpoint{6.664421in}{2.141100in}}%
\pgfpathlineto{\pgfqpoint{6.664421in}{2.141100in}}%
\pgfpathlineto{\pgfqpoint{6.664421in}{2.141100in}}%
\pgfpathlineto{\pgfqpoint{6.664720in}{2.140339in}}%
\pgfpathlineto{\pgfqpoint{6.665465in}{2.140970in}}%
\pgfpathlineto{\pgfqpoint{6.670213in}{2.141699in}}%
\pgfpathlineto{\pgfqpoint{6.670361in}{2.140978in}}%
\pgfpathlineto{\pgfqpoint{6.671248in}{2.140211in}}%
\pgfpathlineto{\pgfqpoint{6.671543in}{2.140615in}}%
\pgfpathlineto{\pgfqpoint{6.673312in}{2.140832in}}%
\pgfpathlineto{\pgfqpoint{6.674782in}{2.142776in}}%
\pgfpathlineto{\pgfqpoint{6.677275in}{2.142176in}}%
\pgfpathlineto{\pgfqpoint{6.677421in}{2.142559in}}%
\pgfpathlineto{\pgfqpoint{6.682958in}{2.147154in}}%
\pgfpathlineto{\pgfqpoint{6.683248in}{2.146174in}}%
\pgfpathlineto{\pgfqpoint{6.686864in}{2.144579in}}%
\pgfpathlineto{\pgfqpoint{6.687009in}{2.144204in}}%
\pgfpathlineto{\pgfqpoint{6.687297in}{2.144712in}}%
\pgfpathlineto{\pgfqpoint{6.687297in}{2.144712in}}%
\pgfpathlineto{\pgfqpoint{6.689312in}{2.147332in}}%
\pgfpathlineto{\pgfqpoint{6.689744in}{2.146980in}}%
\pgfpathlineto{\pgfqpoint{6.690174in}{2.146246in}}%
\pgfpathlineto{\pgfqpoint{6.690892in}{2.146993in}}%
\pgfpathlineto{\pgfqpoint{6.692754in}{2.147797in}}%
\pgfpathlineto{\pgfqpoint{6.696035in}{2.151148in}}%
\pgfpathlineto{\pgfqpoint{6.696177in}{2.150874in}}%
\pgfpathlineto{\pgfqpoint{6.696319in}{2.150398in}}%
\pgfpathlineto{\pgfqpoint{6.696462in}{2.150930in}}%
\pgfpathlineto{\pgfqpoint{6.696462in}{2.150930in}}%
\pgfpathlineto{\pgfqpoint{6.697030in}{2.152528in}}%
\pgfpathlineto{\pgfqpoint{6.697598in}{2.150817in}}%
\pgfpathlineto{\pgfqpoint{6.702127in}{2.150412in}}%
\pgfpathlineto{\pgfqpoint{6.703395in}{2.151676in}}%
\pgfpathlineto{\pgfqpoint{6.703536in}{2.151358in}}%
\pgfpathlineto{\pgfqpoint{6.704942in}{2.150389in}}%
\pgfpathlineto{\pgfqpoint{6.707045in}{2.152276in}}%
\pgfpathlineto{\pgfqpoint{6.708444in}{2.153634in}}%
\pgfpathlineto{\pgfqpoint{6.708723in}{2.152846in}}%
\pgfpathlineto{\pgfqpoint{6.709002in}{2.153261in}}%
\pgfpathlineto{\pgfqpoint{6.709002in}{2.153261in}}%
\pgfpathlineto{\pgfqpoint{6.709281in}{2.154393in}}%
\pgfpathlineto{\pgfqpoint{6.709421in}{2.153753in}}%
\pgfpathlineto{\pgfqpoint{6.709421in}{2.153753in}}%
\pgfpathlineto{\pgfqpoint{6.709700in}{2.152917in}}%
\pgfpathlineto{\pgfqpoint{6.709979in}{2.153173in}}%
\pgfpathlineto{\pgfqpoint{6.709979in}{2.153173in}}%
\pgfpathlineto{\pgfqpoint{6.711371in}{2.155587in}}%
\pgfpathlineto{\pgfqpoint{6.712483in}{2.153423in}}%
\pgfpathlineto{\pgfqpoint{6.712761in}{2.153998in}}%
\pgfpathlineto{\pgfqpoint{6.713038in}{2.155324in}}%
\pgfpathlineto{\pgfqpoint{6.714009in}{2.154639in}}%
\pgfpathlineto{\pgfqpoint{6.714286in}{2.155186in}}%
\pgfpathlineto{\pgfqpoint{6.714563in}{2.154536in}}%
\pgfpathlineto{\pgfqpoint{6.714563in}{2.154536in}}%
\pgfpathlineto{\pgfqpoint{6.715532in}{2.154008in}}%
\pgfpathlineto{\pgfqpoint{6.715670in}{2.154439in}}%
\pgfpathlineto{\pgfqpoint{6.716775in}{2.157093in}}%
\pgfpathlineto{\pgfqpoint{6.717189in}{2.156769in}}%
\pgfpathlineto{\pgfqpoint{6.717327in}{2.157056in}}%
\pgfpathlineto{\pgfqpoint{6.717464in}{2.156639in}}%
\pgfpathlineto{\pgfqpoint{6.717464in}{2.156639in}}%
\pgfpathlineto{\pgfqpoint{6.718291in}{2.154857in}}%
\pgfpathlineto{\pgfqpoint{6.718979in}{2.155261in}}%
\pgfpathlineto{\pgfqpoint{6.719117in}{2.154968in}}%
\pgfpathlineto{\pgfqpoint{6.719254in}{2.155293in}}%
\pgfpathlineto{\pgfqpoint{6.719254in}{2.155293in}}%
\pgfpathlineto{\pgfqpoint{6.719941in}{2.157962in}}%
\pgfpathlineto{\pgfqpoint{6.720628in}{2.156795in}}%
\pgfpathlineto{\pgfqpoint{6.722546in}{2.160032in}}%
\pgfpathlineto{\pgfqpoint{6.722819in}{2.160107in}}%
\pgfpathlineto{\pgfqpoint{6.722956in}{2.161124in}}%
\pgfpathlineto{\pgfqpoint{6.723230in}{2.160185in}}%
\pgfpathlineto{\pgfqpoint{6.723230in}{2.160185in}}%
\pgfpathlineto{\pgfqpoint{6.723503in}{2.158690in}}%
\pgfpathlineto{\pgfqpoint{6.723776in}{2.159416in}}%
\pgfpathlineto{\pgfqpoint{6.723776in}{2.159416in}}%
\pgfpathlineto{\pgfqpoint{6.723913in}{2.160772in}}%
\pgfpathlineto{\pgfqpoint{6.724186in}{2.159644in}}%
\pgfpathlineto{\pgfqpoint{6.724186in}{2.159644in}}%
\pgfpathlineto{\pgfqpoint{6.724595in}{2.159205in}}%
\pgfpathlineto{\pgfqpoint{6.725004in}{2.159743in}}%
\pgfpathlineto{\pgfqpoint{6.725004in}{2.159743in}}%
\pgfpathlineto{\pgfqpoint{6.725140in}{2.160135in}}%
\pgfpathlineto{\pgfqpoint{6.725140in}{2.160135in}}%
\pgfpathlineto{\pgfqpoint{6.725140in}{2.160135in}}%
\pgfpathlineto{\pgfqpoint{6.725277in}{2.159035in}}%
\pgfpathlineto{\pgfqpoint{6.725413in}{2.159853in}}%
\pgfpathlineto{\pgfqpoint{6.725413in}{2.159853in}}%
\pgfpathlineto{\pgfqpoint{6.725822in}{2.160487in}}%
\pgfpathlineto{\pgfqpoint{6.726094in}{2.159883in}}%
\pgfpathlineto{\pgfqpoint{6.726094in}{2.159883in}}%
\pgfpathlineto{\pgfqpoint{6.726230in}{2.158965in}}%
\pgfpathlineto{\pgfqpoint{6.726366in}{2.159126in}}%
\pgfpathlineto{\pgfqpoint{6.726366in}{2.159126in}}%
\pgfpathlineto{\pgfqpoint{6.726774in}{2.162268in}}%
\pgfpathlineto{\pgfqpoint{6.727454in}{2.159415in}}%
\pgfpathlineto{\pgfqpoint{6.728404in}{2.157636in}}%
\pgfpathlineto{\pgfqpoint{6.728539in}{2.158889in}}%
\pgfpathlineto{\pgfqpoint{6.730705in}{2.164150in}}%
\pgfpathlineto{\pgfqpoint{6.731516in}{2.160899in}}%
\pgfpathlineto{\pgfqpoint{6.732056in}{2.162660in}}%
\pgfpathlineto{\pgfqpoint{6.732191in}{2.163222in}}%
\pgfpathlineto{\pgfqpoint{6.732595in}{2.162848in}}%
\pgfpathlineto{\pgfqpoint{6.732595in}{2.162848in}}%
\pgfpathlineto{\pgfqpoint{6.732730in}{2.162368in}}%
\pgfpathlineto{\pgfqpoint{6.732865in}{2.162859in}}%
\pgfpathlineto{\pgfqpoint{6.732865in}{2.162859in}}%
\pgfpathlineto{\pgfqpoint{6.733269in}{2.163804in}}%
\pgfpathlineto{\pgfqpoint{6.733404in}{2.163490in}}%
\pgfusepath{stroke}%
\end{pgfscope}%
\begin{pgfscope}%
\pgfpathrectangle{\pgfqpoint{1.000000in}{0.300000in}}{\pgfqpoint{6.200000in}{2.400000in}} %
\pgfusepath{clip}%
\pgfsetrectcap%
\pgfsetroundjoin%
\pgfsetlinewidth{1.003750pt}%
\definecolor{currentstroke}{rgb}{1.000000,0.000000,0.000000}%
\pgfsetstrokecolor{currentstroke}%
\pgfsetdash{}{0pt}%
\pgfpathmoveto{\pgfqpoint{1.000000in}{0.497039in}}%
\pgfpathlineto{\pgfqpoint{1.466596in}{0.692308in}}%
\pgfpathlineto{\pgfqpoint{1.739538in}{0.795276in}}%
\pgfpathlineto{\pgfqpoint{1.933193in}{0.858682in}}%
\pgfpathlineto{\pgfqpoint{2.309902in}{0.971959in}}%
\pgfpathlineto{\pgfqpoint{2.866386in}{1.132900in}}%
\pgfpathlineto{\pgfqpoint{2.945672in}{1.155466in}}%
\pgfpathlineto{\pgfqpoint{3.049440in}{1.187056in}}%
\pgfpathlineto{\pgfqpoint{3.218614in}{1.238008in}}%
\pgfpathlineto{\pgfqpoint{3.332982in}{1.269423in}}%
\pgfpathlineto{\pgfqpoint{3.393305in}{1.287605in}}%
\pgfpathlineto{\pgfqpoint{3.448665in}{1.304469in}}%
\pgfpathlineto{\pgfqpoint{3.799579in}{1.406377in}}%
\pgfpathlineto{\pgfqpoint{3.897309in}{1.435879in}}%
\pgfpathlineto{\pgfqpoint{3.941322in}{1.448895in}}%
\pgfpathlineto{\pgfqpoint{4.043871in}{1.479495in}}%
\pgfpathlineto{\pgfqpoint{4.086400in}{1.493359in}}%
\pgfpathlineto{\pgfqpoint{4.119898in}{1.502007in}}%
\pgfpathlineto{\pgfqpoint{4.524945in}{1.625851in}}%
\pgfpathlineto{\pgfqpoint{4.535602in}{1.628961in}}%
\pgfpathlineto{\pgfqpoint{4.552997in}{1.634833in}}%
\pgfpathlineto{\pgfqpoint{4.563222in}{1.637750in}}%
\pgfpathlineto{\pgfqpoint{4.579927in}{1.641933in}}%
\pgfpathlineto{\pgfqpoint{4.998665in}{1.769457in}}%
\pgfpathlineto{\pgfqpoint{5.007464in}{1.771740in}}%
\pgfpathlineto{\pgfqpoint{5.021308in}{1.775907in}}%
\pgfpathlineto{\pgfqpoint{5.049815in}{1.784149in}}%
\pgfpathlineto{\pgfqpoint{5.081876in}{1.794044in}}%
\pgfpathlineto{\pgfqpoint{5.091204in}{1.796726in}}%
\pgfpathlineto{\pgfqpoint{5.130195in}{1.809168in}}%
\pgfpathlineto{\pgfqpoint{5.241415in}{1.842110in}}%
\pgfpathlineto{\pgfqpoint{5.264483in}{1.849405in}}%
\pgfpathlineto{\pgfqpoint{5.301632in}{1.860818in}}%
\pgfpathlineto{\pgfqpoint{5.305012in}{1.861593in}}%
\pgfpathlineto{\pgfqpoint{5.312833in}{1.863141in}}%
\pgfpathlineto{\pgfqpoint{5.334690in}{1.869678in}}%
\pgfpathlineto{\pgfqpoint{5.358979in}{1.877564in}}%
\pgfpathlineto{\pgfqpoint{5.365174in}{1.879731in}}%
\pgfpathlineto{\pgfqpoint{5.373346in}{1.882373in}}%
\pgfpathlineto{\pgfqpoint{5.380416in}{1.884268in}}%
\pgfpathlineto{\pgfqpoint{5.391378in}{1.887810in}}%
\pgfpathlineto{\pgfqpoint{5.400216in}{1.890299in}}%
\pgfpathlineto{\pgfqpoint{5.407976in}{1.892807in}}%
\pgfpathlineto{\pgfqpoint{5.411823in}{1.893105in}}%
\pgfpathlineto{\pgfqpoint{5.456345in}{1.906656in}}%
\pgfpathlineto{\pgfqpoint{5.460818in}{1.908371in}}%
\pgfpathlineto{\pgfqpoint{5.467030in}{1.910269in}}%
\pgfpathlineto{\pgfqpoint{5.619005in}{1.957172in}}%
\pgfpathlineto{\pgfqpoint{5.624621in}{1.958803in}}%
\pgfpathlineto{\pgfqpoint{5.629496in}{1.960127in}}%
\pgfpathlineto{\pgfqpoint{5.649327in}{1.966088in}}%
\pgfpathlineto{\pgfqpoint{5.653357in}{1.966965in}}%
\pgfpathlineto{\pgfqpoint{5.673808in}{1.973736in}}%
\pgfpathlineto{\pgfqpoint{5.679630in}{1.976203in}}%
\pgfpathlineto{\pgfqpoint{5.712934in}{1.986445in}}%
\pgfpathlineto{\pgfqpoint{5.715992in}{1.986608in}}%
\pgfpathlineto{\pgfqpoint{5.735836in}{1.993214in}}%
\pgfpathlineto{\pgfqpoint{5.741149in}{1.995296in}}%
\pgfpathlineto{\pgfqpoint{5.769357in}{2.003805in}}%
\pgfpathlineto{\pgfqpoint{5.785511in}{2.008552in}}%
\pgfpathlineto{\pgfqpoint{5.793717in}{2.011597in}}%
\pgfpathlineto{\pgfqpoint{5.815123in}{2.018426in}}%
\pgfpathlineto{\pgfqpoint{5.818276in}{2.019400in}}%
\pgfpathlineto{\pgfqpoint{5.830227in}{2.022609in}}%
\pgfpathlineto{\pgfqpoint{5.841970in}{2.027038in}}%
\pgfpathlineto{\pgfqpoint{5.846007in}{2.028141in}}%
\pgfpathlineto{\pgfqpoint{5.876017in}{2.037078in}}%
\pgfpathlineto{\pgfqpoint{5.879377in}{2.037980in}}%
\pgfpathlineto{\pgfqpoint{5.888885in}{2.041443in}}%
\pgfpathlineto{\pgfqpoint{5.906129in}{2.047084in}}%
\pgfpathlineto{\pgfqpoint{5.911628in}{2.048694in}}%
\pgfpathlineto{\pgfqpoint{5.917082in}{2.050500in}}%
\pgfpathlineto{\pgfqpoint{5.924734in}{2.052293in}}%
\pgfpathlineto{\pgfqpoint{5.945448in}{2.059634in}}%
\pgfpathlineto{\pgfqpoint{6.043049in}{2.090130in}}%
\pgfpathlineto{\pgfqpoint{6.047540in}{2.091141in}}%
\pgfpathlineto{\pgfqpoint{6.049774in}{2.092381in}}%
\pgfpathlineto{\pgfqpoint{6.052371in}{2.093284in}}%
\pgfpathlineto{\pgfqpoint{6.056063in}{2.094414in}}%
\pgfpathlineto{\pgfqpoint{6.064116in}{2.096893in}}%
\pgfpathlineto{\pgfqpoint{6.066296in}{2.097719in}}%
\pgfpathlineto{\pgfqpoint{6.069191in}{2.098393in}}%
\pgfpathlineto{\pgfqpoint{6.072434in}{2.099375in}}%
\pgfpathlineto{\pgfqpoint{6.085602in}{2.103313in}}%
\pgfpathlineto{\pgfqpoint{6.089117in}{2.104423in}}%
\pgfpathlineto{\pgfqpoint{6.094008in}{2.106086in}}%
\pgfpathlineto{\pgfqpoint{6.097133in}{2.107807in}}%
\pgfpathlineto{\pgfqpoint{6.174299in}{2.132652in}}%
\pgfpathlineto{\pgfqpoint{6.177073in}{2.133243in}}%
\pgfpathlineto{\pgfqpoint{6.179530in}{2.134044in}}%
\pgfpathlineto{\pgfqpoint{6.181978in}{2.134561in}}%
\pgfpathlineto{\pgfqpoint{6.191681in}{2.138286in}}%
\pgfpathlineto{\pgfqpoint{6.194685in}{2.139681in}}%
\pgfpathlineto{\pgfqpoint{6.197377in}{2.139982in}}%
\pgfpathlineto{\pgfqpoint{6.218245in}{2.146838in}}%
\pgfpathlineto{\pgfqpoint{6.318364in}{2.180251in}}%
\pgfpathlineto{\pgfqpoint{6.320357in}{2.180704in}}%
\pgfpathlineto{\pgfqpoint{6.322839in}{2.181591in}}%
\pgfpathlineto{\pgfqpoint{6.365090in}{2.195834in}}%
\pgfpathlineto{\pgfqpoint{6.367645in}{2.196672in}}%
\pgfpathlineto{\pgfqpoint{6.372955in}{2.198341in}}%
\pgfpathlineto{\pgfqpoint{6.377082in}{2.200538in}}%
\pgfpathlineto{\pgfqpoint{6.383452in}{2.202164in}}%
\pgfpathlineto{\pgfqpoint{6.385938in}{2.203395in}}%
\pgfpathlineto{\pgfqpoint{6.388190in}{2.203301in}}%
\pgfpathlineto{\pgfqpoint{6.396680in}{2.206281in}}%
\pgfpathlineto{\pgfqpoint{6.398454in}{2.207027in}}%
\pgfpathlineto{\pgfqpoint{6.400444in}{2.207682in}}%
\pgfpathlineto{\pgfqpoint{6.401767in}{2.208038in}}%
\pgfpathlineto{\pgfqpoint{6.409871in}{2.210441in}}%
\pgfpathlineto{\pgfqpoint{6.413562in}{2.211562in}}%
\pgfpathlineto{\pgfqpoint{6.430242in}{2.217153in}}%
\pgfpathlineto{\pgfqpoint{6.434663in}{2.217938in}}%
\pgfpathlineto{\pgfqpoint{6.436968in}{2.219095in}}%
\pgfpathlineto{\pgfqpoint{6.438847in}{2.219358in}}%
\pgfpathlineto{\pgfqpoint{6.449399in}{2.223862in}}%
\pgfpathlineto{\pgfqpoint{6.451040in}{2.223801in}}%
\pgfpathlineto{\pgfqpoint{6.455123in}{2.224347in}}%
\pgfpathlineto{\pgfqpoint{6.458372in}{2.225655in}}%
\pgfpathlineto{\pgfqpoint{6.468824in}{2.229473in}}%
\pgfpathlineto{\pgfqpoint{6.472405in}{2.230411in}}%
\pgfpathlineto{\pgfqpoint{6.473792in}{2.230417in}}%
\pgfpathlineto{\pgfqpoint{6.475374in}{2.231303in}}%
\pgfpathlineto{\pgfqpoint{6.477740in}{2.232369in}}%
\pgfpathlineto{\pgfqpoint{6.481861in}{2.234155in}}%
\pgfpathlineto{\pgfqpoint{6.483229in}{2.234577in}}%
\pgfpathlineto{\pgfqpoint{6.484400in}{2.234315in}}%
\pgfpathlineto{\pgfqpoint{6.484594in}{2.234523in}}%
\pgfpathlineto{\pgfqpoint{6.487899in}{2.235700in}}%
\pgfpathlineto{\pgfqpoint{6.488867in}{2.236107in}}%
\pgfpathlineto{\pgfqpoint{6.490607in}{2.236870in}}%
\pgfpathlineto{\pgfqpoint{6.495801in}{2.238429in}}%
\pgfpathlineto{\pgfqpoint{6.497715in}{2.238562in}}%
\pgfpathlineto{\pgfqpoint{6.500194in}{2.240128in}}%
\pgfpathlineto{\pgfqpoint{6.502285in}{2.240955in}}%
\pgfpathlineto{\pgfqpoint{6.506259in}{2.241595in}}%
\pgfpathlineto{\pgfqpoint{6.514323in}{2.244063in}}%
\pgfpathlineto{\pgfqpoint{6.515626in}{2.244475in}}%
\pgfpathlineto{\pgfqpoint{6.538131in}{2.251191in}}%
\pgfpathlineto{\pgfqpoint{6.539389in}{2.251686in}}%
\pgfpathlineto{\pgfqpoint{6.540466in}{2.251502in}}%
\pgfpathlineto{\pgfqpoint{6.540645in}{2.251699in}}%
\pgfpathlineto{\pgfqpoint{6.543151in}{2.252497in}}%
\pgfpathlineto{\pgfqpoint{6.544578in}{2.252838in}}%
\pgfpathlineto{\pgfqpoint{6.545824in}{2.253592in}}%
\pgfpathlineto{\pgfqpoint{6.546180in}{2.253302in}}%
\pgfpathlineto{\pgfqpoint{6.550610in}{2.254579in}}%
\pgfpathlineto{\pgfqpoint{6.551846in}{2.254795in}}%
\pgfpathlineto{\pgfqpoint{6.554134in}{2.255789in}}%
\pgfpathlineto{\pgfqpoint{6.568046in}{2.259637in}}%
\pgfpathlineto{\pgfqpoint{6.569250in}{2.259864in}}%
\pgfpathlineto{\pgfqpoint{6.572164in}{2.260993in}}%
\pgfpathlineto{\pgfqpoint{6.574385in}{2.261659in}}%
\pgfpathlineto{\pgfqpoint{6.574555in}{2.261326in}}%
\pgfpathlineto{\pgfqpoint{6.575747in}{2.261692in}}%
\pgfpathlineto{\pgfqpoint{6.576598in}{2.262053in}}%
\pgfpathlineto{\pgfqpoint{6.576937in}{2.261506in}}%
\pgfpathlineto{\pgfqpoint{6.579480in}{2.262356in}}%
\pgfpathlineto{\pgfqpoint{6.580495in}{2.262987in}}%
\pgfpathlineto{\pgfqpoint{6.580833in}{2.262700in}}%
\pgfpathlineto{\pgfqpoint{6.587053in}{2.264223in}}%
\pgfpathlineto{\pgfqpoint{6.588056in}{2.264886in}}%
\pgfpathlineto{\pgfqpoint{6.589391in}{2.265758in}}%
\pgfpathlineto{\pgfqpoint{6.589724in}{2.265532in}}%
\pgfpathlineto{\pgfqpoint{6.591555in}{2.265608in}}%
\pgfpathlineto{\pgfqpoint{6.592552in}{2.266069in}}%
\pgfpathlineto{\pgfqpoint{6.593381in}{2.266255in}}%
\pgfpathlineto{\pgfqpoint{6.593547in}{2.265880in}}%
\pgfpathlineto{\pgfqpoint{6.594209in}{2.265656in}}%
\pgfpathlineto{\pgfqpoint{6.594540in}{2.265951in}}%
\pgfpathlineto{\pgfqpoint{6.597348in}{2.267362in}}%
\pgfpathlineto{\pgfqpoint{6.598994in}{2.267561in}}%
\pgfpathlineto{\pgfqpoint{6.600307in}{2.268166in}}%
\pgfpathlineto{\pgfqpoint{6.602437in}{2.268418in}}%
\pgfpathlineto{\pgfqpoint{6.609756in}{2.270727in}}%
\pgfpathlineto{\pgfqpoint{6.609918in}{2.270273in}}%
\pgfpathlineto{\pgfqpoint{6.610242in}{2.269786in}}%
\pgfpathlineto{\pgfqpoint{6.610888in}{2.270219in}}%
\pgfpathlineto{\pgfqpoint{6.612984in}{2.271715in}}%
\pgfpathlineto{\pgfqpoint{6.614592in}{2.271477in}}%
\pgfpathlineto{\pgfqpoint{6.616837in}{2.272204in}}%
\pgfpathlineto{\pgfqpoint{6.618756in}{2.272739in}}%
\pgfpathlineto{\pgfqpoint{6.621146in}{2.273749in}}%
\pgfpathlineto{\pgfqpoint{6.622576in}{2.273532in}}%
\pgfpathlineto{\pgfqpoint{6.623686in}{2.274094in}}%
\pgfpathlineto{\pgfqpoint{6.623844in}{2.273952in}}%
\pgfpathlineto{\pgfqpoint{6.624953in}{2.274563in}}%
\pgfpathlineto{\pgfqpoint{6.625743in}{2.275000in}}%
\pgfpathlineto{\pgfqpoint{6.625901in}{2.274494in}}%
\pgfpathlineto{\pgfqpoint{6.626217in}{2.273954in}}%
\pgfpathlineto{\pgfqpoint{6.626848in}{2.274423in}}%
\pgfpathlineto{\pgfqpoint{6.628108in}{2.274722in}}%
\pgfpathlineto{\pgfqpoint{6.630152in}{2.275772in}}%
\pgfpathlineto{\pgfqpoint{6.630936in}{2.275640in}}%
\pgfpathlineto{\pgfqpoint{6.631093in}{2.275884in}}%
\pgfpathlineto{\pgfqpoint{6.631563in}{2.276792in}}%
\pgfpathlineto{\pgfqpoint{6.631876in}{2.276250in}}%
\pgfpathlineto{\pgfqpoint{6.631876in}{2.276250in}}%
\pgfpathlineto{\pgfqpoint{6.632033in}{2.275627in}}%
\pgfpathlineto{\pgfqpoint{6.632971in}{2.276252in}}%
\pgfpathlineto{\pgfqpoint{6.634064in}{2.276792in}}%
\pgfpathlineto{\pgfqpoint{6.634220in}{2.276519in}}%
\pgfpathlineto{\pgfqpoint{6.636090in}{2.276819in}}%
\pgfpathlineto{\pgfqpoint{6.637644in}{2.277894in}}%
\pgfpathlineto{\pgfqpoint{6.638264in}{2.277680in}}%
\pgfpathlineto{\pgfqpoint{6.638729in}{2.278328in}}%
\pgfpathlineto{\pgfqpoint{6.641358in}{2.279088in}}%
\pgfpathlineto{\pgfqpoint{6.641821in}{2.277942in}}%
\pgfpathlineto{\pgfqpoint{6.642592in}{2.278738in}}%
\pgfpathlineto{\pgfqpoint{6.645206in}{2.279635in}}%
\pgfpathlineto{\pgfqpoint{6.651622in}{2.281431in}}%
\pgfpathlineto{\pgfqpoint{6.654655in}{2.282137in}}%
\pgfpathlineto{\pgfqpoint{6.655563in}{2.281478in}}%
\pgfpathlineto{\pgfqpoint{6.655865in}{2.282233in}}%
\pgfpathlineto{\pgfqpoint{6.656620in}{2.282307in}}%
\pgfpathlineto{\pgfqpoint{6.656771in}{2.281878in}}%
\pgfpathlineto{\pgfqpoint{6.656922in}{2.281424in}}%
\pgfpathlineto{\pgfqpoint{6.657826in}{2.282217in}}%
\pgfpathlineto{\pgfqpoint{6.663376in}{2.283856in}}%
\pgfpathlineto{\pgfqpoint{6.665762in}{2.284350in}}%
\pgfpathlineto{\pgfqpoint{6.669178in}{2.285518in}}%
\pgfpathlineto{\pgfqpoint{6.672870in}{2.286023in}}%
\pgfpathlineto{\pgfqpoint{6.673017in}{2.286341in}}%
\pgfpathlineto{\pgfqpoint{6.673165in}{2.285980in}}%
\pgfpathlineto{\pgfqpoint{6.673165in}{2.285980in}}%
\pgfpathlineto{\pgfqpoint{6.673606in}{2.285751in}}%
\pgfpathlineto{\pgfqpoint{6.674194in}{2.286356in}}%
\pgfpathlineto{\pgfqpoint{6.676249in}{2.287045in}}%
\pgfpathlineto{\pgfqpoint{6.678298in}{2.286788in}}%
\pgfpathlineto{\pgfqpoint{6.678590in}{2.287506in}}%
\pgfpathlineto{\pgfqpoint{6.679320in}{2.287033in}}%
\pgfpathlineto{\pgfqpoint{6.679612in}{2.286041in}}%
\pgfpathlineto{\pgfqpoint{6.680486in}{2.286905in}}%
\pgfpathlineto{\pgfqpoint{6.681505in}{2.287784in}}%
\pgfpathlineto{\pgfqpoint{6.682232in}{2.288656in}}%
\pgfpathlineto{\pgfqpoint{6.682667in}{2.288533in}}%
\pgfpathlineto{\pgfqpoint{6.685131in}{2.287809in}}%
\pgfpathlineto{\pgfqpoint{6.685709in}{2.289285in}}%
\pgfpathlineto{\pgfqpoint{6.686287in}{2.288631in}}%
\pgfpathlineto{\pgfqpoint{6.686864in}{2.288141in}}%
\pgfpathlineto{\pgfqpoint{6.687297in}{2.288927in}}%
\pgfpathlineto{\pgfqpoint{6.687585in}{2.289597in}}%
\pgfpathlineto{\pgfqpoint{6.687873in}{2.288742in}}%
\pgfpathlineto{\pgfqpoint{6.687873in}{2.288742in}}%
\pgfpathlineto{\pgfqpoint{6.688593in}{2.288180in}}%
\pgfpathlineto{\pgfqpoint{6.688881in}{2.288912in}}%
\pgfpathlineto{\pgfqpoint{6.691608in}{2.290338in}}%
\pgfpathlineto{\pgfqpoint{6.691752in}{2.290147in}}%
\pgfpathlineto{\pgfqpoint{6.691895in}{2.289788in}}%
\pgfpathlineto{\pgfqpoint{6.692038in}{2.290214in}}%
\pgfpathlineto{\pgfqpoint{6.692038in}{2.290214in}}%
\pgfpathlineto{\pgfqpoint{6.693039in}{2.291058in}}%
\pgfpathlineto{\pgfqpoint{6.693325in}{2.290763in}}%
\pgfpathlineto{\pgfqpoint{6.694039in}{2.289694in}}%
\pgfpathlineto{\pgfqpoint{6.694467in}{2.291079in}}%
\pgfpathlineto{\pgfqpoint{6.695750in}{2.291208in}}%
\pgfpathlineto{\pgfqpoint{6.697314in}{2.291631in}}%
\pgfpathlineto{\pgfqpoint{6.698592in}{2.291839in}}%
\pgfpathlineto{\pgfqpoint{6.698733in}{2.291732in}}%
\pgfpathlineto{\pgfqpoint{6.699017in}{2.291252in}}%
\pgfpathlineto{\pgfqpoint{6.699442in}{2.291606in}}%
\pgfpathlineto{\pgfqpoint{6.700008in}{2.293288in}}%
\pgfpathlineto{\pgfqpoint{6.700574in}{2.291912in}}%
\pgfpathlineto{\pgfqpoint{6.700715in}{2.291184in}}%
\pgfpathlineto{\pgfqpoint{6.701421in}{2.291878in}}%
\pgfpathlineto{\pgfqpoint{6.701421in}{2.291878in}}%
\pgfpathlineto{\pgfqpoint{6.701845in}{2.293825in}}%
\pgfpathlineto{\pgfqpoint{6.702550in}{2.292129in}}%
\pgfpathlineto{\pgfqpoint{6.703536in}{2.293723in}}%
\pgfpathlineto{\pgfqpoint{6.703676in}{2.294322in}}%
\pgfpathlineto{\pgfqpoint{6.704098in}{2.293846in}}%
\pgfpathlineto{\pgfqpoint{6.704098in}{2.293846in}}%
\pgfpathlineto{\pgfqpoint{6.705082in}{2.293502in}}%
\pgfpathlineto{\pgfqpoint{6.705222in}{2.293631in}}%
\pgfpathlineto{\pgfqpoint{6.707185in}{2.295791in}}%
\pgfpathlineto{\pgfqpoint{6.707605in}{2.295392in}}%
\pgfpathlineto{\pgfqpoint{6.707745in}{2.295855in}}%
\pgfpathlineto{\pgfqpoint{6.707745in}{2.295855in}}%
\pgfpathlineto{\pgfqpoint{6.707884in}{2.296096in}}%
\pgfpathlineto{\pgfqpoint{6.708024in}{2.295766in}}%
\pgfpathlineto{\pgfqpoint{6.708024in}{2.295766in}}%
\pgfpathlineto{\pgfqpoint{6.708164in}{2.295348in}}%
\pgfpathlineto{\pgfqpoint{6.708164in}{2.295348in}}%
\pgfpathlineto{\pgfqpoint{6.708164in}{2.295348in}}%
\pgfpathlineto{\pgfqpoint{6.708723in}{2.297120in}}%
\pgfpathlineto{\pgfqpoint{6.709421in}{2.296296in}}%
\pgfpathlineto{\pgfqpoint{6.709700in}{2.295692in}}%
\pgfpathlineto{\pgfqpoint{6.709979in}{2.296607in}}%
\pgfpathlineto{\pgfqpoint{6.709979in}{2.296607in}}%
\pgfpathlineto{\pgfqpoint{6.711093in}{2.297926in}}%
\pgfpathlineto{\pgfqpoint{6.711371in}{2.297576in}}%
\pgfpathlineto{\pgfqpoint{6.713038in}{2.296047in}}%
\pgfpathlineto{\pgfqpoint{6.713177in}{2.296442in}}%
\pgfpathlineto{\pgfqpoint{6.713871in}{2.298722in}}%
\pgfpathlineto{\pgfqpoint{6.714425in}{2.297264in}}%
\pgfpathlineto{\pgfqpoint{6.714563in}{2.296599in}}%
\pgfpathlineto{\pgfqpoint{6.714840in}{2.297346in}}%
\pgfpathlineto{\pgfqpoint{6.714840in}{2.297346in}}%
\pgfpathlineto{\pgfqpoint{6.714978in}{2.297531in}}%
\pgfpathlineto{\pgfqpoint{6.714978in}{2.297531in}}%
\pgfpathlineto{\pgfqpoint{6.714978in}{2.297531in}}%
\pgfpathlineto{\pgfqpoint{6.715117in}{2.297084in}}%
\pgfpathlineto{\pgfqpoint{6.715393in}{2.297530in}}%
\pgfpathlineto{\pgfqpoint{6.715393in}{2.297530in}}%
\pgfpathlineto{\pgfqpoint{6.715808in}{2.300027in}}%
\pgfpathlineto{\pgfqpoint{6.716223in}{2.297609in}}%
\pgfpathlineto{\pgfqpoint{6.716637in}{2.296383in}}%
\pgfpathlineto{\pgfqpoint{6.717051in}{2.297414in}}%
\pgfpathlineto{\pgfqpoint{6.717051in}{2.297414in}}%
\pgfpathlineto{\pgfqpoint{6.717602in}{2.298901in}}%
\pgfpathlineto{\pgfqpoint{6.718429in}{2.298031in}}%
\pgfpathlineto{\pgfqpoint{6.718566in}{2.297780in}}%
\pgfpathlineto{\pgfqpoint{6.718566in}{2.297780in}}%
\pgfpathlineto{\pgfqpoint{6.718566in}{2.297780in}}%
\pgfpathlineto{\pgfqpoint{6.720216in}{2.299884in}}%
\pgfpathlineto{\pgfqpoint{6.721176in}{2.298939in}}%
\pgfpathlineto{\pgfqpoint{6.721313in}{2.299009in}}%
\pgfpathlineto{\pgfqpoint{6.722683in}{2.304020in}}%
\pgfpathlineto{\pgfqpoint{6.723230in}{2.302970in}}%
\pgfpathlineto{\pgfqpoint{6.723503in}{2.302791in}}%
\pgfpathlineto{\pgfqpoint{6.723503in}{2.302791in}}%
\pgfpathlineto{\pgfqpoint{6.723503in}{2.302791in}}%
\pgfpathlineto{\pgfqpoint{6.724868in}{2.305950in}}%
\pgfpathlineto{\pgfqpoint{6.725140in}{2.305648in}}%
\pgfpathlineto{\pgfqpoint{6.726638in}{2.303047in}}%
\pgfpathlineto{\pgfqpoint{6.727725in}{2.306156in}}%
\pgfpathlineto{\pgfqpoint{6.727861in}{2.305689in}}%
\pgfpathlineto{\pgfqpoint{6.728132in}{2.303617in}}%
\pgfpathlineto{\pgfqpoint{6.728539in}{2.305674in}}%
\pgfpathlineto{\pgfqpoint{6.728539in}{2.305674in}}%
\pgfpathlineto{\pgfqpoint{6.728675in}{2.305895in}}%
\pgfpathlineto{\pgfqpoint{6.728675in}{2.305895in}}%
\pgfpathlineto{\pgfqpoint{6.728675in}{2.305895in}}%
\pgfpathlineto{\pgfqpoint{6.728810in}{2.305503in}}%
\pgfpathlineto{\pgfqpoint{6.728810in}{2.305503in}}%
\pgfpathlineto{\pgfqpoint{6.728810in}{2.305503in}}%
\pgfpathlineto{\pgfqpoint{6.729488in}{2.308633in}}%
\pgfpathlineto{\pgfqpoint{6.729894in}{2.305845in}}%
\pgfpathlineto{\pgfqpoint{6.730300in}{2.302788in}}%
\pgfpathlineto{\pgfqpoint{6.730976in}{2.304987in}}%
\pgfpathlineto{\pgfqpoint{6.731246in}{2.305758in}}%
\pgfpathlineto{\pgfqpoint{6.731381in}{2.305057in}}%
\pgfpathlineto{\pgfqpoint{6.731381in}{2.305057in}}%
\pgfpathlineto{\pgfqpoint{6.731786in}{2.303357in}}%
\pgfpathlineto{\pgfqpoint{6.732191in}{2.304877in}}%
\pgfpathlineto{\pgfqpoint{6.732191in}{2.304877in}}%
\pgfpathlineto{\pgfqpoint{6.732730in}{2.306896in}}%
\pgfpathlineto{\pgfqpoint{6.733269in}{2.305633in}}%
\pgfpathlineto{\pgfqpoint{6.733404in}{2.305000in}}%
\pgfpathlineto{\pgfqpoint{6.733404in}{2.305000in}}%
\pgfusepath{stroke}%
\end{pgfscope}%
\begin{pgfscope}%
\pgfpathrectangle{\pgfqpoint{1.000000in}{0.300000in}}{\pgfqpoint{6.200000in}{2.400000in}} %
\pgfusepath{clip}%
\pgfsetbuttcap%
\pgfsetroundjoin%
\pgfsetlinewidth{0.501875pt}%
\definecolor{currentstroke}{rgb}{0.000000,0.000000,0.000000}%
\pgfsetstrokecolor{currentstroke}%
\pgfsetdash{{1.000000pt}{3.000000pt}}{0.000000pt}%
\pgfpathmoveto{\pgfqpoint{1.000000in}{0.300000in}}%
\pgfpathlineto{\pgfqpoint{1.000000in}{2.700000in}}%
\pgfusepath{stroke}%
\end{pgfscope}%
\begin{pgfscope}%
\pgfsetbuttcap%
\pgfsetroundjoin%
\definecolor{currentfill}{rgb}{0.000000,0.000000,0.000000}%
\pgfsetfillcolor{currentfill}%
\pgfsetlinewidth{0.501875pt}%
\definecolor{currentstroke}{rgb}{0.000000,0.000000,0.000000}%
\pgfsetstrokecolor{currentstroke}%
\pgfsetdash{}{0pt}%
\pgfsys@defobject{currentmarker}{\pgfqpoint{0.000000in}{0.000000in}}{\pgfqpoint{0.000000in}{0.055556in}}{%
\pgfpathmoveto{\pgfqpoint{0.000000in}{0.000000in}}%
\pgfpathlineto{\pgfqpoint{0.000000in}{0.055556in}}%
\pgfusepath{stroke,fill}%
}%
\begin{pgfscope}%
\pgfsys@transformshift{1.000000in}{0.300000in}%
\pgfsys@useobject{currentmarker}{}%
\end{pgfscope}%
\end{pgfscope}%
\begin{pgfscope}%
\pgfsetbuttcap%
\pgfsetroundjoin%
\definecolor{currentfill}{rgb}{0.000000,0.000000,0.000000}%
\pgfsetfillcolor{currentfill}%
\pgfsetlinewidth{0.501875pt}%
\definecolor{currentstroke}{rgb}{0.000000,0.000000,0.000000}%
\pgfsetstrokecolor{currentstroke}%
\pgfsetdash{}{0pt}%
\pgfsys@defobject{currentmarker}{\pgfqpoint{0.000000in}{-0.055556in}}{\pgfqpoint{0.000000in}{0.000000in}}{%
\pgfpathmoveto{\pgfqpoint{0.000000in}{0.000000in}}%
\pgfpathlineto{\pgfqpoint{0.000000in}{-0.055556in}}%
\pgfusepath{stroke,fill}%
}%
\begin{pgfscope}%
\pgfsys@transformshift{1.000000in}{2.700000in}%
\pgfsys@useobject{currentmarker}{}%
\end{pgfscope}%
\end{pgfscope}%
\begin{pgfscope}%
\pgftext[left,bottom,x=0.839506in,y=0.104024in,rotate=0.000000]{{\sffamily\fontsize{12.000000}{14.400000}\selectfont \(\displaystyle {10^{-1}}\)}}
%
\end{pgfscope}%
\begin{pgfscope}%
\pgfpathrectangle{\pgfqpoint{1.000000in}{0.300000in}}{\pgfqpoint{6.200000in}{2.400000in}} %
\pgfusepath{clip}%
\pgfsetbuttcap%
\pgfsetroundjoin%
\pgfsetlinewidth{0.501875pt}%
\definecolor{currentstroke}{rgb}{0.000000,0.000000,0.000000}%
\pgfsetstrokecolor{currentstroke}%
\pgfsetdash{{1.000000pt}{3.000000pt}}{0.000000pt}%
\pgfpathmoveto{\pgfqpoint{2.550000in}{0.300000in}}%
\pgfpathlineto{\pgfqpoint{2.550000in}{2.700000in}}%
\pgfusepath{stroke}%
\end{pgfscope}%
\begin{pgfscope}%
\pgfsetbuttcap%
\pgfsetroundjoin%
\definecolor{currentfill}{rgb}{0.000000,0.000000,0.000000}%
\pgfsetfillcolor{currentfill}%
\pgfsetlinewidth{0.501875pt}%
\definecolor{currentstroke}{rgb}{0.000000,0.000000,0.000000}%
\pgfsetstrokecolor{currentstroke}%
\pgfsetdash{}{0pt}%
\pgfsys@defobject{currentmarker}{\pgfqpoint{0.000000in}{0.000000in}}{\pgfqpoint{0.000000in}{0.055556in}}{%
\pgfpathmoveto{\pgfqpoint{0.000000in}{0.000000in}}%
\pgfpathlineto{\pgfqpoint{0.000000in}{0.055556in}}%
\pgfusepath{stroke,fill}%
}%
\begin{pgfscope}%
\pgfsys@transformshift{2.550000in}{0.300000in}%
\pgfsys@useobject{currentmarker}{}%
\end{pgfscope}%
\end{pgfscope}%
\begin{pgfscope}%
\pgfsetbuttcap%
\pgfsetroundjoin%
\definecolor{currentfill}{rgb}{0.000000,0.000000,0.000000}%
\pgfsetfillcolor{currentfill}%
\pgfsetlinewidth{0.501875pt}%
\definecolor{currentstroke}{rgb}{0.000000,0.000000,0.000000}%
\pgfsetstrokecolor{currentstroke}%
\pgfsetdash{}{0pt}%
\pgfsys@defobject{currentmarker}{\pgfqpoint{0.000000in}{-0.055556in}}{\pgfqpoint{0.000000in}{0.000000in}}{%
\pgfpathmoveto{\pgfqpoint{0.000000in}{0.000000in}}%
\pgfpathlineto{\pgfqpoint{0.000000in}{-0.055556in}}%
\pgfusepath{stroke,fill}%
}%
\begin{pgfscope}%
\pgfsys@transformshift{2.550000in}{2.700000in}%
\pgfsys@useobject{currentmarker}{}%
\end{pgfscope}%
\end{pgfscope}%
\begin{pgfscope}%
\pgftext[left,bottom,x=2.435417in,y=0.104024in,rotate=0.000000]{{\sffamily\fontsize{12.000000}{14.400000}\selectfont \(\displaystyle {10^{0}}\)}}
%
\end{pgfscope}%
\begin{pgfscope}%
\pgfpathrectangle{\pgfqpoint{1.000000in}{0.300000in}}{\pgfqpoint{6.200000in}{2.400000in}} %
\pgfusepath{clip}%
\pgfsetbuttcap%
\pgfsetroundjoin%
\pgfsetlinewidth{0.501875pt}%
\definecolor{currentstroke}{rgb}{0.000000,0.000000,0.000000}%
\pgfsetstrokecolor{currentstroke}%
\pgfsetdash{{1.000000pt}{3.000000pt}}{0.000000pt}%
\pgfpathmoveto{\pgfqpoint{4.100000in}{0.300000in}}%
\pgfpathlineto{\pgfqpoint{4.100000in}{2.700000in}}%
\pgfusepath{stroke}%
\end{pgfscope}%
\begin{pgfscope}%
\pgfsetbuttcap%
\pgfsetroundjoin%
\definecolor{currentfill}{rgb}{0.000000,0.000000,0.000000}%
\pgfsetfillcolor{currentfill}%
\pgfsetlinewidth{0.501875pt}%
\definecolor{currentstroke}{rgb}{0.000000,0.000000,0.000000}%
\pgfsetstrokecolor{currentstroke}%
\pgfsetdash{}{0pt}%
\pgfsys@defobject{currentmarker}{\pgfqpoint{0.000000in}{0.000000in}}{\pgfqpoint{0.000000in}{0.055556in}}{%
\pgfpathmoveto{\pgfqpoint{0.000000in}{0.000000in}}%
\pgfpathlineto{\pgfqpoint{0.000000in}{0.055556in}}%
\pgfusepath{stroke,fill}%
}%
\begin{pgfscope}%
\pgfsys@transformshift{4.100000in}{0.300000in}%
\pgfsys@useobject{currentmarker}{}%
\end{pgfscope}%
\end{pgfscope}%
\begin{pgfscope}%
\pgfsetbuttcap%
\pgfsetroundjoin%
\definecolor{currentfill}{rgb}{0.000000,0.000000,0.000000}%
\pgfsetfillcolor{currentfill}%
\pgfsetlinewidth{0.501875pt}%
\definecolor{currentstroke}{rgb}{0.000000,0.000000,0.000000}%
\pgfsetstrokecolor{currentstroke}%
\pgfsetdash{}{0pt}%
\pgfsys@defobject{currentmarker}{\pgfqpoint{0.000000in}{-0.055556in}}{\pgfqpoint{0.000000in}{0.000000in}}{%
\pgfpathmoveto{\pgfqpoint{0.000000in}{0.000000in}}%
\pgfpathlineto{\pgfqpoint{0.000000in}{-0.055556in}}%
\pgfusepath{stroke,fill}%
}%
\begin{pgfscope}%
\pgfsys@transformshift{4.100000in}{2.700000in}%
\pgfsys@useobject{currentmarker}{}%
\end{pgfscope}%
\end{pgfscope}%
\begin{pgfscope}%
\pgftext[left,bottom,x=3.985417in,y=0.104024in,rotate=0.000000]{{\sffamily\fontsize{12.000000}{14.400000}\selectfont \(\displaystyle {10^{1}}\)}}
%
\end{pgfscope}%
\begin{pgfscope}%
\pgfpathrectangle{\pgfqpoint{1.000000in}{0.300000in}}{\pgfqpoint{6.200000in}{2.400000in}} %
\pgfusepath{clip}%
\pgfsetbuttcap%
\pgfsetroundjoin%
\pgfsetlinewidth{0.501875pt}%
\definecolor{currentstroke}{rgb}{0.000000,0.000000,0.000000}%
\pgfsetstrokecolor{currentstroke}%
\pgfsetdash{{1.000000pt}{3.000000pt}}{0.000000pt}%
\pgfpathmoveto{\pgfqpoint{5.650000in}{0.300000in}}%
\pgfpathlineto{\pgfqpoint{5.650000in}{2.700000in}}%
\pgfusepath{stroke}%
\end{pgfscope}%
\begin{pgfscope}%
\pgfsetbuttcap%
\pgfsetroundjoin%
\definecolor{currentfill}{rgb}{0.000000,0.000000,0.000000}%
\pgfsetfillcolor{currentfill}%
\pgfsetlinewidth{0.501875pt}%
\definecolor{currentstroke}{rgb}{0.000000,0.000000,0.000000}%
\pgfsetstrokecolor{currentstroke}%
\pgfsetdash{}{0pt}%
\pgfsys@defobject{currentmarker}{\pgfqpoint{0.000000in}{0.000000in}}{\pgfqpoint{0.000000in}{0.055556in}}{%
\pgfpathmoveto{\pgfqpoint{0.000000in}{0.000000in}}%
\pgfpathlineto{\pgfqpoint{0.000000in}{0.055556in}}%
\pgfusepath{stroke,fill}%
}%
\begin{pgfscope}%
\pgfsys@transformshift{5.650000in}{0.300000in}%
\pgfsys@useobject{currentmarker}{}%
\end{pgfscope}%
\end{pgfscope}%
\begin{pgfscope}%
\pgfsetbuttcap%
\pgfsetroundjoin%
\definecolor{currentfill}{rgb}{0.000000,0.000000,0.000000}%
\pgfsetfillcolor{currentfill}%
\pgfsetlinewidth{0.501875pt}%
\definecolor{currentstroke}{rgb}{0.000000,0.000000,0.000000}%
\pgfsetstrokecolor{currentstroke}%
\pgfsetdash{}{0pt}%
\pgfsys@defobject{currentmarker}{\pgfqpoint{0.000000in}{-0.055556in}}{\pgfqpoint{0.000000in}{0.000000in}}{%
\pgfpathmoveto{\pgfqpoint{0.000000in}{0.000000in}}%
\pgfpathlineto{\pgfqpoint{0.000000in}{-0.055556in}}%
\pgfusepath{stroke,fill}%
}%
\begin{pgfscope}%
\pgfsys@transformshift{5.650000in}{2.700000in}%
\pgfsys@useobject{currentmarker}{}%
\end{pgfscope}%
\end{pgfscope}%
\begin{pgfscope}%
\pgftext[left,bottom,x=5.535417in,y=0.104024in,rotate=0.000000]{{\sffamily\fontsize{12.000000}{14.400000}\selectfont \(\displaystyle {10^{2}}\)}}
%
\end{pgfscope}%
\begin{pgfscope}%
\pgfpathrectangle{\pgfqpoint{1.000000in}{0.300000in}}{\pgfqpoint{6.200000in}{2.400000in}} %
\pgfusepath{clip}%
\pgfsetbuttcap%
\pgfsetroundjoin%
\pgfsetlinewidth{0.501875pt}%
\definecolor{currentstroke}{rgb}{0.000000,0.000000,0.000000}%
\pgfsetstrokecolor{currentstroke}%
\pgfsetdash{{1.000000pt}{3.000000pt}}{0.000000pt}%
\pgfpathmoveto{\pgfqpoint{7.200000in}{0.300000in}}%
\pgfpathlineto{\pgfqpoint{7.200000in}{2.700000in}}%
\pgfusepath{stroke}%
\end{pgfscope}%
\begin{pgfscope}%
\pgfsetbuttcap%
\pgfsetroundjoin%
\definecolor{currentfill}{rgb}{0.000000,0.000000,0.000000}%
\pgfsetfillcolor{currentfill}%
\pgfsetlinewidth{0.501875pt}%
\definecolor{currentstroke}{rgb}{0.000000,0.000000,0.000000}%
\pgfsetstrokecolor{currentstroke}%
\pgfsetdash{}{0pt}%
\pgfsys@defobject{currentmarker}{\pgfqpoint{0.000000in}{0.000000in}}{\pgfqpoint{0.000000in}{0.055556in}}{%
\pgfpathmoveto{\pgfqpoint{0.000000in}{0.000000in}}%
\pgfpathlineto{\pgfqpoint{0.000000in}{0.055556in}}%
\pgfusepath{stroke,fill}%
}%
\begin{pgfscope}%
\pgfsys@transformshift{7.200000in}{0.300000in}%
\pgfsys@useobject{currentmarker}{}%
\end{pgfscope}%
\end{pgfscope}%
\begin{pgfscope}%
\pgfsetbuttcap%
\pgfsetroundjoin%
\definecolor{currentfill}{rgb}{0.000000,0.000000,0.000000}%
\pgfsetfillcolor{currentfill}%
\pgfsetlinewidth{0.501875pt}%
\definecolor{currentstroke}{rgb}{0.000000,0.000000,0.000000}%
\pgfsetstrokecolor{currentstroke}%
\pgfsetdash{}{0pt}%
\pgfsys@defobject{currentmarker}{\pgfqpoint{0.000000in}{-0.055556in}}{\pgfqpoint{0.000000in}{0.000000in}}{%
\pgfpathmoveto{\pgfqpoint{0.000000in}{0.000000in}}%
\pgfpathlineto{\pgfqpoint{0.000000in}{-0.055556in}}%
\pgfusepath{stroke,fill}%
}%
\begin{pgfscope}%
\pgfsys@transformshift{7.200000in}{2.700000in}%
\pgfsys@useobject{currentmarker}{}%
\end{pgfscope}%
\end{pgfscope}%
\begin{pgfscope}%
\pgftext[left,bottom,x=7.085417in,y=0.104024in,rotate=0.000000]{{\sffamily\fontsize{12.000000}{14.400000}\selectfont \(\displaystyle {10^{3}}\)}}
%
\end{pgfscope}%
\begin{pgfscope}%
\pgfsetbuttcap%
\pgfsetroundjoin%
\definecolor{currentfill}{rgb}{0.000000,0.000000,0.000000}%
\pgfsetfillcolor{currentfill}%
\pgfsetlinewidth{0.501875pt}%
\definecolor{currentstroke}{rgb}{0.000000,0.000000,0.000000}%
\pgfsetstrokecolor{currentstroke}%
\pgfsetdash{}{0pt}%
\pgfsys@defobject{currentmarker}{\pgfqpoint{0.000000in}{0.000000in}}{\pgfqpoint{0.000000in}{0.027778in}}{%
\pgfpathmoveto{\pgfqpoint{0.000000in}{0.000000in}}%
\pgfpathlineto{\pgfqpoint{0.000000in}{0.027778in}}%
\pgfusepath{stroke,fill}%
}%
\begin{pgfscope}%
\pgfsys@transformshift{1.466596in}{0.300000in}%
\pgfsys@useobject{currentmarker}{}%
\end{pgfscope}%
\end{pgfscope}%
\begin{pgfscope}%
\pgfsetbuttcap%
\pgfsetroundjoin%
\definecolor{currentfill}{rgb}{0.000000,0.000000,0.000000}%
\pgfsetfillcolor{currentfill}%
\pgfsetlinewidth{0.501875pt}%
\definecolor{currentstroke}{rgb}{0.000000,0.000000,0.000000}%
\pgfsetstrokecolor{currentstroke}%
\pgfsetdash{}{0pt}%
\pgfsys@defobject{currentmarker}{\pgfqpoint{0.000000in}{-0.027778in}}{\pgfqpoint{0.000000in}{0.000000in}}{%
\pgfpathmoveto{\pgfqpoint{0.000000in}{0.000000in}}%
\pgfpathlineto{\pgfqpoint{0.000000in}{-0.027778in}}%
\pgfusepath{stroke,fill}%
}%
\begin{pgfscope}%
\pgfsys@transformshift{1.466596in}{2.700000in}%
\pgfsys@useobject{currentmarker}{}%
\end{pgfscope}%
\end{pgfscope}%
\begin{pgfscope}%
\pgfsetbuttcap%
\pgfsetroundjoin%
\definecolor{currentfill}{rgb}{0.000000,0.000000,0.000000}%
\pgfsetfillcolor{currentfill}%
\pgfsetlinewidth{0.501875pt}%
\definecolor{currentstroke}{rgb}{0.000000,0.000000,0.000000}%
\pgfsetstrokecolor{currentstroke}%
\pgfsetdash{}{0pt}%
\pgfsys@defobject{currentmarker}{\pgfqpoint{0.000000in}{0.000000in}}{\pgfqpoint{0.000000in}{0.027778in}}{%
\pgfpathmoveto{\pgfqpoint{0.000000in}{0.000000in}}%
\pgfpathlineto{\pgfqpoint{0.000000in}{0.027778in}}%
\pgfusepath{stroke,fill}%
}%
\begin{pgfscope}%
\pgfsys@transformshift{1.739538in}{0.300000in}%
\pgfsys@useobject{currentmarker}{}%
\end{pgfscope}%
\end{pgfscope}%
\begin{pgfscope}%
\pgfsetbuttcap%
\pgfsetroundjoin%
\definecolor{currentfill}{rgb}{0.000000,0.000000,0.000000}%
\pgfsetfillcolor{currentfill}%
\pgfsetlinewidth{0.501875pt}%
\definecolor{currentstroke}{rgb}{0.000000,0.000000,0.000000}%
\pgfsetstrokecolor{currentstroke}%
\pgfsetdash{}{0pt}%
\pgfsys@defobject{currentmarker}{\pgfqpoint{0.000000in}{-0.027778in}}{\pgfqpoint{0.000000in}{0.000000in}}{%
\pgfpathmoveto{\pgfqpoint{0.000000in}{0.000000in}}%
\pgfpathlineto{\pgfqpoint{0.000000in}{-0.027778in}}%
\pgfusepath{stroke,fill}%
}%
\begin{pgfscope}%
\pgfsys@transformshift{1.739538in}{2.700000in}%
\pgfsys@useobject{currentmarker}{}%
\end{pgfscope}%
\end{pgfscope}%
\begin{pgfscope}%
\pgfsetbuttcap%
\pgfsetroundjoin%
\definecolor{currentfill}{rgb}{0.000000,0.000000,0.000000}%
\pgfsetfillcolor{currentfill}%
\pgfsetlinewidth{0.501875pt}%
\definecolor{currentstroke}{rgb}{0.000000,0.000000,0.000000}%
\pgfsetstrokecolor{currentstroke}%
\pgfsetdash{}{0pt}%
\pgfsys@defobject{currentmarker}{\pgfqpoint{0.000000in}{0.000000in}}{\pgfqpoint{0.000000in}{0.027778in}}{%
\pgfpathmoveto{\pgfqpoint{0.000000in}{0.000000in}}%
\pgfpathlineto{\pgfqpoint{0.000000in}{0.027778in}}%
\pgfusepath{stroke,fill}%
}%
\begin{pgfscope}%
\pgfsys@transformshift{1.933193in}{0.300000in}%
\pgfsys@useobject{currentmarker}{}%
\end{pgfscope}%
\end{pgfscope}%
\begin{pgfscope}%
\pgfsetbuttcap%
\pgfsetroundjoin%
\definecolor{currentfill}{rgb}{0.000000,0.000000,0.000000}%
\pgfsetfillcolor{currentfill}%
\pgfsetlinewidth{0.501875pt}%
\definecolor{currentstroke}{rgb}{0.000000,0.000000,0.000000}%
\pgfsetstrokecolor{currentstroke}%
\pgfsetdash{}{0pt}%
\pgfsys@defobject{currentmarker}{\pgfqpoint{0.000000in}{-0.027778in}}{\pgfqpoint{0.000000in}{0.000000in}}{%
\pgfpathmoveto{\pgfqpoint{0.000000in}{0.000000in}}%
\pgfpathlineto{\pgfqpoint{0.000000in}{-0.027778in}}%
\pgfusepath{stroke,fill}%
}%
\begin{pgfscope}%
\pgfsys@transformshift{1.933193in}{2.700000in}%
\pgfsys@useobject{currentmarker}{}%
\end{pgfscope}%
\end{pgfscope}%
\begin{pgfscope}%
\pgfsetbuttcap%
\pgfsetroundjoin%
\definecolor{currentfill}{rgb}{0.000000,0.000000,0.000000}%
\pgfsetfillcolor{currentfill}%
\pgfsetlinewidth{0.501875pt}%
\definecolor{currentstroke}{rgb}{0.000000,0.000000,0.000000}%
\pgfsetstrokecolor{currentstroke}%
\pgfsetdash{}{0pt}%
\pgfsys@defobject{currentmarker}{\pgfqpoint{0.000000in}{0.000000in}}{\pgfqpoint{0.000000in}{0.027778in}}{%
\pgfpathmoveto{\pgfqpoint{0.000000in}{0.000000in}}%
\pgfpathlineto{\pgfqpoint{0.000000in}{0.027778in}}%
\pgfusepath{stroke,fill}%
}%
\begin{pgfscope}%
\pgfsys@transformshift{2.083404in}{0.300000in}%
\pgfsys@useobject{currentmarker}{}%
\end{pgfscope}%
\end{pgfscope}%
\begin{pgfscope}%
\pgfsetbuttcap%
\pgfsetroundjoin%
\definecolor{currentfill}{rgb}{0.000000,0.000000,0.000000}%
\pgfsetfillcolor{currentfill}%
\pgfsetlinewidth{0.501875pt}%
\definecolor{currentstroke}{rgb}{0.000000,0.000000,0.000000}%
\pgfsetstrokecolor{currentstroke}%
\pgfsetdash{}{0pt}%
\pgfsys@defobject{currentmarker}{\pgfqpoint{0.000000in}{-0.027778in}}{\pgfqpoint{0.000000in}{0.000000in}}{%
\pgfpathmoveto{\pgfqpoint{0.000000in}{0.000000in}}%
\pgfpathlineto{\pgfqpoint{0.000000in}{-0.027778in}}%
\pgfusepath{stroke,fill}%
}%
\begin{pgfscope}%
\pgfsys@transformshift{2.083404in}{2.700000in}%
\pgfsys@useobject{currentmarker}{}%
\end{pgfscope}%
\end{pgfscope}%
\begin{pgfscope}%
\pgfsetbuttcap%
\pgfsetroundjoin%
\definecolor{currentfill}{rgb}{0.000000,0.000000,0.000000}%
\pgfsetfillcolor{currentfill}%
\pgfsetlinewidth{0.501875pt}%
\definecolor{currentstroke}{rgb}{0.000000,0.000000,0.000000}%
\pgfsetstrokecolor{currentstroke}%
\pgfsetdash{}{0pt}%
\pgfsys@defobject{currentmarker}{\pgfqpoint{0.000000in}{0.000000in}}{\pgfqpoint{0.000000in}{0.027778in}}{%
\pgfpathmoveto{\pgfqpoint{0.000000in}{0.000000in}}%
\pgfpathlineto{\pgfqpoint{0.000000in}{0.027778in}}%
\pgfusepath{stroke,fill}%
}%
\begin{pgfscope}%
\pgfsys@transformshift{2.206134in}{0.300000in}%
\pgfsys@useobject{currentmarker}{}%
\end{pgfscope}%
\end{pgfscope}%
\begin{pgfscope}%
\pgfsetbuttcap%
\pgfsetroundjoin%
\definecolor{currentfill}{rgb}{0.000000,0.000000,0.000000}%
\pgfsetfillcolor{currentfill}%
\pgfsetlinewidth{0.501875pt}%
\definecolor{currentstroke}{rgb}{0.000000,0.000000,0.000000}%
\pgfsetstrokecolor{currentstroke}%
\pgfsetdash{}{0pt}%
\pgfsys@defobject{currentmarker}{\pgfqpoint{0.000000in}{-0.027778in}}{\pgfqpoint{0.000000in}{0.000000in}}{%
\pgfpathmoveto{\pgfqpoint{0.000000in}{0.000000in}}%
\pgfpathlineto{\pgfqpoint{0.000000in}{-0.027778in}}%
\pgfusepath{stroke,fill}%
}%
\begin{pgfscope}%
\pgfsys@transformshift{2.206134in}{2.700000in}%
\pgfsys@useobject{currentmarker}{}%
\end{pgfscope}%
\end{pgfscope}%
\begin{pgfscope}%
\pgfsetbuttcap%
\pgfsetroundjoin%
\definecolor{currentfill}{rgb}{0.000000,0.000000,0.000000}%
\pgfsetfillcolor{currentfill}%
\pgfsetlinewidth{0.501875pt}%
\definecolor{currentstroke}{rgb}{0.000000,0.000000,0.000000}%
\pgfsetstrokecolor{currentstroke}%
\pgfsetdash{}{0pt}%
\pgfsys@defobject{currentmarker}{\pgfqpoint{0.000000in}{0.000000in}}{\pgfqpoint{0.000000in}{0.027778in}}{%
\pgfpathmoveto{\pgfqpoint{0.000000in}{0.000000in}}%
\pgfpathlineto{\pgfqpoint{0.000000in}{0.027778in}}%
\pgfusepath{stroke,fill}%
}%
\begin{pgfscope}%
\pgfsys@transformshift{2.309902in}{0.300000in}%
\pgfsys@useobject{currentmarker}{}%
\end{pgfscope}%
\end{pgfscope}%
\begin{pgfscope}%
\pgfsetbuttcap%
\pgfsetroundjoin%
\definecolor{currentfill}{rgb}{0.000000,0.000000,0.000000}%
\pgfsetfillcolor{currentfill}%
\pgfsetlinewidth{0.501875pt}%
\definecolor{currentstroke}{rgb}{0.000000,0.000000,0.000000}%
\pgfsetstrokecolor{currentstroke}%
\pgfsetdash{}{0pt}%
\pgfsys@defobject{currentmarker}{\pgfqpoint{0.000000in}{-0.027778in}}{\pgfqpoint{0.000000in}{0.000000in}}{%
\pgfpathmoveto{\pgfqpoint{0.000000in}{0.000000in}}%
\pgfpathlineto{\pgfqpoint{0.000000in}{-0.027778in}}%
\pgfusepath{stroke,fill}%
}%
\begin{pgfscope}%
\pgfsys@transformshift{2.309902in}{2.700000in}%
\pgfsys@useobject{currentmarker}{}%
\end{pgfscope}%
\end{pgfscope}%
\begin{pgfscope}%
\pgfsetbuttcap%
\pgfsetroundjoin%
\definecolor{currentfill}{rgb}{0.000000,0.000000,0.000000}%
\pgfsetfillcolor{currentfill}%
\pgfsetlinewidth{0.501875pt}%
\definecolor{currentstroke}{rgb}{0.000000,0.000000,0.000000}%
\pgfsetstrokecolor{currentstroke}%
\pgfsetdash{}{0pt}%
\pgfsys@defobject{currentmarker}{\pgfqpoint{0.000000in}{0.000000in}}{\pgfqpoint{0.000000in}{0.027778in}}{%
\pgfpathmoveto{\pgfqpoint{0.000000in}{0.000000in}}%
\pgfpathlineto{\pgfqpoint{0.000000in}{0.027778in}}%
\pgfusepath{stroke,fill}%
}%
\begin{pgfscope}%
\pgfsys@transformshift{2.399789in}{0.300000in}%
\pgfsys@useobject{currentmarker}{}%
\end{pgfscope}%
\end{pgfscope}%
\begin{pgfscope}%
\pgfsetbuttcap%
\pgfsetroundjoin%
\definecolor{currentfill}{rgb}{0.000000,0.000000,0.000000}%
\pgfsetfillcolor{currentfill}%
\pgfsetlinewidth{0.501875pt}%
\definecolor{currentstroke}{rgb}{0.000000,0.000000,0.000000}%
\pgfsetstrokecolor{currentstroke}%
\pgfsetdash{}{0pt}%
\pgfsys@defobject{currentmarker}{\pgfqpoint{0.000000in}{-0.027778in}}{\pgfqpoint{0.000000in}{0.000000in}}{%
\pgfpathmoveto{\pgfqpoint{0.000000in}{0.000000in}}%
\pgfpathlineto{\pgfqpoint{0.000000in}{-0.027778in}}%
\pgfusepath{stroke,fill}%
}%
\begin{pgfscope}%
\pgfsys@transformshift{2.399789in}{2.700000in}%
\pgfsys@useobject{currentmarker}{}%
\end{pgfscope}%
\end{pgfscope}%
\begin{pgfscope}%
\pgfsetbuttcap%
\pgfsetroundjoin%
\definecolor{currentfill}{rgb}{0.000000,0.000000,0.000000}%
\pgfsetfillcolor{currentfill}%
\pgfsetlinewidth{0.501875pt}%
\definecolor{currentstroke}{rgb}{0.000000,0.000000,0.000000}%
\pgfsetstrokecolor{currentstroke}%
\pgfsetdash{}{0pt}%
\pgfsys@defobject{currentmarker}{\pgfqpoint{0.000000in}{0.000000in}}{\pgfqpoint{0.000000in}{0.027778in}}{%
\pgfpathmoveto{\pgfqpoint{0.000000in}{0.000000in}}%
\pgfpathlineto{\pgfqpoint{0.000000in}{0.027778in}}%
\pgfusepath{stroke,fill}%
}%
\begin{pgfscope}%
\pgfsys@transformshift{2.479076in}{0.300000in}%
\pgfsys@useobject{currentmarker}{}%
\end{pgfscope}%
\end{pgfscope}%
\begin{pgfscope}%
\pgfsetbuttcap%
\pgfsetroundjoin%
\definecolor{currentfill}{rgb}{0.000000,0.000000,0.000000}%
\pgfsetfillcolor{currentfill}%
\pgfsetlinewidth{0.501875pt}%
\definecolor{currentstroke}{rgb}{0.000000,0.000000,0.000000}%
\pgfsetstrokecolor{currentstroke}%
\pgfsetdash{}{0pt}%
\pgfsys@defobject{currentmarker}{\pgfqpoint{0.000000in}{-0.027778in}}{\pgfqpoint{0.000000in}{0.000000in}}{%
\pgfpathmoveto{\pgfqpoint{0.000000in}{0.000000in}}%
\pgfpathlineto{\pgfqpoint{0.000000in}{-0.027778in}}%
\pgfusepath{stroke,fill}%
}%
\begin{pgfscope}%
\pgfsys@transformshift{2.479076in}{2.700000in}%
\pgfsys@useobject{currentmarker}{}%
\end{pgfscope}%
\end{pgfscope}%
\begin{pgfscope}%
\pgfsetbuttcap%
\pgfsetroundjoin%
\definecolor{currentfill}{rgb}{0.000000,0.000000,0.000000}%
\pgfsetfillcolor{currentfill}%
\pgfsetlinewidth{0.501875pt}%
\definecolor{currentstroke}{rgb}{0.000000,0.000000,0.000000}%
\pgfsetstrokecolor{currentstroke}%
\pgfsetdash{}{0pt}%
\pgfsys@defobject{currentmarker}{\pgfqpoint{0.000000in}{0.000000in}}{\pgfqpoint{0.000000in}{0.027778in}}{%
\pgfpathmoveto{\pgfqpoint{0.000000in}{0.000000in}}%
\pgfpathlineto{\pgfqpoint{0.000000in}{0.027778in}}%
\pgfusepath{stroke,fill}%
}%
\begin{pgfscope}%
\pgfsys@transformshift{3.016596in}{0.300000in}%
\pgfsys@useobject{currentmarker}{}%
\end{pgfscope}%
\end{pgfscope}%
\begin{pgfscope}%
\pgfsetbuttcap%
\pgfsetroundjoin%
\definecolor{currentfill}{rgb}{0.000000,0.000000,0.000000}%
\pgfsetfillcolor{currentfill}%
\pgfsetlinewidth{0.501875pt}%
\definecolor{currentstroke}{rgb}{0.000000,0.000000,0.000000}%
\pgfsetstrokecolor{currentstroke}%
\pgfsetdash{}{0pt}%
\pgfsys@defobject{currentmarker}{\pgfqpoint{0.000000in}{-0.027778in}}{\pgfqpoint{0.000000in}{0.000000in}}{%
\pgfpathmoveto{\pgfqpoint{0.000000in}{0.000000in}}%
\pgfpathlineto{\pgfqpoint{0.000000in}{-0.027778in}}%
\pgfusepath{stroke,fill}%
}%
\begin{pgfscope}%
\pgfsys@transformshift{3.016596in}{2.700000in}%
\pgfsys@useobject{currentmarker}{}%
\end{pgfscope}%
\end{pgfscope}%
\begin{pgfscope}%
\pgfsetbuttcap%
\pgfsetroundjoin%
\definecolor{currentfill}{rgb}{0.000000,0.000000,0.000000}%
\pgfsetfillcolor{currentfill}%
\pgfsetlinewidth{0.501875pt}%
\definecolor{currentstroke}{rgb}{0.000000,0.000000,0.000000}%
\pgfsetstrokecolor{currentstroke}%
\pgfsetdash{}{0pt}%
\pgfsys@defobject{currentmarker}{\pgfqpoint{0.000000in}{0.000000in}}{\pgfqpoint{0.000000in}{0.027778in}}{%
\pgfpathmoveto{\pgfqpoint{0.000000in}{0.000000in}}%
\pgfpathlineto{\pgfqpoint{0.000000in}{0.027778in}}%
\pgfusepath{stroke,fill}%
}%
\begin{pgfscope}%
\pgfsys@transformshift{3.289538in}{0.300000in}%
\pgfsys@useobject{currentmarker}{}%
\end{pgfscope}%
\end{pgfscope}%
\begin{pgfscope}%
\pgfsetbuttcap%
\pgfsetroundjoin%
\definecolor{currentfill}{rgb}{0.000000,0.000000,0.000000}%
\pgfsetfillcolor{currentfill}%
\pgfsetlinewidth{0.501875pt}%
\definecolor{currentstroke}{rgb}{0.000000,0.000000,0.000000}%
\pgfsetstrokecolor{currentstroke}%
\pgfsetdash{}{0pt}%
\pgfsys@defobject{currentmarker}{\pgfqpoint{0.000000in}{-0.027778in}}{\pgfqpoint{0.000000in}{0.000000in}}{%
\pgfpathmoveto{\pgfqpoint{0.000000in}{0.000000in}}%
\pgfpathlineto{\pgfqpoint{0.000000in}{-0.027778in}}%
\pgfusepath{stroke,fill}%
}%
\begin{pgfscope}%
\pgfsys@transformshift{3.289538in}{2.700000in}%
\pgfsys@useobject{currentmarker}{}%
\end{pgfscope}%
\end{pgfscope}%
\begin{pgfscope}%
\pgfsetbuttcap%
\pgfsetroundjoin%
\definecolor{currentfill}{rgb}{0.000000,0.000000,0.000000}%
\pgfsetfillcolor{currentfill}%
\pgfsetlinewidth{0.501875pt}%
\definecolor{currentstroke}{rgb}{0.000000,0.000000,0.000000}%
\pgfsetstrokecolor{currentstroke}%
\pgfsetdash{}{0pt}%
\pgfsys@defobject{currentmarker}{\pgfqpoint{0.000000in}{0.000000in}}{\pgfqpoint{0.000000in}{0.027778in}}{%
\pgfpathmoveto{\pgfqpoint{0.000000in}{0.000000in}}%
\pgfpathlineto{\pgfqpoint{0.000000in}{0.027778in}}%
\pgfusepath{stroke,fill}%
}%
\begin{pgfscope}%
\pgfsys@transformshift{3.483193in}{0.300000in}%
\pgfsys@useobject{currentmarker}{}%
\end{pgfscope}%
\end{pgfscope}%
\begin{pgfscope}%
\pgfsetbuttcap%
\pgfsetroundjoin%
\definecolor{currentfill}{rgb}{0.000000,0.000000,0.000000}%
\pgfsetfillcolor{currentfill}%
\pgfsetlinewidth{0.501875pt}%
\definecolor{currentstroke}{rgb}{0.000000,0.000000,0.000000}%
\pgfsetstrokecolor{currentstroke}%
\pgfsetdash{}{0pt}%
\pgfsys@defobject{currentmarker}{\pgfqpoint{0.000000in}{-0.027778in}}{\pgfqpoint{0.000000in}{0.000000in}}{%
\pgfpathmoveto{\pgfqpoint{0.000000in}{0.000000in}}%
\pgfpathlineto{\pgfqpoint{0.000000in}{-0.027778in}}%
\pgfusepath{stroke,fill}%
}%
\begin{pgfscope}%
\pgfsys@transformshift{3.483193in}{2.700000in}%
\pgfsys@useobject{currentmarker}{}%
\end{pgfscope}%
\end{pgfscope}%
\begin{pgfscope}%
\pgfsetbuttcap%
\pgfsetroundjoin%
\definecolor{currentfill}{rgb}{0.000000,0.000000,0.000000}%
\pgfsetfillcolor{currentfill}%
\pgfsetlinewidth{0.501875pt}%
\definecolor{currentstroke}{rgb}{0.000000,0.000000,0.000000}%
\pgfsetstrokecolor{currentstroke}%
\pgfsetdash{}{0pt}%
\pgfsys@defobject{currentmarker}{\pgfqpoint{0.000000in}{0.000000in}}{\pgfqpoint{0.000000in}{0.027778in}}{%
\pgfpathmoveto{\pgfqpoint{0.000000in}{0.000000in}}%
\pgfpathlineto{\pgfqpoint{0.000000in}{0.027778in}}%
\pgfusepath{stroke,fill}%
}%
\begin{pgfscope}%
\pgfsys@transformshift{3.633404in}{0.300000in}%
\pgfsys@useobject{currentmarker}{}%
\end{pgfscope}%
\end{pgfscope}%
\begin{pgfscope}%
\pgfsetbuttcap%
\pgfsetroundjoin%
\definecolor{currentfill}{rgb}{0.000000,0.000000,0.000000}%
\pgfsetfillcolor{currentfill}%
\pgfsetlinewidth{0.501875pt}%
\definecolor{currentstroke}{rgb}{0.000000,0.000000,0.000000}%
\pgfsetstrokecolor{currentstroke}%
\pgfsetdash{}{0pt}%
\pgfsys@defobject{currentmarker}{\pgfqpoint{0.000000in}{-0.027778in}}{\pgfqpoint{0.000000in}{0.000000in}}{%
\pgfpathmoveto{\pgfqpoint{0.000000in}{0.000000in}}%
\pgfpathlineto{\pgfqpoint{0.000000in}{-0.027778in}}%
\pgfusepath{stroke,fill}%
}%
\begin{pgfscope}%
\pgfsys@transformshift{3.633404in}{2.700000in}%
\pgfsys@useobject{currentmarker}{}%
\end{pgfscope}%
\end{pgfscope}%
\begin{pgfscope}%
\pgfsetbuttcap%
\pgfsetroundjoin%
\definecolor{currentfill}{rgb}{0.000000,0.000000,0.000000}%
\pgfsetfillcolor{currentfill}%
\pgfsetlinewidth{0.501875pt}%
\definecolor{currentstroke}{rgb}{0.000000,0.000000,0.000000}%
\pgfsetstrokecolor{currentstroke}%
\pgfsetdash{}{0pt}%
\pgfsys@defobject{currentmarker}{\pgfqpoint{0.000000in}{0.000000in}}{\pgfqpoint{0.000000in}{0.027778in}}{%
\pgfpathmoveto{\pgfqpoint{0.000000in}{0.000000in}}%
\pgfpathlineto{\pgfqpoint{0.000000in}{0.027778in}}%
\pgfusepath{stroke,fill}%
}%
\begin{pgfscope}%
\pgfsys@transformshift{3.756134in}{0.300000in}%
\pgfsys@useobject{currentmarker}{}%
\end{pgfscope}%
\end{pgfscope}%
\begin{pgfscope}%
\pgfsetbuttcap%
\pgfsetroundjoin%
\definecolor{currentfill}{rgb}{0.000000,0.000000,0.000000}%
\pgfsetfillcolor{currentfill}%
\pgfsetlinewidth{0.501875pt}%
\definecolor{currentstroke}{rgb}{0.000000,0.000000,0.000000}%
\pgfsetstrokecolor{currentstroke}%
\pgfsetdash{}{0pt}%
\pgfsys@defobject{currentmarker}{\pgfqpoint{0.000000in}{-0.027778in}}{\pgfqpoint{0.000000in}{0.000000in}}{%
\pgfpathmoveto{\pgfqpoint{0.000000in}{0.000000in}}%
\pgfpathlineto{\pgfqpoint{0.000000in}{-0.027778in}}%
\pgfusepath{stroke,fill}%
}%
\begin{pgfscope}%
\pgfsys@transformshift{3.756134in}{2.700000in}%
\pgfsys@useobject{currentmarker}{}%
\end{pgfscope}%
\end{pgfscope}%
\begin{pgfscope}%
\pgfsetbuttcap%
\pgfsetroundjoin%
\definecolor{currentfill}{rgb}{0.000000,0.000000,0.000000}%
\pgfsetfillcolor{currentfill}%
\pgfsetlinewidth{0.501875pt}%
\definecolor{currentstroke}{rgb}{0.000000,0.000000,0.000000}%
\pgfsetstrokecolor{currentstroke}%
\pgfsetdash{}{0pt}%
\pgfsys@defobject{currentmarker}{\pgfqpoint{0.000000in}{0.000000in}}{\pgfqpoint{0.000000in}{0.027778in}}{%
\pgfpathmoveto{\pgfqpoint{0.000000in}{0.000000in}}%
\pgfpathlineto{\pgfqpoint{0.000000in}{0.027778in}}%
\pgfusepath{stroke,fill}%
}%
\begin{pgfscope}%
\pgfsys@transformshift{3.859902in}{0.300000in}%
\pgfsys@useobject{currentmarker}{}%
\end{pgfscope}%
\end{pgfscope}%
\begin{pgfscope}%
\pgfsetbuttcap%
\pgfsetroundjoin%
\definecolor{currentfill}{rgb}{0.000000,0.000000,0.000000}%
\pgfsetfillcolor{currentfill}%
\pgfsetlinewidth{0.501875pt}%
\definecolor{currentstroke}{rgb}{0.000000,0.000000,0.000000}%
\pgfsetstrokecolor{currentstroke}%
\pgfsetdash{}{0pt}%
\pgfsys@defobject{currentmarker}{\pgfqpoint{0.000000in}{-0.027778in}}{\pgfqpoint{0.000000in}{0.000000in}}{%
\pgfpathmoveto{\pgfqpoint{0.000000in}{0.000000in}}%
\pgfpathlineto{\pgfqpoint{0.000000in}{-0.027778in}}%
\pgfusepath{stroke,fill}%
}%
\begin{pgfscope}%
\pgfsys@transformshift{3.859902in}{2.700000in}%
\pgfsys@useobject{currentmarker}{}%
\end{pgfscope}%
\end{pgfscope}%
\begin{pgfscope}%
\pgfsetbuttcap%
\pgfsetroundjoin%
\definecolor{currentfill}{rgb}{0.000000,0.000000,0.000000}%
\pgfsetfillcolor{currentfill}%
\pgfsetlinewidth{0.501875pt}%
\definecolor{currentstroke}{rgb}{0.000000,0.000000,0.000000}%
\pgfsetstrokecolor{currentstroke}%
\pgfsetdash{}{0pt}%
\pgfsys@defobject{currentmarker}{\pgfqpoint{0.000000in}{0.000000in}}{\pgfqpoint{0.000000in}{0.027778in}}{%
\pgfpathmoveto{\pgfqpoint{0.000000in}{0.000000in}}%
\pgfpathlineto{\pgfqpoint{0.000000in}{0.027778in}}%
\pgfusepath{stroke,fill}%
}%
\begin{pgfscope}%
\pgfsys@transformshift{3.949789in}{0.300000in}%
\pgfsys@useobject{currentmarker}{}%
\end{pgfscope}%
\end{pgfscope}%
\begin{pgfscope}%
\pgfsetbuttcap%
\pgfsetroundjoin%
\definecolor{currentfill}{rgb}{0.000000,0.000000,0.000000}%
\pgfsetfillcolor{currentfill}%
\pgfsetlinewidth{0.501875pt}%
\definecolor{currentstroke}{rgb}{0.000000,0.000000,0.000000}%
\pgfsetstrokecolor{currentstroke}%
\pgfsetdash{}{0pt}%
\pgfsys@defobject{currentmarker}{\pgfqpoint{0.000000in}{-0.027778in}}{\pgfqpoint{0.000000in}{0.000000in}}{%
\pgfpathmoveto{\pgfqpoint{0.000000in}{0.000000in}}%
\pgfpathlineto{\pgfqpoint{0.000000in}{-0.027778in}}%
\pgfusepath{stroke,fill}%
}%
\begin{pgfscope}%
\pgfsys@transformshift{3.949789in}{2.700000in}%
\pgfsys@useobject{currentmarker}{}%
\end{pgfscope}%
\end{pgfscope}%
\begin{pgfscope}%
\pgfsetbuttcap%
\pgfsetroundjoin%
\definecolor{currentfill}{rgb}{0.000000,0.000000,0.000000}%
\pgfsetfillcolor{currentfill}%
\pgfsetlinewidth{0.501875pt}%
\definecolor{currentstroke}{rgb}{0.000000,0.000000,0.000000}%
\pgfsetstrokecolor{currentstroke}%
\pgfsetdash{}{0pt}%
\pgfsys@defobject{currentmarker}{\pgfqpoint{0.000000in}{0.000000in}}{\pgfqpoint{0.000000in}{0.027778in}}{%
\pgfpathmoveto{\pgfqpoint{0.000000in}{0.000000in}}%
\pgfpathlineto{\pgfqpoint{0.000000in}{0.027778in}}%
\pgfusepath{stroke,fill}%
}%
\begin{pgfscope}%
\pgfsys@transformshift{4.029076in}{0.300000in}%
\pgfsys@useobject{currentmarker}{}%
\end{pgfscope}%
\end{pgfscope}%
\begin{pgfscope}%
\pgfsetbuttcap%
\pgfsetroundjoin%
\definecolor{currentfill}{rgb}{0.000000,0.000000,0.000000}%
\pgfsetfillcolor{currentfill}%
\pgfsetlinewidth{0.501875pt}%
\definecolor{currentstroke}{rgb}{0.000000,0.000000,0.000000}%
\pgfsetstrokecolor{currentstroke}%
\pgfsetdash{}{0pt}%
\pgfsys@defobject{currentmarker}{\pgfqpoint{0.000000in}{-0.027778in}}{\pgfqpoint{0.000000in}{0.000000in}}{%
\pgfpathmoveto{\pgfqpoint{0.000000in}{0.000000in}}%
\pgfpathlineto{\pgfqpoint{0.000000in}{-0.027778in}}%
\pgfusepath{stroke,fill}%
}%
\begin{pgfscope}%
\pgfsys@transformshift{4.029076in}{2.700000in}%
\pgfsys@useobject{currentmarker}{}%
\end{pgfscope}%
\end{pgfscope}%
\begin{pgfscope}%
\pgfsetbuttcap%
\pgfsetroundjoin%
\definecolor{currentfill}{rgb}{0.000000,0.000000,0.000000}%
\pgfsetfillcolor{currentfill}%
\pgfsetlinewidth{0.501875pt}%
\definecolor{currentstroke}{rgb}{0.000000,0.000000,0.000000}%
\pgfsetstrokecolor{currentstroke}%
\pgfsetdash{}{0pt}%
\pgfsys@defobject{currentmarker}{\pgfqpoint{0.000000in}{0.000000in}}{\pgfqpoint{0.000000in}{0.027778in}}{%
\pgfpathmoveto{\pgfqpoint{0.000000in}{0.000000in}}%
\pgfpathlineto{\pgfqpoint{0.000000in}{0.027778in}}%
\pgfusepath{stroke,fill}%
}%
\begin{pgfscope}%
\pgfsys@transformshift{4.566596in}{0.300000in}%
\pgfsys@useobject{currentmarker}{}%
\end{pgfscope}%
\end{pgfscope}%
\begin{pgfscope}%
\pgfsetbuttcap%
\pgfsetroundjoin%
\definecolor{currentfill}{rgb}{0.000000,0.000000,0.000000}%
\pgfsetfillcolor{currentfill}%
\pgfsetlinewidth{0.501875pt}%
\definecolor{currentstroke}{rgb}{0.000000,0.000000,0.000000}%
\pgfsetstrokecolor{currentstroke}%
\pgfsetdash{}{0pt}%
\pgfsys@defobject{currentmarker}{\pgfqpoint{0.000000in}{-0.027778in}}{\pgfqpoint{0.000000in}{0.000000in}}{%
\pgfpathmoveto{\pgfqpoint{0.000000in}{0.000000in}}%
\pgfpathlineto{\pgfqpoint{0.000000in}{-0.027778in}}%
\pgfusepath{stroke,fill}%
}%
\begin{pgfscope}%
\pgfsys@transformshift{4.566596in}{2.700000in}%
\pgfsys@useobject{currentmarker}{}%
\end{pgfscope}%
\end{pgfscope}%
\begin{pgfscope}%
\pgfsetbuttcap%
\pgfsetroundjoin%
\definecolor{currentfill}{rgb}{0.000000,0.000000,0.000000}%
\pgfsetfillcolor{currentfill}%
\pgfsetlinewidth{0.501875pt}%
\definecolor{currentstroke}{rgb}{0.000000,0.000000,0.000000}%
\pgfsetstrokecolor{currentstroke}%
\pgfsetdash{}{0pt}%
\pgfsys@defobject{currentmarker}{\pgfqpoint{0.000000in}{0.000000in}}{\pgfqpoint{0.000000in}{0.027778in}}{%
\pgfpathmoveto{\pgfqpoint{0.000000in}{0.000000in}}%
\pgfpathlineto{\pgfqpoint{0.000000in}{0.027778in}}%
\pgfusepath{stroke,fill}%
}%
\begin{pgfscope}%
\pgfsys@transformshift{4.839538in}{0.300000in}%
\pgfsys@useobject{currentmarker}{}%
\end{pgfscope}%
\end{pgfscope}%
\begin{pgfscope}%
\pgfsetbuttcap%
\pgfsetroundjoin%
\definecolor{currentfill}{rgb}{0.000000,0.000000,0.000000}%
\pgfsetfillcolor{currentfill}%
\pgfsetlinewidth{0.501875pt}%
\definecolor{currentstroke}{rgb}{0.000000,0.000000,0.000000}%
\pgfsetstrokecolor{currentstroke}%
\pgfsetdash{}{0pt}%
\pgfsys@defobject{currentmarker}{\pgfqpoint{0.000000in}{-0.027778in}}{\pgfqpoint{0.000000in}{0.000000in}}{%
\pgfpathmoveto{\pgfqpoint{0.000000in}{0.000000in}}%
\pgfpathlineto{\pgfqpoint{0.000000in}{-0.027778in}}%
\pgfusepath{stroke,fill}%
}%
\begin{pgfscope}%
\pgfsys@transformshift{4.839538in}{2.700000in}%
\pgfsys@useobject{currentmarker}{}%
\end{pgfscope}%
\end{pgfscope}%
\begin{pgfscope}%
\pgfsetbuttcap%
\pgfsetroundjoin%
\definecolor{currentfill}{rgb}{0.000000,0.000000,0.000000}%
\pgfsetfillcolor{currentfill}%
\pgfsetlinewidth{0.501875pt}%
\definecolor{currentstroke}{rgb}{0.000000,0.000000,0.000000}%
\pgfsetstrokecolor{currentstroke}%
\pgfsetdash{}{0pt}%
\pgfsys@defobject{currentmarker}{\pgfqpoint{0.000000in}{0.000000in}}{\pgfqpoint{0.000000in}{0.027778in}}{%
\pgfpathmoveto{\pgfqpoint{0.000000in}{0.000000in}}%
\pgfpathlineto{\pgfqpoint{0.000000in}{0.027778in}}%
\pgfusepath{stroke,fill}%
}%
\begin{pgfscope}%
\pgfsys@transformshift{5.033193in}{0.300000in}%
\pgfsys@useobject{currentmarker}{}%
\end{pgfscope}%
\end{pgfscope}%
\begin{pgfscope}%
\pgfsetbuttcap%
\pgfsetroundjoin%
\definecolor{currentfill}{rgb}{0.000000,0.000000,0.000000}%
\pgfsetfillcolor{currentfill}%
\pgfsetlinewidth{0.501875pt}%
\definecolor{currentstroke}{rgb}{0.000000,0.000000,0.000000}%
\pgfsetstrokecolor{currentstroke}%
\pgfsetdash{}{0pt}%
\pgfsys@defobject{currentmarker}{\pgfqpoint{0.000000in}{-0.027778in}}{\pgfqpoint{0.000000in}{0.000000in}}{%
\pgfpathmoveto{\pgfqpoint{0.000000in}{0.000000in}}%
\pgfpathlineto{\pgfqpoint{0.000000in}{-0.027778in}}%
\pgfusepath{stroke,fill}%
}%
\begin{pgfscope}%
\pgfsys@transformshift{5.033193in}{2.700000in}%
\pgfsys@useobject{currentmarker}{}%
\end{pgfscope}%
\end{pgfscope}%
\begin{pgfscope}%
\pgfsetbuttcap%
\pgfsetroundjoin%
\definecolor{currentfill}{rgb}{0.000000,0.000000,0.000000}%
\pgfsetfillcolor{currentfill}%
\pgfsetlinewidth{0.501875pt}%
\definecolor{currentstroke}{rgb}{0.000000,0.000000,0.000000}%
\pgfsetstrokecolor{currentstroke}%
\pgfsetdash{}{0pt}%
\pgfsys@defobject{currentmarker}{\pgfqpoint{0.000000in}{0.000000in}}{\pgfqpoint{0.000000in}{0.027778in}}{%
\pgfpathmoveto{\pgfqpoint{0.000000in}{0.000000in}}%
\pgfpathlineto{\pgfqpoint{0.000000in}{0.027778in}}%
\pgfusepath{stroke,fill}%
}%
\begin{pgfscope}%
\pgfsys@transformshift{5.183404in}{0.300000in}%
\pgfsys@useobject{currentmarker}{}%
\end{pgfscope}%
\end{pgfscope}%
\begin{pgfscope}%
\pgfsetbuttcap%
\pgfsetroundjoin%
\definecolor{currentfill}{rgb}{0.000000,0.000000,0.000000}%
\pgfsetfillcolor{currentfill}%
\pgfsetlinewidth{0.501875pt}%
\definecolor{currentstroke}{rgb}{0.000000,0.000000,0.000000}%
\pgfsetstrokecolor{currentstroke}%
\pgfsetdash{}{0pt}%
\pgfsys@defobject{currentmarker}{\pgfqpoint{0.000000in}{-0.027778in}}{\pgfqpoint{0.000000in}{0.000000in}}{%
\pgfpathmoveto{\pgfqpoint{0.000000in}{0.000000in}}%
\pgfpathlineto{\pgfqpoint{0.000000in}{-0.027778in}}%
\pgfusepath{stroke,fill}%
}%
\begin{pgfscope}%
\pgfsys@transformshift{5.183404in}{2.700000in}%
\pgfsys@useobject{currentmarker}{}%
\end{pgfscope}%
\end{pgfscope}%
\begin{pgfscope}%
\pgfsetbuttcap%
\pgfsetroundjoin%
\definecolor{currentfill}{rgb}{0.000000,0.000000,0.000000}%
\pgfsetfillcolor{currentfill}%
\pgfsetlinewidth{0.501875pt}%
\definecolor{currentstroke}{rgb}{0.000000,0.000000,0.000000}%
\pgfsetstrokecolor{currentstroke}%
\pgfsetdash{}{0pt}%
\pgfsys@defobject{currentmarker}{\pgfqpoint{0.000000in}{0.000000in}}{\pgfqpoint{0.000000in}{0.027778in}}{%
\pgfpathmoveto{\pgfqpoint{0.000000in}{0.000000in}}%
\pgfpathlineto{\pgfqpoint{0.000000in}{0.027778in}}%
\pgfusepath{stroke,fill}%
}%
\begin{pgfscope}%
\pgfsys@transformshift{5.306134in}{0.300000in}%
\pgfsys@useobject{currentmarker}{}%
\end{pgfscope}%
\end{pgfscope}%
\begin{pgfscope}%
\pgfsetbuttcap%
\pgfsetroundjoin%
\definecolor{currentfill}{rgb}{0.000000,0.000000,0.000000}%
\pgfsetfillcolor{currentfill}%
\pgfsetlinewidth{0.501875pt}%
\definecolor{currentstroke}{rgb}{0.000000,0.000000,0.000000}%
\pgfsetstrokecolor{currentstroke}%
\pgfsetdash{}{0pt}%
\pgfsys@defobject{currentmarker}{\pgfqpoint{0.000000in}{-0.027778in}}{\pgfqpoint{0.000000in}{0.000000in}}{%
\pgfpathmoveto{\pgfqpoint{0.000000in}{0.000000in}}%
\pgfpathlineto{\pgfqpoint{0.000000in}{-0.027778in}}%
\pgfusepath{stroke,fill}%
}%
\begin{pgfscope}%
\pgfsys@transformshift{5.306134in}{2.700000in}%
\pgfsys@useobject{currentmarker}{}%
\end{pgfscope}%
\end{pgfscope}%
\begin{pgfscope}%
\pgfsetbuttcap%
\pgfsetroundjoin%
\definecolor{currentfill}{rgb}{0.000000,0.000000,0.000000}%
\pgfsetfillcolor{currentfill}%
\pgfsetlinewidth{0.501875pt}%
\definecolor{currentstroke}{rgb}{0.000000,0.000000,0.000000}%
\pgfsetstrokecolor{currentstroke}%
\pgfsetdash{}{0pt}%
\pgfsys@defobject{currentmarker}{\pgfqpoint{0.000000in}{0.000000in}}{\pgfqpoint{0.000000in}{0.027778in}}{%
\pgfpathmoveto{\pgfqpoint{0.000000in}{0.000000in}}%
\pgfpathlineto{\pgfqpoint{0.000000in}{0.027778in}}%
\pgfusepath{stroke,fill}%
}%
\begin{pgfscope}%
\pgfsys@transformshift{5.409902in}{0.300000in}%
\pgfsys@useobject{currentmarker}{}%
\end{pgfscope}%
\end{pgfscope}%
\begin{pgfscope}%
\pgfsetbuttcap%
\pgfsetroundjoin%
\definecolor{currentfill}{rgb}{0.000000,0.000000,0.000000}%
\pgfsetfillcolor{currentfill}%
\pgfsetlinewidth{0.501875pt}%
\definecolor{currentstroke}{rgb}{0.000000,0.000000,0.000000}%
\pgfsetstrokecolor{currentstroke}%
\pgfsetdash{}{0pt}%
\pgfsys@defobject{currentmarker}{\pgfqpoint{0.000000in}{-0.027778in}}{\pgfqpoint{0.000000in}{0.000000in}}{%
\pgfpathmoveto{\pgfqpoint{0.000000in}{0.000000in}}%
\pgfpathlineto{\pgfqpoint{0.000000in}{-0.027778in}}%
\pgfusepath{stroke,fill}%
}%
\begin{pgfscope}%
\pgfsys@transformshift{5.409902in}{2.700000in}%
\pgfsys@useobject{currentmarker}{}%
\end{pgfscope}%
\end{pgfscope}%
\begin{pgfscope}%
\pgfsetbuttcap%
\pgfsetroundjoin%
\definecolor{currentfill}{rgb}{0.000000,0.000000,0.000000}%
\pgfsetfillcolor{currentfill}%
\pgfsetlinewidth{0.501875pt}%
\definecolor{currentstroke}{rgb}{0.000000,0.000000,0.000000}%
\pgfsetstrokecolor{currentstroke}%
\pgfsetdash{}{0pt}%
\pgfsys@defobject{currentmarker}{\pgfqpoint{0.000000in}{0.000000in}}{\pgfqpoint{0.000000in}{0.027778in}}{%
\pgfpathmoveto{\pgfqpoint{0.000000in}{0.000000in}}%
\pgfpathlineto{\pgfqpoint{0.000000in}{0.027778in}}%
\pgfusepath{stroke,fill}%
}%
\begin{pgfscope}%
\pgfsys@transformshift{5.499789in}{0.300000in}%
\pgfsys@useobject{currentmarker}{}%
\end{pgfscope}%
\end{pgfscope}%
\begin{pgfscope}%
\pgfsetbuttcap%
\pgfsetroundjoin%
\definecolor{currentfill}{rgb}{0.000000,0.000000,0.000000}%
\pgfsetfillcolor{currentfill}%
\pgfsetlinewidth{0.501875pt}%
\definecolor{currentstroke}{rgb}{0.000000,0.000000,0.000000}%
\pgfsetstrokecolor{currentstroke}%
\pgfsetdash{}{0pt}%
\pgfsys@defobject{currentmarker}{\pgfqpoint{0.000000in}{-0.027778in}}{\pgfqpoint{0.000000in}{0.000000in}}{%
\pgfpathmoveto{\pgfqpoint{0.000000in}{0.000000in}}%
\pgfpathlineto{\pgfqpoint{0.000000in}{-0.027778in}}%
\pgfusepath{stroke,fill}%
}%
\begin{pgfscope}%
\pgfsys@transformshift{5.499789in}{2.700000in}%
\pgfsys@useobject{currentmarker}{}%
\end{pgfscope}%
\end{pgfscope}%
\begin{pgfscope}%
\pgfsetbuttcap%
\pgfsetroundjoin%
\definecolor{currentfill}{rgb}{0.000000,0.000000,0.000000}%
\pgfsetfillcolor{currentfill}%
\pgfsetlinewidth{0.501875pt}%
\definecolor{currentstroke}{rgb}{0.000000,0.000000,0.000000}%
\pgfsetstrokecolor{currentstroke}%
\pgfsetdash{}{0pt}%
\pgfsys@defobject{currentmarker}{\pgfqpoint{0.000000in}{0.000000in}}{\pgfqpoint{0.000000in}{0.027778in}}{%
\pgfpathmoveto{\pgfqpoint{0.000000in}{0.000000in}}%
\pgfpathlineto{\pgfqpoint{0.000000in}{0.027778in}}%
\pgfusepath{stroke,fill}%
}%
\begin{pgfscope}%
\pgfsys@transformshift{5.579076in}{0.300000in}%
\pgfsys@useobject{currentmarker}{}%
\end{pgfscope}%
\end{pgfscope}%
\begin{pgfscope}%
\pgfsetbuttcap%
\pgfsetroundjoin%
\definecolor{currentfill}{rgb}{0.000000,0.000000,0.000000}%
\pgfsetfillcolor{currentfill}%
\pgfsetlinewidth{0.501875pt}%
\definecolor{currentstroke}{rgb}{0.000000,0.000000,0.000000}%
\pgfsetstrokecolor{currentstroke}%
\pgfsetdash{}{0pt}%
\pgfsys@defobject{currentmarker}{\pgfqpoint{0.000000in}{-0.027778in}}{\pgfqpoint{0.000000in}{0.000000in}}{%
\pgfpathmoveto{\pgfqpoint{0.000000in}{0.000000in}}%
\pgfpathlineto{\pgfqpoint{0.000000in}{-0.027778in}}%
\pgfusepath{stroke,fill}%
}%
\begin{pgfscope}%
\pgfsys@transformshift{5.579076in}{2.700000in}%
\pgfsys@useobject{currentmarker}{}%
\end{pgfscope}%
\end{pgfscope}%
\begin{pgfscope}%
\pgfsetbuttcap%
\pgfsetroundjoin%
\definecolor{currentfill}{rgb}{0.000000,0.000000,0.000000}%
\pgfsetfillcolor{currentfill}%
\pgfsetlinewidth{0.501875pt}%
\definecolor{currentstroke}{rgb}{0.000000,0.000000,0.000000}%
\pgfsetstrokecolor{currentstroke}%
\pgfsetdash{}{0pt}%
\pgfsys@defobject{currentmarker}{\pgfqpoint{0.000000in}{0.000000in}}{\pgfqpoint{0.000000in}{0.027778in}}{%
\pgfpathmoveto{\pgfqpoint{0.000000in}{0.000000in}}%
\pgfpathlineto{\pgfqpoint{0.000000in}{0.027778in}}%
\pgfusepath{stroke,fill}%
}%
\begin{pgfscope}%
\pgfsys@transformshift{6.116596in}{0.300000in}%
\pgfsys@useobject{currentmarker}{}%
\end{pgfscope}%
\end{pgfscope}%
\begin{pgfscope}%
\pgfsetbuttcap%
\pgfsetroundjoin%
\definecolor{currentfill}{rgb}{0.000000,0.000000,0.000000}%
\pgfsetfillcolor{currentfill}%
\pgfsetlinewidth{0.501875pt}%
\definecolor{currentstroke}{rgb}{0.000000,0.000000,0.000000}%
\pgfsetstrokecolor{currentstroke}%
\pgfsetdash{}{0pt}%
\pgfsys@defobject{currentmarker}{\pgfqpoint{0.000000in}{-0.027778in}}{\pgfqpoint{0.000000in}{0.000000in}}{%
\pgfpathmoveto{\pgfqpoint{0.000000in}{0.000000in}}%
\pgfpathlineto{\pgfqpoint{0.000000in}{-0.027778in}}%
\pgfusepath{stroke,fill}%
}%
\begin{pgfscope}%
\pgfsys@transformshift{6.116596in}{2.700000in}%
\pgfsys@useobject{currentmarker}{}%
\end{pgfscope}%
\end{pgfscope}%
\begin{pgfscope}%
\pgfsetbuttcap%
\pgfsetroundjoin%
\definecolor{currentfill}{rgb}{0.000000,0.000000,0.000000}%
\pgfsetfillcolor{currentfill}%
\pgfsetlinewidth{0.501875pt}%
\definecolor{currentstroke}{rgb}{0.000000,0.000000,0.000000}%
\pgfsetstrokecolor{currentstroke}%
\pgfsetdash{}{0pt}%
\pgfsys@defobject{currentmarker}{\pgfqpoint{0.000000in}{0.000000in}}{\pgfqpoint{0.000000in}{0.027778in}}{%
\pgfpathmoveto{\pgfqpoint{0.000000in}{0.000000in}}%
\pgfpathlineto{\pgfqpoint{0.000000in}{0.027778in}}%
\pgfusepath{stroke,fill}%
}%
\begin{pgfscope}%
\pgfsys@transformshift{6.389538in}{0.300000in}%
\pgfsys@useobject{currentmarker}{}%
\end{pgfscope}%
\end{pgfscope}%
\begin{pgfscope}%
\pgfsetbuttcap%
\pgfsetroundjoin%
\definecolor{currentfill}{rgb}{0.000000,0.000000,0.000000}%
\pgfsetfillcolor{currentfill}%
\pgfsetlinewidth{0.501875pt}%
\definecolor{currentstroke}{rgb}{0.000000,0.000000,0.000000}%
\pgfsetstrokecolor{currentstroke}%
\pgfsetdash{}{0pt}%
\pgfsys@defobject{currentmarker}{\pgfqpoint{0.000000in}{-0.027778in}}{\pgfqpoint{0.000000in}{0.000000in}}{%
\pgfpathmoveto{\pgfqpoint{0.000000in}{0.000000in}}%
\pgfpathlineto{\pgfqpoint{0.000000in}{-0.027778in}}%
\pgfusepath{stroke,fill}%
}%
\begin{pgfscope}%
\pgfsys@transformshift{6.389538in}{2.700000in}%
\pgfsys@useobject{currentmarker}{}%
\end{pgfscope}%
\end{pgfscope}%
\begin{pgfscope}%
\pgfsetbuttcap%
\pgfsetroundjoin%
\definecolor{currentfill}{rgb}{0.000000,0.000000,0.000000}%
\pgfsetfillcolor{currentfill}%
\pgfsetlinewidth{0.501875pt}%
\definecolor{currentstroke}{rgb}{0.000000,0.000000,0.000000}%
\pgfsetstrokecolor{currentstroke}%
\pgfsetdash{}{0pt}%
\pgfsys@defobject{currentmarker}{\pgfqpoint{0.000000in}{0.000000in}}{\pgfqpoint{0.000000in}{0.027778in}}{%
\pgfpathmoveto{\pgfqpoint{0.000000in}{0.000000in}}%
\pgfpathlineto{\pgfqpoint{0.000000in}{0.027778in}}%
\pgfusepath{stroke,fill}%
}%
\begin{pgfscope}%
\pgfsys@transformshift{6.583193in}{0.300000in}%
\pgfsys@useobject{currentmarker}{}%
\end{pgfscope}%
\end{pgfscope}%
\begin{pgfscope}%
\pgfsetbuttcap%
\pgfsetroundjoin%
\definecolor{currentfill}{rgb}{0.000000,0.000000,0.000000}%
\pgfsetfillcolor{currentfill}%
\pgfsetlinewidth{0.501875pt}%
\definecolor{currentstroke}{rgb}{0.000000,0.000000,0.000000}%
\pgfsetstrokecolor{currentstroke}%
\pgfsetdash{}{0pt}%
\pgfsys@defobject{currentmarker}{\pgfqpoint{0.000000in}{-0.027778in}}{\pgfqpoint{0.000000in}{0.000000in}}{%
\pgfpathmoveto{\pgfqpoint{0.000000in}{0.000000in}}%
\pgfpathlineto{\pgfqpoint{0.000000in}{-0.027778in}}%
\pgfusepath{stroke,fill}%
}%
\begin{pgfscope}%
\pgfsys@transformshift{6.583193in}{2.700000in}%
\pgfsys@useobject{currentmarker}{}%
\end{pgfscope}%
\end{pgfscope}%
\begin{pgfscope}%
\pgfsetbuttcap%
\pgfsetroundjoin%
\definecolor{currentfill}{rgb}{0.000000,0.000000,0.000000}%
\pgfsetfillcolor{currentfill}%
\pgfsetlinewidth{0.501875pt}%
\definecolor{currentstroke}{rgb}{0.000000,0.000000,0.000000}%
\pgfsetstrokecolor{currentstroke}%
\pgfsetdash{}{0pt}%
\pgfsys@defobject{currentmarker}{\pgfqpoint{0.000000in}{0.000000in}}{\pgfqpoint{0.000000in}{0.027778in}}{%
\pgfpathmoveto{\pgfqpoint{0.000000in}{0.000000in}}%
\pgfpathlineto{\pgfqpoint{0.000000in}{0.027778in}}%
\pgfusepath{stroke,fill}%
}%
\begin{pgfscope}%
\pgfsys@transformshift{6.733404in}{0.300000in}%
\pgfsys@useobject{currentmarker}{}%
\end{pgfscope}%
\end{pgfscope}%
\begin{pgfscope}%
\pgfsetbuttcap%
\pgfsetroundjoin%
\definecolor{currentfill}{rgb}{0.000000,0.000000,0.000000}%
\pgfsetfillcolor{currentfill}%
\pgfsetlinewidth{0.501875pt}%
\definecolor{currentstroke}{rgb}{0.000000,0.000000,0.000000}%
\pgfsetstrokecolor{currentstroke}%
\pgfsetdash{}{0pt}%
\pgfsys@defobject{currentmarker}{\pgfqpoint{0.000000in}{-0.027778in}}{\pgfqpoint{0.000000in}{0.000000in}}{%
\pgfpathmoveto{\pgfqpoint{0.000000in}{0.000000in}}%
\pgfpathlineto{\pgfqpoint{0.000000in}{-0.027778in}}%
\pgfusepath{stroke,fill}%
}%
\begin{pgfscope}%
\pgfsys@transformshift{6.733404in}{2.700000in}%
\pgfsys@useobject{currentmarker}{}%
\end{pgfscope}%
\end{pgfscope}%
\begin{pgfscope}%
\pgfsetbuttcap%
\pgfsetroundjoin%
\definecolor{currentfill}{rgb}{0.000000,0.000000,0.000000}%
\pgfsetfillcolor{currentfill}%
\pgfsetlinewidth{0.501875pt}%
\definecolor{currentstroke}{rgb}{0.000000,0.000000,0.000000}%
\pgfsetstrokecolor{currentstroke}%
\pgfsetdash{}{0pt}%
\pgfsys@defobject{currentmarker}{\pgfqpoint{0.000000in}{0.000000in}}{\pgfqpoint{0.000000in}{0.027778in}}{%
\pgfpathmoveto{\pgfqpoint{0.000000in}{0.000000in}}%
\pgfpathlineto{\pgfqpoint{0.000000in}{0.027778in}}%
\pgfusepath{stroke,fill}%
}%
\begin{pgfscope}%
\pgfsys@transformshift{6.856134in}{0.300000in}%
\pgfsys@useobject{currentmarker}{}%
\end{pgfscope}%
\end{pgfscope}%
\begin{pgfscope}%
\pgfsetbuttcap%
\pgfsetroundjoin%
\definecolor{currentfill}{rgb}{0.000000,0.000000,0.000000}%
\pgfsetfillcolor{currentfill}%
\pgfsetlinewidth{0.501875pt}%
\definecolor{currentstroke}{rgb}{0.000000,0.000000,0.000000}%
\pgfsetstrokecolor{currentstroke}%
\pgfsetdash{}{0pt}%
\pgfsys@defobject{currentmarker}{\pgfqpoint{0.000000in}{-0.027778in}}{\pgfqpoint{0.000000in}{0.000000in}}{%
\pgfpathmoveto{\pgfqpoint{0.000000in}{0.000000in}}%
\pgfpathlineto{\pgfqpoint{0.000000in}{-0.027778in}}%
\pgfusepath{stroke,fill}%
}%
\begin{pgfscope}%
\pgfsys@transformshift{6.856134in}{2.700000in}%
\pgfsys@useobject{currentmarker}{}%
\end{pgfscope}%
\end{pgfscope}%
\begin{pgfscope}%
\pgfsetbuttcap%
\pgfsetroundjoin%
\definecolor{currentfill}{rgb}{0.000000,0.000000,0.000000}%
\pgfsetfillcolor{currentfill}%
\pgfsetlinewidth{0.501875pt}%
\definecolor{currentstroke}{rgb}{0.000000,0.000000,0.000000}%
\pgfsetstrokecolor{currentstroke}%
\pgfsetdash{}{0pt}%
\pgfsys@defobject{currentmarker}{\pgfqpoint{0.000000in}{0.000000in}}{\pgfqpoint{0.000000in}{0.027778in}}{%
\pgfpathmoveto{\pgfqpoint{0.000000in}{0.000000in}}%
\pgfpathlineto{\pgfqpoint{0.000000in}{0.027778in}}%
\pgfusepath{stroke,fill}%
}%
\begin{pgfscope}%
\pgfsys@transformshift{6.959902in}{0.300000in}%
\pgfsys@useobject{currentmarker}{}%
\end{pgfscope}%
\end{pgfscope}%
\begin{pgfscope}%
\pgfsetbuttcap%
\pgfsetroundjoin%
\definecolor{currentfill}{rgb}{0.000000,0.000000,0.000000}%
\pgfsetfillcolor{currentfill}%
\pgfsetlinewidth{0.501875pt}%
\definecolor{currentstroke}{rgb}{0.000000,0.000000,0.000000}%
\pgfsetstrokecolor{currentstroke}%
\pgfsetdash{}{0pt}%
\pgfsys@defobject{currentmarker}{\pgfqpoint{0.000000in}{-0.027778in}}{\pgfqpoint{0.000000in}{0.000000in}}{%
\pgfpathmoveto{\pgfqpoint{0.000000in}{0.000000in}}%
\pgfpathlineto{\pgfqpoint{0.000000in}{-0.027778in}}%
\pgfusepath{stroke,fill}%
}%
\begin{pgfscope}%
\pgfsys@transformshift{6.959902in}{2.700000in}%
\pgfsys@useobject{currentmarker}{}%
\end{pgfscope}%
\end{pgfscope}%
\begin{pgfscope}%
\pgfsetbuttcap%
\pgfsetroundjoin%
\definecolor{currentfill}{rgb}{0.000000,0.000000,0.000000}%
\pgfsetfillcolor{currentfill}%
\pgfsetlinewidth{0.501875pt}%
\definecolor{currentstroke}{rgb}{0.000000,0.000000,0.000000}%
\pgfsetstrokecolor{currentstroke}%
\pgfsetdash{}{0pt}%
\pgfsys@defobject{currentmarker}{\pgfqpoint{0.000000in}{0.000000in}}{\pgfqpoint{0.000000in}{0.027778in}}{%
\pgfpathmoveto{\pgfqpoint{0.000000in}{0.000000in}}%
\pgfpathlineto{\pgfqpoint{0.000000in}{0.027778in}}%
\pgfusepath{stroke,fill}%
}%
\begin{pgfscope}%
\pgfsys@transformshift{7.049789in}{0.300000in}%
\pgfsys@useobject{currentmarker}{}%
\end{pgfscope}%
\end{pgfscope}%
\begin{pgfscope}%
\pgfsetbuttcap%
\pgfsetroundjoin%
\definecolor{currentfill}{rgb}{0.000000,0.000000,0.000000}%
\pgfsetfillcolor{currentfill}%
\pgfsetlinewidth{0.501875pt}%
\definecolor{currentstroke}{rgb}{0.000000,0.000000,0.000000}%
\pgfsetstrokecolor{currentstroke}%
\pgfsetdash{}{0pt}%
\pgfsys@defobject{currentmarker}{\pgfqpoint{0.000000in}{-0.027778in}}{\pgfqpoint{0.000000in}{0.000000in}}{%
\pgfpathmoveto{\pgfqpoint{0.000000in}{0.000000in}}%
\pgfpathlineto{\pgfqpoint{0.000000in}{-0.027778in}}%
\pgfusepath{stroke,fill}%
}%
\begin{pgfscope}%
\pgfsys@transformshift{7.049789in}{2.700000in}%
\pgfsys@useobject{currentmarker}{}%
\end{pgfscope}%
\end{pgfscope}%
\begin{pgfscope}%
\pgfsetbuttcap%
\pgfsetroundjoin%
\definecolor{currentfill}{rgb}{0.000000,0.000000,0.000000}%
\pgfsetfillcolor{currentfill}%
\pgfsetlinewidth{0.501875pt}%
\definecolor{currentstroke}{rgb}{0.000000,0.000000,0.000000}%
\pgfsetstrokecolor{currentstroke}%
\pgfsetdash{}{0pt}%
\pgfsys@defobject{currentmarker}{\pgfqpoint{0.000000in}{0.000000in}}{\pgfqpoint{0.000000in}{0.027778in}}{%
\pgfpathmoveto{\pgfqpoint{0.000000in}{0.000000in}}%
\pgfpathlineto{\pgfqpoint{0.000000in}{0.027778in}}%
\pgfusepath{stroke,fill}%
}%
\begin{pgfscope}%
\pgfsys@transformshift{7.129076in}{0.300000in}%
\pgfsys@useobject{currentmarker}{}%
\end{pgfscope}%
\end{pgfscope}%
\begin{pgfscope}%
\pgfsetbuttcap%
\pgfsetroundjoin%
\definecolor{currentfill}{rgb}{0.000000,0.000000,0.000000}%
\pgfsetfillcolor{currentfill}%
\pgfsetlinewidth{0.501875pt}%
\definecolor{currentstroke}{rgb}{0.000000,0.000000,0.000000}%
\pgfsetstrokecolor{currentstroke}%
\pgfsetdash{}{0pt}%
\pgfsys@defobject{currentmarker}{\pgfqpoint{0.000000in}{-0.027778in}}{\pgfqpoint{0.000000in}{0.000000in}}{%
\pgfpathmoveto{\pgfqpoint{0.000000in}{0.000000in}}%
\pgfpathlineto{\pgfqpoint{0.000000in}{-0.027778in}}%
\pgfusepath{stroke,fill}%
}%
\begin{pgfscope}%
\pgfsys@transformshift{7.129076in}{2.700000in}%
\pgfsys@useobject{currentmarker}{}%
\end{pgfscope}%
\end{pgfscope}%
\begin{pgfscope}%
\pgftext[left,bottom,x=3.723901in,y=-0.126716in,rotate=0.000000]{{\sffamily\fontsize{12.000000}{14.400000}\selectfont time [ps]}}
%
\end{pgfscope}%
\begin{pgfscope}%
\pgfpathrectangle{\pgfqpoint{1.000000in}{0.300000in}}{\pgfqpoint{6.200000in}{2.400000in}} %
\pgfusepath{clip}%
\pgfsetbuttcap%
\pgfsetroundjoin%
\pgfsetlinewidth{0.501875pt}%
\definecolor{currentstroke}{rgb}{0.000000,0.000000,0.000000}%
\pgfsetstrokecolor{currentstroke}%
\pgfsetdash{{1.000000pt}{3.000000pt}}{0.000000pt}%
\pgfpathmoveto{\pgfqpoint{1.000000in}{0.300000in}}%
\pgfpathlineto{\pgfqpoint{7.200000in}{0.300000in}}%
\pgfusepath{stroke}%
\end{pgfscope}%
\begin{pgfscope}%
\pgfsetbuttcap%
\pgfsetroundjoin%
\definecolor{currentfill}{rgb}{0.000000,0.000000,0.000000}%
\pgfsetfillcolor{currentfill}%
\pgfsetlinewidth{0.501875pt}%
\definecolor{currentstroke}{rgb}{0.000000,0.000000,0.000000}%
\pgfsetstrokecolor{currentstroke}%
\pgfsetdash{}{0pt}%
\pgfsys@defobject{currentmarker}{\pgfqpoint{0.000000in}{0.000000in}}{\pgfqpoint{0.055556in}{0.000000in}}{%
\pgfpathmoveto{\pgfqpoint{0.000000in}{0.000000in}}%
\pgfpathlineto{\pgfqpoint{0.055556in}{0.000000in}}%
\pgfusepath{stroke,fill}%
}%
\begin{pgfscope}%
\pgfsys@transformshift{1.000000in}{0.300000in}%
\pgfsys@useobject{currentmarker}{}%
\end{pgfscope}%
\end{pgfscope}%
\begin{pgfscope}%
\pgfsetbuttcap%
\pgfsetroundjoin%
\definecolor{currentfill}{rgb}{0.000000,0.000000,0.000000}%
\pgfsetfillcolor{currentfill}%
\pgfsetlinewidth{0.501875pt}%
\definecolor{currentstroke}{rgb}{0.000000,0.000000,0.000000}%
\pgfsetstrokecolor{currentstroke}%
\pgfsetdash{}{0pt}%
\pgfsys@defobject{currentmarker}{\pgfqpoint{-0.055556in}{0.000000in}}{\pgfqpoint{0.000000in}{0.000000in}}{%
\pgfpathmoveto{\pgfqpoint{0.000000in}{0.000000in}}%
\pgfpathlineto{\pgfqpoint{-0.055556in}{0.000000in}}%
\pgfusepath{stroke,fill}%
}%
\begin{pgfscope}%
\pgfsys@transformshift{7.200000in}{0.300000in}%
\pgfsys@useobject{currentmarker}{}%
\end{pgfscope}%
\end{pgfscope}%
\begin{pgfscope}%
\pgftext[left,bottom,x=0.623456in,y=0.229790in,rotate=0.000000]{{\sffamily\fontsize{12.000000}{14.400000}\selectfont \(\displaystyle {10^{-3}}\)}}
%
\end{pgfscope}%
\begin{pgfscope}%
\pgfpathrectangle{\pgfqpoint{1.000000in}{0.300000in}}{\pgfqpoint{6.200000in}{2.400000in}} %
\pgfusepath{clip}%
\pgfsetbuttcap%
\pgfsetroundjoin%
\pgfsetlinewidth{0.501875pt}%
\definecolor{currentstroke}{rgb}{0.000000,0.000000,0.000000}%
\pgfsetstrokecolor{currentstroke}%
\pgfsetdash{{1.000000pt}{3.000000pt}}{0.000000pt}%
\pgfpathmoveto{\pgfqpoint{1.000000in}{0.780000in}}%
\pgfpathlineto{\pgfqpoint{7.200000in}{0.780000in}}%
\pgfusepath{stroke}%
\end{pgfscope}%
\begin{pgfscope}%
\pgfsetbuttcap%
\pgfsetroundjoin%
\definecolor{currentfill}{rgb}{0.000000,0.000000,0.000000}%
\pgfsetfillcolor{currentfill}%
\pgfsetlinewidth{0.501875pt}%
\definecolor{currentstroke}{rgb}{0.000000,0.000000,0.000000}%
\pgfsetstrokecolor{currentstroke}%
\pgfsetdash{}{0pt}%
\pgfsys@defobject{currentmarker}{\pgfqpoint{0.000000in}{0.000000in}}{\pgfqpoint{0.055556in}{0.000000in}}{%
\pgfpathmoveto{\pgfqpoint{0.000000in}{0.000000in}}%
\pgfpathlineto{\pgfqpoint{0.055556in}{0.000000in}}%
\pgfusepath{stroke,fill}%
}%
\begin{pgfscope}%
\pgfsys@transformshift{1.000000in}{0.780000in}%
\pgfsys@useobject{currentmarker}{}%
\end{pgfscope}%
\end{pgfscope}%
\begin{pgfscope}%
\pgfsetbuttcap%
\pgfsetroundjoin%
\definecolor{currentfill}{rgb}{0.000000,0.000000,0.000000}%
\pgfsetfillcolor{currentfill}%
\pgfsetlinewidth{0.501875pt}%
\definecolor{currentstroke}{rgb}{0.000000,0.000000,0.000000}%
\pgfsetstrokecolor{currentstroke}%
\pgfsetdash{}{0pt}%
\pgfsys@defobject{currentmarker}{\pgfqpoint{-0.055556in}{0.000000in}}{\pgfqpoint{0.000000in}{0.000000in}}{%
\pgfpathmoveto{\pgfqpoint{0.000000in}{0.000000in}}%
\pgfpathlineto{\pgfqpoint{-0.055556in}{0.000000in}}%
\pgfusepath{stroke,fill}%
}%
\begin{pgfscope}%
\pgfsys@transformshift{7.200000in}{0.780000in}%
\pgfsys@useobject{currentmarker}{}%
\end{pgfscope}%
\end{pgfscope}%
\begin{pgfscope}%
\pgftext[left,bottom,x=0.623456in,y=0.709790in,rotate=0.000000]{{\sffamily\fontsize{12.000000}{14.400000}\selectfont \(\displaystyle {10^{-2}}\)}}
%
\end{pgfscope}%
\begin{pgfscope}%
\pgfpathrectangle{\pgfqpoint{1.000000in}{0.300000in}}{\pgfqpoint{6.200000in}{2.400000in}} %
\pgfusepath{clip}%
\pgfsetbuttcap%
\pgfsetroundjoin%
\pgfsetlinewidth{0.501875pt}%
\definecolor{currentstroke}{rgb}{0.000000,0.000000,0.000000}%
\pgfsetstrokecolor{currentstroke}%
\pgfsetdash{{1.000000pt}{3.000000pt}}{0.000000pt}%
\pgfpathmoveto{\pgfqpoint{1.000000in}{1.260000in}}%
\pgfpathlineto{\pgfqpoint{7.200000in}{1.260000in}}%
\pgfusepath{stroke}%
\end{pgfscope}%
\begin{pgfscope}%
\pgfsetbuttcap%
\pgfsetroundjoin%
\definecolor{currentfill}{rgb}{0.000000,0.000000,0.000000}%
\pgfsetfillcolor{currentfill}%
\pgfsetlinewidth{0.501875pt}%
\definecolor{currentstroke}{rgb}{0.000000,0.000000,0.000000}%
\pgfsetstrokecolor{currentstroke}%
\pgfsetdash{}{0pt}%
\pgfsys@defobject{currentmarker}{\pgfqpoint{0.000000in}{0.000000in}}{\pgfqpoint{0.055556in}{0.000000in}}{%
\pgfpathmoveto{\pgfqpoint{0.000000in}{0.000000in}}%
\pgfpathlineto{\pgfqpoint{0.055556in}{0.000000in}}%
\pgfusepath{stroke,fill}%
}%
\begin{pgfscope}%
\pgfsys@transformshift{1.000000in}{1.260000in}%
\pgfsys@useobject{currentmarker}{}%
\end{pgfscope}%
\end{pgfscope}%
\begin{pgfscope}%
\pgfsetbuttcap%
\pgfsetroundjoin%
\definecolor{currentfill}{rgb}{0.000000,0.000000,0.000000}%
\pgfsetfillcolor{currentfill}%
\pgfsetlinewidth{0.501875pt}%
\definecolor{currentstroke}{rgb}{0.000000,0.000000,0.000000}%
\pgfsetstrokecolor{currentstroke}%
\pgfsetdash{}{0pt}%
\pgfsys@defobject{currentmarker}{\pgfqpoint{-0.055556in}{0.000000in}}{\pgfqpoint{0.000000in}{0.000000in}}{%
\pgfpathmoveto{\pgfqpoint{0.000000in}{0.000000in}}%
\pgfpathlineto{\pgfqpoint{-0.055556in}{0.000000in}}%
\pgfusepath{stroke,fill}%
}%
\begin{pgfscope}%
\pgfsys@transformshift{7.200000in}{1.260000in}%
\pgfsys@useobject{currentmarker}{}%
\end{pgfscope}%
\end{pgfscope}%
\begin{pgfscope}%
\pgftext[left,bottom,x=0.623456in,y=1.189790in,rotate=0.000000]{{\sffamily\fontsize{12.000000}{14.400000}\selectfont \(\displaystyle {10^{-1}}\)}}
%
\end{pgfscope}%
\begin{pgfscope}%
\pgfpathrectangle{\pgfqpoint{1.000000in}{0.300000in}}{\pgfqpoint{6.200000in}{2.400000in}} %
\pgfusepath{clip}%
\pgfsetbuttcap%
\pgfsetroundjoin%
\pgfsetlinewidth{0.501875pt}%
\definecolor{currentstroke}{rgb}{0.000000,0.000000,0.000000}%
\pgfsetstrokecolor{currentstroke}%
\pgfsetdash{{1.000000pt}{3.000000pt}}{0.000000pt}%
\pgfpathmoveto{\pgfqpoint{1.000000in}{1.740000in}}%
\pgfpathlineto{\pgfqpoint{7.200000in}{1.740000in}}%
\pgfusepath{stroke}%
\end{pgfscope}%
\begin{pgfscope}%
\pgfsetbuttcap%
\pgfsetroundjoin%
\definecolor{currentfill}{rgb}{0.000000,0.000000,0.000000}%
\pgfsetfillcolor{currentfill}%
\pgfsetlinewidth{0.501875pt}%
\definecolor{currentstroke}{rgb}{0.000000,0.000000,0.000000}%
\pgfsetstrokecolor{currentstroke}%
\pgfsetdash{}{0pt}%
\pgfsys@defobject{currentmarker}{\pgfqpoint{0.000000in}{0.000000in}}{\pgfqpoint{0.055556in}{0.000000in}}{%
\pgfpathmoveto{\pgfqpoint{0.000000in}{0.000000in}}%
\pgfpathlineto{\pgfqpoint{0.055556in}{0.000000in}}%
\pgfusepath{stroke,fill}%
}%
\begin{pgfscope}%
\pgfsys@transformshift{1.000000in}{1.740000in}%
\pgfsys@useobject{currentmarker}{}%
\end{pgfscope}%
\end{pgfscope}%
\begin{pgfscope}%
\pgfsetbuttcap%
\pgfsetroundjoin%
\definecolor{currentfill}{rgb}{0.000000,0.000000,0.000000}%
\pgfsetfillcolor{currentfill}%
\pgfsetlinewidth{0.501875pt}%
\definecolor{currentstroke}{rgb}{0.000000,0.000000,0.000000}%
\pgfsetstrokecolor{currentstroke}%
\pgfsetdash{}{0pt}%
\pgfsys@defobject{currentmarker}{\pgfqpoint{-0.055556in}{0.000000in}}{\pgfqpoint{0.000000in}{0.000000in}}{%
\pgfpathmoveto{\pgfqpoint{0.000000in}{0.000000in}}%
\pgfpathlineto{\pgfqpoint{-0.055556in}{0.000000in}}%
\pgfusepath{stroke,fill}%
}%
\begin{pgfscope}%
\pgfsys@transformshift{7.200000in}{1.740000in}%
\pgfsys@useobject{currentmarker}{}%
\end{pgfscope}%
\end{pgfscope}%
\begin{pgfscope}%
\pgftext[left,bottom,x=0.715279in,y=1.669790in,rotate=0.000000]{{\sffamily\fontsize{12.000000}{14.400000}\selectfont \(\displaystyle {10^{0}}\)}}
%
\end{pgfscope}%
\begin{pgfscope}%
\pgfpathrectangle{\pgfqpoint{1.000000in}{0.300000in}}{\pgfqpoint{6.200000in}{2.400000in}} %
\pgfusepath{clip}%
\pgfsetbuttcap%
\pgfsetroundjoin%
\pgfsetlinewidth{0.501875pt}%
\definecolor{currentstroke}{rgb}{0.000000,0.000000,0.000000}%
\pgfsetstrokecolor{currentstroke}%
\pgfsetdash{{1.000000pt}{3.000000pt}}{0.000000pt}%
\pgfpathmoveto{\pgfqpoint{1.000000in}{2.220000in}}%
\pgfpathlineto{\pgfqpoint{7.200000in}{2.220000in}}%
\pgfusepath{stroke}%
\end{pgfscope}%
\begin{pgfscope}%
\pgfsetbuttcap%
\pgfsetroundjoin%
\definecolor{currentfill}{rgb}{0.000000,0.000000,0.000000}%
\pgfsetfillcolor{currentfill}%
\pgfsetlinewidth{0.501875pt}%
\definecolor{currentstroke}{rgb}{0.000000,0.000000,0.000000}%
\pgfsetstrokecolor{currentstroke}%
\pgfsetdash{}{0pt}%
\pgfsys@defobject{currentmarker}{\pgfqpoint{0.000000in}{0.000000in}}{\pgfqpoint{0.055556in}{0.000000in}}{%
\pgfpathmoveto{\pgfqpoint{0.000000in}{0.000000in}}%
\pgfpathlineto{\pgfqpoint{0.055556in}{0.000000in}}%
\pgfusepath{stroke,fill}%
}%
\begin{pgfscope}%
\pgfsys@transformshift{1.000000in}{2.220000in}%
\pgfsys@useobject{currentmarker}{}%
\end{pgfscope}%
\end{pgfscope}%
\begin{pgfscope}%
\pgfsetbuttcap%
\pgfsetroundjoin%
\definecolor{currentfill}{rgb}{0.000000,0.000000,0.000000}%
\pgfsetfillcolor{currentfill}%
\pgfsetlinewidth{0.501875pt}%
\definecolor{currentstroke}{rgb}{0.000000,0.000000,0.000000}%
\pgfsetstrokecolor{currentstroke}%
\pgfsetdash{}{0pt}%
\pgfsys@defobject{currentmarker}{\pgfqpoint{-0.055556in}{0.000000in}}{\pgfqpoint{0.000000in}{0.000000in}}{%
\pgfpathmoveto{\pgfqpoint{0.000000in}{0.000000in}}%
\pgfpathlineto{\pgfqpoint{-0.055556in}{0.000000in}}%
\pgfusepath{stroke,fill}%
}%
\begin{pgfscope}%
\pgfsys@transformshift{7.200000in}{2.220000in}%
\pgfsys@useobject{currentmarker}{}%
\end{pgfscope}%
\end{pgfscope}%
\begin{pgfscope}%
\pgftext[left,bottom,x=0.715279in,y=2.149790in,rotate=0.000000]{{\sffamily\fontsize{12.000000}{14.400000}\selectfont \(\displaystyle {10^{1}}\)}}
%
\end{pgfscope}%
\begin{pgfscope}%
\pgfpathrectangle{\pgfqpoint{1.000000in}{0.300000in}}{\pgfqpoint{6.200000in}{2.400000in}} %
\pgfusepath{clip}%
\pgfsetbuttcap%
\pgfsetroundjoin%
\pgfsetlinewidth{0.501875pt}%
\definecolor{currentstroke}{rgb}{0.000000,0.000000,0.000000}%
\pgfsetstrokecolor{currentstroke}%
\pgfsetdash{{1.000000pt}{3.000000pt}}{0.000000pt}%
\pgfpathmoveto{\pgfqpoint{1.000000in}{2.700000in}}%
\pgfpathlineto{\pgfqpoint{7.200000in}{2.700000in}}%
\pgfusepath{stroke}%
\end{pgfscope}%
\begin{pgfscope}%
\pgfsetbuttcap%
\pgfsetroundjoin%
\definecolor{currentfill}{rgb}{0.000000,0.000000,0.000000}%
\pgfsetfillcolor{currentfill}%
\pgfsetlinewidth{0.501875pt}%
\definecolor{currentstroke}{rgb}{0.000000,0.000000,0.000000}%
\pgfsetstrokecolor{currentstroke}%
\pgfsetdash{}{0pt}%
\pgfsys@defobject{currentmarker}{\pgfqpoint{0.000000in}{0.000000in}}{\pgfqpoint{0.055556in}{0.000000in}}{%
\pgfpathmoveto{\pgfqpoint{0.000000in}{0.000000in}}%
\pgfpathlineto{\pgfqpoint{0.055556in}{0.000000in}}%
\pgfusepath{stroke,fill}%
}%
\begin{pgfscope}%
\pgfsys@transformshift{1.000000in}{2.700000in}%
\pgfsys@useobject{currentmarker}{}%
\end{pgfscope}%
\end{pgfscope}%
\begin{pgfscope}%
\pgfsetbuttcap%
\pgfsetroundjoin%
\definecolor{currentfill}{rgb}{0.000000,0.000000,0.000000}%
\pgfsetfillcolor{currentfill}%
\pgfsetlinewidth{0.501875pt}%
\definecolor{currentstroke}{rgb}{0.000000,0.000000,0.000000}%
\pgfsetstrokecolor{currentstroke}%
\pgfsetdash{}{0pt}%
\pgfsys@defobject{currentmarker}{\pgfqpoint{-0.055556in}{0.000000in}}{\pgfqpoint{0.000000in}{0.000000in}}{%
\pgfpathmoveto{\pgfqpoint{0.000000in}{0.000000in}}%
\pgfpathlineto{\pgfqpoint{-0.055556in}{0.000000in}}%
\pgfusepath{stroke,fill}%
}%
\begin{pgfscope}%
\pgfsys@transformshift{7.200000in}{2.700000in}%
\pgfsys@useobject{currentmarker}{}%
\end{pgfscope}%
\end{pgfscope}%
\begin{pgfscope}%
\pgftext[left,bottom,x=0.715279in,y=2.629790in,rotate=0.000000]{{\sffamily\fontsize{12.000000}{14.400000}\selectfont \(\displaystyle {10^{2}}\)}}
%
\end{pgfscope}%
\begin{pgfscope}%
\pgfsetbuttcap%
\pgfsetroundjoin%
\definecolor{currentfill}{rgb}{0.000000,0.000000,0.000000}%
\pgfsetfillcolor{currentfill}%
\pgfsetlinewidth{0.501875pt}%
\definecolor{currentstroke}{rgb}{0.000000,0.000000,0.000000}%
\pgfsetstrokecolor{currentstroke}%
\pgfsetdash{}{0pt}%
\pgfsys@defobject{currentmarker}{\pgfqpoint{0.000000in}{0.000000in}}{\pgfqpoint{0.027778in}{0.000000in}}{%
\pgfpathmoveto{\pgfqpoint{0.000000in}{0.000000in}}%
\pgfpathlineto{\pgfqpoint{0.027778in}{0.000000in}}%
\pgfusepath{stroke,fill}%
}%
\begin{pgfscope}%
\pgfsys@transformshift{1.000000in}{0.444494in}%
\pgfsys@useobject{currentmarker}{}%
\end{pgfscope}%
\end{pgfscope}%
\begin{pgfscope}%
\pgfsetbuttcap%
\pgfsetroundjoin%
\definecolor{currentfill}{rgb}{0.000000,0.000000,0.000000}%
\pgfsetfillcolor{currentfill}%
\pgfsetlinewidth{0.501875pt}%
\definecolor{currentstroke}{rgb}{0.000000,0.000000,0.000000}%
\pgfsetstrokecolor{currentstroke}%
\pgfsetdash{}{0pt}%
\pgfsys@defobject{currentmarker}{\pgfqpoint{-0.027778in}{0.000000in}}{\pgfqpoint{0.000000in}{0.000000in}}{%
\pgfpathmoveto{\pgfqpoint{0.000000in}{0.000000in}}%
\pgfpathlineto{\pgfqpoint{-0.027778in}{0.000000in}}%
\pgfusepath{stroke,fill}%
}%
\begin{pgfscope}%
\pgfsys@transformshift{7.200000in}{0.444494in}%
\pgfsys@useobject{currentmarker}{}%
\end{pgfscope}%
\end{pgfscope}%
\begin{pgfscope}%
\pgfsetbuttcap%
\pgfsetroundjoin%
\definecolor{currentfill}{rgb}{0.000000,0.000000,0.000000}%
\pgfsetfillcolor{currentfill}%
\pgfsetlinewidth{0.501875pt}%
\definecolor{currentstroke}{rgb}{0.000000,0.000000,0.000000}%
\pgfsetstrokecolor{currentstroke}%
\pgfsetdash{}{0pt}%
\pgfsys@defobject{currentmarker}{\pgfqpoint{0.000000in}{0.000000in}}{\pgfqpoint{0.027778in}{0.000000in}}{%
\pgfpathmoveto{\pgfqpoint{0.000000in}{0.000000in}}%
\pgfpathlineto{\pgfqpoint{0.027778in}{0.000000in}}%
\pgfusepath{stroke,fill}%
}%
\begin{pgfscope}%
\pgfsys@transformshift{1.000000in}{0.529018in}%
\pgfsys@useobject{currentmarker}{}%
\end{pgfscope}%
\end{pgfscope}%
\begin{pgfscope}%
\pgfsetbuttcap%
\pgfsetroundjoin%
\definecolor{currentfill}{rgb}{0.000000,0.000000,0.000000}%
\pgfsetfillcolor{currentfill}%
\pgfsetlinewidth{0.501875pt}%
\definecolor{currentstroke}{rgb}{0.000000,0.000000,0.000000}%
\pgfsetstrokecolor{currentstroke}%
\pgfsetdash{}{0pt}%
\pgfsys@defobject{currentmarker}{\pgfqpoint{-0.027778in}{0.000000in}}{\pgfqpoint{0.000000in}{0.000000in}}{%
\pgfpathmoveto{\pgfqpoint{0.000000in}{0.000000in}}%
\pgfpathlineto{\pgfqpoint{-0.027778in}{0.000000in}}%
\pgfusepath{stroke,fill}%
}%
\begin{pgfscope}%
\pgfsys@transformshift{7.200000in}{0.529018in}%
\pgfsys@useobject{currentmarker}{}%
\end{pgfscope}%
\end{pgfscope}%
\begin{pgfscope}%
\pgfsetbuttcap%
\pgfsetroundjoin%
\definecolor{currentfill}{rgb}{0.000000,0.000000,0.000000}%
\pgfsetfillcolor{currentfill}%
\pgfsetlinewidth{0.501875pt}%
\definecolor{currentstroke}{rgb}{0.000000,0.000000,0.000000}%
\pgfsetstrokecolor{currentstroke}%
\pgfsetdash{}{0pt}%
\pgfsys@defobject{currentmarker}{\pgfqpoint{0.000000in}{0.000000in}}{\pgfqpoint{0.027778in}{0.000000in}}{%
\pgfpathmoveto{\pgfqpoint{0.000000in}{0.000000in}}%
\pgfpathlineto{\pgfqpoint{0.027778in}{0.000000in}}%
\pgfusepath{stroke,fill}%
}%
\begin{pgfscope}%
\pgfsys@transformshift{1.000000in}{0.588989in}%
\pgfsys@useobject{currentmarker}{}%
\end{pgfscope}%
\end{pgfscope}%
\begin{pgfscope}%
\pgfsetbuttcap%
\pgfsetroundjoin%
\definecolor{currentfill}{rgb}{0.000000,0.000000,0.000000}%
\pgfsetfillcolor{currentfill}%
\pgfsetlinewidth{0.501875pt}%
\definecolor{currentstroke}{rgb}{0.000000,0.000000,0.000000}%
\pgfsetstrokecolor{currentstroke}%
\pgfsetdash{}{0pt}%
\pgfsys@defobject{currentmarker}{\pgfqpoint{-0.027778in}{0.000000in}}{\pgfqpoint{0.000000in}{0.000000in}}{%
\pgfpathmoveto{\pgfqpoint{0.000000in}{0.000000in}}%
\pgfpathlineto{\pgfqpoint{-0.027778in}{0.000000in}}%
\pgfusepath{stroke,fill}%
}%
\begin{pgfscope}%
\pgfsys@transformshift{7.200000in}{0.588989in}%
\pgfsys@useobject{currentmarker}{}%
\end{pgfscope}%
\end{pgfscope}%
\begin{pgfscope}%
\pgfsetbuttcap%
\pgfsetroundjoin%
\definecolor{currentfill}{rgb}{0.000000,0.000000,0.000000}%
\pgfsetfillcolor{currentfill}%
\pgfsetlinewidth{0.501875pt}%
\definecolor{currentstroke}{rgb}{0.000000,0.000000,0.000000}%
\pgfsetstrokecolor{currentstroke}%
\pgfsetdash{}{0pt}%
\pgfsys@defobject{currentmarker}{\pgfqpoint{0.000000in}{0.000000in}}{\pgfqpoint{0.027778in}{0.000000in}}{%
\pgfpathmoveto{\pgfqpoint{0.000000in}{0.000000in}}%
\pgfpathlineto{\pgfqpoint{0.027778in}{0.000000in}}%
\pgfusepath{stroke,fill}%
}%
\begin{pgfscope}%
\pgfsys@transformshift{1.000000in}{0.635506in}%
\pgfsys@useobject{currentmarker}{}%
\end{pgfscope}%
\end{pgfscope}%
\begin{pgfscope}%
\pgfsetbuttcap%
\pgfsetroundjoin%
\definecolor{currentfill}{rgb}{0.000000,0.000000,0.000000}%
\pgfsetfillcolor{currentfill}%
\pgfsetlinewidth{0.501875pt}%
\definecolor{currentstroke}{rgb}{0.000000,0.000000,0.000000}%
\pgfsetstrokecolor{currentstroke}%
\pgfsetdash{}{0pt}%
\pgfsys@defobject{currentmarker}{\pgfqpoint{-0.027778in}{0.000000in}}{\pgfqpoint{0.000000in}{0.000000in}}{%
\pgfpathmoveto{\pgfqpoint{0.000000in}{0.000000in}}%
\pgfpathlineto{\pgfqpoint{-0.027778in}{0.000000in}}%
\pgfusepath{stroke,fill}%
}%
\begin{pgfscope}%
\pgfsys@transformshift{7.200000in}{0.635506in}%
\pgfsys@useobject{currentmarker}{}%
\end{pgfscope}%
\end{pgfscope}%
\begin{pgfscope}%
\pgfsetbuttcap%
\pgfsetroundjoin%
\definecolor{currentfill}{rgb}{0.000000,0.000000,0.000000}%
\pgfsetfillcolor{currentfill}%
\pgfsetlinewidth{0.501875pt}%
\definecolor{currentstroke}{rgb}{0.000000,0.000000,0.000000}%
\pgfsetstrokecolor{currentstroke}%
\pgfsetdash{}{0pt}%
\pgfsys@defobject{currentmarker}{\pgfqpoint{0.000000in}{0.000000in}}{\pgfqpoint{0.027778in}{0.000000in}}{%
\pgfpathmoveto{\pgfqpoint{0.000000in}{0.000000in}}%
\pgfpathlineto{\pgfqpoint{0.027778in}{0.000000in}}%
\pgfusepath{stroke,fill}%
}%
\begin{pgfscope}%
\pgfsys@transformshift{1.000000in}{0.673513in}%
\pgfsys@useobject{currentmarker}{}%
\end{pgfscope}%
\end{pgfscope}%
\begin{pgfscope}%
\pgfsetbuttcap%
\pgfsetroundjoin%
\definecolor{currentfill}{rgb}{0.000000,0.000000,0.000000}%
\pgfsetfillcolor{currentfill}%
\pgfsetlinewidth{0.501875pt}%
\definecolor{currentstroke}{rgb}{0.000000,0.000000,0.000000}%
\pgfsetstrokecolor{currentstroke}%
\pgfsetdash{}{0pt}%
\pgfsys@defobject{currentmarker}{\pgfqpoint{-0.027778in}{0.000000in}}{\pgfqpoint{0.000000in}{0.000000in}}{%
\pgfpathmoveto{\pgfqpoint{0.000000in}{0.000000in}}%
\pgfpathlineto{\pgfqpoint{-0.027778in}{0.000000in}}%
\pgfusepath{stroke,fill}%
}%
\begin{pgfscope}%
\pgfsys@transformshift{7.200000in}{0.673513in}%
\pgfsys@useobject{currentmarker}{}%
\end{pgfscope}%
\end{pgfscope}%
\begin{pgfscope}%
\pgfsetbuttcap%
\pgfsetroundjoin%
\definecolor{currentfill}{rgb}{0.000000,0.000000,0.000000}%
\pgfsetfillcolor{currentfill}%
\pgfsetlinewidth{0.501875pt}%
\definecolor{currentstroke}{rgb}{0.000000,0.000000,0.000000}%
\pgfsetstrokecolor{currentstroke}%
\pgfsetdash{}{0pt}%
\pgfsys@defobject{currentmarker}{\pgfqpoint{0.000000in}{0.000000in}}{\pgfqpoint{0.027778in}{0.000000in}}{%
\pgfpathmoveto{\pgfqpoint{0.000000in}{0.000000in}}%
\pgfpathlineto{\pgfqpoint{0.027778in}{0.000000in}}%
\pgfusepath{stroke,fill}%
}%
\begin{pgfscope}%
\pgfsys@transformshift{1.000000in}{0.705647in}%
\pgfsys@useobject{currentmarker}{}%
\end{pgfscope}%
\end{pgfscope}%
\begin{pgfscope}%
\pgfsetbuttcap%
\pgfsetroundjoin%
\definecolor{currentfill}{rgb}{0.000000,0.000000,0.000000}%
\pgfsetfillcolor{currentfill}%
\pgfsetlinewidth{0.501875pt}%
\definecolor{currentstroke}{rgb}{0.000000,0.000000,0.000000}%
\pgfsetstrokecolor{currentstroke}%
\pgfsetdash{}{0pt}%
\pgfsys@defobject{currentmarker}{\pgfqpoint{-0.027778in}{0.000000in}}{\pgfqpoint{0.000000in}{0.000000in}}{%
\pgfpathmoveto{\pgfqpoint{0.000000in}{0.000000in}}%
\pgfpathlineto{\pgfqpoint{-0.027778in}{0.000000in}}%
\pgfusepath{stroke,fill}%
}%
\begin{pgfscope}%
\pgfsys@transformshift{7.200000in}{0.705647in}%
\pgfsys@useobject{currentmarker}{}%
\end{pgfscope}%
\end{pgfscope}%
\begin{pgfscope}%
\pgfsetbuttcap%
\pgfsetroundjoin%
\definecolor{currentfill}{rgb}{0.000000,0.000000,0.000000}%
\pgfsetfillcolor{currentfill}%
\pgfsetlinewidth{0.501875pt}%
\definecolor{currentstroke}{rgb}{0.000000,0.000000,0.000000}%
\pgfsetstrokecolor{currentstroke}%
\pgfsetdash{}{0pt}%
\pgfsys@defobject{currentmarker}{\pgfqpoint{0.000000in}{0.000000in}}{\pgfqpoint{0.027778in}{0.000000in}}{%
\pgfpathmoveto{\pgfqpoint{0.000000in}{0.000000in}}%
\pgfpathlineto{\pgfqpoint{0.027778in}{0.000000in}}%
\pgfusepath{stroke,fill}%
}%
\begin{pgfscope}%
\pgfsys@transformshift{1.000000in}{0.733483in}%
\pgfsys@useobject{currentmarker}{}%
\end{pgfscope}%
\end{pgfscope}%
\begin{pgfscope}%
\pgfsetbuttcap%
\pgfsetroundjoin%
\definecolor{currentfill}{rgb}{0.000000,0.000000,0.000000}%
\pgfsetfillcolor{currentfill}%
\pgfsetlinewidth{0.501875pt}%
\definecolor{currentstroke}{rgb}{0.000000,0.000000,0.000000}%
\pgfsetstrokecolor{currentstroke}%
\pgfsetdash{}{0pt}%
\pgfsys@defobject{currentmarker}{\pgfqpoint{-0.027778in}{0.000000in}}{\pgfqpoint{0.000000in}{0.000000in}}{%
\pgfpathmoveto{\pgfqpoint{0.000000in}{0.000000in}}%
\pgfpathlineto{\pgfqpoint{-0.027778in}{0.000000in}}%
\pgfusepath{stroke,fill}%
}%
\begin{pgfscope}%
\pgfsys@transformshift{7.200000in}{0.733483in}%
\pgfsys@useobject{currentmarker}{}%
\end{pgfscope}%
\end{pgfscope}%
\begin{pgfscope}%
\pgfsetbuttcap%
\pgfsetroundjoin%
\definecolor{currentfill}{rgb}{0.000000,0.000000,0.000000}%
\pgfsetfillcolor{currentfill}%
\pgfsetlinewidth{0.501875pt}%
\definecolor{currentstroke}{rgb}{0.000000,0.000000,0.000000}%
\pgfsetstrokecolor{currentstroke}%
\pgfsetdash{}{0pt}%
\pgfsys@defobject{currentmarker}{\pgfqpoint{0.000000in}{0.000000in}}{\pgfqpoint{0.027778in}{0.000000in}}{%
\pgfpathmoveto{\pgfqpoint{0.000000in}{0.000000in}}%
\pgfpathlineto{\pgfqpoint{0.027778in}{0.000000in}}%
\pgfusepath{stroke,fill}%
}%
\begin{pgfscope}%
\pgfsys@transformshift{1.000000in}{0.758036in}%
\pgfsys@useobject{currentmarker}{}%
\end{pgfscope}%
\end{pgfscope}%
\begin{pgfscope}%
\pgfsetbuttcap%
\pgfsetroundjoin%
\definecolor{currentfill}{rgb}{0.000000,0.000000,0.000000}%
\pgfsetfillcolor{currentfill}%
\pgfsetlinewidth{0.501875pt}%
\definecolor{currentstroke}{rgb}{0.000000,0.000000,0.000000}%
\pgfsetstrokecolor{currentstroke}%
\pgfsetdash{}{0pt}%
\pgfsys@defobject{currentmarker}{\pgfqpoint{-0.027778in}{0.000000in}}{\pgfqpoint{0.000000in}{0.000000in}}{%
\pgfpathmoveto{\pgfqpoint{0.000000in}{0.000000in}}%
\pgfpathlineto{\pgfqpoint{-0.027778in}{0.000000in}}%
\pgfusepath{stroke,fill}%
}%
\begin{pgfscope}%
\pgfsys@transformshift{7.200000in}{0.758036in}%
\pgfsys@useobject{currentmarker}{}%
\end{pgfscope}%
\end{pgfscope}%
\begin{pgfscope}%
\pgfsetbuttcap%
\pgfsetroundjoin%
\definecolor{currentfill}{rgb}{0.000000,0.000000,0.000000}%
\pgfsetfillcolor{currentfill}%
\pgfsetlinewidth{0.501875pt}%
\definecolor{currentstroke}{rgb}{0.000000,0.000000,0.000000}%
\pgfsetstrokecolor{currentstroke}%
\pgfsetdash{}{0pt}%
\pgfsys@defobject{currentmarker}{\pgfqpoint{0.000000in}{0.000000in}}{\pgfqpoint{0.027778in}{0.000000in}}{%
\pgfpathmoveto{\pgfqpoint{0.000000in}{0.000000in}}%
\pgfpathlineto{\pgfqpoint{0.027778in}{0.000000in}}%
\pgfusepath{stroke,fill}%
}%
\begin{pgfscope}%
\pgfsys@transformshift{1.000000in}{0.924494in}%
\pgfsys@useobject{currentmarker}{}%
\end{pgfscope}%
\end{pgfscope}%
\begin{pgfscope}%
\pgfsetbuttcap%
\pgfsetroundjoin%
\definecolor{currentfill}{rgb}{0.000000,0.000000,0.000000}%
\pgfsetfillcolor{currentfill}%
\pgfsetlinewidth{0.501875pt}%
\definecolor{currentstroke}{rgb}{0.000000,0.000000,0.000000}%
\pgfsetstrokecolor{currentstroke}%
\pgfsetdash{}{0pt}%
\pgfsys@defobject{currentmarker}{\pgfqpoint{-0.027778in}{0.000000in}}{\pgfqpoint{0.000000in}{0.000000in}}{%
\pgfpathmoveto{\pgfqpoint{0.000000in}{0.000000in}}%
\pgfpathlineto{\pgfqpoint{-0.027778in}{0.000000in}}%
\pgfusepath{stroke,fill}%
}%
\begin{pgfscope}%
\pgfsys@transformshift{7.200000in}{0.924494in}%
\pgfsys@useobject{currentmarker}{}%
\end{pgfscope}%
\end{pgfscope}%
\begin{pgfscope}%
\pgfsetbuttcap%
\pgfsetroundjoin%
\definecolor{currentfill}{rgb}{0.000000,0.000000,0.000000}%
\pgfsetfillcolor{currentfill}%
\pgfsetlinewidth{0.501875pt}%
\definecolor{currentstroke}{rgb}{0.000000,0.000000,0.000000}%
\pgfsetstrokecolor{currentstroke}%
\pgfsetdash{}{0pt}%
\pgfsys@defobject{currentmarker}{\pgfqpoint{0.000000in}{0.000000in}}{\pgfqpoint{0.027778in}{0.000000in}}{%
\pgfpathmoveto{\pgfqpoint{0.000000in}{0.000000in}}%
\pgfpathlineto{\pgfqpoint{0.027778in}{0.000000in}}%
\pgfusepath{stroke,fill}%
}%
\begin{pgfscope}%
\pgfsys@transformshift{1.000000in}{1.009018in}%
\pgfsys@useobject{currentmarker}{}%
\end{pgfscope}%
\end{pgfscope}%
\begin{pgfscope}%
\pgfsetbuttcap%
\pgfsetroundjoin%
\definecolor{currentfill}{rgb}{0.000000,0.000000,0.000000}%
\pgfsetfillcolor{currentfill}%
\pgfsetlinewidth{0.501875pt}%
\definecolor{currentstroke}{rgb}{0.000000,0.000000,0.000000}%
\pgfsetstrokecolor{currentstroke}%
\pgfsetdash{}{0pt}%
\pgfsys@defobject{currentmarker}{\pgfqpoint{-0.027778in}{0.000000in}}{\pgfqpoint{0.000000in}{0.000000in}}{%
\pgfpathmoveto{\pgfqpoint{0.000000in}{0.000000in}}%
\pgfpathlineto{\pgfqpoint{-0.027778in}{0.000000in}}%
\pgfusepath{stroke,fill}%
}%
\begin{pgfscope}%
\pgfsys@transformshift{7.200000in}{1.009018in}%
\pgfsys@useobject{currentmarker}{}%
\end{pgfscope}%
\end{pgfscope}%
\begin{pgfscope}%
\pgfsetbuttcap%
\pgfsetroundjoin%
\definecolor{currentfill}{rgb}{0.000000,0.000000,0.000000}%
\pgfsetfillcolor{currentfill}%
\pgfsetlinewidth{0.501875pt}%
\definecolor{currentstroke}{rgb}{0.000000,0.000000,0.000000}%
\pgfsetstrokecolor{currentstroke}%
\pgfsetdash{}{0pt}%
\pgfsys@defobject{currentmarker}{\pgfqpoint{0.000000in}{0.000000in}}{\pgfqpoint{0.027778in}{0.000000in}}{%
\pgfpathmoveto{\pgfqpoint{0.000000in}{0.000000in}}%
\pgfpathlineto{\pgfqpoint{0.027778in}{0.000000in}}%
\pgfusepath{stroke,fill}%
}%
\begin{pgfscope}%
\pgfsys@transformshift{1.000000in}{1.068989in}%
\pgfsys@useobject{currentmarker}{}%
\end{pgfscope}%
\end{pgfscope}%
\begin{pgfscope}%
\pgfsetbuttcap%
\pgfsetroundjoin%
\definecolor{currentfill}{rgb}{0.000000,0.000000,0.000000}%
\pgfsetfillcolor{currentfill}%
\pgfsetlinewidth{0.501875pt}%
\definecolor{currentstroke}{rgb}{0.000000,0.000000,0.000000}%
\pgfsetstrokecolor{currentstroke}%
\pgfsetdash{}{0pt}%
\pgfsys@defobject{currentmarker}{\pgfqpoint{-0.027778in}{0.000000in}}{\pgfqpoint{0.000000in}{0.000000in}}{%
\pgfpathmoveto{\pgfqpoint{0.000000in}{0.000000in}}%
\pgfpathlineto{\pgfqpoint{-0.027778in}{0.000000in}}%
\pgfusepath{stroke,fill}%
}%
\begin{pgfscope}%
\pgfsys@transformshift{7.200000in}{1.068989in}%
\pgfsys@useobject{currentmarker}{}%
\end{pgfscope}%
\end{pgfscope}%
\begin{pgfscope}%
\pgfsetbuttcap%
\pgfsetroundjoin%
\definecolor{currentfill}{rgb}{0.000000,0.000000,0.000000}%
\pgfsetfillcolor{currentfill}%
\pgfsetlinewidth{0.501875pt}%
\definecolor{currentstroke}{rgb}{0.000000,0.000000,0.000000}%
\pgfsetstrokecolor{currentstroke}%
\pgfsetdash{}{0pt}%
\pgfsys@defobject{currentmarker}{\pgfqpoint{0.000000in}{0.000000in}}{\pgfqpoint{0.027778in}{0.000000in}}{%
\pgfpathmoveto{\pgfqpoint{0.000000in}{0.000000in}}%
\pgfpathlineto{\pgfqpoint{0.027778in}{0.000000in}}%
\pgfusepath{stroke,fill}%
}%
\begin{pgfscope}%
\pgfsys@transformshift{1.000000in}{1.115506in}%
\pgfsys@useobject{currentmarker}{}%
\end{pgfscope}%
\end{pgfscope}%
\begin{pgfscope}%
\pgfsetbuttcap%
\pgfsetroundjoin%
\definecolor{currentfill}{rgb}{0.000000,0.000000,0.000000}%
\pgfsetfillcolor{currentfill}%
\pgfsetlinewidth{0.501875pt}%
\definecolor{currentstroke}{rgb}{0.000000,0.000000,0.000000}%
\pgfsetstrokecolor{currentstroke}%
\pgfsetdash{}{0pt}%
\pgfsys@defobject{currentmarker}{\pgfqpoint{-0.027778in}{0.000000in}}{\pgfqpoint{0.000000in}{0.000000in}}{%
\pgfpathmoveto{\pgfqpoint{0.000000in}{0.000000in}}%
\pgfpathlineto{\pgfqpoint{-0.027778in}{0.000000in}}%
\pgfusepath{stroke,fill}%
}%
\begin{pgfscope}%
\pgfsys@transformshift{7.200000in}{1.115506in}%
\pgfsys@useobject{currentmarker}{}%
\end{pgfscope}%
\end{pgfscope}%
\begin{pgfscope}%
\pgfsetbuttcap%
\pgfsetroundjoin%
\definecolor{currentfill}{rgb}{0.000000,0.000000,0.000000}%
\pgfsetfillcolor{currentfill}%
\pgfsetlinewidth{0.501875pt}%
\definecolor{currentstroke}{rgb}{0.000000,0.000000,0.000000}%
\pgfsetstrokecolor{currentstroke}%
\pgfsetdash{}{0pt}%
\pgfsys@defobject{currentmarker}{\pgfqpoint{0.000000in}{0.000000in}}{\pgfqpoint{0.027778in}{0.000000in}}{%
\pgfpathmoveto{\pgfqpoint{0.000000in}{0.000000in}}%
\pgfpathlineto{\pgfqpoint{0.027778in}{0.000000in}}%
\pgfusepath{stroke,fill}%
}%
\begin{pgfscope}%
\pgfsys@transformshift{1.000000in}{1.153513in}%
\pgfsys@useobject{currentmarker}{}%
\end{pgfscope}%
\end{pgfscope}%
\begin{pgfscope}%
\pgfsetbuttcap%
\pgfsetroundjoin%
\definecolor{currentfill}{rgb}{0.000000,0.000000,0.000000}%
\pgfsetfillcolor{currentfill}%
\pgfsetlinewidth{0.501875pt}%
\definecolor{currentstroke}{rgb}{0.000000,0.000000,0.000000}%
\pgfsetstrokecolor{currentstroke}%
\pgfsetdash{}{0pt}%
\pgfsys@defobject{currentmarker}{\pgfqpoint{-0.027778in}{0.000000in}}{\pgfqpoint{0.000000in}{0.000000in}}{%
\pgfpathmoveto{\pgfqpoint{0.000000in}{0.000000in}}%
\pgfpathlineto{\pgfqpoint{-0.027778in}{0.000000in}}%
\pgfusepath{stroke,fill}%
}%
\begin{pgfscope}%
\pgfsys@transformshift{7.200000in}{1.153513in}%
\pgfsys@useobject{currentmarker}{}%
\end{pgfscope}%
\end{pgfscope}%
\begin{pgfscope}%
\pgfsetbuttcap%
\pgfsetroundjoin%
\definecolor{currentfill}{rgb}{0.000000,0.000000,0.000000}%
\pgfsetfillcolor{currentfill}%
\pgfsetlinewidth{0.501875pt}%
\definecolor{currentstroke}{rgb}{0.000000,0.000000,0.000000}%
\pgfsetstrokecolor{currentstroke}%
\pgfsetdash{}{0pt}%
\pgfsys@defobject{currentmarker}{\pgfqpoint{0.000000in}{0.000000in}}{\pgfqpoint{0.027778in}{0.000000in}}{%
\pgfpathmoveto{\pgfqpoint{0.000000in}{0.000000in}}%
\pgfpathlineto{\pgfqpoint{0.027778in}{0.000000in}}%
\pgfusepath{stroke,fill}%
}%
\begin{pgfscope}%
\pgfsys@transformshift{1.000000in}{1.185647in}%
\pgfsys@useobject{currentmarker}{}%
\end{pgfscope}%
\end{pgfscope}%
\begin{pgfscope}%
\pgfsetbuttcap%
\pgfsetroundjoin%
\definecolor{currentfill}{rgb}{0.000000,0.000000,0.000000}%
\pgfsetfillcolor{currentfill}%
\pgfsetlinewidth{0.501875pt}%
\definecolor{currentstroke}{rgb}{0.000000,0.000000,0.000000}%
\pgfsetstrokecolor{currentstroke}%
\pgfsetdash{}{0pt}%
\pgfsys@defobject{currentmarker}{\pgfqpoint{-0.027778in}{0.000000in}}{\pgfqpoint{0.000000in}{0.000000in}}{%
\pgfpathmoveto{\pgfqpoint{0.000000in}{0.000000in}}%
\pgfpathlineto{\pgfqpoint{-0.027778in}{0.000000in}}%
\pgfusepath{stroke,fill}%
}%
\begin{pgfscope}%
\pgfsys@transformshift{7.200000in}{1.185647in}%
\pgfsys@useobject{currentmarker}{}%
\end{pgfscope}%
\end{pgfscope}%
\begin{pgfscope}%
\pgfsetbuttcap%
\pgfsetroundjoin%
\definecolor{currentfill}{rgb}{0.000000,0.000000,0.000000}%
\pgfsetfillcolor{currentfill}%
\pgfsetlinewidth{0.501875pt}%
\definecolor{currentstroke}{rgb}{0.000000,0.000000,0.000000}%
\pgfsetstrokecolor{currentstroke}%
\pgfsetdash{}{0pt}%
\pgfsys@defobject{currentmarker}{\pgfqpoint{0.000000in}{0.000000in}}{\pgfqpoint{0.027778in}{0.000000in}}{%
\pgfpathmoveto{\pgfqpoint{0.000000in}{0.000000in}}%
\pgfpathlineto{\pgfqpoint{0.027778in}{0.000000in}}%
\pgfusepath{stroke,fill}%
}%
\begin{pgfscope}%
\pgfsys@transformshift{1.000000in}{1.213483in}%
\pgfsys@useobject{currentmarker}{}%
\end{pgfscope}%
\end{pgfscope}%
\begin{pgfscope}%
\pgfsetbuttcap%
\pgfsetroundjoin%
\definecolor{currentfill}{rgb}{0.000000,0.000000,0.000000}%
\pgfsetfillcolor{currentfill}%
\pgfsetlinewidth{0.501875pt}%
\definecolor{currentstroke}{rgb}{0.000000,0.000000,0.000000}%
\pgfsetstrokecolor{currentstroke}%
\pgfsetdash{}{0pt}%
\pgfsys@defobject{currentmarker}{\pgfqpoint{-0.027778in}{0.000000in}}{\pgfqpoint{0.000000in}{0.000000in}}{%
\pgfpathmoveto{\pgfqpoint{0.000000in}{0.000000in}}%
\pgfpathlineto{\pgfqpoint{-0.027778in}{0.000000in}}%
\pgfusepath{stroke,fill}%
}%
\begin{pgfscope}%
\pgfsys@transformshift{7.200000in}{1.213483in}%
\pgfsys@useobject{currentmarker}{}%
\end{pgfscope}%
\end{pgfscope}%
\begin{pgfscope}%
\pgfsetbuttcap%
\pgfsetroundjoin%
\definecolor{currentfill}{rgb}{0.000000,0.000000,0.000000}%
\pgfsetfillcolor{currentfill}%
\pgfsetlinewidth{0.501875pt}%
\definecolor{currentstroke}{rgb}{0.000000,0.000000,0.000000}%
\pgfsetstrokecolor{currentstroke}%
\pgfsetdash{}{0pt}%
\pgfsys@defobject{currentmarker}{\pgfqpoint{0.000000in}{0.000000in}}{\pgfqpoint{0.027778in}{0.000000in}}{%
\pgfpathmoveto{\pgfqpoint{0.000000in}{0.000000in}}%
\pgfpathlineto{\pgfqpoint{0.027778in}{0.000000in}}%
\pgfusepath{stroke,fill}%
}%
\begin{pgfscope}%
\pgfsys@transformshift{1.000000in}{1.238036in}%
\pgfsys@useobject{currentmarker}{}%
\end{pgfscope}%
\end{pgfscope}%
\begin{pgfscope}%
\pgfsetbuttcap%
\pgfsetroundjoin%
\definecolor{currentfill}{rgb}{0.000000,0.000000,0.000000}%
\pgfsetfillcolor{currentfill}%
\pgfsetlinewidth{0.501875pt}%
\definecolor{currentstroke}{rgb}{0.000000,0.000000,0.000000}%
\pgfsetstrokecolor{currentstroke}%
\pgfsetdash{}{0pt}%
\pgfsys@defobject{currentmarker}{\pgfqpoint{-0.027778in}{0.000000in}}{\pgfqpoint{0.000000in}{0.000000in}}{%
\pgfpathmoveto{\pgfqpoint{0.000000in}{0.000000in}}%
\pgfpathlineto{\pgfqpoint{-0.027778in}{0.000000in}}%
\pgfusepath{stroke,fill}%
}%
\begin{pgfscope}%
\pgfsys@transformshift{7.200000in}{1.238036in}%
\pgfsys@useobject{currentmarker}{}%
\end{pgfscope}%
\end{pgfscope}%
\begin{pgfscope}%
\pgfsetbuttcap%
\pgfsetroundjoin%
\definecolor{currentfill}{rgb}{0.000000,0.000000,0.000000}%
\pgfsetfillcolor{currentfill}%
\pgfsetlinewidth{0.501875pt}%
\definecolor{currentstroke}{rgb}{0.000000,0.000000,0.000000}%
\pgfsetstrokecolor{currentstroke}%
\pgfsetdash{}{0pt}%
\pgfsys@defobject{currentmarker}{\pgfqpoint{0.000000in}{0.000000in}}{\pgfqpoint{0.027778in}{0.000000in}}{%
\pgfpathmoveto{\pgfqpoint{0.000000in}{0.000000in}}%
\pgfpathlineto{\pgfqpoint{0.027778in}{0.000000in}}%
\pgfusepath{stroke,fill}%
}%
\begin{pgfscope}%
\pgfsys@transformshift{1.000000in}{1.404494in}%
\pgfsys@useobject{currentmarker}{}%
\end{pgfscope}%
\end{pgfscope}%
\begin{pgfscope}%
\pgfsetbuttcap%
\pgfsetroundjoin%
\definecolor{currentfill}{rgb}{0.000000,0.000000,0.000000}%
\pgfsetfillcolor{currentfill}%
\pgfsetlinewidth{0.501875pt}%
\definecolor{currentstroke}{rgb}{0.000000,0.000000,0.000000}%
\pgfsetstrokecolor{currentstroke}%
\pgfsetdash{}{0pt}%
\pgfsys@defobject{currentmarker}{\pgfqpoint{-0.027778in}{0.000000in}}{\pgfqpoint{0.000000in}{0.000000in}}{%
\pgfpathmoveto{\pgfqpoint{0.000000in}{0.000000in}}%
\pgfpathlineto{\pgfqpoint{-0.027778in}{0.000000in}}%
\pgfusepath{stroke,fill}%
}%
\begin{pgfscope}%
\pgfsys@transformshift{7.200000in}{1.404494in}%
\pgfsys@useobject{currentmarker}{}%
\end{pgfscope}%
\end{pgfscope}%
\begin{pgfscope}%
\pgfsetbuttcap%
\pgfsetroundjoin%
\definecolor{currentfill}{rgb}{0.000000,0.000000,0.000000}%
\pgfsetfillcolor{currentfill}%
\pgfsetlinewidth{0.501875pt}%
\definecolor{currentstroke}{rgb}{0.000000,0.000000,0.000000}%
\pgfsetstrokecolor{currentstroke}%
\pgfsetdash{}{0pt}%
\pgfsys@defobject{currentmarker}{\pgfqpoint{0.000000in}{0.000000in}}{\pgfqpoint{0.027778in}{0.000000in}}{%
\pgfpathmoveto{\pgfqpoint{0.000000in}{0.000000in}}%
\pgfpathlineto{\pgfqpoint{0.027778in}{0.000000in}}%
\pgfusepath{stroke,fill}%
}%
\begin{pgfscope}%
\pgfsys@transformshift{1.000000in}{1.489018in}%
\pgfsys@useobject{currentmarker}{}%
\end{pgfscope}%
\end{pgfscope}%
\begin{pgfscope}%
\pgfsetbuttcap%
\pgfsetroundjoin%
\definecolor{currentfill}{rgb}{0.000000,0.000000,0.000000}%
\pgfsetfillcolor{currentfill}%
\pgfsetlinewidth{0.501875pt}%
\definecolor{currentstroke}{rgb}{0.000000,0.000000,0.000000}%
\pgfsetstrokecolor{currentstroke}%
\pgfsetdash{}{0pt}%
\pgfsys@defobject{currentmarker}{\pgfqpoint{-0.027778in}{0.000000in}}{\pgfqpoint{0.000000in}{0.000000in}}{%
\pgfpathmoveto{\pgfqpoint{0.000000in}{0.000000in}}%
\pgfpathlineto{\pgfqpoint{-0.027778in}{0.000000in}}%
\pgfusepath{stroke,fill}%
}%
\begin{pgfscope}%
\pgfsys@transformshift{7.200000in}{1.489018in}%
\pgfsys@useobject{currentmarker}{}%
\end{pgfscope}%
\end{pgfscope}%
\begin{pgfscope}%
\pgfsetbuttcap%
\pgfsetroundjoin%
\definecolor{currentfill}{rgb}{0.000000,0.000000,0.000000}%
\pgfsetfillcolor{currentfill}%
\pgfsetlinewidth{0.501875pt}%
\definecolor{currentstroke}{rgb}{0.000000,0.000000,0.000000}%
\pgfsetstrokecolor{currentstroke}%
\pgfsetdash{}{0pt}%
\pgfsys@defobject{currentmarker}{\pgfqpoint{0.000000in}{0.000000in}}{\pgfqpoint{0.027778in}{0.000000in}}{%
\pgfpathmoveto{\pgfqpoint{0.000000in}{0.000000in}}%
\pgfpathlineto{\pgfqpoint{0.027778in}{0.000000in}}%
\pgfusepath{stroke,fill}%
}%
\begin{pgfscope}%
\pgfsys@transformshift{1.000000in}{1.548989in}%
\pgfsys@useobject{currentmarker}{}%
\end{pgfscope}%
\end{pgfscope}%
\begin{pgfscope}%
\pgfsetbuttcap%
\pgfsetroundjoin%
\definecolor{currentfill}{rgb}{0.000000,0.000000,0.000000}%
\pgfsetfillcolor{currentfill}%
\pgfsetlinewidth{0.501875pt}%
\definecolor{currentstroke}{rgb}{0.000000,0.000000,0.000000}%
\pgfsetstrokecolor{currentstroke}%
\pgfsetdash{}{0pt}%
\pgfsys@defobject{currentmarker}{\pgfqpoint{-0.027778in}{0.000000in}}{\pgfqpoint{0.000000in}{0.000000in}}{%
\pgfpathmoveto{\pgfqpoint{0.000000in}{0.000000in}}%
\pgfpathlineto{\pgfqpoint{-0.027778in}{0.000000in}}%
\pgfusepath{stroke,fill}%
}%
\begin{pgfscope}%
\pgfsys@transformshift{7.200000in}{1.548989in}%
\pgfsys@useobject{currentmarker}{}%
\end{pgfscope}%
\end{pgfscope}%
\begin{pgfscope}%
\pgfsetbuttcap%
\pgfsetroundjoin%
\definecolor{currentfill}{rgb}{0.000000,0.000000,0.000000}%
\pgfsetfillcolor{currentfill}%
\pgfsetlinewidth{0.501875pt}%
\definecolor{currentstroke}{rgb}{0.000000,0.000000,0.000000}%
\pgfsetstrokecolor{currentstroke}%
\pgfsetdash{}{0pt}%
\pgfsys@defobject{currentmarker}{\pgfqpoint{0.000000in}{0.000000in}}{\pgfqpoint{0.027778in}{0.000000in}}{%
\pgfpathmoveto{\pgfqpoint{0.000000in}{0.000000in}}%
\pgfpathlineto{\pgfqpoint{0.027778in}{0.000000in}}%
\pgfusepath{stroke,fill}%
}%
\begin{pgfscope}%
\pgfsys@transformshift{1.000000in}{1.595506in}%
\pgfsys@useobject{currentmarker}{}%
\end{pgfscope}%
\end{pgfscope}%
\begin{pgfscope}%
\pgfsetbuttcap%
\pgfsetroundjoin%
\definecolor{currentfill}{rgb}{0.000000,0.000000,0.000000}%
\pgfsetfillcolor{currentfill}%
\pgfsetlinewidth{0.501875pt}%
\definecolor{currentstroke}{rgb}{0.000000,0.000000,0.000000}%
\pgfsetstrokecolor{currentstroke}%
\pgfsetdash{}{0pt}%
\pgfsys@defobject{currentmarker}{\pgfqpoint{-0.027778in}{0.000000in}}{\pgfqpoint{0.000000in}{0.000000in}}{%
\pgfpathmoveto{\pgfqpoint{0.000000in}{0.000000in}}%
\pgfpathlineto{\pgfqpoint{-0.027778in}{0.000000in}}%
\pgfusepath{stroke,fill}%
}%
\begin{pgfscope}%
\pgfsys@transformshift{7.200000in}{1.595506in}%
\pgfsys@useobject{currentmarker}{}%
\end{pgfscope}%
\end{pgfscope}%
\begin{pgfscope}%
\pgfsetbuttcap%
\pgfsetroundjoin%
\definecolor{currentfill}{rgb}{0.000000,0.000000,0.000000}%
\pgfsetfillcolor{currentfill}%
\pgfsetlinewidth{0.501875pt}%
\definecolor{currentstroke}{rgb}{0.000000,0.000000,0.000000}%
\pgfsetstrokecolor{currentstroke}%
\pgfsetdash{}{0pt}%
\pgfsys@defobject{currentmarker}{\pgfqpoint{0.000000in}{0.000000in}}{\pgfqpoint{0.027778in}{0.000000in}}{%
\pgfpathmoveto{\pgfqpoint{0.000000in}{0.000000in}}%
\pgfpathlineto{\pgfqpoint{0.027778in}{0.000000in}}%
\pgfusepath{stroke,fill}%
}%
\begin{pgfscope}%
\pgfsys@transformshift{1.000000in}{1.633513in}%
\pgfsys@useobject{currentmarker}{}%
\end{pgfscope}%
\end{pgfscope}%
\begin{pgfscope}%
\pgfsetbuttcap%
\pgfsetroundjoin%
\definecolor{currentfill}{rgb}{0.000000,0.000000,0.000000}%
\pgfsetfillcolor{currentfill}%
\pgfsetlinewidth{0.501875pt}%
\definecolor{currentstroke}{rgb}{0.000000,0.000000,0.000000}%
\pgfsetstrokecolor{currentstroke}%
\pgfsetdash{}{0pt}%
\pgfsys@defobject{currentmarker}{\pgfqpoint{-0.027778in}{0.000000in}}{\pgfqpoint{0.000000in}{0.000000in}}{%
\pgfpathmoveto{\pgfqpoint{0.000000in}{0.000000in}}%
\pgfpathlineto{\pgfqpoint{-0.027778in}{0.000000in}}%
\pgfusepath{stroke,fill}%
}%
\begin{pgfscope}%
\pgfsys@transformshift{7.200000in}{1.633513in}%
\pgfsys@useobject{currentmarker}{}%
\end{pgfscope}%
\end{pgfscope}%
\begin{pgfscope}%
\pgfsetbuttcap%
\pgfsetroundjoin%
\definecolor{currentfill}{rgb}{0.000000,0.000000,0.000000}%
\pgfsetfillcolor{currentfill}%
\pgfsetlinewidth{0.501875pt}%
\definecolor{currentstroke}{rgb}{0.000000,0.000000,0.000000}%
\pgfsetstrokecolor{currentstroke}%
\pgfsetdash{}{0pt}%
\pgfsys@defobject{currentmarker}{\pgfqpoint{0.000000in}{0.000000in}}{\pgfqpoint{0.027778in}{0.000000in}}{%
\pgfpathmoveto{\pgfqpoint{0.000000in}{0.000000in}}%
\pgfpathlineto{\pgfqpoint{0.027778in}{0.000000in}}%
\pgfusepath{stroke,fill}%
}%
\begin{pgfscope}%
\pgfsys@transformshift{1.000000in}{1.665647in}%
\pgfsys@useobject{currentmarker}{}%
\end{pgfscope}%
\end{pgfscope}%
\begin{pgfscope}%
\pgfsetbuttcap%
\pgfsetroundjoin%
\definecolor{currentfill}{rgb}{0.000000,0.000000,0.000000}%
\pgfsetfillcolor{currentfill}%
\pgfsetlinewidth{0.501875pt}%
\definecolor{currentstroke}{rgb}{0.000000,0.000000,0.000000}%
\pgfsetstrokecolor{currentstroke}%
\pgfsetdash{}{0pt}%
\pgfsys@defobject{currentmarker}{\pgfqpoint{-0.027778in}{0.000000in}}{\pgfqpoint{0.000000in}{0.000000in}}{%
\pgfpathmoveto{\pgfqpoint{0.000000in}{0.000000in}}%
\pgfpathlineto{\pgfqpoint{-0.027778in}{0.000000in}}%
\pgfusepath{stroke,fill}%
}%
\begin{pgfscope}%
\pgfsys@transformshift{7.200000in}{1.665647in}%
\pgfsys@useobject{currentmarker}{}%
\end{pgfscope}%
\end{pgfscope}%
\begin{pgfscope}%
\pgfsetbuttcap%
\pgfsetroundjoin%
\definecolor{currentfill}{rgb}{0.000000,0.000000,0.000000}%
\pgfsetfillcolor{currentfill}%
\pgfsetlinewidth{0.501875pt}%
\definecolor{currentstroke}{rgb}{0.000000,0.000000,0.000000}%
\pgfsetstrokecolor{currentstroke}%
\pgfsetdash{}{0pt}%
\pgfsys@defobject{currentmarker}{\pgfqpoint{0.000000in}{0.000000in}}{\pgfqpoint{0.027778in}{0.000000in}}{%
\pgfpathmoveto{\pgfqpoint{0.000000in}{0.000000in}}%
\pgfpathlineto{\pgfqpoint{0.027778in}{0.000000in}}%
\pgfusepath{stroke,fill}%
}%
\begin{pgfscope}%
\pgfsys@transformshift{1.000000in}{1.693483in}%
\pgfsys@useobject{currentmarker}{}%
\end{pgfscope}%
\end{pgfscope}%
\begin{pgfscope}%
\pgfsetbuttcap%
\pgfsetroundjoin%
\definecolor{currentfill}{rgb}{0.000000,0.000000,0.000000}%
\pgfsetfillcolor{currentfill}%
\pgfsetlinewidth{0.501875pt}%
\definecolor{currentstroke}{rgb}{0.000000,0.000000,0.000000}%
\pgfsetstrokecolor{currentstroke}%
\pgfsetdash{}{0pt}%
\pgfsys@defobject{currentmarker}{\pgfqpoint{-0.027778in}{0.000000in}}{\pgfqpoint{0.000000in}{0.000000in}}{%
\pgfpathmoveto{\pgfqpoint{0.000000in}{0.000000in}}%
\pgfpathlineto{\pgfqpoint{-0.027778in}{0.000000in}}%
\pgfusepath{stroke,fill}%
}%
\begin{pgfscope}%
\pgfsys@transformshift{7.200000in}{1.693483in}%
\pgfsys@useobject{currentmarker}{}%
\end{pgfscope}%
\end{pgfscope}%
\begin{pgfscope}%
\pgfsetbuttcap%
\pgfsetroundjoin%
\definecolor{currentfill}{rgb}{0.000000,0.000000,0.000000}%
\pgfsetfillcolor{currentfill}%
\pgfsetlinewidth{0.501875pt}%
\definecolor{currentstroke}{rgb}{0.000000,0.000000,0.000000}%
\pgfsetstrokecolor{currentstroke}%
\pgfsetdash{}{0pt}%
\pgfsys@defobject{currentmarker}{\pgfqpoint{0.000000in}{0.000000in}}{\pgfqpoint{0.027778in}{0.000000in}}{%
\pgfpathmoveto{\pgfqpoint{0.000000in}{0.000000in}}%
\pgfpathlineto{\pgfqpoint{0.027778in}{0.000000in}}%
\pgfusepath{stroke,fill}%
}%
\begin{pgfscope}%
\pgfsys@transformshift{1.000000in}{1.718036in}%
\pgfsys@useobject{currentmarker}{}%
\end{pgfscope}%
\end{pgfscope}%
\begin{pgfscope}%
\pgfsetbuttcap%
\pgfsetroundjoin%
\definecolor{currentfill}{rgb}{0.000000,0.000000,0.000000}%
\pgfsetfillcolor{currentfill}%
\pgfsetlinewidth{0.501875pt}%
\definecolor{currentstroke}{rgb}{0.000000,0.000000,0.000000}%
\pgfsetstrokecolor{currentstroke}%
\pgfsetdash{}{0pt}%
\pgfsys@defobject{currentmarker}{\pgfqpoint{-0.027778in}{0.000000in}}{\pgfqpoint{0.000000in}{0.000000in}}{%
\pgfpathmoveto{\pgfqpoint{0.000000in}{0.000000in}}%
\pgfpathlineto{\pgfqpoint{-0.027778in}{0.000000in}}%
\pgfusepath{stroke,fill}%
}%
\begin{pgfscope}%
\pgfsys@transformshift{7.200000in}{1.718036in}%
\pgfsys@useobject{currentmarker}{}%
\end{pgfscope}%
\end{pgfscope}%
\begin{pgfscope}%
\pgfsetbuttcap%
\pgfsetroundjoin%
\definecolor{currentfill}{rgb}{0.000000,0.000000,0.000000}%
\pgfsetfillcolor{currentfill}%
\pgfsetlinewidth{0.501875pt}%
\definecolor{currentstroke}{rgb}{0.000000,0.000000,0.000000}%
\pgfsetstrokecolor{currentstroke}%
\pgfsetdash{}{0pt}%
\pgfsys@defobject{currentmarker}{\pgfqpoint{0.000000in}{0.000000in}}{\pgfqpoint{0.027778in}{0.000000in}}{%
\pgfpathmoveto{\pgfqpoint{0.000000in}{0.000000in}}%
\pgfpathlineto{\pgfqpoint{0.027778in}{0.000000in}}%
\pgfusepath{stroke,fill}%
}%
\begin{pgfscope}%
\pgfsys@transformshift{1.000000in}{1.884494in}%
\pgfsys@useobject{currentmarker}{}%
\end{pgfscope}%
\end{pgfscope}%
\begin{pgfscope}%
\pgfsetbuttcap%
\pgfsetroundjoin%
\definecolor{currentfill}{rgb}{0.000000,0.000000,0.000000}%
\pgfsetfillcolor{currentfill}%
\pgfsetlinewidth{0.501875pt}%
\definecolor{currentstroke}{rgb}{0.000000,0.000000,0.000000}%
\pgfsetstrokecolor{currentstroke}%
\pgfsetdash{}{0pt}%
\pgfsys@defobject{currentmarker}{\pgfqpoint{-0.027778in}{0.000000in}}{\pgfqpoint{0.000000in}{0.000000in}}{%
\pgfpathmoveto{\pgfqpoint{0.000000in}{0.000000in}}%
\pgfpathlineto{\pgfqpoint{-0.027778in}{0.000000in}}%
\pgfusepath{stroke,fill}%
}%
\begin{pgfscope}%
\pgfsys@transformshift{7.200000in}{1.884494in}%
\pgfsys@useobject{currentmarker}{}%
\end{pgfscope}%
\end{pgfscope}%
\begin{pgfscope}%
\pgfsetbuttcap%
\pgfsetroundjoin%
\definecolor{currentfill}{rgb}{0.000000,0.000000,0.000000}%
\pgfsetfillcolor{currentfill}%
\pgfsetlinewidth{0.501875pt}%
\definecolor{currentstroke}{rgb}{0.000000,0.000000,0.000000}%
\pgfsetstrokecolor{currentstroke}%
\pgfsetdash{}{0pt}%
\pgfsys@defobject{currentmarker}{\pgfqpoint{0.000000in}{0.000000in}}{\pgfqpoint{0.027778in}{0.000000in}}{%
\pgfpathmoveto{\pgfqpoint{0.000000in}{0.000000in}}%
\pgfpathlineto{\pgfqpoint{0.027778in}{0.000000in}}%
\pgfusepath{stroke,fill}%
}%
\begin{pgfscope}%
\pgfsys@transformshift{1.000000in}{1.969018in}%
\pgfsys@useobject{currentmarker}{}%
\end{pgfscope}%
\end{pgfscope}%
\begin{pgfscope}%
\pgfsetbuttcap%
\pgfsetroundjoin%
\definecolor{currentfill}{rgb}{0.000000,0.000000,0.000000}%
\pgfsetfillcolor{currentfill}%
\pgfsetlinewidth{0.501875pt}%
\definecolor{currentstroke}{rgb}{0.000000,0.000000,0.000000}%
\pgfsetstrokecolor{currentstroke}%
\pgfsetdash{}{0pt}%
\pgfsys@defobject{currentmarker}{\pgfqpoint{-0.027778in}{0.000000in}}{\pgfqpoint{0.000000in}{0.000000in}}{%
\pgfpathmoveto{\pgfqpoint{0.000000in}{0.000000in}}%
\pgfpathlineto{\pgfqpoint{-0.027778in}{0.000000in}}%
\pgfusepath{stroke,fill}%
}%
\begin{pgfscope}%
\pgfsys@transformshift{7.200000in}{1.969018in}%
\pgfsys@useobject{currentmarker}{}%
\end{pgfscope}%
\end{pgfscope}%
\begin{pgfscope}%
\pgfsetbuttcap%
\pgfsetroundjoin%
\definecolor{currentfill}{rgb}{0.000000,0.000000,0.000000}%
\pgfsetfillcolor{currentfill}%
\pgfsetlinewidth{0.501875pt}%
\definecolor{currentstroke}{rgb}{0.000000,0.000000,0.000000}%
\pgfsetstrokecolor{currentstroke}%
\pgfsetdash{}{0pt}%
\pgfsys@defobject{currentmarker}{\pgfqpoint{0.000000in}{0.000000in}}{\pgfqpoint{0.027778in}{0.000000in}}{%
\pgfpathmoveto{\pgfqpoint{0.000000in}{0.000000in}}%
\pgfpathlineto{\pgfqpoint{0.027778in}{0.000000in}}%
\pgfusepath{stroke,fill}%
}%
\begin{pgfscope}%
\pgfsys@transformshift{1.000000in}{2.028989in}%
\pgfsys@useobject{currentmarker}{}%
\end{pgfscope}%
\end{pgfscope}%
\begin{pgfscope}%
\pgfsetbuttcap%
\pgfsetroundjoin%
\definecolor{currentfill}{rgb}{0.000000,0.000000,0.000000}%
\pgfsetfillcolor{currentfill}%
\pgfsetlinewidth{0.501875pt}%
\definecolor{currentstroke}{rgb}{0.000000,0.000000,0.000000}%
\pgfsetstrokecolor{currentstroke}%
\pgfsetdash{}{0pt}%
\pgfsys@defobject{currentmarker}{\pgfqpoint{-0.027778in}{0.000000in}}{\pgfqpoint{0.000000in}{0.000000in}}{%
\pgfpathmoveto{\pgfqpoint{0.000000in}{0.000000in}}%
\pgfpathlineto{\pgfqpoint{-0.027778in}{0.000000in}}%
\pgfusepath{stroke,fill}%
}%
\begin{pgfscope}%
\pgfsys@transformshift{7.200000in}{2.028989in}%
\pgfsys@useobject{currentmarker}{}%
\end{pgfscope}%
\end{pgfscope}%
\begin{pgfscope}%
\pgfsetbuttcap%
\pgfsetroundjoin%
\definecolor{currentfill}{rgb}{0.000000,0.000000,0.000000}%
\pgfsetfillcolor{currentfill}%
\pgfsetlinewidth{0.501875pt}%
\definecolor{currentstroke}{rgb}{0.000000,0.000000,0.000000}%
\pgfsetstrokecolor{currentstroke}%
\pgfsetdash{}{0pt}%
\pgfsys@defobject{currentmarker}{\pgfqpoint{0.000000in}{0.000000in}}{\pgfqpoint{0.027778in}{0.000000in}}{%
\pgfpathmoveto{\pgfqpoint{0.000000in}{0.000000in}}%
\pgfpathlineto{\pgfqpoint{0.027778in}{0.000000in}}%
\pgfusepath{stroke,fill}%
}%
\begin{pgfscope}%
\pgfsys@transformshift{1.000000in}{2.075506in}%
\pgfsys@useobject{currentmarker}{}%
\end{pgfscope}%
\end{pgfscope}%
\begin{pgfscope}%
\pgfsetbuttcap%
\pgfsetroundjoin%
\definecolor{currentfill}{rgb}{0.000000,0.000000,0.000000}%
\pgfsetfillcolor{currentfill}%
\pgfsetlinewidth{0.501875pt}%
\definecolor{currentstroke}{rgb}{0.000000,0.000000,0.000000}%
\pgfsetstrokecolor{currentstroke}%
\pgfsetdash{}{0pt}%
\pgfsys@defobject{currentmarker}{\pgfqpoint{-0.027778in}{0.000000in}}{\pgfqpoint{0.000000in}{0.000000in}}{%
\pgfpathmoveto{\pgfqpoint{0.000000in}{0.000000in}}%
\pgfpathlineto{\pgfqpoint{-0.027778in}{0.000000in}}%
\pgfusepath{stroke,fill}%
}%
\begin{pgfscope}%
\pgfsys@transformshift{7.200000in}{2.075506in}%
\pgfsys@useobject{currentmarker}{}%
\end{pgfscope}%
\end{pgfscope}%
\begin{pgfscope}%
\pgfsetbuttcap%
\pgfsetroundjoin%
\definecolor{currentfill}{rgb}{0.000000,0.000000,0.000000}%
\pgfsetfillcolor{currentfill}%
\pgfsetlinewidth{0.501875pt}%
\definecolor{currentstroke}{rgb}{0.000000,0.000000,0.000000}%
\pgfsetstrokecolor{currentstroke}%
\pgfsetdash{}{0pt}%
\pgfsys@defobject{currentmarker}{\pgfqpoint{0.000000in}{0.000000in}}{\pgfqpoint{0.027778in}{0.000000in}}{%
\pgfpathmoveto{\pgfqpoint{0.000000in}{0.000000in}}%
\pgfpathlineto{\pgfqpoint{0.027778in}{0.000000in}}%
\pgfusepath{stroke,fill}%
}%
\begin{pgfscope}%
\pgfsys@transformshift{1.000000in}{2.113513in}%
\pgfsys@useobject{currentmarker}{}%
\end{pgfscope}%
\end{pgfscope}%
\begin{pgfscope}%
\pgfsetbuttcap%
\pgfsetroundjoin%
\definecolor{currentfill}{rgb}{0.000000,0.000000,0.000000}%
\pgfsetfillcolor{currentfill}%
\pgfsetlinewidth{0.501875pt}%
\definecolor{currentstroke}{rgb}{0.000000,0.000000,0.000000}%
\pgfsetstrokecolor{currentstroke}%
\pgfsetdash{}{0pt}%
\pgfsys@defobject{currentmarker}{\pgfqpoint{-0.027778in}{0.000000in}}{\pgfqpoint{0.000000in}{0.000000in}}{%
\pgfpathmoveto{\pgfqpoint{0.000000in}{0.000000in}}%
\pgfpathlineto{\pgfqpoint{-0.027778in}{0.000000in}}%
\pgfusepath{stroke,fill}%
}%
\begin{pgfscope}%
\pgfsys@transformshift{7.200000in}{2.113513in}%
\pgfsys@useobject{currentmarker}{}%
\end{pgfscope}%
\end{pgfscope}%
\begin{pgfscope}%
\pgfsetbuttcap%
\pgfsetroundjoin%
\definecolor{currentfill}{rgb}{0.000000,0.000000,0.000000}%
\pgfsetfillcolor{currentfill}%
\pgfsetlinewidth{0.501875pt}%
\definecolor{currentstroke}{rgb}{0.000000,0.000000,0.000000}%
\pgfsetstrokecolor{currentstroke}%
\pgfsetdash{}{0pt}%
\pgfsys@defobject{currentmarker}{\pgfqpoint{0.000000in}{0.000000in}}{\pgfqpoint{0.027778in}{0.000000in}}{%
\pgfpathmoveto{\pgfqpoint{0.000000in}{0.000000in}}%
\pgfpathlineto{\pgfqpoint{0.027778in}{0.000000in}}%
\pgfusepath{stroke,fill}%
}%
\begin{pgfscope}%
\pgfsys@transformshift{1.000000in}{2.145647in}%
\pgfsys@useobject{currentmarker}{}%
\end{pgfscope}%
\end{pgfscope}%
\begin{pgfscope}%
\pgfsetbuttcap%
\pgfsetroundjoin%
\definecolor{currentfill}{rgb}{0.000000,0.000000,0.000000}%
\pgfsetfillcolor{currentfill}%
\pgfsetlinewidth{0.501875pt}%
\definecolor{currentstroke}{rgb}{0.000000,0.000000,0.000000}%
\pgfsetstrokecolor{currentstroke}%
\pgfsetdash{}{0pt}%
\pgfsys@defobject{currentmarker}{\pgfqpoint{-0.027778in}{0.000000in}}{\pgfqpoint{0.000000in}{0.000000in}}{%
\pgfpathmoveto{\pgfqpoint{0.000000in}{0.000000in}}%
\pgfpathlineto{\pgfqpoint{-0.027778in}{0.000000in}}%
\pgfusepath{stroke,fill}%
}%
\begin{pgfscope}%
\pgfsys@transformshift{7.200000in}{2.145647in}%
\pgfsys@useobject{currentmarker}{}%
\end{pgfscope}%
\end{pgfscope}%
\begin{pgfscope}%
\pgfsetbuttcap%
\pgfsetroundjoin%
\definecolor{currentfill}{rgb}{0.000000,0.000000,0.000000}%
\pgfsetfillcolor{currentfill}%
\pgfsetlinewidth{0.501875pt}%
\definecolor{currentstroke}{rgb}{0.000000,0.000000,0.000000}%
\pgfsetstrokecolor{currentstroke}%
\pgfsetdash{}{0pt}%
\pgfsys@defobject{currentmarker}{\pgfqpoint{0.000000in}{0.000000in}}{\pgfqpoint{0.027778in}{0.000000in}}{%
\pgfpathmoveto{\pgfqpoint{0.000000in}{0.000000in}}%
\pgfpathlineto{\pgfqpoint{0.027778in}{0.000000in}}%
\pgfusepath{stroke,fill}%
}%
\begin{pgfscope}%
\pgfsys@transformshift{1.000000in}{2.173483in}%
\pgfsys@useobject{currentmarker}{}%
\end{pgfscope}%
\end{pgfscope}%
\begin{pgfscope}%
\pgfsetbuttcap%
\pgfsetroundjoin%
\definecolor{currentfill}{rgb}{0.000000,0.000000,0.000000}%
\pgfsetfillcolor{currentfill}%
\pgfsetlinewidth{0.501875pt}%
\definecolor{currentstroke}{rgb}{0.000000,0.000000,0.000000}%
\pgfsetstrokecolor{currentstroke}%
\pgfsetdash{}{0pt}%
\pgfsys@defobject{currentmarker}{\pgfqpoint{-0.027778in}{0.000000in}}{\pgfqpoint{0.000000in}{0.000000in}}{%
\pgfpathmoveto{\pgfqpoint{0.000000in}{0.000000in}}%
\pgfpathlineto{\pgfqpoint{-0.027778in}{0.000000in}}%
\pgfusepath{stroke,fill}%
}%
\begin{pgfscope}%
\pgfsys@transformshift{7.200000in}{2.173483in}%
\pgfsys@useobject{currentmarker}{}%
\end{pgfscope}%
\end{pgfscope}%
\begin{pgfscope}%
\pgfsetbuttcap%
\pgfsetroundjoin%
\definecolor{currentfill}{rgb}{0.000000,0.000000,0.000000}%
\pgfsetfillcolor{currentfill}%
\pgfsetlinewidth{0.501875pt}%
\definecolor{currentstroke}{rgb}{0.000000,0.000000,0.000000}%
\pgfsetstrokecolor{currentstroke}%
\pgfsetdash{}{0pt}%
\pgfsys@defobject{currentmarker}{\pgfqpoint{0.000000in}{0.000000in}}{\pgfqpoint{0.027778in}{0.000000in}}{%
\pgfpathmoveto{\pgfqpoint{0.000000in}{0.000000in}}%
\pgfpathlineto{\pgfqpoint{0.027778in}{0.000000in}}%
\pgfusepath{stroke,fill}%
}%
\begin{pgfscope}%
\pgfsys@transformshift{1.000000in}{2.198036in}%
\pgfsys@useobject{currentmarker}{}%
\end{pgfscope}%
\end{pgfscope}%
\begin{pgfscope}%
\pgfsetbuttcap%
\pgfsetroundjoin%
\definecolor{currentfill}{rgb}{0.000000,0.000000,0.000000}%
\pgfsetfillcolor{currentfill}%
\pgfsetlinewidth{0.501875pt}%
\definecolor{currentstroke}{rgb}{0.000000,0.000000,0.000000}%
\pgfsetstrokecolor{currentstroke}%
\pgfsetdash{}{0pt}%
\pgfsys@defobject{currentmarker}{\pgfqpoint{-0.027778in}{0.000000in}}{\pgfqpoint{0.000000in}{0.000000in}}{%
\pgfpathmoveto{\pgfqpoint{0.000000in}{0.000000in}}%
\pgfpathlineto{\pgfqpoint{-0.027778in}{0.000000in}}%
\pgfusepath{stroke,fill}%
}%
\begin{pgfscope}%
\pgfsys@transformshift{7.200000in}{2.198036in}%
\pgfsys@useobject{currentmarker}{}%
\end{pgfscope}%
\end{pgfscope}%
\begin{pgfscope}%
\pgfsetbuttcap%
\pgfsetroundjoin%
\definecolor{currentfill}{rgb}{0.000000,0.000000,0.000000}%
\pgfsetfillcolor{currentfill}%
\pgfsetlinewidth{0.501875pt}%
\definecolor{currentstroke}{rgb}{0.000000,0.000000,0.000000}%
\pgfsetstrokecolor{currentstroke}%
\pgfsetdash{}{0pt}%
\pgfsys@defobject{currentmarker}{\pgfqpoint{0.000000in}{0.000000in}}{\pgfqpoint{0.027778in}{0.000000in}}{%
\pgfpathmoveto{\pgfqpoint{0.000000in}{0.000000in}}%
\pgfpathlineto{\pgfqpoint{0.027778in}{0.000000in}}%
\pgfusepath{stroke,fill}%
}%
\begin{pgfscope}%
\pgfsys@transformshift{1.000000in}{2.364494in}%
\pgfsys@useobject{currentmarker}{}%
\end{pgfscope}%
\end{pgfscope}%
\begin{pgfscope}%
\pgfsetbuttcap%
\pgfsetroundjoin%
\definecolor{currentfill}{rgb}{0.000000,0.000000,0.000000}%
\pgfsetfillcolor{currentfill}%
\pgfsetlinewidth{0.501875pt}%
\definecolor{currentstroke}{rgb}{0.000000,0.000000,0.000000}%
\pgfsetstrokecolor{currentstroke}%
\pgfsetdash{}{0pt}%
\pgfsys@defobject{currentmarker}{\pgfqpoint{-0.027778in}{0.000000in}}{\pgfqpoint{0.000000in}{0.000000in}}{%
\pgfpathmoveto{\pgfqpoint{0.000000in}{0.000000in}}%
\pgfpathlineto{\pgfqpoint{-0.027778in}{0.000000in}}%
\pgfusepath{stroke,fill}%
}%
\begin{pgfscope}%
\pgfsys@transformshift{7.200000in}{2.364494in}%
\pgfsys@useobject{currentmarker}{}%
\end{pgfscope}%
\end{pgfscope}%
\begin{pgfscope}%
\pgfsetbuttcap%
\pgfsetroundjoin%
\definecolor{currentfill}{rgb}{0.000000,0.000000,0.000000}%
\pgfsetfillcolor{currentfill}%
\pgfsetlinewidth{0.501875pt}%
\definecolor{currentstroke}{rgb}{0.000000,0.000000,0.000000}%
\pgfsetstrokecolor{currentstroke}%
\pgfsetdash{}{0pt}%
\pgfsys@defobject{currentmarker}{\pgfqpoint{0.000000in}{0.000000in}}{\pgfqpoint{0.027778in}{0.000000in}}{%
\pgfpathmoveto{\pgfqpoint{0.000000in}{0.000000in}}%
\pgfpathlineto{\pgfqpoint{0.027778in}{0.000000in}}%
\pgfusepath{stroke,fill}%
}%
\begin{pgfscope}%
\pgfsys@transformshift{1.000000in}{2.449018in}%
\pgfsys@useobject{currentmarker}{}%
\end{pgfscope}%
\end{pgfscope}%
\begin{pgfscope}%
\pgfsetbuttcap%
\pgfsetroundjoin%
\definecolor{currentfill}{rgb}{0.000000,0.000000,0.000000}%
\pgfsetfillcolor{currentfill}%
\pgfsetlinewidth{0.501875pt}%
\definecolor{currentstroke}{rgb}{0.000000,0.000000,0.000000}%
\pgfsetstrokecolor{currentstroke}%
\pgfsetdash{}{0pt}%
\pgfsys@defobject{currentmarker}{\pgfqpoint{-0.027778in}{0.000000in}}{\pgfqpoint{0.000000in}{0.000000in}}{%
\pgfpathmoveto{\pgfqpoint{0.000000in}{0.000000in}}%
\pgfpathlineto{\pgfqpoint{-0.027778in}{0.000000in}}%
\pgfusepath{stroke,fill}%
}%
\begin{pgfscope}%
\pgfsys@transformshift{7.200000in}{2.449018in}%
\pgfsys@useobject{currentmarker}{}%
\end{pgfscope}%
\end{pgfscope}%
\begin{pgfscope}%
\pgfsetbuttcap%
\pgfsetroundjoin%
\definecolor{currentfill}{rgb}{0.000000,0.000000,0.000000}%
\pgfsetfillcolor{currentfill}%
\pgfsetlinewidth{0.501875pt}%
\definecolor{currentstroke}{rgb}{0.000000,0.000000,0.000000}%
\pgfsetstrokecolor{currentstroke}%
\pgfsetdash{}{0pt}%
\pgfsys@defobject{currentmarker}{\pgfqpoint{0.000000in}{0.000000in}}{\pgfqpoint{0.027778in}{0.000000in}}{%
\pgfpathmoveto{\pgfqpoint{0.000000in}{0.000000in}}%
\pgfpathlineto{\pgfqpoint{0.027778in}{0.000000in}}%
\pgfusepath{stroke,fill}%
}%
\begin{pgfscope}%
\pgfsys@transformshift{1.000000in}{2.508989in}%
\pgfsys@useobject{currentmarker}{}%
\end{pgfscope}%
\end{pgfscope}%
\begin{pgfscope}%
\pgfsetbuttcap%
\pgfsetroundjoin%
\definecolor{currentfill}{rgb}{0.000000,0.000000,0.000000}%
\pgfsetfillcolor{currentfill}%
\pgfsetlinewidth{0.501875pt}%
\definecolor{currentstroke}{rgb}{0.000000,0.000000,0.000000}%
\pgfsetstrokecolor{currentstroke}%
\pgfsetdash{}{0pt}%
\pgfsys@defobject{currentmarker}{\pgfqpoint{-0.027778in}{0.000000in}}{\pgfqpoint{0.000000in}{0.000000in}}{%
\pgfpathmoveto{\pgfqpoint{0.000000in}{0.000000in}}%
\pgfpathlineto{\pgfqpoint{-0.027778in}{0.000000in}}%
\pgfusepath{stroke,fill}%
}%
\begin{pgfscope}%
\pgfsys@transformshift{7.200000in}{2.508989in}%
\pgfsys@useobject{currentmarker}{}%
\end{pgfscope}%
\end{pgfscope}%
\begin{pgfscope}%
\pgfsetbuttcap%
\pgfsetroundjoin%
\definecolor{currentfill}{rgb}{0.000000,0.000000,0.000000}%
\pgfsetfillcolor{currentfill}%
\pgfsetlinewidth{0.501875pt}%
\definecolor{currentstroke}{rgb}{0.000000,0.000000,0.000000}%
\pgfsetstrokecolor{currentstroke}%
\pgfsetdash{}{0pt}%
\pgfsys@defobject{currentmarker}{\pgfqpoint{0.000000in}{0.000000in}}{\pgfqpoint{0.027778in}{0.000000in}}{%
\pgfpathmoveto{\pgfqpoint{0.000000in}{0.000000in}}%
\pgfpathlineto{\pgfqpoint{0.027778in}{0.000000in}}%
\pgfusepath{stroke,fill}%
}%
\begin{pgfscope}%
\pgfsys@transformshift{1.000000in}{2.555506in}%
\pgfsys@useobject{currentmarker}{}%
\end{pgfscope}%
\end{pgfscope}%
\begin{pgfscope}%
\pgfsetbuttcap%
\pgfsetroundjoin%
\definecolor{currentfill}{rgb}{0.000000,0.000000,0.000000}%
\pgfsetfillcolor{currentfill}%
\pgfsetlinewidth{0.501875pt}%
\definecolor{currentstroke}{rgb}{0.000000,0.000000,0.000000}%
\pgfsetstrokecolor{currentstroke}%
\pgfsetdash{}{0pt}%
\pgfsys@defobject{currentmarker}{\pgfqpoint{-0.027778in}{0.000000in}}{\pgfqpoint{0.000000in}{0.000000in}}{%
\pgfpathmoveto{\pgfqpoint{0.000000in}{0.000000in}}%
\pgfpathlineto{\pgfqpoint{-0.027778in}{0.000000in}}%
\pgfusepath{stroke,fill}%
}%
\begin{pgfscope}%
\pgfsys@transformshift{7.200000in}{2.555506in}%
\pgfsys@useobject{currentmarker}{}%
\end{pgfscope}%
\end{pgfscope}%
\begin{pgfscope}%
\pgfsetbuttcap%
\pgfsetroundjoin%
\definecolor{currentfill}{rgb}{0.000000,0.000000,0.000000}%
\pgfsetfillcolor{currentfill}%
\pgfsetlinewidth{0.501875pt}%
\definecolor{currentstroke}{rgb}{0.000000,0.000000,0.000000}%
\pgfsetstrokecolor{currentstroke}%
\pgfsetdash{}{0pt}%
\pgfsys@defobject{currentmarker}{\pgfqpoint{0.000000in}{0.000000in}}{\pgfqpoint{0.027778in}{0.000000in}}{%
\pgfpathmoveto{\pgfqpoint{0.000000in}{0.000000in}}%
\pgfpathlineto{\pgfqpoint{0.027778in}{0.000000in}}%
\pgfusepath{stroke,fill}%
}%
\begin{pgfscope}%
\pgfsys@transformshift{1.000000in}{2.593513in}%
\pgfsys@useobject{currentmarker}{}%
\end{pgfscope}%
\end{pgfscope}%
\begin{pgfscope}%
\pgfsetbuttcap%
\pgfsetroundjoin%
\definecolor{currentfill}{rgb}{0.000000,0.000000,0.000000}%
\pgfsetfillcolor{currentfill}%
\pgfsetlinewidth{0.501875pt}%
\definecolor{currentstroke}{rgb}{0.000000,0.000000,0.000000}%
\pgfsetstrokecolor{currentstroke}%
\pgfsetdash{}{0pt}%
\pgfsys@defobject{currentmarker}{\pgfqpoint{-0.027778in}{0.000000in}}{\pgfqpoint{0.000000in}{0.000000in}}{%
\pgfpathmoveto{\pgfqpoint{0.000000in}{0.000000in}}%
\pgfpathlineto{\pgfqpoint{-0.027778in}{0.000000in}}%
\pgfusepath{stroke,fill}%
}%
\begin{pgfscope}%
\pgfsys@transformshift{7.200000in}{2.593513in}%
\pgfsys@useobject{currentmarker}{}%
\end{pgfscope}%
\end{pgfscope}%
\begin{pgfscope}%
\pgfsetbuttcap%
\pgfsetroundjoin%
\definecolor{currentfill}{rgb}{0.000000,0.000000,0.000000}%
\pgfsetfillcolor{currentfill}%
\pgfsetlinewidth{0.501875pt}%
\definecolor{currentstroke}{rgb}{0.000000,0.000000,0.000000}%
\pgfsetstrokecolor{currentstroke}%
\pgfsetdash{}{0pt}%
\pgfsys@defobject{currentmarker}{\pgfqpoint{0.000000in}{0.000000in}}{\pgfqpoint{0.027778in}{0.000000in}}{%
\pgfpathmoveto{\pgfqpoint{0.000000in}{0.000000in}}%
\pgfpathlineto{\pgfqpoint{0.027778in}{0.000000in}}%
\pgfusepath{stroke,fill}%
}%
\begin{pgfscope}%
\pgfsys@transformshift{1.000000in}{2.625647in}%
\pgfsys@useobject{currentmarker}{}%
\end{pgfscope}%
\end{pgfscope}%
\begin{pgfscope}%
\pgfsetbuttcap%
\pgfsetroundjoin%
\definecolor{currentfill}{rgb}{0.000000,0.000000,0.000000}%
\pgfsetfillcolor{currentfill}%
\pgfsetlinewidth{0.501875pt}%
\definecolor{currentstroke}{rgb}{0.000000,0.000000,0.000000}%
\pgfsetstrokecolor{currentstroke}%
\pgfsetdash{}{0pt}%
\pgfsys@defobject{currentmarker}{\pgfqpoint{-0.027778in}{0.000000in}}{\pgfqpoint{0.000000in}{0.000000in}}{%
\pgfpathmoveto{\pgfqpoint{0.000000in}{0.000000in}}%
\pgfpathlineto{\pgfqpoint{-0.027778in}{0.000000in}}%
\pgfusepath{stroke,fill}%
}%
\begin{pgfscope}%
\pgfsys@transformshift{7.200000in}{2.625647in}%
\pgfsys@useobject{currentmarker}{}%
\end{pgfscope}%
\end{pgfscope}%
\begin{pgfscope}%
\pgfsetbuttcap%
\pgfsetroundjoin%
\definecolor{currentfill}{rgb}{0.000000,0.000000,0.000000}%
\pgfsetfillcolor{currentfill}%
\pgfsetlinewidth{0.501875pt}%
\definecolor{currentstroke}{rgb}{0.000000,0.000000,0.000000}%
\pgfsetstrokecolor{currentstroke}%
\pgfsetdash{}{0pt}%
\pgfsys@defobject{currentmarker}{\pgfqpoint{0.000000in}{0.000000in}}{\pgfqpoint{0.027778in}{0.000000in}}{%
\pgfpathmoveto{\pgfqpoint{0.000000in}{0.000000in}}%
\pgfpathlineto{\pgfqpoint{0.027778in}{0.000000in}}%
\pgfusepath{stroke,fill}%
}%
\begin{pgfscope}%
\pgfsys@transformshift{1.000000in}{2.653483in}%
\pgfsys@useobject{currentmarker}{}%
\end{pgfscope}%
\end{pgfscope}%
\begin{pgfscope}%
\pgfsetbuttcap%
\pgfsetroundjoin%
\definecolor{currentfill}{rgb}{0.000000,0.000000,0.000000}%
\pgfsetfillcolor{currentfill}%
\pgfsetlinewidth{0.501875pt}%
\definecolor{currentstroke}{rgb}{0.000000,0.000000,0.000000}%
\pgfsetstrokecolor{currentstroke}%
\pgfsetdash{}{0pt}%
\pgfsys@defobject{currentmarker}{\pgfqpoint{-0.027778in}{0.000000in}}{\pgfqpoint{0.000000in}{0.000000in}}{%
\pgfpathmoveto{\pgfqpoint{0.000000in}{0.000000in}}%
\pgfpathlineto{\pgfqpoint{-0.027778in}{0.000000in}}%
\pgfusepath{stroke,fill}%
}%
\begin{pgfscope}%
\pgfsys@transformshift{7.200000in}{2.653483in}%
\pgfsys@useobject{currentmarker}{}%
\end{pgfscope}%
\end{pgfscope}%
\begin{pgfscope}%
\pgfsetbuttcap%
\pgfsetroundjoin%
\definecolor{currentfill}{rgb}{0.000000,0.000000,0.000000}%
\pgfsetfillcolor{currentfill}%
\pgfsetlinewidth{0.501875pt}%
\definecolor{currentstroke}{rgb}{0.000000,0.000000,0.000000}%
\pgfsetstrokecolor{currentstroke}%
\pgfsetdash{}{0pt}%
\pgfsys@defobject{currentmarker}{\pgfqpoint{0.000000in}{0.000000in}}{\pgfqpoint{0.027778in}{0.000000in}}{%
\pgfpathmoveto{\pgfqpoint{0.000000in}{0.000000in}}%
\pgfpathlineto{\pgfqpoint{0.027778in}{0.000000in}}%
\pgfusepath{stroke,fill}%
}%
\begin{pgfscope}%
\pgfsys@transformshift{1.000000in}{2.678036in}%
\pgfsys@useobject{currentmarker}{}%
\end{pgfscope}%
\end{pgfscope}%
\begin{pgfscope}%
\pgfsetbuttcap%
\pgfsetroundjoin%
\definecolor{currentfill}{rgb}{0.000000,0.000000,0.000000}%
\pgfsetfillcolor{currentfill}%
\pgfsetlinewidth{0.501875pt}%
\definecolor{currentstroke}{rgb}{0.000000,0.000000,0.000000}%
\pgfsetstrokecolor{currentstroke}%
\pgfsetdash{}{0pt}%
\pgfsys@defobject{currentmarker}{\pgfqpoint{-0.027778in}{0.000000in}}{\pgfqpoint{0.000000in}{0.000000in}}{%
\pgfpathmoveto{\pgfqpoint{0.000000in}{0.000000in}}%
\pgfpathlineto{\pgfqpoint{-0.027778in}{0.000000in}}%
\pgfusepath{stroke,fill}%
}%
\begin{pgfscope}%
\pgfsys@transformshift{7.200000in}{2.678036in}%
\pgfsys@useobject{currentmarker}{}%
\end{pgfscope}%
\end{pgfscope}%
\begin{pgfscope}%
\pgftext[left,bottom,x=0.554012in,y=0.106703in,rotate=90.000000]{{\sffamily\fontsize{12.000000}{14.400000}\selectfont mean square displacement [nm\(\displaystyle ^2\)]}}
%
\end{pgfscope}%
\begin{pgfscope}%
\pgfsetrectcap%
\pgfsetroundjoin%
\pgfsetlinewidth{1.003750pt}%
\definecolor{currentstroke}{rgb}{0.000000,0.000000,0.000000}%
\pgfsetstrokecolor{currentstroke}%
\pgfsetdash{}{0pt}%
\pgfpathmoveto{\pgfqpoint{1.000000in}{2.700000in}}%
\pgfpathlineto{\pgfqpoint{7.200000in}{2.700000in}}%
\pgfusepath{stroke}%
\end{pgfscope}%
\begin{pgfscope}%
\pgfsetrectcap%
\pgfsetroundjoin%
\pgfsetlinewidth{1.003750pt}%
\definecolor{currentstroke}{rgb}{0.000000,0.000000,0.000000}%
\pgfsetstrokecolor{currentstroke}%
\pgfsetdash{}{0pt}%
\pgfpathmoveto{\pgfqpoint{7.200000in}{0.300000in}}%
\pgfpathlineto{\pgfqpoint{7.200000in}{2.700000in}}%
\pgfusepath{stroke}%
\end{pgfscope}%
\begin{pgfscope}%
\pgfsetrectcap%
\pgfsetroundjoin%
\pgfsetlinewidth{1.003750pt}%
\definecolor{currentstroke}{rgb}{0.000000,0.000000,0.000000}%
\pgfsetstrokecolor{currentstroke}%
\pgfsetdash{}{0pt}%
\pgfpathmoveto{\pgfqpoint{1.000000in}{0.300000in}}%
\pgfpathlineto{\pgfqpoint{7.200000in}{0.300000in}}%
\pgfusepath{stroke}%
\end{pgfscope}%
\begin{pgfscope}%
\pgfsetrectcap%
\pgfsetroundjoin%
\pgfsetlinewidth{1.003750pt}%
\definecolor{currentstroke}{rgb}{0.000000,0.000000,0.000000}%
\pgfsetstrokecolor{currentstroke}%
\pgfsetdash{}{0pt}%
\pgfpathmoveto{\pgfqpoint{1.000000in}{0.300000in}}%
\pgfpathlineto{\pgfqpoint{1.000000in}{2.700000in}}%
\pgfusepath{stroke}%
\end{pgfscope}%
\begin{pgfscope}%
\pgfsetrectcap%
\pgfsetroundjoin%
\definecolor{currentfill}{rgb}{1.000000,1.000000,1.000000}%
\pgfsetfillcolor{currentfill}%
\pgfsetlinewidth{1.003750pt}%
\definecolor{currentstroke}{rgb}{0.000000,0.000000,0.000000}%
\pgfsetstrokecolor{currentstroke}%
\pgfsetdash{}{0pt}%
\pgfpathmoveto{\pgfqpoint{1.069417in}{1.977606in}}%
\pgfpathlineto{\pgfqpoint{1.926808in}{1.977606in}}%
\pgfpathlineto{\pgfqpoint{1.926808in}{2.630583in}}%
\pgfpathlineto{\pgfqpoint{1.069417in}{2.630583in}}%
\pgfpathlineto{\pgfqpoint{1.069417in}{1.977606in}}%
\pgfpathclose%
\pgfusepath{stroke,fill}%
\end{pgfscope}%
\begin{pgfscope}%
\pgfsetrectcap%
\pgfsetroundjoin%
\pgfsetlinewidth{1.003750pt}%
\definecolor{currentstroke}{rgb}{0.000000,0.000000,1.000000}%
\pgfsetstrokecolor{currentstroke}%
\pgfsetdash{}{0pt}%
\pgfpathmoveto{\pgfqpoint{1.166600in}{2.518161in}}%
\pgfpathlineto{\pgfqpoint{1.360967in}{2.518161in}}%
\pgfusepath{stroke}%
\end{pgfscope}%
\begin{pgfscope}%
\pgftext[left,bottom,x=1.513683in,y=2.440691in,rotate=0.000000]{{\sffamily\fontsize{9.996000}{11.995200}\selectfont spc}}
%
\end{pgfscope}%
\begin{pgfscope}%
\pgfsetrectcap%
\pgfsetroundjoin%
\pgfsetlinewidth{1.003750pt}%
\definecolor{currentstroke}{rgb}{0.000000,0.500000,0.000000}%
\pgfsetstrokecolor{currentstroke}%
\pgfsetdash{}{0pt}%
\pgfpathmoveto{\pgfqpoint{1.166600in}{2.314385in}}%
\pgfpathlineto{\pgfqpoint{1.360967in}{2.314385in}}%
\pgfusepath{stroke}%
\end{pgfscope}%
\begin{pgfscope}%
\pgftext[left,bottom,x=1.513683in,y=2.236915in,rotate=0.000000]{{\sffamily\fontsize{9.996000}{11.995200}\selectfont spce}}
%
\end{pgfscope}%
\begin{pgfscope}%
\pgfsetrectcap%
\pgfsetroundjoin%
\pgfsetlinewidth{1.003750pt}%
\definecolor{currentstroke}{rgb}{1.000000,0.000000,0.000000}%
\pgfsetstrokecolor{currentstroke}%
\pgfsetdash{}{0pt}%
\pgfpathmoveto{\pgfqpoint{1.166600in}{2.110609in}}%
\pgfpathlineto{\pgfqpoint{1.360967in}{2.110609in}}%
\pgfusepath{stroke}%
\end{pgfscope}%
\begin{pgfscope}%
\pgftext[left,bottom,x=1.513683in,y=2.033139in,rotate=0.000000]{{\sffamily\fontsize{9.996000}{11.995200}\selectfont tip3p}}
%
\end{pgfscope}%
\end{pgfpicture}%
\makeatother%
\endgroup%
}
    \caption{Mean-square displacement.} \label{fig:msd}
\end{figure}

The full-logarithmic plot of the mean-square displacement $\langle \Delta r(t)^2 \rangle$ shows the existence of two regimes. Until $\unit[0.3]{ps}$ the slope is higher than afterwards. In theory the slope at the beginning has to be twice as high because of the quadratic time-dependence of the means-square displacement in the ballistic regime. In the following linear regime the displacement is proportional to the time itself.

\begin{figure}[H]
	\resizebox{\linewidth}{!}{%% Creator: Matplotlib, PGF backend
%%
%% To include the figure in your LaTeX document, write
%%   \input{<filename>.pgf}
%%
%% Make sure the required packages are loaded in your preamble
%%   \usepackage{pgf}
%%
%% Figures using additional raster images can only be included by \input if
%% they are in the same directory as the main LaTeX file. For loading figures
%% from other directories you can use the `import` package
%%   \usepackage{import}
%% and then include the figures with
%%   \import{<path to file>}{<filename>.pgf}
%%
%% Matplotlib used the following preamble
%%   \usepackage{fontspec}
%%   \setmainfont{DejaVu Serif}
%%   \setsansfont{DejaVu Sans}
%%   \setmonofont{DejaVu Sans Mono}
%%
\begingroup%
\makeatletter%
\begin{pgfpicture}%
\pgfpathrectangle{\pgfpointorigin}{\pgfqpoint{8.000000in}{3.500000in}}%
\pgfusepath{use as bounding box}%
\begin{pgfscope}%
\pgfsetrectcap%
\pgfsetroundjoin%
\definecolor{currentfill}{rgb}{1.000000,1.000000,1.000000}%
\pgfsetfillcolor{currentfill}%
\pgfsetlinewidth{0.000000pt}%
\definecolor{currentstroke}{rgb}{1.000000,1.000000,1.000000}%
\pgfsetstrokecolor{currentstroke}%
\pgfsetdash{}{0pt}%
\pgfpathmoveto{\pgfqpoint{0.000000in}{0.000000in}}%
\pgfpathlineto{\pgfqpoint{8.000000in}{0.000000in}}%
\pgfpathlineto{\pgfqpoint{8.000000in}{3.500000in}}%
\pgfpathlineto{\pgfqpoint{0.000000in}{3.500000in}}%
\pgfpathclose%
\pgfusepath{fill}%
\end{pgfscope}%
\begin{pgfscope}%
\pgfsetrectcap%
\pgfsetroundjoin%
\definecolor{currentfill}{rgb}{1.000000,1.000000,1.000000}%
\pgfsetfillcolor{currentfill}%
\pgfsetlinewidth{0.000000pt}%
\definecolor{currentstroke}{rgb}{0.000000,0.000000,0.000000}%
\pgfsetstrokecolor{currentstroke}%
\pgfsetdash{}{0pt}%
\pgfpathmoveto{\pgfqpoint{1.000000in}{0.350000in}}%
\pgfpathlineto{\pgfqpoint{7.200000in}{0.350000in}}%
\pgfpathlineto{\pgfqpoint{7.200000in}{3.150000in}}%
\pgfpathlineto{\pgfqpoint{1.000000in}{3.150000in}}%
\pgfpathclose%
\pgfusepath{fill}%
\end{pgfscope}%
\begin{pgfscope}%
\pgfpathrectangle{\pgfqpoint{1.000000in}{0.350000in}}{\pgfqpoint{6.200000in}{2.800000in}} %
\pgfusepath{clip}%
\pgfsetrectcap%
\pgfsetroundjoin%
\pgfsetlinewidth{1.003750pt}%
\definecolor{currentstroke}{rgb}{0.000000,0.000000,1.000000}%
\pgfsetstrokecolor{currentstroke}%
\pgfsetdash{}{0pt}%
\pgfpathmoveto{\pgfqpoint{1.001240in}{1.693679in}}%
\pgfpathlineto{\pgfqpoint{1.002480in}{2.293444in}}%
\pgfpathlineto{\pgfqpoint{1.003720in}{2.509775in}}%
\pgfpathlineto{\pgfqpoint{1.004960in}{2.524485in}}%
\pgfpathlineto{\pgfqpoint{1.009920in}{2.188602in}}%
\pgfpathlineto{\pgfqpoint{1.019840in}{1.847099in}}%
\pgfpathlineto{\pgfqpoint{1.021080in}{1.840417in}}%
\pgfpathlineto{\pgfqpoint{1.023560in}{1.807506in}}%
\pgfpathlineto{\pgfqpoint{1.026040in}{1.781259in}}%
\pgfpathlineto{\pgfqpoint{1.027280in}{1.766197in}}%
\pgfpathlineto{\pgfqpoint{1.028520in}{1.765199in}}%
\pgfpathlineto{\pgfqpoint{1.029760in}{1.760003in}}%
\pgfpathlineto{\pgfqpoint{1.043400in}{1.648181in}}%
\pgfpathlineto{\pgfqpoint{1.044640in}{1.648676in}}%
\pgfpathlineto{\pgfqpoint{1.047120in}{1.632141in}}%
\pgfpathlineto{\pgfqpoint{1.049600in}{1.632737in}}%
\pgfpathlineto{\pgfqpoint{1.050840in}{1.622928in}}%
\pgfpathlineto{\pgfqpoint{1.054560in}{1.629162in}}%
\pgfpathlineto{\pgfqpoint{1.058280in}{1.618009in}}%
\pgfpathlineto{\pgfqpoint{1.059520in}{1.619420in}}%
\pgfpathlineto{\pgfqpoint{1.060760in}{1.611418in}}%
\pgfpathlineto{\pgfqpoint{1.063240in}{1.612318in}}%
\pgfpathlineto{\pgfqpoint{1.064480in}{1.613311in}}%
\pgfpathlineto{\pgfqpoint{1.068200in}{1.601138in}}%
\pgfpathlineto{\pgfqpoint{1.069440in}{1.602630in}}%
\pgfpathlineto{\pgfqpoint{1.070680in}{1.601795in}}%
\pgfpathlineto{\pgfqpoint{1.073160in}{1.609824in}}%
\pgfpathlineto{\pgfqpoint{1.075640in}{1.608793in}}%
\pgfpathlineto{\pgfqpoint{1.076880in}{1.609381in}}%
\pgfpathlineto{\pgfqpoint{1.081840in}{1.602175in}}%
\pgfpathlineto{\pgfqpoint{1.083080in}{1.601224in}}%
\pgfpathlineto{\pgfqpoint{1.086800in}{1.591837in}}%
\pgfpathlineto{\pgfqpoint{1.090520in}{1.583108in}}%
\pgfpathlineto{\pgfqpoint{1.091760in}{1.582659in}}%
\pgfpathlineto{\pgfqpoint{1.093000in}{1.584019in}}%
\pgfpathlineto{\pgfqpoint{1.096720in}{1.572188in}}%
\pgfpathlineto{\pgfqpoint{1.099200in}{1.574119in}}%
\pgfpathlineto{\pgfqpoint{1.100440in}{1.573435in}}%
\pgfpathlineto{\pgfqpoint{1.101680in}{1.574096in}}%
\pgfpathlineto{\pgfqpoint{1.107880in}{1.548874in}}%
\pgfpathlineto{\pgfqpoint{1.109120in}{1.550960in}}%
\pgfpathlineto{\pgfqpoint{1.110360in}{1.550784in}}%
\pgfpathlineto{\pgfqpoint{1.111600in}{1.549114in}}%
\pgfpathlineto{\pgfqpoint{1.114080in}{1.544892in}}%
\pgfpathlineto{\pgfqpoint{1.116560in}{1.546476in}}%
\pgfpathlineto{\pgfqpoint{1.117800in}{1.545955in}}%
\pgfpathlineto{\pgfqpoint{1.120280in}{1.537643in}}%
\pgfpathlineto{\pgfqpoint{1.121520in}{1.537492in}}%
\pgfpathlineto{\pgfqpoint{1.122760in}{1.538539in}}%
\pgfpathlineto{\pgfqpoint{1.125240in}{1.534532in}}%
\pgfpathlineto{\pgfqpoint{1.126480in}{1.535028in}}%
\pgfpathlineto{\pgfqpoint{1.128960in}{1.539462in}}%
\pgfpathlineto{\pgfqpoint{1.131440in}{1.535744in}}%
\pgfpathlineto{\pgfqpoint{1.133920in}{1.532136in}}%
\pgfpathlineto{\pgfqpoint{1.135160in}{1.529958in}}%
\pgfpathlineto{\pgfqpoint{1.137640in}{1.534380in}}%
\pgfpathlineto{\pgfqpoint{1.142600in}{1.515521in}}%
\pgfpathlineto{\pgfqpoint{1.143840in}{1.515880in}}%
\pgfpathlineto{\pgfqpoint{1.145080in}{1.518475in}}%
\pgfpathlineto{\pgfqpoint{1.146320in}{1.517836in}}%
\pgfpathlineto{\pgfqpoint{1.148800in}{1.524427in}}%
\pgfpathlineto{\pgfqpoint{1.150040in}{1.522966in}}%
\pgfpathlineto{\pgfqpoint{1.152520in}{1.525275in}}%
\pgfpathlineto{\pgfqpoint{1.157480in}{1.518839in}}%
\pgfpathlineto{\pgfqpoint{1.158720in}{1.520394in}}%
\pgfpathlineto{\pgfqpoint{1.161200in}{1.519172in}}%
\pgfpathlineto{\pgfqpoint{1.163680in}{1.518301in}}%
\pgfpathlineto{\pgfqpoint{1.164920in}{1.514000in}}%
\pgfpathlineto{\pgfqpoint{1.168640in}{1.515942in}}%
\pgfpathlineto{\pgfqpoint{1.172360in}{1.521128in}}%
\pgfpathlineto{\pgfqpoint{1.174840in}{1.518572in}}%
\pgfpathlineto{\pgfqpoint{1.177320in}{1.530670in}}%
\pgfpathlineto{\pgfqpoint{1.178560in}{1.532114in}}%
\pgfpathlineto{\pgfqpoint{1.181040in}{1.528841in}}%
\pgfpathlineto{\pgfqpoint{1.183520in}{1.530337in}}%
\pgfpathlineto{\pgfqpoint{1.186000in}{1.527023in}}%
\pgfpathlineto{\pgfqpoint{1.188480in}{1.530053in}}%
\pgfpathlineto{\pgfqpoint{1.190960in}{1.527852in}}%
\pgfpathlineto{\pgfqpoint{1.192200in}{1.524214in}}%
\pgfpathlineto{\pgfqpoint{1.194680in}{1.526192in}}%
\pgfpathlineto{\pgfqpoint{1.197160in}{1.536220in}}%
\pgfpathlineto{\pgfqpoint{1.198400in}{1.537803in}}%
\pgfpathlineto{\pgfqpoint{1.200880in}{1.543002in}}%
\pgfpathlineto{\pgfqpoint{1.202120in}{1.540678in}}%
\pgfpathlineto{\pgfqpoint{1.204600in}{1.541254in}}%
\pgfpathlineto{\pgfqpoint{1.205840in}{1.536776in}}%
\pgfpathlineto{\pgfqpoint{1.207080in}{1.538382in}}%
\pgfpathlineto{\pgfqpoint{1.209560in}{1.536990in}}%
\pgfpathlineto{\pgfqpoint{1.214520in}{1.525407in}}%
\pgfpathlineto{\pgfqpoint{1.217000in}{1.522427in}}%
\pgfpathlineto{\pgfqpoint{1.219480in}{1.517510in}}%
\pgfpathlineto{\pgfqpoint{1.220720in}{1.517156in}}%
\pgfpathlineto{\pgfqpoint{1.221960in}{1.518996in}}%
\pgfpathlineto{\pgfqpoint{1.224440in}{1.519279in}}%
\pgfpathlineto{\pgfqpoint{1.225680in}{1.522023in}}%
\pgfpathlineto{\pgfqpoint{1.230640in}{1.518557in}}%
\pgfpathlineto{\pgfqpoint{1.231880in}{1.515070in}}%
\pgfpathlineto{\pgfqpoint{1.234360in}{1.515761in}}%
\pgfpathlineto{\pgfqpoint{1.235600in}{1.517191in}}%
\pgfpathlineto{\pgfqpoint{1.238080in}{1.515630in}}%
\pgfpathlineto{\pgfqpoint{1.239320in}{1.514867in}}%
\pgfpathlineto{\pgfqpoint{1.241800in}{1.509732in}}%
\pgfpathlineto{\pgfqpoint{1.244280in}{1.506802in}}%
\pgfpathlineto{\pgfqpoint{1.246760in}{1.510794in}}%
\pgfpathlineto{\pgfqpoint{1.248000in}{1.509366in}}%
\pgfpathlineto{\pgfqpoint{1.250480in}{1.514577in}}%
\pgfpathlineto{\pgfqpoint{1.251720in}{1.516427in}}%
\pgfpathlineto{\pgfqpoint{1.252960in}{1.515594in}}%
\pgfpathlineto{\pgfqpoint{1.255440in}{1.512800in}}%
\pgfpathlineto{\pgfqpoint{1.256680in}{1.513200in}}%
\pgfpathlineto{\pgfqpoint{1.259160in}{1.509780in}}%
\pgfpathlineto{\pgfqpoint{1.261640in}{1.512464in}}%
\pgfpathlineto{\pgfqpoint{1.265360in}{1.505327in}}%
\pgfpathlineto{\pgfqpoint{1.266600in}{1.504232in}}%
\pgfpathlineto{\pgfqpoint{1.270320in}{1.506606in}}%
\pgfpathlineto{\pgfqpoint{1.272800in}{1.509860in}}%
\pgfpathlineto{\pgfqpoint{1.274040in}{1.508872in}}%
\pgfpathlineto{\pgfqpoint{1.277760in}{1.514472in}}%
\pgfpathlineto{\pgfqpoint{1.279000in}{1.513592in}}%
\pgfpathlineto{\pgfqpoint{1.281480in}{1.507955in}}%
\pgfpathlineto{\pgfqpoint{1.283960in}{1.513913in}}%
\pgfpathlineto{\pgfqpoint{1.285200in}{1.514227in}}%
\pgfpathlineto{\pgfqpoint{1.287680in}{1.509993in}}%
\pgfpathlineto{\pgfqpoint{1.290160in}{1.506199in}}%
\pgfpathlineto{\pgfqpoint{1.293880in}{1.502726in}}%
\pgfpathlineto{\pgfqpoint{1.295120in}{1.504118in}}%
\pgfpathlineto{\pgfqpoint{1.296360in}{1.508019in}}%
\pgfpathlineto{\pgfqpoint{1.297600in}{1.506318in}}%
\pgfpathlineto{\pgfqpoint{1.298840in}{1.502682in}}%
\pgfpathlineto{\pgfqpoint{1.300080in}{1.506755in}}%
\pgfpathlineto{\pgfqpoint{1.302560in}{1.507359in}}%
\pgfpathlineto{\pgfqpoint{1.305040in}{1.504281in}}%
\pgfpathlineto{\pgfqpoint{1.306280in}{1.503790in}}%
\pgfpathlineto{\pgfqpoint{1.307520in}{1.505519in}}%
\pgfpathlineto{\pgfqpoint{1.310000in}{1.501808in}}%
\pgfpathlineto{\pgfqpoint{1.312480in}{1.504877in}}%
\pgfpathlineto{\pgfqpoint{1.317440in}{1.498498in}}%
\pgfpathlineto{\pgfqpoint{1.318680in}{1.499089in}}%
\pgfpathlineto{\pgfqpoint{1.323640in}{1.508291in}}%
\pgfpathlineto{\pgfqpoint{1.324880in}{1.509095in}}%
\pgfpathlineto{\pgfqpoint{1.326120in}{1.507850in}}%
\pgfpathlineto{\pgfqpoint{1.327360in}{1.509517in}}%
\pgfpathlineto{\pgfqpoint{1.328600in}{1.508163in}}%
\pgfpathlineto{\pgfqpoint{1.329840in}{1.503520in}}%
\pgfpathlineto{\pgfqpoint{1.331080in}{1.503776in}}%
\pgfpathlineto{\pgfqpoint{1.333560in}{1.500147in}}%
\pgfpathlineto{\pgfqpoint{1.337280in}{1.489643in}}%
\pgfpathlineto{\pgfqpoint{1.339760in}{1.491581in}}%
\pgfpathlineto{\pgfqpoint{1.341000in}{1.491367in}}%
\pgfpathlineto{\pgfqpoint{1.343480in}{1.489045in}}%
\pgfpathlineto{\pgfqpoint{1.350920in}{1.491818in}}%
\pgfpathlineto{\pgfqpoint{1.352160in}{1.491533in}}%
\pgfpathlineto{\pgfqpoint{1.355880in}{1.486560in}}%
\pgfpathlineto{\pgfqpoint{1.358360in}{1.488821in}}%
\pgfpathlineto{\pgfqpoint{1.360840in}{1.489672in}}%
\pgfpathlineto{\pgfqpoint{1.362080in}{1.488929in}}%
\pgfpathlineto{\pgfqpoint{1.363320in}{1.489792in}}%
\pgfpathlineto{\pgfqpoint{1.365800in}{1.487897in}}%
\pgfpathlineto{\pgfqpoint{1.368280in}{1.487427in}}%
\pgfpathlineto{\pgfqpoint{1.370760in}{1.491369in}}%
\pgfpathlineto{\pgfqpoint{1.372000in}{1.490927in}}%
\pgfpathlineto{\pgfqpoint{1.373240in}{1.495233in}}%
\pgfpathlineto{\pgfqpoint{1.374480in}{1.495054in}}%
\pgfpathlineto{\pgfqpoint{1.375720in}{1.496662in}}%
\pgfpathlineto{\pgfqpoint{1.379440in}{1.492642in}}%
\pgfpathlineto{\pgfqpoint{1.381920in}{1.493606in}}%
\pgfpathlineto{\pgfqpoint{1.383160in}{1.493973in}}%
\pgfpathlineto{\pgfqpoint{1.385640in}{1.496806in}}%
\pgfpathlineto{\pgfqpoint{1.386880in}{1.494244in}}%
\pgfpathlineto{\pgfqpoint{1.388120in}{1.494485in}}%
\pgfpathlineto{\pgfqpoint{1.391840in}{1.490872in}}%
\pgfpathlineto{\pgfqpoint{1.395560in}{1.494307in}}%
\pgfpathlineto{\pgfqpoint{1.396800in}{1.496428in}}%
\pgfpathlineto{\pgfqpoint{1.399280in}{1.497027in}}%
\pgfpathlineto{\pgfqpoint{1.400520in}{1.499409in}}%
\pgfpathlineto{\pgfqpoint{1.403000in}{1.497777in}}%
\pgfpathlineto{\pgfqpoint{1.405480in}{1.492612in}}%
\pgfpathlineto{\pgfqpoint{1.407960in}{1.496723in}}%
\pgfpathlineto{\pgfqpoint{1.409200in}{1.495747in}}%
\pgfpathlineto{\pgfqpoint{1.410440in}{1.496975in}}%
\pgfpathlineto{\pgfqpoint{1.419120in}{1.485599in}}%
\pgfpathlineto{\pgfqpoint{1.420360in}{1.488413in}}%
\pgfpathlineto{\pgfqpoint{1.422840in}{1.486428in}}%
\pgfpathlineto{\pgfqpoint{1.425320in}{1.490154in}}%
\pgfpathlineto{\pgfqpoint{1.426560in}{1.489535in}}%
\pgfpathlineto{\pgfqpoint{1.430280in}{1.485657in}}%
\pgfpathlineto{\pgfqpoint{1.431520in}{1.487168in}}%
\pgfpathlineto{\pgfqpoint{1.434000in}{1.481582in}}%
\pgfpathlineto{\pgfqpoint{1.436480in}{1.483361in}}%
\pgfpathlineto{\pgfqpoint{1.438960in}{1.479371in}}%
\pgfpathlineto{\pgfqpoint{1.440200in}{1.475305in}}%
\pgfpathlineto{\pgfqpoint{1.442680in}{1.476893in}}%
\pgfpathlineto{\pgfqpoint{1.445160in}{1.481113in}}%
\pgfpathlineto{\pgfqpoint{1.447640in}{1.485128in}}%
\pgfpathlineto{\pgfqpoint{1.450120in}{1.486958in}}%
\pgfpathlineto{\pgfqpoint{1.451360in}{1.487866in}}%
\pgfpathlineto{\pgfqpoint{1.452600in}{1.486377in}}%
\pgfpathlineto{\pgfqpoint{1.453840in}{1.481699in}}%
\pgfpathlineto{\pgfqpoint{1.456320in}{1.481157in}}%
\pgfpathlineto{\pgfqpoint{1.457560in}{1.481081in}}%
\pgfpathlineto{\pgfqpoint{1.461280in}{1.471759in}}%
\pgfpathlineto{\pgfqpoint{1.463760in}{1.472171in}}%
\pgfpathlineto{\pgfqpoint{1.466240in}{1.472358in}}%
\pgfpathlineto{\pgfqpoint{1.469960in}{1.475371in}}%
\pgfpathlineto{\pgfqpoint{1.473680in}{1.478603in}}%
\pgfpathlineto{\pgfqpoint{1.477400in}{1.473049in}}%
\pgfpathlineto{\pgfqpoint{1.479880in}{1.469839in}}%
\pgfpathlineto{\pgfqpoint{1.483600in}{1.472965in}}%
\pgfpathlineto{\pgfqpoint{1.486080in}{1.472222in}}%
\pgfpathlineto{\pgfqpoint{1.487320in}{1.472486in}}%
\pgfpathlineto{\pgfqpoint{1.488560in}{1.470134in}}%
\pgfpathlineto{\pgfqpoint{1.493520in}{1.471381in}}%
\pgfpathlineto{\pgfqpoint{1.494760in}{1.473524in}}%
\pgfpathlineto{\pgfqpoint{1.496000in}{1.472494in}}%
\pgfpathlineto{\pgfqpoint{1.497240in}{1.475296in}}%
\pgfpathlineto{\pgfqpoint{1.498480in}{1.474638in}}%
\pgfpathlineto{\pgfqpoint{1.499720in}{1.476613in}}%
\pgfpathlineto{\pgfqpoint{1.503440in}{1.473962in}}%
\pgfpathlineto{\pgfqpoint{1.504680in}{1.474153in}}%
\pgfpathlineto{\pgfqpoint{1.507160in}{1.472214in}}%
\pgfpathlineto{\pgfqpoint{1.509640in}{1.475296in}}%
\pgfpathlineto{\pgfqpoint{1.510880in}{1.473302in}}%
\pgfpathlineto{\pgfqpoint{1.513360in}{1.473324in}}%
\pgfpathlineto{\pgfqpoint{1.514600in}{1.471387in}}%
\pgfpathlineto{\pgfqpoint{1.519560in}{1.473534in}}%
\pgfpathlineto{\pgfqpoint{1.524520in}{1.478995in}}%
\pgfpathlineto{\pgfqpoint{1.527000in}{1.476984in}}%
\pgfpathlineto{\pgfqpoint{1.528240in}{1.472692in}}%
\pgfpathlineto{\pgfqpoint{1.530720in}{1.473736in}}%
\pgfpathlineto{\pgfqpoint{1.531960in}{1.474598in}}%
\pgfpathlineto{\pgfqpoint{1.536920in}{1.468625in}}%
\pgfpathlineto{\pgfqpoint{1.541880in}{1.463887in}}%
\pgfpathlineto{\pgfqpoint{1.545600in}{1.467525in}}%
\pgfpathlineto{\pgfqpoint{1.546840in}{1.466567in}}%
\pgfpathlineto{\pgfqpoint{1.549320in}{1.471545in}}%
\pgfpathlineto{\pgfqpoint{1.550560in}{1.471074in}}%
\pgfpathlineto{\pgfqpoint{1.554280in}{1.466636in}}%
\pgfpathlineto{\pgfqpoint{1.555520in}{1.469097in}}%
\pgfpathlineto{\pgfqpoint{1.558000in}{1.465460in}}%
\pgfpathlineto{\pgfqpoint{1.561720in}{1.468947in}}%
\pgfpathlineto{\pgfqpoint{1.565440in}{1.465156in}}%
\pgfpathlineto{\pgfqpoint{1.569160in}{1.470565in}}%
\pgfpathlineto{\pgfqpoint{1.572880in}{1.475320in}}%
\pgfpathlineto{\pgfqpoint{1.574120in}{1.474704in}}%
\pgfpathlineto{\pgfqpoint{1.575360in}{1.476012in}}%
\pgfpathlineto{\pgfqpoint{1.576600in}{1.474773in}}%
\pgfpathlineto{\pgfqpoint{1.579080in}{1.469956in}}%
\pgfpathlineto{\pgfqpoint{1.584040in}{1.464870in}}%
\pgfpathlineto{\pgfqpoint{1.586520in}{1.462787in}}%
\pgfpathlineto{\pgfqpoint{1.587760in}{1.462255in}}%
\pgfpathlineto{\pgfqpoint{1.593960in}{1.466536in}}%
\pgfpathlineto{\pgfqpoint{1.597680in}{1.467741in}}%
\pgfpathlineto{\pgfqpoint{1.598920in}{1.466178in}}%
\pgfpathlineto{\pgfqpoint{1.600160in}{1.466422in}}%
\pgfpathlineto{\pgfqpoint{1.603880in}{1.461823in}}%
\pgfpathlineto{\pgfqpoint{1.606360in}{1.463532in}}%
\pgfpathlineto{\pgfqpoint{1.607600in}{1.463841in}}%
\pgfpathlineto{\pgfqpoint{1.610080in}{1.462391in}}%
\pgfpathlineto{\pgfqpoint{1.611320in}{1.462448in}}%
\pgfpathlineto{\pgfqpoint{1.615040in}{1.458751in}}%
\pgfpathlineto{\pgfqpoint{1.617520in}{1.459088in}}%
\pgfpathlineto{\pgfqpoint{1.618760in}{1.461936in}}%
\pgfpathlineto{\pgfqpoint{1.620000in}{1.461849in}}%
\pgfpathlineto{\pgfqpoint{1.621240in}{1.463418in}}%
\pgfpathlineto{\pgfqpoint{1.622480in}{1.462873in}}%
\pgfpathlineto{\pgfqpoint{1.623720in}{1.465258in}}%
\pgfpathlineto{\pgfqpoint{1.626200in}{1.465072in}}%
\pgfpathlineto{\pgfqpoint{1.628680in}{1.465214in}}%
\pgfpathlineto{\pgfqpoint{1.631160in}{1.463827in}}%
\pgfpathlineto{\pgfqpoint{1.632400in}{1.466299in}}%
\pgfpathlineto{\pgfqpoint{1.634880in}{1.465422in}}%
\pgfpathlineto{\pgfqpoint{1.636120in}{1.466361in}}%
\pgfpathlineto{\pgfqpoint{1.637360in}{1.464896in}}%
\pgfpathlineto{\pgfqpoint{1.638600in}{1.461220in}}%
\pgfpathlineto{\pgfqpoint{1.642320in}{1.461692in}}%
\pgfpathlineto{\pgfqpoint{1.647280in}{1.466186in}}%
\pgfpathlineto{\pgfqpoint{1.648520in}{1.468136in}}%
\pgfpathlineto{\pgfqpoint{1.651000in}{1.466148in}}%
\pgfpathlineto{\pgfqpoint{1.652240in}{1.462291in}}%
\pgfpathlineto{\pgfqpoint{1.654720in}{1.463263in}}%
\pgfpathlineto{\pgfqpoint{1.655960in}{1.463903in}}%
\pgfpathlineto{\pgfqpoint{1.665880in}{1.455902in}}%
\pgfpathlineto{\pgfqpoint{1.668360in}{1.459899in}}%
\pgfpathlineto{\pgfqpoint{1.670840in}{1.456409in}}%
\pgfpathlineto{\pgfqpoint{1.673320in}{1.461539in}}%
\pgfpathlineto{\pgfqpoint{1.674560in}{1.461517in}}%
\pgfpathlineto{\pgfqpoint{1.678280in}{1.459156in}}%
\pgfpathlineto{\pgfqpoint{1.679520in}{1.460727in}}%
\pgfpathlineto{\pgfqpoint{1.682000in}{1.459611in}}%
\pgfpathlineto{\pgfqpoint{1.684480in}{1.463819in}}%
\pgfpathlineto{\pgfqpoint{1.689440in}{1.459592in}}%
\pgfpathlineto{\pgfqpoint{1.699360in}{1.468180in}}%
\pgfpathlineto{\pgfqpoint{1.705560in}{1.461306in}}%
\pgfpathlineto{\pgfqpoint{1.706800in}{1.460121in}}%
\pgfpathlineto{\pgfqpoint{1.709280in}{1.455804in}}%
\pgfpathlineto{\pgfqpoint{1.711760in}{1.454625in}}%
\pgfpathlineto{\pgfqpoint{1.715480in}{1.458174in}}%
\pgfpathlineto{\pgfqpoint{1.716720in}{1.457835in}}%
\pgfpathlineto{\pgfqpoint{1.721680in}{1.461411in}}%
\pgfpathlineto{\pgfqpoint{1.724160in}{1.459239in}}%
\pgfpathlineto{\pgfqpoint{1.729120in}{1.456067in}}%
\pgfpathlineto{\pgfqpoint{1.732840in}{1.454787in}}%
\pgfpathlineto{\pgfqpoint{1.735320in}{1.455800in}}%
\pgfpathlineto{\pgfqpoint{1.736560in}{1.453537in}}%
\pgfpathlineto{\pgfqpoint{1.737800in}{1.453704in}}%
\pgfpathlineto{\pgfqpoint{1.739040in}{1.452652in}}%
\pgfpathlineto{\pgfqpoint{1.741520in}{1.454601in}}%
\pgfpathlineto{\pgfqpoint{1.742760in}{1.457431in}}%
\pgfpathlineto{\pgfqpoint{1.745240in}{1.456092in}}%
\pgfpathlineto{\pgfqpoint{1.746480in}{1.455719in}}%
\pgfpathlineto{\pgfqpoint{1.747720in}{1.457789in}}%
\pgfpathlineto{\pgfqpoint{1.751440in}{1.456652in}}%
\pgfpathlineto{\pgfqpoint{1.752680in}{1.456674in}}%
\pgfpathlineto{\pgfqpoint{1.755160in}{1.455253in}}%
\pgfpathlineto{\pgfqpoint{1.756400in}{1.456961in}}%
\pgfpathlineto{\pgfqpoint{1.758880in}{1.455851in}}%
\pgfpathlineto{\pgfqpoint{1.760120in}{1.455992in}}%
\pgfpathlineto{\pgfqpoint{1.761360in}{1.454647in}}%
\pgfpathlineto{\pgfqpoint{1.763840in}{1.450354in}}%
\pgfpathlineto{\pgfqpoint{1.767560in}{1.452191in}}%
\pgfpathlineto{\pgfqpoint{1.772520in}{1.455002in}}%
\pgfpathlineto{\pgfqpoint{1.777480in}{1.449011in}}%
\pgfpathlineto{\pgfqpoint{1.781200in}{1.451267in}}%
\pgfpathlineto{\pgfqpoint{1.783680in}{1.449095in}}%
\pgfpathlineto{\pgfqpoint{1.788640in}{1.445884in}}%
\pgfpathlineto{\pgfqpoint{1.789880in}{1.446199in}}%
\pgfpathlineto{\pgfqpoint{1.792360in}{1.450395in}}%
\pgfpathlineto{\pgfqpoint{1.794840in}{1.446493in}}%
\pgfpathlineto{\pgfqpoint{1.797320in}{1.450291in}}%
\pgfpathlineto{\pgfqpoint{1.802280in}{1.449500in}}%
\pgfpathlineto{\pgfqpoint{1.803520in}{1.451244in}}%
\pgfpathlineto{\pgfqpoint{1.806000in}{1.449850in}}%
\pgfpathlineto{\pgfqpoint{1.808480in}{1.452997in}}%
\pgfpathlineto{\pgfqpoint{1.814680in}{1.448044in}}%
\pgfpathlineto{\pgfqpoint{1.819640in}{1.452664in}}%
\pgfpathlineto{\pgfqpoint{1.820880in}{1.454162in}}%
\pgfpathlineto{\pgfqpoint{1.822120in}{1.453889in}}%
\pgfpathlineto{\pgfqpoint{1.823360in}{1.454929in}}%
\pgfpathlineto{\pgfqpoint{1.824600in}{1.453860in}}%
\pgfpathlineto{\pgfqpoint{1.827080in}{1.450037in}}%
\pgfpathlineto{\pgfqpoint{1.830800in}{1.446984in}}%
\pgfpathlineto{\pgfqpoint{1.834520in}{1.441894in}}%
\pgfpathlineto{\pgfqpoint{1.838240in}{1.445808in}}%
\pgfpathlineto{\pgfqpoint{1.839480in}{1.447207in}}%
\pgfpathlineto{\pgfqpoint{1.841960in}{1.446495in}}%
\pgfpathlineto{\pgfqpoint{1.845680in}{1.448879in}}%
\pgfpathlineto{\pgfqpoint{1.850640in}{1.442880in}}%
\pgfpathlineto{\pgfqpoint{1.851880in}{1.442014in}}%
\pgfpathlineto{\pgfqpoint{1.854360in}{1.444112in}}%
\pgfpathlineto{\pgfqpoint{1.859320in}{1.444920in}}%
\pgfpathlineto{\pgfqpoint{1.861800in}{1.442470in}}%
\pgfpathlineto{\pgfqpoint{1.863040in}{1.441258in}}%
\pgfpathlineto{\pgfqpoint{1.865520in}{1.442871in}}%
\pgfpathlineto{\pgfqpoint{1.866760in}{1.445343in}}%
\pgfpathlineto{\pgfqpoint{1.868000in}{1.444504in}}%
\pgfpathlineto{\pgfqpoint{1.869240in}{1.447676in}}%
\pgfpathlineto{\pgfqpoint{1.870480in}{1.447235in}}%
\pgfpathlineto{\pgfqpoint{1.871720in}{1.448964in}}%
\pgfpathlineto{\pgfqpoint{1.879160in}{1.447212in}}%
\pgfpathlineto{\pgfqpoint{1.880400in}{1.448953in}}%
\pgfpathlineto{\pgfqpoint{1.884120in}{1.447470in}}%
\pgfpathlineto{\pgfqpoint{1.885360in}{1.446354in}}%
\pgfpathlineto{\pgfqpoint{1.887840in}{1.442655in}}%
\pgfpathlineto{\pgfqpoint{1.894040in}{1.443837in}}%
\pgfpathlineto{\pgfqpoint{1.896520in}{1.446964in}}%
\pgfpathlineto{\pgfqpoint{1.899000in}{1.444782in}}%
\pgfpathlineto{\pgfqpoint{1.901480in}{1.441057in}}%
\pgfpathlineto{\pgfqpoint{1.905200in}{1.441616in}}%
\pgfpathlineto{\pgfqpoint{1.907680in}{1.439682in}}%
\pgfpathlineto{\pgfqpoint{1.910160in}{1.436445in}}%
\pgfpathlineto{\pgfqpoint{1.911400in}{1.436617in}}%
\pgfpathlineto{\pgfqpoint{1.913880in}{1.434865in}}%
\pgfpathlineto{\pgfqpoint{1.916360in}{1.436756in}}%
\pgfpathlineto{\pgfqpoint{1.918840in}{1.432283in}}%
\pgfpathlineto{\pgfqpoint{1.921320in}{1.436488in}}%
\pgfpathlineto{\pgfqpoint{1.926280in}{1.435488in}}%
\pgfpathlineto{\pgfqpoint{1.928760in}{1.436577in}}%
\pgfpathlineto{\pgfqpoint{1.930000in}{1.435515in}}%
\pgfpathlineto{\pgfqpoint{1.932480in}{1.439441in}}%
\pgfpathlineto{\pgfqpoint{1.938680in}{1.433532in}}%
\pgfpathlineto{\pgfqpoint{1.947360in}{1.441183in}}%
\pgfpathlineto{\pgfqpoint{1.953560in}{1.436165in}}%
\pgfpathlineto{\pgfqpoint{1.957280in}{1.431057in}}%
\pgfpathlineto{\pgfqpoint{1.958520in}{1.430558in}}%
\pgfpathlineto{\pgfqpoint{1.969680in}{1.439234in}}%
\pgfpathlineto{\pgfqpoint{1.974640in}{1.430893in}}%
\pgfpathlineto{\pgfqpoint{1.975880in}{1.430060in}}%
\pgfpathlineto{\pgfqpoint{1.978360in}{1.433329in}}%
\pgfpathlineto{\pgfqpoint{1.979600in}{1.434058in}}%
\pgfpathlineto{\pgfqpoint{1.982080in}{1.432957in}}%
\pgfpathlineto{\pgfqpoint{1.983320in}{1.433148in}}%
\pgfpathlineto{\pgfqpoint{1.987040in}{1.429991in}}%
\pgfpathlineto{\pgfqpoint{1.989520in}{1.430819in}}%
\pgfpathlineto{\pgfqpoint{1.990760in}{1.432996in}}%
\pgfpathlineto{\pgfqpoint{1.992000in}{1.431383in}}%
\pgfpathlineto{\pgfqpoint{1.993240in}{1.437206in}}%
\pgfpathlineto{\pgfqpoint{1.994480in}{1.436975in}}%
\pgfpathlineto{\pgfqpoint{1.995720in}{1.439328in}}%
\pgfpathlineto{\pgfqpoint{2.000680in}{1.437090in}}%
\pgfpathlineto{\pgfqpoint{2.003160in}{1.437017in}}%
\pgfpathlineto{\pgfqpoint{2.005640in}{1.438339in}}%
\pgfpathlineto{\pgfqpoint{2.011840in}{1.433869in}}%
\pgfpathlineto{\pgfqpoint{2.014320in}{1.434578in}}%
\pgfpathlineto{\pgfqpoint{2.015560in}{1.435356in}}%
\pgfpathlineto{\pgfqpoint{2.018040in}{1.433925in}}%
\pgfpathlineto{\pgfqpoint{2.020520in}{1.437427in}}%
\pgfpathlineto{\pgfqpoint{2.024240in}{1.431419in}}%
\pgfpathlineto{\pgfqpoint{2.029200in}{1.432867in}}%
\pgfpathlineto{\pgfqpoint{2.031680in}{1.430750in}}%
\pgfpathlineto{\pgfqpoint{2.034160in}{1.427735in}}%
\pgfpathlineto{\pgfqpoint{2.035400in}{1.428581in}}%
\pgfpathlineto{\pgfqpoint{2.037880in}{1.426970in}}%
\pgfpathlineto{\pgfqpoint{2.040360in}{1.428469in}}%
\pgfpathlineto{\pgfqpoint{2.042840in}{1.424141in}}%
\pgfpathlineto{\pgfqpoint{2.045320in}{1.427670in}}%
\pgfpathlineto{\pgfqpoint{2.047800in}{1.427058in}}%
\pgfpathlineto{\pgfqpoint{2.052760in}{1.428555in}}%
\pgfpathlineto{\pgfqpoint{2.054000in}{1.426688in}}%
\pgfpathlineto{\pgfqpoint{2.056480in}{1.429358in}}%
\pgfpathlineto{\pgfqpoint{2.062680in}{1.425239in}}%
\pgfpathlineto{\pgfqpoint{2.071360in}{1.431003in}}%
\pgfpathlineto{\pgfqpoint{2.072600in}{1.430760in}}%
\pgfpathlineto{\pgfqpoint{2.078800in}{1.422356in}}%
\pgfpathlineto{\pgfqpoint{2.083760in}{1.420094in}}%
\pgfpathlineto{\pgfqpoint{2.089960in}{1.423906in}}%
\pgfpathlineto{\pgfqpoint{2.092440in}{1.425399in}}%
\pgfpathlineto{\pgfqpoint{2.093680in}{1.427072in}}%
\pgfpathlineto{\pgfqpoint{2.098640in}{1.417832in}}%
\pgfpathlineto{\pgfqpoint{2.099880in}{1.417131in}}%
\pgfpathlineto{\pgfqpoint{2.103600in}{1.420851in}}%
\pgfpathlineto{\pgfqpoint{2.106080in}{1.419253in}}%
\pgfpathlineto{\pgfqpoint{2.107320in}{1.419089in}}%
\pgfpathlineto{\pgfqpoint{2.109800in}{1.417522in}}%
\pgfpathlineto{\pgfqpoint{2.112280in}{1.417142in}}%
\pgfpathlineto{\pgfqpoint{2.113520in}{1.416912in}}%
\pgfpathlineto{\pgfqpoint{2.114760in}{1.418840in}}%
\pgfpathlineto{\pgfqpoint{2.116000in}{1.417180in}}%
\pgfpathlineto{\pgfqpoint{2.117240in}{1.422925in}}%
\pgfpathlineto{\pgfqpoint{2.118480in}{1.423023in}}%
\pgfpathlineto{\pgfqpoint{2.119720in}{1.425705in}}%
\pgfpathlineto{\pgfqpoint{2.127160in}{1.423681in}}%
\pgfpathlineto{\pgfqpoint{2.129640in}{1.425103in}}%
\pgfpathlineto{\pgfqpoint{2.132120in}{1.423578in}}%
\pgfpathlineto{\pgfqpoint{2.133360in}{1.423299in}}%
\pgfpathlineto{\pgfqpoint{2.135840in}{1.420628in}}%
\pgfpathlineto{\pgfqpoint{2.138320in}{1.421616in}}%
\pgfpathlineto{\pgfqpoint{2.139560in}{1.423102in}}%
\pgfpathlineto{\pgfqpoint{2.142040in}{1.421785in}}%
\pgfpathlineto{\pgfqpoint{2.144520in}{1.424609in}}%
\pgfpathlineto{\pgfqpoint{2.148240in}{1.419022in}}%
\pgfpathlineto{\pgfqpoint{2.151960in}{1.421743in}}%
\pgfpathlineto{\pgfqpoint{2.155680in}{1.419606in}}%
\pgfpathlineto{\pgfqpoint{2.158160in}{1.416760in}}%
\pgfpathlineto{\pgfqpoint{2.159400in}{1.417327in}}%
\pgfpathlineto{\pgfqpoint{2.161880in}{1.416344in}}%
\pgfpathlineto{\pgfqpoint{2.164360in}{1.417960in}}%
\pgfpathlineto{\pgfqpoint{2.166840in}{1.413707in}}%
\pgfpathlineto{\pgfqpoint{2.169320in}{1.417598in}}%
\pgfpathlineto{\pgfqpoint{2.170560in}{1.416775in}}%
\pgfpathlineto{\pgfqpoint{2.174280in}{1.418586in}}%
\pgfpathlineto{\pgfqpoint{2.175520in}{1.420314in}}%
\pgfpathlineto{\pgfqpoint{2.176760in}{1.419711in}}%
\pgfpathlineto{\pgfqpoint{2.178000in}{1.417744in}}%
\pgfpathlineto{\pgfqpoint{2.180480in}{1.421089in}}%
\pgfpathlineto{\pgfqpoint{2.185440in}{1.417302in}}%
\pgfpathlineto{\pgfqpoint{2.189160in}{1.418256in}}%
\pgfpathlineto{\pgfqpoint{2.194120in}{1.421573in}}%
\pgfpathlineto{\pgfqpoint{2.196600in}{1.422135in}}%
\pgfpathlineto{\pgfqpoint{2.200320in}{1.416777in}}%
\pgfpathlineto{\pgfqpoint{2.201560in}{1.415901in}}%
\pgfpathlineto{\pgfqpoint{2.204040in}{1.412071in}}%
\pgfpathlineto{\pgfqpoint{2.206520in}{1.409568in}}%
\pgfpathlineto{\pgfqpoint{2.211480in}{1.414836in}}%
\pgfpathlineto{\pgfqpoint{2.213960in}{1.414271in}}%
\pgfpathlineto{\pgfqpoint{2.217680in}{1.417008in}}%
\pgfpathlineto{\pgfqpoint{2.222640in}{1.408293in}}%
\pgfpathlineto{\pgfqpoint{2.223880in}{1.408146in}}%
\pgfpathlineto{\pgfqpoint{2.226360in}{1.411867in}}%
\pgfpathlineto{\pgfqpoint{2.227600in}{1.412795in}}%
\pgfpathlineto{\pgfqpoint{2.230080in}{1.410914in}}%
\pgfpathlineto{\pgfqpoint{2.231320in}{1.410681in}}%
\pgfpathlineto{\pgfqpoint{2.235040in}{1.407570in}}%
\pgfpathlineto{\pgfqpoint{2.237520in}{1.406614in}}%
\pgfpathlineto{\pgfqpoint{2.238760in}{1.408587in}}%
\pgfpathlineto{\pgfqpoint{2.240000in}{1.407176in}}%
\pgfpathlineto{\pgfqpoint{2.241240in}{1.413517in}}%
\pgfpathlineto{\pgfqpoint{2.242480in}{1.413853in}}%
\pgfpathlineto{\pgfqpoint{2.244960in}{1.416878in}}%
\pgfpathlineto{\pgfqpoint{2.249920in}{1.413148in}}%
\pgfpathlineto{\pgfqpoint{2.251160in}{1.413533in}}%
\pgfpathlineto{\pgfqpoint{2.252400in}{1.415345in}}%
\pgfpathlineto{\pgfqpoint{2.253640in}{1.415122in}}%
\pgfpathlineto{\pgfqpoint{2.254880in}{1.413609in}}%
\pgfpathlineto{\pgfqpoint{2.257360in}{1.413914in}}%
\pgfpathlineto{\pgfqpoint{2.259840in}{1.411126in}}%
\pgfpathlineto{\pgfqpoint{2.261080in}{1.410449in}}%
\pgfpathlineto{\pgfqpoint{2.264800in}{1.411590in}}%
\pgfpathlineto{\pgfqpoint{2.266040in}{1.411211in}}%
\pgfpathlineto{\pgfqpoint{2.268520in}{1.413789in}}%
\pgfpathlineto{\pgfqpoint{2.272240in}{1.408536in}}%
\pgfpathlineto{\pgfqpoint{2.277200in}{1.410949in}}%
\pgfpathlineto{\pgfqpoint{2.279680in}{1.410129in}}%
\pgfpathlineto{\pgfqpoint{2.282160in}{1.408189in}}%
\pgfpathlineto{\pgfqpoint{2.283400in}{1.408209in}}%
\pgfpathlineto{\pgfqpoint{2.284640in}{1.406827in}}%
\pgfpathlineto{\pgfqpoint{2.288360in}{1.408017in}}%
\pgfpathlineto{\pgfqpoint{2.290840in}{1.403165in}}%
\pgfpathlineto{\pgfqpoint{2.293320in}{1.405789in}}%
\pgfpathlineto{\pgfqpoint{2.294560in}{1.404967in}}%
\pgfpathlineto{\pgfqpoint{2.300760in}{1.408233in}}%
\pgfpathlineto{\pgfqpoint{2.302000in}{1.406000in}}%
\pgfpathlineto{\pgfqpoint{2.304480in}{1.409818in}}%
\pgfpathlineto{\pgfqpoint{2.309440in}{1.406452in}}%
\pgfpathlineto{\pgfqpoint{2.313160in}{1.407592in}}%
\pgfpathlineto{\pgfqpoint{2.319360in}{1.412729in}}%
\pgfpathlineto{\pgfqpoint{2.320600in}{1.412228in}}%
\pgfpathlineto{\pgfqpoint{2.324320in}{1.406768in}}%
\pgfpathlineto{\pgfqpoint{2.325560in}{1.406235in}}%
\pgfpathlineto{\pgfqpoint{2.328040in}{1.402656in}}%
\pgfpathlineto{\pgfqpoint{2.330520in}{1.400703in}}%
\pgfpathlineto{\pgfqpoint{2.335480in}{1.405476in}}%
\pgfpathlineto{\pgfqpoint{2.336720in}{1.404353in}}%
\pgfpathlineto{\pgfqpoint{2.339200in}{1.405953in}}%
\pgfpathlineto{\pgfqpoint{2.341680in}{1.408123in}}%
\pgfpathlineto{\pgfqpoint{2.346640in}{1.401542in}}%
\pgfpathlineto{\pgfqpoint{2.347880in}{1.400985in}}%
\pgfpathlineto{\pgfqpoint{2.351600in}{1.406046in}}%
\pgfpathlineto{\pgfqpoint{2.359040in}{1.400871in}}%
\pgfpathlineto{\pgfqpoint{2.364000in}{1.400453in}}%
\pgfpathlineto{\pgfqpoint{2.365240in}{1.405800in}}%
\pgfpathlineto{\pgfqpoint{2.366480in}{1.405962in}}%
\pgfpathlineto{\pgfqpoint{2.367720in}{1.408398in}}%
\pgfpathlineto{\pgfqpoint{2.372680in}{1.406569in}}%
\pgfpathlineto{\pgfqpoint{2.375160in}{1.406833in}}%
\pgfpathlineto{\pgfqpoint{2.377640in}{1.408385in}}%
\pgfpathlineto{\pgfqpoint{2.378880in}{1.406882in}}%
\pgfpathlineto{\pgfqpoint{2.381360in}{1.406968in}}%
\pgfpathlineto{\pgfqpoint{2.383840in}{1.403856in}}%
\pgfpathlineto{\pgfqpoint{2.385080in}{1.402959in}}%
\pgfpathlineto{\pgfqpoint{2.388800in}{1.404148in}}%
\pgfpathlineto{\pgfqpoint{2.390040in}{1.403735in}}%
\pgfpathlineto{\pgfqpoint{2.392520in}{1.405478in}}%
\pgfpathlineto{\pgfqpoint{2.397480in}{1.400858in}}%
\pgfpathlineto{\pgfqpoint{2.399960in}{1.403966in}}%
\pgfpathlineto{\pgfqpoint{2.403680in}{1.402932in}}%
\pgfpathlineto{\pgfqpoint{2.406160in}{1.401431in}}%
\pgfpathlineto{\pgfqpoint{2.407400in}{1.401656in}}%
\pgfpathlineto{\pgfqpoint{2.408640in}{1.400493in}}%
\pgfpathlineto{\pgfqpoint{2.412360in}{1.402406in}}%
\pgfpathlineto{\pgfqpoint{2.414840in}{1.397253in}}%
\pgfpathlineto{\pgfqpoint{2.423520in}{1.403279in}}%
\pgfpathlineto{\pgfqpoint{2.426000in}{1.400537in}}%
\pgfpathlineto{\pgfqpoint{2.428480in}{1.403952in}}%
\pgfpathlineto{\pgfqpoint{2.434680in}{1.402140in}}%
\pgfpathlineto{\pgfqpoint{2.437160in}{1.403689in}}%
\pgfpathlineto{\pgfqpoint{2.443360in}{1.408197in}}%
\pgfpathlineto{\pgfqpoint{2.445840in}{1.405421in}}%
\pgfpathlineto{\pgfqpoint{2.448320in}{1.402637in}}%
\pgfpathlineto{\pgfqpoint{2.449560in}{1.402313in}}%
\pgfpathlineto{\pgfqpoint{2.452040in}{1.399400in}}%
\pgfpathlineto{\pgfqpoint{2.454520in}{1.397964in}}%
\pgfpathlineto{\pgfqpoint{2.459480in}{1.401866in}}%
\pgfpathlineto{\pgfqpoint{2.460720in}{1.400656in}}%
\pgfpathlineto{\pgfqpoint{2.465680in}{1.404619in}}%
\pgfpathlineto{\pgfqpoint{2.468160in}{1.402409in}}%
\pgfpathlineto{\pgfqpoint{2.471880in}{1.398028in}}%
\pgfpathlineto{\pgfqpoint{2.475600in}{1.402889in}}%
\pgfpathlineto{\pgfqpoint{2.480560in}{1.400343in}}%
\pgfpathlineto{\pgfqpoint{2.488000in}{1.399896in}}%
\pgfpathlineto{\pgfqpoint{2.489240in}{1.404212in}}%
\pgfpathlineto{\pgfqpoint{2.490480in}{1.404163in}}%
\pgfpathlineto{\pgfqpoint{2.491720in}{1.405942in}}%
\pgfpathlineto{\pgfqpoint{2.499160in}{1.404209in}}%
\pgfpathlineto{\pgfqpoint{2.501640in}{1.405389in}}%
\pgfpathlineto{\pgfqpoint{2.502880in}{1.404646in}}%
\pgfpathlineto{\pgfqpoint{2.505360in}{1.405394in}}%
\pgfpathlineto{\pgfqpoint{2.509080in}{1.400935in}}%
\pgfpathlineto{\pgfqpoint{2.516520in}{1.403436in}}%
\pgfpathlineto{\pgfqpoint{2.520240in}{1.397856in}}%
\pgfpathlineto{\pgfqpoint{2.521480in}{1.398379in}}%
\pgfpathlineto{\pgfqpoint{2.523960in}{1.401306in}}%
\pgfpathlineto{\pgfqpoint{2.527680in}{1.400404in}}%
\pgfpathlineto{\pgfqpoint{2.530160in}{1.398601in}}%
\pgfpathlineto{\pgfqpoint{2.533880in}{1.399017in}}%
\pgfpathlineto{\pgfqpoint{2.536360in}{1.400814in}}%
\pgfpathlineto{\pgfqpoint{2.538840in}{1.395611in}}%
\pgfpathlineto{\pgfqpoint{2.545040in}{1.397915in}}%
\pgfpathlineto{\pgfqpoint{2.546280in}{1.397374in}}%
\pgfpathlineto{\pgfqpoint{2.547520in}{1.398970in}}%
\pgfpathlineto{\pgfqpoint{2.548760in}{1.398395in}}%
\pgfpathlineto{\pgfqpoint{2.550000in}{1.396304in}}%
\pgfpathlineto{\pgfqpoint{2.552480in}{1.399843in}}%
\pgfpathlineto{\pgfqpoint{2.558680in}{1.397977in}}%
\pgfpathlineto{\pgfqpoint{2.561160in}{1.399969in}}%
\pgfpathlineto{\pgfqpoint{2.566120in}{1.403590in}}%
\pgfpathlineto{\pgfqpoint{2.568600in}{1.402326in}}%
\pgfpathlineto{\pgfqpoint{2.576040in}{1.393777in}}%
\pgfpathlineto{\pgfqpoint{2.578520in}{1.392430in}}%
\pgfpathlineto{\pgfqpoint{2.584720in}{1.395549in}}%
\pgfpathlineto{\pgfqpoint{2.588440in}{1.397181in}}%
\pgfpathlineto{\pgfqpoint{2.589680in}{1.398152in}}%
\pgfpathlineto{\pgfqpoint{2.595880in}{1.391614in}}%
\pgfpathlineto{\pgfqpoint{2.599600in}{1.396829in}}%
\pgfpathlineto{\pgfqpoint{2.602080in}{1.396059in}}%
\pgfpathlineto{\pgfqpoint{2.612000in}{1.394312in}}%
\pgfpathlineto{\pgfqpoint{2.613240in}{1.396833in}}%
\pgfpathlineto{\pgfqpoint{2.614480in}{1.396611in}}%
\pgfpathlineto{\pgfqpoint{2.616960in}{1.398910in}}%
\pgfpathlineto{\pgfqpoint{2.619440in}{1.398948in}}%
\pgfpathlineto{\pgfqpoint{2.626880in}{1.399731in}}%
\pgfpathlineto{\pgfqpoint{2.629360in}{1.400588in}}%
\pgfpathlineto{\pgfqpoint{2.633080in}{1.395014in}}%
\pgfpathlineto{\pgfqpoint{2.636800in}{1.395805in}}%
\pgfpathlineto{\pgfqpoint{2.641760in}{1.394086in}}%
\pgfpathlineto{\pgfqpoint{2.644240in}{1.390838in}}%
\pgfpathlineto{\pgfqpoint{2.650440in}{1.395335in}}%
\pgfpathlineto{\pgfqpoint{2.652920in}{1.393813in}}%
\pgfpathlineto{\pgfqpoint{2.656640in}{1.391609in}}%
\pgfpathlineto{\pgfqpoint{2.660360in}{1.393853in}}%
\pgfpathlineto{\pgfqpoint{2.662840in}{1.388754in}}%
\pgfpathlineto{\pgfqpoint{2.666560in}{1.389545in}}%
\pgfpathlineto{\pgfqpoint{2.667800in}{1.389790in}}%
\pgfpathlineto{\pgfqpoint{2.669040in}{1.391182in}}%
\pgfpathlineto{\pgfqpoint{2.670280in}{1.390594in}}%
\pgfpathlineto{\pgfqpoint{2.671520in}{1.392676in}}%
\pgfpathlineto{\pgfqpoint{2.672760in}{1.392495in}}%
\pgfpathlineto{\pgfqpoint{2.674000in}{1.390870in}}%
\pgfpathlineto{\pgfqpoint{2.676480in}{1.394853in}}%
\pgfpathlineto{\pgfqpoint{2.682680in}{1.393113in}}%
\pgfpathlineto{\pgfqpoint{2.685160in}{1.395355in}}%
\pgfpathlineto{\pgfqpoint{2.690120in}{1.399203in}}%
\pgfpathlineto{\pgfqpoint{2.691360in}{1.399460in}}%
\pgfpathlineto{\pgfqpoint{2.695080in}{1.395147in}}%
\pgfpathlineto{\pgfqpoint{2.698800in}{1.390459in}}%
\pgfpathlineto{\pgfqpoint{2.703760in}{1.388584in}}%
\pgfpathlineto{\pgfqpoint{2.708720in}{1.392147in}}%
\pgfpathlineto{\pgfqpoint{2.714920in}{1.393028in}}%
\pgfpathlineto{\pgfqpoint{2.717400in}{1.392361in}}%
\pgfpathlineto{\pgfqpoint{2.719880in}{1.388338in}}%
\pgfpathlineto{\pgfqpoint{2.723600in}{1.393364in}}%
\pgfpathlineto{\pgfqpoint{2.726080in}{1.392952in}}%
\pgfpathlineto{\pgfqpoint{2.729800in}{1.391750in}}%
\pgfpathlineto{\pgfqpoint{2.732280in}{1.391461in}}%
\pgfpathlineto{\pgfqpoint{2.733520in}{1.391510in}}%
\pgfpathlineto{\pgfqpoint{2.734760in}{1.392945in}}%
\pgfpathlineto{\pgfqpoint{2.736000in}{1.391067in}}%
\pgfpathlineto{\pgfqpoint{2.739720in}{1.392706in}}%
\pgfpathlineto{\pgfqpoint{2.744680in}{1.394179in}}%
\pgfpathlineto{\pgfqpoint{2.747160in}{1.393543in}}%
\pgfpathlineto{\pgfqpoint{2.752120in}{1.396351in}}%
\pgfpathlineto{\pgfqpoint{2.753360in}{1.396593in}}%
\pgfpathlineto{\pgfqpoint{2.757080in}{1.390843in}}%
\pgfpathlineto{\pgfqpoint{2.760800in}{1.392230in}}%
\pgfpathlineto{\pgfqpoint{2.762040in}{1.391744in}}%
\pgfpathlineto{\pgfqpoint{2.764520in}{1.392670in}}%
\pgfpathlineto{\pgfqpoint{2.768240in}{1.387684in}}%
\pgfpathlineto{\pgfqpoint{2.769480in}{1.388309in}}%
\pgfpathlineto{\pgfqpoint{2.771960in}{1.391573in}}%
\pgfpathlineto{\pgfqpoint{2.775680in}{1.391266in}}%
\pgfpathlineto{\pgfqpoint{2.778160in}{1.389286in}}%
\pgfpathlineto{\pgfqpoint{2.779400in}{1.389306in}}%
\pgfpathlineto{\pgfqpoint{2.780640in}{1.387947in}}%
\pgfpathlineto{\pgfqpoint{2.784360in}{1.390330in}}%
\pgfpathlineto{\pgfqpoint{2.786840in}{1.385274in}}%
\pgfpathlineto{\pgfqpoint{2.789320in}{1.387396in}}%
\pgfpathlineto{\pgfqpoint{2.791800in}{1.386649in}}%
\pgfpathlineto{\pgfqpoint{2.793040in}{1.387596in}}%
\pgfpathlineto{\pgfqpoint{2.794280in}{1.386313in}}%
\pgfpathlineto{\pgfqpoint{2.796760in}{1.388225in}}%
\pgfpathlineto{\pgfqpoint{2.798000in}{1.387109in}}%
\pgfpathlineto{\pgfqpoint{2.800480in}{1.391101in}}%
\pgfpathlineto{\pgfqpoint{2.802960in}{1.390104in}}%
\pgfpathlineto{\pgfqpoint{2.804200in}{1.388049in}}%
\pgfpathlineto{\pgfqpoint{2.806680in}{1.388610in}}%
\pgfpathlineto{\pgfqpoint{2.807920in}{1.390800in}}%
\pgfpathlineto{\pgfqpoint{2.809160in}{1.390534in}}%
\pgfpathlineto{\pgfqpoint{2.814120in}{1.394636in}}%
\pgfpathlineto{\pgfqpoint{2.816600in}{1.392884in}}%
\pgfpathlineto{\pgfqpoint{2.824040in}{1.384259in}}%
\pgfpathlineto{\pgfqpoint{2.827760in}{1.382387in}}%
\pgfpathlineto{\pgfqpoint{2.833960in}{1.386960in}}%
\pgfpathlineto{\pgfqpoint{2.836440in}{1.386750in}}%
\pgfpathlineto{\pgfqpoint{2.837680in}{1.387163in}}%
\pgfpathlineto{\pgfqpoint{2.840160in}{1.385716in}}%
\pgfpathlineto{\pgfqpoint{2.841400in}{1.385354in}}%
\pgfpathlineto{\pgfqpoint{2.843880in}{1.381395in}}%
\pgfpathlineto{\pgfqpoint{2.848840in}{1.385465in}}%
\pgfpathlineto{\pgfqpoint{2.853800in}{1.384512in}}%
\pgfpathlineto{\pgfqpoint{2.856280in}{1.384855in}}%
\pgfpathlineto{\pgfqpoint{2.857520in}{1.384974in}}%
\pgfpathlineto{\pgfqpoint{2.858760in}{1.386394in}}%
\pgfpathlineto{\pgfqpoint{2.860000in}{1.384749in}}%
\pgfpathlineto{\pgfqpoint{2.861240in}{1.386042in}}%
\pgfpathlineto{\pgfqpoint{2.862480in}{1.385373in}}%
\pgfpathlineto{\pgfqpoint{2.864960in}{1.388257in}}%
\pgfpathlineto{\pgfqpoint{2.868680in}{1.389349in}}%
\pgfpathlineto{\pgfqpoint{2.871160in}{1.388563in}}%
\pgfpathlineto{\pgfqpoint{2.874880in}{1.390233in}}%
\pgfpathlineto{\pgfqpoint{2.877360in}{1.390400in}}%
\pgfpathlineto{\pgfqpoint{2.882320in}{1.385035in}}%
\pgfpathlineto{\pgfqpoint{2.884800in}{1.386471in}}%
\pgfpathlineto{\pgfqpoint{2.887280in}{1.386581in}}%
\pgfpathlineto{\pgfqpoint{2.888520in}{1.386642in}}%
\pgfpathlineto{\pgfqpoint{2.891000in}{1.383093in}}%
\pgfpathlineto{\pgfqpoint{2.892240in}{1.382287in}}%
\pgfpathlineto{\pgfqpoint{2.897200in}{1.386407in}}%
\pgfpathlineto{\pgfqpoint{2.900920in}{1.384581in}}%
\pgfpathlineto{\pgfqpoint{2.905880in}{1.382530in}}%
\pgfpathlineto{\pgfqpoint{2.908360in}{1.384470in}}%
\pgfpathlineto{\pgfqpoint{2.910840in}{1.379877in}}%
\pgfpathlineto{\pgfqpoint{2.914560in}{1.380535in}}%
\pgfpathlineto{\pgfqpoint{2.915800in}{1.380405in}}%
\pgfpathlineto{\pgfqpoint{2.917040in}{1.381496in}}%
\pgfpathlineto{\pgfqpoint{2.918280in}{1.380159in}}%
\pgfpathlineto{\pgfqpoint{2.920760in}{1.381607in}}%
\pgfpathlineto{\pgfqpoint{2.922000in}{1.380728in}}%
\pgfpathlineto{\pgfqpoint{2.924480in}{1.384277in}}%
\pgfpathlineto{\pgfqpoint{2.926960in}{1.383493in}}%
\pgfpathlineto{\pgfqpoint{2.929440in}{1.381912in}}%
\pgfpathlineto{\pgfqpoint{2.930680in}{1.382035in}}%
\pgfpathlineto{\pgfqpoint{2.931920in}{1.384068in}}%
\pgfpathlineto{\pgfqpoint{2.933160in}{1.383551in}}%
\pgfpathlineto{\pgfqpoint{2.938120in}{1.387044in}}%
\pgfpathlineto{\pgfqpoint{2.940600in}{1.384625in}}%
\pgfpathlineto{\pgfqpoint{2.946800in}{1.374989in}}%
\pgfpathlineto{\pgfqpoint{2.948040in}{1.375122in}}%
\pgfpathlineto{\pgfqpoint{2.951760in}{1.372292in}}%
\pgfpathlineto{\pgfqpoint{2.956720in}{1.375740in}}%
\pgfpathlineto{\pgfqpoint{2.957960in}{1.376181in}}%
\pgfpathlineto{\pgfqpoint{2.960440in}{1.375906in}}%
\pgfpathlineto{\pgfqpoint{2.965400in}{1.375927in}}%
\pgfpathlineto{\pgfqpoint{2.967880in}{1.372014in}}%
\pgfpathlineto{\pgfqpoint{2.970360in}{1.375649in}}%
\pgfpathlineto{\pgfqpoint{2.972840in}{1.376901in}}%
\pgfpathlineto{\pgfqpoint{2.981520in}{1.376433in}}%
\pgfpathlineto{\pgfqpoint{2.982760in}{1.377153in}}%
\pgfpathlineto{\pgfqpoint{2.985240in}{1.374710in}}%
\pgfpathlineto{\pgfqpoint{2.986480in}{1.374272in}}%
\pgfpathlineto{\pgfqpoint{2.988960in}{1.376971in}}%
\pgfpathlineto{\pgfqpoint{2.993920in}{1.377486in}}%
\pgfpathlineto{\pgfqpoint{2.995160in}{1.377343in}}%
\pgfpathlineto{\pgfqpoint{2.998880in}{1.379613in}}%
\pgfpathlineto{\pgfqpoint{3.001360in}{1.379590in}}%
\pgfpathlineto{\pgfqpoint{3.006320in}{1.373641in}}%
\pgfpathlineto{\pgfqpoint{3.008800in}{1.374765in}}%
\pgfpathlineto{\pgfqpoint{3.012520in}{1.375695in}}%
\pgfpathlineto{\pgfqpoint{3.015000in}{1.371906in}}%
\pgfpathlineto{\pgfqpoint{3.016240in}{1.371246in}}%
\pgfpathlineto{\pgfqpoint{3.021200in}{1.374427in}}%
\pgfpathlineto{\pgfqpoint{3.026160in}{1.371424in}}%
\pgfpathlineto{\pgfqpoint{3.029880in}{1.369071in}}%
\pgfpathlineto{\pgfqpoint{3.032360in}{1.370409in}}%
\pgfpathlineto{\pgfqpoint{3.034840in}{1.365746in}}%
\pgfpathlineto{\pgfqpoint{3.037320in}{1.367647in}}%
\pgfpathlineto{\pgfqpoint{3.039800in}{1.365899in}}%
\pgfpathlineto{\pgfqpoint{3.041040in}{1.367178in}}%
\pgfpathlineto{\pgfqpoint{3.042280in}{1.366661in}}%
\pgfpathlineto{\pgfqpoint{3.044760in}{1.368403in}}%
\pgfpathlineto{\pgfqpoint{3.046000in}{1.367811in}}%
\pgfpathlineto{\pgfqpoint{3.048480in}{1.371293in}}%
\pgfpathlineto{\pgfqpoint{3.050960in}{1.370597in}}%
\pgfpathlineto{\pgfqpoint{3.053440in}{1.369427in}}%
\pgfpathlineto{\pgfqpoint{3.054680in}{1.369601in}}%
\pgfpathlineto{\pgfqpoint{3.055920in}{1.371769in}}%
\pgfpathlineto{\pgfqpoint{3.057160in}{1.370816in}}%
\pgfpathlineto{\pgfqpoint{3.063360in}{1.374043in}}%
\pgfpathlineto{\pgfqpoint{3.065840in}{1.370044in}}%
\pgfpathlineto{\pgfqpoint{3.069560in}{1.363823in}}%
\pgfpathlineto{\pgfqpoint{3.070800in}{1.362850in}}%
\pgfpathlineto{\pgfqpoint{3.072040in}{1.363454in}}%
\pgfpathlineto{\pgfqpoint{3.075760in}{1.360498in}}%
\pgfpathlineto{\pgfqpoint{3.078240in}{1.362813in}}%
\pgfpathlineto{\pgfqpoint{3.079480in}{1.364936in}}%
\pgfpathlineto{\pgfqpoint{3.081960in}{1.364697in}}%
\pgfpathlineto{\pgfqpoint{3.084440in}{1.364612in}}%
\pgfpathlineto{\pgfqpoint{3.085680in}{1.364687in}}%
\pgfpathlineto{\pgfqpoint{3.088160in}{1.363827in}}%
\pgfpathlineto{\pgfqpoint{3.089400in}{1.364173in}}%
\pgfpathlineto{\pgfqpoint{3.091880in}{1.360555in}}%
\pgfpathlineto{\pgfqpoint{3.093120in}{1.361222in}}%
\pgfpathlineto{\pgfqpoint{3.095600in}{1.365456in}}%
\pgfpathlineto{\pgfqpoint{3.104280in}{1.365209in}}%
\pgfpathlineto{\pgfqpoint{3.106760in}{1.366064in}}%
\pgfpathlineto{\pgfqpoint{3.110480in}{1.361812in}}%
\pgfpathlineto{\pgfqpoint{3.114200in}{1.364313in}}%
\pgfpathlineto{\pgfqpoint{3.121640in}{1.365339in}}%
\pgfpathlineto{\pgfqpoint{3.125360in}{1.365728in}}%
\pgfpathlineto{\pgfqpoint{3.130320in}{1.360785in}}%
\pgfpathlineto{\pgfqpoint{3.132800in}{1.362305in}}%
\pgfpathlineto{\pgfqpoint{3.136520in}{1.362510in}}%
\pgfpathlineto{\pgfqpoint{3.139000in}{1.358601in}}%
\pgfpathlineto{\pgfqpoint{3.140240in}{1.357957in}}%
\pgfpathlineto{\pgfqpoint{3.143960in}{1.359975in}}%
\pgfpathlineto{\pgfqpoint{3.150160in}{1.357200in}}%
\pgfpathlineto{\pgfqpoint{3.153880in}{1.354935in}}%
\pgfpathlineto{\pgfqpoint{3.156360in}{1.356481in}}%
\pgfpathlineto{\pgfqpoint{3.158840in}{1.352620in}}%
\pgfpathlineto{\pgfqpoint{3.161320in}{1.353951in}}%
\pgfpathlineto{\pgfqpoint{3.163800in}{1.351944in}}%
\pgfpathlineto{\pgfqpoint{3.165040in}{1.353200in}}%
\pgfpathlineto{\pgfqpoint{3.166280in}{1.352459in}}%
\pgfpathlineto{\pgfqpoint{3.168760in}{1.353661in}}%
\pgfpathlineto{\pgfqpoint{3.170000in}{1.353153in}}%
\pgfpathlineto{\pgfqpoint{3.172480in}{1.356553in}}%
\pgfpathlineto{\pgfqpoint{3.174960in}{1.355648in}}%
\pgfpathlineto{\pgfqpoint{3.177440in}{1.353989in}}%
\pgfpathlineto{\pgfqpoint{3.178680in}{1.353884in}}%
\pgfpathlineto{\pgfqpoint{3.179920in}{1.356126in}}%
\pgfpathlineto{\pgfqpoint{3.181160in}{1.355224in}}%
\pgfpathlineto{\pgfqpoint{3.186120in}{1.358208in}}%
\pgfpathlineto{\pgfqpoint{3.188600in}{1.355015in}}%
\pgfpathlineto{\pgfqpoint{3.192320in}{1.347526in}}%
\pgfpathlineto{\pgfqpoint{3.199760in}{1.343404in}}%
\pgfpathlineto{\pgfqpoint{3.205960in}{1.346634in}}%
\pgfpathlineto{\pgfqpoint{3.209680in}{1.346830in}}%
\pgfpathlineto{\pgfqpoint{3.212160in}{1.345782in}}%
\pgfpathlineto{\pgfqpoint{3.213400in}{1.346346in}}%
\pgfpathlineto{\pgfqpoint{3.215880in}{1.342782in}}%
\pgfpathlineto{\pgfqpoint{3.217120in}{1.343560in}}%
\pgfpathlineto{\pgfqpoint{3.219600in}{1.347456in}}%
\pgfpathlineto{\pgfqpoint{3.229520in}{1.347228in}}%
\pgfpathlineto{\pgfqpoint{3.230760in}{1.348565in}}%
\pgfpathlineto{\pgfqpoint{3.234480in}{1.343605in}}%
\pgfpathlineto{\pgfqpoint{3.238200in}{1.347337in}}%
\pgfpathlineto{\pgfqpoint{3.240680in}{1.347564in}}%
\pgfpathlineto{\pgfqpoint{3.243160in}{1.345664in}}%
\pgfpathlineto{\pgfqpoint{3.248120in}{1.347998in}}%
\pgfpathlineto{\pgfqpoint{3.254320in}{1.341888in}}%
\pgfpathlineto{\pgfqpoint{3.256800in}{1.343128in}}%
\pgfpathlineto{\pgfqpoint{3.260520in}{1.344213in}}%
\pgfpathlineto{\pgfqpoint{3.263000in}{1.340140in}}%
\pgfpathlineto{\pgfqpoint{3.264240in}{1.339036in}}%
\pgfpathlineto{\pgfqpoint{3.269200in}{1.340444in}}%
\pgfpathlineto{\pgfqpoint{3.277880in}{1.336427in}}%
\pgfpathlineto{\pgfqpoint{3.279120in}{1.337871in}}%
\pgfpathlineto{\pgfqpoint{3.280360in}{1.337328in}}%
\pgfpathlineto{\pgfqpoint{3.282840in}{1.332622in}}%
\pgfpathlineto{\pgfqpoint{3.285320in}{1.333441in}}%
\pgfpathlineto{\pgfqpoint{3.287800in}{1.331279in}}%
\pgfpathlineto{\pgfqpoint{3.289040in}{1.332753in}}%
\pgfpathlineto{\pgfqpoint{3.291520in}{1.332312in}}%
\pgfpathlineto{\pgfqpoint{3.295240in}{1.334631in}}%
\pgfpathlineto{\pgfqpoint{3.297720in}{1.335627in}}%
\pgfpathlineto{\pgfqpoint{3.298960in}{1.334861in}}%
\pgfpathlineto{\pgfqpoint{3.301440in}{1.332626in}}%
\pgfpathlineto{\pgfqpoint{3.302680in}{1.332728in}}%
\pgfpathlineto{\pgfqpoint{3.303920in}{1.334834in}}%
\pgfpathlineto{\pgfqpoint{3.305160in}{1.334393in}}%
\pgfpathlineto{\pgfqpoint{3.310120in}{1.336321in}}%
\pgfpathlineto{\pgfqpoint{3.312600in}{1.333945in}}%
\pgfpathlineto{\pgfqpoint{3.318800in}{1.324404in}}%
\pgfpathlineto{\pgfqpoint{3.321280in}{1.324061in}}%
\pgfpathlineto{\pgfqpoint{3.323760in}{1.322109in}}%
\pgfpathlineto{\pgfqpoint{3.329960in}{1.326780in}}%
\pgfpathlineto{\pgfqpoint{3.338640in}{1.325557in}}%
\pgfpathlineto{\pgfqpoint{3.339880in}{1.324188in}}%
\pgfpathlineto{\pgfqpoint{3.341120in}{1.324963in}}%
\pgfpathlineto{\pgfqpoint{3.343600in}{1.328919in}}%
\pgfpathlineto{\pgfqpoint{3.347320in}{1.328863in}}%
\pgfpathlineto{\pgfqpoint{3.352280in}{1.329237in}}%
\pgfpathlineto{\pgfqpoint{3.354760in}{1.330947in}}%
\pgfpathlineto{\pgfqpoint{3.358480in}{1.325431in}}%
\pgfpathlineto{\pgfqpoint{3.363440in}{1.330092in}}%
\pgfpathlineto{\pgfqpoint{3.365920in}{1.328956in}}%
\pgfpathlineto{\pgfqpoint{3.368400in}{1.328615in}}%
\pgfpathlineto{\pgfqpoint{3.372120in}{1.329824in}}%
\pgfpathlineto{\pgfqpoint{3.378320in}{1.324339in}}%
\pgfpathlineto{\pgfqpoint{3.380800in}{1.325927in}}%
\pgfpathlineto{\pgfqpoint{3.383280in}{1.325905in}}%
\pgfpathlineto{\pgfqpoint{3.384520in}{1.326174in}}%
\pgfpathlineto{\pgfqpoint{3.387000in}{1.323204in}}%
\pgfpathlineto{\pgfqpoint{3.388240in}{1.322080in}}%
\pgfpathlineto{\pgfqpoint{3.391960in}{1.323992in}}%
\pgfpathlineto{\pgfqpoint{3.394440in}{1.323547in}}%
\pgfpathlineto{\pgfqpoint{3.395680in}{1.324369in}}%
\pgfpathlineto{\pgfqpoint{3.399400in}{1.322258in}}%
\pgfpathlineto{\pgfqpoint{3.401880in}{1.320505in}}%
\pgfpathlineto{\pgfqpoint{3.404360in}{1.321494in}}%
\pgfpathlineto{\pgfqpoint{3.406840in}{1.317169in}}%
\pgfpathlineto{\pgfqpoint{3.409320in}{1.318938in}}%
\pgfpathlineto{\pgfqpoint{3.411800in}{1.316417in}}%
\pgfpathlineto{\pgfqpoint{3.413040in}{1.317503in}}%
\pgfpathlineto{\pgfqpoint{3.414280in}{1.316436in}}%
\pgfpathlineto{\pgfqpoint{3.419240in}{1.319712in}}%
\pgfpathlineto{\pgfqpoint{3.421720in}{1.320859in}}%
\pgfpathlineto{\pgfqpoint{3.426680in}{1.316829in}}%
\pgfpathlineto{\pgfqpoint{3.427920in}{1.318788in}}%
\pgfpathlineto{\pgfqpoint{3.431640in}{1.318115in}}%
\pgfpathlineto{\pgfqpoint{3.434120in}{1.319204in}}%
\pgfpathlineto{\pgfqpoint{3.437840in}{1.314555in}}%
\pgfpathlineto{\pgfqpoint{3.441560in}{1.310020in}}%
\pgfpathlineto{\pgfqpoint{3.442800in}{1.308454in}}%
\pgfpathlineto{\pgfqpoint{3.445280in}{1.308781in}}%
\pgfpathlineto{\pgfqpoint{3.447760in}{1.307111in}}%
\pgfpathlineto{\pgfqpoint{3.453960in}{1.312799in}}%
\pgfpathlineto{\pgfqpoint{3.457680in}{1.314057in}}%
\pgfpathlineto{\pgfqpoint{3.458920in}{1.312729in}}%
\pgfpathlineto{\pgfqpoint{3.461400in}{1.313812in}}%
\pgfpathlineto{\pgfqpoint{3.463880in}{1.310306in}}%
\pgfpathlineto{\pgfqpoint{3.465120in}{1.311054in}}%
\pgfpathlineto{\pgfqpoint{3.467600in}{1.315226in}}%
\pgfpathlineto{\pgfqpoint{3.477520in}{1.316713in}}%
\pgfpathlineto{\pgfqpoint{3.478760in}{1.317848in}}%
\pgfpathlineto{\pgfqpoint{3.480000in}{1.315844in}}%
\pgfpathlineto{\pgfqpoint{3.482480in}{1.316900in}}%
\pgfpathlineto{\pgfqpoint{3.488680in}{1.320914in}}%
\pgfpathlineto{\pgfqpoint{3.492400in}{1.318959in}}%
\pgfpathlineto{\pgfqpoint{3.496120in}{1.320328in}}%
\pgfpathlineto{\pgfqpoint{3.502320in}{1.313833in}}%
\pgfpathlineto{\pgfqpoint{3.504800in}{1.314593in}}%
\pgfpathlineto{\pgfqpoint{3.511000in}{1.311587in}}%
\pgfpathlineto{\pgfqpoint{3.512240in}{1.310641in}}%
\pgfpathlineto{\pgfqpoint{3.515960in}{1.312707in}}%
\pgfpathlineto{\pgfqpoint{3.518440in}{1.312739in}}%
\pgfpathlineto{\pgfqpoint{3.519680in}{1.313510in}}%
\pgfpathlineto{\pgfqpoint{3.523400in}{1.309965in}}%
\pgfpathlineto{\pgfqpoint{3.524640in}{1.308169in}}%
\pgfpathlineto{\pgfqpoint{3.528360in}{1.310569in}}%
\pgfpathlineto{\pgfqpoint{3.530840in}{1.306442in}}%
\pgfpathlineto{\pgfqpoint{3.533320in}{1.307775in}}%
\pgfpathlineto{\pgfqpoint{3.538280in}{1.305477in}}%
\pgfpathlineto{\pgfqpoint{3.542000in}{1.305379in}}%
\pgfpathlineto{\pgfqpoint{3.544480in}{1.309249in}}%
\pgfpathlineto{\pgfqpoint{3.546960in}{1.308865in}}%
\pgfpathlineto{\pgfqpoint{3.549440in}{1.306142in}}%
\pgfpathlineto{\pgfqpoint{3.550680in}{1.305713in}}%
\pgfpathlineto{\pgfqpoint{3.553160in}{1.308249in}}%
\pgfpathlineto{\pgfqpoint{3.554400in}{1.308821in}}%
\pgfpathlineto{\pgfqpoint{3.555640in}{1.308165in}}%
\pgfpathlineto{\pgfqpoint{3.558120in}{1.308955in}}%
\pgfpathlineto{\pgfqpoint{3.561840in}{1.304009in}}%
\pgfpathlineto{\pgfqpoint{3.566800in}{1.298208in}}%
\pgfpathlineto{\pgfqpoint{3.569280in}{1.298684in}}%
\pgfpathlineto{\pgfqpoint{3.571760in}{1.296894in}}%
\pgfpathlineto{\pgfqpoint{3.574240in}{1.298884in}}%
\pgfpathlineto{\pgfqpoint{3.575480in}{1.300794in}}%
\pgfpathlineto{\pgfqpoint{3.576720in}{1.300506in}}%
\pgfpathlineto{\pgfqpoint{3.579200in}{1.302474in}}%
\pgfpathlineto{\pgfqpoint{3.581680in}{1.303217in}}%
\pgfpathlineto{\pgfqpoint{3.582920in}{1.301874in}}%
\pgfpathlineto{\pgfqpoint{3.585400in}{1.303542in}}%
\pgfpathlineto{\pgfqpoint{3.587880in}{1.300322in}}%
\pgfpathlineto{\pgfqpoint{3.589120in}{1.300627in}}%
\pgfpathlineto{\pgfqpoint{3.591600in}{1.305210in}}%
\pgfpathlineto{\pgfqpoint{3.601520in}{1.306867in}}%
\pgfpathlineto{\pgfqpoint{3.602760in}{1.307725in}}%
\pgfpathlineto{\pgfqpoint{3.604000in}{1.305862in}}%
\pgfpathlineto{\pgfqpoint{3.608960in}{1.308964in}}%
\pgfpathlineto{\pgfqpoint{3.611440in}{1.310382in}}%
\pgfpathlineto{\pgfqpoint{3.613920in}{1.309451in}}%
\pgfpathlineto{\pgfqpoint{3.616400in}{1.309136in}}%
\pgfpathlineto{\pgfqpoint{3.620120in}{1.310739in}}%
\pgfpathlineto{\pgfqpoint{3.626320in}{1.303638in}}%
\pgfpathlineto{\pgfqpoint{3.632520in}{1.304690in}}%
\pgfpathlineto{\pgfqpoint{3.635000in}{1.302254in}}%
\pgfpathlineto{\pgfqpoint{3.637480in}{1.301197in}}%
\pgfpathlineto{\pgfqpoint{3.641200in}{1.302131in}}%
\pgfpathlineto{\pgfqpoint{3.644920in}{1.302382in}}%
\pgfpathlineto{\pgfqpoint{3.649880in}{1.298042in}}%
\pgfpathlineto{\pgfqpoint{3.652360in}{1.299623in}}%
\pgfpathlineto{\pgfqpoint{3.654840in}{1.295917in}}%
\pgfpathlineto{\pgfqpoint{3.657320in}{1.296849in}}%
\pgfpathlineto{\pgfqpoint{3.659800in}{1.294458in}}%
\pgfpathlineto{\pgfqpoint{3.661040in}{1.295569in}}%
\pgfpathlineto{\pgfqpoint{3.663520in}{1.294389in}}%
\pgfpathlineto{\pgfqpoint{3.664760in}{1.294707in}}%
\pgfpathlineto{\pgfqpoint{3.666000in}{1.293448in}}%
\pgfpathlineto{\pgfqpoint{3.668480in}{1.296769in}}%
\pgfpathlineto{\pgfqpoint{3.670960in}{1.296978in}}%
\pgfpathlineto{\pgfqpoint{3.673440in}{1.294329in}}%
\pgfpathlineto{\pgfqpoint{3.674680in}{1.294204in}}%
\pgfpathlineto{\pgfqpoint{3.677160in}{1.297239in}}%
\pgfpathlineto{\pgfqpoint{3.682120in}{1.299395in}}%
\pgfpathlineto{\pgfqpoint{3.685840in}{1.295423in}}%
\pgfpathlineto{\pgfqpoint{3.690800in}{1.289823in}}%
\pgfpathlineto{\pgfqpoint{3.693280in}{1.290404in}}%
\pgfpathlineto{\pgfqpoint{3.695760in}{1.288337in}}%
\pgfpathlineto{\pgfqpoint{3.698240in}{1.290649in}}%
\pgfpathlineto{\pgfqpoint{3.700720in}{1.292313in}}%
\pgfpathlineto{\pgfqpoint{3.704440in}{1.293247in}}%
\pgfpathlineto{\pgfqpoint{3.710640in}{1.291597in}}%
\pgfpathlineto{\pgfqpoint{3.713120in}{1.291054in}}%
\pgfpathlineto{\pgfqpoint{3.716840in}{1.295498in}}%
\pgfpathlineto{\pgfqpoint{3.721800in}{1.296968in}}%
\pgfpathlineto{\pgfqpoint{3.725520in}{1.297639in}}%
\pgfpathlineto{\pgfqpoint{3.726760in}{1.298014in}}%
\pgfpathlineto{\pgfqpoint{3.728000in}{1.296263in}}%
\pgfpathlineto{\pgfqpoint{3.729240in}{1.297826in}}%
\pgfpathlineto{\pgfqpoint{3.730480in}{1.297410in}}%
\pgfpathlineto{\pgfqpoint{3.736680in}{1.301621in}}%
\pgfpathlineto{\pgfqpoint{3.740400in}{1.299555in}}%
\pgfpathlineto{\pgfqpoint{3.742880in}{1.300572in}}%
\pgfpathlineto{\pgfqpoint{3.744120in}{1.301370in}}%
\pgfpathlineto{\pgfqpoint{3.750320in}{1.293769in}}%
\pgfpathlineto{\pgfqpoint{3.754040in}{1.296113in}}%
\pgfpathlineto{\pgfqpoint{3.761480in}{1.293313in}}%
\pgfpathlineto{\pgfqpoint{3.767680in}{1.295850in}}%
\pgfpathlineto{\pgfqpoint{3.771400in}{1.291407in}}%
\pgfpathlineto{\pgfqpoint{3.772640in}{1.289928in}}%
\pgfpathlineto{\pgfqpoint{3.776360in}{1.292036in}}%
\pgfpathlineto{\pgfqpoint{3.778840in}{1.287938in}}%
\pgfpathlineto{\pgfqpoint{3.781320in}{1.288137in}}%
\pgfpathlineto{\pgfqpoint{3.783800in}{1.285897in}}%
\pgfpathlineto{\pgfqpoint{3.785040in}{1.287429in}}%
\pgfpathlineto{\pgfqpoint{3.787520in}{1.286332in}}%
\pgfpathlineto{\pgfqpoint{3.788760in}{1.286520in}}%
\pgfpathlineto{\pgfqpoint{3.790000in}{1.284970in}}%
\pgfpathlineto{\pgfqpoint{3.792480in}{1.288053in}}%
\pgfpathlineto{\pgfqpoint{3.794960in}{1.287373in}}%
\pgfpathlineto{\pgfqpoint{3.798680in}{1.284459in}}%
\pgfpathlineto{\pgfqpoint{3.801160in}{1.287341in}}%
\pgfpathlineto{\pgfqpoint{3.806120in}{1.289845in}}%
\pgfpathlineto{\pgfqpoint{3.811080in}{1.284820in}}%
\pgfpathlineto{\pgfqpoint{3.813560in}{1.281299in}}%
\pgfpathlineto{\pgfqpoint{3.814800in}{1.280140in}}%
\pgfpathlineto{\pgfqpoint{3.817280in}{1.280252in}}%
\pgfpathlineto{\pgfqpoint{3.819760in}{1.277323in}}%
\pgfpathlineto{\pgfqpoint{3.822240in}{1.279579in}}%
\pgfpathlineto{\pgfqpoint{3.823480in}{1.281361in}}%
\pgfpathlineto{\pgfqpoint{3.825960in}{1.281836in}}%
\pgfpathlineto{\pgfqpoint{3.829680in}{1.282179in}}%
\pgfpathlineto{\pgfqpoint{3.830920in}{1.280953in}}%
\pgfpathlineto{\pgfqpoint{3.833400in}{1.282879in}}%
\pgfpathlineto{\pgfqpoint{3.835880in}{1.279529in}}%
\pgfpathlineto{\pgfqpoint{3.837120in}{1.280165in}}%
\pgfpathlineto{\pgfqpoint{3.840840in}{1.285316in}}%
\pgfpathlineto{\pgfqpoint{3.844560in}{1.286046in}}%
\pgfpathlineto{\pgfqpoint{3.849520in}{1.288437in}}%
\pgfpathlineto{\pgfqpoint{3.850760in}{1.288575in}}%
\pgfpathlineto{\pgfqpoint{3.852000in}{1.286859in}}%
\pgfpathlineto{\pgfqpoint{3.853240in}{1.289003in}}%
\pgfpathlineto{\pgfqpoint{3.854480in}{1.288334in}}%
\pgfpathlineto{\pgfqpoint{3.860680in}{1.293065in}}%
\pgfpathlineto{\pgfqpoint{3.864400in}{1.291719in}}%
\pgfpathlineto{\pgfqpoint{3.869360in}{1.291337in}}%
\pgfpathlineto{\pgfqpoint{3.873080in}{1.285647in}}%
\pgfpathlineto{\pgfqpoint{3.875560in}{1.285525in}}%
\pgfpathlineto{\pgfqpoint{3.878040in}{1.286662in}}%
\pgfpathlineto{\pgfqpoint{3.886720in}{1.284224in}}%
\pgfpathlineto{\pgfqpoint{3.892920in}{1.286187in}}%
\pgfpathlineto{\pgfqpoint{3.896640in}{1.281904in}}%
\pgfpathlineto{\pgfqpoint{3.900360in}{1.283622in}}%
\pgfpathlineto{\pgfqpoint{3.902840in}{1.278912in}}%
\pgfpathlineto{\pgfqpoint{3.905320in}{1.279460in}}%
\pgfpathlineto{\pgfqpoint{3.907800in}{1.276903in}}%
\pgfpathlineto{\pgfqpoint{3.909040in}{1.278192in}}%
\pgfpathlineto{\pgfqpoint{3.911520in}{1.277084in}}%
\pgfpathlineto{\pgfqpoint{3.912760in}{1.277011in}}%
\pgfpathlineto{\pgfqpoint{3.914000in}{1.275050in}}%
\pgfpathlineto{\pgfqpoint{3.916480in}{1.277897in}}%
\pgfpathlineto{\pgfqpoint{3.918960in}{1.276857in}}%
\pgfpathlineto{\pgfqpoint{3.922680in}{1.274230in}}%
\pgfpathlineto{\pgfqpoint{3.925160in}{1.277517in}}%
\pgfpathlineto{\pgfqpoint{3.928880in}{1.279680in}}%
\pgfpathlineto{\pgfqpoint{3.930120in}{1.279906in}}%
\pgfpathlineto{\pgfqpoint{3.935080in}{1.275482in}}%
\pgfpathlineto{\pgfqpoint{3.937560in}{1.271947in}}%
\pgfpathlineto{\pgfqpoint{3.938800in}{1.271090in}}%
\pgfpathlineto{\pgfqpoint{3.941280in}{1.271949in}}%
\pgfpathlineto{\pgfqpoint{3.943760in}{1.269280in}}%
\pgfpathlineto{\pgfqpoint{3.945000in}{1.270007in}}%
\pgfpathlineto{\pgfqpoint{3.947480in}{1.274219in}}%
\pgfpathlineto{\pgfqpoint{3.951200in}{1.274763in}}%
\pgfpathlineto{\pgfqpoint{3.953680in}{1.274887in}}%
\pgfpathlineto{\pgfqpoint{3.954920in}{1.273585in}}%
\pgfpathlineto{\pgfqpoint{3.957400in}{1.275159in}}%
\pgfpathlineto{\pgfqpoint{3.959880in}{1.272259in}}%
\pgfpathlineto{\pgfqpoint{3.961120in}{1.273054in}}%
\pgfpathlineto{\pgfqpoint{3.964840in}{1.278484in}}%
\pgfpathlineto{\pgfqpoint{3.966080in}{1.278257in}}%
\pgfpathlineto{\pgfqpoint{3.967320in}{1.279325in}}%
\pgfpathlineto{\pgfqpoint{3.968560in}{1.278919in}}%
\pgfpathlineto{\pgfqpoint{3.974760in}{1.282305in}}%
\pgfpathlineto{\pgfqpoint{3.976000in}{1.280192in}}%
\pgfpathlineto{\pgfqpoint{3.977240in}{1.281239in}}%
\pgfpathlineto{\pgfqpoint{3.978480in}{1.280419in}}%
\pgfpathlineto{\pgfqpoint{3.984680in}{1.285720in}}%
\pgfpathlineto{\pgfqpoint{3.988400in}{1.284035in}}%
\pgfpathlineto{\pgfqpoint{3.989640in}{1.284340in}}%
\pgfpathlineto{\pgfqpoint{3.992120in}{1.286484in}}%
\pgfpathlineto{\pgfqpoint{3.998320in}{1.278230in}}%
\pgfpathlineto{\pgfqpoint{4.005760in}{1.278469in}}%
\pgfpathlineto{\pgfqpoint{4.007000in}{1.278253in}}%
\pgfpathlineto{\pgfqpoint{4.009480in}{1.276492in}}%
\pgfpathlineto{\pgfqpoint{4.016920in}{1.278334in}}%
\pgfpathlineto{\pgfqpoint{4.020640in}{1.275194in}}%
\pgfpathlineto{\pgfqpoint{4.024360in}{1.277763in}}%
\pgfpathlineto{\pgfqpoint{4.026840in}{1.274195in}}%
\pgfpathlineto{\pgfqpoint{4.029320in}{1.273693in}}%
\pgfpathlineto{\pgfqpoint{4.031800in}{1.271377in}}%
\pgfpathlineto{\pgfqpoint{4.033040in}{1.272081in}}%
\pgfpathlineto{\pgfqpoint{4.035520in}{1.270523in}}%
\pgfpathlineto{\pgfqpoint{4.036760in}{1.270304in}}%
\pgfpathlineto{\pgfqpoint{4.038000in}{1.268408in}}%
\pgfpathlineto{\pgfqpoint{4.039240in}{1.270737in}}%
\pgfpathlineto{\pgfqpoint{4.044200in}{1.269575in}}%
\pgfpathlineto{\pgfqpoint{4.046680in}{1.267092in}}%
\pgfpathlineto{\pgfqpoint{4.049160in}{1.269876in}}%
\pgfpathlineto{\pgfqpoint{4.051640in}{1.270679in}}%
\pgfpathlineto{\pgfqpoint{4.054120in}{1.270955in}}%
\pgfpathlineto{\pgfqpoint{4.057840in}{1.266680in}}%
\pgfpathlineto{\pgfqpoint{4.059080in}{1.266443in}}%
\pgfpathlineto{\pgfqpoint{4.061560in}{1.263466in}}%
\pgfpathlineto{\pgfqpoint{4.064040in}{1.263285in}}%
\pgfpathlineto{\pgfqpoint{4.065280in}{1.263394in}}%
\pgfpathlineto{\pgfqpoint{4.067760in}{1.260639in}}%
\pgfpathlineto{\pgfqpoint{4.069000in}{1.261520in}}%
\pgfpathlineto{\pgfqpoint{4.071480in}{1.266658in}}%
\pgfpathlineto{\pgfqpoint{4.080160in}{1.266266in}}%
\pgfpathlineto{\pgfqpoint{4.081400in}{1.266707in}}%
\pgfpathlineto{\pgfqpoint{4.083880in}{1.264023in}}%
\pgfpathlineto{\pgfqpoint{4.085120in}{1.264393in}}%
\pgfpathlineto{\pgfqpoint{4.088840in}{1.269595in}}%
\pgfpathlineto{\pgfqpoint{4.090080in}{1.269430in}}%
\pgfpathlineto{\pgfqpoint{4.091320in}{1.270904in}}%
\pgfpathlineto{\pgfqpoint{4.092560in}{1.270634in}}%
\pgfpathlineto{\pgfqpoint{4.098760in}{1.274805in}}%
\pgfpathlineto{\pgfqpoint{4.100000in}{1.272697in}}%
\pgfpathlineto{\pgfqpoint{4.101240in}{1.273324in}}%
\pgfpathlineto{\pgfqpoint{4.102480in}{1.272039in}}%
\pgfpathlineto{\pgfqpoint{4.104960in}{1.273874in}}%
\pgfpathlineto{\pgfqpoint{4.108680in}{1.275685in}}%
\pgfpathlineto{\pgfqpoint{4.111160in}{1.272802in}}%
\pgfpathlineto{\pgfqpoint{4.113640in}{1.274132in}}%
\pgfpathlineto{\pgfqpoint{4.116120in}{1.276367in}}%
\pgfpathlineto{\pgfqpoint{4.121080in}{1.268810in}}%
\pgfpathlineto{\pgfqpoint{4.123560in}{1.268056in}}%
\pgfpathlineto{\pgfqpoint{4.126040in}{1.269528in}}%
\pgfpathlineto{\pgfqpoint{4.127280in}{1.269233in}}%
\pgfpathlineto{\pgfqpoint{4.128520in}{1.270127in}}%
\pgfpathlineto{\pgfqpoint{4.131000in}{1.268523in}}%
\pgfpathlineto{\pgfqpoint{4.133480in}{1.266783in}}%
\pgfpathlineto{\pgfqpoint{4.140920in}{1.268328in}}%
\pgfpathlineto{\pgfqpoint{4.142160in}{1.267520in}}%
\pgfpathlineto{\pgfqpoint{4.144640in}{1.264367in}}%
\pgfpathlineto{\pgfqpoint{4.148360in}{1.268441in}}%
\pgfpathlineto{\pgfqpoint{4.150840in}{1.264678in}}%
\pgfpathlineto{\pgfqpoint{4.153320in}{1.264762in}}%
\pgfpathlineto{\pgfqpoint{4.155800in}{1.262434in}}%
\pgfpathlineto{\pgfqpoint{4.157040in}{1.263081in}}%
\pgfpathlineto{\pgfqpoint{4.159520in}{1.261412in}}%
\pgfpathlineto{\pgfqpoint{4.160760in}{1.261250in}}%
\pgfpathlineto{\pgfqpoint{4.162000in}{1.259689in}}%
\pgfpathlineto{\pgfqpoint{4.164480in}{1.262625in}}%
\pgfpathlineto{\pgfqpoint{4.166960in}{1.261886in}}%
\pgfpathlineto{\pgfqpoint{4.169440in}{1.260503in}}%
\pgfpathlineto{\pgfqpoint{4.170680in}{1.259032in}}%
\pgfpathlineto{\pgfqpoint{4.173160in}{1.261624in}}%
\pgfpathlineto{\pgfqpoint{4.176880in}{1.262487in}}%
\pgfpathlineto{\pgfqpoint{4.178120in}{1.262273in}}%
\pgfpathlineto{\pgfqpoint{4.181840in}{1.258394in}}%
\pgfpathlineto{\pgfqpoint{4.183080in}{1.257886in}}%
\pgfpathlineto{\pgfqpoint{4.185560in}{1.254410in}}%
\pgfpathlineto{\pgfqpoint{4.188040in}{1.254392in}}%
\pgfpathlineto{\pgfqpoint{4.189280in}{1.254992in}}%
\pgfpathlineto{\pgfqpoint{4.191760in}{1.252355in}}%
\pgfpathlineto{\pgfqpoint{4.193000in}{1.253201in}}%
\pgfpathlineto{\pgfqpoint{4.195480in}{1.258641in}}%
\pgfpathlineto{\pgfqpoint{4.204160in}{1.257672in}}%
\pgfpathlineto{\pgfqpoint{4.205400in}{1.258396in}}%
\pgfpathlineto{\pgfqpoint{4.207880in}{1.255750in}}%
\pgfpathlineto{\pgfqpoint{4.209120in}{1.255809in}}%
\pgfpathlineto{\pgfqpoint{4.211600in}{1.259930in}}%
\pgfpathlineto{\pgfqpoint{4.214080in}{1.258984in}}%
\pgfpathlineto{\pgfqpoint{4.215320in}{1.260617in}}%
\pgfpathlineto{\pgfqpoint{4.216560in}{1.260018in}}%
\pgfpathlineto{\pgfqpoint{4.221520in}{1.264253in}}%
\pgfpathlineto{\pgfqpoint{4.222760in}{1.264783in}}%
\pgfpathlineto{\pgfqpoint{4.225240in}{1.262656in}}%
\pgfpathlineto{\pgfqpoint{4.226480in}{1.260955in}}%
\pgfpathlineto{\pgfqpoint{4.232680in}{1.264033in}}%
\pgfpathlineto{\pgfqpoint{4.235160in}{1.262124in}}%
\pgfpathlineto{\pgfqpoint{4.240120in}{1.266502in}}%
\pgfpathlineto{\pgfqpoint{4.245080in}{1.259411in}}%
\pgfpathlineto{\pgfqpoint{4.247560in}{1.259076in}}%
\pgfpathlineto{\pgfqpoint{4.250040in}{1.261105in}}%
\pgfpathlineto{\pgfqpoint{4.258720in}{1.258252in}}%
\pgfpathlineto{\pgfqpoint{4.261200in}{1.258827in}}%
\pgfpathlineto{\pgfqpoint{4.264920in}{1.258445in}}%
\pgfpathlineto{\pgfqpoint{4.266160in}{1.257730in}}%
\pgfpathlineto{\pgfqpoint{4.268640in}{1.254850in}}%
\pgfpathlineto{\pgfqpoint{4.272360in}{1.259669in}}%
\pgfpathlineto{\pgfqpoint{4.274840in}{1.255646in}}%
\pgfpathlineto{\pgfqpoint{4.276080in}{1.256454in}}%
\pgfpathlineto{\pgfqpoint{4.279800in}{1.253332in}}%
\pgfpathlineto{\pgfqpoint{4.281040in}{1.254907in}}%
\pgfpathlineto{\pgfqpoint{4.286000in}{1.252175in}}%
\pgfpathlineto{\pgfqpoint{4.287240in}{1.254738in}}%
\pgfpathlineto{\pgfqpoint{4.293440in}{1.253308in}}%
\pgfpathlineto{\pgfqpoint{4.294680in}{1.251902in}}%
\pgfpathlineto{\pgfqpoint{4.297160in}{1.254239in}}%
\pgfpathlineto{\pgfqpoint{4.300880in}{1.255397in}}%
\pgfpathlineto{\pgfqpoint{4.302120in}{1.255240in}}%
\pgfpathlineto{\pgfqpoint{4.308320in}{1.246734in}}%
\pgfpathlineto{\pgfqpoint{4.310800in}{1.245301in}}%
\pgfpathlineto{\pgfqpoint{4.313280in}{1.246164in}}%
\pgfpathlineto{\pgfqpoint{4.315760in}{1.243888in}}%
\pgfpathlineto{\pgfqpoint{4.317000in}{1.244655in}}%
\pgfpathlineto{\pgfqpoint{4.320720in}{1.249020in}}%
\pgfpathlineto{\pgfqpoint{4.326920in}{1.246857in}}%
\pgfpathlineto{\pgfqpoint{4.329400in}{1.248738in}}%
\pgfpathlineto{\pgfqpoint{4.331880in}{1.246356in}}%
\pgfpathlineto{\pgfqpoint{4.333120in}{1.246586in}}%
\pgfpathlineto{\pgfqpoint{4.335600in}{1.250318in}}%
\pgfpathlineto{\pgfqpoint{4.338080in}{1.249083in}}%
\pgfpathlineto{\pgfqpoint{4.339320in}{1.251213in}}%
\pgfpathlineto{\pgfqpoint{4.340560in}{1.250544in}}%
\pgfpathlineto{\pgfqpoint{4.345520in}{1.254559in}}%
\pgfpathlineto{\pgfqpoint{4.346760in}{1.255042in}}%
\pgfpathlineto{\pgfqpoint{4.348000in}{1.252510in}}%
\pgfpathlineto{\pgfqpoint{4.349240in}{1.254864in}}%
\pgfpathlineto{\pgfqpoint{4.350480in}{1.252441in}}%
\pgfpathlineto{\pgfqpoint{4.351720in}{1.252802in}}%
\pgfpathlineto{\pgfqpoint{4.354200in}{1.255481in}}%
\pgfpathlineto{\pgfqpoint{4.361640in}{1.256392in}}%
\pgfpathlineto{\pgfqpoint{4.364120in}{1.258896in}}%
\pgfpathlineto{\pgfqpoint{4.369080in}{1.251215in}}%
\pgfpathlineto{\pgfqpoint{4.371560in}{1.251913in}}%
\pgfpathlineto{\pgfqpoint{4.374040in}{1.253434in}}%
\pgfpathlineto{\pgfqpoint{4.376520in}{1.253407in}}%
\pgfpathlineto{\pgfqpoint{4.377760in}{1.251575in}}%
\pgfpathlineto{\pgfqpoint{4.379000in}{1.251777in}}%
\pgfpathlineto{\pgfqpoint{4.381480in}{1.249648in}}%
\pgfpathlineto{\pgfqpoint{4.383960in}{1.250543in}}%
\pgfpathlineto{\pgfqpoint{4.386440in}{1.248850in}}%
\pgfpathlineto{\pgfqpoint{4.388920in}{1.248428in}}%
\pgfpathlineto{\pgfqpoint{4.390160in}{1.247947in}}%
\pgfpathlineto{\pgfqpoint{4.391400in}{1.245867in}}%
\pgfpathlineto{\pgfqpoint{4.392640in}{1.246400in}}%
\pgfpathlineto{\pgfqpoint{4.396360in}{1.251548in}}%
\pgfpathlineto{\pgfqpoint{4.398840in}{1.247671in}}%
\pgfpathlineto{\pgfqpoint{4.400080in}{1.248667in}}%
\pgfpathlineto{\pgfqpoint{4.401320in}{1.248226in}}%
\pgfpathlineto{\pgfqpoint{4.403800in}{1.245433in}}%
\pgfpathlineto{\pgfqpoint{4.405040in}{1.247505in}}%
\pgfpathlineto{\pgfqpoint{4.410000in}{1.243485in}}%
\pgfpathlineto{\pgfqpoint{4.411240in}{1.246002in}}%
\pgfpathlineto{\pgfqpoint{4.413720in}{1.245069in}}%
\pgfpathlineto{\pgfqpoint{4.417440in}{1.244429in}}%
\pgfpathlineto{\pgfqpoint{4.418680in}{1.243450in}}%
\pgfpathlineto{\pgfqpoint{4.421160in}{1.246073in}}%
\pgfpathlineto{\pgfqpoint{4.424880in}{1.246342in}}%
\pgfpathlineto{\pgfqpoint{4.426120in}{1.246179in}}%
\pgfpathlineto{\pgfqpoint{4.433560in}{1.236863in}}%
\pgfpathlineto{\pgfqpoint{4.434800in}{1.236468in}}%
\pgfpathlineto{\pgfqpoint{4.437280in}{1.237763in}}%
\pgfpathlineto{\pgfqpoint{4.439760in}{1.236401in}}%
\pgfpathlineto{\pgfqpoint{4.441000in}{1.237169in}}%
\pgfpathlineto{\pgfqpoint{4.444720in}{1.242083in}}%
\pgfpathlineto{\pgfqpoint{4.450920in}{1.240354in}}%
\pgfpathlineto{\pgfqpoint{4.453400in}{1.242693in}}%
\pgfpathlineto{\pgfqpoint{4.455880in}{1.239727in}}%
\pgfpathlineto{\pgfqpoint{4.457120in}{1.240011in}}%
\pgfpathlineto{\pgfqpoint{4.459600in}{1.243439in}}%
\pgfpathlineto{\pgfqpoint{4.462080in}{1.242680in}}%
\pgfpathlineto{\pgfqpoint{4.463320in}{1.245149in}}%
\pgfpathlineto{\pgfqpoint{4.464560in}{1.244146in}}%
\pgfpathlineto{\pgfqpoint{4.469520in}{1.247133in}}%
\pgfpathlineto{\pgfqpoint{4.470760in}{1.247383in}}%
\pgfpathlineto{\pgfqpoint{4.472000in}{1.245326in}}%
\pgfpathlineto{\pgfqpoint{4.473240in}{1.246302in}}%
\pgfpathlineto{\pgfqpoint{4.475720in}{1.243657in}}%
\pgfpathlineto{\pgfqpoint{4.478200in}{1.245475in}}%
\pgfpathlineto{\pgfqpoint{4.485640in}{1.246626in}}%
\pgfpathlineto{\pgfqpoint{4.488120in}{1.248384in}}%
\pgfpathlineto{\pgfqpoint{4.493080in}{1.241594in}}%
\pgfpathlineto{\pgfqpoint{4.494320in}{1.241473in}}%
\pgfpathlineto{\pgfqpoint{4.499280in}{1.244806in}}%
\pgfpathlineto{\pgfqpoint{4.500520in}{1.244787in}}%
\pgfpathlineto{\pgfqpoint{4.501760in}{1.242920in}}%
\pgfpathlineto{\pgfqpoint{4.503000in}{1.243551in}}%
\pgfpathlineto{\pgfqpoint{4.505480in}{1.241173in}}%
\pgfpathlineto{\pgfqpoint{4.507960in}{1.242567in}}%
\pgfpathlineto{\pgfqpoint{4.510440in}{1.240738in}}%
\pgfpathlineto{\pgfqpoint{4.512920in}{1.240783in}}%
\pgfpathlineto{\pgfqpoint{4.516640in}{1.238426in}}%
\pgfpathlineto{\pgfqpoint{4.519120in}{1.242020in}}%
\pgfpathlineto{\pgfqpoint{4.520360in}{1.243032in}}%
\pgfpathlineto{\pgfqpoint{4.522840in}{1.239129in}}%
\pgfpathlineto{\pgfqpoint{4.525320in}{1.239463in}}%
\pgfpathlineto{\pgfqpoint{4.527800in}{1.236136in}}%
\pgfpathlineto{\pgfqpoint{4.529040in}{1.238230in}}%
\pgfpathlineto{\pgfqpoint{4.534000in}{1.234371in}}%
\pgfpathlineto{\pgfqpoint{4.535240in}{1.237407in}}%
\pgfpathlineto{\pgfqpoint{4.538960in}{1.236823in}}%
\pgfpathlineto{\pgfqpoint{4.541440in}{1.236532in}}%
\pgfpathlineto{\pgfqpoint{4.542680in}{1.235547in}}%
\pgfpathlineto{\pgfqpoint{4.545160in}{1.237927in}}%
\pgfpathlineto{\pgfqpoint{4.547640in}{1.238292in}}%
\pgfpathlineto{\pgfqpoint{4.552600in}{1.235043in}}%
\pgfpathlineto{\pgfqpoint{4.557560in}{1.229837in}}%
\pgfpathlineto{\pgfqpoint{4.562520in}{1.229150in}}%
\pgfpathlineto{\pgfqpoint{4.563760in}{1.228782in}}%
\pgfpathlineto{\pgfqpoint{4.566240in}{1.231774in}}%
\pgfpathlineto{\pgfqpoint{4.567480in}{1.234246in}}%
\pgfpathlineto{\pgfqpoint{4.571200in}{1.233192in}}%
\pgfpathlineto{\pgfqpoint{4.573680in}{1.234443in}}%
\pgfpathlineto{\pgfqpoint{4.574920in}{1.232133in}}%
\pgfpathlineto{\pgfqpoint{4.577400in}{1.233728in}}%
\pgfpathlineto{\pgfqpoint{4.581120in}{1.230634in}}%
\pgfpathlineto{\pgfqpoint{4.583600in}{1.233638in}}%
\pgfpathlineto{\pgfqpoint{4.586080in}{1.232862in}}%
\pgfpathlineto{\pgfqpoint{4.587320in}{1.235457in}}%
\pgfpathlineto{\pgfqpoint{4.588560in}{1.234337in}}%
\pgfpathlineto{\pgfqpoint{4.594760in}{1.237831in}}%
\pgfpathlineto{\pgfqpoint{4.596000in}{1.235683in}}%
\pgfpathlineto{\pgfqpoint{4.597240in}{1.236851in}}%
\pgfpathlineto{\pgfqpoint{4.599720in}{1.234679in}}%
\pgfpathlineto{\pgfqpoint{4.602200in}{1.236444in}}%
\pgfpathlineto{\pgfqpoint{4.607160in}{1.235524in}}%
\pgfpathlineto{\pgfqpoint{4.610880in}{1.238636in}}%
\pgfpathlineto{\pgfqpoint{4.612120in}{1.239477in}}%
\pgfpathlineto{\pgfqpoint{4.618320in}{1.231727in}}%
\pgfpathlineto{\pgfqpoint{4.623280in}{1.235233in}}%
\pgfpathlineto{\pgfqpoint{4.624520in}{1.235072in}}%
\pgfpathlineto{\pgfqpoint{4.625760in}{1.232616in}}%
\pgfpathlineto{\pgfqpoint{4.627000in}{1.233240in}}%
\pgfpathlineto{\pgfqpoint{4.629480in}{1.231308in}}%
\pgfpathlineto{\pgfqpoint{4.631960in}{1.232901in}}%
\pgfpathlineto{\pgfqpoint{4.634440in}{1.230777in}}%
\pgfpathlineto{\pgfqpoint{4.638160in}{1.230294in}}%
\pgfpathlineto{\pgfqpoint{4.639400in}{1.228439in}}%
\pgfpathlineto{\pgfqpoint{4.640640in}{1.229154in}}%
\pgfpathlineto{\pgfqpoint{4.643120in}{1.232304in}}%
\pgfpathlineto{\pgfqpoint{4.644360in}{1.233083in}}%
\pgfpathlineto{\pgfqpoint{4.646840in}{1.229165in}}%
\pgfpathlineto{\pgfqpoint{4.649320in}{1.229994in}}%
\pgfpathlineto{\pgfqpoint{4.651800in}{1.227252in}}%
\pgfpathlineto{\pgfqpoint{4.653040in}{1.229222in}}%
\pgfpathlineto{\pgfqpoint{4.655520in}{1.227067in}}%
\pgfpathlineto{\pgfqpoint{4.656760in}{1.226917in}}%
\pgfpathlineto{\pgfqpoint{4.658000in}{1.225298in}}%
\pgfpathlineto{\pgfqpoint{4.659240in}{1.228333in}}%
\pgfpathlineto{\pgfqpoint{4.667920in}{1.228852in}}%
\pgfpathlineto{\pgfqpoint{4.670400in}{1.229719in}}%
\pgfpathlineto{\pgfqpoint{4.675360in}{1.226478in}}%
\pgfpathlineto{\pgfqpoint{4.681560in}{1.219319in}}%
\pgfpathlineto{\pgfqpoint{4.685280in}{1.219653in}}%
\pgfpathlineto{\pgfqpoint{4.687760in}{1.218865in}}%
\pgfpathlineto{\pgfqpoint{4.690240in}{1.222484in}}%
\pgfpathlineto{\pgfqpoint{4.691480in}{1.224717in}}%
\pgfpathlineto{\pgfqpoint{4.693960in}{1.224032in}}%
\pgfpathlineto{\pgfqpoint{4.695200in}{1.223947in}}%
\pgfpathlineto{\pgfqpoint{4.697680in}{1.226091in}}%
\pgfpathlineto{\pgfqpoint{4.698920in}{1.224278in}}%
\pgfpathlineto{\pgfqpoint{4.701400in}{1.226166in}}%
\pgfpathlineto{\pgfqpoint{4.703880in}{1.223087in}}%
\pgfpathlineto{\pgfqpoint{4.705120in}{1.223069in}}%
\pgfpathlineto{\pgfqpoint{4.708840in}{1.225702in}}%
\pgfpathlineto{\pgfqpoint{4.710080in}{1.224955in}}%
\pgfpathlineto{\pgfqpoint{4.711320in}{1.227279in}}%
\pgfpathlineto{\pgfqpoint{4.712560in}{1.226300in}}%
\pgfpathlineto{\pgfqpoint{4.718760in}{1.230254in}}%
\pgfpathlineto{\pgfqpoint{4.722480in}{1.225315in}}%
\pgfpathlineto{\pgfqpoint{4.723720in}{1.225190in}}%
\pgfpathlineto{\pgfqpoint{4.726200in}{1.227061in}}%
\pgfpathlineto{\pgfqpoint{4.728680in}{1.227612in}}%
\pgfpathlineto{\pgfqpoint{4.731160in}{1.225940in}}%
\pgfpathlineto{\pgfqpoint{4.734880in}{1.230731in}}%
\pgfpathlineto{\pgfqpoint{4.736120in}{1.231829in}}%
\pgfpathlineto{\pgfqpoint{4.742320in}{1.224482in}}%
\pgfpathlineto{\pgfqpoint{4.748520in}{1.226915in}}%
\pgfpathlineto{\pgfqpoint{4.749760in}{1.224616in}}%
\pgfpathlineto{\pgfqpoint{4.751000in}{1.225012in}}%
\pgfpathlineto{\pgfqpoint{4.753480in}{1.223198in}}%
\pgfpathlineto{\pgfqpoint{4.755960in}{1.225466in}}%
\pgfpathlineto{\pgfqpoint{4.759680in}{1.222746in}}%
\pgfpathlineto{\pgfqpoint{4.762160in}{1.221873in}}%
\pgfpathlineto{\pgfqpoint{4.763400in}{1.219529in}}%
\pgfpathlineto{\pgfqpoint{4.764640in}{1.219885in}}%
\pgfpathlineto{\pgfqpoint{4.767120in}{1.222842in}}%
\pgfpathlineto{\pgfqpoint{4.768360in}{1.223404in}}%
\pgfpathlineto{\pgfqpoint{4.770840in}{1.220259in}}%
\pgfpathlineto{\pgfqpoint{4.773320in}{1.222556in}}%
\pgfpathlineto{\pgfqpoint{4.775800in}{1.220079in}}%
\pgfpathlineto{\pgfqpoint{4.777040in}{1.222507in}}%
\pgfpathlineto{\pgfqpoint{4.782000in}{1.218190in}}%
\pgfpathlineto{\pgfqpoint{4.783240in}{1.220775in}}%
\pgfpathlineto{\pgfqpoint{4.790680in}{1.219900in}}%
\pgfpathlineto{\pgfqpoint{4.793160in}{1.222643in}}%
\pgfpathlineto{\pgfqpoint{4.795640in}{1.222934in}}%
\pgfpathlineto{\pgfqpoint{4.800600in}{1.218617in}}%
\pgfpathlineto{\pgfqpoint{4.805560in}{1.211869in}}%
\pgfpathlineto{\pgfqpoint{4.806800in}{1.212092in}}%
\pgfpathlineto{\pgfqpoint{4.810520in}{1.210572in}}%
\pgfpathlineto{\pgfqpoint{4.811760in}{1.210542in}}%
\pgfpathlineto{\pgfqpoint{4.817960in}{1.215719in}}%
\pgfpathlineto{\pgfqpoint{4.820440in}{1.216432in}}%
\pgfpathlineto{\pgfqpoint{4.821680in}{1.217238in}}%
\pgfpathlineto{\pgfqpoint{4.822920in}{1.215393in}}%
\pgfpathlineto{\pgfqpoint{4.825400in}{1.216771in}}%
\pgfpathlineto{\pgfqpoint{4.829120in}{1.213102in}}%
\pgfpathlineto{\pgfqpoint{4.832840in}{1.215871in}}%
\pgfpathlineto{\pgfqpoint{4.834080in}{1.214866in}}%
\pgfpathlineto{\pgfqpoint{4.835320in}{1.217098in}}%
\pgfpathlineto{\pgfqpoint{4.836560in}{1.216147in}}%
\pgfpathlineto{\pgfqpoint{4.842760in}{1.220425in}}%
\pgfpathlineto{\pgfqpoint{4.844000in}{1.217856in}}%
\pgfpathlineto{\pgfqpoint{4.845240in}{1.217937in}}%
\pgfpathlineto{\pgfqpoint{4.846480in}{1.215431in}}%
\pgfpathlineto{\pgfqpoint{4.848960in}{1.216494in}}%
\pgfpathlineto{\pgfqpoint{4.852680in}{1.217129in}}%
\pgfpathlineto{\pgfqpoint{4.855160in}{1.216214in}}%
\pgfpathlineto{\pgfqpoint{4.858880in}{1.221218in}}%
\pgfpathlineto{\pgfqpoint{4.860120in}{1.222210in}}%
\pgfpathlineto{\pgfqpoint{4.866320in}{1.215819in}}%
\pgfpathlineto{\pgfqpoint{4.867560in}{1.216227in}}%
\pgfpathlineto{\pgfqpoint{4.870040in}{1.218450in}}%
\pgfpathlineto{\pgfqpoint{4.872520in}{1.217824in}}%
\pgfpathlineto{\pgfqpoint{4.873760in}{1.215634in}}%
\pgfpathlineto{\pgfqpoint{4.875000in}{1.216040in}}%
\pgfpathlineto{\pgfqpoint{4.877480in}{1.214638in}}%
\pgfpathlineto{\pgfqpoint{4.879960in}{1.217197in}}%
\pgfpathlineto{\pgfqpoint{4.886160in}{1.212994in}}%
\pgfpathlineto{\pgfqpoint{4.887400in}{1.210420in}}%
\pgfpathlineto{\pgfqpoint{4.888640in}{1.210969in}}%
\pgfpathlineto{\pgfqpoint{4.892360in}{1.215799in}}%
\pgfpathlineto{\pgfqpoint{4.894840in}{1.213174in}}%
\pgfpathlineto{\pgfqpoint{4.897320in}{1.215825in}}%
\pgfpathlineto{\pgfqpoint{4.899800in}{1.212343in}}%
\pgfpathlineto{\pgfqpoint{4.901040in}{1.214696in}}%
\pgfpathlineto{\pgfqpoint{4.907240in}{1.213456in}}%
\pgfpathlineto{\pgfqpoint{4.913440in}{1.212377in}}%
\pgfpathlineto{\pgfqpoint{4.914680in}{1.211677in}}%
\pgfpathlineto{\pgfqpoint{4.918400in}{1.215509in}}%
\pgfpathlineto{\pgfqpoint{4.920880in}{1.215185in}}%
\pgfpathlineto{\pgfqpoint{4.922120in}{1.214714in}}%
\pgfpathlineto{\pgfqpoint{4.925840in}{1.208838in}}%
\pgfpathlineto{\pgfqpoint{4.932040in}{1.203346in}}%
\pgfpathlineto{\pgfqpoint{4.937000in}{1.204777in}}%
\pgfpathlineto{\pgfqpoint{4.940720in}{1.207964in}}%
\pgfpathlineto{\pgfqpoint{4.944440in}{1.207272in}}%
\pgfpathlineto{\pgfqpoint{4.945680in}{1.207856in}}%
\pgfpathlineto{\pgfqpoint{4.946920in}{1.206398in}}%
\pgfpathlineto{\pgfqpoint{4.949400in}{1.208354in}}%
\pgfpathlineto{\pgfqpoint{4.951880in}{1.205261in}}%
\pgfpathlineto{\pgfqpoint{4.953120in}{1.205055in}}%
\pgfpathlineto{\pgfqpoint{4.955600in}{1.208039in}}%
\pgfpathlineto{\pgfqpoint{4.958080in}{1.204782in}}%
\pgfpathlineto{\pgfqpoint{4.959320in}{1.206882in}}%
\pgfpathlineto{\pgfqpoint{4.961800in}{1.206560in}}%
\pgfpathlineto{\pgfqpoint{4.964280in}{1.209391in}}%
\pgfpathlineto{\pgfqpoint{4.966760in}{1.210193in}}%
\pgfpathlineto{\pgfqpoint{4.970480in}{1.204030in}}%
\pgfpathlineto{\pgfqpoint{4.975440in}{1.205012in}}%
\pgfpathlineto{\pgfqpoint{4.976680in}{1.205654in}}%
\pgfpathlineto{\pgfqpoint{4.979160in}{1.205108in}}%
\pgfpathlineto{\pgfqpoint{4.982880in}{1.209840in}}%
\pgfpathlineto{\pgfqpoint{4.984120in}{1.211005in}}%
\pgfpathlineto{\pgfqpoint{4.989080in}{1.203359in}}%
\pgfpathlineto{\pgfqpoint{4.990320in}{1.204502in}}%
\pgfpathlineto{\pgfqpoint{4.996520in}{1.207761in}}%
\pgfpathlineto{\pgfqpoint{4.997760in}{1.205964in}}%
\pgfpathlineto{\pgfqpoint{4.999000in}{1.206560in}}%
\pgfpathlineto{\pgfqpoint{5.001480in}{1.204841in}}%
\pgfpathlineto{\pgfqpoint{5.003960in}{1.208072in}}%
\pgfpathlineto{\pgfqpoint{5.010160in}{1.204279in}}%
\pgfpathlineto{\pgfqpoint{5.011400in}{1.201605in}}%
\pgfpathlineto{\pgfqpoint{5.013880in}{1.204151in}}%
\pgfpathlineto{\pgfqpoint{5.016360in}{1.206665in}}%
\pgfpathlineto{\pgfqpoint{5.018840in}{1.202860in}}%
\pgfpathlineto{\pgfqpoint{5.021320in}{1.206382in}}%
\pgfpathlineto{\pgfqpoint{5.023800in}{1.203256in}}%
\pgfpathlineto{\pgfqpoint{5.025040in}{1.205821in}}%
\pgfpathlineto{\pgfqpoint{5.027520in}{1.206268in}}%
\pgfpathlineto{\pgfqpoint{5.028760in}{1.206480in}}%
\pgfpathlineto{\pgfqpoint{5.030000in}{1.204043in}}%
\pgfpathlineto{\pgfqpoint{5.032480in}{1.206429in}}%
\pgfpathlineto{\pgfqpoint{5.034960in}{1.206069in}}%
\pgfpathlineto{\pgfqpoint{5.038680in}{1.204561in}}%
\pgfpathlineto{\pgfqpoint{5.043640in}{1.209125in}}%
\pgfpathlineto{\pgfqpoint{5.046120in}{1.208545in}}%
\pgfpathlineto{\pgfqpoint{5.049840in}{1.201936in}}%
\pgfpathlineto{\pgfqpoint{5.053560in}{1.196687in}}%
\pgfpathlineto{\pgfqpoint{5.054800in}{1.197691in}}%
\pgfpathlineto{\pgfqpoint{5.056040in}{1.195992in}}%
\pgfpathlineto{\pgfqpoint{5.057280in}{1.196510in}}%
\pgfpathlineto{\pgfqpoint{5.058520in}{1.195760in}}%
\pgfpathlineto{\pgfqpoint{5.062240in}{1.198721in}}%
\pgfpathlineto{\pgfqpoint{5.064720in}{1.200826in}}%
\pgfpathlineto{\pgfqpoint{5.068440in}{1.200709in}}%
\pgfpathlineto{\pgfqpoint{5.069680in}{1.201043in}}%
\pgfpathlineto{\pgfqpoint{5.070920in}{1.199622in}}%
\pgfpathlineto{\pgfqpoint{5.073400in}{1.200557in}}%
\pgfpathlineto{\pgfqpoint{5.077120in}{1.197658in}}%
\pgfpathlineto{\pgfqpoint{5.079600in}{1.200364in}}%
\pgfpathlineto{\pgfqpoint{5.082080in}{1.197208in}}%
\pgfpathlineto{\pgfqpoint{5.083320in}{1.198534in}}%
\pgfpathlineto{\pgfqpoint{5.085800in}{1.197887in}}%
\pgfpathlineto{\pgfqpoint{5.089520in}{1.201675in}}%
\pgfpathlineto{\pgfqpoint{5.090760in}{1.201540in}}%
\pgfpathlineto{\pgfqpoint{5.094480in}{1.196674in}}%
\pgfpathlineto{\pgfqpoint{5.098200in}{1.197057in}}%
\pgfpathlineto{\pgfqpoint{5.103160in}{1.197497in}}%
\pgfpathlineto{\pgfqpoint{5.106880in}{1.202844in}}%
\pgfpathlineto{\pgfqpoint{5.108120in}{1.204011in}}%
\pgfpathlineto{\pgfqpoint{5.110600in}{1.201694in}}%
\pgfpathlineto{\pgfqpoint{5.113080in}{1.196722in}}%
\pgfpathlineto{\pgfqpoint{5.116800in}{1.201139in}}%
\pgfpathlineto{\pgfqpoint{5.120520in}{1.202439in}}%
\pgfpathlineto{\pgfqpoint{5.121760in}{1.200913in}}%
\pgfpathlineto{\pgfqpoint{5.123000in}{1.201287in}}%
\pgfpathlineto{\pgfqpoint{5.125480in}{1.198382in}}%
\pgfpathlineto{\pgfqpoint{5.127960in}{1.200570in}}%
\pgfpathlineto{\pgfqpoint{5.134160in}{1.198513in}}%
\pgfpathlineto{\pgfqpoint{5.135400in}{1.195799in}}%
\pgfpathlineto{\pgfqpoint{5.139120in}{1.199435in}}%
\pgfpathlineto{\pgfqpoint{5.140360in}{1.200681in}}%
\pgfpathlineto{\pgfqpoint{5.142840in}{1.197052in}}%
\pgfpathlineto{\pgfqpoint{5.145320in}{1.200287in}}%
\pgfpathlineto{\pgfqpoint{5.147800in}{1.197323in}}%
\pgfpathlineto{\pgfqpoint{5.149040in}{1.199989in}}%
\pgfpathlineto{\pgfqpoint{5.151520in}{1.199035in}}%
\pgfpathlineto{\pgfqpoint{5.152760in}{1.199321in}}%
\pgfpathlineto{\pgfqpoint{5.154000in}{1.197451in}}%
\pgfpathlineto{\pgfqpoint{5.155240in}{1.200627in}}%
\pgfpathlineto{\pgfqpoint{5.158960in}{1.201394in}}%
\pgfpathlineto{\pgfqpoint{5.161440in}{1.200772in}}%
\pgfpathlineto{\pgfqpoint{5.162680in}{1.200343in}}%
\pgfpathlineto{\pgfqpoint{5.167640in}{1.203777in}}%
\pgfpathlineto{\pgfqpoint{5.170120in}{1.204274in}}%
\pgfpathlineto{\pgfqpoint{5.176320in}{1.191998in}}%
\pgfpathlineto{\pgfqpoint{5.177560in}{1.191410in}}%
\pgfpathlineto{\pgfqpoint{5.178800in}{1.192249in}}%
\pgfpathlineto{\pgfqpoint{5.180040in}{1.190520in}}%
\pgfpathlineto{\pgfqpoint{5.181280in}{1.191369in}}%
\pgfpathlineto{\pgfqpoint{5.183760in}{1.191322in}}%
\pgfpathlineto{\pgfqpoint{5.186240in}{1.192525in}}%
\pgfpathlineto{\pgfqpoint{5.187480in}{1.195248in}}%
\pgfpathlineto{\pgfqpoint{5.192440in}{1.193711in}}%
\pgfpathlineto{\pgfqpoint{5.193680in}{1.194596in}}%
\pgfpathlineto{\pgfqpoint{5.194920in}{1.193473in}}%
\pgfpathlineto{\pgfqpoint{5.197400in}{1.194577in}}%
\pgfpathlineto{\pgfqpoint{5.199880in}{1.192152in}}%
\pgfpathlineto{\pgfqpoint{5.202360in}{1.193580in}}%
\pgfpathlineto{\pgfqpoint{5.203600in}{1.195026in}}%
\pgfpathlineto{\pgfqpoint{5.208560in}{1.192206in}}%
\pgfpathlineto{\pgfqpoint{5.209800in}{1.192935in}}%
\pgfpathlineto{\pgfqpoint{5.213520in}{1.197068in}}%
\pgfpathlineto{\pgfqpoint{5.214760in}{1.197203in}}%
\pgfpathlineto{\pgfqpoint{5.217240in}{1.194476in}}%
\pgfpathlineto{\pgfqpoint{5.218480in}{1.192341in}}%
\pgfpathlineto{\pgfqpoint{5.222200in}{1.193381in}}%
\pgfpathlineto{\pgfqpoint{5.225920in}{1.192346in}}%
\pgfpathlineto{\pgfqpoint{5.227160in}{1.193295in}}%
\pgfpathlineto{\pgfqpoint{5.230880in}{1.199063in}}%
\pgfpathlineto{\pgfqpoint{5.232120in}{1.199957in}}%
\pgfpathlineto{\pgfqpoint{5.234600in}{1.197088in}}%
\pgfpathlineto{\pgfqpoint{5.237080in}{1.191026in}}%
\pgfpathlineto{\pgfqpoint{5.242040in}{1.197306in}}%
\pgfpathlineto{\pgfqpoint{5.243280in}{1.196513in}}%
\pgfpathlineto{\pgfqpoint{5.244520in}{1.197341in}}%
\pgfpathlineto{\pgfqpoint{5.245760in}{1.196318in}}%
\pgfpathlineto{\pgfqpoint{5.247000in}{1.196931in}}%
\pgfpathlineto{\pgfqpoint{5.249480in}{1.194013in}}%
\pgfpathlineto{\pgfqpoint{5.251960in}{1.197512in}}%
\pgfpathlineto{\pgfqpoint{5.253200in}{1.197320in}}%
\pgfpathlineto{\pgfqpoint{5.255680in}{1.196137in}}%
\pgfpathlineto{\pgfqpoint{5.258160in}{1.194983in}}%
\pgfpathlineto{\pgfqpoint{5.259400in}{1.191929in}}%
\pgfpathlineto{\pgfqpoint{5.264360in}{1.197014in}}%
\pgfpathlineto{\pgfqpoint{5.266840in}{1.192369in}}%
\pgfpathlineto{\pgfqpoint{5.269320in}{1.194819in}}%
\pgfpathlineto{\pgfqpoint{5.271800in}{1.192628in}}%
\pgfpathlineto{\pgfqpoint{5.273040in}{1.195447in}}%
\pgfpathlineto{\pgfqpoint{5.275520in}{1.194689in}}%
\pgfpathlineto{\pgfqpoint{5.276760in}{1.195087in}}%
\pgfpathlineto{\pgfqpoint{5.278000in}{1.192994in}}%
\pgfpathlineto{\pgfqpoint{5.279240in}{1.196797in}}%
\pgfpathlineto{\pgfqpoint{5.282960in}{1.197620in}}%
\pgfpathlineto{\pgfqpoint{5.284200in}{1.198668in}}%
\pgfpathlineto{\pgfqpoint{5.286680in}{1.197905in}}%
\pgfpathlineto{\pgfqpoint{5.294120in}{1.200568in}}%
\pgfpathlineto{\pgfqpoint{5.299080in}{1.192722in}}%
\pgfpathlineto{\pgfqpoint{5.300320in}{1.188146in}}%
\pgfpathlineto{\pgfqpoint{5.310240in}{1.190844in}}%
\pgfpathlineto{\pgfqpoint{5.312720in}{1.193173in}}%
\pgfpathlineto{\pgfqpoint{5.321400in}{1.194318in}}%
\pgfpathlineto{\pgfqpoint{5.323880in}{1.192305in}}%
\pgfpathlineto{\pgfqpoint{5.328840in}{1.193603in}}%
\pgfpathlineto{\pgfqpoint{5.330080in}{1.190790in}}%
\pgfpathlineto{\pgfqpoint{5.331320in}{1.192185in}}%
\pgfpathlineto{\pgfqpoint{5.332560in}{1.191790in}}%
\pgfpathlineto{\pgfqpoint{5.337520in}{1.195031in}}%
\pgfpathlineto{\pgfqpoint{5.340000in}{1.193627in}}%
\pgfpathlineto{\pgfqpoint{5.342480in}{1.186982in}}%
\pgfpathlineto{\pgfqpoint{5.351160in}{1.189190in}}%
\pgfpathlineto{\pgfqpoint{5.354880in}{1.194482in}}%
\pgfpathlineto{\pgfqpoint{5.357360in}{1.194586in}}%
\pgfpathlineto{\pgfqpoint{5.359840in}{1.189853in}}%
\pgfpathlineto{\pgfqpoint{5.361080in}{1.186465in}}%
\pgfpathlineto{\pgfqpoint{5.364800in}{1.191541in}}%
\pgfpathlineto{\pgfqpoint{5.371000in}{1.191968in}}%
\pgfpathlineto{\pgfqpoint{5.373480in}{1.188362in}}%
\pgfpathlineto{\pgfqpoint{5.375960in}{1.192439in}}%
\pgfpathlineto{\pgfqpoint{5.379680in}{1.191950in}}%
\pgfpathlineto{\pgfqpoint{5.382160in}{1.191790in}}%
\pgfpathlineto{\pgfqpoint{5.383400in}{1.189378in}}%
\pgfpathlineto{\pgfqpoint{5.388360in}{1.193698in}}%
\pgfpathlineto{\pgfqpoint{5.390840in}{1.189069in}}%
\pgfpathlineto{\pgfqpoint{5.393320in}{1.191493in}}%
\pgfpathlineto{\pgfqpoint{5.395800in}{1.188964in}}%
\pgfpathlineto{\pgfqpoint{5.397040in}{1.191489in}}%
\pgfpathlineto{\pgfqpoint{5.398280in}{1.189997in}}%
\pgfpathlineto{\pgfqpoint{5.400760in}{1.190935in}}%
\pgfpathlineto{\pgfqpoint{5.402000in}{1.189159in}}%
\pgfpathlineto{\pgfqpoint{5.403240in}{1.193385in}}%
\pgfpathlineto{\pgfqpoint{5.405720in}{1.193648in}}%
\pgfpathlineto{\pgfqpoint{5.409440in}{1.194398in}}%
\pgfpathlineto{\pgfqpoint{5.414400in}{1.195858in}}%
\pgfpathlineto{\pgfqpoint{5.419360in}{1.194807in}}%
\pgfpathlineto{\pgfqpoint{5.421840in}{1.191329in}}%
\pgfpathlineto{\pgfqpoint{5.425560in}{1.183846in}}%
\pgfpathlineto{\pgfqpoint{5.426800in}{1.184571in}}%
\pgfpathlineto{\pgfqpoint{5.428040in}{1.183239in}}%
\pgfpathlineto{\pgfqpoint{5.430520in}{1.184059in}}%
\pgfpathlineto{\pgfqpoint{5.441680in}{1.188515in}}%
\pgfpathlineto{\pgfqpoint{5.442920in}{1.187291in}}%
\pgfpathlineto{\pgfqpoint{5.445400in}{1.188285in}}%
\pgfpathlineto{\pgfqpoint{5.447880in}{1.185394in}}%
\pgfpathlineto{\pgfqpoint{5.451600in}{1.187007in}}%
\pgfpathlineto{\pgfqpoint{5.452840in}{1.185535in}}%
\pgfpathlineto{\pgfqpoint{5.454080in}{1.182089in}}%
\pgfpathlineto{\pgfqpoint{5.456560in}{1.183707in}}%
\pgfpathlineto{\pgfqpoint{5.461520in}{1.186827in}}%
\pgfpathlineto{\pgfqpoint{5.464000in}{1.185449in}}%
\pgfpathlineto{\pgfqpoint{5.467720in}{1.182021in}}%
\pgfpathlineto{\pgfqpoint{5.468960in}{1.182798in}}%
\pgfpathlineto{\pgfqpoint{5.471440in}{1.182112in}}%
\pgfpathlineto{\pgfqpoint{5.475160in}{1.182848in}}%
\pgfpathlineto{\pgfqpoint{5.480120in}{1.189923in}}%
\pgfpathlineto{\pgfqpoint{5.482600in}{1.187834in}}%
\pgfpathlineto{\pgfqpoint{5.485080in}{1.181700in}}%
\pgfpathlineto{\pgfqpoint{5.487560in}{1.185303in}}%
\pgfpathlineto{\pgfqpoint{5.488800in}{1.187021in}}%
\pgfpathlineto{\pgfqpoint{5.491280in}{1.186590in}}%
\pgfpathlineto{\pgfqpoint{5.492520in}{1.188363in}}%
\pgfpathlineto{\pgfqpoint{5.493760in}{1.186984in}}%
\pgfpathlineto{\pgfqpoint{5.495000in}{1.187640in}}%
\pgfpathlineto{\pgfqpoint{5.497480in}{1.184773in}}%
\pgfpathlineto{\pgfqpoint{5.499960in}{1.189434in}}%
\pgfpathlineto{\pgfqpoint{5.501200in}{1.189846in}}%
\pgfpathlineto{\pgfqpoint{5.502440in}{1.188766in}}%
\pgfpathlineto{\pgfqpoint{5.503680in}{1.189309in}}%
\pgfpathlineto{\pgfqpoint{5.504920in}{1.188159in}}%
\pgfpathlineto{\pgfqpoint{5.506160in}{1.188916in}}%
\pgfpathlineto{\pgfqpoint{5.507400in}{1.186890in}}%
\pgfpathlineto{\pgfqpoint{5.513600in}{1.188860in}}%
\pgfpathlineto{\pgfqpoint{5.514840in}{1.185762in}}%
\pgfpathlineto{\pgfqpoint{5.517320in}{1.188517in}}%
\pgfpathlineto{\pgfqpoint{5.519800in}{1.185348in}}%
\pgfpathlineto{\pgfqpoint{5.521040in}{1.187454in}}%
\pgfpathlineto{\pgfqpoint{5.523520in}{1.186918in}}%
\pgfpathlineto{\pgfqpoint{5.524760in}{1.187982in}}%
\pgfpathlineto{\pgfqpoint{5.526000in}{1.185838in}}%
\pgfpathlineto{\pgfqpoint{5.527240in}{1.189707in}}%
\pgfpathlineto{\pgfqpoint{5.530960in}{1.189038in}}%
\pgfpathlineto{\pgfqpoint{5.533440in}{1.189688in}}%
\pgfpathlineto{\pgfqpoint{5.538400in}{1.192420in}}%
\pgfpathlineto{\pgfqpoint{5.540880in}{1.192583in}}%
\pgfpathlineto{\pgfqpoint{5.542120in}{1.193494in}}%
\pgfpathlineto{\pgfqpoint{5.545840in}{1.187808in}}%
\pgfpathlineto{\pgfqpoint{5.548320in}{1.180153in}}%
\pgfpathlineto{\pgfqpoint{5.550800in}{1.180607in}}%
\pgfpathlineto{\pgfqpoint{5.552040in}{1.178734in}}%
\pgfpathlineto{\pgfqpoint{5.554520in}{1.179850in}}%
\pgfpathlineto{\pgfqpoint{5.558240in}{1.180233in}}%
\pgfpathlineto{\pgfqpoint{5.559480in}{1.182061in}}%
\pgfpathlineto{\pgfqpoint{5.561960in}{1.181747in}}%
\pgfpathlineto{\pgfqpoint{5.564440in}{1.183492in}}%
\pgfpathlineto{\pgfqpoint{5.565680in}{1.184404in}}%
\pgfpathlineto{\pgfqpoint{5.566920in}{1.182516in}}%
\pgfpathlineto{\pgfqpoint{5.569400in}{1.184467in}}%
\pgfpathlineto{\pgfqpoint{5.570640in}{1.182615in}}%
\pgfpathlineto{\pgfqpoint{5.575600in}{1.185945in}}%
\pgfpathlineto{\pgfqpoint{5.578080in}{1.181860in}}%
\pgfpathlineto{\pgfqpoint{5.579320in}{1.183367in}}%
\pgfpathlineto{\pgfqpoint{5.580560in}{1.182774in}}%
\pgfpathlineto{\pgfqpoint{5.585520in}{1.185255in}}%
\pgfpathlineto{\pgfqpoint{5.588000in}{1.183621in}}%
\pgfpathlineto{\pgfqpoint{5.589240in}{1.186923in}}%
\pgfpathlineto{\pgfqpoint{5.590480in}{1.184227in}}%
\pgfpathlineto{\pgfqpoint{5.594200in}{1.185091in}}%
\pgfpathlineto{\pgfqpoint{5.597920in}{1.185819in}}%
\pgfpathlineto{\pgfqpoint{5.600400in}{1.187557in}}%
\pgfpathlineto{\pgfqpoint{5.604120in}{1.190419in}}%
\pgfpathlineto{\pgfqpoint{5.606600in}{1.188467in}}%
\pgfpathlineto{\pgfqpoint{5.609080in}{1.181775in}}%
\pgfpathlineto{\pgfqpoint{5.610320in}{1.184843in}}%
\pgfpathlineto{\pgfqpoint{5.611560in}{1.184627in}}%
\pgfpathlineto{\pgfqpoint{5.612800in}{1.186262in}}%
\pgfpathlineto{\pgfqpoint{5.615280in}{1.185770in}}%
\pgfpathlineto{\pgfqpoint{5.616520in}{1.186671in}}%
\pgfpathlineto{\pgfqpoint{5.617760in}{1.184976in}}%
\pgfpathlineto{\pgfqpoint{5.619000in}{1.186186in}}%
\pgfpathlineto{\pgfqpoint{5.621480in}{1.183862in}}%
\pgfpathlineto{\pgfqpoint{5.623960in}{1.188184in}}%
\pgfpathlineto{\pgfqpoint{5.625200in}{1.188621in}}%
\pgfpathlineto{\pgfqpoint{5.626440in}{1.186929in}}%
\pgfpathlineto{\pgfqpoint{5.627680in}{1.188594in}}%
\pgfpathlineto{\pgfqpoint{5.628920in}{1.188097in}}%
\pgfpathlineto{\pgfqpoint{5.630160in}{1.188903in}}%
\pgfpathlineto{\pgfqpoint{5.631400in}{1.187451in}}%
\pgfpathlineto{\pgfqpoint{5.636360in}{1.188843in}}%
\pgfpathlineto{\pgfqpoint{5.640080in}{1.185305in}}%
\pgfpathlineto{\pgfqpoint{5.641320in}{1.187063in}}%
\pgfpathlineto{\pgfqpoint{5.642560in}{1.183746in}}%
\pgfpathlineto{\pgfqpoint{5.643800in}{1.184417in}}%
\pgfpathlineto{\pgfqpoint{5.645040in}{1.186999in}}%
\pgfpathlineto{\pgfqpoint{5.648760in}{1.186863in}}%
\pgfpathlineto{\pgfqpoint{5.650000in}{1.184732in}}%
\pgfpathlineto{\pgfqpoint{5.651240in}{1.189195in}}%
\pgfpathlineto{\pgfqpoint{5.652480in}{1.188693in}}%
\pgfpathlineto{\pgfqpoint{5.656200in}{1.190784in}}%
\pgfpathlineto{\pgfqpoint{5.658680in}{1.191328in}}%
\pgfpathlineto{\pgfqpoint{5.661160in}{1.192690in}}%
\pgfpathlineto{\pgfqpoint{5.667360in}{1.192471in}}%
\pgfpathlineto{\pgfqpoint{5.671080in}{1.183511in}}%
\pgfpathlineto{\pgfqpoint{5.672320in}{1.180400in}}%
\pgfpathlineto{\pgfqpoint{5.674800in}{1.180356in}}%
\pgfpathlineto{\pgfqpoint{5.676040in}{1.178395in}}%
\pgfpathlineto{\pgfqpoint{5.679760in}{1.179811in}}%
\pgfpathlineto{\pgfqpoint{5.689680in}{1.182756in}}%
\pgfpathlineto{\pgfqpoint{5.692160in}{1.180334in}}%
\pgfpathlineto{\pgfqpoint{5.693400in}{1.182422in}}%
\pgfpathlineto{\pgfqpoint{5.694640in}{1.180397in}}%
\pgfpathlineto{\pgfqpoint{5.697120in}{1.182296in}}%
\pgfpathlineto{\pgfqpoint{5.698360in}{1.183169in}}%
\pgfpathlineto{\pgfqpoint{5.700840in}{1.180486in}}%
\pgfpathlineto{\pgfqpoint{5.702080in}{1.177691in}}%
\pgfpathlineto{\pgfqpoint{5.703320in}{1.178870in}}%
\pgfpathlineto{\pgfqpoint{5.704560in}{1.177407in}}%
\pgfpathlineto{\pgfqpoint{5.708280in}{1.179220in}}%
\pgfpathlineto{\pgfqpoint{5.713240in}{1.181228in}}%
\pgfpathlineto{\pgfqpoint{5.715720in}{1.178223in}}%
\pgfpathlineto{\pgfqpoint{5.718200in}{1.179621in}}%
\pgfpathlineto{\pgfqpoint{5.725640in}{1.182277in}}%
\pgfpathlineto{\pgfqpoint{5.728120in}{1.183252in}}%
\pgfpathlineto{\pgfqpoint{5.729360in}{1.183676in}}%
\pgfpathlineto{\pgfqpoint{5.730600in}{1.182729in}}%
\pgfpathlineto{\pgfqpoint{5.733080in}{1.175598in}}%
\pgfpathlineto{\pgfqpoint{5.735560in}{1.179741in}}%
\pgfpathlineto{\pgfqpoint{5.736800in}{1.181622in}}%
\pgfpathlineto{\pgfqpoint{5.739280in}{1.180967in}}%
\pgfpathlineto{\pgfqpoint{5.740520in}{1.181916in}}%
\pgfpathlineto{\pgfqpoint{5.741760in}{1.180164in}}%
\pgfpathlineto{\pgfqpoint{5.743000in}{1.181311in}}%
\pgfpathlineto{\pgfqpoint{5.745480in}{1.179470in}}%
\pgfpathlineto{\pgfqpoint{5.747960in}{1.182928in}}%
\pgfpathlineto{\pgfqpoint{5.749200in}{1.183301in}}%
\pgfpathlineto{\pgfqpoint{5.750440in}{1.181484in}}%
\pgfpathlineto{\pgfqpoint{5.751680in}{1.182785in}}%
\pgfpathlineto{\pgfqpoint{5.752920in}{1.182151in}}%
\pgfpathlineto{\pgfqpoint{5.754160in}{1.183606in}}%
\pgfpathlineto{\pgfqpoint{5.756640in}{1.183687in}}%
\pgfpathlineto{\pgfqpoint{5.760360in}{1.184479in}}%
\pgfpathlineto{\pgfqpoint{5.764080in}{1.180113in}}%
\pgfpathlineto{\pgfqpoint{5.765320in}{1.181411in}}%
\pgfpathlineto{\pgfqpoint{5.766560in}{1.178321in}}%
\pgfpathlineto{\pgfqpoint{5.769040in}{1.183024in}}%
\pgfpathlineto{\pgfqpoint{5.770280in}{1.182120in}}%
\pgfpathlineto{\pgfqpoint{5.771520in}{1.181171in}}%
\pgfpathlineto{\pgfqpoint{5.772760in}{1.182181in}}%
\pgfpathlineto{\pgfqpoint{5.774000in}{1.180077in}}%
\pgfpathlineto{\pgfqpoint{5.776480in}{1.184774in}}%
\pgfpathlineto{\pgfqpoint{5.780200in}{1.188098in}}%
\pgfpathlineto{\pgfqpoint{5.782680in}{1.190436in}}%
\pgfpathlineto{\pgfqpoint{5.785160in}{1.192429in}}%
\pgfpathlineto{\pgfqpoint{5.786400in}{1.193377in}}%
\pgfpathlineto{\pgfqpoint{5.787640in}{1.192831in}}%
\pgfpathlineto{\pgfqpoint{5.790120in}{1.195036in}}%
\pgfpathlineto{\pgfqpoint{5.792600in}{1.191489in}}%
\pgfpathlineto{\pgfqpoint{5.795080in}{1.184757in}}%
\pgfpathlineto{\pgfqpoint{5.796320in}{1.182104in}}%
\pgfpathlineto{\pgfqpoint{5.812440in}{1.182123in}}%
\pgfpathlineto{\pgfqpoint{5.813680in}{1.183551in}}%
\pgfpathlineto{\pgfqpoint{5.816160in}{1.181488in}}%
\pgfpathlineto{\pgfqpoint{5.817400in}{1.183700in}}%
\pgfpathlineto{\pgfqpoint{5.818640in}{1.181223in}}%
\pgfpathlineto{\pgfqpoint{5.819880in}{1.181321in}}%
\pgfpathlineto{\pgfqpoint{5.822360in}{1.184019in}}%
\pgfpathlineto{\pgfqpoint{5.823600in}{1.184100in}}%
\pgfpathlineto{\pgfqpoint{5.826080in}{1.179329in}}%
\pgfpathlineto{\pgfqpoint{5.827320in}{1.180847in}}%
\pgfpathlineto{\pgfqpoint{5.828560in}{1.179603in}}%
\pgfpathlineto{\pgfqpoint{5.831040in}{1.180814in}}%
\pgfpathlineto{\pgfqpoint{5.836000in}{1.179289in}}%
\pgfpathlineto{\pgfqpoint{5.837240in}{1.182214in}}%
\pgfpathlineto{\pgfqpoint{5.838480in}{1.179919in}}%
\pgfpathlineto{\pgfqpoint{5.839720in}{1.180305in}}%
\pgfpathlineto{\pgfqpoint{5.842200in}{1.182636in}}%
\pgfpathlineto{\pgfqpoint{5.844680in}{1.182910in}}%
\pgfpathlineto{\pgfqpoint{5.848400in}{1.183258in}}%
\pgfpathlineto{\pgfqpoint{5.853360in}{1.185435in}}%
\pgfpathlineto{\pgfqpoint{5.855840in}{1.181012in}}%
\pgfpathlineto{\pgfqpoint{5.857080in}{1.176529in}}%
\pgfpathlineto{\pgfqpoint{5.859560in}{1.179421in}}%
\pgfpathlineto{\pgfqpoint{5.860800in}{1.181622in}}%
\pgfpathlineto{\pgfqpoint{5.863280in}{1.181407in}}%
\pgfpathlineto{\pgfqpoint{5.864520in}{1.183077in}}%
\pgfpathlineto{\pgfqpoint{5.865760in}{1.181218in}}%
\pgfpathlineto{\pgfqpoint{5.867000in}{1.181912in}}%
\pgfpathlineto{\pgfqpoint{5.869480in}{1.179857in}}%
\pgfpathlineto{\pgfqpoint{5.873200in}{1.183575in}}%
\pgfpathlineto{\pgfqpoint{5.874440in}{1.182500in}}%
\pgfpathlineto{\pgfqpoint{5.875680in}{1.183259in}}%
\pgfpathlineto{\pgfqpoint{5.876920in}{1.181578in}}%
\pgfpathlineto{\pgfqpoint{5.879400in}{1.182484in}}%
\pgfpathlineto{\pgfqpoint{5.883120in}{1.180952in}}%
\pgfpathlineto{\pgfqpoint{5.884360in}{1.181525in}}%
\pgfpathlineto{\pgfqpoint{5.886840in}{1.176293in}}%
\pgfpathlineto{\pgfqpoint{5.889320in}{1.178738in}}%
\pgfpathlineto{\pgfqpoint{5.890560in}{1.175930in}}%
\pgfpathlineto{\pgfqpoint{5.893040in}{1.180765in}}%
\pgfpathlineto{\pgfqpoint{5.895520in}{1.179301in}}%
\pgfpathlineto{\pgfqpoint{5.896760in}{1.180906in}}%
\pgfpathlineto{\pgfqpoint{5.898000in}{1.178590in}}%
\pgfpathlineto{\pgfqpoint{5.900480in}{1.184705in}}%
\pgfpathlineto{\pgfqpoint{5.904200in}{1.188532in}}%
\pgfpathlineto{\pgfqpoint{5.905440in}{1.188335in}}%
\pgfpathlineto{\pgfqpoint{5.910400in}{1.193983in}}%
\pgfpathlineto{\pgfqpoint{5.912880in}{1.193879in}}%
\pgfpathlineto{\pgfqpoint{5.914120in}{1.195443in}}%
\pgfpathlineto{\pgfqpoint{5.917840in}{1.187916in}}%
\pgfpathlineto{\pgfqpoint{5.920320in}{1.181830in}}%
\pgfpathlineto{\pgfqpoint{5.922800in}{1.181576in}}%
\pgfpathlineto{\pgfqpoint{5.931480in}{1.181974in}}%
\pgfpathlineto{\pgfqpoint{5.933960in}{1.180344in}}%
\pgfpathlineto{\pgfqpoint{5.938920in}{1.182046in}}%
\pgfpathlineto{\pgfqpoint{5.940160in}{1.182048in}}%
\pgfpathlineto{\pgfqpoint{5.941400in}{1.184402in}}%
\pgfpathlineto{\pgfqpoint{5.943880in}{1.181625in}}%
\pgfpathlineto{\pgfqpoint{5.947600in}{1.184045in}}%
\pgfpathlineto{\pgfqpoint{5.950080in}{1.179559in}}%
\pgfpathlineto{\pgfqpoint{5.951320in}{1.181640in}}%
\pgfpathlineto{\pgfqpoint{5.952560in}{1.179970in}}%
\pgfpathlineto{\pgfqpoint{5.955040in}{1.180631in}}%
\pgfpathlineto{\pgfqpoint{5.960000in}{1.178893in}}%
\pgfpathlineto{\pgfqpoint{5.961240in}{1.184692in}}%
\pgfpathlineto{\pgfqpoint{5.962480in}{1.182998in}}%
\pgfpathlineto{\pgfqpoint{5.963720in}{1.183611in}}%
\pgfpathlineto{\pgfqpoint{5.967440in}{1.188090in}}%
\pgfpathlineto{\pgfqpoint{5.972400in}{1.186747in}}%
\pgfpathlineto{\pgfqpoint{5.977360in}{1.188920in}}%
\pgfpathlineto{\pgfqpoint{5.978600in}{1.187464in}}%
\pgfpathlineto{\pgfqpoint{5.981080in}{1.178742in}}%
\pgfpathlineto{\pgfqpoint{5.984800in}{1.183765in}}%
\pgfpathlineto{\pgfqpoint{5.987280in}{1.182997in}}%
\pgfpathlineto{\pgfqpoint{5.988520in}{1.185113in}}%
\pgfpathlineto{\pgfqpoint{5.989760in}{1.183725in}}%
\pgfpathlineto{\pgfqpoint{5.991000in}{1.185313in}}%
\pgfpathlineto{\pgfqpoint{5.993480in}{1.183495in}}%
\pgfpathlineto{\pgfqpoint{5.997200in}{1.187447in}}%
\pgfpathlineto{\pgfqpoint{5.998440in}{1.186046in}}%
\pgfpathlineto{\pgfqpoint{5.999680in}{1.186677in}}%
\pgfpathlineto{\pgfqpoint{6.000920in}{1.184163in}}%
\pgfpathlineto{\pgfqpoint{6.003400in}{1.184970in}}%
\pgfpathlineto{\pgfqpoint{6.005880in}{1.183388in}}%
\pgfpathlineto{\pgfqpoint{6.008360in}{1.184167in}}%
\pgfpathlineto{\pgfqpoint{6.010840in}{1.179173in}}%
\pgfpathlineto{\pgfqpoint{6.012080in}{1.179607in}}%
\pgfpathlineto{\pgfqpoint{6.013320in}{1.182249in}}%
\pgfpathlineto{\pgfqpoint{6.014560in}{1.179699in}}%
\pgfpathlineto{\pgfqpoint{6.019520in}{1.184313in}}%
\pgfpathlineto{\pgfqpoint{6.020760in}{1.185830in}}%
\pgfpathlineto{\pgfqpoint{6.022000in}{1.182907in}}%
\pgfpathlineto{\pgfqpoint{6.024480in}{1.188014in}}%
\pgfpathlineto{\pgfqpoint{6.030680in}{1.194253in}}%
\pgfpathlineto{\pgfqpoint{6.031920in}{1.196989in}}%
\pgfpathlineto{\pgfqpoint{6.034400in}{1.195752in}}%
\pgfpathlineto{\pgfqpoint{6.036880in}{1.195272in}}%
\pgfpathlineto{\pgfqpoint{6.038120in}{1.197004in}}%
\pgfpathlineto{\pgfqpoint{6.043080in}{1.185948in}}%
\pgfpathlineto{\pgfqpoint{6.044320in}{1.182630in}}%
\pgfpathlineto{\pgfqpoint{6.055480in}{1.184025in}}%
\pgfpathlineto{\pgfqpoint{6.057960in}{1.181257in}}%
\pgfpathlineto{\pgfqpoint{6.060440in}{1.182910in}}%
\pgfpathlineto{\pgfqpoint{6.061680in}{1.185237in}}%
\pgfpathlineto{\pgfqpoint{6.064160in}{1.184551in}}%
\pgfpathlineto{\pgfqpoint{6.065400in}{1.186718in}}%
\pgfpathlineto{\pgfqpoint{6.066640in}{1.184178in}}%
\pgfpathlineto{\pgfqpoint{6.071600in}{1.184871in}}%
\pgfpathlineto{\pgfqpoint{6.074080in}{1.180175in}}%
\pgfpathlineto{\pgfqpoint{6.075320in}{1.183159in}}%
\pgfpathlineto{\pgfqpoint{6.076560in}{1.182072in}}%
\pgfpathlineto{\pgfqpoint{6.079040in}{1.183043in}}%
\pgfpathlineto{\pgfqpoint{6.082760in}{1.181087in}}%
\pgfpathlineto{\pgfqpoint{6.084000in}{1.180930in}}%
\pgfpathlineto{\pgfqpoint{6.085240in}{1.189077in}}%
\pgfpathlineto{\pgfqpoint{6.086480in}{1.187495in}}%
\pgfpathlineto{\pgfqpoint{6.092680in}{1.194288in}}%
\pgfpathlineto{\pgfqpoint{6.097640in}{1.195091in}}%
\pgfpathlineto{\pgfqpoint{6.100120in}{1.197430in}}%
\pgfpathlineto{\pgfqpoint{6.102600in}{1.194753in}}%
\pgfpathlineto{\pgfqpoint{6.105080in}{1.186730in}}%
\pgfpathlineto{\pgfqpoint{6.106320in}{1.188492in}}%
\pgfpathlineto{\pgfqpoint{6.107560in}{1.188119in}}%
\pgfpathlineto{\pgfqpoint{6.108800in}{1.189776in}}%
\pgfpathlineto{\pgfqpoint{6.111280in}{1.188380in}}%
\pgfpathlineto{\pgfqpoint{6.112520in}{1.190662in}}%
\pgfpathlineto{\pgfqpoint{6.113760in}{1.188861in}}%
\pgfpathlineto{\pgfqpoint{6.115000in}{1.190923in}}%
\pgfpathlineto{\pgfqpoint{6.117480in}{1.188344in}}%
\pgfpathlineto{\pgfqpoint{6.121200in}{1.194625in}}%
\pgfpathlineto{\pgfqpoint{6.122440in}{1.193798in}}%
\pgfpathlineto{\pgfqpoint{6.123680in}{1.195072in}}%
\pgfpathlineto{\pgfqpoint{6.126160in}{1.192112in}}%
\pgfpathlineto{\pgfqpoint{6.132360in}{1.191416in}}%
\pgfpathlineto{\pgfqpoint{6.134840in}{1.186168in}}%
\pgfpathlineto{\pgfqpoint{6.136080in}{1.187304in}}%
\pgfpathlineto{\pgfqpoint{6.137320in}{1.190488in}}%
\pgfpathlineto{\pgfqpoint{6.138560in}{1.186459in}}%
\pgfpathlineto{\pgfqpoint{6.142280in}{1.190620in}}%
\pgfpathlineto{\pgfqpoint{6.143520in}{1.190043in}}%
\pgfpathlineto{\pgfqpoint{6.144760in}{1.191375in}}%
\pgfpathlineto{\pgfqpoint{6.146000in}{1.188968in}}%
\pgfpathlineto{\pgfqpoint{6.148480in}{1.193866in}}%
\pgfpathlineto{\pgfqpoint{6.155920in}{1.203695in}}%
\pgfpathlineto{\pgfqpoint{6.158400in}{1.201724in}}%
\pgfpathlineto{\pgfqpoint{6.159640in}{1.200565in}}%
\pgfpathlineto{\pgfqpoint{6.160880in}{1.201240in}}%
\pgfpathlineto{\pgfqpoint{6.162120in}{1.203223in}}%
\pgfpathlineto{\pgfqpoint{6.170800in}{1.188309in}}%
\pgfpathlineto{\pgfqpoint{6.175760in}{1.186944in}}%
\pgfpathlineto{\pgfqpoint{6.177000in}{1.185047in}}%
\pgfpathlineto{\pgfqpoint{6.179480in}{1.185233in}}%
\pgfpathlineto{\pgfqpoint{6.181960in}{1.183000in}}%
\pgfpathlineto{\pgfqpoint{6.184440in}{1.185689in}}%
\pgfpathlineto{\pgfqpoint{6.185680in}{1.189301in}}%
\pgfpathlineto{\pgfqpoint{6.186920in}{1.188354in}}%
\pgfpathlineto{\pgfqpoint{6.188160in}{1.188895in}}%
\pgfpathlineto{\pgfqpoint{6.189400in}{1.191162in}}%
\pgfpathlineto{\pgfqpoint{6.190640in}{1.188442in}}%
\pgfpathlineto{\pgfqpoint{6.191880in}{1.189029in}}%
\pgfpathlineto{\pgfqpoint{6.193120in}{1.191072in}}%
\pgfpathlineto{\pgfqpoint{6.195600in}{1.190022in}}%
\pgfpathlineto{\pgfqpoint{6.198080in}{1.184335in}}%
\pgfpathlineto{\pgfqpoint{6.199320in}{1.187180in}}%
\pgfpathlineto{\pgfqpoint{6.201800in}{1.185039in}}%
\pgfpathlineto{\pgfqpoint{6.206760in}{1.184569in}}%
\pgfpathlineto{\pgfqpoint{6.208000in}{1.183030in}}%
\pgfpathlineto{\pgfqpoint{6.209240in}{1.188228in}}%
\pgfpathlineto{\pgfqpoint{6.210480in}{1.186947in}}%
\pgfpathlineto{\pgfqpoint{6.215440in}{1.193691in}}%
\pgfpathlineto{\pgfqpoint{6.217920in}{1.194165in}}%
\pgfpathlineto{\pgfqpoint{6.219160in}{1.192459in}}%
\pgfpathlineto{\pgfqpoint{6.221640in}{1.193857in}}%
\pgfpathlineto{\pgfqpoint{6.222880in}{1.195503in}}%
\pgfpathlineto{\pgfqpoint{6.225360in}{1.194558in}}%
\pgfpathlineto{\pgfqpoint{6.227840in}{1.187902in}}%
\pgfpathlineto{\pgfqpoint{6.229080in}{1.182800in}}%
\pgfpathlineto{\pgfqpoint{6.232800in}{1.186279in}}%
\pgfpathlineto{\pgfqpoint{6.235280in}{1.182992in}}%
\pgfpathlineto{\pgfqpoint{6.236520in}{1.185592in}}%
\pgfpathlineto{\pgfqpoint{6.237760in}{1.185100in}}%
\pgfpathlineto{\pgfqpoint{6.239000in}{1.188910in}}%
\pgfpathlineto{\pgfqpoint{6.241480in}{1.189568in}}%
\pgfpathlineto{\pgfqpoint{6.245200in}{1.197068in}}%
\pgfpathlineto{\pgfqpoint{6.246440in}{1.195640in}}%
\pgfpathlineto{\pgfqpoint{6.247680in}{1.196491in}}%
\pgfpathlineto{\pgfqpoint{6.248920in}{1.193023in}}%
\pgfpathlineto{\pgfqpoint{6.251400in}{1.196330in}}%
\pgfpathlineto{\pgfqpoint{6.252640in}{1.195002in}}%
\pgfpathlineto{\pgfqpoint{6.256360in}{1.199952in}}%
\pgfpathlineto{\pgfqpoint{6.258840in}{1.194294in}}%
\pgfpathlineto{\pgfqpoint{6.260080in}{1.194554in}}%
\pgfpathlineto{\pgfqpoint{6.261320in}{1.196223in}}%
\pgfpathlineto{\pgfqpoint{6.262560in}{1.191205in}}%
\pgfpathlineto{\pgfqpoint{6.266280in}{1.197084in}}%
\pgfpathlineto{\pgfqpoint{6.267520in}{1.196618in}}%
\pgfpathlineto{\pgfqpoint{6.268760in}{1.197725in}}%
\pgfpathlineto{\pgfqpoint{6.270000in}{1.194553in}}%
\pgfpathlineto{\pgfqpoint{6.272480in}{1.198127in}}%
\pgfpathlineto{\pgfqpoint{6.276200in}{1.201959in}}%
\pgfpathlineto{\pgfqpoint{6.277440in}{1.201075in}}%
\pgfpathlineto{\pgfqpoint{6.279920in}{1.205176in}}%
\pgfpathlineto{\pgfqpoint{6.283640in}{1.200857in}}%
\pgfpathlineto{\pgfqpoint{6.284880in}{1.200833in}}%
\pgfpathlineto{\pgfqpoint{6.286120in}{1.202438in}}%
\pgfpathlineto{\pgfqpoint{6.287360in}{1.200023in}}%
\pgfpathlineto{\pgfqpoint{6.289840in}{1.193240in}}%
\pgfpathlineto{\pgfqpoint{6.292320in}{1.188059in}}%
\pgfpathlineto{\pgfqpoint{6.296040in}{1.188599in}}%
\pgfpathlineto{\pgfqpoint{6.298520in}{1.187978in}}%
\pgfpathlineto{\pgfqpoint{6.302240in}{1.186517in}}%
\pgfpathlineto{\pgfqpoint{6.303480in}{1.186644in}}%
\pgfpathlineto{\pgfqpoint{6.305960in}{1.183934in}}%
\pgfpathlineto{\pgfqpoint{6.308440in}{1.186199in}}%
\pgfpathlineto{\pgfqpoint{6.309680in}{1.190429in}}%
\pgfpathlineto{\pgfqpoint{6.312160in}{1.191026in}}%
\pgfpathlineto{\pgfqpoint{6.313400in}{1.193425in}}%
\pgfpathlineto{\pgfqpoint{6.314640in}{1.191307in}}%
\pgfpathlineto{\pgfqpoint{6.319600in}{1.193636in}}%
\pgfpathlineto{\pgfqpoint{6.322080in}{1.190135in}}%
\pgfpathlineto{\pgfqpoint{6.323320in}{1.192317in}}%
\pgfpathlineto{\pgfqpoint{6.325800in}{1.189884in}}%
\pgfpathlineto{\pgfqpoint{6.329520in}{1.189206in}}%
\pgfpathlineto{\pgfqpoint{6.333240in}{1.186533in}}%
\pgfpathlineto{\pgfqpoint{6.334480in}{1.186092in}}%
\pgfpathlineto{\pgfqpoint{6.336960in}{1.190699in}}%
\pgfpathlineto{\pgfqpoint{6.341920in}{1.193731in}}%
\pgfpathlineto{\pgfqpoint{6.343160in}{1.191348in}}%
\pgfpathlineto{\pgfqpoint{6.348120in}{1.196107in}}%
\pgfpathlineto{\pgfqpoint{6.349360in}{1.194839in}}%
\pgfpathlineto{\pgfqpoint{6.353080in}{1.182236in}}%
\pgfpathlineto{\pgfqpoint{6.354320in}{1.184534in}}%
\pgfpathlineto{\pgfqpoint{6.355560in}{1.183284in}}%
\pgfpathlineto{\pgfqpoint{6.356800in}{1.184611in}}%
\pgfpathlineto{\pgfqpoint{6.359280in}{1.180613in}}%
\pgfpathlineto{\pgfqpoint{6.364240in}{1.189154in}}%
\pgfpathlineto{\pgfqpoint{6.366720in}{1.190982in}}%
\pgfpathlineto{\pgfqpoint{6.367960in}{1.192685in}}%
\pgfpathlineto{\pgfqpoint{6.369200in}{1.196438in}}%
\pgfpathlineto{\pgfqpoint{6.370440in}{1.195495in}}%
\pgfpathlineto{\pgfqpoint{6.371680in}{1.198376in}}%
\pgfpathlineto{\pgfqpoint{6.372920in}{1.194976in}}%
\pgfpathlineto{\pgfqpoint{6.380360in}{1.201745in}}%
\pgfpathlineto{\pgfqpoint{6.382840in}{1.197303in}}%
\pgfpathlineto{\pgfqpoint{6.385320in}{1.198591in}}%
\pgfpathlineto{\pgfqpoint{6.386560in}{1.194166in}}%
\pgfpathlineto{\pgfqpoint{6.387800in}{1.194898in}}%
\pgfpathlineto{\pgfqpoint{6.389040in}{1.198543in}}%
\pgfpathlineto{\pgfqpoint{6.390280in}{1.198457in}}%
\pgfpathlineto{\pgfqpoint{6.391520in}{1.196721in}}%
\pgfpathlineto{\pgfqpoint{6.392760in}{1.197468in}}%
\pgfpathlineto{\pgfqpoint{6.394000in}{1.194313in}}%
\pgfpathlineto{\pgfqpoint{6.396480in}{1.198135in}}%
\pgfpathlineto{\pgfqpoint{6.397720in}{1.197793in}}%
\pgfpathlineto{\pgfqpoint{6.400200in}{1.201266in}}%
\pgfpathlineto{\pgfqpoint{6.401440in}{1.200030in}}%
\pgfpathlineto{\pgfqpoint{6.403920in}{1.203930in}}%
\pgfpathlineto{\pgfqpoint{6.406400in}{1.202019in}}%
\pgfpathlineto{\pgfqpoint{6.410120in}{1.202781in}}%
\pgfpathlineto{\pgfqpoint{6.417560in}{1.191424in}}%
\pgfpathlineto{\pgfqpoint{6.420040in}{1.189046in}}%
\pgfpathlineto{\pgfqpoint{6.421280in}{1.187875in}}%
\pgfpathlineto{\pgfqpoint{6.422520in}{1.188961in}}%
\pgfpathlineto{\pgfqpoint{6.426240in}{1.187846in}}%
\pgfpathlineto{\pgfqpoint{6.427480in}{1.187505in}}%
\pgfpathlineto{\pgfqpoint{6.429960in}{1.183916in}}%
\pgfpathlineto{\pgfqpoint{6.432440in}{1.184229in}}%
\pgfpathlineto{\pgfqpoint{6.433680in}{1.187548in}}%
\pgfpathlineto{\pgfqpoint{6.434920in}{1.186583in}}%
\pgfpathlineto{\pgfqpoint{6.437400in}{1.188985in}}%
\pgfpathlineto{\pgfqpoint{6.438640in}{1.186127in}}%
\pgfpathlineto{\pgfqpoint{6.443600in}{1.185597in}}%
\pgfpathlineto{\pgfqpoint{6.449800in}{1.180496in}}%
\pgfpathlineto{\pgfqpoint{6.452280in}{1.182427in}}%
\pgfpathlineto{\pgfqpoint{6.454760in}{1.181207in}}%
\pgfpathlineto{\pgfqpoint{6.456000in}{1.178538in}}%
\pgfpathlineto{\pgfqpoint{6.457240in}{1.184536in}}%
\pgfpathlineto{\pgfqpoint{6.458480in}{1.183854in}}%
\pgfpathlineto{\pgfqpoint{6.460960in}{1.188159in}}%
\pgfpathlineto{\pgfqpoint{6.464680in}{1.188506in}}%
\pgfpathlineto{\pgfqpoint{6.465920in}{1.190047in}}%
\pgfpathlineto{\pgfqpoint{6.467160in}{1.187737in}}%
\pgfpathlineto{\pgfqpoint{6.468400in}{1.189814in}}%
\pgfpathlineto{\pgfqpoint{6.469640in}{1.188773in}}%
\pgfpathlineto{\pgfqpoint{6.472120in}{1.190475in}}%
\pgfpathlineto{\pgfqpoint{6.474600in}{1.187426in}}%
\pgfpathlineto{\pgfqpoint{6.477080in}{1.180482in}}%
\pgfpathlineto{\pgfqpoint{6.480800in}{1.186725in}}%
\pgfpathlineto{\pgfqpoint{6.483280in}{1.181065in}}%
\pgfpathlineto{\pgfqpoint{6.488240in}{1.190752in}}%
\pgfpathlineto{\pgfqpoint{6.489480in}{1.190752in}}%
\pgfpathlineto{\pgfqpoint{6.491960in}{1.195015in}}%
\pgfpathlineto{\pgfqpoint{6.494440in}{1.199556in}}%
\pgfpathlineto{\pgfqpoint{6.495680in}{1.202390in}}%
\pgfpathlineto{\pgfqpoint{6.496920in}{1.199961in}}%
\pgfpathlineto{\pgfqpoint{6.499400in}{1.202045in}}%
\pgfpathlineto{\pgfqpoint{6.501880in}{1.202771in}}%
\pgfpathlineto{\pgfqpoint{6.504360in}{1.204267in}}%
\pgfpathlineto{\pgfqpoint{6.506840in}{1.198570in}}%
\pgfpathlineto{\pgfqpoint{6.509320in}{1.199808in}}%
\pgfpathlineto{\pgfqpoint{6.510560in}{1.193923in}}%
\pgfpathlineto{\pgfqpoint{6.511800in}{1.194414in}}%
\pgfpathlineto{\pgfqpoint{6.513040in}{1.198469in}}%
\pgfpathlineto{\pgfqpoint{6.515520in}{1.196512in}}%
\pgfpathlineto{\pgfqpoint{6.516760in}{1.197895in}}%
\pgfpathlineto{\pgfqpoint{6.518000in}{1.195255in}}%
\pgfpathlineto{\pgfqpoint{6.520480in}{1.200405in}}%
\pgfpathlineto{\pgfqpoint{6.522960in}{1.202568in}}%
\pgfpathlineto{\pgfqpoint{6.524200in}{1.205373in}}%
\pgfpathlineto{\pgfqpoint{6.525440in}{1.203312in}}%
\pgfpathlineto{\pgfqpoint{6.527920in}{1.207230in}}%
\pgfpathlineto{\pgfqpoint{6.529160in}{1.206638in}}%
\pgfpathlineto{\pgfqpoint{6.530400in}{1.208399in}}%
\pgfpathlineto{\pgfqpoint{6.532880in}{1.205291in}}%
\pgfpathlineto{\pgfqpoint{6.534120in}{1.206321in}}%
\pgfpathlineto{\pgfqpoint{6.545280in}{1.194218in}}%
\pgfpathlineto{\pgfqpoint{6.546520in}{1.195590in}}%
\pgfpathlineto{\pgfqpoint{6.549000in}{1.191400in}}%
\pgfpathlineto{\pgfqpoint{6.550240in}{1.193075in}}%
\pgfpathlineto{\pgfqpoint{6.551480in}{1.192203in}}%
\pgfpathlineto{\pgfqpoint{6.552720in}{1.189363in}}%
\pgfpathlineto{\pgfqpoint{6.556440in}{1.190481in}}%
\pgfpathlineto{\pgfqpoint{6.557680in}{1.193274in}}%
\pgfpathlineto{\pgfqpoint{6.558920in}{1.192554in}}%
\pgfpathlineto{\pgfqpoint{6.560160in}{1.193439in}}%
\pgfpathlineto{\pgfqpoint{6.561400in}{1.195908in}}%
\pgfpathlineto{\pgfqpoint{6.562640in}{1.192647in}}%
\pgfpathlineto{\pgfqpoint{6.565120in}{1.193228in}}%
\pgfpathlineto{\pgfqpoint{6.566360in}{1.193126in}}%
\pgfpathlineto{\pgfqpoint{6.570080in}{1.189238in}}%
\pgfpathlineto{\pgfqpoint{6.571320in}{1.189467in}}%
\pgfpathlineto{\pgfqpoint{6.573800in}{1.186357in}}%
\pgfpathlineto{\pgfqpoint{6.576280in}{1.187025in}}%
\pgfpathlineto{\pgfqpoint{6.578760in}{1.183214in}}%
\pgfpathlineto{\pgfqpoint{6.580000in}{1.178851in}}%
\pgfpathlineto{\pgfqpoint{6.581240in}{1.188247in}}%
\pgfpathlineto{\pgfqpoint{6.582480in}{1.187796in}}%
\pgfpathlineto{\pgfqpoint{6.584960in}{1.194004in}}%
\pgfpathlineto{\pgfqpoint{6.588680in}{1.192234in}}%
\pgfpathlineto{\pgfqpoint{6.589920in}{1.193441in}}%
\pgfpathlineto{\pgfqpoint{6.591160in}{1.188452in}}%
\pgfpathlineto{\pgfqpoint{6.594880in}{1.192397in}}%
\pgfpathlineto{\pgfqpoint{6.596120in}{1.195423in}}%
\pgfpathlineto{\pgfqpoint{6.598600in}{1.191038in}}%
\pgfpathlineto{\pgfqpoint{6.601080in}{1.181537in}}%
\pgfpathlineto{\pgfqpoint{6.604800in}{1.188798in}}%
\pgfpathlineto{\pgfqpoint{6.607280in}{1.183767in}}%
\pgfpathlineto{\pgfqpoint{6.613480in}{1.197637in}}%
\pgfpathlineto{\pgfqpoint{6.615960in}{1.200899in}}%
\pgfpathlineto{\pgfqpoint{6.617200in}{1.204734in}}%
\pgfpathlineto{\pgfqpoint{6.618440in}{1.204623in}}%
\pgfpathlineto{\pgfqpoint{6.619680in}{1.208702in}}%
\pgfpathlineto{\pgfqpoint{6.620920in}{1.204491in}}%
\pgfpathlineto{\pgfqpoint{6.623400in}{1.208953in}}%
\pgfpathlineto{\pgfqpoint{6.627120in}{1.209534in}}%
\pgfpathlineto{\pgfqpoint{6.628360in}{1.211911in}}%
\pgfpathlineto{\pgfqpoint{6.630840in}{1.208813in}}%
\pgfpathlineto{\pgfqpoint{6.633320in}{1.211040in}}%
\pgfpathlineto{\pgfqpoint{6.635800in}{1.204966in}}%
\pgfpathlineto{\pgfqpoint{6.637040in}{1.210034in}}%
\pgfpathlineto{\pgfqpoint{6.639520in}{1.207047in}}%
\pgfpathlineto{\pgfqpoint{6.640760in}{1.208436in}}%
\pgfpathlineto{\pgfqpoint{6.642000in}{1.204658in}}%
\pgfpathlineto{\pgfqpoint{6.645720in}{1.212552in}}%
\pgfpathlineto{\pgfqpoint{6.648200in}{1.216244in}}%
\pgfpathlineto{\pgfqpoint{6.649440in}{1.213311in}}%
\pgfpathlineto{\pgfqpoint{6.651920in}{1.219020in}}%
\pgfpathlineto{\pgfqpoint{6.653160in}{1.217970in}}%
\pgfpathlineto{\pgfqpoint{6.654400in}{1.220579in}}%
\pgfpathlineto{\pgfqpoint{6.660600in}{1.215023in}}%
\pgfpathlineto{\pgfqpoint{6.661840in}{1.216987in}}%
\pgfpathlineto{\pgfqpoint{6.665560in}{1.208078in}}%
\pgfpathlineto{\pgfqpoint{6.669280in}{1.204553in}}%
\pgfpathlineto{\pgfqpoint{6.670520in}{1.204740in}}%
\pgfpathlineto{\pgfqpoint{6.673000in}{1.198435in}}%
\pgfpathlineto{\pgfqpoint{6.675480in}{1.199811in}}%
\pgfpathlineto{\pgfqpoint{6.677960in}{1.195904in}}%
\pgfpathlineto{\pgfqpoint{6.679200in}{1.194325in}}%
\pgfpathlineto{\pgfqpoint{6.682920in}{1.196381in}}%
\pgfpathlineto{\pgfqpoint{6.685400in}{1.203759in}}%
\pgfpathlineto{\pgfqpoint{6.686640in}{1.200271in}}%
\pgfpathlineto{\pgfqpoint{6.687880in}{1.200251in}}%
\pgfpathlineto{\pgfqpoint{6.691600in}{1.192802in}}%
\pgfpathlineto{\pgfqpoint{6.695320in}{1.190239in}}%
\pgfpathlineto{\pgfqpoint{6.697800in}{1.186524in}}%
\pgfpathlineto{\pgfqpoint{6.700280in}{1.187340in}}%
\pgfpathlineto{\pgfqpoint{6.701520in}{1.185049in}}%
\pgfpathlineto{\pgfqpoint{6.702760in}{1.185481in}}%
\pgfpathlineto{\pgfqpoint{6.704000in}{1.182301in}}%
\pgfpathlineto{\pgfqpoint{6.706480in}{1.193328in}}%
\pgfpathlineto{\pgfqpoint{6.708960in}{1.199316in}}%
\pgfpathlineto{\pgfqpoint{6.710200in}{1.196589in}}%
\pgfpathlineto{\pgfqpoint{6.711440in}{1.199019in}}%
\pgfpathlineto{\pgfqpoint{6.713920in}{1.198417in}}%
\pgfpathlineto{\pgfqpoint{6.715160in}{1.193827in}}%
\pgfpathlineto{\pgfqpoint{6.716400in}{1.194688in}}%
\pgfpathlineto{\pgfqpoint{6.718880in}{1.193818in}}%
\pgfpathlineto{\pgfqpoint{6.720120in}{1.196789in}}%
\pgfpathlineto{\pgfqpoint{6.721360in}{1.195253in}}%
\pgfpathlineto{\pgfqpoint{6.725080in}{1.185009in}}%
\pgfpathlineto{\pgfqpoint{6.727560in}{1.189623in}}%
\pgfpathlineto{\pgfqpoint{6.728800in}{1.192007in}}%
\pgfpathlineto{\pgfqpoint{6.731280in}{1.186109in}}%
\pgfpathlineto{\pgfqpoint{6.736240in}{1.197734in}}%
\pgfpathlineto{\pgfqpoint{6.737480in}{1.198858in}}%
\pgfpathlineto{\pgfqpoint{6.741200in}{1.209513in}}%
\pgfpathlineto{\pgfqpoint{6.742440in}{1.207769in}}%
\pgfpathlineto{\pgfqpoint{6.743680in}{1.211332in}}%
\pgfpathlineto{\pgfqpoint{6.744920in}{1.204590in}}%
\pgfpathlineto{\pgfqpoint{6.749880in}{1.210895in}}%
\pgfpathlineto{\pgfqpoint{6.751120in}{1.210050in}}%
\pgfpathlineto{\pgfqpoint{6.752360in}{1.210654in}}%
\pgfpathlineto{\pgfqpoint{6.754840in}{1.207049in}}%
\pgfpathlineto{\pgfqpoint{6.757320in}{1.210969in}}%
\pgfpathlineto{\pgfqpoint{6.759800in}{1.203762in}}%
\pgfpathlineto{\pgfqpoint{6.761040in}{1.209564in}}%
\pgfpathlineto{\pgfqpoint{6.763520in}{1.205995in}}%
\pgfpathlineto{\pgfqpoint{6.764760in}{1.206931in}}%
\pgfpathlineto{\pgfqpoint{6.766000in}{1.204494in}}%
\pgfpathlineto{\pgfqpoint{6.769720in}{1.213812in}}%
\pgfpathlineto{\pgfqpoint{6.772200in}{1.216301in}}%
\pgfpathlineto{\pgfqpoint{6.773440in}{1.212222in}}%
\pgfpathlineto{\pgfqpoint{6.778400in}{1.224282in}}%
\pgfpathlineto{\pgfqpoint{6.784600in}{1.212326in}}%
\pgfpathlineto{\pgfqpoint{6.785840in}{1.214266in}}%
\pgfpathlineto{\pgfqpoint{6.788320in}{1.206695in}}%
\pgfpathlineto{\pgfqpoint{6.790800in}{1.204487in}}%
\pgfpathlineto{\pgfqpoint{6.792040in}{1.205181in}}%
\pgfpathlineto{\pgfqpoint{6.794520in}{1.208220in}}%
\pgfpathlineto{\pgfqpoint{6.798240in}{1.201832in}}%
\pgfpathlineto{\pgfqpoint{6.799480in}{1.201231in}}%
\pgfpathlineto{\pgfqpoint{6.800720in}{1.197336in}}%
\pgfpathlineto{\pgfqpoint{6.801960in}{1.198025in}}%
\pgfpathlineto{\pgfqpoint{6.803200in}{1.195327in}}%
\pgfpathlineto{\pgfqpoint{6.804440in}{1.197654in}}%
\pgfpathlineto{\pgfqpoint{6.805680in}{1.196583in}}%
\pgfpathlineto{\pgfqpoint{6.806920in}{1.197554in}}%
\pgfpathlineto{\pgfqpoint{6.809400in}{1.206692in}}%
\pgfpathlineto{\pgfqpoint{6.811880in}{1.203422in}}%
\pgfpathlineto{\pgfqpoint{6.814360in}{1.201531in}}%
\pgfpathlineto{\pgfqpoint{6.816840in}{1.198054in}}%
\pgfpathlineto{\pgfqpoint{6.818080in}{1.196076in}}%
\pgfpathlineto{\pgfqpoint{6.819320in}{1.196399in}}%
\pgfpathlineto{\pgfqpoint{6.821800in}{1.189450in}}%
\pgfpathlineto{\pgfqpoint{6.824280in}{1.186178in}}%
\pgfpathlineto{\pgfqpoint{6.825520in}{1.183700in}}%
\pgfpathlineto{\pgfqpoint{6.826760in}{1.186032in}}%
\pgfpathlineto{\pgfqpoint{6.828000in}{1.183182in}}%
\pgfpathlineto{\pgfqpoint{6.832960in}{1.197414in}}%
\pgfpathlineto{\pgfqpoint{6.834200in}{1.195149in}}%
\pgfpathlineto{\pgfqpoint{6.836680in}{1.202659in}}%
\pgfpathlineto{\pgfqpoint{6.837920in}{1.204014in}}%
\pgfpathlineto{\pgfqpoint{6.839160in}{1.200509in}}%
\pgfpathlineto{\pgfqpoint{6.844120in}{1.204745in}}%
\pgfpathlineto{\pgfqpoint{6.847840in}{1.193936in}}%
\pgfpathlineto{\pgfqpoint{6.849080in}{1.189305in}}%
\pgfpathlineto{\pgfqpoint{6.851560in}{1.194000in}}%
\pgfpathlineto{\pgfqpoint{6.852800in}{1.195616in}}%
\pgfpathlineto{\pgfqpoint{6.855280in}{1.190336in}}%
\pgfpathlineto{\pgfqpoint{6.856520in}{1.195353in}}%
\pgfpathlineto{\pgfqpoint{6.857760in}{1.194832in}}%
\pgfpathlineto{\pgfqpoint{6.862720in}{1.210793in}}%
\pgfpathlineto{\pgfqpoint{6.865200in}{1.216461in}}%
\pgfpathlineto{\pgfqpoint{6.866440in}{1.212681in}}%
\pgfpathlineto{\pgfqpoint{6.867680in}{1.218477in}}%
\pgfpathlineto{\pgfqpoint{6.868920in}{1.212239in}}%
\pgfpathlineto{\pgfqpoint{6.870160in}{1.214038in}}%
\pgfpathlineto{\pgfqpoint{6.873880in}{1.227896in}}%
\pgfpathlineto{\pgfqpoint{6.875120in}{1.226824in}}%
\pgfpathlineto{\pgfqpoint{6.876360in}{1.228571in}}%
\pgfpathlineto{\pgfqpoint{6.878840in}{1.217515in}}%
\pgfpathlineto{\pgfqpoint{6.881320in}{1.224098in}}%
\pgfpathlineto{\pgfqpoint{6.883800in}{1.218499in}}%
\pgfpathlineto{\pgfqpoint{6.885040in}{1.223316in}}%
\pgfpathlineto{\pgfqpoint{6.887520in}{1.219300in}}%
\pgfpathlineto{\pgfqpoint{6.890000in}{1.224984in}}%
\pgfpathlineto{\pgfqpoint{6.893720in}{1.230847in}}%
\pgfpathlineto{\pgfqpoint{6.894960in}{1.230426in}}%
\pgfpathlineto{\pgfqpoint{6.896200in}{1.232238in}}%
\pgfpathlineto{\pgfqpoint{6.897440in}{1.229002in}}%
\pgfpathlineto{\pgfqpoint{6.901160in}{1.243591in}}%
\pgfpathlineto{\pgfqpoint{6.902400in}{1.242839in}}%
\pgfpathlineto{\pgfqpoint{6.907360in}{1.230038in}}%
\pgfpathlineto{\pgfqpoint{6.909840in}{1.229362in}}%
\pgfpathlineto{\pgfqpoint{6.913560in}{1.216592in}}%
\pgfpathlineto{\pgfqpoint{6.916040in}{1.213771in}}%
\pgfpathlineto{\pgfqpoint{6.918520in}{1.221678in}}%
\pgfpathlineto{\pgfqpoint{6.919760in}{1.221496in}}%
\pgfpathlineto{\pgfqpoint{6.921000in}{1.218260in}}%
\pgfpathlineto{\pgfqpoint{6.922240in}{1.219485in}}%
\pgfpathlineto{\pgfqpoint{6.923480in}{1.218919in}}%
\pgfpathlineto{\pgfqpoint{6.927200in}{1.211385in}}%
\pgfpathlineto{\pgfqpoint{6.928440in}{1.213613in}}%
\pgfpathlineto{\pgfqpoint{6.930920in}{1.213926in}}%
\pgfpathlineto{\pgfqpoint{6.933400in}{1.225329in}}%
\pgfpathlineto{\pgfqpoint{6.939600in}{1.213941in}}%
\pgfpathlineto{\pgfqpoint{6.940840in}{1.215046in}}%
\pgfpathlineto{\pgfqpoint{6.942080in}{1.214158in}}%
\pgfpathlineto{\pgfqpoint{6.943320in}{1.214817in}}%
\pgfpathlineto{\pgfqpoint{6.945800in}{1.207923in}}%
\pgfpathlineto{\pgfqpoint{6.948280in}{1.204285in}}%
\pgfpathlineto{\pgfqpoint{6.949520in}{1.201081in}}%
\pgfpathlineto{\pgfqpoint{6.950760in}{1.204621in}}%
\pgfpathlineto{\pgfqpoint{6.952000in}{1.201559in}}%
\pgfpathlineto{\pgfqpoint{6.954480in}{1.205424in}}%
\pgfpathlineto{\pgfqpoint{6.955720in}{1.211417in}}%
\pgfpathlineto{\pgfqpoint{6.958200in}{1.204322in}}%
\pgfpathlineto{\pgfqpoint{6.961920in}{1.214640in}}%
\pgfpathlineto{\pgfqpoint{6.963160in}{1.209751in}}%
\pgfpathlineto{\pgfqpoint{6.964400in}{1.210497in}}%
\pgfpathlineto{\pgfqpoint{6.968120in}{1.215057in}}%
\pgfpathlineto{\pgfqpoint{6.969360in}{1.213041in}}%
\pgfpathlineto{\pgfqpoint{6.973080in}{1.195927in}}%
\pgfpathlineto{\pgfqpoint{6.974320in}{1.198207in}}%
\pgfpathlineto{\pgfqpoint{6.976800in}{1.204768in}}%
\pgfpathlineto{\pgfqpoint{6.978040in}{1.197568in}}%
\pgfpathlineto{\pgfqpoint{6.979280in}{1.197529in}}%
\pgfpathlineto{\pgfqpoint{6.986720in}{1.231054in}}%
\pgfpathlineto{\pgfqpoint{6.987960in}{1.228741in}}%
\pgfpathlineto{\pgfqpoint{6.989200in}{1.231816in}}%
\pgfpathlineto{\pgfqpoint{6.990440in}{1.230561in}}%
\pgfpathlineto{\pgfqpoint{6.991680in}{1.238523in}}%
\pgfpathlineto{\pgfqpoint{6.994160in}{1.225599in}}%
\pgfpathlineto{\pgfqpoint{6.997880in}{1.236573in}}%
\pgfpathlineto{\pgfqpoint{6.999120in}{1.237010in}}%
\pgfpathlineto{\pgfqpoint{7.000360in}{1.243496in}}%
\pgfpathlineto{\pgfqpoint{7.002840in}{1.226946in}}%
\pgfpathlineto{\pgfqpoint{7.005320in}{1.233965in}}%
\pgfpathlineto{\pgfqpoint{7.006560in}{1.225590in}}%
\pgfpathlineto{\pgfqpoint{7.007800in}{1.226207in}}%
\pgfpathlineto{\pgfqpoint{7.009040in}{1.234169in}}%
\pgfpathlineto{\pgfqpoint{7.011520in}{1.224474in}}%
\pgfpathlineto{\pgfqpoint{7.012760in}{1.226586in}}%
\pgfpathlineto{\pgfqpoint{7.014000in}{1.223566in}}%
\pgfpathlineto{\pgfqpoint{7.016480in}{1.227835in}}%
\pgfpathlineto{\pgfqpoint{7.017720in}{1.235225in}}%
\pgfpathlineto{\pgfqpoint{7.020200in}{1.237054in}}%
\pgfpathlineto{\pgfqpoint{7.021440in}{1.233448in}}%
\pgfpathlineto{\pgfqpoint{7.026400in}{1.253310in}}%
\pgfpathlineto{\pgfqpoint{7.027640in}{1.247642in}}%
\pgfpathlineto{\pgfqpoint{7.028880in}{1.233781in}}%
\pgfpathlineto{\pgfqpoint{7.031360in}{1.232944in}}%
\pgfpathlineto{\pgfqpoint{7.032600in}{1.231309in}}%
\pgfpathlineto{\pgfqpoint{7.033840in}{1.233638in}}%
\pgfpathlineto{\pgfqpoint{7.036320in}{1.223256in}}%
\pgfpathlineto{\pgfqpoint{7.037560in}{1.223509in}}%
\pgfpathlineto{\pgfqpoint{7.041280in}{1.228287in}}%
\pgfpathlineto{\pgfqpoint{7.042520in}{1.228202in}}%
\pgfpathlineto{\pgfqpoint{7.043760in}{1.229550in}}%
\pgfpathlineto{\pgfqpoint{7.045000in}{1.228837in}}%
\pgfpathlineto{\pgfqpoint{7.051200in}{1.207672in}}%
\pgfpathlineto{\pgfqpoint{7.052440in}{1.208361in}}%
\pgfpathlineto{\pgfqpoint{7.054920in}{1.206729in}}%
\pgfpathlineto{\pgfqpoint{7.056160in}{1.210456in}}%
\pgfpathlineto{\pgfqpoint{7.057400in}{1.218553in}}%
\pgfpathlineto{\pgfqpoint{7.058640in}{1.217664in}}%
\pgfpathlineto{\pgfqpoint{7.059880in}{1.219532in}}%
\pgfpathlineto{\pgfqpoint{7.061120in}{1.218614in}}%
\pgfpathlineto{\pgfqpoint{7.062360in}{1.215980in}}%
\pgfpathlineto{\pgfqpoint{7.064840in}{1.217437in}}%
\pgfpathlineto{\pgfqpoint{7.066080in}{1.216666in}}%
\pgfpathlineto{\pgfqpoint{7.067320in}{1.221969in}}%
\pgfpathlineto{\pgfqpoint{7.068560in}{1.218170in}}%
\pgfpathlineto{\pgfqpoint{7.069800in}{1.219539in}}%
\pgfpathlineto{\pgfqpoint{7.071040in}{1.222795in}}%
\pgfpathlineto{\pgfqpoint{7.073520in}{1.215367in}}%
\pgfpathlineto{\pgfqpoint{7.074760in}{1.218675in}}%
\pgfpathlineto{\pgfqpoint{7.076000in}{1.216967in}}%
\pgfpathlineto{\pgfqpoint{7.077240in}{1.233312in}}%
\pgfpathlineto{\pgfqpoint{7.078480in}{1.234612in}}%
\pgfpathlineto{\pgfqpoint{7.079720in}{1.240101in}}%
\pgfpathlineto{\pgfqpoint{7.080960in}{1.238523in}}%
\pgfpathlineto{\pgfqpoint{7.082200in}{1.233374in}}%
\pgfpathlineto{\pgfqpoint{7.083440in}{1.235899in}}%
\pgfpathlineto{\pgfqpoint{7.084680in}{1.245609in}}%
\pgfpathlineto{\pgfqpoint{7.087160in}{1.234830in}}%
\pgfpathlineto{\pgfqpoint{7.089640in}{1.229592in}}%
\pgfpathlineto{\pgfqpoint{7.090880in}{1.226309in}}%
\pgfpathlineto{\pgfqpoint{7.093360in}{1.217595in}}%
\pgfpathlineto{\pgfqpoint{7.097080in}{1.203863in}}%
\pgfpathlineto{\pgfqpoint{7.098320in}{1.203122in}}%
\pgfpathlineto{\pgfqpoint{7.100800in}{1.214429in}}%
\pgfpathlineto{\pgfqpoint{7.103280in}{1.203619in}}%
\pgfpathlineto{\pgfqpoint{7.107000in}{1.218500in}}%
\pgfpathlineto{\pgfqpoint{7.110720in}{1.253262in}}%
\pgfpathlineto{\pgfqpoint{7.113200in}{1.242714in}}%
\pgfpathlineto{\pgfqpoint{7.114440in}{1.243564in}}%
\pgfpathlineto{\pgfqpoint{7.115680in}{1.250553in}}%
\pgfpathlineto{\pgfqpoint{7.118160in}{1.237787in}}%
\pgfpathlineto{\pgfqpoint{7.119400in}{1.244542in}}%
\pgfpathlineto{\pgfqpoint{7.120640in}{1.245012in}}%
\pgfpathlineto{\pgfqpoint{7.121880in}{1.249473in}}%
\pgfpathlineto{\pgfqpoint{7.123120in}{1.249626in}}%
\pgfpathlineto{\pgfqpoint{7.124360in}{1.255575in}}%
\pgfpathlineto{\pgfqpoint{7.126840in}{1.239531in}}%
\pgfpathlineto{\pgfqpoint{7.129320in}{1.256143in}}%
\pgfpathlineto{\pgfqpoint{7.130560in}{1.250842in}}%
\pgfpathlineto{\pgfqpoint{7.131800in}{1.253596in}}%
\pgfpathlineto{\pgfqpoint{7.133040in}{1.263226in}}%
\pgfpathlineto{\pgfqpoint{7.134280in}{1.261595in}}%
\pgfpathlineto{\pgfqpoint{7.136760in}{1.254730in}}%
\pgfpathlineto{\pgfqpoint{7.139240in}{1.260125in}}%
\pgfpathlineto{\pgfqpoint{7.140480in}{1.261282in}}%
\pgfpathlineto{\pgfqpoint{7.142960in}{1.275065in}}%
\pgfpathlineto{\pgfqpoint{7.144200in}{1.271194in}}%
\pgfpathlineto{\pgfqpoint{7.146680in}{1.258689in}}%
\pgfpathlineto{\pgfqpoint{7.150400in}{1.274824in}}%
\pgfpathlineto{\pgfqpoint{7.152880in}{1.254408in}}%
\pgfpathlineto{\pgfqpoint{7.154120in}{1.263169in}}%
\pgfpathlineto{\pgfqpoint{7.156600in}{1.257851in}}%
\pgfpathlineto{\pgfqpoint{7.157840in}{1.263077in}}%
\pgfpathlineto{\pgfqpoint{7.165280in}{1.239407in}}%
\pgfpathlineto{\pgfqpoint{7.167760in}{1.252977in}}%
\pgfpathlineto{\pgfqpoint{7.170240in}{1.250655in}}%
\pgfpathlineto{\pgfqpoint{7.172720in}{1.236231in}}%
\pgfpathlineto{\pgfqpoint{7.173960in}{1.239657in}}%
\pgfpathlineto{\pgfqpoint{7.176440in}{1.226953in}}%
\pgfpathlineto{\pgfqpoint{7.177680in}{1.225070in}}%
\pgfpathlineto{\pgfqpoint{7.180160in}{1.231232in}}%
\pgfpathlineto{\pgfqpoint{7.181400in}{1.243662in}}%
\pgfpathlineto{\pgfqpoint{7.182640in}{1.241964in}}%
\pgfpathlineto{\pgfqpoint{7.185120in}{1.246430in}}%
\pgfpathlineto{\pgfqpoint{7.186360in}{1.242218in}}%
\pgfpathlineto{\pgfqpoint{7.187600in}{1.242766in}}%
\pgfpathlineto{\pgfqpoint{7.190080in}{1.246210in}}%
\pgfpathlineto{\pgfqpoint{7.191320in}{1.248606in}}%
\pgfpathlineto{\pgfqpoint{7.193800in}{1.240855in}}%
\pgfpathlineto{\pgfqpoint{7.195040in}{1.241134in}}%
\pgfpathlineto{\pgfqpoint{7.196280in}{1.236722in}}%
\pgfpathlineto{\pgfqpoint{7.197520in}{1.237976in}}%
\pgfpathlineto{\pgfqpoint{7.200000in}{1.245004in}}%
\pgfpathlineto{\pgfqpoint{7.200000in}{1.245004in}}%
\pgfusepath{stroke}%
\end{pgfscope}%
\begin{pgfscope}%
\pgfpathrectangle{\pgfqpoint{1.000000in}{0.350000in}}{\pgfqpoint{6.200000in}{2.800000in}} %
\pgfusepath{clip}%
\pgfsetrectcap%
\pgfsetroundjoin%
\pgfsetlinewidth{1.003750pt}%
\definecolor{currentstroke}{rgb}{0.000000,0.500000,0.000000}%
\pgfsetstrokecolor{currentstroke}%
\pgfsetdash{}{0pt}%
\pgfpathmoveto{\pgfqpoint{1.001240in}{1.559994in}}%
\pgfpathlineto{\pgfqpoint{1.002480in}{2.017291in}}%
\pgfpathlineto{\pgfqpoint{1.003720in}{2.067136in}}%
\pgfpathlineto{\pgfqpoint{1.011160in}{1.344012in}}%
\pgfpathlineto{\pgfqpoint{1.016120in}{1.117958in}}%
\pgfpathlineto{\pgfqpoint{1.021080in}{0.988717in}}%
\pgfpathlineto{\pgfqpoint{1.027280in}{0.901845in}}%
\pgfpathlineto{\pgfqpoint{1.031000in}{0.867336in}}%
\pgfpathlineto{\pgfqpoint{1.032240in}{0.867657in}}%
\pgfpathlineto{\pgfqpoint{1.035960in}{0.856464in}}%
\pgfpathlineto{\pgfqpoint{1.037200in}{0.856154in}}%
\pgfpathlineto{\pgfqpoint{1.038440in}{0.853431in}}%
\pgfpathlineto{\pgfqpoint{1.042160in}{0.837417in}}%
\pgfpathlineto{\pgfqpoint{1.044640in}{0.840081in}}%
\pgfpathlineto{\pgfqpoint{1.047120in}{0.821691in}}%
\pgfpathlineto{\pgfqpoint{1.050840in}{0.798015in}}%
\pgfpathlineto{\pgfqpoint{1.053320in}{0.792598in}}%
\pgfpathlineto{\pgfqpoint{1.055800in}{0.782629in}}%
\pgfpathlineto{\pgfqpoint{1.059520in}{0.777147in}}%
\pgfpathlineto{\pgfqpoint{1.062000in}{0.766540in}}%
\pgfpathlineto{\pgfqpoint{1.066960in}{0.741524in}}%
\pgfpathlineto{\pgfqpoint{1.068200in}{0.740294in}}%
\pgfpathlineto{\pgfqpoint{1.073160in}{0.744642in}}%
\pgfpathlineto{\pgfqpoint{1.078120in}{0.739982in}}%
\pgfpathlineto{\pgfqpoint{1.079360in}{0.740315in}}%
\pgfpathlineto{\pgfqpoint{1.080600in}{0.738251in}}%
\pgfpathlineto{\pgfqpoint{1.081840in}{0.740649in}}%
\pgfpathlineto{\pgfqpoint{1.086800in}{0.723733in}}%
\pgfpathlineto{\pgfqpoint{1.089280in}{0.713006in}}%
\pgfpathlineto{\pgfqpoint{1.090520in}{0.711869in}}%
\pgfpathlineto{\pgfqpoint{1.091760in}{0.714224in}}%
\pgfpathlineto{\pgfqpoint{1.097960in}{0.697867in}}%
\pgfpathlineto{\pgfqpoint{1.100440in}{0.697369in}}%
\pgfpathlineto{\pgfqpoint{1.101680in}{0.702340in}}%
\pgfpathlineto{\pgfqpoint{1.109120in}{0.697254in}}%
\pgfpathlineto{\pgfqpoint{1.111600in}{0.700945in}}%
\pgfpathlineto{\pgfqpoint{1.112840in}{0.699345in}}%
\pgfpathlineto{\pgfqpoint{1.114080in}{0.700180in}}%
\pgfpathlineto{\pgfqpoint{1.115320in}{0.702782in}}%
\pgfpathlineto{\pgfqpoint{1.116560in}{0.702703in}}%
\pgfpathlineto{\pgfqpoint{1.117800in}{0.700879in}}%
\pgfpathlineto{\pgfqpoint{1.120280in}{0.703394in}}%
\pgfpathlineto{\pgfqpoint{1.122760in}{0.699764in}}%
\pgfpathlineto{\pgfqpoint{1.124000in}{0.700601in}}%
\pgfpathlineto{\pgfqpoint{1.126480in}{0.693595in}}%
\pgfpathlineto{\pgfqpoint{1.127720in}{0.694266in}}%
\pgfpathlineto{\pgfqpoint{1.131440in}{0.701272in}}%
\pgfpathlineto{\pgfqpoint{1.132680in}{0.701042in}}%
\pgfpathlineto{\pgfqpoint{1.133920in}{0.702689in}}%
\pgfpathlineto{\pgfqpoint{1.142600in}{0.688868in}}%
\pgfpathlineto{\pgfqpoint{1.143840in}{0.688628in}}%
\pgfpathlineto{\pgfqpoint{1.155000in}{0.662231in}}%
\pgfpathlineto{\pgfqpoint{1.157480in}{0.668991in}}%
\pgfpathlineto{\pgfqpoint{1.159960in}{0.674142in}}%
\pgfpathlineto{\pgfqpoint{1.164920in}{0.666663in}}%
\pgfpathlineto{\pgfqpoint{1.166160in}{0.666157in}}%
\pgfpathlineto{\pgfqpoint{1.168640in}{0.669331in}}%
\pgfpathlineto{\pgfqpoint{1.171120in}{0.663877in}}%
\pgfpathlineto{\pgfqpoint{1.172360in}{0.662394in}}%
\pgfpathlineto{\pgfqpoint{1.173600in}{0.663852in}}%
\pgfpathlineto{\pgfqpoint{1.176080in}{0.661162in}}%
\pgfpathlineto{\pgfqpoint{1.181040in}{0.661729in}}%
\pgfpathlineto{\pgfqpoint{1.183520in}{0.665645in}}%
\pgfpathlineto{\pgfqpoint{1.184760in}{0.665795in}}%
\pgfpathlineto{\pgfqpoint{1.187240in}{0.656819in}}%
\pgfpathlineto{\pgfqpoint{1.188480in}{0.657884in}}%
\pgfpathlineto{\pgfqpoint{1.192200in}{0.651002in}}%
\pgfpathlineto{\pgfqpoint{1.195920in}{0.652729in}}%
\pgfpathlineto{\pgfqpoint{1.197160in}{0.655554in}}%
\pgfpathlineto{\pgfqpoint{1.199640in}{0.655134in}}%
\pgfpathlineto{\pgfqpoint{1.200880in}{0.655491in}}%
\pgfpathlineto{\pgfqpoint{1.202120in}{0.658108in}}%
\pgfpathlineto{\pgfqpoint{1.204600in}{0.656152in}}%
\pgfpathlineto{\pgfqpoint{1.205840in}{0.658119in}}%
\pgfpathlineto{\pgfqpoint{1.208320in}{0.653679in}}%
\pgfpathlineto{\pgfqpoint{1.210800in}{0.651796in}}%
\pgfpathlineto{\pgfqpoint{1.213280in}{0.648088in}}%
\pgfpathlineto{\pgfqpoint{1.214520in}{0.647720in}}%
\pgfpathlineto{\pgfqpoint{1.217000in}{0.650237in}}%
\pgfpathlineto{\pgfqpoint{1.221960in}{0.644594in}}%
\pgfpathlineto{\pgfqpoint{1.224440in}{0.647120in}}%
\pgfpathlineto{\pgfqpoint{1.225680in}{0.647846in}}%
\pgfpathlineto{\pgfqpoint{1.228160in}{0.642659in}}%
\pgfpathlineto{\pgfqpoint{1.229400in}{0.641957in}}%
\pgfpathlineto{\pgfqpoint{1.230640in}{0.644117in}}%
\pgfpathlineto{\pgfqpoint{1.231880in}{0.643542in}}%
\pgfpathlineto{\pgfqpoint{1.233120in}{0.644519in}}%
\pgfpathlineto{\pgfqpoint{1.235600in}{0.651166in}}%
\pgfpathlineto{\pgfqpoint{1.241800in}{0.649587in}}%
\pgfpathlineto{\pgfqpoint{1.244280in}{0.651825in}}%
\pgfpathlineto{\pgfqpoint{1.249240in}{0.642512in}}%
\pgfpathlineto{\pgfqpoint{1.251720in}{0.645745in}}%
\pgfpathlineto{\pgfqpoint{1.254200in}{0.648626in}}%
\pgfpathlineto{\pgfqpoint{1.256680in}{0.648772in}}%
\pgfpathlineto{\pgfqpoint{1.257920in}{0.651558in}}%
\pgfpathlineto{\pgfqpoint{1.262880in}{0.646617in}}%
\pgfpathlineto{\pgfqpoint{1.265360in}{0.647756in}}%
\pgfpathlineto{\pgfqpoint{1.267840in}{0.651543in}}%
\pgfpathlineto{\pgfqpoint{1.270320in}{0.648619in}}%
\pgfpathlineto{\pgfqpoint{1.275280in}{0.643935in}}%
\pgfpathlineto{\pgfqpoint{1.279000in}{0.635950in}}%
\pgfpathlineto{\pgfqpoint{1.283960in}{0.641294in}}%
\pgfpathlineto{\pgfqpoint{1.286440in}{0.638121in}}%
\pgfpathlineto{\pgfqpoint{1.287680in}{0.637550in}}%
\pgfpathlineto{\pgfqpoint{1.292640in}{0.640858in}}%
\pgfpathlineto{\pgfqpoint{1.295120in}{0.638549in}}%
\pgfpathlineto{\pgfqpoint{1.296360in}{0.638574in}}%
\pgfpathlineto{\pgfqpoint{1.297600in}{0.640556in}}%
\pgfpathlineto{\pgfqpoint{1.300080in}{0.638592in}}%
\pgfpathlineto{\pgfqpoint{1.302560in}{0.640825in}}%
\pgfpathlineto{\pgfqpoint{1.307520in}{0.644305in}}%
\pgfpathlineto{\pgfqpoint{1.308760in}{0.644031in}}%
\pgfpathlineto{\pgfqpoint{1.311240in}{0.637941in}}%
\pgfpathlineto{\pgfqpoint{1.312480in}{0.638374in}}%
\pgfpathlineto{\pgfqpoint{1.316200in}{0.630033in}}%
\pgfpathlineto{\pgfqpoint{1.321160in}{0.634112in}}%
\pgfpathlineto{\pgfqpoint{1.323640in}{0.632007in}}%
\pgfpathlineto{\pgfqpoint{1.327360in}{0.636399in}}%
\pgfpathlineto{\pgfqpoint{1.328600in}{0.634766in}}%
\pgfpathlineto{\pgfqpoint{1.329840in}{0.636476in}}%
\pgfpathlineto{\pgfqpoint{1.336040in}{0.629480in}}%
\pgfpathlineto{\pgfqpoint{1.338520in}{0.629392in}}%
\pgfpathlineto{\pgfqpoint{1.341000in}{0.634672in}}%
\pgfpathlineto{\pgfqpoint{1.342240in}{0.633059in}}%
\pgfpathlineto{\pgfqpoint{1.343480in}{0.633373in}}%
\pgfpathlineto{\pgfqpoint{1.344720in}{0.632096in}}%
\pgfpathlineto{\pgfqpoint{1.345960in}{0.628409in}}%
\pgfpathlineto{\pgfqpoint{1.349680in}{0.631423in}}%
\pgfpathlineto{\pgfqpoint{1.352160in}{0.628401in}}%
\pgfpathlineto{\pgfqpoint{1.353400in}{0.628519in}}%
\pgfpathlineto{\pgfqpoint{1.355880in}{0.629606in}}%
\pgfpathlineto{\pgfqpoint{1.357120in}{0.629201in}}%
\pgfpathlineto{\pgfqpoint{1.362080in}{0.635309in}}%
\pgfpathlineto{\pgfqpoint{1.363320in}{0.634102in}}%
\pgfpathlineto{\pgfqpoint{1.364560in}{0.634656in}}%
\pgfpathlineto{\pgfqpoint{1.365800in}{0.633751in}}%
\pgfpathlineto{\pgfqpoint{1.367040in}{0.635255in}}%
\pgfpathlineto{\pgfqpoint{1.368280in}{0.634923in}}%
\pgfpathlineto{\pgfqpoint{1.372000in}{0.628773in}}%
\pgfpathlineto{\pgfqpoint{1.373240in}{0.628171in}}%
\pgfpathlineto{\pgfqpoint{1.376960in}{0.632893in}}%
\pgfpathlineto{\pgfqpoint{1.379440in}{0.632457in}}%
\pgfpathlineto{\pgfqpoint{1.381920in}{0.636502in}}%
\pgfpathlineto{\pgfqpoint{1.386880in}{0.631137in}}%
\pgfpathlineto{\pgfqpoint{1.391840in}{0.636294in}}%
\pgfpathlineto{\pgfqpoint{1.395560in}{0.633953in}}%
\pgfpathlineto{\pgfqpoint{1.396800in}{0.634333in}}%
\pgfpathlineto{\pgfqpoint{1.398040in}{0.632252in}}%
\pgfpathlineto{\pgfqpoint{1.400520in}{0.631477in}}%
\pgfpathlineto{\pgfqpoint{1.404240in}{0.628384in}}%
\pgfpathlineto{\pgfqpoint{1.407960in}{0.633001in}}%
\pgfpathlineto{\pgfqpoint{1.412920in}{0.630087in}}%
\pgfpathlineto{\pgfqpoint{1.415400in}{0.630458in}}%
\pgfpathlineto{\pgfqpoint{1.416640in}{0.633725in}}%
\pgfpathlineto{\pgfqpoint{1.420360in}{0.631903in}}%
\pgfpathlineto{\pgfqpoint{1.421600in}{0.632885in}}%
\pgfpathlineto{\pgfqpoint{1.424080in}{0.631998in}}%
\pgfpathlineto{\pgfqpoint{1.425320in}{0.633141in}}%
\pgfpathlineto{\pgfqpoint{1.426560in}{0.632309in}}%
\pgfpathlineto{\pgfqpoint{1.432760in}{0.635976in}}%
\pgfpathlineto{\pgfqpoint{1.435240in}{0.629672in}}%
\pgfpathlineto{\pgfqpoint{1.436480in}{0.630713in}}%
\pgfpathlineto{\pgfqpoint{1.440200in}{0.624529in}}%
\pgfpathlineto{\pgfqpoint{1.442680in}{0.627929in}}%
\pgfpathlineto{\pgfqpoint{1.446400in}{0.629476in}}%
\pgfpathlineto{\pgfqpoint{1.447640in}{0.628995in}}%
\pgfpathlineto{\pgfqpoint{1.451360in}{0.632111in}}%
\pgfpathlineto{\pgfqpoint{1.452600in}{0.631239in}}%
\pgfpathlineto{\pgfqpoint{1.453840in}{0.632468in}}%
\pgfpathlineto{\pgfqpoint{1.456320in}{0.630335in}}%
\pgfpathlineto{\pgfqpoint{1.457560in}{0.630169in}}%
\pgfpathlineto{\pgfqpoint{1.460040in}{0.625592in}}%
\pgfpathlineto{\pgfqpoint{1.462520in}{0.627381in}}%
\pgfpathlineto{\pgfqpoint{1.465000in}{0.633700in}}%
\pgfpathlineto{\pgfqpoint{1.466240in}{0.633031in}}%
\pgfpathlineto{\pgfqpoint{1.467480in}{0.633895in}}%
\pgfpathlineto{\pgfqpoint{1.469960in}{0.629590in}}%
\pgfpathlineto{\pgfqpoint{1.473680in}{0.632085in}}%
\pgfpathlineto{\pgfqpoint{1.474920in}{0.630084in}}%
\pgfpathlineto{\pgfqpoint{1.482360in}{0.632786in}}%
\pgfpathlineto{\pgfqpoint{1.484840in}{0.636002in}}%
\pgfpathlineto{\pgfqpoint{1.486080in}{0.636148in}}%
\pgfpathlineto{\pgfqpoint{1.489800in}{0.633547in}}%
\pgfpathlineto{\pgfqpoint{1.492280in}{0.635762in}}%
\pgfpathlineto{\pgfqpoint{1.496000in}{0.629017in}}%
\pgfpathlineto{\pgfqpoint{1.497240in}{0.629506in}}%
\pgfpathlineto{\pgfqpoint{1.500960in}{0.635034in}}%
\pgfpathlineto{\pgfqpoint{1.503440in}{0.636340in}}%
\pgfpathlineto{\pgfqpoint{1.505920in}{0.640004in}}%
\pgfpathlineto{\pgfqpoint{1.510880in}{0.635424in}}%
\pgfpathlineto{\pgfqpoint{1.515840in}{0.637189in}}%
\pgfpathlineto{\pgfqpoint{1.519560in}{0.635712in}}%
\pgfpathlineto{\pgfqpoint{1.520800in}{0.636388in}}%
\pgfpathlineto{\pgfqpoint{1.522040in}{0.633742in}}%
\pgfpathlineto{\pgfqpoint{1.524520in}{0.632780in}}%
\pgfpathlineto{\pgfqpoint{1.525760in}{0.632209in}}%
\pgfpathlineto{\pgfqpoint{1.528240in}{0.629849in}}%
\pgfpathlineto{\pgfqpoint{1.531960in}{0.634201in}}%
\pgfpathlineto{\pgfqpoint{1.534440in}{0.634697in}}%
\pgfpathlineto{\pgfqpoint{1.538160in}{0.635333in}}%
\pgfpathlineto{\pgfqpoint{1.541880in}{0.636260in}}%
\pgfpathlineto{\pgfqpoint{1.544360in}{0.634006in}}%
\pgfpathlineto{\pgfqpoint{1.546840in}{0.633972in}}%
\pgfpathlineto{\pgfqpoint{1.548080in}{0.633397in}}%
\pgfpathlineto{\pgfqpoint{1.549320in}{0.634523in}}%
\pgfpathlineto{\pgfqpoint{1.550560in}{0.633919in}}%
\pgfpathlineto{\pgfqpoint{1.553040in}{0.635227in}}%
\pgfpathlineto{\pgfqpoint{1.555520in}{0.635294in}}%
\pgfpathlineto{\pgfqpoint{1.556760in}{0.635892in}}%
\pgfpathlineto{\pgfqpoint{1.559240in}{0.629765in}}%
\pgfpathlineto{\pgfqpoint{1.560480in}{0.630858in}}%
\pgfpathlineto{\pgfqpoint{1.564200in}{0.626720in}}%
\pgfpathlineto{\pgfqpoint{1.566680in}{0.629923in}}%
\pgfpathlineto{\pgfqpoint{1.569160in}{0.630386in}}%
\pgfpathlineto{\pgfqpoint{1.572880in}{0.631167in}}%
\pgfpathlineto{\pgfqpoint{1.575360in}{0.633475in}}%
\pgfpathlineto{\pgfqpoint{1.576600in}{0.631722in}}%
\pgfpathlineto{\pgfqpoint{1.577840in}{0.633142in}}%
\pgfpathlineto{\pgfqpoint{1.580320in}{0.630993in}}%
\pgfpathlineto{\pgfqpoint{1.581560in}{0.631132in}}%
\pgfpathlineto{\pgfqpoint{1.584040in}{0.627380in}}%
\pgfpathlineto{\pgfqpoint{1.586520in}{0.628908in}}%
\pgfpathlineto{\pgfqpoint{1.589000in}{0.635323in}}%
\pgfpathlineto{\pgfqpoint{1.590240in}{0.634558in}}%
\pgfpathlineto{\pgfqpoint{1.592720in}{0.634717in}}%
\pgfpathlineto{\pgfqpoint{1.593960in}{0.632715in}}%
\pgfpathlineto{\pgfqpoint{1.595200in}{0.633539in}}%
\pgfpathlineto{\pgfqpoint{1.600160in}{0.630780in}}%
\pgfpathlineto{\pgfqpoint{1.606360in}{0.631325in}}%
\pgfpathlineto{\pgfqpoint{1.608840in}{0.634735in}}%
\pgfpathlineto{\pgfqpoint{1.610080in}{0.636043in}}%
\pgfpathlineto{\pgfqpoint{1.613800in}{0.634207in}}%
\pgfpathlineto{\pgfqpoint{1.616280in}{0.635966in}}%
\pgfpathlineto{\pgfqpoint{1.621240in}{0.630929in}}%
\pgfpathlineto{\pgfqpoint{1.624960in}{0.635609in}}%
\pgfpathlineto{\pgfqpoint{1.627440in}{0.635683in}}%
\pgfpathlineto{\pgfqpoint{1.629920in}{0.640446in}}%
\pgfpathlineto{\pgfqpoint{1.634880in}{0.636313in}}%
\pgfpathlineto{\pgfqpoint{1.638600in}{0.636625in}}%
\pgfpathlineto{\pgfqpoint{1.639840in}{0.636606in}}%
\pgfpathlineto{\pgfqpoint{1.643560in}{0.634418in}}%
\pgfpathlineto{\pgfqpoint{1.644800in}{0.635432in}}%
\pgfpathlineto{\pgfqpoint{1.647280in}{0.633957in}}%
\pgfpathlineto{\pgfqpoint{1.652240in}{0.630313in}}%
\pgfpathlineto{\pgfqpoint{1.655960in}{0.633648in}}%
\pgfpathlineto{\pgfqpoint{1.663400in}{0.635292in}}%
\pgfpathlineto{\pgfqpoint{1.664640in}{0.636803in}}%
\pgfpathlineto{\pgfqpoint{1.668360in}{0.633958in}}%
\pgfpathlineto{\pgfqpoint{1.669600in}{0.634319in}}%
\pgfpathlineto{\pgfqpoint{1.670840in}{0.633243in}}%
\pgfpathlineto{\pgfqpoint{1.674560in}{0.634594in}}%
\pgfpathlineto{\pgfqpoint{1.680760in}{0.635151in}}%
\pgfpathlineto{\pgfqpoint{1.683240in}{0.629187in}}%
\pgfpathlineto{\pgfqpoint{1.684480in}{0.630445in}}%
\pgfpathlineto{\pgfqpoint{1.688200in}{0.628182in}}%
\pgfpathlineto{\pgfqpoint{1.690680in}{0.631044in}}%
\pgfpathlineto{\pgfqpoint{1.696880in}{0.632489in}}%
\pgfpathlineto{\pgfqpoint{1.699360in}{0.634450in}}%
\pgfpathlineto{\pgfqpoint{1.700600in}{0.633150in}}%
\pgfpathlineto{\pgfqpoint{1.701840in}{0.634929in}}%
\pgfpathlineto{\pgfqpoint{1.704320in}{0.632812in}}%
\pgfpathlineto{\pgfqpoint{1.705560in}{0.632422in}}%
\pgfpathlineto{\pgfqpoint{1.708040in}{0.628362in}}%
\pgfpathlineto{\pgfqpoint{1.710520in}{0.630264in}}%
\pgfpathlineto{\pgfqpoint{1.713000in}{0.635732in}}%
\pgfpathlineto{\pgfqpoint{1.714240in}{0.634510in}}%
\pgfpathlineto{\pgfqpoint{1.716720in}{0.635686in}}%
\pgfpathlineto{\pgfqpoint{1.717960in}{0.633991in}}%
\pgfpathlineto{\pgfqpoint{1.719200in}{0.635407in}}%
\pgfpathlineto{\pgfqpoint{1.724160in}{0.633520in}}%
\pgfpathlineto{\pgfqpoint{1.725400in}{0.633148in}}%
\pgfpathlineto{\pgfqpoint{1.726640in}{0.634392in}}%
\pgfpathlineto{\pgfqpoint{1.729120in}{0.632298in}}%
\pgfpathlineto{\pgfqpoint{1.730360in}{0.633162in}}%
\pgfpathlineto{\pgfqpoint{1.732840in}{0.636149in}}%
\pgfpathlineto{\pgfqpoint{1.734080in}{0.637554in}}%
\pgfpathlineto{\pgfqpoint{1.737800in}{0.634959in}}%
\pgfpathlineto{\pgfqpoint{1.740280in}{0.636278in}}%
\pgfpathlineto{\pgfqpoint{1.744000in}{0.631747in}}%
\pgfpathlineto{\pgfqpoint{1.753920in}{0.639731in}}%
\pgfpathlineto{\pgfqpoint{1.756400in}{0.636470in}}%
\pgfpathlineto{\pgfqpoint{1.757640in}{0.636353in}}%
\pgfpathlineto{\pgfqpoint{1.758880in}{0.635090in}}%
\pgfpathlineto{\pgfqpoint{1.763840in}{0.635823in}}%
\pgfpathlineto{\pgfqpoint{1.766320in}{0.633665in}}%
\pgfpathlineto{\pgfqpoint{1.768800in}{0.635919in}}%
\pgfpathlineto{\pgfqpoint{1.776240in}{0.631897in}}%
\pgfpathlineto{\pgfqpoint{1.778720in}{0.634353in}}%
\pgfpathlineto{\pgfqpoint{1.787400in}{0.635399in}}%
\pgfpathlineto{\pgfqpoint{1.788640in}{0.637812in}}%
\pgfpathlineto{\pgfqpoint{1.797320in}{0.635369in}}%
\pgfpathlineto{\pgfqpoint{1.801040in}{0.636166in}}%
\pgfpathlineto{\pgfqpoint{1.803520in}{0.635267in}}%
\pgfpathlineto{\pgfqpoint{1.804760in}{0.636162in}}%
\pgfpathlineto{\pgfqpoint{1.807240in}{0.630263in}}%
\pgfpathlineto{\pgfqpoint{1.808480in}{0.631039in}}%
\pgfpathlineto{\pgfqpoint{1.812200in}{0.628478in}}%
\pgfpathlineto{\pgfqpoint{1.814680in}{0.631834in}}%
\pgfpathlineto{\pgfqpoint{1.819640in}{0.632709in}}%
\pgfpathlineto{\pgfqpoint{1.823360in}{0.636033in}}%
\pgfpathlineto{\pgfqpoint{1.824600in}{0.634371in}}%
\pgfpathlineto{\pgfqpoint{1.825840in}{0.635842in}}%
\pgfpathlineto{\pgfqpoint{1.828320in}{0.633582in}}%
\pgfpathlineto{\pgfqpoint{1.829560in}{0.632865in}}%
\pgfpathlineto{\pgfqpoint{1.832040in}{0.629255in}}%
\pgfpathlineto{\pgfqpoint{1.834520in}{0.632122in}}%
\pgfpathlineto{\pgfqpoint{1.837000in}{0.637504in}}%
\pgfpathlineto{\pgfqpoint{1.838240in}{0.636164in}}%
\pgfpathlineto{\pgfqpoint{1.840720in}{0.637525in}}%
\pgfpathlineto{\pgfqpoint{1.841960in}{0.636506in}}%
\pgfpathlineto{\pgfqpoint{1.843200in}{0.637488in}}%
\pgfpathlineto{\pgfqpoint{1.854360in}{0.632800in}}%
\pgfpathlineto{\pgfqpoint{1.856840in}{0.636063in}}%
\pgfpathlineto{\pgfqpoint{1.858080in}{0.637238in}}%
\pgfpathlineto{\pgfqpoint{1.861800in}{0.635148in}}%
\pgfpathlineto{\pgfqpoint{1.864280in}{0.636680in}}%
\pgfpathlineto{\pgfqpoint{1.868000in}{0.632756in}}%
\pgfpathlineto{\pgfqpoint{1.871720in}{0.638025in}}%
\pgfpathlineto{\pgfqpoint{1.872960in}{0.639059in}}%
\pgfpathlineto{\pgfqpoint{1.875440in}{0.638079in}}%
\pgfpathlineto{\pgfqpoint{1.877920in}{0.641601in}}%
\pgfpathlineto{\pgfqpoint{1.882880in}{0.638375in}}%
\pgfpathlineto{\pgfqpoint{1.887840in}{0.638082in}}%
\pgfpathlineto{\pgfqpoint{1.890320in}{0.635055in}}%
\pgfpathlineto{\pgfqpoint{1.891560in}{0.635394in}}%
\pgfpathlineto{\pgfqpoint{1.894040in}{0.637292in}}%
\pgfpathlineto{\pgfqpoint{1.900240in}{0.634528in}}%
\pgfpathlineto{\pgfqpoint{1.902720in}{0.637023in}}%
\pgfpathlineto{\pgfqpoint{1.906440in}{0.637888in}}%
\pgfpathlineto{\pgfqpoint{1.911400in}{0.637675in}}%
\pgfpathlineto{\pgfqpoint{1.912640in}{0.639934in}}%
\pgfpathlineto{\pgfqpoint{1.916360in}{0.638686in}}%
\pgfpathlineto{\pgfqpoint{1.917600in}{0.639908in}}%
\pgfpathlineto{\pgfqpoint{1.920080in}{0.638120in}}%
\pgfpathlineto{\pgfqpoint{1.922560in}{0.638112in}}%
\pgfpathlineto{\pgfqpoint{1.925040in}{0.639300in}}%
\pgfpathlineto{\pgfqpoint{1.926280in}{0.637524in}}%
\pgfpathlineto{\pgfqpoint{1.928760in}{0.639429in}}%
\pgfpathlineto{\pgfqpoint{1.930000in}{0.637633in}}%
\pgfpathlineto{\pgfqpoint{1.931240in}{0.633535in}}%
\pgfpathlineto{\pgfqpoint{1.933720in}{0.633760in}}%
\pgfpathlineto{\pgfqpoint{1.936200in}{0.630838in}}%
\pgfpathlineto{\pgfqpoint{1.938680in}{0.632877in}}%
\pgfpathlineto{\pgfqpoint{1.944880in}{0.635389in}}%
\pgfpathlineto{\pgfqpoint{1.947360in}{0.637656in}}%
\pgfpathlineto{\pgfqpoint{1.948600in}{0.636426in}}%
\pgfpathlineto{\pgfqpoint{1.949840in}{0.637554in}}%
\pgfpathlineto{\pgfqpoint{1.952320in}{0.635685in}}%
\pgfpathlineto{\pgfqpoint{1.953560in}{0.635098in}}%
\pgfpathlineto{\pgfqpoint{1.956040in}{0.631116in}}%
\pgfpathlineto{\pgfqpoint{1.961000in}{0.639179in}}%
\pgfpathlineto{\pgfqpoint{1.962240in}{0.637470in}}%
\pgfpathlineto{\pgfqpoint{1.967200in}{0.638575in}}%
\pgfpathlineto{\pgfqpoint{1.969680in}{0.638447in}}%
\pgfpathlineto{\pgfqpoint{1.972160in}{0.637196in}}%
\pgfpathlineto{\pgfqpoint{1.977120in}{0.634159in}}%
\pgfpathlineto{\pgfqpoint{1.978360in}{0.635008in}}%
\pgfpathlineto{\pgfqpoint{1.980840in}{0.638640in}}%
\pgfpathlineto{\pgfqpoint{1.982080in}{0.639375in}}%
\pgfpathlineto{\pgfqpoint{1.984560in}{0.637536in}}%
\pgfpathlineto{\pgfqpoint{1.985800in}{0.636992in}}%
\pgfpathlineto{\pgfqpoint{1.988280in}{0.639759in}}%
\pgfpathlineto{\pgfqpoint{1.992000in}{0.635717in}}%
\pgfpathlineto{\pgfqpoint{1.994480in}{0.638586in}}%
\pgfpathlineto{\pgfqpoint{1.998200in}{0.640383in}}%
\pgfpathlineto{\pgfqpoint{1.999440in}{0.640306in}}%
\pgfpathlineto{\pgfqpoint{2.001920in}{0.643900in}}%
\pgfpathlineto{\pgfqpoint{2.006880in}{0.640383in}}%
\pgfpathlineto{\pgfqpoint{2.010600in}{0.640218in}}%
\pgfpathlineto{\pgfqpoint{2.013080in}{0.638149in}}%
\pgfpathlineto{\pgfqpoint{2.014320in}{0.636783in}}%
\pgfpathlineto{\pgfqpoint{2.015560in}{0.637116in}}%
\pgfpathlineto{\pgfqpoint{2.018040in}{0.638804in}}%
\pgfpathlineto{\pgfqpoint{2.024240in}{0.636453in}}%
\pgfpathlineto{\pgfqpoint{2.026720in}{0.638128in}}%
\pgfpathlineto{\pgfqpoint{2.031680in}{0.638937in}}%
\pgfpathlineto{\pgfqpoint{2.035400in}{0.638582in}}%
\pgfpathlineto{\pgfqpoint{2.036640in}{0.640457in}}%
\pgfpathlineto{\pgfqpoint{2.040360in}{0.638665in}}%
\pgfpathlineto{\pgfqpoint{2.041600in}{0.639790in}}%
\pgfpathlineto{\pgfqpoint{2.044080in}{0.638350in}}%
\pgfpathlineto{\pgfqpoint{2.046560in}{0.638871in}}%
\pgfpathlineto{\pgfqpoint{2.049040in}{0.639610in}}%
\pgfpathlineto{\pgfqpoint{2.050280in}{0.637665in}}%
\pgfpathlineto{\pgfqpoint{2.052760in}{0.639405in}}%
\pgfpathlineto{\pgfqpoint{2.054000in}{0.637601in}}%
\pgfpathlineto{\pgfqpoint{2.055240in}{0.633571in}}%
\pgfpathlineto{\pgfqpoint{2.057720in}{0.633659in}}%
\pgfpathlineto{\pgfqpoint{2.060200in}{0.631164in}}%
\pgfpathlineto{\pgfqpoint{2.063920in}{0.633590in}}%
\pgfpathlineto{\pgfqpoint{2.065160in}{0.632801in}}%
\pgfpathlineto{\pgfqpoint{2.071360in}{0.636650in}}%
\pgfpathlineto{\pgfqpoint{2.072600in}{0.634712in}}%
\pgfpathlineto{\pgfqpoint{2.073840in}{0.635245in}}%
\pgfpathlineto{\pgfqpoint{2.076320in}{0.633369in}}%
\pgfpathlineto{\pgfqpoint{2.077560in}{0.633127in}}%
\pgfpathlineto{\pgfqpoint{2.080040in}{0.629381in}}%
\pgfpathlineto{\pgfqpoint{2.085000in}{0.637289in}}%
\pgfpathlineto{\pgfqpoint{2.086240in}{0.635446in}}%
\pgfpathlineto{\pgfqpoint{2.089960in}{0.635864in}}%
\pgfpathlineto{\pgfqpoint{2.093680in}{0.636008in}}%
\pgfpathlineto{\pgfqpoint{2.096160in}{0.634634in}}%
\pgfpathlineto{\pgfqpoint{2.101120in}{0.631869in}}%
\pgfpathlineto{\pgfqpoint{2.106080in}{0.636801in}}%
\pgfpathlineto{\pgfqpoint{2.108560in}{0.635452in}}%
\pgfpathlineto{\pgfqpoint{2.109800in}{0.635052in}}%
\pgfpathlineto{\pgfqpoint{2.112280in}{0.638151in}}%
\pgfpathlineto{\pgfqpoint{2.116000in}{0.634494in}}%
\pgfpathlineto{\pgfqpoint{2.120960in}{0.639361in}}%
\pgfpathlineto{\pgfqpoint{2.123440in}{0.638459in}}%
\pgfpathlineto{\pgfqpoint{2.125920in}{0.641787in}}%
\pgfpathlineto{\pgfqpoint{2.128400in}{0.639675in}}%
\pgfpathlineto{\pgfqpoint{2.129640in}{0.640029in}}%
\pgfpathlineto{\pgfqpoint{2.132120in}{0.639075in}}%
\pgfpathlineto{\pgfqpoint{2.134600in}{0.639634in}}%
\pgfpathlineto{\pgfqpoint{2.135840in}{0.639317in}}%
\pgfpathlineto{\pgfqpoint{2.138320in}{0.636417in}}%
\pgfpathlineto{\pgfqpoint{2.148240in}{0.635712in}}%
\pgfpathlineto{\pgfqpoint{2.150720in}{0.637275in}}%
\pgfpathlineto{\pgfqpoint{2.154440in}{0.639134in}}%
\pgfpathlineto{\pgfqpoint{2.156920in}{0.638347in}}%
\pgfpathlineto{\pgfqpoint{2.159400in}{0.638574in}}%
\pgfpathlineto{\pgfqpoint{2.160640in}{0.640714in}}%
\pgfpathlineto{\pgfqpoint{2.169320in}{0.640106in}}%
\pgfpathlineto{\pgfqpoint{2.171800in}{0.639953in}}%
\pgfpathlineto{\pgfqpoint{2.173040in}{0.639713in}}%
\pgfpathlineto{\pgfqpoint{2.174280in}{0.637687in}}%
\pgfpathlineto{\pgfqpoint{2.176760in}{0.639868in}}%
\pgfpathlineto{\pgfqpoint{2.178000in}{0.638526in}}%
\pgfpathlineto{\pgfqpoint{2.179240in}{0.634558in}}%
\pgfpathlineto{\pgfqpoint{2.181720in}{0.634538in}}%
\pgfpathlineto{\pgfqpoint{2.184200in}{0.632769in}}%
\pgfpathlineto{\pgfqpoint{2.187920in}{0.634027in}}%
\pgfpathlineto{\pgfqpoint{2.190400in}{0.633662in}}%
\pgfpathlineto{\pgfqpoint{2.194120in}{0.635654in}}%
\pgfpathlineto{\pgfqpoint{2.195360in}{0.636293in}}%
\pgfpathlineto{\pgfqpoint{2.196600in}{0.634460in}}%
\pgfpathlineto{\pgfqpoint{2.197840in}{0.634756in}}%
\pgfpathlineto{\pgfqpoint{2.200320in}{0.633032in}}%
\pgfpathlineto{\pgfqpoint{2.201560in}{0.632729in}}%
\pgfpathlineto{\pgfqpoint{2.204040in}{0.630139in}}%
\pgfpathlineto{\pgfqpoint{2.209000in}{0.637679in}}%
\pgfpathlineto{\pgfqpoint{2.210240in}{0.636322in}}%
\pgfpathlineto{\pgfqpoint{2.213960in}{0.636970in}}%
\pgfpathlineto{\pgfqpoint{2.220160in}{0.635944in}}%
\pgfpathlineto{\pgfqpoint{2.225120in}{0.632876in}}%
\pgfpathlineto{\pgfqpoint{2.230080in}{0.636523in}}%
\pgfpathlineto{\pgfqpoint{2.232560in}{0.634778in}}%
\pgfpathlineto{\pgfqpoint{2.233800in}{0.634805in}}%
\pgfpathlineto{\pgfqpoint{2.236280in}{0.637406in}}%
\pgfpathlineto{\pgfqpoint{2.240000in}{0.634304in}}%
\pgfpathlineto{\pgfqpoint{2.243720in}{0.638819in}}%
\pgfpathlineto{\pgfqpoint{2.244960in}{0.640143in}}%
\pgfpathlineto{\pgfqpoint{2.247440in}{0.639051in}}%
\pgfpathlineto{\pgfqpoint{2.249920in}{0.642171in}}%
\pgfpathlineto{\pgfqpoint{2.254880in}{0.640442in}}%
\pgfpathlineto{\pgfqpoint{2.259840in}{0.640452in}}%
\pgfpathlineto{\pgfqpoint{2.262320in}{0.638088in}}%
\pgfpathlineto{\pgfqpoint{2.268520in}{0.637887in}}%
\pgfpathlineto{\pgfqpoint{2.271000in}{0.636273in}}%
\pgfpathlineto{\pgfqpoint{2.282160in}{0.640230in}}%
\pgfpathlineto{\pgfqpoint{2.283400in}{0.639849in}}%
\pgfpathlineto{\pgfqpoint{2.284640in}{0.642062in}}%
\pgfpathlineto{\pgfqpoint{2.292080in}{0.640699in}}%
\pgfpathlineto{\pgfqpoint{2.297040in}{0.641848in}}%
\pgfpathlineto{\pgfqpoint{2.298280in}{0.639786in}}%
\pgfpathlineto{\pgfqpoint{2.300760in}{0.641675in}}%
\pgfpathlineto{\pgfqpoint{2.302000in}{0.640262in}}%
\pgfpathlineto{\pgfqpoint{2.303240in}{0.636808in}}%
\pgfpathlineto{\pgfqpoint{2.304480in}{0.637581in}}%
\pgfpathlineto{\pgfqpoint{2.308200in}{0.634669in}}%
\pgfpathlineto{\pgfqpoint{2.311920in}{0.636381in}}%
\pgfpathlineto{\pgfqpoint{2.315640in}{0.635669in}}%
\pgfpathlineto{\pgfqpoint{2.319360in}{0.638567in}}%
\pgfpathlineto{\pgfqpoint{2.320600in}{0.636427in}}%
\pgfpathlineto{\pgfqpoint{2.321840in}{0.636528in}}%
\pgfpathlineto{\pgfqpoint{2.324320in}{0.635185in}}%
\pgfpathlineto{\pgfqpoint{2.328040in}{0.632518in}}%
\pgfpathlineto{\pgfqpoint{2.333000in}{0.638653in}}%
\pgfpathlineto{\pgfqpoint{2.334240in}{0.637556in}}%
\pgfpathlineto{\pgfqpoint{2.337960in}{0.638823in}}%
\pgfpathlineto{\pgfqpoint{2.346640in}{0.636799in}}%
\pgfpathlineto{\pgfqpoint{2.349120in}{0.634413in}}%
\pgfpathlineto{\pgfqpoint{2.354080in}{0.638253in}}%
\pgfpathlineto{\pgfqpoint{2.356560in}{0.636882in}}%
\pgfpathlineto{\pgfqpoint{2.357800in}{0.636781in}}%
\pgfpathlineto{\pgfqpoint{2.360280in}{0.639434in}}%
\pgfpathlineto{\pgfqpoint{2.365240in}{0.637978in}}%
\pgfpathlineto{\pgfqpoint{2.370200in}{0.640530in}}%
\pgfpathlineto{\pgfqpoint{2.371440in}{0.639986in}}%
\pgfpathlineto{\pgfqpoint{2.375160in}{0.642764in}}%
\pgfpathlineto{\pgfqpoint{2.377640in}{0.643292in}}%
\pgfpathlineto{\pgfqpoint{2.383840in}{0.642331in}}%
\pgfpathlineto{\pgfqpoint{2.386320in}{0.640064in}}%
\pgfpathlineto{\pgfqpoint{2.390040in}{0.640352in}}%
\pgfpathlineto{\pgfqpoint{2.393760in}{0.638442in}}%
\pgfpathlineto{\pgfqpoint{2.396240in}{0.638059in}}%
\pgfpathlineto{\pgfqpoint{2.398720in}{0.640260in}}%
\pgfpathlineto{\pgfqpoint{2.404920in}{0.642073in}}%
\pgfpathlineto{\pgfqpoint{2.407400in}{0.641403in}}%
\pgfpathlineto{\pgfqpoint{2.408640in}{0.643584in}}%
\pgfpathlineto{\pgfqpoint{2.412360in}{0.641681in}}%
\pgfpathlineto{\pgfqpoint{2.413600in}{0.642590in}}%
\pgfpathlineto{\pgfqpoint{2.414840in}{0.641670in}}%
\pgfpathlineto{\pgfqpoint{2.419800in}{0.642662in}}%
\pgfpathlineto{\pgfqpoint{2.421040in}{0.642768in}}%
\pgfpathlineto{\pgfqpoint{2.422280in}{0.640624in}}%
\pgfpathlineto{\pgfqpoint{2.424760in}{0.642672in}}%
\pgfpathlineto{\pgfqpoint{2.426000in}{0.641732in}}%
\pgfpathlineto{\pgfqpoint{2.427240in}{0.638568in}}%
\pgfpathlineto{\pgfqpoint{2.428480in}{0.639587in}}%
\pgfpathlineto{\pgfqpoint{2.433440in}{0.636755in}}%
\pgfpathlineto{\pgfqpoint{2.437160in}{0.637158in}}%
\pgfpathlineto{\pgfqpoint{2.439640in}{0.637056in}}%
\pgfpathlineto{\pgfqpoint{2.443360in}{0.638847in}}%
\pgfpathlineto{\pgfqpoint{2.444600in}{0.637210in}}%
\pgfpathlineto{\pgfqpoint{2.445840in}{0.637658in}}%
\pgfpathlineto{\pgfqpoint{2.448320in}{0.636446in}}%
\pgfpathlineto{\pgfqpoint{2.452040in}{0.634242in}}%
\pgfpathlineto{\pgfqpoint{2.457000in}{0.639488in}}%
\pgfpathlineto{\pgfqpoint{2.458240in}{0.637857in}}%
\pgfpathlineto{\pgfqpoint{2.463200in}{0.638454in}}%
\pgfpathlineto{\pgfqpoint{2.465680in}{0.638519in}}%
\pgfpathlineto{\pgfqpoint{2.466920in}{0.637030in}}%
\pgfpathlineto{\pgfqpoint{2.469400in}{0.637080in}}%
\pgfpathlineto{\pgfqpoint{2.470640in}{0.637196in}}%
\pgfpathlineto{\pgfqpoint{2.473120in}{0.635185in}}%
\pgfpathlineto{\pgfqpoint{2.475600in}{0.637965in}}%
\pgfpathlineto{\pgfqpoint{2.476840in}{0.637105in}}%
\pgfpathlineto{\pgfqpoint{2.478080in}{0.638004in}}%
\pgfpathlineto{\pgfqpoint{2.480560in}{0.636201in}}%
\pgfpathlineto{\pgfqpoint{2.483040in}{0.636980in}}%
\pgfpathlineto{\pgfqpoint{2.484280in}{0.638534in}}%
\pgfpathlineto{\pgfqpoint{2.486760in}{0.637585in}}%
\pgfpathlineto{\pgfqpoint{2.489240in}{0.635215in}}%
\pgfpathlineto{\pgfqpoint{2.499160in}{0.639598in}}%
\pgfpathlineto{\pgfqpoint{2.500400in}{0.639076in}}%
\pgfpathlineto{\pgfqpoint{2.504120in}{0.640336in}}%
\pgfpathlineto{\pgfqpoint{2.506600in}{0.640507in}}%
\pgfpathlineto{\pgfqpoint{2.511560in}{0.637580in}}%
\pgfpathlineto{\pgfqpoint{2.515280in}{0.637527in}}%
\pgfpathlineto{\pgfqpoint{2.520240in}{0.636040in}}%
\pgfpathlineto{\pgfqpoint{2.522720in}{0.637448in}}%
\pgfpathlineto{\pgfqpoint{2.528920in}{0.639655in}}%
\pgfpathlineto{\pgfqpoint{2.531400in}{0.638800in}}%
\pgfpathlineto{\pgfqpoint{2.532640in}{0.641201in}}%
\pgfpathlineto{\pgfqpoint{2.536360in}{0.640128in}}%
\pgfpathlineto{\pgfqpoint{2.538840in}{0.640253in}}%
\pgfpathlineto{\pgfqpoint{2.542560in}{0.642154in}}%
\pgfpathlineto{\pgfqpoint{2.546280in}{0.639031in}}%
\pgfpathlineto{\pgfqpoint{2.548760in}{0.641043in}}%
\pgfpathlineto{\pgfqpoint{2.550000in}{0.640296in}}%
\pgfpathlineto{\pgfqpoint{2.551240in}{0.637370in}}%
\pgfpathlineto{\pgfqpoint{2.552480in}{0.638772in}}%
\pgfpathlineto{\pgfqpoint{2.554960in}{0.636161in}}%
\pgfpathlineto{\pgfqpoint{2.556200in}{0.634644in}}%
\pgfpathlineto{\pgfqpoint{2.561160in}{0.635832in}}%
\pgfpathlineto{\pgfqpoint{2.569840in}{0.635644in}}%
\pgfpathlineto{\pgfqpoint{2.574800in}{0.633307in}}%
\pgfpathlineto{\pgfqpoint{2.576040in}{0.632611in}}%
\pgfpathlineto{\pgfqpoint{2.581000in}{0.638296in}}%
\pgfpathlineto{\pgfqpoint{2.583480in}{0.636554in}}%
\pgfpathlineto{\pgfqpoint{2.585960in}{0.637005in}}%
\pgfpathlineto{\pgfqpoint{2.589680in}{0.636666in}}%
\pgfpathlineto{\pgfqpoint{2.590920in}{0.635310in}}%
\pgfpathlineto{\pgfqpoint{2.592160in}{0.636114in}}%
\pgfpathlineto{\pgfqpoint{2.597120in}{0.633631in}}%
\pgfpathlineto{\pgfqpoint{2.599600in}{0.636455in}}%
\pgfpathlineto{\pgfqpoint{2.600840in}{0.635318in}}%
\pgfpathlineto{\pgfqpoint{2.602080in}{0.636004in}}%
\pgfpathlineto{\pgfqpoint{2.604560in}{0.634400in}}%
\pgfpathlineto{\pgfqpoint{2.605800in}{0.634108in}}%
\pgfpathlineto{\pgfqpoint{2.609520in}{0.636703in}}%
\pgfpathlineto{\pgfqpoint{2.614480in}{0.634835in}}%
\pgfpathlineto{\pgfqpoint{2.616960in}{0.637150in}}%
\pgfpathlineto{\pgfqpoint{2.619440in}{0.636225in}}%
\pgfpathlineto{\pgfqpoint{2.623160in}{0.638334in}}%
\pgfpathlineto{\pgfqpoint{2.624400in}{0.637829in}}%
\pgfpathlineto{\pgfqpoint{2.626880in}{0.639074in}}%
\pgfpathlineto{\pgfqpoint{2.631840in}{0.638566in}}%
\pgfpathlineto{\pgfqpoint{2.634320in}{0.635846in}}%
\pgfpathlineto{\pgfqpoint{2.641760in}{0.636023in}}%
\pgfpathlineto{\pgfqpoint{2.644240in}{0.635216in}}%
\pgfpathlineto{\pgfqpoint{2.646720in}{0.636458in}}%
\pgfpathlineto{\pgfqpoint{2.649200in}{0.637727in}}%
\pgfpathlineto{\pgfqpoint{2.650440in}{0.639155in}}%
\pgfpathlineto{\pgfqpoint{2.651680in}{0.638642in}}%
\pgfpathlineto{\pgfqpoint{2.654160in}{0.639033in}}%
\pgfpathlineto{\pgfqpoint{2.655400in}{0.638809in}}%
\pgfpathlineto{\pgfqpoint{2.656640in}{0.640758in}}%
\pgfpathlineto{\pgfqpoint{2.660360in}{0.639069in}}%
\pgfpathlineto{\pgfqpoint{2.666560in}{0.641574in}}%
\pgfpathlineto{\pgfqpoint{2.671520in}{0.639201in}}%
\pgfpathlineto{\pgfqpoint{2.672760in}{0.640094in}}%
\pgfpathlineto{\pgfqpoint{2.674000in}{0.639318in}}%
\pgfpathlineto{\pgfqpoint{2.675240in}{0.636740in}}%
\pgfpathlineto{\pgfqpoint{2.676480in}{0.638008in}}%
\pgfpathlineto{\pgfqpoint{2.681440in}{0.633633in}}%
\pgfpathlineto{\pgfqpoint{2.686400in}{0.634468in}}%
\pgfpathlineto{\pgfqpoint{2.693840in}{0.632968in}}%
\pgfpathlineto{\pgfqpoint{2.696320in}{0.631925in}}%
\pgfpathlineto{\pgfqpoint{2.700040in}{0.630272in}}%
\pgfpathlineto{\pgfqpoint{2.705000in}{0.635626in}}%
\pgfpathlineto{\pgfqpoint{2.706240in}{0.634213in}}%
\pgfpathlineto{\pgfqpoint{2.711200in}{0.635133in}}%
\pgfpathlineto{\pgfqpoint{2.712440in}{0.634220in}}%
\pgfpathlineto{\pgfqpoint{2.713680in}{0.634757in}}%
\pgfpathlineto{\pgfqpoint{2.714920in}{0.633569in}}%
\pgfpathlineto{\pgfqpoint{2.716160in}{0.634346in}}%
\pgfpathlineto{\pgfqpoint{2.721120in}{0.631427in}}%
\pgfpathlineto{\pgfqpoint{2.724840in}{0.633183in}}%
\pgfpathlineto{\pgfqpoint{2.726080in}{0.633740in}}%
\pgfpathlineto{\pgfqpoint{2.728560in}{0.632299in}}%
\pgfpathlineto{\pgfqpoint{2.729800in}{0.631970in}}%
\pgfpathlineto{\pgfqpoint{2.733520in}{0.634495in}}%
\pgfpathlineto{\pgfqpoint{2.736000in}{0.632874in}}%
\pgfpathlineto{\pgfqpoint{2.737240in}{0.631732in}}%
\pgfpathlineto{\pgfqpoint{2.739720in}{0.633437in}}%
\pgfpathlineto{\pgfqpoint{2.742200in}{0.634672in}}%
\pgfpathlineto{\pgfqpoint{2.744680in}{0.635874in}}%
\pgfpathlineto{\pgfqpoint{2.745920in}{0.637667in}}%
\pgfpathlineto{\pgfqpoint{2.749640in}{0.637401in}}%
\pgfpathlineto{\pgfqpoint{2.754600in}{0.637673in}}%
\pgfpathlineto{\pgfqpoint{2.755840in}{0.637119in}}%
\pgfpathlineto{\pgfqpoint{2.758320in}{0.634044in}}%
\pgfpathlineto{\pgfqpoint{2.765760in}{0.634086in}}%
\pgfpathlineto{\pgfqpoint{2.767000in}{0.632991in}}%
\pgfpathlineto{\pgfqpoint{2.773200in}{0.635881in}}%
\pgfpathlineto{\pgfqpoint{2.776920in}{0.637244in}}%
\pgfpathlineto{\pgfqpoint{2.779400in}{0.636237in}}%
\pgfpathlineto{\pgfqpoint{2.780640in}{0.638139in}}%
\pgfpathlineto{\pgfqpoint{2.783120in}{0.636382in}}%
\pgfpathlineto{\pgfqpoint{2.788080in}{0.637271in}}%
\pgfpathlineto{\pgfqpoint{2.791800in}{0.637471in}}%
\pgfpathlineto{\pgfqpoint{2.793040in}{0.637731in}}%
\pgfpathlineto{\pgfqpoint{2.794280in}{0.635640in}}%
\pgfpathlineto{\pgfqpoint{2.798000in}{0.636709in}}%
\pgfpathlineto{\pgfqpoint{2.799240in}{0.634345in}}%
\pgfpathlineto{\pgfqpoint{2.800480in}{0.635757in}}%
\pgfpathlineto{\pgfqpoint{2.805440in}{0.630577in}}%
\pgfpathlineto{\pgfqpoint{2.807920in}{0.630811in}}%
\pgfpathlineto{\pgfqpoint{2.811640in}{0.631079in}}%
\pgfpathlineto{\pgfqpoint{2.814120in}{0.632212in}}%
\pgfpathlineto{\pgfqpoint{2.815360in}{0.632479in}}%
\pgfpathlineto{\pgfqpoint{2.819080in}{0.629654in}}%
\pgfpathlineto{\pgfqpoint{2.822800in}{0.629845in}}%
\pgfpathlineto{\pgfqpoint{2.824040in}{0.629239in}}%
\pgfpathlineto{\pgfqpoint{2.829000in}{0.634408in}}%
\pgfpathlineto{\pgfqpoint{2.830240in}{0.633218in}}%
\pgfpathlineto{\pgfqpoint{2.835200in}{0.635150in}}%
\pgfpathlineto{\pgfqpoint{2.836440in}{0.634061in}}%
\pgfpathlineto{\pgfqpoint{2.837680in}{0.634576in}}%
\pgfpathlineto{\pgfqpoint{2.838920in}{0.633378in}}%
\pgfpathlineto{\pgfqpoint{2.840160in}{0.634598in}}%
\pgfpathlineto{\pgfqpoint{2.841400in}{0.633799in}}%
\pgfpathlineto{\pgfqpoint{2.842640in}{0.634376in}}%
\pgfpathlineto{\pgfqpoint{2.845120in}{0.631645in}}%
\pgfpathlineto{\pgfqpoint{2.847600in}{0.634127in}}%
\pgfpathlineto{\pgfqpoint{2.848840in}{0.633192in}}%
\pgfpathlineto{\pgfqpoint{2.850080in}{0.633795in}}%
\pgfpathlineto{\pgfqpoint{2.852560in}{0.632041in}}%
\pgfpathlineto{\pgfqpoint{2.855040in}{0.632600in}}%
\pgfpathlineto{\pgfqpoint{2.857520in}{0.634285in}}%
\pgfpathlineto{\pgfqpoint{2.860000in}{0.632931in}}%
\pgfpathlineto{\pgfqpoint{2.861240in}{0.631409in}}%
\pgfpathlineto{\pgfqpoint{2.866200in}{0.634465in}}%
\pgfpathlineto{\pgfqpoint{2.868680in}{0.635615in}}%
\pgfpathlineto{\pgfqpoint{2.869920in}{0.637214in}}%
\pgfpathlineto{\pgfqpoint{2.872400in}{0.636126in}}%
\pgfpathlineto{\pgfqpoint{2.874880in}{0.637229in}}%
\pgfpathlineto{\pgfqpoint{2.879840in}{0.637436in}}%
\pgfpathlineto{\pgfqpoint{2.882320in}{0.634413in}}%
\pgfpathlineto{\pgfqpoint{2.888520in}{0.634848in}}%
\pgfpathlineto{\pgfqpoint{2.892240in}{0.633670in}}%
\pgfpathlineto{\pgfqpoint{2.894720in}{0.634914in}}%
\pgfpathlineto{\pgfqpoint{2.897200in}{0.635565in}}%
\pgfpathlineto{\pgfqpoint{2.898440in}{0.636679in}}%
\pgfpathlineto{\pgfqpoint{2.899680in}{0.636173in}}%
\pgfpathlineto{\pgfqpoint{2.902160in}{0.636612in}}%
\pgfpathlineto{\pgfqpoint{2.903400in}{0.636297in}}%
\pgfpathlineto{\pgfqpoint{2.904640in}{0.638183in}}%
\pgfpathlineto{\pgfqpoint{2.907120in}{0.635758in}}%
\pgfpathlineto{\pgfqpoint{2.913320in}{0.638065in}}%
\pgfpathlineto{\pgfqpoint{2.915800in}{0.638451in}}%
\pgfpathlineto{\pgfqpoint{2.917040in}{0.638861in}}%
\pgfpathlineto{\pgfqpoint{2.918280in}{0.636757in}}%
\pgfpathlineto{\pgfqpoint{2.922000in}{0.637473in}}%
\pgfpathlineto{\pgfqpoint{2.923240in}{0.635356in}}%
\pgfpathlineto{\pgfqpoint{2.924480in}{0.636909in}}%
\pgfpathlineto{\pgfqpoint{2.929440in}{0.632259in}}%
\pgfpathlineto{\pgfqpoint{2.931920in}{0.632283in}}%
\pgfpathlineto{\pgfqpoint{2.935640in}{0.631880in}}%
\pgfpathlineto{\pgfqpoint{2.939360in}{0.633123in}}%
\pgfpathlineto{\pgfqpoint{2.943080in}{0.630483in}}%
\pgfpathlineto{\pgfqpoint{2.949280in}{0.632025in}}%
\pgfpathlineto{\pgfqpoint{2.953000in}{0.635241in}}%
\pgfpathlineto{\pgfqpoint{2.954240in}{0.633830in}}%
\pgfpathlineto{\pgfqpoint{2.959200in}{0.636183in}}%
\pgfpathlineto{\pgfqpoint{2.960440in}{0.635084in}}%
\pgfpathlineto{\pgfqpoint{2.961680in}{0.635598in}}%
\pgfpathlineto{\pgfqpoint{2.962920in}{0.634612in}}%
\pgfpathlineto{\pgfqpoint{2.964160in}{0.635833in}}%
\pgfpathlineto{\pgfqpoint{2.969120in}{0.632730in}}%
\pgfpathlineto{\pgfqpoint{2.971600in}{0.635120in}}%
\pgfpathlineto{\pgfqpoint{2.972840in}{0.634198in}}%
\pgfpathlineto{\pgfqpoint{2.974080in}{0.634892in}}%
\pgfpathlineto{\pgfqpoint{2.977800in}{0.632809in}}%
\pgfpathlineto{\pgfqpoint{2.984000in}{0.634168in}}%
\pgfpathlineto{\pgfqpoint{2.985240in}{0.632772in}}%
\pgfpathlineto{\pgfqpoint{2.992680in}{0.636860in}}%
\pgfpathlineto{\pgfqpoint{2.993920in}{0.638532in}}%
\pgfpathlineto{\pgfqpoint{2.996400in}{0.637285in}}%
\pgfpathlineto{\pgfqpoint{2.998880in}{0.638994in}}%
\pgfpathlineto{\pgfqpoint{3.003840in}{0.638997in}}%
\pgfpathlineto{\pgfqpoint{3.006320in}{0.636268in}}%
\pgfpathlineto{\pgfqpoint{3.008800in}{0.637413in}}%
\pgfpathlineto{\pgfqpoint{3.013760in}{0.634732in}}%
\pgfpathlineto{\pgfqpoint{3.015000in}{0.634040in}}%
\pgfpathlineto{\pgfqpoint{3.024920in}{0.637760in}}%
\pgfpathlineto{\pgfqpoint{3.027400in}{0.636964in}}%
\pgfpathlineto{\pgfqpoint{3.028640in}{0.638450in}}%
\pgfpathlineto{\pgfqpoint{3.031120in}{0.636337in}}%
\pgfpathlineto{\pgfqpoint{3.036080in}{0.638697in}}%
\pgfpathlineto{\pgfqpoint{3.038560in}{0.639859in}}%
\pgfpathlineto{\pgfqpoint{3.043520in}{0.637501in}}%
\pgfpathlineto{\pgfqpoint{3.046000in}{0.638139in}}%
\pgfpathlineto{\pgfqpoint{3.047240in}{0.636275in}}%
\pgfpathlineto{\pgfqpoint{3.048480in}{0.637910in}}%
\pgfpathlineto{\pgfqpoint{3.053440in}{0.633125in}}%
\pgfpathlineto{\pgfqpoint{3.055920in}{0.633090in}}%
\pgfpathlineto{\pgfqpoint{3.059640in}{0.632547in}}%
\pgfpathlineto{\pgfqpoint{3.063360in}{0.633820in}}%
\pgfpathlineto{\pgfqpoint{3.067080in}{0.631176in}}%
\pgfpathlineto{\pgfqpoint{3.070800in}{0.631577in}}%
\pgfpathlineto{\pgfqpoint{3.072040in}{0.630782in}}%
\pgfpathlineto{\pgfqpoint{3.077000in}{0.635343in}}%
\pgfpathlineto{\pgfqpoint{3.078240in}{0.633904in}}%
\pgfpathlineto{\pgfqpoint{3.085680in}{0.634532in}}%
\pgfpathlineto{\pgfqpoint{3.086920in}{0.633672in}}%
\pgfpathlineto{\pgfqpoint{3.088160in}{0.634865in}}%
\pgfpathlineto{\pgfqpoint{3.089400in}{0.633828in}}%
\pgfpathlineto{\pgfqpoint{3.090640in}{0.634250in}}%
\pgfpathlineto{\pgfqpoint{3.093120in}{0.631585in}}%
\pgfpathlineto{\pgfqpoint{3.098080in}{0.634811in}}%
\pgfpathlineto{\pgfqpoint{3.101800in}{0.633077in}}%
\pgfpathlineto{\pgfqpoint{3.106760in}{0.634871in}}%
\pgfpathlineto{\pgfqpoint{3.110480in}{0.634210in}}%
\pgfpathlineto{\pgfqpoint{3.117920in}{0.639290in}}%
\pgfpathlineto{\pgfqpoint{3.120400in}{0.638314in}}%
\pgfpathlineto{\pgfqpoint{3.122880in}{0.639686in}}%
\pgfpathlineto{\pgfqpoint{3.127840in}{0.639845in}}%
\pgfpathlineto{\pgfqpoint{3.130320in}{0.637008in}}%
\pgfpathlineto{\pgfqpoint{3.132800in}{0.638351in}}%
\pgfpathlineto{\pgfqpoint{3.137760in}{0.634806in}}%
\pgfpathlineto{\pgfqpoint{3.139000in}{0.633992in}}%
\pgfpathlineto{\pgfqpoint{3.143960in}{0.636316in}}%
\pgfpathlineto{\pgfqpoint{3.152640in}{0.638883in}}%
\pgfpathlineto{\pgfqpoint{3.155120in}{0.636635in}}%
\pgfpathlineto{\pgfqpoint{3.157600in}{0.637986in}}%
\pgfpathlineto{\pgfqpoint{3.160080in}{0.639150in}}%
\pgfpathlineto{\pgfqpoint{3.162560in}{0.640031in}}%
\pgfpathlineto{\pgfqpoint{3.167520in}{0.637641in}}%
\pgfpathlineto{\pgfqpoint{3.170000in}{0.638385in}}%
\pgfpathlineto{\pgfqpoint{3.171240in}{0.636657in}}%
\pgfpathlineto{\pgfqpoint{3.172480in}{0.638429in}}%
\pgfpathlineto{\pgfqpoint{3.177440in}{0.634146in}}%
\pgfpathlineto{\pgfqpoint{3.179920in}{0.633646in}}%
\pgfpathlineto{\pgfqpoint{3.183640in}{0.632520in}}%
\pgfpathlineto{\pgfqpoint{3.187360in}{0.633469in}}%
\pgfpathlineto{\pgfqpoint{3.191080in}{0.631414in}}%
\pgfpathlineto{\pgfqpoint{3.193560in}{0.631706in}}%
\pgfpathlineto{\pgfqpoint{3.194800in}{0.631758in}}%
\pgfpathlineto{\pgfqpoint{3.196040in}{0.630627in}}%
\pgfpathlineto{\pgfqpoint{3.201000in}{0.636240in}}%
\pgfpathlineto{\pgfqpoint{3.202240in}{0.634917in}}%
\pgfpathlineto{\pgfqpoint{3.207200in}{0.636898in}}%
\pgfpathlineto{\pgfqpoint{3.209680in}{0.635639in}}%
\pgfpathlineto{\pgfqpoint{3.210920in}{0.634548in}}%
\pgfpathlineto{\pgfqpoint{3.212160in}{0.635685in}}%
\pgfpathlineto{\pgfqpoint{3.213400in}{0.634735in}}%
\pgfpathlineto{\pgfqpoint{3.214640in}{0.635394in}}%
\pgfpathlineto{\pgfqpoint{3.217120in}{0.632633in}}%
\pgfpathlineto{\pgfqpoint{3.222080in}{0.635509in}}%
\pgfpathlineto{\pgfqpoint{3.224560in}{0.634399in}}%
\pgfpathlineto{\pgfqpoint{3.227040in}{0.634469in}}%
\pgfpathlineto{\pgfqpoint{3.229520in}{0.636339in}}%
\pgfpathlineto{\pgfqpoint{3.233240in}{0.634537in}}%
\pgfpathlineto{\pgfqpoint{3.235720in}{0.636338in}}%
\pgfpathlineto{\pgfqpoint{3.238200in}{0.638273in}}%
\pgfpathlineto{\pgfqpoint{3.248120in}{0.640862in}}%
\pgfpathlineto{\pgfqpoint{3.251840in}{0.640233in}}%
\pgfpathlineto{\pgfqpoint{3.254320in}{0.637581in}}%
\pgfpathlineto{\pgfqpoint{3.256800in}{0.639071in}}%
\pgfpathlineto{\pgfqpoint{3.263000in}{0.634932in}}%
\pgfpathlineto{\pgfqpoint{3.270440in}{0.639011in}}%
\pgfpathlineto{\pgfqpoint{3.272920in}{0.639522in}}%
\pgfpathlineto{\pgfqpoint{3.277880in}{0.639174in}}%
\pgfpathlineto{\pgfqpoint{3.279120in}{0.637916in}}%
\pgfpathlineto{\pgfqpoint{3.282840in}{0.640010in}}%
\pgfpathlineto{\pgfqpoint{3.286560in}{0.641675in}}%
\pgfpathlineto{\pgfqpoint{3.291520in}{0.638737in}}%
\pgfpathlineto{\pgfqpoint{3.294000in}{0.639502in}}%
\pgfpathlineto{\pgfqpoint{3.295240in}{0.637608in}}%
\pgfpathlineto{\pgfqpoint{3.296480in}{0.639301in}}%
\pgfpathlineto{\pgfqpoint{3.301440in}{0.635682in}}%
\pgfpathlineto{\pgfqpoint{3.303920in}{0.635141in}}%
\pgfpathlineto{\pgfqpoint{3.307640in}{0.633984in}}%
\pgfpathlineto{\pgfqpoint{3.311360in}{0.635035in}}%
\pgfpathlineto{\pgfqpoint{3.313840in}{0.633112in}}%
\pgfpathlineto{\pgfqpoint{3.315080in}{0.632316in}}%
\pgfpathlineto{\pgfqpoint{3.317560in}{0.632841in}}%
\pgfpathlineto{\pgfqpoint{3.318800in}{0.632712in}}%
\pgfpathlineto{\pgfqpoint{3.320040in}{0.631275in}}%
\pgfpathlineto{\pgfqpoint{3.325000in}{0.636582in}}%
\pgfpathlineto{\pgfqpoint{3.326240in}{0.635296in}}%
\pgfpathlineto{\pgfqpoint{3.331200in}{0.637458in}}%
\pgfpathlineto{\pgfqpoint{3.334920in}{0.635287in}}%
\pgfpathlineto{\pgfqpoint{3.336160in}{0.636308in}}%
\pgfpathlineto{\pgfqpoint{3.341120in}{0.633279in}}%
\pgfpathlineto{\pgfqpoint{3.344840in}{0.635544in}}%
\pgfpathlineto{\pgfqpoint{3.346080in}{0.636336in}}%
\pgfpathlineto{\pgfqpoint{3.349800in}{0.634455in}}%
\pgfpathlineto{\pgfqpoint{3.351040in}{0.634841in}}%
\pgfpathlineto{\pgfqpoint{3.352280in}{0.636861in}}%
\pgfpathlineto{\pgfqpoint{3.357240in}{0.636274in}}%
\pgfpathlineto{\pgfqpoint{3.359720in}{0.638572in}}%
\pgfpathlineto{\pgfqpoint{3.364680in}{0.641980in}}%
\pgfpathlineto{\pgfqpoint{3.365920in}{0.642952in}}%
\pgfpathlineto{\pgfqpoint{3.368400in}{0.642748in}}%
\pgfpathlineto{\pgfqpoint{3.370880in}{0.643429in}}%
\pgfpathlineto{\pgfqpoint{3.375840in}{0.642422in}}%
\pgfpathlineto{\pgfqpoint{3.378320in}{0.639889in}}%
\pgfpathlineto{\pgfqpoint{3.380800in}{0.641537in}}%
\pgfpathlineto{\pgfqpoint{3.387000in}{0.638113in}}%
\pgfpathlineto{\pgfqpoint{3.391960in}{0.640397in}}%
\pgfpathlineto{\pgfqpoint{3.395680in}{0.641573in}}%
\pgfpathlineto{\pgfqpoint{3.398160in}{0.642354in}}%
\pgfpathlineto{\pgfqpoint{3.399400in}{0.642101in}}%
\pgfpathlineto{\pgfqpoint{3.400640in}{0.643381in}}%
\pgfpathlineto{\pgfqpoint{3.404360in}{0.640852in}}%
\pgfpathlineto{\pgfqpoint{3.408080in}{0.643355in}}%
\pgfpathlineto{\pgfqpoint{3.410560in}{0.644121in}}%
\pgfpathlineto{\pgfqpoint{3.415520in}{0.641196in}}%
\pgfpathlineto{\pgfqpoint{3.418000in}{0.641939in}}%
\pgfpathlineto{\pgfqpoint{3.419240in}{0.640105in}}%
\pgfpathlineto{\pgfqpoint{3.420480in}{0.642054in}}%
\pgfpathlineto{\pgfqpoint{3.425440in}{0.637497in}}%
\pgfpathlineto{\pgfqpoint{3.430400in}{0.636212in}}%
\pgfpathlineto{\pgfqpoint{3.437840in}{0.634858in}}%
\pgfpathlineto{\pgfqpoint{3.439080in}{0.633991in}}%
\pgfpathlineto{\pgfqpoint{3.441560in}{0.634661in}}%
\pgfpathlineto{\pgfqpoint{3.445280in}{0.635017in}}%
\pgfpathlineto{\pgfqpoint{3.447760in}{0.637134in}}%
\pgfpathlineto{\pgfqpoint{3.449000in}{0.638890in}}%
\pgfpathlineto{\pgfqpoint{3.450240in}{0.637961in}}%
\pgfpathlineto{\pgfqpoint{3.456440in}{0.639507in}}%
\pgfpathlineto{\pgfqpoint{3.463880in}{0.638090in}}%
\pgfpathlineto{\pgfqpoint{3.465120in}{0.636635in}}%
\pgfpathlineto{\pgfqpoint{3.468840in}{0.638745in}}%
\pgfpathlineto{\pgfqpoint{3.470080in}{0.639850in}}%
\pgfpathlineto{\pgfqpoint{3.473800in}{0.638104in}}%
\pgfpathlineto{\pgfqpoint{3.475040in}{0.638417in}}%
\pgfpathlineto{\pgfqpoint{3.476280in}{0.640204in}}%
\pgfpathlineto{\pgfqpoint{3.480000in}{0.638664in}}%
\pgfpathlineto{\pgfqpoint{3.486200in}{0.645124in}}%
\pgfpathlineto{\pgfqpoint{3.488680in}{0.646411in}}%
\pgfpathlineto{\pgfqpoint{3.491160in}{0.647107in}}%
\pgfpathlineto{\pgfqpoint{3.497360in}{0.648830in}}%
\pgfpathlineto{\pgfqpoint{3.511000in}{0.643085in}}%
\pgfpathlineto{\pgfqpoint{3.512240in}{0.643728in}}%
\pgfpathlineto{\pgfqpoint{3.514720in}{0.645867in}}%
\pgfpathlineto{\pgfqpoint{3.517200in}{0.646051in}}%
\pgfpathlineto{\pgfqpoint{3.520920in}{0.648399in}}%
\pgfpathlineto{\pgfqpoint{3.525880in}{0.647829in}}%
\pgfpathlineto{\pgfqpoint{3.528360in}{0.646474in}}%
\pgfpathlineto{\pgfqpoint{3.532080in}{0.648714in}}%
\pgfpathlineto{\pgfqpoint{3.534560in}{0.649899in}}%
\pgfpathlineto{\pgfqpoint{3.543240in}{0.646276in}}%
\pgfpathlineto{\pgfqpoint{3.544480in}{0.647941in}}%
\pgfpathlineto{\pgfqpoint{3.549440in}{0.643104in}}%
\pgfpathlineto{\pgfqpoint{3.561840in}{0.641028in}}%
\pgfpathlineto{\pgfqpoint{3.563080in}{0.640150in}}%
\pgfpathlineto{\pgfqpoint{3.565560in}{0.641493in}}%
\pgfpathlineto{\pgfqpoint{3.568040in}{0.639830in}}%
\pgfpathlineto{\pgfqpoint{3.573000in}{0.645178in}}%
\pgfpathlineto{\pgfqpoint{3.574240in}{0.644187in}}%
\pgfpathlineto{\pgfqpoint{3.580440in}{0.645648in}}%
\pgfpathlineto{\pgfqpoint{3.587880in}{0.644534in}}%
\pgfpathlineto{\pgfqpoint{3.589120in}{0.643231in}}%
\pgfpathlineto{\pgfqpoint{3.591600in}{0.645799in}}%
\pgfpathlineto{\pgfqpoint{3.592840in}{0.645107in}}%
\pgfpathlineto{\pgfqpoint{3.594080in}{0.646180in}}%
\pgfpathlineto{\pgfqpoint{3.597800in}{0.644548in}}%
\pgfpathlineto{\pgfqpoint{3.599040in}{0.644779in}}%
\pgfpathlineto{\pgfqpoint{3.601520in}{0.646747in}}%
\pgfpathlineto{\pgfqpoint{3.604000in}{0.645654in}}%
\pgfpathlineto{\pgfqpoint{3.607720in}{0.649607in}}%
\pgfpathlineto{\pgfqpoint{3.611440in}{0.651721in}}%
\pgfpathlineto{\pgfqpoint{3.620120in}{0.655177in}}%
\pgfpathlineto{\pgfqpoint{3.622600in}{0.654796in}}%
\pgfpathlineto{\pgfqpoint{3.625080in}{0.652436in}}%
\pgfpathlineto{\pgfqpoint{3.626320in}{0.651397in}}%
\pgfpathlineto{\pgfqpoint{3.628800in}{0.652614in}}%
\pgfpathlineto{\pgfqpoint{3.635000in}{0.649121in}}%
\pgfpathlineto{\pgfqpoint{3.644920in}{0.654693in}}%
\pgfpathlineto{\pgfqpoint{3.647400in}{0.654568in}}%
\pgfpathlineto{\pgfqpoint{3.648640in}{0.655737in}}%
\pgfpathlineto{\pgfqpoint{3.652360in}{0.652729in}}%
\pgfpathlineto{\pgfqpoint{3.656080in}{0.655248in}}%
\pgfpathlineto{\pgfqpoint{3.661040in}{0.656390in}}%
\pgfpathlineto{\pgfqpoint{3.663520in}{0.653700in}}%
\pgfpathlineto{\pgfqpoint{3.666000in}{0.654831in}}%
\pgfpathlineto{\pgfqpoint{3.667240in}{0.652930in}}%
\pgfpathlineto{\pgfqpoint{3.668480in}{0.654485in}}%
\pgfpathlineto{\pgfqpoint{3.673440in}{0.650063in}}%
\pgfpathlineto{\pgfqpoint{3.677160in}{0.649257in}}%
\pgfpathlineto{\pgfqpoint{3.682120in}{0.648900in}}%
\pgfpathlineto{\pgfqpoint{3.683360in}{0.649490in}}%
\pgfpathlineto{\pgfqpoint{3.685840in}{0.647317in}}%
\pgfpathlineto{\pgfqpoint{3.687080in}{0.646505in}}%
\pgfpathlineto{\pgfqpoint{3.689560in}{0.647958in}}%
\pgfpathlineto{\pgfqpoint{3.692040in}{0.646208in}}%
\pgfpathlineto{\pgfqpoint{3.695760in}{0.649915in}}%
\pgfpathlineto{\pgfqpoint{3.697000in}{0.651547in}}%
\pgfpathlineto{\pgfqpoint{3.698240in}{0.650588in}}%
\pgfpathlineto{\pgfqpoint{3.703200in}{0.652648in}}%
\pgfpathlineto{\pgfqpoint{3.706920in}{0.650532in}}%
\pgfpathlineto{\pgfqpoint{3.708160in}{0.651947in}}%
\pgfpathlineto{\pgfqpoint{3.709400in}{0.651481in}}%
\pgfpathlineto{\pgfqpoint{3.710640in}{0.652264in}}%
\pgfpathlineto{\pgfqpoint{3.713120in}{0.649774in}}%
\pgfpathlineto{\pgfqpoint{3.715600in}{0.652071in}}%
\pgfpathlineto{\pgfqpoint{3.716840in}{0.651569in}}%
\pgfpathlineto{\pgfqpoint{3.718080in}{0.652732in}}%
\pgfpathlineto{\pgfqpoint{3.719320in}{0.651530in}}%
\pgfpathlineto{\pgfqpoint{3.720560in}{0.652093in}}%
\pgfpathlineto{\pgfqpoint{3.723040in}{0.651232in}}%
\pgfpathlineto{\pgfqpoint{3.725520in}{0.653388in}}%
\pgfpathlineto{\pgfqpoint{3.728000in}{0.651938in}}%
\pgfpathlineto{\pgfqpoint{3.731720in}{0.655759in}}%
\pgfpathlineto{\pgfqpoint{3.737920in}{0.659820in}}%
\pgfpathlineto{\pgfqpoint{3.740400in}{0.660238in}}%
\pgfpathlineto{\pgfqpoint{3.742880in}{0.661566in}}%
\pgfpathlineto{\pgfqpoint{3.746600in}{0.661434in}}%
\pgfpathlineto{\pgfqpoint{3.751560in}{0.658929in}}%
\pgfpathlineto{\pgfqpoint{3.754040in}{0.658604in}}%
\pgfpathlineto{\pgfqpoint{3.760240in}{0.656396in}}%
\pgfpathlineto{\pgfqpoint{3.762720in}{0.658844in}}%
\pgfpathlineto{\pgfqpoint{3.765200in}{0.659149in}}%
\pgfpathlineto{\pgfqpoint{3.768920in}{0.660795in}}%
\pgfpathlineto{\pgfqpoint{3.771400in}{0.660605in}}%
\pgfpathlineto{\pgfqpoint{3.772640in}{0.661835in}}%
\pgfpathlineto{\pgfqpoint{3.776360in}{0.658138in}}%
\pgfpathlineto{\pgfqpoint{3.780080in}{0.661379in}}%
\pgfpathlineto{\pgfqpoint{3.782560in}{0.662844in}}%
\pgfpathlineto{\pgfqpoint{3.785040in}{0.663120in}}%
\pgfpathlineto{\pgfqpoint{3.787520in}{0.660165in}}%
\pgfpathlineto{\pgfqpoint{3.790000in}{0.661821in}}%
\pgfpathlineto{\pgfqpoint{3.791240in}{0.659844in}}%
\pgfpathlineto{\pgfqpoint{3.792480in}{0.661498in}}%
\pgfpathlineto{\pgfqpoint{3.796200in}{0.656696in}}%
\pgfpathlineto{\pgfqpoint{3.798680in}{0.656732in}}%
\pgfpathlineto{\pgfqpoint{3.802400in}{0.654760in}}%
\pgfpathlineto{\pgfqpoint{3.806120in}{0.655310in}}%
\pgfpathlineto{\pgfqpoint{3.807360in}{0.655669in}}%
\pgfpathlineto{\pgfqpoint{3.809840in}{0.653148in}}%
\pgfpathlineto{\pgfqpoint{3.811080in}{0.652227in}}%
\pgfpathlineto{\pgfqpoint{3.813560in}{0.654018in}}%
\pgfpathlineto{\pgfqpoint{3.816040in}{0.652417in}}%
\pgfpathlineto{\pgfqpoint{3.818520in}{0.655080in}}%
\pgfpathlineto{\pgfqpoint{3.824720in}{0.658298in}}%
\pgfpathlineto{\pgfqpoint{3.825960in}{0.657780in}}%
\pgfpathlineto{\pgfqpoint{3.828440in}{0.658615in}}%
\pgfpathlineto{\pgfqpoint{3.833400in}{0.658279in}}%
\pgfpathlineto{\pgfqpoint{3.834640in}{0.659224in}}%
\pgfpathlineto{\pgfqpoint{3.837120in}{0.656751in}}%
\pgfpathlineto{\pgfqpoint{3.839600in}{0.659141in}}%
\pgfpathlineto{\pgfqpoint{3.840840in}{0.658929in}}%
\pgfpathlineto{\pgfqpoint{3.842080in}{0.659980in}}%
\pgfpathlineto{\pgfqpoint{3.843320in}{0.658922in}}%
\pgfpathlineto{\pgfqpoint{3.844560in}{0.659814in}}%
\pgfpathlineto{\pgfqpoint{3.847040in}{0.659313in}}%
\pgfpathlineto{\pgfqpoint{3.849520in}{0.661120in}}%
\pgfpathlineto{\pgfqpoint{3.852000in}{0.659542in}}%
\pgfpathlineto{\pgfqpoint{3.859440in}{0.664385in}}%
\pgfpathlineto{\pgfqpoint{3.868120in}{0.667581in}}%
\pgfpathlineto{\pgfqpoint{3.870600in}{0.666876in}}%
\pgfpathlineto{\pgfqpoint{3.874320in}{0.663039in}}%
\pgfpathlineto{\pgfqpoint{3.876800in}{0.664598in}}%
\pgfpathlineto{\pgfqpoint{3.883000in}{0.660848in}}%
\pgfpathlineto{\pgfqpoint{3.890440in}{0.664955in}}%
\pgfpathlineto{\pgfqpoint{3.891680in}{0.664473in}}%
\pgfpathlineto{\pgfqpoint{3.894160in}{0.665297in}}%
\pgfpathlineto{\pgfqpoint{3.895400in}{0.665144in}}%
\pgfpathlineto{\pgfqpoint{3.896640in}{0.666434in}}%
\pgfpathlineto{\pgfqpoint{3.900360in}{0.662978in}}%
\pgfpathlineto{\pgfqpoint{3.902840in}{0.664939in}}%
\pgfpathlineto{\pgfqpoint{3.906560in}{0.667243in}}%
\pgfpathlineto{\pgfqpoint{3.909040in}{0.667888in}}%
\pgfpathlineto{\pgfqpoint{3.911520in}{0.664949in}}%
\pgfpathlineto{\pgfqpoint{3.912760in}{0.666457in}}%
\pgfpathlineto{\pgfqpoint{3.914000in}{0.666177in}}%
\pgfpathlineto{\pgfqpoint{3.915240in}{0.664256in}}%
\pgfpathlineto{\pgfqpoint{3.916480in}{0.665904in}}%
\pgfpathlineto{\pgfqpoint{3.920200in}{0.661472in}}%
\pgfpathlineto{\pgfqpoint{3.922680in}{0.661740in}}%
\pgfpathlineto{\pgfqpoint{3.926400in}{0.659832in}}%
\pgfpathlineto{\pgfqpoint{3.930120in}{0.659597in}}%
\pgfpathlineto{\pgfqpoint{3.931360in}{0.659769in}}%
\pgfpathlineto{\pgfqpoint{3.932600in}{0.657366in}}%
\pgfpathlineto{\pgfqpoint{3.938800in}{0.657501in}}%
\pgfpathlineto{\pgfqpoint{3.940040in}{0.656344in}}%
\pgfpathlineto{\pgfqpoint{3.942520in}{0.659131in}}%
\pgfpathlineto{\pgfqpoint{3.948720in}{0.661373in}}%
\pgfpathlineto{\pgfqpoint{3.954920in}{0.659801in}}%
\pgfpathlineto{\pgfqpoint{3.956160in}{0.661564in}}%
\pgfpathlineto{\pgfqpoint{3.961120in}{0.659270in}}%
\pgfpathlineto{\pgfqpoint{3.963600in}{0.661710in}}%
\pgfpathlineto{\pgfqpoint{3.964840in}{0.661475in}}%
\pgfpathlineto{\pgfqpoint{3.966080in}{0.662542in}}%
\pgfpathlineto{\pgfqpoint{3.967320in}{0.661432in}}%
\pgfpathlineto{\pgfqpoint{3.968560in}{0.662398in}}%
\pgfpathlineto{\pgfqpoint{3.971040in}{0.661306in}}%
\pgfpathlineto{\pgfqpoint{3.972280in}{0.662967in}}%
\pgfpathlineto{\pgfqpoint{3.977240in}{0.662860in}}%
\pgfpathlineto{\pgfqpoint{3.984680in}{0.667444in}}%
\pgfpathlineto{\pgfqpoint{3.990880in}{0.669865in}}%
\pgfpathlineto{\pgfqpoint{3.994600in}{0.669214in}}%
\pgfpathlineto{\pgfqpoint{3.998320in}{0.666116in}}%
\pgfpathlineto{\pgfqpoint{4.000800in}{0.667535in}}%
\pgfpathlineto{\pgfqpoint{4.008240in}{0.664044in}}%
\pgfpathlineto{\pgfqpoint{4.010720in}{0.666049in}}%
\pgfpathlineto{\pgfqpoint{4.013200in}{0.666531in}}%
\pgfpathlineto{\pgfqpoint{4.014440in}{0.667536in}}%
\pgfpathlineto{\pgfqpoint{4.016920in}{0.667692in}}%
\pgfpathlineto{\pgfqpoint{4.021880in}{0.667511in}}%
\pgfpathlineto{\pgfqpoint{4.023120in}{0.665743in}}%
\pgfpathlineto{\pgfqpoint{4.026840in}{0.667339in}}%
\pgfpathlineto{\pgfqpoint{4.030560in}{0.669716in}}%
\pgfpathlineto{\pgfqpoint{4.034280in}{0.668443in}}%
\pgfpathlineto{\pgfqpoint{4.035520in}{0.668059in}}%
\pgfpathlineto{\pgfqpoint{4.038000in}{0.669780in}}%
\pgfpathlineto{\pgfqpoint{4.039240in}{0.667906in}}%
\pgfpathlineto{\pgfqpoint{4.040480in}{0.669396in}}%
\pgfpathlineto{\pgfqpoint{4.044200in}{0.664140in}}%
\pgfpathlineto{\pgfqpoint{4.046680in}{0.664623in}}%
\pgfpathlineto{\pgfqpoint{4.050400in}{0.662254in}}%
\pgfpathlineto{\pgfqpoint{4.054120in}{0.662585in}}%
\pgfpathlineto{\pgfqpoint{4.055360in}{0.662896in}}%
\pgfpathlineto{\pgfqpoint{4.057840in}{0.660627in}}%
\pgfpathlineto{\pgfqpoint{4.059080in}{0.659383in}}%
\pgfpathlineto{\pgfqpoint{4.062800in}{0.661048in}}%
\pgfpathlineto{\pgfqpoint{4.064040in}{0.659902in}}%
\pgfpathlineto{\pgfqpoint{4.067760in}{0.663354in}}%
\pgfpathlineto{\pgfqpoint{4.069000in}{0.664895in}}%
\pgfpathlineto{\pgfqpoint{4.071480in}{0.664548in}}%
\pgfpathlineto{\pgfqpoint{4.072720in}{0.665564in}}%
\pgfpathlineto{\pgfqpoint{4.078920in}{0.662306in}}%
\pgfpathlineto{\pgfqpoint{4.080160in}{0.664242in}}%
\pgfpathlineto{\pgfqpoint{4.083880in}{0.663780in}}%
\pgfpathlineto{\pgfqpoint{4.085120in}{0.662249in}}%
\pgfpathlineto{\pgfqpoint{4.090080in}{0.665506in}}%
\pgfpathlineto{\pgfqpoint{4.091320in}{0.664127in}}%
\pgfpathlineto{\pgfqpoint{4.092560in}{0.665053in}}%
\pgfpathlineto{\pgfqpoint{4.095040in}{0.663724in}}%
\pgfpathlineto{\pgfqpoint{4.097520in}{0.664827in}}%
\pgfpathlineto{\pgfqpoint{4.101240in}{0.665100in}}%
\pgfpathlineto{\pgfqpoint{4.107440in}{0.669077in}}%
\pgfpathlineto{\pgfqpoint{4.117360in}{0.672636in}}%
\pgfpathlineto{\pgfqpoint{4.119840in}{0.671218in}}%
\pgfpathlineto{\pgfqpoint{4.122320in}{0.669369in}}%
\pgfpathlineto{\pgfqpoint{4.124800in}{0.670424in}}%
\pgfpathlineto{\pgfqpoint{4.132240in}{0.666452in}}%
\pgfpathlineto{\pgfqpoint{4.134720in}{0.668528in}}%
\pgfpathlineto{\pgfqpoint{4.137200in}{0.668620in}}%
\pgfpathlineto{\pgfqpoint{4.138440in}{0.669449in}}%
\pgfpathlineto{\pgfqpoint{4.142160in}{0.668298in}}%
\pgfpathlineto{\pgfqpoint{4.143400in}{0.667882in}}%
\pgfpathlineto{\pgfqpoint{4.144640in}{0.669007in}}%
\pgfpathlineto{\pgfqpoint{4.147120in}{0.665602in}}%
\pgfpathlineto{\pgfqpoint{4.149600in}{0.666541in}}%
\pgfpathlineto{\pgfqpoint{4.157040in}{0.669319in}}%
\pgfpathlineto{\pgfqpoint{4.159520in}{0.667436in}}%
\pgfpathlineto{\pgfqpoint{4.160760in}{0.669956in}}%
\pgfpathlineto{\pgfqpoint{4.162000in}{0.669867in}}%
\pgfpathlineto{\pgfqpoint{4.163240in}{0.668099in}}%
\pgfpathlineto{\pgfqpoint{4.164480in}{0.669568in}}%
\pgfpathlineto{\pgfqpoint{4.168200in}{0.664297in}}%
\pgfpathlineto{\pgfqpoint{4.170680in}{0.664591in}}%
\pgfpathlineto{\pgfqpoint{4.174400in}{0.662221in}}%
\pgfpathlineto{\pgfqpoint{4.179360in}{0.662515in}}%
\pgfpathlineto{\pgfqpoint{4.181840in}{0.660161in}}%
\pgfpathlineto{\pgfqpoint{4.183080in}{0.659212in}}%
\pgfpathlineto{\pgfqpoint{4.186800in}{0.660853in}}%
\pgfpathlineto{\pgfqpoint{4.188040in}{0.659893in}}%
\pgfpathlineto{\pgfqpoint{4.191760in}{0.663150in}}%
\pgfpathlineto{\pgfqpoint{4.193000in}{0.664401in}}%
\pgfpathlineto{\pgfqpoint{4.194240in}{0.663406in}}%
\pgfpathlineto{\pgfqpoint{4.200440in}{0.665001in}}%
\pgfpathlineto{\pgfqpoint{4.202920in}{0.662472in}}%
\pgfpathlineto{\pgfqpoint{4.204160in}{0.664667in}}%
\pgfpathlineto{\pgfqpoint{4.210360in}{0.664171in}}%
\pgfpathlineto{\pgfqpoint{4.214080in}{0.666756in}}%
\pgfpathlineto{\pgfqpoint{4.215320in}{0.665314in}}%
\pgfpathlineto{\pgfqpoint{4.216560in}{0.666488in}}%
\pgfpathlineto{\pgfqpoint{4.219040in}{0.664902in}}%
\pgfpathlineto{\pgfqpoint{4.220280in}{0.666019in}}%
\pgfpathlineto{\pgfqpoint{4.225240in}{0.664592in}}%
\pgfpathlineto{\pgfqpoint{4.233920in}{0.669247in}}%
\pgfpathlineto{\pgfqpoint{4.236400in}{0.668974in}}%
\pgfpathlineto{\pgfqpoint{4.240120in}{0.671279in}}%
\pgfpathlineto{\pgfqpoint{4.242600in}{0.670899in}}%
\pgfpathlineto{\pgfqpoint{4.246320in}{0.668622in}}%
\pgfpathlineto{\pgfqpoint{4.250040in}{0.668807in}}%
\pgfpathlineto{\pgfqpoint{4.256240in}{0.665416in}}%
\pgfpathlineto{\pgfqpoint{4.261200in}{0.668126in}}%
\pgfpathlineto{\pgfqpoint{4.266160in}{0.667332in}}%
\pgfpathlineto{\pgfqpoint{4.267400in}{0.666900in}}%
\pgfpathlineto{\pgfqpoint{4.268640in}{0.667858in}}%
\pgfpathlineto{\pgfqpoint{4.271120in}{0.664590in}}%
\pgfpathlineto{\pgfqpoint{4.281040in}{0.668468in}}%
\pgfpathlineto{\pgfqpoint{4.283520in}{0.667248in}}%
\pgfpathlineto{\pgfqpoint{4.286000in}{0.669799in}}%
\pgfpathlineto{\pgfqpoint{4.287240in}{0.668070in}}%
\pgfpathlineto{\pgfqpoint{4.288480in}{0.669350in}}%
\pgfpathlineto{\pgfqpoint{4.292200in}{0.663451in}}%
\pgfpathlineto{\pgfqpoint{4.294680in}{0.663562in}}%
\pgfpathlineto{\pgfqpoint{4.298400in}{0.661419in}}%
\pgfpathlineto{\pgfqpoint{4.303360in}{0.663042in}}%
\pgfpathlineto{\pgfqpoint{4.305840in}{0.660345in}}%
\pgfpathlineto{\pgfqpoint{4.307080in}{0.659175in}}%
\pgfpathlineto{\pgfqpoint{4.310800in}{0.660696in}}%
\pgfpathlineto{\pgfqpoint{4.312040in}{0.659651in}}%
\pgfpathlineto{\pgfqpoint{4.314520in}{0.661643in}}%
\pgfpathlineto{\pgfqpoint{4.323200in}{0.664892in}}%
\pgfpathlineto{\pgfqpoint{4.326920in}{0.661057in}}%
\pgfpathlineto{\pgfqpoint{4.329400in}{0.663193in}}%
\pgfpathlineto{\pgfqpoint{4.331880in}{0.663010in}}%
\pgfpathlineto{\pgfqpoint{4.333120in}{0.662238in}}%
\pgfpathlineto{\pgfqpoint{4.338080in}{0.666219in}}%
\pgfpathlineto{\pgfqpoint{4.339320in}{0.664843in}}%
\pgfpathlineto{\pgfqpoint{4.340560in}{0.666046in}}%
\pgfpathlineto{\pgfqpoint{4.343040in}{0.665154in}}%
\pgfpathlineto{\pgfqpoint{4.344280in}{0.666444in}}%
\pgfpathlineto{\pgfqpoint{4.349240in}{0.664398in}}%
\pgfpathlineto{\pgfqpoint{4.355440in}{0.667653in}}%
\pgfpathlineto{\pgfqpoint{4.356680in}{0.667160in}}%
\pgfpathlineto{\pgfqpoint{4.359160in}{0.667558in}}%
\pgfpathlineto{\pgfqpoint{4.360400in}{0.667904in}}%
\pgfpathlineto{\pgfqpoint{4.362880in}{0.670522in}}%
\pgfpathlineto{\pgfqpoint{4.366600in}{0.670553in}}%
\pgfpathlineto{\pgfqpoint{4.371560in}{0.668487in}}%
\pgfpathlineto{\pgfqpoint{4.374040in}{0.668612in}}%
\pgfpathlineto{\pgfqpoint{4.377760in}{0.667092in}}%
\pgfpathlineto{\pgfqpoint{4.380240in}{0.665444in}}%
\pgfpathlineto{\pgfqpoint{4.386440in}{0.668315in}}%
\pgfpathlineto{\pgfqpoint{4.388920in}{0.667604in}}%
\pgfpathlineto{\pgfqpoint{4.393880in}{0.665499in}}%
\pgfpathlineto{\pgfqpoint{4.395120in}{0.664255in}}%
\pgfpathlineto{\pgfqpoint{4.405040in}{0.666938in}}%
\pgfpathlineto{\pgfqpoint{4.406280in}{0.665536in}}%
\pgfpathlineto{\pgfqpoint{4.407520in}{0.666013in}}%
\pgfpathlineto{\pgfqpoint{4.410000in}{0.668714in}}%
\pgfpathlineto{\pgfqpoint{4.411240in}{0.667233in}}%
\pgfpathlineto{\pgfqpoint{4.412480in}{0.668116in}}%
\pgfpathlineto{\pgfqpoint{4.416200in}{0.662944in}}%
\pgfpathlineto{\pgfqpoint{4.418680in}{0.663105in}}%
\pgfpathlineto{\pgfqpoint{4.421160in}{0.660976in}}%
\pgfpathlineto{\pgfqpoint{4.423640in}{0.661502in}}%
\pgfpathlineto{\pgfqpoint{4.427360in}{0.662300in}}%
\pgfpathlineto{\pgfqpoint{4.429840in}{0.659319in}}%
\pgfpathlineto{\pgfqpoint{4.431080in}{0.658072in}}%
\pgfpathlineto{\pgfqpoint{4.434800in}{0.660471in}}%
\pgfpathlineto{\pgfqpoint{4.436040in}{0.659356in}}%
\pgfpathlineto{\pgfqpoint{4.438520in}{0.661339in}}%
\pgfpathlineto{\pgfqpoint{4.441000in}{0.662778in}}%
\pgfpathlineto{\pgfqpoint{4.442240in}{0.661346in}}%
\pgfpathlineto{\pgfqpoint{4.448440in}{0.662666in}}%
\pgfpathlineto{\pgfqpoint{4.450920in}{0.659956in}}%
\pgfpathlineto{\pgfqpoint{4.453400in}{0.661964in}}%
\pgfpathlineto{\pgfqpoint{4.455880in}{0.661893in}}%
\pgfpathlineto{\pgfqpoint{4.457120in}{0.661472in}}%
\pgfpathlineto{\pgfqpoint{4.462080in}{0.664669in}}%
\pgfpathlineto{\pgfqpoint{4.463320in}{0.663369in}}%
\pgfpathlineto{\pgfqpoint{4.465800in}{0.664013in}}%
\pgfpathlineto{\pgfqpoint{4.470760in}{0.664216in}}%
\pgfpathlineto{\pgfqpoint{4.472000in}{0.662807in}}%
\pgfpathlineto{\pgfqpoint{4.479440in}{0.665834in}}%
\pgfpathlineto{\pgfqpoint{4.480680in}{0.665685in}}%
\pgfpathlineto{\pgfqpoint{4.486880in}{0.669910in}}%
\pgfpathlineto{\pgfqpoint{4.490600in}{0.670675in}}%
\pgfpathlineto{\pgfqpoint{4.494320in}{0.668615in}}%
\pgfpathlineto{\pgfqpoint{4.499280in}{0.668729in}}%
\pgfpathlineto{\pgfqpoint{4.501760in}{0.667508in}}%
\pgfpathlineto{\pgfqpoint{4.504240in}{0.666740in}}%
\pgfpathlineto{\pgfqpoint{4.510440in}{0.669056in}}%
\pgfpathlineto{\pgfqpoint{4.512920in}{0.668868in}}%
\pgfpathlineto{\pgfqpoint{4.516640in}{0.668900in}}%
\pgfpathlineto{\pgfqpoint{4.519120in}{0.665777in}}%
\pgfpathlineto{\pgfqpoint{4.529040in}{0.668763in}}%
\pgfpathlineto{\pgfqpoint{4.530280in}{0.667132in}}%
\pgfpathlineto{\pgfqpoint{4.531520in}{0.667775in}}%
\pgfpathlineto{\pgfqpoint{4.532760in}{0.670346in}}%
\pgfpathlineto{\pgfqpoint{4.536480in}{0.669111in}}%
\pgfpathlineto{\pgfqpoint{4.538960in}{0.665261in}}%
\pgfpathlineto{\pgfqpoint{4.540200in}{0.664427in}}%
\pgfpathlineto{\pgfqpoint{4.542680in}{0.664806in}}%
\pgfpathlineto{\pgfqpoint{4.545160in}{0.662178in}}%
\pgfpathlineto{\pgfqpoint{4.547640in}{0.662731in}}%
\pgfpathlineto{\pgfqpoint{4.551360in}{0.663939in}}%
\pgfpathlineto{\pgfqpoint{4.553840in}{0.660536in}}%
\pgfpathlineto{\pgfqpoint{4.555080in}{0.659483in}}%
\pgfpathlineto{\pgfqpoint{4.558800in}{0.661476in}}%
\pgfpathlineto{\pgfqpoint{4.560040in}{0.660631in}}%
\pgfpathlineto{\pgfqpoint{4.565000in}{0.664199in}}%
\pgfpathlineto{\pgfqpoint{4.567480in}{0.663259in}}%
\pgfpathlineto{\pgfqpoint{4.569960in}{0.664382in}}%
\pgfpathlineto{\pgfqpoint{4.572440in}{0.664092in}}%
\pgfpathlineto{\pgfqpoint{4.574920in}{0.661219in}}%
\pgfpathlineto{\pgfqpoint{4.576160in}{0.663225in}}%
\pgfpathlineto{\pgfqpoint{4.581120in}{0.662522in}}%
\pgfpathlineto{\pgfqpoint{4.583600in}{0.664997in}}%
\pgfpathlineto{\pgfqpoint{4.584840in}{0.664530in}}%
\pgfpathlineto{\pgfqpoint{4.586080in}{0.665680in}}%
\pgfpathlineto{\pgfqpoint{4.587320in}{0.664979in}}%
\pgfpathlineto{\pgfqpoint{4.589800in}{0.665395in}}%
\pgfpathlineto{\pgfqpoint{4.591040in}{0.665580in}}%
\pgfpathlineto{\pgfqpoint{4.592280in}{0.667214in}}%
\pgfpathlineto{\pgfqpoint{4.594760in}{0.665801in}}%
\pgfpathlineto{\pgfqpoint{4.597240in}{0.663456in}}%
\pgfpathlineto{\pgfqpoint{4.608400in}{0.667661in}}%
\pgfpathlineto{\pgfqpoint{4.610880in}{0.669833in}}%
\pgfpathlineto{\pgfqpoint{4.614600in}{0.670500in}}%
\pgfpathlineto{\pgfqpoint{4.619560in}{0.668764in}}%
\pgfpathlineto{\pgfqpoint{4.622040in}{0.669174in}}%
\pgfpathlineto{\pgfqpoint{4.627000in}{0.666011in}}%
\pgfpathlineto{\pgfqpoint{4.629480in}{0.666241in}}%
\pgfpathlineto{\pgfqpoint{4.631960in}{0.668065in}}%
\pgfpathlineto{\pgfqpoint{4.634440in}{0.668601in}}%
\pgfpathlineto{\pgfqpoint{4.639400in}{0.666507in}}%
\pgfpathlineto{\pgfqpoint{4.640640in}{0.667543in}}%
\pgfpathlineto{\pgfqpoint{4.643120in}{0.664126in}}%
\pgfpathlineto{\pgfqpoint{4.653040in}{0.666831in}}%
\pgfpathlineto{\pgfqpoint{4.654280in}{0.664707in}}%
\pgfpathlineto{\pgfqpoint{4.655520in}{0.665276in}}%
\pgfpathlineto{\pgfqpoint{4.658000in}{0.667952in}}%
\pgfpathlineto{\pgfqpoint{4.659240in}{0.666734in}}%
\pgfpathlineto{\pgfqpoint{4.660480in}{0.667522in}}%
\pgfpathlineto{\pgfqpoint{4.662960in}{0.663969in}}%
\pgfpathlineto{\pgfqpoint{4.665440in}{0.663174in}}%
\pgfpathlineto{\pgfqpoint{4.666680in}{0.662784in}}%
\pgfpathlineto{\pgfqpoint{4.669160in}{0.660978in}}%
\pgfpathlineto{\pgfqpoint{4.675360in}{0.662556in}}%
\pgfpathlineto{\pgfqpoint{4.677840in}{0.659681in}}%
\pgfpathlineto{\pgfqpoint{4.679080in}{0.658448in}}%
\pgfpathlineto{\pgfqpoint{4.682800in}{0.661059in}}%
\pgfpathlineto{\pgfqpoint{4.684040in}{0.660256in}}%
\pgfpathlineto{\pgfqpoint{4.689000in}{0.663339in}}%
\pgfpathlineto{\pgfqpoint{4.690240in}{0.661906in}}%
\pgfpathlineto{\pgfqpoint{4.695200in}{0.663394in}}%
\pgfpathlineto{\pgfqpoint{4.696440in}{0.663024in}}%
\pgfpathlineto{\pgfqpoint{4.698920in}{0.660374in}}%
\pgfpathlineto{\pgfqpoint{4.700160in}{0.662133in}}%
\pgfpathlineto{\pgfqpoint{4.705120in}{0.660837in}}%
\pgfpathlineto{\pgfqpoint{4.710080in}{0.664674in}}%
\pgfpathlineto{\pgfqpoint{4.712560in}{0.664592in}}%
\pgfpathlineto{\pgfqpoint{4.715040in}{0.664618in}}%
\pgfpathlineto{\pgfqpoint{4.716280in}{0.666502in}}%
\pgfpathlineto{\pgfqpoint{4.718760in}{0.665115in}}%
\pgfpathlineto{\pgfqpoint{4.722480in}{0.662538in}}%
\pgfpathlineto{\pgfqpoint{4.727440in}{0.664454in}}%
\pgfpathlineto{\pgfqpoint{4.729920in}{0.664862in}}%
\pgfpathlineto{\pgfqpoint{4.732400in}{0.666396in}}%
\pgfpathlineto{\pgfqpoint{4.734880in}{0.668910in}}%
\pgfpathlineto{\pgfqpoint{4.738600in}{0.669394in}}%
\pgfpathlineto{\pgfqpoint{4.743560in}{0.666789in}}%
\pgfpathlineto{\pgfqpoint{4.746040in}{0.666593in}}%
\pgfpathlineto{\pgfqpoint{4.747280in}{0.665723in}}%
\pgfpathlineto{\pgfqpoint{4.749760in}{0.665761in}}%
\pgfpathlineto{\pgfqpoint{4.752240in}{0.664871in}}%
\pgfpathlineto{\pgfqpoint{4.753480in}{0.665085in}}%
\pgfpathlineto{\pgfqpoint{4.754720in}{0.666858in}}%
\pgfpathlineto{\pgfqpoint{4.765880in}{0.663778in}}%
\pgfpathlineto{\pgfqpoint{4.768360in}{0.661681in}}%
\pgfpathlineto{\pgfqpoint{4.770840in}{0.662202in}}%
\pgfpathlineto{\pgfqpoint{4.772080in}{0.662168in}}%
\pgfpathlineto{\pgfqpoint{4.774560in}{0.663986in}}%
\pgfpathlineto{\pgfqpoint{4.779520in}{0.663979in}}%
\pgfpathlineto{\pgfqpoint{4.780760in}{0.666400in}}%
\pgfpathlineto{\pgfqpoint{4.784480in}{0.665610in}}%
\pgfpathlineto{\pgfqpoint{4.788200in}{0.660269in}}%
\pgfpathlineto{\pgfqpoint{4.790680in}{0.659842in}}%
\pgfpathlineto{\pgfqpoint{4.793160in}{0.657911in}}%
\pgfpathlineto{\pgfqpoint{4.795640in}{0.658641in}}%
\pgfpathlineto{\pgfqpoint{4.799360in}{0.659294in}}%
\pgfpathlineto{\pgfqpoint{4.801840in}{0.657106in}}%
\pgfpathlineto{\pgfqpoint{4.803080in}{0.656374in}}%
\pgfpathlineto{\pgfqpoint{4.806800in}{0.658953in}}%
\pgfpathlineto{\pgfqpoint{4.808040in}{0.658001in}}%
\pgfpathlineto{\pgfqpoint{4.813000in}{0.661253in}}%
\pgfpathlineto{\pgfqpoint{4.814240in}{0.660295in}}%
\pgfpathlineto{\pgfqpoint{4.817960in}{0.662250in}}%
\pgfpathlineto{\pgfqpoint{4.822920in}{0.658111in}}%
\pgfpathlineto{\pgfqpoint{4.824160in}{0.659615in}}%
\pgfpathlineto{\pgfqpoint{4.829120in}{0.657931in}}%
\pgfpathlineto{\pgfqpoint{4.832840in}{0.661084in}}%
\pgfpathlineto{\pgfqpoint{4.834080in}{0.661983in}}%
\pgfpathlineto{\pgfqpoint{4.835320in}{0.661356in}}%
\pgfpathlineto{\pgfqpoint{4.837800in}{0.661801in}}%
\pgfpathlineto{\pgfqpoint{4.839040in}{0.661831in}}%
\pgfpathlineto{\pgfqpoint{4.841520in}{0.663762in}}%
\pgfpathlineto{\pgfqpoint{4.845240in}{0.660680in}}%
\pgfpathlineto{\pgfqpoint{4.846480in}{0.660473in}}%
\pgfpathlineto{\pgfqpoint{4.850200in}{0.662404in}}%
\pgfpathlineto{\pgfqpoint{4.853920in}{0.662863in}}%
\pgfpathlineto{\pgfqpoint{4.856400in}{0.664359in}}%
\pgfpathlineto{\pgfqpoint{4.858880in}{0.667459in}}%
\pgfpathlineto{\pgfqpoint{4.862600in}{0.667476in}}%
\pgfpathlineto{\pgfqpoint{4.867560in}{0.665051in}}%
\pgfpathlineto{\pgfqpoint{4.870040in}{0.665537in}}%
\pgfpathlineto{\pgfqpoint{4.871280in}{0.664682in}}%
\pgfpathlineto{\pgfqpoint{4.873760in}{0.664836in}}%
\pgfpathlineto{\pgfqpoint{4.876240in}{0.663473in}}%
\pgfpathlineto{\pgfqpoint{4.877480in}{0.663625in}}%
\pgfpathlineto{\pgfqpoint{4.879960in}{0.664816in}}%
\pgfpathlineto{\pgfqpoint{4.882440in}{0.665122in}}%
\pgfpathlineto{\pgfqpoint{4.887400in}{0.662907in}}%
\pgfpathlineto{\pgfqpoint{4.888640in}{0.664153in}}%
\pgfpathlineto{\pgfqpoint{4.892360in}{0.660450in}}%
\pgfpathlineto{\pgfqpoint{4.894840in}{0.660741in}}%
\pgfpathlineto{\pgfqpoint{4.896080in}{0.660398in}}%
\pgfpathlineto{\pgfqpoint{4.898560in}{0.662556in}}%
\pgfpathlineto{\pgfqpoint{4.903520in}{0.661793in}}%
\pgfpathlineto{\pgfqpoint{4.906000in}{0.664268in}}%
\pgfpathlineto{\pgfqpoint{4.907240in}{0.663254in}}%
\pgfpathlineto{\pgfqpoint{4.908480in}{0.664283in}}%
\pgfpathlineto{\pgfqpoint{4.912200in}{0.659201in}}%
\pgfpathlineto{\pgfqpoint{4.914680in}{0.658549in}}%
\pgfpathlineto{\pgfqpoint{4.917160in}{0.655905in}}%
\pgfpathlineto{\pgfqpoint{4.922120in}{0.655783in}}%
\pgfpathlineto{\pgfqpoint{4.923360in}{0.656660in}}%
\pgfpathlineto{\pgfqpoint{4.927080in}{0.653724in}}%
\pgfpathlineto{\pgfqpoint{4.930800in}{0.656570in}}%
\pgfpathlineto{\pgfqpoint{4.932040in}{0.655846in}}%
\pgfpathlineto{\pgfqpoint{4.937000in}{0.658247in}}%
\pgfpathlineto{\pgfqpoint{4.939480in}{0.657312in}}%
\pgfpathlineto{\pgfqpoint{4.940720in}{0.658833in}}%
\pgfpathlineto{\pgfqpoint{4.945680in}{0.656112in}}%
\pgfpathlineto{\pgfqpoint{4.946920in}{0.654647in}}%
\pgfpathlineto{\pgfqpoint{4.948160in}{0.656382in}}%
\pgfpathlineto{\pgfqpoint{4.953120in}{0.654914in}}%
\pgfpathlineto{\pgfqpoint{4.955600in}{0.657883in}}%
\pgfpathlineto{\pgfqpoint{4.956840in}{0.657338in}}%
\pgfpathlineto{\pgfqpoint{4.958080in}{0.658207in}}%
\pgfpathlineto{\pgfqpoint{4.959320in}{0.657477in}}%
\pgfpathlineto{\pgfqpoint{4.965520in}{0.660197in}}%
\pgfpathlineto{\pgfqpoint{4.970480in}{0.655291in}}%
\pgfpathlineto{\pgfqpoint{4.975440in}{0.656741in}}%
\pgfpathlineto{\pgfqpoint{4.977920in}{0.656243in}}%
\pgfpathlineto{\pgfqpoint{4.979160in}{0.656338in}}%
\pgfpathlineto{\pgfqpoint{4.984120in}{0.661786in}}%
\pgfpathlineto{\pgfqpoint{5.001480in}{0.658412in}}%
\pgfpathlineto{\pgfqpoint{5.003960in}{0.659213in}}%
\pgfpathlineto{\pgfqpoint{5.006440in}{0.659089in}}%
\pgfpathlineto{\pgfqpoint{5.007680in}{0.657739in}}%
\pgfpathlineto{\pgfqpoint{5.010160in}{0.658591in}}%
\pgfpathlineto{\pgfqpoint{5.011400in}{0.657553in}}%
\pgfpathlineto{\pgfqpoint{5.012640in}{0.658948in}}%
\pgfpathlineto{\pgfqpoint{5.016360in}{0.654564in}}%
\pgfpathlineto{\pgfqpoint{5.018840in}{0.655368in}}%
\pgfpathlineto{\pgfqpoint{5.020080in}{0.654982in}}%
\pgfpathlineto{\pgfqpoint{5.022560in}{0.656556in}}%
\pgfpathlineto{\pgfqpoint{5.027520in}{0.655613in}}%
\pgfpathlineto{\pgfqpoint{5.030000in}{0.658795in}}%
\pgfpathlineto{\pgfqpoint{5.031240in}{0.658186in}}%
\pgfpathlineto{\pgfqpoint{5.032480in}{0.659134in}}%
\pgfpathlineto{\pgfqpoint{5.036200in}{0.653793in}}%
\pgfpathlineto{\pgfqpoint{5.038680in}{0.654125in}}%
\pgfpathlineto{\pgfqpoint{5.041160in}{0.651918in}}%
\pgfpathlineto{\pgfqpoint{5.048600in}{0.651189in}}%
\pgfpathlineto{\pgfqpoint{5.051080in}{0.648976in}}%
\pgfpathlineto{\pgfqpoint{5.059760in}{0.653110in}}%
\pgfpathlineto{\pgfqpoint{5.061000in}{0.652995in}}%
\pgfpathlineto{\pgfqpoint{5.063480in}{0.651408in}}%
\pgfpathlineto{\pgfqpoint{5.064720in}{0.652686in}}%
\pgfpathlineto{\pgfqpoint{5.069680in}{0.651104in}}%
\pgfpathlineto{\pgfqpoint{5.070920in}{0.649487in}}%
\pgfpathlineto{\pgfqpoint{5.073400in}{0.651019in}}%
\pgfpathlineto{\pgfqpoint{5.075880in}{0.650839in}}%
\pgfpathlineto{\pgfqpoint{5.077120in}{0.649252in}}%
\pgfpathlineto{\pgfqpoint{5.079600in}{0.651910in}}%
\pgfpathlineto{\pgfqpoint{5.080840in}{0.651765in}}%
\pgfpathlineto{\pgfqpoint{5.082080in}{0.653103in}}%
\pgfpathlineto{\pgfqpoint{5.084560in}{0.652151in}}%
\pgfpathlineto{\pgfqpoint{5.087040in}{0.652663in}}%
\pgfpathlineto{\pgfqpoint{5.089520in}{0.655244in}}%
\pgfpathlineto{\pgfqpoint{5.092000in}{0.652900in}}%
\pgfpathlineto{\pgfqpoint{5.098200in}{0.655479in}}%
\pgfpathlineto{\pgfqpoint{5.101920in}{0.654935in}}%
\pgfpathlineto{\pgfqpoint{5.103160in}{0.654708in}}%
\pgfpathlineto{\pgfqpoint{5.106880in}{0.660284in}}%
\pgfpathlineto{\pgfqpoint{5.109360in}{0.660251in}}%
\pgfpathlineto{\pgfqpoint{5.114320in}{0.658389in}}%
\pgfpathlineto{\pgfqpoint{5.116800in}{0.657313in}}%
\pgfpathlineto{\pgfqpoint{5.120520in}{0.657394in}}%
\pgfpathlineto{\pgfqpoint{5.125480in}{0.655737in}}%
\pgfpathlineto{\pgfqpoint{5.127960in}{0.656535in}}%
\pgfpathlineto{\pgfqpoint{5.130440in}{0.657328in}}%
\pgfpathlineto{\pgfqpoint{5.131680in}{0.656212in}}%
\pgfpathlineto{\pgfqpoint{5.134160in}{0.656713in}}%
\pgfpathlineto{\pgfqpoint{5.135400in}{0.655452in}}%
\pgfpathlineto{\pgfqpoint{5.136640in}{0.656843in}}%
\pgfpathlineto{\pgfqpoint{5.140360in}{0.652666in}}%
\pgfpathlineto{\pgfqpoint{5.146560in}{0.655064in}}%
\pgfpathlineto{\pgfqpoint{5.151520in}{0.653618in}}%
\pgfpathlineto{\pgfqpoint{5.154000in}{0.656461in}}%
\pgfpathlineto{\pgfqpoint{5.155240in}{0.656073in}}%
\pgfpathlineto{\pgfqpoint{5.156480in}{0.656894in}}%
\pgfpathlineto{\pgfqpoint{5.158960in}{0.653016in}}%
\pgfpathlineto{\pgfqpoint{5.160200in}{0.652153in}}%
\pgfpathlineto{\pgfqpoint{5.162680in}{0.652860in}}%
\pgfpathlineto{\pgfqpoint{5.163920in}{0.650979in}}%
\pgfpathlineto{\pgfqpoint{5.168880in}{0.651311in}}%
\pgfpathlineto{\pgfqpoint{5.170120in}{0.650402in}}%
\pgfpathlineto{\pgfqpoint{5.171360in}{0.650929in}}%
\pgfpathlineto{\pgfqpoint{5.177560in}{0.648140in}}%
\pgfpathlineto{\pgfqpoint{5.178800in}{0.649676in}}%
\pgfpathlineto{\pgfqpoint{5.180040in}{0.649289in}}%
\pgfpathlineto{\pgfqpoint{5.183760in}{0.652019in}}%
\pgfpathlineto{\pgfqpoint{5.185000in}{0.652289in}}%
\pgfpathlineto{\pgfqpoint{5.187480in}{0.650738in}}%
\pgfpathlineto{\pgfqpoint{5.189960in}{0.651563in}}%
\pgfpathlineto{\pgfqpoint{5.193680in}{0.651575in}}%
\pgfpathlineto{\pgfqpoint{5.194920in}{0.649815in}}%
\pgfpathlineto{\pgfqpoint{5.196160in}{0.651433in}}%
\pgfpathlineto{\pgfqpoint{5.198640in}{0.651047in}}%
\pgfpathlineto{\pgfqpoint{5.199880in}{0.651577in}}%
\pgfpathlineto{\pgfqpoint{5.201120in}{0.650364in}}%
\pgfpathlineto{\pgfqpoint{5.206080in}{0.653856in}}%
\pgfpathlineto{\pgfqpoint{5.208560in}{0.652635in}}%
\pgfpathlineto{\pgfqpoint{5.211040in}{0.652515in}}%
\pgfpathlineto{\pgfqpoint{5.213520in}{0.654449in}}%
\pgfpathlineto{\pgfqpoint{5.217240in}{0.652299in}}%
\pgfpathlineto{\pgfqpoint{5.223440in}{0.653880in}}%
\pgfpathlineto{\pgfqpoint{5.227160in}{0.653947in}}%
\pgfpathlineto{\pgfqpoint{5.230880in}{0.658898in}}%
\pgfpathlineto{\pgfqpoint{5.233360in}{0.658505in}}%
\pgfpathlineto{\pgfqpoint{5.248240in}{0.653047in}}%
\pgfpathlineto{\pgfqpoint{5.250720in}{0.654719in}}%
\pgfpathlineto{\pgfqpoint{5.255680in}{0.652808in}}%
\pgfpathlineto{\pgfqpoint{5.258160in}{0.653187in}}%
\pgfpathlineto{\pgfqpoint{5.259400in}{0.652392in}}%
\pgfpathlineto{\pgfqpoint{5.260640in}{0.653901in}}%
\pgfpathlineto{\pgfqpoint{5.264360in}{0.650994in}}%
\pgfpathlineto{\pgfqpoint{5.270560in}{0.653674in}}%
\pgfpathlineto{\pgfqpoint{5.275520in}{0.651637in}}%
\pgfpathlineto{\pgfqpoint{5.278000in}{0.654621in}}%
\pgfpathlineto{\pgfqpoint{5.279240in}{0.654121in}}%
\pgfpathlineto{\pgfqpoint{5.280480in}{0.654835in}}%
\pgfpathlineto{\pgfqpoint{5.282960in}{0.650456in}}%
\pgfpathlineto{\pgfqpoint{5.284200in}{0.649624in}}%
\pgfpathlineto{\pgfqpoint{5.286680in}{0.650570in}}%
\pgfpathlineto{\pgfqpoint{5.287920in}{0.648531in}}%
\pgfpathlineto{\pgfqpoint{5.295360in}{0.649588in}}%
\pgfpathlineto{\pgfqpoint{5.301560in}{0.646289in}}%
\pgfpathlineto{\pgfqpoint{5.306520in}{0.650215in}}%
\pgfpathlineto{\pgfqpoint{5.309000in}{0.650774in}}%
\pgfpathlineto{\pgfqpoint{5.311480in}{0.649601in}}%
\pgfpathlineto{\pgfqpoint{5.312720in}{0.650210in}}%
\pgfpathlineto{\pgfqpoint{5.313960in}{0.649399in}}%
\pgfpathlineto{\pgfqpoint{5.317680in}{0.649972in}}%
\pgfpathlineto{\pgfqpoint{5.318920in}{0.647608in}}%
\pgfpathlineto{\pgfqpoint{5.323880in}{0.649090in}}%
\pgfpathlineto{\pgfqpoint{5.325120in}{0.648112in}}%
\pgfpathlineto{\pgfqpoint{5.330080in}{0.652534in}}%
\pgfpathlineto{\pgfqpoint{5.333800in}{0.651067in}}%
\pgfpathlineto{\pgfqpoint{5.337520in}{0.652733in}}%
\pgfpathlineto{\pgfqpoint{5.340000in}{0.651055in}}%
\pgfpathlineto{\pgfqpoint{5.347440in}{0.653690in}}%
\pgfpathlineto{\pgfqpoint{5.351160in}{0.654095in}}%
\pgfpathlineto{\pgfqpoint{5.354880in}{0.659986in}}%
\pgfpathlineto{\pgfqpoint{5.357360in}{0.659057in}}%
\pgfpathlineto{\pgfqpoint{5.373480in}{0.652387in}}%
\pgfpathlineto{\pgfqpoint{5.374720in}{0.653098in}}%
\pgfpathlineto{\pgfqpoint{5.379680in}{0.651598in}}%
\pgfpathlineto{\pgfqpoint{5.385880in}{0.653347in}}%
\pgfpathlineto{\pgfqpoint{5.388360in}{0.651813in}}%
\pgfpathlineto{\pgfqpoint{5.392080in}{0.652134in}}%
\pgfpathlineto{\pgfqpoint{5.394560in}{0.654969in}}%
\pgfpathlineto{\pgfqpoint{5.399520in}{0.653450in}}%
\pgfpathlineto{\pgfqpoint{5.402000in}{0.656119in}}%
\pgfpathlineto{\pgfqpoint{5.404480in}{0.657105in}}%
\pgfpathlineto{\pgfqpoint{5.406960in}{0.653384in}}%
\pgfpathlineto{\pgfqpoint{5.409440in}{0.653120in}}%
\pgfpathlineto{\pgfqpoint{5.410680in}{0.653776in}}%
\pgfpathlineto{\pgfqpoint{5.413160in}{0.651833in}}%
\pgfpathlineto{\pgfqpoint{5.418120in}{0.651166in}}%
\pgfpathlineto{\pgfqpoint{5.419360in}{0.652539in}}%
\pgfpathlineto{\pgfqpoint{5.421840in}{0.650389in}}%
\pgfpathlineto{\pgfqpoint{5.424320in}{0.648738in}}%
\pgfpathlineto{\pgfqpoint{5.425560in}{0.648213in}}%
\pgfpathlineto{\pgfqpoint{5.429280in}{0.651867in}}%
\pgfpathlineto{\pgfqpoint{5.433000in}{0.653766in}}%
\pgfpathlineto{\pgfqpoint{5.434240in}{0.652884in}}%
\pgfpathlineto{\pgfqpoint{5.436720in}{0.653917in}}%
\pgfpathlineto{\pgfqpoint{5.437960in}{0.652216in}}%
\pgfpathlineto{\pgfqpoint{5.441680in}{0.653349in}}%
\pgfpathlineto{\pgfqpoint{5.442920in}{0.651166in}}%
\pgfpathlineto{\pgfqpoint{5.447880in}{0.652623in}}%
\pgfpathlineto{\pgfqpoint{5.449120in}{0.652082in}}%
\pgfpathlineto{\pgfqpoint{5.454080in}{0.656492in}}%
\pgfpathlineto{\pgfqpoint{5.456560in}{0.655819in}}%
\pgfpathlineto{\pgfqpoint{5.461520in}{0.656851in}}%
\pgfpathlineto{\pgfqpoint{5.464000in}{0.655137in}}%
\pgfpathlineto{\pgfqpoint{5.473920in}{0.660016in}}%
\pgfpathlineto{\pgfqpoint{5.475160in}{0.659548in}}%
\pgfpathlineto{\pgfqpoint{5.480120in}{0.665365in}}%
\pgfpathlineto{\pgfqpoint{5.483840in}{0.662000in}}%
\pgfpathlineto{\pgfqpoint{5.486320in}{0.660076in}}%
\pgfpathlineto{\pgfqpoint{5.487560in}{0.658976in}}%
\pgfpathlineto{\pgfqpoint{5.490040in}{0.659364in}}%
\pgfpathlineto{\pgfqpoint{5.491280in}{0.657927in}}%
\pgfpathlineto{\pgfqpoint{5.493760in}{0.657843in}}%
\pgfpathlineto{\pgfqpoint{5.507400in}{0.655300in}}%
\pgfpathlineto{\pgfqpoint{5.508640in}{0.656693in}}%
\pgfpathlineto{\pgfqpoint{5.511120in}{0.653472in}}%
\pgfpathlineto{\pgfqpoint{5.514840in}{0.654535in}}%
\pgfpathlineto{\pgfqpoint{5.516080in}{0.653637in}}%
\pgfpathlineto{\pgfqpoint{5.518560in}{0.656304in}}%
\pgfpathlineto{\pgfqpoint{5.523520in}{0.654382in}}%
\pgfpathlineto{\pgfqpoint{5.526000in}{0.657668in}}%
\pgfpathlineto{\pgfqpoint{5.528480in}{0.658922in}}%
\pgfpathlineto{\pgfqpoint{5.530960in}{0.654805in}}%
\pgfpathlineto{\pgfqpoint{5.532200in}{0.654128in}}%
\pgfpathlineto{\pgfqpoint{5.534680in}{0.655964in}}%
\pgfpathlineto{\pgfqpoint{5.537160in}{0.653263in}}%
\pgfpathlineto{\pgfqpoint{5.542120in}{0.653665in}}%
\pgfpathlineto{\pgfqpoint{5.543360in}{0.655145in}}%
\pgfpathlineto{\pgfqpoint{5.549560in}{0.650755in}}%
\pgfpathlineto{\pgfqpoint{5.554520in}{0.656046in}}%
\pgfpathlineto{\pgfqpoint{5.557000in}{0.657341in}}%
\pgfpathlineto{\pgfqpoint{5.558240in}{0.656555in}}%
\pgfpathlineto{\pgfqpoint{5.560720in}{0.657257in}}%
\pgfpathlineto{\pgfqpoint{5.563200in}{0.655261in}}%
\pgfpathlineto{\pgfqpoint{5.565680in}{0.655825in}}%
\pgfpathlineto{\pgfqpoint{5.566920in}{0.654474in}}%
\pgfpathlineto{\pgfqpoint{5.571880in}{0.655646in}}%
\pgfpathlineto{\pgfqpoint{5.573120in}{0.654905in}}%
\pgfpathlineto{\pgfqpoint{5.576840in}{0.657335in}}%
\pgfpathlineto{\pgfqpoint{5.578080in}{0.657976in}}%
\pgfpathlineto{\pgfqpoint{5.579320in}{0.657118in}}%
\pgfpathlineto{\pgfqpoint{5.584280in}{0.660460in}}%
\pgfpathlineto{\pgfqpoint{5.589240in}{0.657284in}}%
\pgfpathlineto{\pgfqpoint{5.600400in}{0.663336in}}%
\pgfpathlineto{\pgfqpoint{5.602880in}{0.667729in}}%
\pgfpathlineto{\pgfqpoint{5.604120in}{0.668256in}}%
\pgfpathlineto{\pgfqpoint{5.606600in}{0.665424in}}%
\pgfpathlineto{\pgfqpoint{5.612800in}{0.661730in}}%
\pgfpathlineto{\pgfqpoint{5.619000in}{0.659488in}}%
\pgfpathlineto{\pgfqpoint{5.621480in}{0.658699in}}%
\pgfpathlineto{\pgfqpoint{5.622720in}{0.658730in}}%
\pgfpathlineto{\pgfqpoint{5.625200in}{0.657088in}}%
\pgfpathlineto{\pgfqpoint{5.626440in}{0.657809in}}%
\pgfpathlineto{\pgfqpoint{5.628920in}{0.657364in}}%
\pgfpathlineto{\pgfqpoint{5.630160in}{0.658331in}}%
\pgfpathlineto{\pgfqpoint{5.631400in}{0.657441in}}%
\pgfpathlineto{\pgfqpoint{5.632640in}{0.659110in}}%
\pgfpathlineto{\pgfqpoint{5.635120in}{0.655826in}}%
\pgfpathlineto{\pgfqpoint{5.638840in}{0.657635in}}%
\pgfpathlineto{\pgfqpoint{5.640080in}{0.656871in}}%
\pgfpathlineto{\pgfqpoint{5.642560in}{0.659530in}}%
\pgfpathlineto{\pgfqpoint{5.647520in}{0.656897in}}%
\pgfpathlineto{\pgfqpoint{5.650000in}{0.660293in}}%
\pgfpathlineto{\pgfqpoint{5.652480in}{0.662197in}}%
\pgfpathlineto{\pgfqpoint{5.654960in}{0.658277in}}%
\pgfpathlineto{\pgfqpoint{5.656200in}{0.657358in}}%
\pgfpathlineto{\pgfqpoint{5.658680in}{0.659847in}}%
\pgfpathlineto{\pgfqpoint{5.661160in}{0.657367in}}%
\pgfpathlineto{\pgfqpoint{5.667360in}{0.657370in}}%
\pgfpathlineto{\pgfqpoint{5.673560in}{0.650537in}}%
\pgfpathlineto{\pgfqpoint{5.674800in}{0.651541in}}%
\pgfpathlineto{\pgfqpoint{5.676040in}{0.651115in}}%
\pgfpathlineto{\pgfqpoint{5.678520in}{0.654613in}}%
\pgfpathlineto{\pgfqpoint{5.681000in}{0.656350in}}%
\pgfpathlineto{\pgfqpoint{5.682240in}{0.655533in}}%
\pgfpathlineto{\pgfqpoint{5.684720in}{0.656719in}}%
\pgfpathlineto{\pgfqpoint{5.687200in}{0.654796in}}%
\pgfpathlineto{\pgfqpoint{5.689680in}{0.655970in}}%
\pgfpathlineto{\pgfqpoint{5.690920in}{0.654677in}}%
\pgfpathlineto{\pgfqpoint{5.694640in}{0.655485in}}%
\pgfpathlineto{\pgfqpoint{5.698360in}{0.654898in}}%
\pgfpathlineto{\pgfqpoint{5.699600in}{0.656812in}}%
\pgfpathlineto{\pgfqpoint{5.703320in}{0.656532in}}%
\pgfpathlineto{\pgfqpoint{5.709520in}{0.659238in}}%
\pgfpathlineto{\pgfqpoint{5.712000in}{0.657116in}}%
\pgfpathlineto{\pgfqpoint{5.716960in}{0.658658in}}%
\pgfpathlineto{\pgfqpoint{5.718200in}{0.658639in}}%
\pgfpathlineto{\pgfqpoint{5.721920in}{0.661944in}}%
\pgfpathlineto{\pgfqpoint{5.723160in}{0.660379in}}%
\pgfpathlineto{\pgfqpoint{5.724400in}{0.662243in}}%
\pgfpathlineto{\pgfqpoint{5.726880in}{0.668220in}}%
\pgfpathlineto{\pgfqpoint{5.728120in}{0.668672in}}%
\pgfpathlineto{\pgfqpoint{5.730600in}{0.666643in}}%
\pgfpathlineto{\pgfqpoint{5.735560in}{0.663270in}}%
\pgfpathlineto{\pgfqpoint{5.738040in}{0.662718in}}%
\pgfpathlineto{\pgfqpoint{5.740520in}{0.662207in}}%
\pgfpathlineto{\pgfqpoint{5.741760in}{0.662643in}}%
\pgfpathlineto{\pgfqpoint{5.745480in}{0.660599in}}%
\pgfpathlineto{\pgfqpoint{5.746720in}{0.660973in}}%
\pgfpathlineto{\pgfqpoint{5.749200in}{0.660142in}}%
\pgfpathlineto{\pgfqpoint{5.756640in}{0.662036in}}%
\pgfpathlineto{\pgfqpoint{5.759120in}{0.658995in}}%
\pgfpathlineto{\pgfqpoint{5.762840in}{0.661538in}}%
\pgfpathlineto{\pgfqpoint{5.764080in}{0.660793in}}%
\pgfpathlineto{\pgfqpoint{5.766560in}{0.663016in}}%
\pgfpathlineto{\pgfqpoint{5.769040in}{0.661610in}}%
\pgfpathlineto{\pgfqpoint{5.770280in}{0.659151in}}%
\pgfpathlineto{\pgfqpoint{5.771520in}{0.659774in}}%
\pgfpathlineto{\pgfqpoint{5.774000in}{0.663166in}}%
\pgfpathlineto{\pgfqpoint{5.776480in}{0.665879in}}%
\pgfpathlineto{\pgfqpoint{5.780200in}{0.659672in}}%
\pgfpathlineto{\pgfqpoint{5.782680in}{0.661232in}}%
\pgfpathlineto{\pgfqpoint{5.785160in}{0.658139in}}%
\pgfpathlineto{\pgfqpoint{5.788880in}{0.658093in}}%
\pgfpathlineto{\pgfqpoint{5.793840in}{0.654287in}}%
\pgfpathlineto{\pgfqpoint{5.795080in}{0.651923in}}%
\pgfpathlineto{\pgfqpoint{5.798800in}{0.653972in}}%
\pgfpathlineto{\pgfqpoint{5.800040in}{0.653337in}}%
\pgfpathlineto{\pgfqpoint{5.803760in}{0.658539in}}%
\pgfpathlineto{\pgfqpoint{5.805000in}{0.659187in}}%
\pgfpathlineto{\pgfqpoint{5.807480in}{0.658629in}}%
\pgfpathlineto{\pgfqpoint{5.808720in}{0.659167in}}%
\pgfpathlineto{\pgfqpoint{5.811200in}{0.657359in}}%
\pgfpathlineto{\pgfqpoint{5.813680in}{0.658528in}}%
\pgfpathlineto{\pgfqpoint{5.814920in}{0.657455in}}%
\pgfpathlineto{\pgfqpoint{5.816160in}{0.658412in}}%
\pgfpathlineto{\pgfqpoint{5.817400in}{0.657676in}}%
\pgfpathlineto{\pgfqpoint{5.819880in}{0.658352in}}%
\pgfpathlineto{\pgfqpoint{5.822360in}{0.657327in}}%
\pgfpathlineto{\pgfqpoint{5.824840in}{0.658683in}}%
\pgfpathlineto{\pgfqpoint{5.831040in}{0.660025in}}%
\pgfpathlineto{\pgfqpoint{5.832280in}{0.662476in}}%
\pgfpathlineto{\pgfqpoint{5.833520in}{0.661936in}}%
\pgfpathlineto{\pgfqpoint{5.836000in}{0.659082in}}%
\pgfpathlineto{\pgfqpoint{5.840960in}{0.662320in}}%
\pgfpathlineto{\pgfqpoint{5.842200in}{0.662550in}}%
\pgfpathlineto{\pgfqpoint{5.845920in}{0.665845in}}%
\pgfpathlineto{\pgfqpoint{5.847160in}{0.664632in}}%
\pgfpathlineto{\pgfqpoint{5.848400in}{0.665819in}}%
\pgfpathlineto{\pgfqpoint{5.850880in}{0.672489in}}%
\pgfpathlineto{\pgfqpoint{5.852120in}{0.673376in}}%
\pgfpathlineto{\pgfqpoint{5.855840in}{0.669797in}}%
\pgfpathlineto{\pgfqpoint{5.857080in}{0.670062in}}%
\pgfpathlineto{\pgfqpoint{5.859560in}{0.667573in}}%
\pgfpathlineto{\pgfqpoint{5.860800in}{0.667931in}}%
\pgfpathlineto{\pgfqpoint{5.863280in}{0.665804in}}%
\pgfpathlineto{\pgfqpoint{5.865760in}{0.667908in}}%
\pgfpathlineto{\pgfqpoint{5.868240in}{0.665108in}}%
\pgfpathlineto{\pgfqpoint{5.870720in}{0.666904in}}%
\pgfpathlineto{\pgfqpoint{5.873200in}{0.666180in}}%
\pgfpathlineto{\pgfqpoint{5.878160in}{0.667351in}}%
\pgfpathlineto{\pgfqpoint{5.879400in}{0.666416in}}%
\pgfpathlineto{\pgfqpoint{5.880640in}{0.667348in}}%
\pgfpathlineto{\pgfqpoint{5.883120in}{0.664943in}}%
\pgfpathlineto{\pgfqpoint{5.884360in}{0.665947in}}%
\pgfpathlineto{\pgfqpoint{5.885600in}{0.665138in}}%
\pgfpathlineto{\pgfqpoint{5.886840in}{0.665692in}}%
\pgfpathlineto{\pgfqpoint{5.888080in}{0.664179in}}%
\pgfpathlineto{\pgfqpoint{5.890560in}{0.665952in}}%
\pgfpathlineto{\pgfqpoint{5.893040in}{0.665001in}}%
\pgfpathlineto{\pgfqpoint{5.894280in}{0.662488in}}%
\pgfpathlineto{\pgfqpoint{5.895520in}{0.663170in}}%
\pgfpathlineto{\pgfqpoint{5.898000in}{0.665711in}}%
\pgfpathlineto{\pgfqpoint{5.900480in}{0.667986in}}%
\pgfpathlineto{\pgfqpoint{5.904200in}{0.661945in}}%
\pgfpathlineto{\pgfqpoint{5.906680in}{0.663402in}}%
\pgfpathlineto{\pgfqpoint{5.909160in}{0.660095in}}%
\pgfpathlineto{\pgfqpoint{5.911640in}{0.659917in}}%
\pgfpathlineto{\pgfqpoint{5.912880in}{0.659768in}}%
\pgfpathlineto{\pgfqpoint{5.914120in}{0.658413in}}%
\pgfpathlineto{\pgfqpoint{5.915360in}{0.659400in}}%
\pgfpathlineto{\pgfqpoint{5.917840in}{0.656258in}}%
\pgfpathlineto{\pgfqpoint{5.919080in}{0.653870in}}%
\pgfpathlineto{\pgfqpoint{5.922800in}{0.656798in}}%
\pgfpathlineto{\pgfqpoint{5.924040in}{0.656371in}}%
\pgfpathlineto{\pgfqpoint{5.929000in}{0.662395in}}%
\pgfpathlineto{\pgfqpoint{5.931480in}{0.661395in}}%
\pgfpathlineto{\pgfqpoint{5.936440in}{0.660400in}}%
\pgfpathlineto{\pgfqpoint{5.938920in}{0.661158in}}%
\pgfpathlineto{\pgfqpoint{5.942640in}{0.662415in}}%
\pgfpathlineto{\pgfqpoint{5.946360in}{0.661392in}}%
\pgfpathlineto{\pgfqpoint{5.948840in}{0.662532in}}%
\pgfpathlineto{\pgfqpoint{5.955040in}{0.662209in}}%
\pgfpathlineto{\pgfqpoint{5.956280in}{0.664707in}}%
\pgfpathlineto{\pgfqpoint{5.957520in}{0.664548in}}%
\pgfpathlineto{\pgfqpoint{5.960000in}{0.662094in}}%
\pgfpathlineto{\pgfqpoint{5.964960in}{0.666988in}}%
\pgfpathlineto{\pgfqpoint{5.967440in}{0.668497in}}%
\pgfpathlineto{\pgfqpoint{5.969920in}{0.670364in}}%
\pgfpathlineto{\pgfqpoint{5.971160in}{0.669050in}}%
\pgfpathlineto{\pgfqpoint{5.972400in}{0.669689in}}%
\pgfpathlineto{\pgfqpoint{5.974880in}{0.676623in}}%
\pgfpathlineto{\pgfqpoint{5.976120in}{0.678100in}}%
\pgfpathlineto{\pgfqpoint{5.978600in}{0.674664in}}%
\pgfpathlineto{\pgfqpoint{5.981080in}{0.672700in}}%
\pgfpathlineto{\pgfqpoint{5.983560in}{0.670731in}}%
\pgfpathlineto{\pgfqpoint{5.984800in}{0.671549in}}%
\pgfpathlineto{\pgfqpoint{5.987280in}{0.668921in}}%
\pgfpathlineto{\pgfqpoint{5.989760in}{0.672069in}}%
\pgfpathlineto{\pgfqpoint{5.992240in}{0.670271in}}%
\pgfpathlineto{\pgfqpoint{5.997200in}{0.672961in}}%
\pgfpathlineto{\pgfqpoint{6.000920in}{0.672762in}}%
\pgfpathlineto{\pgfqpoint{6.002160in}{0.674021in}}%
\pgfpathlineto{\pgfqpoint{6.003400in}{0.672895in}}%
\pgfpathlineto{\pgfqpoint{6.004640in}{0.673976in}}%
\pgfpathlineto{\pgfqpoint{6.007120in}{0.671379in}}%
\pgfpathlineto{\pgfqpoint{6.008360in}{0.672935in}}%
\pgfpathlineto{\pgfqpoint{6.012080in}{0.672577in}}%
\pgfpathlineto{\pgfqpoint{6.014560in}{0.674215in}}%
\pgfpathlineto{\pgfqpoint{6.018280in}{0.669442in}}%
\pgfpathlineto{\pgfqpoint{6.024480in}{0.675144in}}%
\pgfpathlineto{\pgfqpoint{6.028200in}{0.668152in}}%
\pgfpathlineto{\pgfqpoint{6.030680in}{0.669309in}}%
\pgfpathlineto{\pgfqpoint{6.033160in}{0.666796in}}%
\pgfpathlineto{\pgfqpoint{6.034400in}{0.667556in}}%
\pgfpathlineto{\pgfqpoint{6.038120in}{0.665976in}}%
\pgfpathlineto{\pgfqpoint{6.039360in}{0.667192in}}%
\pgfpathlineto{\pgfqpoint{6.043080in}{0.661708in}}%
\pgfpathlineto{\pgfqpoint{6.045560in}{0.663206in}}%
\pgfpathlineto{\pgfqpoint{6.046800in}{0.664189in}}%
\pgfpathlineto{\pgfqpoint{6.048040in}{0.663131in}}%
\pgfpathlineto{\pgfqpoint{6.055480in}{0.667857in}}%
\pgfpathlineto{\pgfqpoint{6.056720in}{0.668351in}}%
\pgfpathlineto{\pgfqpoint{6.059200in}{0.667328in}}%
\pgfpathlineto{\pgfqpoint{6.064160in}{0.670379in}}%
\pgfpathlineto{\pgfqpoint{6.069120in}{0.667834in}}%
\pgfpathlineto{\pgfqpoint{6.075320in}{0.669338in}}%
\pgfpathlineto{\pgfqpoint{6.079040in}{0.668626in}}%
\pgfpathlineto{\pgfqpoint{6.080280in}{0.671669in}}%
\pgfpathlineto{\pgfqpoint{6.081520in}{0.671297in}}%
\pgfpathlineto{\pgfqpoint{6.084000in}{0.668861in}}%
\pgfpathlineto{\pgfqpoint{6.086480in}{0.669803in}}%
\pgfpathlineto{\pgfqpoint{6.088960in}{0.672207in}}%
\pgfpathlineto{\pgfqpoint{6.093920in}{0.674334in}}%
\pgfpathlineto{\pgfqpoint{6.095160in}{0.673988in}}%
\pgfpathlineto{\pgfqpoint{6.096400in}{0.675077in}}%
\pgfpathlineto{\pgfqpoint{6.098880in}{0.680573in}}%
\pgfpathlineto{\pgfqpoint{6.100120in}{0.681834in}}%
\pgfpathlineto{\pgfqpoint{6.102600in}{0.677643in}}%
\pgfpathlineto{\pgfqpoint{6.106320in}{0.674665in}}%
\pgfpathlineto{\pgfqpoint{6.107560in}{0.674017in}}%
\pgfpathlineto{\pgfqpoint{6.108800in}{0.674749in}}%
\pgfpathlineto{\pgfqpoint{6.112520in}{0.673695in}}%
\pgfpathlineto{\pgfqpoint{6.113760in}{0.674909in}}%
\pgfpathlineto{\pgfqpoint{6.116240in}{0.672889in}}%
\pgfpathlineto{\pgfqpoint{6.119960in}{0.673631in}}%
\pgfpathlineto{\pgfqpoint{6.122440in}{0.675385in}}%
\pgfpathlineto{\pgfqpoint{6.124920in}{0.674733in}}%
\pgfpathlineto{\pgfqpoint{6.126160in}{0.676376in}}%
\pgfpathlineto{\pgfqpoint{6.127400in}{0.674988in}}%
\pgfpathlineto{\pgfqpoint{6.128640in}{0.676531in}}%
\pgfpathlineto{\pgfqpoint{6.131120in}{0.673552in}}%
\pgfpathlineto{\pgfqpoint{6.132360in}{0.675804in}}%
\pgfpathlineto{\pgfqpoint{6.136080in}{0.674606in}}%
\pgfpathlineto{\pgfqpoint{6.138560in}{0.676807in}}%
\pgfpathlineto{\pgfqpoint{6.143520in}{0.672699in}}%
\pgfpathlineto{\pgfqpoint{6.147240in}{0.677315in}}%
\pgfpathlineto{\pgfqpoint{6.148480in}{0.678522in}}%
\pgfpathlineto{\pgfqpoint{6.152200in}{0.670962in}}%
\pgfpathlineto{\pgfqpoint{6.154680in}{0.673320in}}%
\pgfpathlineto{\pgfqpoint{6.157160in}{0.670646in}}%
\pgfpathlineto{\pgfqpoint{6.158400in}{0.671207in}}%
\pgfpathlineto{\pgfqpoint{6.162120in}{0.669396in}}%
\pgfpathlineto{\pgfqpoint{6.163360in}{0.670109in}}%
\pgfpathlineto{\pgfqpoint{6.165840in}{0.667349in}}%
\pgfpathlineto{\pgfqpoint{6.167080in}{0.664150in}}%
\pgfpathlineto{\pgfqpoint{6.174520in}{0.667100in}}%
\pgfpathlineto{\pgfqpoint{6.177000in}{0.670893in}}%
\pgfpathlineto{\pgfqpoint{6.178240in}{0.669947in}}%
\pgfpathlineto{\pgfqpoint{6.180720in}{0.670791in}}%
\pgfpathlineto{\pgfqpoint{6.181960in}{0.670310in}}%
\pgfpathlineto{\pgfqpoint{6.185680in}{0.672543in}}%
\pgfpathlineto{\pgfqpoint{6.186920in}{0.671977in}}%
\pgfpathlineto{\pgfqpoint{6.188160in}{0.673407in}}%
\pgfpathlineto{\pgfqpoint{6.189400in}{0.672131in}}%
\pgfpathlineto{\pgfqpoint{6.190640in}{0.672829in}}%
\pgfpathlineto{\pgfqpoint{6.193120in}{0.669342in}}%
\pgfpathlineto{\pgfqpoint{6.200560in}{0.670653in}}%
\pgfpathlineto{\pgfqpoint{6.203040in}{0.669099in}}%
\pgfpathlineto{\pgfqpoint{6.204280in}{0.671667in}}%
\pgfpathlineto{\pgfqpoint{6.206760in}{0.669987in}}%
\pgfpathlineto{\pgfqpoint{6.208000in}{0.668077in}}%
\pgfpathlineto{\pgfqpoint{6.211720in}{0.673216in}}%
\pgfpathlineto{\pgfqpoint{6.212960in}{0.672744in}}%
\pgfpathlineto{\pgfqpoint{6.215440in}{0.673767in}}%
\pgfpathlineto{\pgfqpoint{6.220400in}{0.674935in}}%
\pgfpathlineto{\pgfqpoint{6.222880in}{0.681108in}}%
\pgfpathlineto{\pgfqpoint{6.224120in}{0.682873in}}%
\pgfpathlineto{\pgfqpoint{6.226600in}{0.677578in}}%
\pgfpathlineto{\pgfqpoint{6.227840in}{0.676394in}}%
\pgfpathlineto{\pgfqpoint{6.230320in}{0.676824in}}%
\pgfpathlineto{\pgfqpoint{6.234040in}{0.676506in}}%
\pgfpathlineto{\pgfqpoint{6.235280in}{0.673326in}}%
\pgfpathlineto{\pgfqpoint{6.236520in}{0.673321in}}%
\pgfpathlineto{\pgfqpoint{6.237760in}{0.675032in}}%
\pgfpathlineto{\pgfqpoint{6.241480in}{0.672330in}}%
\pgfpathlineto{\pgfqpoint{6.246440in}{0.675058in}}%
\pgfpathlineto{\pgfqpoint{6.248920in}{0.673454in}}%
\pgfpathlineto{\pgfqpoint{6.250160in}{0.675905in}}%
\pgfpathlineto{\pgfqpoint{6.251400in}{0.675066in}}%
\pgfpathlineto{\pgfqpoint{6.252640in}{0.677548in}}%
\pgfpathlineto{\pgfqpoint{6.255120in}{0.673679in}}%
\pgfpathlineto{\pgfqpoint{6.256360in}{0.676692in}}%
\pgfpathlineto{\pgfqpoint{6.260080in}{0.674928in}}%
\pgfpathlineto{\pgfqpoint{6.262560in}{0.677302in}}%
\pgfpathlineto{\pgfqpoint{6.267520in}{0.674243in}}%
\pgfpathlineto{\pgfqpoint{6.272480in}{0.679713in}}%
\pgfpathlineto{\pgfqpoint{6.276200in}{0.671667in}}%
\pgfpathlineto{\pgfqpoint{6.278680in}{0.672492in}}%
\pgfpathlineto{\pgfqpoint{6.281160in}{0.670727in}}%
\pgfpathlineto{\pgfqpoint{6.282400in}{0.672265in}}%
\pgfpathlineto{\pgfqpoint{6.284880in}{0.671530in}}%
\pgfpathlineto{\pgfqpoint{6.286120in}{0.669360in}}%
\pgfpathlineto{\pgfqpoint{6.287360in}{0.670005in}}%
\pgfpathlineto{\pgfqpoint{6.292320in}{0.664906in}}%
\pgfpathlineto{\pgfqpoint{6.294800in}{0.666556in}}%
\pgfpathlineto{\pgfqpoint{6.297280in}{0.665638in}}%
\pgfpathlineto{\pgfqpoint{6.298520in}{0.666261in}}%
\pgfpathlineto{\pgfqpoint{6.301000in}{0.670834in}}%
\pgfpathlineto{\pgfqpoint{6.302240in}{0.670591in}}%
\pgfpathlineto{\pgfqpoint{6.304720in}{0.672221in}}%
\pgfpathlineto{\pgfqpoint{6.307200in}{0.671983in}}%
\pgfpathlineto{\pgfqpoint{6.308440in}{0.672266in}}%
\pgfpathlineto{\pgfqpoint{6.309680in}{0.674131in}}%
\pgfpathlineto{\pgfqpoint{6.310920in}{0.673440in}}%
\pgfpathlineto{\pgfqpoint{6.314640in}{0.675602in}}%
\pgfpathlineto{\pgfqpoint{6.318360in}{0.670688in}}%
\pgfpathlineto{\pgfqpoint{6.322080in}{0.672699in}}%
\pgfpathlineto{\pgfqpoint{6.323320in}{0.671097in}}%
\pgfpathlineto{\pgfqpoint{6.325800in}{0.672034in}}%
\pgfpathlineto{\pgfqpoint{6.327040in}{0.671817in}}%
\pgfpathlineto{\pgfqpoint{6.328280in}{0.674783in}}%
\pgfpathlineto{\pgfqpoint{6.332000in}{0.671551in}}%
\pgfpathlineto{\pgfqpoint{6.335720in}{0.677181in}}%
\pgfpathlineto{\pgfqpoint{6.338200in}{0.678330in}}%
\pgfpathlineto{\pgfqpoint{6.341920in}{0.680915in}}%
\pgfpathlineto{\pgfqpoint{6.344400in}{0.680608in}}%
\pgfpathlineto{\pgfqpoint{6.346880in}{0.685099in}}%
\pgfpathlineto{\pgfqpoint{6.348120in}{0.687488in}}%
\pgfpathlineto{\pgfqpoint{6.350600in}{0.681364in}}%
\pgfpathlineto{\pgfqpoint{6.353080in}{0.679445in}}%
\pgfpathlineto{\pgfqpoint{6.358040in}{0.678857in}}%
\pgfpathlineto{\pgfqpoint{6.359280in}{0.675680in}}%
\pgfpathlineto{\pgfqpoint{6.360520in}{0.676605in}}%
\pgfpathlineto{\pgfqpoint{6.361760in}{0.679151in}}%
\pgfpathlineto{\pgfqpoint{6.367960in}{0.676136in}}%
\pgfpathlineto{\pgfqpoint{6.370440in}{0.677374in}}%
\pgfpathlineto{\pgfqpoint{6.372920in}{0.674105in}}%
\pgfpathlineto{\pgfqpoint{6.374160in}{0.675449in}}%
\pgfpathlineto{\pgfqpoint{6.375400in}{0.674534in}}%
\pgfpathlineto{\pgfqpoint{6.376640in}{0.677245in}}%
\pgfpathlineto{\pgfqpoint{6.379120in}{0.673898in}}%
\pgfpathlineto{\pgfqpoint{6.380360in}{0.676821in}}%
\pgfpathlineto{\pgfqpoint{6.382840in}{0.675270in}}%
\pgfpathlineto{\pgfqpoint{6.385320in}{0.676942in}}%
\pgfpathlineto{\pgfqpoint{6.389040in}{0.677492in}}%
\pgfpathlineto{\pgfqpoint{6.390280in}{0.674950in}}%
\pgfpathlineto{\pgfqpoint{6.394000in}{0.678421in}}%
\pgfpathlineto{\pgfqpoint{6.396480in}{0.681900in}}%
\pgfpathlineto{\pgfqpoint{6.400200in}{0.671793in}}%
\pgfpathlineto{\pgfqpoint{6.405160in}{0.670271in}}%
\pgfpathlineto{\pgfqpoint{6.406400in}{0.672992in}}%
\pgfpathlineto{\pgfqpoint{6.407640in}{0.672899in}}%
\pgfpathlineto{\pgfqpoint{6.416320in}{0.665368in}}%
\pgfpathlineto{\pgfqpoint{6.421280in}{0.668986in}}%
\pgfpathlineto{\pgfqpoint{6.422520in}{0.669802in}}%
\pgfpathlineto{\pgfqpoint{6.425000in}{0.674217in}}%
\pgfpathlineto{\pgfqpoint{6.427480in}{0.674493in}}%
\pgfpathlineto{\pgfqpoint{6.433680in}{0.676177in}}%
\pgfpathlineto{\pgfqpoint{6.437400in}{0.673974in}}%
\pgfpathlineto{\pgfqpoint{6.438640in}{0.674309in}}%
\pgfpathlineto{\pgfqpoint{6.441120in}{0.669151in}}%
\pgfpathlineto{\pgfqpoint{6.442360in}{0.669374in}}%
\pgfpathlineto{\pgfqpoint{6.444840in}{0.671550in}}%
\pgfpathlineto{\pgfqpoint{6.446080in}{0.672006in}}%
\pgfpathlineto{\pgfqpoint{6.447320in}{0.670365in}}%
\pgfpathlineto{\pgfqpoint{6.449800in}{0.671227in}}%
\pgfpathlineto{\pgfqpoint{6.451040in}{0.671151in}}%
\pgfpathlineto{\pgfqpoint{6.452280in}{0.674300in}}%
\pgfpathlineto{\pgfqpoint{6.456000in}{0.670006in}}%
\pgfpathlineto{\pgfqpoint{6.457240in}{0.666462in}}%
\pgfpathlineto{\pgfqpoint{6.460960in}{0.668828in}}%
\pgfpathlineto{\pgfqpoint{6.464680in}{0.673256in}}%
\pgfpathlineto{\pgfqpoint{6.468400in}{0.671945in}}%
\pgfpathlineto{\pgfqpoint{6.472120in}{0.680844in}}%
\pgfpathlineto{\pgfqpoint{6.474600in}{0.676769in}}%
\pgfpathlineto{\pgfqpoint{6.475840in}{0.675993in}}%
\pgfpathlineto{\pgfqpoint{6.480800in}{0.678729in}}%
\pgfpathlineto{\pgfqpoint{6.482040in}{0.677981in}}%
\pgfpathlineto{\pgfqpoint{6.483280in}{0.674554in}}%
\pgfpathlineto{\pgfqpoint{6.485760in}{0.678490in}}%
\pgfpathlineto{\pgfqpoint{6.491960in}{0.673273in}}%
\pgfpathlineto{\pgfqpoint{6.494440in}{0.675025in}}%
\pgfpathlineto{\pgfqpoint{6.495680in}{0.672006in}}%
\pgfpathlineto{\pgfqpoint{6.499400in}{0.671795in}}%
\pgfpathlineto{\pgfqpoint{6.501880in}{0.674899in}}%
\pgfpathlineto{\pgfqpoint{6.503120in}{0.671584in}}%
\pgfpathlineto{\pgfqpoint{6.504360in}{0.673414in}}%
\pgfpathlineto{\pgfqpoint{6.508080in}{0.670634in}}%
\pgfpathlineto{\pgfqpoint{6.511800in}{0.675674in}}%
\pgfpathlineto{\pgfqpoint{6.513040in}{0.675430in}}%
\pgfpathlineto{\pgfqpoint{6.514280in}{0.672701in}}%
\pgfpathlineto{\pgfqpoint{6.518000in}{0.674455in}}%
\pgfpathlineto{\pgfqpoint{6.520480in}{0.677708in}}%
\pgfpathlineto{\pgfqpoint{6.524200in}{0.666037in}}%
\pgfpathlineto{\pgfqpoint{6.529160in}{0.661567in}}%
\pgfpathlineto{\pgfqpoint{6.531640in}{0.664491in}}%
\pgfpathlineto{\pgfqpoint{6.537840in}{0.659068in}}%
\pgfpathlineto{\pgfqpoint{6.539080in}{0.657487in}}%
\pgfpathlineto{\pgfqpoint{6.541560in}{0.659445in}}%
\pgfpathlineto{\pgfqpoint{6.542800in}{0.661847in}}%
\pgfpathlineto{\pgfqpoint{6.544040in}{0.661501in}}%
\pgfpathlineto{\pgfqpoint{6.549000in}{0.669373in}}%
\pgfpathlineto{\pgfqpoint{6.550240in}{0.668785in}}%
\pgfpathlineto{\pgfqpoint{6.552720in}{0.671023in}}%
\pgfpathlineto{\pgfqpoint{6.556440in}{0.670625in}}%
\pgfpathlineto{\pgfqpoint{6.558920in}{0.671870in}}%
\pgfpathlineto{\pgfqpoint{6.562640in}{0.671188in}}%
\pgfpathlineto{\pgfqpoint{6.566360in}{0.666697in}}%
\pgfpathlineto{\pgfqpoint{6.568840in}{0.667979in}}%
\pgfpathlineto{\pgfqpoint{6.570080in}{0.668390in}}%
\pgfpathlineto{\pgfqpoint{6.571320in}{0.666891in}}%
\pgfpathlineto{\pgfqpoint{6.575040in}{0.667648in}}%
\pgfpathlineto{\pgfqpoint{6.576280in}{0.670508in}}%
\pgfpathlineto{\pgfqpoint{6.580000in}{0.665179in}}%
\pgfpathlineto{\pgfqpoint{6.581240in}{0.664977in}}%
\pgfpathlineto{\pgfqpoint{6.584960in}{0.667834in}}%
\pgfpathlineto{\pgfqpoint{6.587440in}{0.669686in}}%
\pgfpathlineto{\pgfqpoint{6.589920in}{0.672098in}}%
\pgfpathlineto{\pgfqpoint{6.592400in}{0.673213in}}%
\pgfpathlineto{\pgfqpoint{6.594880in}{0.677563in}}%
\pgfpathlineto{\pgfqpoint{6.596120in}{0.679368in}}%
\pgfpathlineto{\pgfqpoint{6.598600in}{0.672942in}}%
\pgfpathlineto{\pgfqpoint{6.601080in}{0.674372in}}%
\pgfpathlineto{\pgfqpoint{6.603560in}{0.676216in}}%
\pgfpathlineto{\pgfqpoint{6.606040in}{0.675854in}}%
\pgfpathlineto{\pgfqpoint{6.607280in}{0.672559in}}%
\pgfpathlineto{\pgfqpoint{6.609760in}{0.678620in}}%
\pgfpathlineto{\pgfqpoint{6.614720in}{0.673396in}}%
\pgfpathlineto{\pgfqpoint{6.615960in}{0.673185in}}%
\pgfpathlineto{\pgfqpoint{6.618440in}{0.675322in}}%
\pgfpathlineto{\pgfqpoint{6.619680in}{0.672175in}}%
\pgfpathlineto{\pgfqpoint{6.622160in}{0.673611in}}%
\pgfpathlineto{\pgfqpoint{6.623400in}{0.671283in}}%
\pgfpathlineto{\pgfqpoint{6.625880in}{0.675836in}}%
\pgfpathlineto{\pgfqpoint{6.627120in}{0.671920in}}%
\pgfpathlineto{\pgfqpoint{6.628360in}{0.672779in}}%
\pgfpathlineto{\pgfqpoint{6.632080in}{0.666828in}}%
\pgfpathlineto{\pgfqpoint{6.634560in}{0.669907in}}%
\pgfpathlineto{\pgfqpoint{6.635800in}{0.670078in}}%
\pgfpathlineto{\pgfqpoint{6.637040in}{0.668947in}}%
\pgfpathlineto{\pgfqpoint{6.638280in}{0.665942in}}%
\pgfpathlineto{\pgfqpoint{6.640760in}{0.667727in}}%
\pgfpathlineto{\pgfqpoint{6.642000in}{0.668484in}}%
\pgfpathlineto{\pgfqpoint{6.643240in}{0.671256in}}%
\pgfpathlineto{\pgfqpoint{6.644480in}{0.670377in}}%
\pgfpathlineto{\pgfqpoint{6.646960in}{0.660676in}}%
\pgfpathlineto{\pgfqpoint{6.648200in}{0.658452in}}%
\pgfpathlineto{\pgfqpoint{6.651920in}{0.658240in}}%
\pgfpathlineto{\pgfqpoint{6.653160in}{0.656860in}}%
\pgfpathlineto{\pgfqpoint{6.654400in}{0.659571in}}%
\pgfpathlineto{\pgfqpoint{6.656880in}{0.658390in}}%
\pgfpathlineto{\pgfqpoint{6.658120in}{0.656011in}}%
\pgfpathlineto{\pgfqpoint{6.659360in}{0.656719in}}%
\pgfpathlineto{\pgfqpoint{6.664320in}{0.651156in}}%
\pgfpathlineto{\pgfqpoint{6.666800in}{0.654714in}}%
\pgfpathlineto{\pgfqpoint{6.668040in}{0.653493in}}%
\pgfpathlineto{\pgfqpoint{6.670520in}{0.656728in}}%
\pgfpathlineto{\pgfqpoint{6.673000in}{0.663217in}}%
\pgfpathlineto{\pgfqpoint{6.682920in}{0.667845in}}%
\pgfpathlineto{\pgfqpoint{6.687880in}{0.663659in}}%
\pgfpathlineto{\pgfqpoint{6.689120in}{0.659509in}}%
\pgfpathlineto{\pgfqpoint{6.690360in}{0.659730in}}%
\pgfpathlineto{\pgfqpoint{6.692840in}{0.662505in}}%
\pgfpathlineto{\pgfqpoint{6.694080in}{0.663275in}}%
\pgfpathlineto{\pgfqpoint{6.695320in}{0.659944in}}%
\pgfpathlineto{\pgfqpoint{6.699040in}{0.659985in}}%
\pgfpathlineto{\pgfqpoint{6.700280in}{0.663159in}}%
\pgfpathlineto{\pgfqpoint{6.704000in}{0.657596in}}%
\pgfpathlineto{\pgfqpoint{6.706480in}{0.660719in}}%
\pgfpathlineto{\pgfqpoint{6.708960in}{0.660129in}}%
\pgfpathlineto{\pgfqpoint{6.711440in}{0.664545in}}%
\pgfpathlineto{\pgfqpoint{6.712680in}{0.666691in}}%
\pgfpathlineto{\pgfqpoint{6.713920in}{0.665961in}}%
\pgfpathlineto{\pgfqpoint{6.715160in}{0.666585in}}%
\pgfpathlineto{\pgfqpoint{6.716400in}{0.665646in}}%
\pgfpathlineto{\pgfqpoint{6.718880in}{0.671650in}}%
\pgfpathlineto{\pgfqpoint{6.720120in}{0.673316in}}%
\pgfpathlineto{\pgfqpoint{6.722600in}{0.667019in}}%
\pgfpathlineto{\pgfqpoint{6.723840in}{0.667970in}}%
\pgfpathlineto{\pgfqpoint{6.727560in}{0.673275in}}%
\pgfpathlineto{\pgfqpoint{6.730040in}{0.675393in}}%
\pgfpathlineto{\pgfqpoint{6.731280in}{0.674140in}}%
\pgfpathlineto{\pgfqpoint{6.733760in}{0.681948in}}%
\pgfpathlineto{\pgfqpoint{6.735000in}{0.680760in}}%
\pgfpathlineto{\pgfqpoint{6.737480in}{0.676231in}}%
\pgfpathlineto{\pgfqpoint{6.742440in}{0.677181in}}%
\pgfpathlineto{\pgfqpoint{6.743680in}{0.674392in}}%
\pgfpathlineto{\pgfqpoint{6.746160in}{0.680095in}}%
\pgfpathlineto{\pgfqpoint{6.747400in}{0.679028in}}%
\pgfpathlineto{\pgfqpoint{6.749880in}{0.684553in}}%
\pgfpathlineto{\pgfqpoint{6.751120in}{0.681449in}}%
\pgfpathlineto{\pgfqpoint{6.752360in}{0.681745in}}%
\pgfpathlineto{\pgfqpoint{6.756080in}{0.673184in}}%
\pgfpathlineto{\pgfqpoint{6.757320in}{0.673137in}}%
\pgfpathlineto{\pgfqpoint{6.758560in}{0.675244in}}%
\pgfpathlineto{\pgfqpoint{6.761040in}{0.673758in}}%
\pgfpathlineto{\pgfqpoint{6.762280in}{0.670781in}}%
\pgfpathlineto{\pgfqpoint{6.766000in}{0.673533in}}%
\pgfpathlineto{\pgfqpoint{6.767240in}{0.675856in}}%
\pgfpathlineto{\pgfqpoint{6.768480in}{0.672919in}}%
\pgfpathlineto{\pgfqpoint{6.770960in}{0.663123in}}%
\pgfpathlineto{\pgfqpoint{6.773440in}{0.660826in}}%
\pgfpathlineto{\pgfqpoint{6.774680in}{0.660815in}}%
\pgfpathlineto{\pgfqpoint{6.777160in}{0.656608in}}%
\pgfpathlineto{\pgfqpoint{6.778400in}{0.659478in}}%
\pgfpathlineto{\pgfqpoint{6.780880in}{0.659401in}}%
\pgfpathlineto{\pgfqpoint{6.784600in}{0.655240in}}%
\pgfpathlineto{\pgfqpoint{6.787080in}{0.650272in}}%
\pgfpathlineto{\pgfqpoint{6.788320in}{0.651516in}}%
\pgfpathlineto{\pgfqpoint{6.790800in}{0.655073in}}%
\pgfpathlineto{\pgfqpoint{6.792040in}{0.652741in}}%
\pgfpathlineto{\pgfqpoint{6.795760in}{0.659618in}}%
\pgfpathlineto{\pgfqpoint{6.798240in}{0.662515in}}%
\pgfpathlineto{\pgfqpoint{6.799480in}{0.662382in}}%
\pgfpathlineto{\pgfqpoint{6.801960in}{0.665081in}}%
\pgfpathlineto{\pgfqpoint{6.803200in}{0.664608in}}%
\pgfpathlineto{\pgfqpoint{6.805680in}{0.667767in}}%
\pgfpathlineto{\pgfqpoint{6.806920in}{0.668264in}}%
\pgfpathlineto{\pgfqpoint{6.809400in}{0.665140in}}%
\pgfpathlineto{\pgfqpoint{6.810640in}{0.664632in}}%
\pgfpathlineto{\pgfqpoint{6.814360in}{0.658156in}}%
\pgfpathlineto{\pgfqpoint{6.815600in}{0.660689in}}%
\pgfpathlineto{\pgfqpoint{6.816840in}{0.660565in}}%
\pgfpathlineto{\pgfqpoint{6.818080in}{0.662585in}}%
\pgfpathlineto{\pgfqpoint{6.819320in}{0.661562in}}%
\pgfpathlineto{\pgfqpoint{6.821800in}{0.661624in}}%
\pgfpathlineto{\pgfqpoint{6.823040in}{0.660656in}}%
\pgfpathlineto{\pgfqpoint{6.824280in}{0.662712in}}%
\pgfpathlineto{\pgfqpoint{6.825520in}{0.662067in}}%
\pgfpathlineto{\pgfqpoint{6.828000in}{0.657155in}}%
\pgfpathlineto{\pgfqpoint{6.829240in}{0.662708in}}%
\pgfpathlineto{\pgfqpoint{6.832960in}{0.656453in}}%
\pgfpathlineto{\pgfqpoint{6.839160in}{0.666225in}}%
\pgfpathlineto{\pgfqpoint{6.840400in}{0.664337in}}%
\pgfpathlineto{\pgfqpoint{6.842880in}{0.671856in}}%
\pgfpathlineto{\pgfqpoint{6.844120in}{0.674195in}}%
\pgfpathlineto{\pgfqpoint{6.845360in}{0.668228in}}%
\pgfpathlineto{\pgfqpoint{6.847840in}{0.669228in}}%
\pgfpathlineto{\pgfqpoint{6.850320in}{0.675616in}}%
\pgfpathlineto{\pgfqpoint{6.851560in}{0.674338in}}%
\pgfpathlineto{\pgfqpoint{6.855280in}{0.673076in}}%
\pgfpathlineto{\pgfqpoint{6.857760in}{0.682568in}}%
\pgfpathlineto{\pgfqpoint{6.859000in}{0.681145in}}%
\pgfpathlineto{\pgfqpoint{6.861480in}{0.676407in}}%
\pgfpathlineto{\pgfqpoint{6.863960in}{0.679030in}}%
\pgfpathlineto{\pgfqpoint{6.865200in}{0.681319in}}%
\pgfpathlineto{\pgfqpoint{6.867680in}{0.674999in}}%
\pgfpathlineto{\pgfqpoint{6.870160in}{0.681241in}}%
\pgfpathlineto{\pgfqpoint{6.871400in}{0.681438in}}%
\pgfpathlineto{\pgfqpoint{6.873880in}{0.689442in}}%
\pgfpathlineto{\pgfqpoint{6.880080in}{0.674534in}}%
\pgfpathlineto{\pgfqpoint{6.881320in}{0.674123in}}%
\pgfpathlineto{\pgfqpoint{6.882560in}{0.675471in}}%
\pgfpathlineto{\pgfqpoint{6.886280in}{0.664917in}}%
\pgfpathlineto{\pgfqpoint{6.888760in}{0.668948in}}%
\pgfpathlineto{\pgfqpoint{6.891240in}{0.674206in}}%
\pgfpathlineto{\pgfqpoint{6.892480in}{0.672743in}}%
\pgfpathlineto{\pgfqpoint{6.894960in}{0.667374in}}%
\pgfpathlineto{\pgfqpoint{6.897440in}{0.666270in}}%
\pgfpathlineto{\pgfqpoint{6.898680in}{0.668174in}}%
\pgfpathlineto{\pgfqpoint{6.901160in}{0.662852in}}%
\pgfpathlineto{\pgfqpoint{6.902400in}{0.664922in}}%
\pgfpathlineto{\pgfqpoint{6.907360in}{0.659792in}}%
\pgfpathlineto{\pgfqpoint{6.909840in}{0.660228in}}%
\pgfpathlineto{\pgfqpoint{6.911080in}{0.656869in}}%
\pgfpathlineto{\pgfqpoint{6.912320in}{0.658092in}}%
\pgfpathlineto{\pgfqpoint{6.913560in}{0.656579in}}%
\pgfpathlineto{\pgfqpoint{6.914800in}{0.659407in}}%
\pgfpathlineto{\pgfqpoint{6.916040in}{0.658372in}}%
\pgfpathlineto{\pgfqpoint{6.923480in}{0.668067in}}%
\pgfpathlineto{\pgfqpoint{6.927200in}{0.665159in}}%
\pgfpathlineto{\pgfqpoint{6.929680in}{0.668140in}}%
\pgfpathlineto{\pgfqpoint{6.933400in}{0.660851in}}%
\pgfpathlineto{\pgfqpoint{6.935880in}{0.659419in}}%
\pgfpathlineto{\pgfqpoint{6.938360in}{0.656046in}}%
\pgfpathlineto{\pgfqpoint{6.939600in}{0.657457in}}%
\pgfpathlineto{\pgfqpoint{6.940840in}{0.654985in}}%
\pgfpathlineto{\pgfqpoint{6.942080in}{0.656136in}}%
\pgfpathlineto{\pgfqpoint{6.944560in}{0.655093in}}%
\pgfpathlineto{\pgfqpoint{6.947040in}{0.658880in}}%
\pgfpathlineto{\pgfqpoint{6.948280in}{0.662618in}}%
\pgfpathlineto{\pgfqpoint{6.949520in}{0.662592in}}%
\pgfpathlineto{\pgfqpoint{6.952000in}{0.654446in}}%
\pgfpathlineto{\pgfqpoint{6.953240in}{0.663727in}}%
\pgfpathlineto{\pgfqpoint{6.955720in}{0.657162in}}%
\pgfpathlineto{\pgfqpoint{6.956960in}{0.652623in}}%
\pgfpathlineto{\pgfqpoint{6.960680in}{0.664317in}}%
\pgfpathlineto{\pgfqpoint{6.961920in}{0.664167in}}%
\pgfpathlineto{\pgfqpoint{6.963160in}{0.667854in}}%
\pgfpathlineto{\pgfqpoint{6.964400in}{0.667978in}}%
\pgfpathlineto{\pgfqpoint{6.966880in}{0.672410in}}%
\pgfpathlineto{\pgfqpoint{6.968120in}{0.673860in}}%
\pgfpathlineto{\pgfqpoint{6.970600in}{0.664269in}}%
\pgfpathlineto{\pgfqpoint{6.973080in}{0.667682in}}%
\pgfpathlineto{\pgfqpoint{6.974320in}{0.671096in}}%
\pgfpathlineto{\pgfqpoint{6.976800in}{0.664545in}}%
\pgfpathlineto{\pgfqpoint{6.979280in}{0.667004in}}%
\pgfpathlineto{\pgfqpoint{6.981760in}{0.674852in}}%
\pgfpathlineto{\pgfqpoint{6.985480in}{0.662811in}}%
\pgfpathlineto{\pgfqpoint{6.987960in}{0.664081in}}%
\pgfpathlineto{\pgfqpoint{6.989200in}{0.670122in}}%
\pgfpathlineto{\pgfqpoint{6.991680in}{0.665489in}}%
\pgfpathlineto{\pgfqpoint{6.997880in}{0.679521in}}%
\pgfpathlineto{\pgfqpoint{6.999120in}{0.676617in}}%
\pgfpathlineto{\pgfqpoint{7.000360in}{0.678974in}}%
\pgfpathlineto{\pgfqpoint{7.002840in}{0.670837in}}%
\pgfpathlineto{\pgfqpoint{7.005320in}{0.667505in}}%
\pgfpathlineto{\pgfqpoint{7.006560in}{0.667970in}}%
\pgfpathlineto{\pgfqpoint{7.007800in}{0.663849in}}%
\pgfpathlineto{\pgfqpoint{7.009040in}{0.665490in}}%
\pgfpathlineto{\pgfqpoint{7.010280in}{0.660157in}}%
\pgfpathlineto{\pgfqpoint{7.014000in}{0.668700in}}%
\pgfpathlineto{\pgfqpoint{7.015240in}{0.673104in}}%
\pgfpathlineto{\pgfqpoint{7.016480in}{0.670323in}}%
\pgfpathlineto{\pgfqpoint{7.018960in}{0.662746in}}%
\pgfpathlineto{\pgfqpoint{7.022680in}{0.668490in}}%
\pgfpathlineto{\pgfqpoint{7.023920in}{0.665711in}}%
\pgfpathlineto{\pgfqpoint{7.026400in}{0.669168in}}%
\pgfpathlineto{\pgfqpoint{7.030120in}{0.661452in}}%
\pgfpathlineto{\pgfqpoint{7.033840in}{0.657906in}}%
\pgfpathlineto{\pgfqpoint{7.037560in}{0.657482in}}%
\pgfpathlineto{\pgfqpoint{7.038800in}{0.660372in}}%
\pgfpathlineto{\pgfqpoint{7.041280in}{0.669110in}}%
\pgfpathlineto{\pgfqpoint{7.042520in}{0.668972in}}%
\pgfpathlineto{\pgfqpoint{7.045000in}{0.671777in}}%
\pgfpathlineto{\pgfqpoint{7.047480in}{0.678054in}}%
\pgfpathlineto{\pgfqpoint{7.051200in}{0.676157in}}%
\pgfpathlineto{\pgfqpoint{7.052440in}{0.674309in}}%
\pgfpathlineto{\pgfqpoint{7.053680in}{0.676150in}}%
\pgfpathlineto{\pgfqpoint{7.058640in}{0.662412in}}%
\pgfpathlineto{\pgfqpoint{7.059880in}{0.663127in}}%
\pgfpathlineto{\pgfqpoint{7.062360in}{0.657444in}}%
\pgfpathlineto{\pgfqpoint{7.063600in}{0.662704in}}%
\pgfpathlineto{\pgfqpoint{7.066080in}{0.658913in}}%
\pgfpathlineto{\pgfqpoint{7.067320in}{0.659931in}}%
\pgfpathlineto{\pgfqpoint{7.069800in}{0.656366in}}%
\pgfpathlineto{\pgfqpoint{7.071040in}{0.658463in}}%
\pgfpathlineto{\pgfqpoint{7.073520in}{0.665979in}}%
\pgfpathlineto{\pgfqpoint{7.074760in}{0.664648in}}%
\pgfpathlineto{\pgfqpoint{7.076000in}{0.659566in}}%
\pgfpathlineto{\pgfqpoint{7.077240in}{0.676873in}}%
\pgfpathlineto{\pgfqpoint{7.080960in}{0.663359in}}%
\pgfpathlineto{\pgfqpoint{7.083440in}{0.666510in}}%
\pgfpathlineto{\pgfqpoint{7.088400in}{0.678327in}}%
\pgfpathlineto{\pgfqpoint{7.090880in}{0.678339in}}%
\pgfpathlineto{\pgfqpoint{7.092120in}{0.684444in}}%
\pgfpathlineto{\pgfqpoint{7.094600in}{0.670954in}}%
\pgfpathlineto{\pgfqpoint{7.097080in}{0.675069in}}%
\pgfpathlineto{\pgfqpoint{7.098320in}{0.685550in}}%
\pgfpathlineto{\pgfqpoint{7.099560in}{0.684555in}}%
\pgfpathlineto{\pgfqpoint{7.100800in}{0.686426in}}%
\pgfpathlineto{\pgfqpoint{7.103280in}{0.686350in}}%
\pgfpathlineto{\pgfqpoint{7.104520in}{0.693760in}}%
\pgfpathlineto{\pgfqpoint{7.105760in}{0.692313in}}%
\pgfpathlineto{\pgfqpoint{7.109480in}{0.674051in}}%
\pgfpathlineto{\pgfqpoint{7.111960in}{0.678883in}}%
\pgfpathlineto{\pgfqpoint{7.113200in}{0.688831in}}%
\pgfpathlineto{\pgfqpoint{7.116920in}{0.676220in}}%
\pgfpathlineto{\pgfqpoint{7.118160in}{0.677589in}}%
\pgfpathlineto{\pgfqpoint{7.119400in}{0.675380in}}%
\pgfpathlineto{\pgfqpoint{7.120640in}{0.678974in}}%
\pgfpathlineto{\pgfqpoint{7.123120in}{0.678483in}}%
\pgfpathlineto{\pgfqpoint{7.124360in}{0.681121in}}%
\pgfpathlineto{\pgfqpoint{7.125600in}{0.672510in}}%
\pgfpathlineto{\pgfqpoint{7.128080in}{0.680704in}}%
\pgfpathlineto{\pgfqpoint{7.130560in}{0.682185in}}%
\pgfpathlineto{\pgfqpoint{7.134280in}{0.669771in}}%
\pgfpathlineto{\pgfqpoint{7.135520in}{0.670662in}}%
\pgfpathlineto{\pgfqpoint{7.139240in}{0.693466in}}%
\pgfpathlineto{\pgfqpoint{7.142960in}{0.672030in}}%
\pgfpathlineto{\pgfqpoint{7.146680in}{0.667507in}}%
\pgfpathlineto{\pgfqpoint{7.147920in}{0.660304in}}%
\pgfpathlineto{\pgfqpoint{7.150400in}{0.663064in}}%
\pgfpathlineto{\pgfqpoint{7.151640in}{0.660852in}}%
\pgfpathlineto{\pgfqpoint{7.152880in}{0.654916in}}%
\pgfpathlineto{\pgfqpoint{7.154120in}{0.654822in}}%
\pgfpathlineto{\pgfqpoint{7.156600in}{0.667966in}}%
\pgfpathlineto{\pgfqpoint{7.157840in}{0.670473in}}%
\pgfpathlineto{\pgfqpoint{7.159080in}{0.669147in}}%
\pgfpathlineto{\pgfqpoint{7.164040in}{0.688037in}}%
\pgfpathlineto{\pgfqpoint{7.165280in}{0.690161in}}%
\pgfpathlineto{\pgfqpoint{7.167760in}{0.686318in}}%
\pgfpathlineto{\pgfqpoint{7.169000in}{0.686504in}}%
\pgfpathlineto{\pgfqpoint{7.170240in}{0.693975in}}%
\pgfpathlineto{\pgfqpoint{7.171480in}{0.691926in}}%
\pgfpathlineto{\pgfqpoint{7.175200in}{0.698540in}}%
\pgfpathlineto{\pgfqpoint{7.181400in}{0.674119in}}%
\pgfpathlineto{\pgfqpoint{7.182640in}{0.671997in}}%
\pgfpathlineto{\pgfqpoint{7.183880in}{0.680242in}}%
\pgfpathlineto{\pgfqpoint{7.185120in}{0.678864in}}%
\pgfpathlineto{\pgfqpoint{7.186360in}{0.675599in}}%
\pgfpathlineto{\pgfqpoint{7.188840in}{0.687967in}}%
\pgfpathlineto{\pgfqpoint{7.191320in}{0.686265in}}%
\pgfpathlineto{\pgfqpoint{7.193800in}{0.680232in}}%
\pgfpathlineto{\pgfqpoint{7.197520in}{0.689050in}}%
\pgfpathlineto{\pgfqpoint{7.198760in}{0.689851in}}%
\pgfpathlineto{\pgfqpoint{7.200000in}{0.687151in}}%
\pgfpathlineto{\pgfqpoint{7.200000in}{0.687151in}}%
\pgfusepath{stroke}%
\end{pgfscope}%
\begin{pgfscope}%
\pgfpathrectangle{\pgfqpoint{1.000000in}{0.350000in}}{\pgfqpoint{6.200000in}{2.800000in}} %
\pgfusepath{clip}%
\pgfsetrectcap%
\pgfsetroundjoin%
\pgfsetlinewidth{1.003750pt}%
\definecolor{currentstroke}{rgb}{1.000000,0.000000,0.000000}%
\pgfsetstrokecolor{currentstroke}%
\pgfsetdash{}{0pt}%
\pgfpathmoveto{\pgfqpoint{1.001240in}{1.774215in}}%
\pgfpathlineto{\pgfqpoint{1.002480in}{2.510200in}}%
\pgfpathlineto{\pgfqpoint{1.003720in}{2.825165in}}%
\pgfpathlineto{\pgfqpoint{1.004960in}{2.886981in}}%
\pgfpathlineto{\pgfqpoint{1.008680in}{2.826133in}}%
\pgfpathlineto{\pgfqpoint{1.014880in}{2.690100in}}%
\pgfpathlineto{\pgfqpoint{1.017360in}{2.659437in}}%
\pgfpathlineto{\pgfqpoint{1.023560in}{2.585453in}}%
\pgfpathlineto{\pgfqpoint{1.024800in}{2.586676in}}%
\pgfpathlineto{\pgfqpoint{1.027280in}{2.583759in}}%
\pgfpathlineto{\pgfqpoint{1.029760in}{2.565573in}}%
\pgfpathlineto{\pgfqpoint{1.031000in}{2.566123in}}%
\pgfpathlineto{\pgfqpoint{1.032240in}{2.565215in}}%
\pgfpathlineto{\pgfqpoint{1.033480in}{2.561869in}}%
\pgfpathlineto{\pgfqpoint{1.038440in}{2.504394in}}%
\pgfpathlineto{\pgfqpoint{1.042160in}{2.482819in}}%
\pgfpathlineto{\pgfqpoint{1.044640in}{2.490704in}}%
\pgfpathlineto{\pgfqpoint{1.045880in}{2.491285in}}%
\pgfpathlineto{\pgfqpoint{1.050840in}{2.464797in}}%
\pgfpathlineto{\pgfqpoint{1.052080in}{2.463037in}}%
\pgfpathlineto{\pgfqpoint{1.057040in}{2.427839in}}%
\pgfpathlineto{\pgfqpoint{1.059520in}{2.412494in}}%
\pgfpathlineto{\pgfqpoint{1.060760in}{2.414698in}}%
\pgfpathlineto{\pgfqpoint{1.062000in}{2.412791in}}%
\pgfpathlineto{\pgfqpoint{1.066960in}{2.392809in}}%
\pgfpathlineto{\pgfqpoint{1.068200in}{2.394792in}}%
\pgfpathlineto{\pgfqpoint{1.069440in}{2.399444in}}%
\pgfpathlineto{\pgfqpoint{1.070680in}{2.396984in}}%
\pgfpathlineto{\pgfqpoint{1.073160in}{2.380747in}}%
\pgfpathlineto{\pgfqpoint{1.074400in}{2.379257in}}%
\pgfpathlineto{\pgfqpoint{1.075640in}{2.375826in}}%
\pgfpathlineto{\pgfqpoint{1.076880in}{2.376772in}}%
\pgfpathlineto{\pgfqpoint{1.078120in}{2.372469in}}%
\pgfpathlineto{\pgfqpoint{1.079360in}{2.375690in}}%
\pgfpathlineto{\pgfqpoint{1.081840in}{2.368404in}}%
\pgfpathlineto{\pgfqpoint{1.084320in}{2.375882in}}%
\pgfpathlineto{\pgfqpoint{1.085560in}{2.374537in}}%
\pgfpathlineto{\pgfqpoint{1.090520in}{2.363940in}}%
\pgfpathlineto{\pgfqpoint{1.091760in}{2.363758in}}%
\pgfpathlineto{\pgfqpoint{1.093000in}{2.364949in}}%
\pgfpathlineto{\pgfqpoint{1.094240in}{2.369657in}}%
\pgfpathlineto{\pgfqpoint{1.096720in}{2.356760in}}%
\pgfpathlineto{\pgfqpoint{1.100440in}{2.346911in}}%
\pgfpathlineto{\pgfqpoint{1.101680in}{2.348092in}}%
\pgfpathlineto{\pgfqpoint{1.102920in}{2.349620in}}%
\pgfpathlineto{\pgfqpoint{1.104160in}{2.348469in}}%
\pgfpathlineto{\pgfqpoint{1.105400in}{2.352567in}}%
\pgfpathlineto{\pgfqpoint{1.106640in}{2.349769in}}%
\pgfpathlineto{\pgfqpoint{1.110360in}{2.336066in}}%
\pgfpathlineto{\pgfqpoint{1.114080in}{2.336196in}}%
\pgfpathlineto{\pgfqpoint{1.119040in}{2.350196in}}%
\pgfpathlineto{\pgfqpoint{1.120280in}{2.351546in}}%
\pgfpathlineto{\pgfqpoint{1.122760in}{2.338828in}}%
\pgfpathlineto{\pgfqpoint{1.126480in}{2.318693in}}%
\pgfpathlineto{\pgfqpoint{1.128960in}{2.320081in}}%
\pgfpathlineto{\pgfqpoint{1.131440in}{2.318834in}}%
\pgfpathlineto{\pgfqpoint{1.133920in}{2.313658in}}%
\pgfpathlineto{\pgfqpoint{1.135160in}{2.313844in}}%
\pgfpathlineto{\pgfqpoint{1.137640in}{2.310168in}}%
\pgfpathlineto{\pgfqpoint{1.140120in}{2.318654in}}%
\pgfpathlineto{\pgfqpoint{1.141360in}{2.317341in}}%
\pgfpathlineto{\pgfqpoint{1.142600in}{2.313382in}}%
\pgfpathlineto{\pgfqpoint{1.143840in}{2.313862in}}%
\pgfpathlineto{\pgfqpoint{1.145080in}{2.313049in}}%
\pgfpathlineto{\pgfqpoint{1.146320in}{2.319518in}}%
\pgfpathlineto{\pgfqpoint{1.148800in}{2.316136in}}%
\pgfpathlineto{\pgfqpoint{1.152520in}{2.317824in}}%
\pgfpathlineto{\pgfqpoint{1.155000in}{2.317575in}}%
\pgfpathlineto{\pgfqpoint{1.157480in}{2.326168in}}%
\pgfpathlineto{\pgfqpoint{1.161200in}{2.322117in}}%
\pgfpathlineto{\pgfqpoint{1.162440in}{2.322872in}}%
\pgfpathlineto{\pgfqpoint{1.166160in}{2.313552in}}%
\pgfpathlineto{\pgfqpoint{1.167400in}{2.319502in}}%
\pgfpathlineto{\pgfqpoint{1.168640in}{2.319890in}}%
\pgfpathlineto{\pgfqpoint{1.169880in}{2.324360in}}%
\pgfpathlineto{\pgfqpoint{1.172360in}{2.320475in}}%
\pgfpathlineto{\pgfqpoint{1.174840in}{2.317252in}}%
\pgfpathlineto{\pgfqpoint{1.176080in}{2.318274in}}%
\pgfpathlineto{\pgfqpoint{1.177320in}{2.316643in}}%
\pgfpathlineto{\pgfqpoint{1.179800in}{2.307597in}}%
\pgfpathlineto{\pgfqpoint{1.182280in}{2.306550in}}%
\pgfpathlineto{\pgfqpoint{1.183520in}{2.300807in}}%
\pgfpathlineto{\pgfqpoint{1.188480in}{2.300667in}}%
\pgfpathlineto{\pgfqpoint{1.189720in}{2.303358in}}%
\pgfpathlineto{\pgfqpoint{1.190960in}{2.302236in}}%
\pgfpathlineto{\pgfqpoint{1.192200in}{2.307678in}}%
\pgfpathlineto{\pgfqpoint{1.193440in}{2.308019in}}%
\pgfpathlineto{\pgfqpoint{1.194680in}{2.310106in}}%
\pgfpathlineto{\pgfqpoint{1.195920in}{2.308182in}}%
\pgfpathlineto{\pgfqpoint{1.197160in}{2.308846in}}%
\pgfpathlineto{\pgfqpoint{1.199640in}{2.305614in}}%
\pgfpathlineto{\pgfqpoint{1.200880in}{2.308067in}}%
\pgfpathlineto{\pgfqpoint{1.204600in}{2.305431in}}%
\pgfpathlineto{\pgfqpoint{1.205840in}{2.305365in}}%
\pgfpathlineto{\pgfqpoint{1.207080in}{2.309776in}}%
\pgfpathlineto{\pgfqpoint{1.209560in}{2.309871in}}%
\pgfpathlineto{\pgfqpoint{1.213280in}{2.314245in}}%
\pgfpathlineto{\pgfqpoint{1.214520in}{2.311079in}}%
\pgfpathlineto{\pgfqpoint{1.217000in}{2.314273in}}%
\pgfpathlineto{\pgfqpoint{1.218240in}{2.317843in}}%
\pgfpathlineto{\pgfqpoint{1.219480in}{2.315390in}}%
\pgfpathlineto{\pgfqpoint{1.220720in}{2.316275in}}%
\pgfpathlineto{\pgfqpoint{1.223200in}{2.311060in}}%
\pgfpathlineto{\pgfqpoint{1.224440in}{2.305932in}}%
\pgfpathlineto{\pgfqpoint{1.225680in}{2.305726in}}%
\pgfpathlineto{\pgfqpoint{1.228160in}{2.302882in}}%
\pgfpathlineto{\pgfqpoint{1.229400in}{2.304260in}}%
\pgfpathlineto{\pgfqpoint{1.230640in}{2.303287in}}%
\pgfpathlineto{\pgfqpoint{1.234360in}{2.290159in}}%
\pgfpathlineto{\pgfqpoint{1.235600in}{2.290296in}}%
\pgfpathlineto{\pgfqpoint{1.236840in}{2.292902in}}%
\pgfpathlineto{\pgfqpoint{1.240560in}{2.288816in}}%
\pgfpathlineto{\pgfqpoint{1.244280in}{2.304620in}}%
\pgfpathlineto{\pgfqpoint{1.245520in}{2.303987in}}%
\pgfpathlineto{\pgfqpoint{1.250480in}{2.278648in}}%
\pgfpathlineto{\pgfqpoint{1.255440in}{2.280968in}}%
\pgfpathlineto{\pgfqpoint{1.256680in}{2.280021in}}%
\pgfpathlineto{\pgfqpoint{1.261640in}{2.284595in}}%
\pgfpathlineto{\pgfqpoint{1.264120in}{2.291752in}}%
\pgfpathlineto{\pgfqpoint{1.265360in}{2.290844in}}%
\pgfpathlineto{\pgfqpoint{1.269080in}{2.283878in}}%
\pgfpathlineto{\pgfqpoint{1.270320in}{2.286429in}}%
\pgfpathlineto{\pgfqpoint{1.274040in}{2.285313in}}%
\pgfpathlineto{\pgfqpoint{1.275280in}{2.287040in}}%
\pgfpathlineto{\pgfqpoint{1.277760in}{2.284347in}}%
\pgfpathlineto{\pgfqpoint{1.279000in}{2.280880in}}%
\pgfpathlineto{\pgfqpoint{1.280240in}{2.281798in}}%
\pgfpathlineto{\pgfqpoint{1.281480in}{2.279510in}}%
\pgfpathlineto{\pgfqpoint{1.283960in}{2.282009in}}%
\pgfpathlineto{\pgfqpoint{1.285200in}{2.281071in}}%
\pgfpathlineto{\pgfqpoint{1.286440in}{2.282045in}}%
\pgfpathlineto{\pgfqpoint{1.287680in}{2.279010in}}%
\pgfpathlineto{\pgfqpoint{1.290160in}{2.279953in}}%
\pgfpathlineto{\pgfqpoint{1.291400in}{2.283121in}}%
\pgfpathlineto{\pgfqpoint{1.292640in}{2.280699in}}%
\pgfpathlineto{\pgfqpoint{1.293880in}{2.282105in}}%
\pgfpathlineto{\pgfqpoint{1.297600in}{2.279070in}}%
\pgfpathlineto{\pgfqpoint{1.298840in}{2.279457in}}%
\pgfpathlineto{\pgfqpoint{1.300080in}{2.278653in}}%
\pgfpathlineto{\pgfqpoint{1.301320in}{2.280297in}}%
\pgfpathlineto{\pgfqpoint{1.302560in}{2.278812in}}%
\pgfpathlineto{\pgfqpoint{1.305040in}{2.281507in}}%
\pgfpathlineto{\pgfqpoint{1.306280in}{2.280056in}}%
\pgfpathlineto{\pgfqpoint{1.308760in}{2.271638in}}%
\pgfpathlineto{\pgfqpoint{1.312480in}{2.273465in}}%
\pgfpathlineto{\pgfqpoint{1.318680in}{2.288480in}}%
\pgfpathlineto{\pgfqpoint{1.319920in}{2.286124in}}%
\pgfpathlineto{\pgfqpoint{1.321160in}{2.287309in}}%
\pgfpathlineto{\pgfqpoint{1.323640in}{2.285012in}}%
\pgfpathlineto{\pgfqpoint{1.326120in}{2.285580in}}%
\pgfpathlineto{\pgfqpoint{1.327360in}{2.284993in}}%
\pgfpathlineto{\pgfqpoint{1.328600in}{2.281824in}}%
\pgfpathlineto{\pgfqpoint{1.329840in}{2.282710in}}%
\pgfpathlineto{\pgfqpoint{1.331080in}{2.285674in}}%
\pgfpathlineto{\pgfqpoint{1.332320in}{2.285703in}}%
\pgfpathlineto{\pgfqpoint{1.333560in}{2.283176in}}%
\pgfpathlineto{\pgfqpoint{1.337280in}{2.287219in}}%
\pgfpathlineto{\pgfqpoint{1.338520in}{2.284424in}}%
\pgfpathlineto{\pgfqpoint{1.342240in}{2.289083in}}%
\pgfpathlineto{\pgfqpoint{1.343480in}{2.285658in}}%
\pgfpathlineto{\pgfqpoint{1.344720in}{2.286230in}}%
\pgfpathlineto{\pgfqpoint{1.345960in}{2.285379in}}%
\pgfpathlineto{\pgfqpoint{1.347200in}{2.282807in}}%
\pgfpathlineto{\pgfqpoint{1.349680in}{2.274377in}}%
\pgfpathlineto{\pgfqpoint{1.352160in}{2.271246in}}%
\pgfpathlineto{\pgfqpoint{1.354640in}{2.273232in}}%
\pgfpathlineto{\pgfqpoint{1.357120in}{2.269674in}}%
\pgfpathlineto{\pgfqpoint{1.358360in}{2.265845in}}%
\pgfpathlineto{\pgfqpoint{1.359600in}{2.265972in}}%
\pgfpathlineto{\pgfqpoint{1.360840in}{2.270122in}}%
\pgfpathlineto{\pgfqpoint{1.362080in}{2.269218in}}%
\pgfpathlineto{\pgfqpoint{1.364560in}{2.265122in}}%
\pgfpathlineto{\pgfqpoint{1.368280in}{2.276588in}}%
\pgfpathlineto{\pgfqpoint{1.369520in}{2.276275in}}%
\pgfpathlineto{\pgfqpoint{1.373240in}{2.262031in}}%
\pgfpathlineto{\pgfqpoint{1.378200in}{2.263381in}}%
\pgfpathlineto{\pgfqpoint{1.381920in}{2.261384in}}%
\pgfpathlineto{\pgfqpoint{1.383160in}{2.263172in}}%
\pgfpathlineto{\pgfqpoint{1.384400in}{2.261875in}}%
\pgfpathlineto{\pgfqpoint{1.388120in}{2.268721in}}%
\pgfpathlineto{\pgfqpoint{1.389360in}{2.269067in}}%
\pgfpathlineto{\pgfqpoint{1.393080in}{2.263664in}}%
\pgfpathlineto{\pgfqpoint{1.394320in}{2.266109in}}%
\pgfpathlineto{\pgfqpoint{1.396800in}{2.260925in}}%
\pgfpathlineto{\pgfqpoint{1.400520in}{2.264380in}}%
\pgfpathlineto{\pgfqpoint{1.401760in}{2.263721in}}%
\pgfpathlineto{\pgfqpoint{1.403000in}{2.260468in}}%
\pgfpathlineto{\pgfqpoint{1.404240in}{2.262042in}}%
\pgfpathlineto{\pgfqpoint{1.405480in}{2.260060in}}%
\pgfpathlineto{\pgfqpoint{1.406720in}{2.262831in}}%
\pgfpathlineto{\pgfqpoint{1.407960in}{2.262876in}}%
\pgfpathlineto{\pgfqpoint{1.412920in}{2.255532in}}%
\pgfpathlineto{\pgfqpoint{1.415400in}{2.260563in}}%
\pgfpathlineto{\pgfqpoint{1.416640in}{2.257685in}}%
\pgfpathlineto{\pgfqpoint{1.417880in}{2.257687in}}%
\pgfpathlineto{\pgfqpoint{1.421600in}{2.253573in}}%
\pgfpathlineto{\pgfqpoint{1.424080in}{2.253691in}}%
\pgfpathlineto{\pgfqpoint{1.425320in}{2.255941in}}%
\pgfpathlineto{\pgfqpoint{1.426560in}{2.255525in}}%
\pgfpathlineto{\pgfqpoint{1.429040in}{2.258538in}}%
\pgfpathlineto{\pgfqpoint{1.430280in}{2.257580in}}%
\pgfpathlineto{\pgfqpoint{1.432760in}{2.251670in}}%
\pgfpathlineto{\pgfqpoint{1.434000in}{2.251274in}}%
\pgfpathlineto{\pgfqpoint{1.436480in}{2.249162in}}%
\pgfpathlineto{\pgfqpoint{1.440200in}{2.258528in}}%
\pgfpathlineto{\pgfqpoint{1.442680in}{2.263319in}}%
\pgfpathlineto{\pgfqpoint{1.445160in}{2.260546in}}%
\pgfpathlineto{\pgfqpoint{1.447640in}{2.260015in}}%
\pgfpathlineto{\pgfqpoint{1.450120in}{2.259719in}}%
\pgfpathlineto{\pgfqpoint{1.453840in}{2.255201in}}%
\pgfpathlineto{\pgfqpoint{1.455080in}{2.256877in}}%
\pgfpathlineto{\pgfqpoint{1.456320in}{2.255894in}}%
\pgfpathlineto{\pgfqpoint{1.457560in}{2.253427in}}%
\pgfpathlineto{\pgfqpoint{1.461280in}{2.255667in}}%
\pgfpathlineto{\pgfqpoint{1.462520in}{2.251531in}}%
\pgfpathlineto{\pgfqpoint{1.466240in}{2.255280in}}%
\pgfpathlineto{\pgfqpoint{1.468720in}{2.250658in}}%
\pgfpathlineto{\pgfqpoint{1.469960in}{2.250760in}}%
\pgfpathlineto{\pgfqpoint{1.476160in}{2.241647in}}%
\pgfpathlineto{\pgfqpoint{1.477400in}{2.242148in}}%
\pgfpathlineto{\pgfqpoint{1.478640in}{2.244070in}}%
\pgfpathlineto{\pgfqpoint{1.481120in}{2.241417in}}%
\pgfpathlineto{\pgfqpoint{1.482360in}{2.237674in}}%
\pgfpathlineto{\pgfqpoint{1.483600in}{2.239162in}}%
\pgfpathlineto{\pgfqpoint{1.486080in}{2.243757in}}%
\pgfpathlineto{\pgfqpoint{1.488560in}{2.238881in}}%
\pgfpathlineto{\pgfqpoint{1.491040in}{2.243300in}}%
\pgfpathlineto{\pgfqpoint{1.492280in}{2.248274in}}%
\pgfpathlineto{\pgfqpoint{1.493520in}{2.246162in}}%
\pgfpathlineto{\pgfqpoint{1.497240in}{2.230735in}}%
\pgfpathlineto{\pgfqpoint{1.499720in}{2.233367in}}%
\pgfpathlineto{\pgfqpoint{1.500960in}{2.233069in}}%
\pgfpathlineto{\pgfqpoint{1.503440in}{2.230026in}}%
\pgfpathlineto{\pgfqpoint{1.505920in}{2.231536in}}%
\pgfpathlineto{\pgfqpoint{1.507160in}{2.233655in}}%
\pgfpathlineto{\pgfqpoint{1.508400in}{2.231539in}}%
\pgfpathlineto{\pgfqpoint{1.509640in}{2.232411in}}%
\pgfpathlineto{\pgfqpoint{1.512120in}{2.234796in}}%
\pgfpathlineto{\pgfqpoint{1.513360in}{2.234802in}}%
\pgfpathlineto{\pgfqpoint{1.515840in}{2.230353in}}%
\pgfpathlineto{\pgfqpoint{1.517080in}{2.230494in}}%
\pgfpathlineto{\pgfqpoint{1.518320in}{2.232644in}}%
\pgfpathlineto{\pgfqpoint{1.520800in}{2.228173in}}%
\pgfpathlineto{\pgfqpoint{1.523280in}{2.230305in}}%
\pgfpathlineto{\pgfqpoint{1.525760in}{2.230314in}}%
\pgfpathlineto{\pgfqpoint{1.527000in}{2.228037in}}%
\pgfpathlineto{\pgfqpoint{1.528240in}{2.229715in}}%
\pgfpathlineto{\pgfqpoint{1.529480in}{2.229271in}}%
\pgfpathlineto{\pgfqpoint{1.531960in}{2.231993in}}%
\pgfpathlineto{\pgfqpoint{1.533200in}{2.231040in}}%
\pgfpathlineto{\pgfqpoint{1.535680in}{2.227637in}}%
\pgfpathlineto{\pgfqpoint{1.536920in}{2.227109in}}%
\pgfpathlineto{\pgfqpoint{1.538160in}{2.227873in}}%
\pgfpathlineto{\pgfqpoint{1.539400in}{2.230327in}}%
\pgfpathlineto{\pgfqpoint{1.540640in}{2.227941in}}%
\pgfpathlineto{\pgfqpoint{1.541880in}{2.228841in}}%
\pgfpathlineto{\pgfqpoint{1.544360in}{2.226399in}}%
\pgfpathlineto{\pgfqpoint{1.546840in}{2.227249in}}%
\pgfpathlineto{\pgfqpoint{1.548080in}{2.227905in}}%
\pgfpathlineto{\pgfqpoint{1.549320in}{2.230385in}}%
\pgfpathlineto{\pgfqpoint{1.550560in}{2.229606in}}%
\pgfpathlineto{\pgfqpoint{1.553040in}{2.231106in}}%
\pgfpathlineto{\pgfqpoint{1.559240in}{2.224025in}}%
\pgfpathlineto{\pgfqpoint{1.560480in}{2.225247in}}%
\pgfpathlineto{\pgfqpoint{1.564200in}{2.235311in}}%
\pgfpathlineto{\pgfqpoint{1.566680in}{2.240199in}}%
\pgfpathlineto{\pgfqpoint{1.569160in}{2.234425in}}%
\pgfpathlineto{\pgfqpoint{1.574120in}{2.232950in}}%
\pgfpathlineto{\pgfqpoint{1.577840in}{2.228595in}}%
\pgfpathlineto{\pgfqpoint{1.580320in}{2.231513in}}%
\pgfpathlineto{\pgfqpoint{1.581560in}{2.228939in}}%
\pgfpathlineto{\pgfqpoint{1.585280in}{2.231563in}}%
\pgfpathlineto{\pgfqpoint{1.586520in}{2.228089in}}%
\pgfpathlineto{\pgfqpoint{1.589000in}{2.230225in}}%
\pgfpathlineto{\pgfqpoint{1.590240in}{2.232789in}}%
\pgfpathlineto{\pgfqpoint{1.592720in}{2.228404in}}%
\pgfpathlineto{\pgfqpoint{1.593960in}{2.229375in}}%
\pgfpathlineto{\pgfqpoint{1.595200in}{2.228873in}}%
\pgfpathlineto{\pgfqpoint{1.597680in}{2.223035in}}%
\pgfpathlineto{\pgfqpoint{1.598920in}{2.221218in}}%
\pgfpathlineto{\pgfqpoint{1.601400in}{2.221884in}}%
\pgfpathlineto{\pgfqpoint{1.602640in}{2.223452in}}%
\pgfpathlineto{\pgfqpoint{1.603880in}{2.222842in}}%
\pgfpathlineto{\pgfqpoint{1.606360in}{2.217473in}}%
\pgfpathlineto{\pgfqpoint{1.607600in}{2.218974in}}%
\pgfpathlineto{\pgfqpoint{1.610080in}{2.222843in}}%
\pgfpathlineto{\pgfqpoint{1.612560in}{2.216289in}}%
\pgfpathlineto{\pgfqpoint{1.613800in}{2.217169in}}%
\pgfpathlineto{\pgfqpoint{1.616280in}{2.223824in}}%
\pgfpathlineto{\pgfqpoint{1.617520in}{2.221623in}}%
\pgfpathlineto{\pgfqpoint{1.620000in}{2.211689in}}%
\pgfpathlineto{\pgfqpoint{1.624960in}{2.212819in}}%
\pgfpathlineto{\pgfqpoint{1.628680in}{2.207176in}}%
\pgfpathlineto{\pgfqpoint{1.629920in}{2.207297in}}%
\pgfpathlineto{\pgfqpoint{1.631160in}{2.209210in}}%
\pgfpathlineto{\pgfqpoint{1.632400in}{2.207476in}}%
\pgfpathlineto{\pgfqpoint{1.636120in}{2.211188in}}%
\pgfpathlineto{\pgfqpoint{1.637360in}{2.211643in}}%
\pgfpathlineto{\pgfqpoint{1.639840in}{2.207181in}}%
\pgfpathlineto{\pgfqpoint{1.641080in}{2.206517in}}%
\pgfpathlineto{\pgfqpoint{1.642320in}{2.208639in}}%
\pgfpathlineto{\pgfqpoint{1.646040in}{2.203848in}}%
\pgfpathlineto{\pgfqpoint{1.648520in}{2.203976in}}%
\pgfpathlineto{\pgfqpoint{1.649760in}{2.204911in}}%
\pgfpathlineto{\pgfqpoint{1.651000in}{2.203649in}}%
\pgfpathlineto{\pgfqpoint{1.655960in}{2.210335in}}%
\pgfpathlineto{\pgfqpoint{1.658440in}{2.206081in}}%
\pgfpathlineto{\pgfqpoint{1.660920in}{2.206627in}}%
\pgfpathlineto{\pgfqpoint{1.662160in}{2.207131in}}%
\pgfpathlineto{\pgfqpoint{1.663400in}{2.209592in}}%
\pgfpathlineto{\pgfqpoint{1.664640in}{2.207922in}}%
\pgfpathlineto{\pgfqpoint{1.665880in}{2.208556in}}%
\pgfpathlineto{\pgfqpoint{1.668360in}{2.205719in}}%
\pgfpathlineto{\pgfqpoint{1.677040in}{2.212013in}}%
\pgfpathlineto{\pgfqpoint{1.683240in}{2.205694in}}%
\pgfpathlineto{\pgfqpoint{1.684480in}{2.206766in}}%
\pgfpathlineto{\pgfqpoint{1.688200in}{2.215658in}}%
\pgfpathlineto{\pgfqpoint{1.690680in}{2.219720in}}%
\pgfpathlineto{\pgfqpoint{1.694400in}{2.213444in}}%
\pgfpathlineto{\pgfqpoint{1.700600in}{2.212812in}}%
\pgfpathlineto{\pgfqpoint{1.701840in}{2.211243in}}%
\pgfpathlineto{\pgfqpoint{1.704320in}{2.212851in}}%
\pgfpathlineto{\pgfqpoint{1.705560in}{2.210871in}}%
\pgfpathlineto{\pgfqpoint{1.709280in}{2.214999in}}%
\pgfpathlineto{\pgfqpoint{1.710520in}{2.211834in}}%
\pgfpathlineto{\pgfqpoint{1.713000in}{2.213763in}}%
\pgfpathlineto{\pgfqpoint{1.714240in}{2.216505in}}%
\pgfpathlineto{\pgfqpoint{1.715480in}{2.213683in}}%
\pgfpathlineto{\pgfqpoint{1.719200in}{2.215201in}}%
\pgfpathlineto{\pgfqpoint{1.721680in}{2.210181in}}%
\pgfpathlineto{\pgfqpoint{1.724160in}{2.208196in}}%
\pgfpathlineto{\pgfqpoint{1.725400in}{2.208335in}}%
\pgfpathlineto{\pgfqpoint{1.727880in}{2.210625in}}%
\pgfpathlineto{\pgfqpoint{1.731600in}{2.208775in}}%
\pgfpathlineto{\pgfqpoint{1.734080in}{2.211902in}}%
\pgfpathlineto{\pgfqpoint{1.736560in}{2.206925in}}%
\pgfpathlineto{\pgfqpoint{1.737800in}{2.208405in}}%
\pgfpathlineto{\pgfqpoint{1.740280in}{2.215520in}}%
\pgfpathlineto{\pgfqpoint{1.741520in}{2.214048in}}%
\pgfpathlineto{\pgfqpoint{1.745240in}{2.199291in}}%
\pgfpathlineto{\pgfqpoint{1.747720in}{2.201051in}}%
\pgfpathlineto{\pgfqpoint{1.750200in}{2.197314in}}%
\pgfpathlineto{\pgfqpoint{1.752680in}{2.193893in}}%
\pgfpathlineto{\pgfqpoint{1.753920in}{2.192736in}}%
\pgfpathlineto{\pgfqpoint{1.755160in}{2.194598in}}%
\pgfpathlineto{\pgfqpoint{1.756400in}{2.193410in}}%
\pgfpathlineto{\pgfqpoint{1.758880in}{2.197248in}}%
\pgfpathlineto{\pgfqpoint{1.761360in}{2.196635in}}%
\pgfpathlineto{\pgfqpoint{1.763840in}{2.191044in}}%
\pgfpathlineto{\pgfqpoint{1.765080in}{2.190968in}}%
\pgfpathlineto{\pgfqpoint{1.766320in}{2.192570in}}%
\pgfpathlineto{\pgfqpoint{1.770040in}{2.187559in}}%
\pgfpathlineto{\pgfqpoint{1.775000in}{2.188741in}}%
\pgfpathlineto{\pgfqpoint{1.778720in}{2.194654in}}%
\pgfpathlineto{\pgfqpoint{1.779960in}{2.196289in}}%
\pgfpathlineto{\pgfqpoint{1.783680in}{2.192152in}}%
\pgfpathlineto{\pgfqpoint{1.786160in}{2.191804in}}%
\pgfpathlineto{\pgfqpoint{1.787400in}{2.194652in}}%
\pgfpathlineto{\pgfqpoint{1.788640in}{2.192812in}}%
\pgfpathlineto{\pgfqpoint{1.789880in}{2.193647in}}%
\pgfpathlineto{\pgfqpoint{1.792360in}{2.191869in}}%
\pgfpathlineto{\pgfqpoint{1.799800in}{2.200074in}}%
\pgfpathlineto{\pgfqpoint{1.801040in}{2.201240in}}%
\pgfpathlineto{\pgfqpoint{1.808480in}{2.193413in}}%
\pgfpathlineto{\pgfqpoint{1.812200in}{2.200549in}}%
\pgfpathlineto{\pgfqpoint{1.814680in}{2.205491in}}%
\pgfpathlineto{\pgfqpoint{1.817160in}{2.202205in}}%
\pgfpathlineto{\pgfqpoint{1.820880in}{2.201362in}}%
\pgfpathlineto{\pgfqpoint{1.822120in}{2.202198in}}%
\pgfpathlineto{\pgfqpoint{1.825840in}{2.197172in}}%
\pgfpathlineto{\pgfqpoint{1.827080in}{2.197154in}}%
\pgfpathlineto{\pgfqpoint{1.828320in}{2.198627in}}%
\pgfpathlineto{\pgfqpoint{1.829560in}{2.197089in}}%
\pgfpathlineto{\pgfqpoint{1.833280in}{2.200586in}}%
\pgfpathlineto{\pgfqpoint{1.834520in}{2.198284in}}%
\pgfpathlineto{\pgfqpoint{1.837000in}{2.199997in}}%
\pgfpathlineto{\pgfqpoint{1.838240in}{2.202185in}}%
\pgfpathlineto{\pgfqpoint{1.839480in}{2.200062in}}%
\pgfpathlineto{\pgfqpoint{1.841960in}{2.203288in}}%
\pgfpathlineto{\pgfqpoint{1.843200in}{2.203124in}}%
\pgfpathlineto{\pgfqpoint{1.846920in}{2.196230in}}%
\pgfpathlineto{\pgfqpoint{1.849400in}{2.195411in}}%
\pgfpathlineto{\pgfqpoint{1.851880in}{2.197798in}}%
\pgfpathlineto{\pgfqpoint{1.853120in}{2.197418in}}%
\pgfpathlineto{\pgfqpoint{1.854360in}{2.195715in}}%
\pgfpathlineto{\pgfqpoint{1.855600in}{2.196332in}}%
\pgfpathlineto{\pgfqpoint{1.858080in}{2.198848in}}%
\pgfpathlineto{\pgfqpoint{1.860560in}{2.192713in}}%
\pgfpathlineto{\pgfqpoint{1.861800in}{2.193359in}}%
\pgfpathlineto{\pgfqpoint{1.864280in}{2.198857in}}%
\pgfpathlineto{\pgfqpoint{1.865520in}{2.197887in}}%
\pgfpathlineto{\pgfqpoint{1.869240in}{2.184165in}}%
\pgfpathlineto{\pgfqpoint{1.872960in}{2.184480in}}%
\pgfpathlineto{\pgfqpoint{1.875440in}{2.180576in}}%
\pgfpathlineto{\pgfqpoint{1.877920in}{2.180502in}}%
\pgfpathlineto{\pgfqpoint{1.879160in}{2.182338in}}%
\pgfpathlineto{\pgfqpoint{1.880400in}{2.181714in}}%
\pgfpathlineto{\pgfqpoint{1.882880in}{2.185643in}}%
\pgfpathlineto{\pgfqpoint{1.885360in}{2.184306in}}%
\pgfpathlineto{\pgfqpoint{1.887840in}{2.178690in}}%
\pgfpathlineto{\pgfqpoint{1.889080in}{2.178324in}}%
\pgfpathlineto{\pgfqpoint{1.890320in}{2.180804in}}%
\pgfpathlineto{\pgfqpoint{1.894040in}{2.176278in}}%
\pgfpathlineto{\pgfqpoint{1.897760in}{2.178480in}}%
\pgfpathlineto{\pgfqpoint{1.899000in}{2.177660in}}%
\pgfpathlineto{\pgfqpoint{1.903960in}{2.183123in}}%
\pgfpathlineto{\pgfqpoint{1.908920in}{2.176614in}}%
\pgfpathlineto{\pgfqpoint{1.910160in}{2.176852in}}%
\pgfpathlineto{\pgfqpoint{1.911400in}{2.179658in}}%
\pgfpathlineto{\pgfqpoint{1.912640in}{2.178961in}}%
\pgfpathlineto{\pgfqpoint{1.913880in}{2.180560in}}%
\pgfpathlineto{\pgfqpoint{1.917600in}{2.180272in}}%
\pgfpathlineto{\pgfqpoint{1.918840in}{2.181189in}}%
\pgfpathlineto{\pgfqpoint{1.921320in}{2.185276in}}%
\pgfpathlineto{\pgfqpoint{1.926280in}{2.185917in}}%
\pgfpathlineto{\pgfqpoint{1.928760in}{2.183008in}}%
\pgfpathlineto{\pgfqpoint{1.931240in}{2.179093in}}%
\pgfpathlineto{\pgfqpoint{1.932480in}{2.179667in}}%
\pgfpathlineto{\pgfqpoint{1.936200in}{2.186050in}}%
\pgfpathlineto{\pgfqpoint{1.938680in}{2.190761in}}%
\pgfpathlineto{\pgfqpoint{1.941160in}{2.187127in}}%
\pgfpathlineto{\pgfqpoint{1.946120in}{2.185556in}}%
\pgfpathlineto{\pgfqpoint{1.948600in}{2.182899in}}%
\pgfpathlineto{\pgfqpoint{1.951080in}{2.181756in}}%
\pgfpathlineto{\pgfqpoint{1.952320in}{2.182828in}}%
\pgfpathlineto{\pgfqpoint{1.953560in}{2.181321in}}%
\pgfpathlineto{\pgfqpoint{1.957280in}{2.184924in}}%
\pgfpathlineto{\pgfqpoint{1.958520in}{2.182416in}}%
\pgfpathlineto{\pgfqpoint{1.961000in}{2.185060in}}%
\pgfpathlineto{\pgfqpoint{1.962240in}{2.187346in}}%
\pgfpathlineto{\pgfqpoint{1.963480in}{2.185022in}}%
\pgfpathlineto{\pgfqpoint{1.967200in}{2.188587in}}%
\pgfpathlineto{\pgfqpoint{1.969680in}{2.183460in}}%
\pgfpathlineto{\pgfqpoint{1.972160in}{2.181561in}}%
\pgfpathlineto{\pgfqpoint{1.973400in}{2.181011in}}%
\pgfpathlineto{\pgfqpoint{1.975880in}{2.182762in}}%
\pgfpathlineto{\pgfqpoint{1.979600in}{2.181893in}}%
\pgfpathlineto{\pgfqpoint{1.980840in}{2.184769in}}%
\pgfpathlineto{\pgfqpoint{1.982080in}{2.184626in}}%
\pgfpathlineto{\pgfqpoint{1.984560in}{2.180880in}}%
\pgfpathlineto{\pgfqpoint{1.985800in}{2.181577in}}%
\pgfpathlineto{\pgfqpoint{1.988280in}{2.187091in}}%
\pgfpathlineto{\pgfqpoint{1.989520in}{2.185153in}}%
\pgfpathlineto{\pgfqpoint{1.993240in}{2.173295in}}%
\pgfpathlineto{\pgfqpoint{1.995720in}{2.173726in}}%
\pgfpathlineto{\pgfqpoint{1.999440in}{2.171647in}}%
\pgfpathlineto{\pgfqpoint{2.001920in}{2.171527in}}%
\pgfpathlineto{\pgfqpoint{2.003160in}{2.173724in}}%
\pgfpathlineto{\pgfqpoint{2.004400in}{2.172970in}}%
\pgfpathlineto{\pgfqpoint{2.006880in}{2.176654in}}%
\pgfpathlineto{\pgfqpoint{2.009360in}{2.175007in}}%
\pgfpathlineto{\pgfqpoint{2.011840in}{2.169060in}}%
\pgfpathlineto{\pgfqpoint{2.013080in}{2.168902in}}%
\pgfpathlineto{\pgfqpoint{2.014320in}{2.171065in}}%
\pgfpathlineto{\pgfqpoint{2.018040in}{2.167417in}}%
\pgfpathlineto{\pgfqpoint{2.021760in}{2.169862in}}%
\pgfpathlineto{\pgfqpoint{2.023000in}{2.170082in}}%
\pgfpathlineto{\pgfqpoint{2.027960in}{2.175811in}}%
\pgfpathlineto{\pgfqpoint{2.032920in}{2.170591in}}%
\pgfpathlineto{\pgfqpoint{2.034160in}{2.170709in}}%
\pgfpathlineto{\pgfqpoint{2.035400in}{2.173869in}}%
\pgfpathlineto{\pgfqpoint{2.036640in}{2.173920in}}%
\pgfpathlineto{\pgfqpoint{2.039120in}{2.176348in}}%
\pgfpathlineto{\pgfqpoint{2.041600in}{2.177333in}}%
\pgfpathlineto{\pgfqpoint{2.049040in}{2.182945in}}%
\pgfpathlineto{\pgfqpoint{2.051520in}{2.181606in}}%
\pgfpathlineto{\pgfqpoint{2.055240in}{2.177476in}}%
\pgfpathlineto{\pgfqpoint{2.056480in}{2.178618in}}%
\pgfpathlineto{\pgfqpoint{2.058960in}{2.183418in}}%
\pgfpathlineto{\pgfqpoint{2.060200in}{2.184015in}}%
\pgfpathlineto{\pgfqpoint{2.062680in}{2.188641in}}%
\pgfpathlineto{\pgfqpoint{2.065160in}{2.185049in}}%
\pgfpathlineto{\pgfqpoint{2.067640in}{2.185158in}}%
\pgfpathlineto{\pgfqpoint{2.072600in}{2.180065in}}%
\pgfpathlineto{\pgfqpoint{2.073840in}{2.179532in}}%
\pgfpathlineto{\pgfqpoint{2.076320in}{2.181780in}}%
\pgfpathlineto{\pgfqpoint{2.077560in}{2.180412in}}%
\pgfpathlineto{\pgfqpoint{2.081280in}{2.182321in}}%
\pgfpathlineto{\pgfqpoint{2.083760in}{2.179931in}}%
\pgfpathlineto{\pgfqpoint{2.086240in}{2.183200in}}%
\pgfpathlineto{\pgfqpoint{2.087480in}{2.181109in}}%
\pgfpathlineto{\pgfqpoint{2.091200in}{2.185933in}}%
\pgfpathlineto{\pgfqpoint{2.093680in}{2.181911in}}%
\pgfpathlineto{\pgfqpoint{2.096160in}{2.180359in}}%
\pgfpathlineto{\pgfqpoint{2.097400in}{2.179896in}}%
\pgfpathlineto{\pgfqpoint{2.101120in}{2.181597in}}%
\pgfpathlineto{\pgfqpoint{2.102360in}{2.179408in}}%
\pgfpathlineto{\pgfqpoint{2.103600in}{2.179799in}}%
\pgfpathlineto{\pgfqpoint{2.104840in}{2.182297in}}%
\pgfpathlineto{\pgfqpoint{2.106080in}{2.181847in}}%
\pgfpathlineto{\pgfqpoint{2.108560in}{2.178735in}}%
\pgfpathlineto{\pgfqpoint{2.109800in}{2.178603in}}%
\pgfpathlineto{\pgfqpoint{2.112280in}{2.183660in}}%
\pgfpathlineto{\pgfqpoint{2.113520in}{2.181941in}}%
\pgfpathlineto{\pgfqpoint{2.116000in}{2.173849in}}%
\pgfpathlineto{\pgfqpoint{2.119720in}{2.173543in}}%
\pgfpathlineto{\pgfqpoint{2.124680in}{2.171789in}}%
\pgfpathlineto{\pgfqpoint{2.125920in}{2.172067in}}%
\pgfpathlineto{\pgfqpoint{2.127160in}{2.174523in}}%
\pgfpathlineto{\pgfqpoint{2.128400in}{2.174205in}}%
\pgfpathlineto{\pgfqpoint{2.132120in}{2.177764in}}%
\pgfpathlineto{\pgfqpoint{2.133360in}{2.175650in}}%
\pgfpathlineto{\pgfqpoint{2.134600in}{2.170912in}}%
\pgfpathlineto{\pgfqpoint{2.137080in}{2.170752in}}%
\pgfpathlineto{\pgfqpoint{2.138320in}{2.172192in}}%
\pgfpathlineto{\pgfqpoint{2.142040in}{2.168542in}}%
\pgfpathlineto{\pgfqpoint{2.145760in}{2.171347in}}%
\pgfpathlineto{\pgfqpoint{2.149480in}{2.174066in}}%
\pgfpathlineto{\pgfqpoint{2.151960in}{2.177730in}}%
\pgfpathlineto{\pgfqpoint{2.156920in}{2.171662in}}%
\pgfpathlineto{\pgfqpoint{2.158160in}{2.172065in}}%
\pgfpathlineto{\pgfqpoint{2.160640in}{2.174720in}}%
\pgfpathlineto{\pgfqpoint{2.161880in}{2.176945in}}%
\pgfpathlineto{\pgfqpoint{2.164360in}{2.176676in}}%
\pgfpathlineto{\pgfqpoint{2.166840in}{2.178855in}}%
\pgfpathlineto{\pgfqpoint{2.169320in}{2.181729in}}%
\pgfpathlineto{\pgfqpoint{2.171800in}{2.181726in}}%
\pgfpathlineto{\pgfqpoint{2.175520in}{2.181924in}}%
\pgfpathlineto{\pgfqpoint{2.179240in}{2.176526in}}%
\pgfpathlineto{\pgfqpoint{2.186680in}{2.186997in}}%
\pgfpathlineto{\pgfqpoint{2.189160in}{2.184918in}}%
\pgfpathlineto{\pgfqpoint{2.191640in}{2.184541in}}%
\pgfpathlineto{\pgfqpoint{2.195360in}{2.179144in}}%
\pgfpathlineto{\pgfqpoint{2.199080in}{2.178574in}}%
\pgfpathlineto{\pgfqpoint{2.200320in}{2.179953in}}%
\pgfpathlineto{\pgfqpoint{2.201560in}{2.178385in}}%
\pgfpathlineto{\pgfqpoint{2.205280in}{2.179175in}}%
\pgfpathlineto{\pgfqpoint{2.207760in}{2.176729in}}%
\pgfpathlineto{\pgfqpoint{2.210240in}{2.180222in}}%
\pgfpathlineto{\pgfqpoint{2.211480in}{2.178391in}}%
\pgfpathlineto{\pgfqpoint{2.215200in}{2.183418in}}%
\pgfpathlineto{\pgfqpoint{2.218920in}{2.178255in}}%
\pgfpathlineto{\pgfqpoint{2.221400in}{2.177501in}}%
\pgfpathlineto{\pgfqpoint{2.225120in}{2.180202in}}%
\pgfpathlineto{\pgfqpoint{2.226360in}{2.177890in}}%
\pgfpathlineto{\pgfqpoint{2.227600in}{2.178663in}}%
\pgfpathlineto{\pgfqpoint{2.228840in}{2.180794in}}%
\pgfpathlineto{\pgfqpoint{2.230080in}{2.180506in}}%
\pgfpathlineto{\pgfqpoint{2.232560in}{2.177449in}}%
\pgfpathlineto{\pgfqpoint{2.233800in}{2.177155in}}%
\pgfpathlineto{\pgfqpoint{2.236280in}{2.183059in}}%
\pgfpathlineto{\pgfqpoint{2.237520in}{2.181326in}}%
\pgfpathlineto{\pgfqpoint{2.240000in}{2.172883in}}%
\pgfpathlineto{\pgfqpoint{2.241240in}{2.171725in}}%
\pgfpathlineto{\pgfqpoint{2.243720in}{2.171659in}}%
\pgfpathlineto{\pgfqpoint{2.247440in}{2.170567in}}%
\pgfpathlineto{\pgfqpoint{2.256120in}{2.177785in}}%
\pgfpathlineto{\pgfqpoint{2.257360in}{2.175768in}}%
\pgfpathlineto{\pgfqpoint{2.258600in}{2.170802in}}%
\pgfpathlineto{\pgfqpoint{2.262320in}{2.172190in}}%
\pgfpathlineto{\pgfqpoint{2.266040in}{2.168461in}}%
\pgfpathlineto{\pgfqpoint{2.274720in}{2.177011in}}%
\pgfpathlineto{\pgfqpoint{2.275960in}{2.178612in}}%
\pgfpathlineto{\pgfqpoint{2.282160in}{2.173570in}}%
\pgfpathlineto{\pgfqpoint{2.285880in}{2.180515in}}%
\pgfpathlineto{\pgfqpoint{2.288360in}{2.179607in}}%
\pgfpathlineto{\pgfqpoint{2.290840in}{2.181999in}}%
\pgfpathlineto{\pgfqpoint{2.293320in}{2.185426in}}%
\pgfpathlineto{\pgfqpoint{2.294560in}{2.185526in}}%
\pgfpathlineto{\pgfqpoint{2.297040in}{2.187350in}}%
\pgfpathlineto{\pgfqpoint{2.298280in}{2.187695in}}%
\pgfpathlineto{\pgfqpoint{2.302000in}{2.182662in}}%
\pgfpathlineto{\pgfqpoint{2.303240in}{2.180237in}}%
\pgfpathlineto{\pgfqpoint{2.305720in}{2.183684in}}%
\pgfpathlineto{\pgfqpoint{2.309440in}{2.188207in}}%
\pgfpathlineto{\pgfqpoint{2.310680in}{2.189892in}}%
\pgfpathlineto{\pgfqpoint{2.313160in}{2.188249in}}%
\pgfpathlineto{\pgfqpoint{2.315640in}{2.189182in}}%
\pgfpathlineto{\pgfqpoint{2.320600in}{2.183219in}}%
\pgfpathlineto{\pgfqpoint{2.321840in}{2.182355in}}%
\pgfpathlineto{\pgfqpoint{2.324320in}{2.185449in}}%
\pgfpathlineto{\pgfqpoint{2.325560in}{2.184052in}}%
\pgfpathlineto{\pgfqpoint{2.328040in}{2.185255in}}%
\pgfpathlineto{\pgfqpoint{2.329280in}{2.184666in}}%
\pgfpathlineto{\pgfqpoint{2.330520in}{2.182096in}}%
\pgfpathlineto{\pgfqpoint{2.331760in}{2.182688in}}%
\pgfpathlineto{\pgfqpoint{2.334240in}{2.187049in}}%
\pgfpathlineto{\pgfqpoint{2.335480in}{2.185912in}}%
\pgfpathlineto{\pgfqpoint{2.339200in}{2.190520in}}%
\pgfpathlineto{\pgfqpoint{2.342920in}{2.185498in}}%
\pgfpathlineto{\pgfqpoint{2.346640in}{2.185689in}}%
\pgfpathlineto{\pgfqpoint{2.347880in}{2.187301in}}%
\pgfpathlineto{\pgfqpoint{2.349120in}{2.186480in}}%
\pgfpathlineto{\pgfqpoint{2.350360in}{2.183775in}}%
\pgfpathlineto{\pgfqpoint{2.354080in}{2.185290in}}%
\pgfpathlineto{\pgfqpoint{2.356560in}{2.182882in}}%
\pgfpathlineto{\pgfqpoint{2.357800in}{2.182658in}}%
\pgfpathlineto{\pgfqpoint{2.360280in}{2.188080in}}%
\pgfpathlineto{\pgfqpoint{2.361520in}{2.186037in}}%
\pgfpathlineto{\pgfqpoint{2.365240in}{2.174856in}}%
\pgfpathlineto{\pgfqpoint{2.367720in}{2.174480in}}%
\pgfpathlineto{\pgfqpoint{2.371440in}{2.174398in}}%
\pgfpathlineto{\pgfqpoint{2.373920in}{2.175999in}}%
\pgfpathlineto{\pgfqpoint{2.375160in}{2.179459in}}%
\pgfpathlineto{\pgfqpoint{2.376400in}{2.179530in}}%
\pgfpathlineto{\pgfqpoint{2.380120in}{2.182639in}}%
\pgfpathlineto{\pgfqpoint{2.382600in}{2.175771in}}%
\pgfpathlineto{\pgfqpoint{2.383840in}{2.176858in}}%
\pgfpathlineto{\pgfqpoint{2.386320in}{2.178153in}}%
\pgfpathlineto{\pgfqpoint{2.390040in}{2.173196in}}%
\pgfpathlineto{\pgfqpoint{2.397480in}{2.179795in}}%
\pgfpathlineto{\pgfqpoint{2.399960in}{2.183774in}}%
\pgfpathlineto{\pgfqpoint{2.406160in}{2.178075in}}%
\pgfpathlineto{\pgfqpoint{2.409880in}{2.184023in}}%
\pgfpathlineto{\pgfqpoint{2.412360in}{2.182651in}}%
\pgfpathlineto{\pgfqpoint{2.414840in}{2.184601in}}%
\pgfpathlineto{\pgfqpoint{2.416080in}{2.187799in}}%
\pgfpathlineto{\pgfqpoint{2.421040in}{2.187990in}}%
\pgfpathlineto{\pgfqpoint{2.422280in}{2.188024in}}%
\pgfpathlineto{\pgfqpoint{2.428480in}{2.182810in}}%
\pgfpathlineto{\pgfqpoint{2.434680in}{2.192511in}}%
\pgfpathlineto{\pgfqpoint{2.437160in}{2.190700in}}%
\pgfpathlineto{\pgfqpoint{2.439640in}{2.191513in}}%
\pgfpathlineto{\pgfqpoint{2.443360in}{2.186215in}}%
\pgfpathlineto{\pgfqpoint{2.445840in}{2.184880in}}%
\pgfpathlineto{\pgfqpoint{2.448320in}{2.186550in}}%
\pgfpathlineto{\pgfqpoint{2.449560in}{2.185130in}}%
\pgfpathlineto{\pgfqpoint{2.452040in}{2.186007in}}%
\pgfpathlineto{\pgfqpoint{2.453280in}{2.185432in}}%
\pgfpathlineto{\pgfqpoint{2.454520in}{2.183169in}}%
\pgfpathlineto{\pgfqpoint{2.455760in}{2.184151in}}%
\pgfpathlineto{\pgfqpoint{2.458240in}{2.188333in}}%
\pgfpathlineto{\pgfqpoint{2.459480in}{2.187406in}}%
\pgfpathlineto{\pgfqpoint{2.463200in}{2.192428in}}%
\pgfpathlineto{\pgfqpoint{2.465680in}{2.190460in}}%
\pgfpathlineto{\pgfqpoint{2.468160in}{2.187763in}}%
\pgfpathlineto{\pgfqpoint{2.470640in}{2.187191in}}%
\pgfpathlineto{\pgfqpoint{2.471880in}{2.188683in}}%
\pgfpathlineto{\pgfqpoint{2.473120in}{2.187851in}}%
\pgfpathlineto{\pgfqpoint{2.474360in}{2.185468in}}%
\pgfpathlineto{\pgfqpoint{2.478080in}{2.187174in}}%
\pgfpathlineto{\pgfqpoint{2.480560in}{2.184374in}}%
\pgfpathlineto{\pgfqpoint{2.481800in}{2.183707in}}%
\pgfpathlineto{\pgfqpoint{2.484280in}{2.188813in}}%
\pgfpathlineto{\pgfqpoint{2.485520in}{2.186906in}}%
\pgfpathlineto{\pgfqpoint{2.488000in}{2.178841in}}%
\pgfpathlineto{\pgfqpoint{2.489240in}{2.177232in}}%
\pgfpathlineto{\pgfqpoint{2.490480in}{2.177756in}}%
\pgfpathlineto{\pgfqpoint{2.491720in}{2.176720in}}%
\pgfpathlineto{\pgfqpoint{2.497920in}{2.179452in}}%
\pgfpathlineto{\pgfqpoint{2.500400in}{2.183344in}}%
\pgfpathlineto{\pgfqpoint{2.504120in}{2.187505in}}%
\pgfpathlineto{\pgfqpoint{2.506600in}{2.181621in}}%
\pgfpathlineto{\pgfqpoint{2.507840in}{2.182468in}}%
\pgfpathlineto{\pgfqpoint{2.510320in}{2.183401in}}%
\pgfpathlineto{\pgfqpoint{2.514040in}{2.178616in}}%
\pgfpathlineto{\pgfqpoint{2.525200in}{2.187646in}}%
\pgfpathlineto{\pgfqpoint{2.527680in}{2.185025in}}%
\pgfpathlineto{\pgfqpoint{2.530160in}{2.183766in}}%
\pgfpathlineto{\pgfqpoint{2.533880in}{2.189269in}}%
\pgfpathlineto{\pgfqpoint{2.536360in}{2.187757in}}%
\pgfpathlineto{\pgfqpoint{2.538840in}{2.189525in}}%
\pgfpathlineto{\pgfqpoint{2.540080in}{2.193154in}}%
\pgfpathlineto{\pgfqpoint{2.547520in}{2.191480in}}%
\pgfpathlineto{\pgfqpoint{2.551240in}{2.186874in}}%
\pgfpathlineto{\pgfqpoint{2.552480in}{2.187527in}}%
\pgfpathlineto{\pgfqpoint{2.557440in}{2.193259in}}%
\pgfpathlineto{\pgfqpoint{2.558680in}{2.195150in}}%
\pgfpathlineto{\pgfqpoint{2.561160in}{2.193125in}}%
\pgfpathlineto{\pgfqpoint{2.563640in}{2.193895in}}%
\pgfpathlineto{\pgfqpoint{2.567360in}{2.188402in}}%
\pgfpathlineto{\pgfqpoint{2.569840in}{2.186891in}}%
\pgfpathlineto{\pgfqpoint{2.572320in}{2.188304in}}%
\pgfpathlineto{\pgfqpoint{2.573560in}{2.186307in}}%
\pgfpathlineto{\pgfqpoint{2.576040in}{2.187571in}}%
\pgfpathlineto{\pgfqpoint{2.579760in}{2.186117in}}%
\pgfpathlineto{\pgfqpoint{2.582240in}{2.191115in}}%
\pgfpathlineto{\pgfqpoint{2.583480in}{2.189884in}}%
\pgfpathlineto{\pgfqpoint{2.584720in}{2.190992in}}%
\pgfpathlineto{\pgfqpoint{2.587200in}{2.195780in}}%
\pgfpathlineto{\pgfqpoint{2.589680in}{2.194187in}}%
\pgfpathlineto{\pgfqpoint{2.592160in}{2.190967in}}%
\pgfpathlineto{\pgfqpoint{2.594640in}{2.190088in}}%
\pgfpathlineto{\pgfqpoint{2.595880in}{2.191365in}}%
\pgfpathlineto{\pgfqpoint{2.598360in}{2.187881in}}%
\pgfpathlineto{\pgfqpoint{2.602080in}{2.189922in}}%
\pgfpathlineto{\pgfqpoint{2.605800in}{2.185628in}}%
\pgfpathlineto{\pgfqpoint{2.608280in}{2.190633in}}%
\pgfpathlineto{\pgfqpoint{2.609520in}{2.188320in}}%
\pgfpathlineto{\pgfqpoint{2.612000in}{2.180338in}}%
\pgfpathlineto{\pgfqpoint{2.614480in}{2.182616in}}%
\pgfpathlineto{\pgfqpoint{2.615720in}{2.181417in}}%
\pgfpathlineto{\pgfqpoint{2.621920in}{2.184882in}}%
\pgfpathlineto{\pgfqpoint{2.624400in}{2.188857in}}%
\pgfpathlineto{\pgfqpoint{2.628120in}{2.193470in}}%
\pgfpathlineto{\pgfqpoint{2.629360in}{2.191948in}}%
\pgfpathlineto{\pgfqpoint{2.630600in}{2.188142in}}%
\pgfpathlineto{\pgfqpoint{2.634320in}{2.189883in}}%
\pgfpathlineto{\pgfqpoint{2.638040in}{2.185964in}}%
\pgfpathlineto{\pgfqpoint{2.640520in}{2.187719in}}%
\pgfpathlineto{\pgfqpoint{2.643000in}{2.190432in}}%
\pgfpathlineto{\pgfqpoint{2.645480in}{2.190706in}}%
\pgfpathlineto{\pgfqpoint{2.647960in}{2.196129in}}%
\pgfpathlineto{\pgfqpoint{2.649200in}{2.195411in}}%
\pgfpathlineto{\pgfqpoint{2.651680in}{2.192306in}}%
\pgfpathlineto{\pgfqpoint{2.654160in}{2.190305in}}%
\pgfpathlineto{\pgfqpoint{2.657880in}{2.194786in}}%
\pgfpathlineto{\pgfqpoint{2.660360in}{2.193393in}}%
\pgfpathlineto{\pgfqpoint{2.665320in}{2.198242in}}%
\pgfpathlineto{\pgfqpoint{2.667800in}{2.198825in}}%
\pgfpathlineto{\pgfqpoint{2.670280in}{2.198682in}}%
\pgfpathlineto{\pgfqpoint{2.674000in}{2.193867in}}%
\pgfpathlineto{\pgfqpoint{2.675240in}{2.191251in}}%
\pgfpathlineto{\pgfqpoint{2.676480in}{2.191753in}}%
\pgfpathlineto{\pgfqpoint{2.682680in}{2.199204in}}%
\pgfpathlineto{\pgfqpoint{2.685160in}{2.196140in}}%
\pgfpathlineto{\pgfqpoint{2.687640in}{2.196764in}}%
\pgfpathlineto{\pgfqpoint{2.690120in}{2.194914in}}%
\pgfpathlineto{\pgfqpoint{2.693840in}{2.191447in}}%
\pgfpathlineto{\pgfqpoint{2.701280in}{2.191565in}}%
\pgfpathlineto{\pgfqpoint{2.703760in}{2.190220in}}%
\pgfpathlineto{\pgfqpoint{2.706240in}{2.194991in}}%
\pgfpathlineto{\pgfqpoint{2.707480in}{2.194359in}}%
\pgfpathlineto{\pgfqpoint{2.713680in}{2.199730in}}%
\pgfpathlineto{\pgfqpoint{2.716160in}{2.196548in}}%
\pgfpathlineto{\pgfqpoint{2.718640in}{2.195051in}}%
\pgfpathlineto{\pgfqpoint{2.719880in}{2.195889in}}%
\pgfpathlineto{\pgfqpoint{2.722360in}{2.193201in}}%
\pgfpathlineto{\pgfqpoint{2.724840in}{2.196932in}}%
\pgfpathlineto{\pgfqpoint{2.726080in}{2.196179in}}%
\pgfpathlineto{\pgfqpoint{2.729800in}{2.190566in}}%
\pgfpathlineto{\pgfqpoint{2.732280in}{2.194597in}}%
\pgfpathlineto{\pgfqpoint{2.733520in}{2.192588in}}%
\pgfpathlineto{\pgfqpoint{2.736000in}{2.185667in}}%
\pgfpathlineto{\pgfqpoint{2.738480in}{2.187161in}}%
\pgfpathlineto{\pgfqpoint{2.740960in}{2.185652in}}%
\pgfpathlineto{\pgfqpoint{2.743440in}{2.185902in}}%
\pgfpathlineto{\pgfqpoint{2.745920in}{2.188443in}}%
\pgfpathlineto{\pgfqpoint{2.748400in}{2.193374in}}%
\pgfpathlineto{\pgfqpoint{2.752120in}{2.197320in}}%
\pgfpathlineto{\pgfqpoint{2.753360in}{2.196161in}}%
\pgfpathlineto{\pgfqpoint{2.754600in}{2.192430in}}%
\pgfpathlineto{\pgfqpoint{2.758320in}{2.193622in}}%
\pgfpathlineto{\pgfqpoint{2.762040in}{2.189847in}}%
\pgfpathlineto{\pgfqpoint{2.763280in}{2.189457in}}%
\pgfpathlineto{\pgfqpoint{2.773200in}{2.197935in}}%
\pgfpathlineto{\pgfqpoint{2.778160in}{2.193185in}}%
\pgfpathlineto{\pgfqpoint{2.781880in}{2.199837in}}%
\pgfpathlineto{\pgfqpoint{2.784360in}{2.197302in}}%
\pgfpathlineto{\pgfqpoint{2.786840in}{2.199028in}}%
\pgfpathlineto{\pgfqpoint{2.789320in}{2.202135in}}%
\pgfpathlineto{\pgfqpoint{2.794280in}{2.202613in}}%
\pgfpathlineto{\pgfqpoint{2.800480in}{2.194983in}}%
\pgfpathlineto{\pgfqpoint{2.802960in}{2.199185in}}%
\pgfpathlineto{\pgfqpoint{2.805440in}{2.200620in}}%
\pgfpathlineto{\pgfqpoint{2.806680in}{2.202097in}}%
\pgfpathlineto{\pgfqpoint{2.809160in}{2.199558in}}%
\pgfpathlineto{\pgfqpoint{2.811640in}{2.200440in}}%
\pgfpathlineto{\pgfqpoint{2.815360in}{2.196613in}}%
\pgfpathlineto{\pgfqpoint{2.819080in}{2.194731in}}%
\pgfpathlineto{\pgfqpoint{2.820320in}{2.195947in}}%
\pgfpathlineto{\pgfqpoint{2.821560in}{2.194832in}}%
\pgfpathlineto{\pgfqpoint{2.824040in}{2.196589in}}%
\pgfpathlineto{\pgfqpoint{2.827760in}{2.194343in}}%
\pgfpathlineto{\pgfqpoint{2.830240in}{2.198132in}}%
\pgfpathlineto{\pgfqpoint{2.831480in}{2.197211in}}%
\pgfpathlineto{\pgfqpoint{2.836440in}{2.201773in}}%
\pgfpathlineto{\pgfqpoint{2.837680in}{2.201776in}}%
\pgfpathlineto{\pgfqpoint{2.840160in}{2.198630in}}%
\pgfpathlineto{\pgfqpoint{2.842640in}{2.197650in}}%
\pgfpathlineto{\pgfqpoint{2.843880in}{2.198402in}}%
\pgfpathlineto{\pgfqpoint{2.846360in}{2.194672in}}%
\pgfpathlineto{\pgfqpoint{2.850080in}{2.197942in}}%
\pgfpathlineto{\pgfqpoint{2.853800in}{2.193311in}}%
\pgfpathlineto{\pgfqpoint{2.856280in}{2.196896in}}%
\pgfpathlineto{\pgfqpoint{2.857520in}{2.194915in}}%
\pgfpathlineto{\pgfqpoint{2.860000in}{2.188383in}}%
\pgfpathlineto{\pgfqpoint{2.862480in}{2.190817in}}%
\pgfpathlineto{\pgfqpoint{2.863720in}{2.189192in}}%
\pgfpathlineto{\pgfqpoint{2.869920in}{2.193045in}}%
\pgfpathlineto{\pgfqpoint{2.872400in}{2.197629in}}%
\pgfpathlineto{\pgfqpoint{2.876120in}{2.201635in}}%
\pgfpathlineto{\pgfqpoint{2.877360in}{2.200762in}}%
\pgfpathlineto{\pgfqpoint{2.878600in}{2.197754in}}%
\pgfpathlineto{\pgfqpoint{2.882320in}{2.200250in}}%
\pgfpathlineto{\pgfqpoint{2.887280in}{2.196130in}}%
\pgfpathlineto{\pgfqpoint{2.892240in}{2.200866in}}%
\pgfpathlineto{\pgfqpoint{2.893480in}{2.200965in}}%
\pgfpathlineto{\pgfqpoint{2.895960in}{2.204641in}}%
\pgfpathlineto{\pgfqpoint{2.898440in}{2.202977in}}%
\pgfpathlineto{\pgfqpoint{2.902160in}{2.200408in}}%
\pgfpathlineto{\pgfqpoint{2.905880in}{2.207821in}}%
\pgfpathlineto{\pgfqpoint{2.908360in}{2.204854in}}%
\pgfpathlineto{\pgfqpoint{2.910840in}{2.206579in}}%
\pgfpathlineto{\pgfqpoint{2.913320in}{2.210551in}}%
\pgfpathlineto{\pgfqpoint{2.919520in}{2.210088in}}%
\pgfpathlineto{\pgfqpoint{2.924480in}{2.203684in}}%
\pgfpathlineto{\pgfqpoint{2.926960in}{2.207938in}}%
\pgfpathlineto{\pgfqpoint{2.929440in}{2.208182in}}%
\pgfpathlineto{\pgfqpoint{2.930680in}{2.209114in}}%
\pgfpathlineto{\pgfqpoint{2.933160in}{2.206310in}}%
\pgfpathlineto{\pgfqpoint{2.935640in}{2.207094in}}%
\pgfpathlineto{\pgfqpoint{2.939360in}{2.202132in}}%
\pgfpathlineto{\pgfqpoint{2.941840in}{2.200389in}}%
\pgfpathlineto{\pgfqpoint{2.948040in}{2.203689in}}%
\pgfpathlineto{\pgfqpoint{2.951760in}{2.201796in}}%
\pgfpathlineto{\pgfqpoint{2.954240in}{2.205862in}}%
\pgfpathlineto{\pgfqpoint{2.955480in}{2.205126in}}%
\pgfpathlineto{\pgfqpoint{2.960440in}{2.209314in}}%
\pgfpathlineto{\pgfqpoint{2.961680in}{2.208610in}}%
\pgfpathlineto{\pgfqpoint{2.962920in}{2.205222in}}%
\pgfpathlineto{\pgfqpoint{2.964160in}{2.205432in}}%
\pgfpathlineto{\pgfqpoint{2.966640in}{2.204014in}}%
\pgfpathlineto{\pgfqpoint{2.967880in}{2.204088in}}%
\pgfpathlineto{\pgfqpoint{2.970360in}{2.200220in}}%
\pgfpathlineto{\pgfqpoint{2.974080in}{2.203528in}}%
\pgfpathlineto{\pgfqpoint{2.976560in}{2.199706in}}%
\pgfpathlineto{\pgfqpoint{2.977800in}{2.199541in}}%
\pgfpathlineto{\pgfqpoint{2.980280in}{2.203692in}}%
\pgfpathlineto{\pgfqpoint{2.981520in}{2.201961in}}%
\pgfpathlineto{\pgfqpoint{2.984000in}{2.195654in}}%
\pgfpathlineto{\pgfqpoint{2.986480in}{2.198476in}}%
\pgfpathlineto{\pgfqpoint{2.988960in}{2.197031in}}%
\pgfpathlineto{\pgfqpoint{2.993920in}{2.199995in}}%
\pgfpathlineto{\pgfqpoint{2.996400in}{2.204808in}}%
\pgfpathlineto{\pgfqpoint{3.000120in}{2.209453in}}%
\pgfpathlineto{\pgfqpoint{3.001360in}{2.208198in}}%
\pgfpathlineto{\pgfqpoint{3.002600in}{2.205243in}}%
\pgfpathlineto{\pgfqpoint{3.006320in}{2.208033in}}%
\pgfpathlineto{\pgfqpoint{3.007560in}{2.206969in}}%
\pgfpathlineto{\pgfqpoint{3.010040in}{2.203264in}}%
\pgfpathlineto{\pgfqpoint{3.011280in}{2.202684in}}%
\pgfpathlineto{\pgfqpoint{3.016240in}{2.206807in}}%
\pgfpathlineto{\pgfqpoint{3.017480in}{2.206653in}}%
\pgfpathlineto{\pgfqpoint{3.019960in}{2.210250in}}%
\pgfpathlineto{\pgfqpoint{3.022440in}{2.209177in}}%
\pgfpathlineto{\pgfqpoint{3.023680in}{2.208654in}}%
\pgfpathlineto{\pgfqpoint{3.026160in}{2.206563in}}%
\pgfpathlineto{\pgfqpoint{3.029880in}{2.213867in}}%
\pgfpathlineto{\pgfqpoint{3.032360in}{2.210801in}}%
\pgfpathlineto{\pgfqpoint{3.034840in}{2.212692in}}%
\pgfpathlineto{\pgfqpoint{3.037320in}{2.216818in}}%
\pgfpathlineto{\pgfqpoint{3.041040in}{2.217498in}}%
\pgfpathlineto{\pgfqpoint{3.043520in}{2.215364in}}%
\pgfpathlineto{\pgfqpoint{3.047240in}{2.208118in}}%
\pgfpathlineto{\pgfqpoint{3.048480in}{2.208242in}}%
\pgfpathlineto{\pgfqpoint{3.050960in}{2.211513in}}%
\pgfpathlineto{\pgfqpoint{3.053440in}{2.211481in}}%
\pgfpathlineto{\pgfqpoint{3.054680in}{2.212761in}}%
\pgfpathlineto{\pgfqpoint{3.057160in}{2.210097in}}%
\pgfpathlineto{\pgfqpoint{3.059640in}{2.211516in}}%
\pgfpathlineto{\pgfqpoint{3.064600in}{2.205649in}}%
\pgfpathlineto{\pgfqpoint{3.065840in}{2.205064in}}%
\pgfpathlineto{\pgfqpoint{3.069560in}{2.208247in}}%
\pgfpathlineto{\pgfqpoint{3.072040in}{2.209517in}}%
\pgfpathlineto{\pgfqpoint{3.075760in}{2.207380in}}%
\pgfpathlineto{\pgfqpoint{3.078240in}{2.210454in}}%
\pgfpathlineto{\pgfqpoint{3.079480in}{2.209722in}}%
\pgfpathlineto{\pgfqpoint{3.084440in}{2.215451in}}%
\pgfpathlineto{\pgfqpoint{3.085680in}{2.214657in}}%
\pgfpathlineto{\pgfqpoint{3.086920in}{2.211503in}}%
\pgfpathlineto{\pgfqpoint{3.088160in}{2.211531in}}%
\pgfpathlineto{\pgfqpoint{3.090640in}{2.209575in}}%
\pgfpathlineto{\pgfqpoint{3.093120in}{2.208207in}}%
\pgfpathlineto{\pgfqpoint{3.094360in}{2.205719in}}%
\pgfpathlineto{\pgfqpoint{3.098080in}{2.209293in}}%
\pgfpathlineto{\pgfqpoint{3.101800in}{2.205654in}}%
\pgfpathlineto{\pgfqpoint{3.104280in}{2.210024in}}%
\pgfpathlineto{\pgfqpoint{3.108000in}{2.200190in}}%
\pgfpathlineto{\pgfqpoint{3.109240in}{2.201760in}}%
\pgfpathlineto{\pgfqpoint{3.110480in}{2.202427in}}%
\pgfpathlineto{\pgfqpoint{3.111720in}{2.201212in}}%
\pgfpathlineto{\pgfqpoint{3.117920in}{2.204736in}}%
\pgfpathlineto{\pgfqpoint{3.120400in}{2.209080in}}%
\pgfpathlineto{\pgfqpoint{3.124120in}{2.212780in}}%
\pgfpathlineto{\pgfqpoint{3.127840in}{2.210001in}}%
\pgfpathlineto{\pgfqpoint{3.129080in}{2.210252in}}%
\pgfpathlineto{\pgfqpoint{3.130320in}{2.211801in}}%
\pgfpathlineto{\pgfqpoint{3.131560in}{2.211098in}}%
\pgfpathlineto{\pgfqpoint{3.134040in}{2.208225in}}%
\pgfpathlineto{\pgfqpoint{3.135280in}{2.207320in}}%
\pgfpathlineto{\pgfqpoint{3.140240in}{2.210472in}}%
\pgfpathlineto{\pgfqpoint{3.141480in}{2.210464in}}%
\pgfpathlineto{\pgfqpoint{3.143960in}{2.213848in}}%
\pgfpathlineto{\pgfqpoint{3.147680in}{2.212918in}}%
\pgfpathlineto{\pgfqpoint{3.150160in}{2.210614in}}%
\pgfpathlineto{\pgfqpoint{3.153880in}{2.217646in}}%
\pgfpathlineto{\pgfqpoint{3.156360in}{2.214490in}}%
\pgfpathlineto{\pgfqpoint{3.158840in}{2.216125in}}%
\pgfpathlineto{\pgfqpoint{3.160080in}{2.219506in}}%
\pgfpathlineto{\pgfqpoint{3.166280in}{2.218453in}}%
\pgfpathlineto{\pgfqpoint{3.168760in}{2.214141in}}%
\pgfpathlineto{\pgfqpoint{3.172480in}{2.210286in}}%
\pgfpathlineto{\pgfqpoint{3.174960in}{2.213858in}}%
\pgfpathlineto{\pgfqpoint{3.177440in}{2.213974in}}%
\pgfpathlineto{\pgfqpoint{3.178680in}{2.215563in}}%
\pgfpathlineto{\pgfqpoint{3.181160in}{2.213241in}}%
\pgfpathlineto{\pgfqpoint{3.183640in}{2.214158in}}%
\pgfpathlineto{\pgfqpoint{3.188600in}{2.208088in}}%
\pgfpathlineto{\pgfqpoint{3.189840in}{2.207477in}}%
\pgfpathlineto{\pgfqpoint{3.192320in}{2.211429in}}%
\pgfpathlineto{\pgfqpoint{3.193560in}{2.210265in}}%
\pgfpathlineto{\pgfqpoint{3.196040in}{2.210752in}}%
\pgfpathlineto{\pgfqpoint{3.199760in}{2.208126in}}%
\pgfpathlineto{\pgfqpoint{3.202240in}{2.210898in}}%
\pgfpathlineto{\pgfqpoint{3.203480in}{2.210271in}}%
\pgfpathlineto{\pgfqpoint{3.207200in}{2.215501in}}%
\pgfpathlineto{\pgfqpoint{3.209680in}{2.215433in}}%
\pgfpathlineto{\pgfqpoint{3.210920in}{2.212665in}}%
\pgfpathlineto{\pgfqpoint{3.212160in}{2.212911in}}%
\pgfpathlineto{\pgfqpoint{3.214640in}{2.210791in}}%
\pgfpathlineto{\pgfqpoint{3.217120in}{2.209741in}}%
\pgfpathlineto{\pgfqpoint{3.218360in}{2.207473in}}%
\pgfpathlineto{\pgfqpoint{3.220840in}{2.209904in}}%
\pgfpathlineto{\pgfqpoint{3.222080in}{2.209659in}}%
\pgfpathlineto{\pgfqpoint{3.225800in}{2.205372in}}%
\pgfpathlineto{\pgfqpoint{3.228280in}{2.209811in}}%
\pgfpathlineto{\pgfqpoint{3.232000in}{2.200207in}}%
\pgfpathlineto{\pgfqpoint{3.233240in}{2.201223in}}%
\pgfpathlineto{\pgfqpoint{3.234480in}{2.201796in}}%
\pgfpathlineto{\pgfqpoint{3.235720in}{2.200838in}}%
\pgfpathlineto{\pgfqpoint{3.238200in}{2.201987in}}%
\pgfpathlineto{\pgfqpoint{3.239440in}{2.201031in}}%
\pgfpathlineto{\pgfqpoint{3.243160in}{2.207793in}}%
\pgfpathlineto{\pgfqpoint{3.244400in}{2.207935in}}%
\pgfpathlineto{\pgfqpoint{3.248120in}{2.212977in}}%
\pgfpathlineto{\pgfqpoint{3.249360in}{2.211824in}}%
\pgfpathlineto{\pgfqpoint{3.250600in}{2.209016in}}%
\pgfpathlineto{\pgfqpoint{3.254320in}{2.211607in}}%
\pgfpathlineto{\pgfqpoint{3.259280in}{2.206154in}}%
\pgfpathlineto{\pgfqpoint{3.265480in}{2.208817in}}%
\pgfpathlineto{\pgfqpoint{3.267960in}{2.212210in}}%
\pgfpathlineto{\pgfqpoint{3.272920in}{2.210506in}}%
\pgfpathlineto{\pgfqpoint{3.274160in}{2.209712in}}%
\pgfpathlineto{\pgfqpoint{3.277880in}{2.216994in}}%
\pgfpathlineto{\pgfqpoint{3.280360in}{2.213915in}}%
\pgfpathlineto{\pgfqpoint{3.282840in}{2.215625in}}%
\pgfpathlineto{\pgfqpoint{3.284080in}{2.219228in}}%
\pgfpathlineto{\pgfqpoint{3.286560in}{2.219625in}}%
\pgfpathlineto{\pgfqpoint{3.289040in}{2.219897in}}%
\pgfpathlineto{\pgfqpoint{3.291520in}{2.216689in}}%
\pgfpathlineto{\pgfqpoint{3.295240in}{2.208812in}}%
\pgfpathlineto{\pgfqpoint{3.296480in}{2.209606in}}%
\pgfpathlineto{\pgfqpoint{3.298960in}{2.214032in}}%
\pgfpathlineto{\pgfqpoint{3.301440in}{2.213443in}}%
\pgfpathlineto{\pgfqpoint{3.302680in}{2.215131in}}%
\pgfpathlineto{\pgfqpoint{3.305160in}{2.213520in}}%
\pgfpathlineto{\pgfqpoint{3.307640in}{2.214393in}}%
\pgfpathlineto{\pgfqpoint{3.313840in}{2.207658in}}%
\pgfpathlineto{\pgfqpoint{3.316320in}{2.210799in}}%
\pgfpathlineto{\pgfqpoint{3.317560in}{2.209365in}}%
\pgfpathlineto{\pgfqpoint{3.320040in}{2.210836in}}%
\pgfpathlineto{\pgfqpoint{3.325000in}{2.210671in}}%
\pgfpathlineto{\pgfqpoint{3.326240in}{2.211885in}}%
\pgfpathlineto{\pgfqpoint{3.327480in}{2.211379in}}%
\pgfpathlineto{\pgfqpoint{3.331200in}{2.216170in}}%
\pgfpathlineto{\pgfqpoint{3.333680in}{2.216507in}}%
\pgfpathlineto{\pgfqpoint{3.336160in}{2.213392in}}%
\pgfpathlineto{\pgfqpoint{3.338640in}{2.210772in}}%
\pgfpathlineto{\pgfqpoint{3.341120in}{2.209878in}}%
\pgfpathlineto{\pgfqpoint{3.342360in}{2.207801in}}%
\pgfpathlineto{\pgfqpoint{3.344840in}{2.209855in}}%
\pgfpathlineto{\pgfqpoint{3.346080in}{2.209580in}}%
\pgfpathlineto{\pgfqpoint{3.349800in}{2.206510in}}%
\pgfpathlineto{\pgfqpoint{3.352280in}{2.210533in}}%
\pgfpathlineto{\pgfqpoint{3.357240in}{2.201714in}}%
\pgfpathlineto{\pgfqpoint{3.363440in}{2.201476in}}%
\pgfpathlineto{\pgfqpoint{3.367160in}{2.207779in}}%
\pgfpathlineto{\pgfqpoint{3.368400in}{2.207708in}}%
\pgfpathlineto{\pgfqpoint{3.372120in}{2.212021in}}%
\pgfpathlineto{\pgfqpoint{3.373360in}{2.211075in}}%
\pgfpathlineto{\pgfqpoint{3.374600in}{2.207878in}}%
\pgfpathlineto{\pgfqpoint{3.378320in}{2.211233in}}%
\pgfpathlineto{\pgfqpoint{3.383280in}{2.206118in}}%
\pgfpathlineto{\pgfqpoint{3.395680in}{2.211165in}}%
\pgfpathlineto{\pgfqpoint{3.398160in}{2.210232in}}%
\pgfpathlineto{\pgfqpoint{3.401880in}{2.217673in}}%
\pgfpathlineto{\pgfqpoint{3.404360in}{2.214297in}}%
\pgfpathlineto{\pgfqpoint{3.406840in}{2.216672in}}%
\pgfpathlineto{\pgfqpoint{3.408080in}{2.221366in}}%
\pgfpathlineto{\pgfqpoint{3.413040in}{2.221314in}}%
\pgfpathlineto{\pgfqpoint{3.416760in}{2.214666in}}%
\pgfpathlineto{\pgfqpoint{3.419240in}{2.209672in}}%
\pgfpathlineto{\pgfqpoint{3.420480in}{2.210716in}}%
\pgfpathlineto{\pgfqpoint{3.422960in}{2.214799in}}%
\pgfpathlineto{\pgfqpoint{3.425440in}{2.213745in}}%
\pgfpathlineto{\pgfqpoint{3.426680in}{2.215761in}}%
\pgfpathlineto{\pgfqpoint{3.429160in}{2.214242in}}%
\pgfpathlineto{\pgfqpoint{3.431640in}{2.216434in}}%
\pgfpathlineto{\pgfqpoint{3.437840in}{2.210026in}}%
\pgfpathlineto{\pgfqpoint{3.441560in}{2.212796in}}%
\pgfpathlineto{\pgfqpoint{3.445280in}{2.213122in}}%
\pgfpathlineto{\pgfqpoint{3.447760in}{2.211939in}}%
\pgfpathlineto{\pgfqpoint{3.452720in}{2.216709in}}%
\pgfpathlineto{\pgfqpoint{3.455200in}{2.218756in}}%
\pgfpathlineto{\pgfqpoint{3.457680in}{2.217718in}}%
\pgfpathlineto{\pgfqpoint{3.458920in}{2.215155in}}%
\pgfpathlineto{\pgfqpoint{3.460160in}{2.215684in}}%
\pgfpathlineto{\pgfqpoint{3.462640in}{2.213889in}}%
\pgfpathlineto{\pgfqpoint{3.465120in}{2.213385in}}%
\pgfpathlineto{\pgfqpoint{3.466360in}{2.211594in}}%
\pgfpathlineto{\pgfqpoint{3.468840in}{2.213646in}}%
\pgfpathlineto{\pgfqpoint{3.471320in}{2.211573in}}%
\pgfpathlineto{\pgfqpoint{3.473800in}{2.210543in}}%
\pgfpathlineto{\pgfqpoint{3.476280in}{2.214834in}}%
\pgfpathlineto{\pgfqpoint{3.478760in}{2.208906in}}%
\pgfpathlineto{\pgfqpoint{3.480000in}{2.206058in}}%
\pgfpathlineto{\pgfqpoint{3.482480in}{2.207534in}}%
\pgfpathlineto{\pgfqpoint{3.483720in}{2.206762in}}%
\pgfpathlineto{\pgfqpoint{3.486200in}{2.207760in}}%
\pgfpathlineto{\pgfqpoint{3.487440in}{2.206674in}}%
\pgfpathlineto{\pgfqpoint{3.491160in}{2.212898in}}%
\pgfpathlineto{\pgfqpoint{3.492400in}{2.212519in}}%
\pgfpathlineto{\pgfqpoint{3.496120in}{2.217110in}}%
\pgfpathlineto{\pgfqpoint{3.497360in}{2.216569in}}%
\pgfpathlineto{\pgfqpoint{3.498600in}{2.213306in}}%
\pgfpathlineto{\pgfqpoint{3.502320in}{2.216055in}}%
\pgfpathlineto{\pgfqpoint{3.508520in}{2.210833in}}%
\pgfpathlineto{\pgfqpoint{3.511000in}{2.212908in}}%
\pgfpathlineto{\pgfqpoint{3.513480in}{2.212840in}}%
\pgfpathlineto{\pgfqpoint{3.515960in}{2.215895in}}%
\pgfpathlineto{\pgfqpoint{3.517200in}{2.216151in}}%
\pgfpathlineto{\pgfqpoint{3.519680in}{2.214692in}}%
\pgfpathlineto{\pgfqpoint{3.522160in}{2.214500in}}%
\pgfpathlineto{\pgfqpoint{3.525880in}{2.221763in}}%
\pgfpathlineto{\pgfqpoint{3.528360in}{2.217857in}}%
\pgfpathlineto{\pgfqpoint{3.530840in}{2.221224in}}%
\pgfpathlineto{\pgfqpoint{3.532080in}{2.226230in}}%
\pgfpathlineto{\pgfqpoint{3.537040in}{2.225851in}}%
\pgfpathlineto{\pgfqpoint{3.540760in}{2.217755in}}%
\pgfpathlineto{\pgfqpoint{3.543240in}{2.212394in}}%
\pgfpathlineto{\pgfqpoint{3.544480in}{2.213715in}}%
\pgfpathlineto{\pgfqpoint{3.546960in}{2.217873in}}%
\pgfpathlineto{\pgfqpoint{3.549440in}{2.218003in}}%
\pgfpathlineto{\pgfqpoint{3.550680in}{2.220881in}}%
\pgfpathlineto{\pgfqpoint{3.553160in}{2.218994in}}%
\pgfpathlineto{\pgfqpoint{3.555640in}{2.220893in}}%
\pgfpathlineto{\pgfqpoint{3.561840in}{2.214102in}}%
\pgfpathlineto{\pgfqpoint{3.565560in}{2.217408in}}%
\pgfpathlineto{\pgfqpoint{3.573000in}{2.219640in}}%
\pgfpathlineto{\pgfqpoint{3.579200in}{2.226409in}}%
\pgfpathlineto{\pgfqpoint{3.580440in}{2.226853in}}%
\pgfpathlineto{\pgfqpoint{3.581680in}{2.225603in}}%
\pgfpathlineto{\pgfqpoint{3.582920in}{2.222571in}}%
\pgfpathlineto{\pgfqpoint{3.584160in}{2.222792in}}%
\pgfpathlineto{\pgfqpoint{3.586640in}{2.221116in}}%
\pgfpathlineto{\pgfqpoint{3.590360in}{2.219486in}}%
\pgfpathlineto{\pgfqpoint{3.592840in}{2.221780in}}%
\pgfpathlineto{\pgfqpoint{3.595320in}{2.220041in}}%
\pgfpathlineto{\pgfqpoint{3.597800in}{2.218746in}}%
\pgfpathlineto{\pgfqpoint{3.600280in}{2.223872in}}%
\pgfpathlineto{\pgfqpoint{3.601520in}{2.222168in}}%
\pgfpathlineto{\pgfqpoint{3.604000in}{2.215407in}}%
\pgfpathlineto{\pgfqpoint{3.605240in}{2.218246in}}%
\pgfpathlineto{\pgfqpoint{3.611440in}{2.217519in}}%
\pgfpathlineto{\pgfqpoint{3.613920in}{2.221503in}}%
\pgfpathlineto{\pgfqpoint{3.615160in}{2.223226in}}%
\pgfpathlineto{\pgfqpoint{3.616400in}{2.222456in}}%
\pgfpathlineto{\pgfqpoint{3.618880in}{2.226880in}}%
\pgfpathlineto{\pgfqpoint{3.620120in}{2.226855in}}%
\pgfpathlineto{\pgfqpoint{3.621360in}{2.225708in}}%
\pgfpathlineto{\pgfqpoint{3.622600in}{2.222840in}}%
\pgfpathlineto{\pgfqpoint{3.623840in}{2.224244in}}%
\pgfpathlineto{\pgfqpoint{3.625080in}{2.223603in}}%
\pgfpathlineto{\pgfqpoint{3.626320in}{2.224941in}}%
\pgfpathlineto{\pgfqpoint{3.632520in}{2.219006in}}%
\pgfpathlineto{\pgfqpoint{3.635000in}{2.221073in}}%
\pgfpathlineto{\pgfqpoint{3.637480in}{2.220791in}}%
\pgfpathlineto{\pgfqpoint{3.638720in}{2.223553in}}%
\pgfpathlineto{\pgfqpoint{3.641200in}{2.223157in}}%
\pgfpathlineto{\pgfqpoint{3.643680in}{2.220981in}}%
\pgfpathlineto{\pgfqpoint{3.646160in}{2.220919in}}%
\pgfpathlineto{\pgfqpoint{3.649880in}{2.228435in}}%
\pgfpathlineto{\pgfqpoint{3.652360in}{2.224152in}}%
\pgfpathlineto{\pgfqpoint{3.654840in}{2.227633in}}%
\pgfpathlineto{\pgfqpoint{3.656080in}{2.232929in}}%
\pgfpathlineto{\pgfqpoint{3.661040in}{2.233720in}}%
\pgfpathlineto{\pgfqpoint{3.664760in}{2.226381in}}%
\pgfpathlineto{\pgfqpoint{3.667240in}{2.221406in}}%
\pgfpathlineto{\pgfqpoint{3.668480in}{2.222729in}}%
\pgfpathlineto{\pgfqpoint{3.670960in}{2.226353in}}%
\pgfpathlineto{\pgfqpoint{3.673440in}{2.227042in}}%
\pgfpathlineto{\pgfqpoint{3.674680in}{2.229931in}}%
\pgfpathlineto{\pgfqpoint{3.677160in}{2.228608in}}%
\pgfpathlineto{\pgfqpoint{3.679640in}{2.231572in}}%
\pgfpathlineto{\pgfqpoint{3.685840in}{2.225534in}}%
\pgfpathlineto{\pgfqpoint{3.689560in}{2.228926in}}%
\pgfpathlineto{\pgfqpoint{3.695760in}{2.229498in}}%
\pgfpathlineto{\pgfqpoint{3.704440in}{2.237541in}}%
\pgfpathlineto{\pgfqpoint{3.709400in}{2.231973in}}%
\pgfpathlineto{\pgfqpoint{3.713120in}{2.231658in}}%
\pgfpathlineto{\pgfqpoint{3.714360in}{2.229804in}}%
\pgfpathlineto{\pgfqpoint{3.718080in}{2.232307in}}%
\pgfpathlineto{\pgfqpoint{3.721800in}{2.229319in}}%
\pgfpathlineto{\pgfqpoint{3.724280in}{2.234434in}}%
\pgfpathlineto{\pgfqpoint{3.726760in}{2.229305in}}%
\pgfpathlineto{\pgfqpoint{3.728000in}{2.226919in}}%
\pgfpathlineto{\pgfqpoint{3.729240in}{2.228592in}}%
\pgfpathlineto{\pgfqpoint{3.732960in}{2.228651in}}%
\pgfpathlineto{\pgfqpoint{3.735440in}{2.227582in}}%
\pgfpathlineto{\pgfqpoint{3.737920in}{2.232067in}}%
\pgfpathlineto{\pgfqpoint{3.739160in}{2.233525in}}%
\pgfpathlineto{\pgfqpoint{3.740400in}{2.232654in}}%
\pgfpathlineto{\pgfqpoint{3.742880in}{2.237886in}}%
\pgfpathlineto{\pgfqpoint{3.745360in}{2.236841in}}%
\pgfpathlineto{\pgfqpoint{3.746600in}{2.234139in}}%
\pgfpathlineto{\pgfqpoint{3.747840in}{2.234838in}}%
\pgfpathlineto{\pgfqpoint{3.749080in}{2.234077in}}%
\pgfpathlineto{\pgfqpoint{3.750320in}{2.235079in}}%
\pgfpathlineto{\pgfqpoint{3.754040in}{2.232243in}}%
\pgfpathlineto{\pgfqpoint{3.755280in}{2.230192in}}%
\pgfpathlineto{\pgfqpoint{3.756520in}{2.230354in}}%
\pgfpathlineto{\pgfqpoint{3.759000in}{2.232528in}}%
\pgfpathlineto{\pgfqpoint{3.761480in}{2.231703in}}%
\pgfpathlineto{\pgfqpoint{3.763960in}{2.235049in}}%
\pgfpathlineto{\pgfqpoint{3.765200in}{2.235259in}}%
\pgfpathlineto{\pgfqpoint{3.767680in}{2.233542in}}%
\pgfpathlineto{\pgfqpoint{3.768920in}{2.233051in}}%
\pgfpathlineto{\pgfqpoint{3.770160in}{2.233787in}}%
\pgfpathlineto{\pgfqpoint{3.773880in}{2.241969in}}%
\pgfpathlineto{\pgfqpoint{3.776360in}{2.237000in}}%
\pgfpathlineto{\pgfqpoint{3.778840in}{2.240522in}}%
\pgfpathlineto{\pgfqpoint{3.781320in}{2.245683in}}%
\pgfpathlineto{\pgfqpoint{3.785040in}{2.246448in}}%
\pgfpathlineto{\pgfqpoint{3.791240in}{2.233252in}}%
\pgfpathlineto{\pgfqpoint{3.796200in}{2.238438in}}%
\pgfpathlineto{\pgfqpoint{3.797440in}{2.238777in}}%
\pgfpathlineto{\pgfqpoint{3.798680in}{2.241010in}}%
\pgfpathlineto{\pgfqpoint{3.801160in}{2.239042in}}%
\pgfpathlineto{\pgfqpoint{3.803640in}{2.241705in}}%
\pgfpathlineto{\pgfqpoint{3.806120in}{2.239300in}}%
\pgfpathlineto{\pgfqpoint{3.809840in}{2.236863in}}%
\pgfpathlineto{\pgfqpoint{3.813560in}{2.240109in}}%
\pgfpathlineto{\pgfqpoint{3.819760in}{2.240572in}}%
\pgfpathlineto{\pgfqpoint{3.822240in}{2.242830in}}%
\pgfpathlineto{\pgfqpoint{3.823480in}{2.242932in}}%
\pgfpathlineto{\pgfqpoint{3.827200in}{2.248419in}}%
\pgfpathlineto{\pgfqpoint{3.828440in}{2.249146in}}%
\pgfpathlineto{\pgfqpoint{3.833400in}{2.243053in}}%
\pgfpathlineto{\pgfqpoint{3.835880in}{2.243144in}}%
\pgfpathlineto{\pgfqpoint{3.837120in}{2.242783in}}%
\pgfpathlineto{\pgfqpoint{3.838360in}{2.240860in}}%
\pgfpathlineto{\pgfqpoint{3.842080in}{2.243597in}}%
\pgfpathlineto{\pgfqpoint{3.843320in}{2.242979in}}%
\pgfpathlineto{\pgfqpoint{3.845800in}{2.240837in}}%
\pgfpathlineto{\pgfqpoint{3.848280in}{2.246676in}}%
\pgfpathlineto{\pgfqpoint{3.850760in}{2.241530in}}%
\pgfpathlineto{\pgfqpoint{3.852000in}{2.238729in}}%
\pgfpathlineto{\pgfqpoint{3.853240in}{2.243307in}}%
\pgfpathlineto{\pgfqpoint{3.859440in}{2.240214in}}%
\pgfpathlineto{\pgfqpoint{3.861920in}{2.245113in}}%
\pgfpathlineto{\pgfqpoint{3.863160in}{2.246570in}}%
\pgfpathlineto{\pgfqpoint{3.864400in}{2.245223in}}%
\pgfpathlineto{\pgfqpoint{3.866880in}{2.249859in}}%
\pgfpathlineto{\pgfqpoint{3.869360in}{2.248723in}}%
\pgfpathlineto{\pgfqpoint{3.870600in}{2.246201in}}%
\pgfpathlineto{\pgfqpoint{3.871840in}{2.247590in}}%
\pgfpathlineto{\pgfqpoint{3.873080in}{2.246887in}}%
\pgfpathlineto{\pgfqpoint{3.875560in}{2.247013in}}%
\pgfpathlineto{\pgfqpoint{3.879280in}{2.244022in}}%
\pgfpathlineto{\pgfqpoint{3.883000in}{2.245869in}}%
\pgfpathlineto{\pgfqpoint{3.885480in}{2.244511in}}%
\pgfpathlineto{\pgfqpoint{3.887960in}{2.247969in}}%
\pgfpathlineto{\pgfqpoint{3.889200in}{2.247893in}}%
\pgfpathlineto{\pgfqpoint{3.891680in}{2.246644in}}%
\pgfpathlineto{\pgfqpoint{3.894160in}{2.246036in}}%
\pgfpathlineto{\pgfqpoint{3.897880in}{2.253014in}}%
\pgfpathlineto{\pgfqpoint{3.900360in}{2.248851in}}%
\pgfpathlineto{\pgfqpoint{3.902840in}{2.252274in}}%
\pgfpathlineto{\pgfqpoint{3.904080in}{2.257630in}}%
\pgfpathlineto{\pgfqpoint{3.909040in}{2.258820in}}%
\pgfpathlineto{\pgfqpoint{3.915240in}{2.246093in}}%
\pgfpathlineto{\pgfqpoint{3.922680in}{2.253753in}}%
\pgfpathlineto{\pgfqpoint{3.925160in}{2.251075in}}%
\pgfpathlineto{\pgfqpoint{3.927640in}{2.255008in}}%
\pgfpathlineto{\pgfqpoint{3.935080in}{2.251576in}}%
\pgfpathlineto{\pgfqpoint{3.936320in}{2.253856in}}%
\pgfpathlineto{\pgfqpoint{3.941280in}{2.253050in}}%
\pgfpathlineto{\pgfqpoint{3.942520in}{2.252712in}}%
\pgfpathlineto{\pgfqpoint{3.946240in}{2.256232in}}%
\pgfpathlineto{\pgfqpoint{3.947480in}{2.256346in}}%
\pgfpathlineto{\pgfqpoint{3.951200in}{2.262667in}}%
\pgfpathlineto{\pgfqpoint{3.952440in}{2.263100in}}%
\pgfpathlineto{\pgfqpoint{3.957400in}{2.256439in}}%
\pgfpathlineto{\pgfqpoint{3.958640in}{2.257096in}}%
\pgfpathlineto{\pgfqpoint{3.961120in}{2.255621in}}%
\pgfpathlineto{\pgfqpoint{3.962360in}{2.253773in}}%
\pgfpathlineto{\pgfqpoint{3.967320in}{2.254681in}}%
\pgfpathlineto{\pgfqpoint{3.969800in}{2.253069in}}%
\pgfpathlineto{\pgfqpoint{3.972280in}{2.259636in}}%
\pgfpathlineto{\pgfqpoint{3.976000in}{2.252233in}}%
\pgfpathlineto{\pgfqpoint{3.977240in}{2.255669in}}%
\pgfpathlineto{\pgfqpoint{3.979720in}{2.253669in}}%
\pgfpathlineto{\pgfqpoint{3.982200in}{2.253202in}}%
\pgfpathlineto{\pgfqpoint{3.983440in}{2.251773in}}%
\pgfpathlineto{\pgfqpoint{3.987160in}{2.257916in}}%
\pgfpathlineto{\pgfqpoint{3.988400in}{2.256233in}}%
\pgfpathlineto{\pgfqpoint{3.992120in}{2.259945in}}%
\pgfpathlineto{\pgfqpoint{3.993360in}{2.259751in}}%
\pgfpathlineto{\pgfqpoint{3.994600in}{2.257538in}}%
\pgfpathlineto{\pgfqpoint{3.995840in}{2.259505in}}%
\pgfpathlineto{\pgfqpoint{3.997080in}{2.258942in}}%
\pgfpathlineto{\pgfqpoint{3.998320in}{2.260220in}}%
\pgfpathlineto{\pgfqpoint{4.002040in}{2.258028in}}%
\pgfpathlineto{\pgfqpoint{4.003280in}{2.256624in}}%
\pgfpathlineto{\pgfqpoint{4.009480in}{2.257776in}}%
\pgfpathlineto{\pgfqpoint{4.011960in}{2.261419in}}%
\pgfpathlineto{\pgfqpoint{4.014440in}{2.259722in}}%
\pgfpathlineto{\pgfqpoint{4.018160in}{2.258515in}}%
\pgfpathlineto{\pgfqpoint{4.021880in}{2.265680in}}%
\pgfpathlineto{\pgfqpoint{4.024360in}{2.261490in}}%
\pgfpathlineto{\pgfqpoint{4.026840in}{2.265468in}}%
\pgfpathlineto{\pgfqpoint{4.029320in}{2.271673in}}%
\pgfpathlineto{\pgfqpoint{4.031800in}{2.273193in}}%
\pgfpathlineto{\pgfqpoint{4.033040in}{2.272809in}}%
\pgfpathlineto{\pgfqpoint{4.036760in}{2.264614in}}%
\pgfpathlineto{\pgfqpoint{4.039240in}{2.260848in}}%
\pgfpathlineto{\pgfqpoint{4.040480in}{2.261515in}}%
\pgfpathlineto{\pgfqpoint{4.042960in}{2.264626in}}%
\pgfpathlineto{\pgfqpoint{4.046680in}{2.267282in}}%
\pgfpathlineto{\pgfqpoint{4.049160in}{2.264550in}}%
\pgfpathlineto{\pgfqpoint{4.051640in}{2.269058in}}%
\pgfpathlineto{\pgfqpoint{4.055360in}{2.265896in}}%
\pgfpathlineto{\pgfqpoint{4.057840in}{2.264229in}}%
\pgfpathlineto{\pgfqpoint{4.062800in}{2.266931in}}%
\pgfpathlineto{\pgfqpoint{4.064040in}{2.265828in}}%
\pgfpathlineto{\pgfqpoint{4.071480in}{2.269557in}}%
\pgfpathlineto{\pgfqpoint{4.075200in}{2.275351in}}%
\pgfpathlineto{\pgfqpoint{4.076440in}{2.275039in}}%
\pgfpathlineto{\pgfqpoint{4.081400in}{2.266866in}}%
\pgfpathlineto{\pgfqpoint{4.082640in}{2.266983in}}%
\pgfpathlineto{\pgfqpoint{4.086360in}{2.263827in}}%
\pgfpathlineto{\pgfqpoint{4.088840in}{2.264181in}}%
\pgfpathlineto{\pgfqpoint{4.091320in}{2.263797in}}%
\pgfpathlineto{\pgfqpoint{4.093800in}{2.262591in}}%
\pgfpathlineto{\pgfqpoint{4.096280in}{2.269411in}}%
\pgfpathlineto{\pgfqpoint{4.100000in}{2.262649in}}%
\pgfpathlineto{\pgfqpoint{4.101240in}{2.264293in}}%
\pgfpathlineto{\pgfqpoint{4.107440in}{2.260809in}}%
\pgfpathlineto{\pgfqpoint{4.111160in}{2.267427in}}%
\pgfpathlineto{\pgfqpoint{4.112400in}{2.266192in}}%
\pgfpathlineto{\pgfqpoint{4.114880in}{2.270224in}}%
\pgfpathlineto{\pgfqpoint{4.118600in}{2.264840in}}%
\pgfpathlineto{\pgfqpoint{4.119840in}{2.266058in}}%
\pgfpathlineto{\pgfqpoint{4.129760in}{2.263802in}}%
\pgfpathlineto{\pgfqpoint{4.133480in}{2.264304in}}%
\pgfpathlineto{\pgfqpoint{4.135960in}{2.267346in}}%
\pgfpathlineto{\pgfqpoint{4.138440in}{2.266597in}}%
\pgfpathlineto{\pgfqpoint{4.142160in}{2.265502in}}%
\pgfpathlineto{\pgfqpoint{4.145880in}{2.272565in}}%
\pgfpathlineto{\pgfqpoint{4.148360in}{2.267546in}}%
\pgfpathlineto{\pgfqpoint{4.150840in}{2.272012in}}%
\pgfpathlineto{\pgfqpoint{4.153320in}{2.278090in}}%
\pgfpathlineto{\pgfqpoint{4.155800in}{2.279459in}}%
\pgfpathlineto{\pgfqpoint{4.159520in}{2.276626in}}%
\pgfpathlineto{\pgfqpoint{4.162000in}{2.268966in}}%
\pgfpathlineto{\pgfqpoint{4.163240in}{2.268210in}}%
\pgfpathlineto{\pgfqpoint{4.164480in}{2.268945in}}%
\pgfpathlineto{\pgfqpoint{4.166960in}{2.272421in}}%
\pgfpathlineto{\pgfqpoint{4.170680in}{2.276232in}}%
\pgfpathlineto{\pgfqpoint{4.173160in}{2.273657in}}%
\pgfpathlineto{\pgfqpoint{4.175640in}{2.278209in}}%
\pgfpathlineto{\pgfqpoint{4.176880in}{2.277788in}}%
\pgfpathlineto{\pgfqpoint{4.179360in}{2.275047in}}%
\pgfpathlineto{\pgfqpoint{4.181840in}{2.274011in}}%
\pgfpathlineto{\pgfqpoint{4.186800in}{2.275822in}}%
\pgfpathlineto{\pgfqpoint{4.189280in}{2.274518in}}%
\pgfpathlineto{\pgfqpoint{4.190520in}{2.274121in}}%
\pgfpathlineto{\pgfqpoint{4.193000in}{2.278329in}}%
\pgfpathlineto{\pgfqpoint{4.195480in}{2.280579in}}%
\pgfpathlineto{\pgfqpoint{4.197960in}{2.285970in}}%
\pgfpathlineto{\pgfqpoint{4.199200in}{2.286618in}}%
\pgfpathlineto{\pgfqpoint{4.200440in}{2.285772in}}%
\pgfpathlineto{\pgfqpoint{4.205400in}{2.275904in}}%
\pgfpathlineto{\pgfqpoint{4.206640in}{2.275753in}}%
\pgfpathlineto{\pgfqpoint{4.210360in}{2.271703in}}%
\pgfpathlineto{\pgfqpoint{4.214080in}{2.273508in}}%
\pgfpathlineto{\pgfqpoint{4.217800in}{2.270708in}}%
\pgfpathlineto{\pgfqpoint{4.220280in}{2.278701in}}%
\pgfpathlineto{\pgfqpoint{4.222760in}{2.273981in}}%
\pgfpathlineto{\pgfqpoint{4.224000in}{2.272001in}}%
\pgfpathlineto{\pgfqpoint{4.225240in}{2.273670in}}%
\pgfpathlineto{\pgfqpoint{4.227720in}{2.273423in}}%
\pgfpathlineto{\pgfqpoint{4.231440in}{2.269474in}}%
\pgfpathlineto{\pgfqpoint{4.235160in}{2.276250in}}%
\pgfpathlineto{\pgfqpoint{4.236400in}{2.275182in}}%
\pgfpathlineto{\pgfqpoint{4.238880in}{2.280247in}}%
\pgfpathlineto{\pgfqpoint{4.242600in}{2.273563in}}%
\pgfpathlineto{\pgfqpoint{4.245080in}{2.274867in}}%
\pgfpathlineto{\pgfqpoint{4.246320in}{2.275177in}}%
\pgfpathlineto{\pgfqpoint{4.252520in}{2.270128in}}%
\pgfpathlineto{\pgfqpoint{4.255000in}{2.271975in}}%
\pgfpathlineto{\pgfqpoint{4.257480in}{2.272067in}}%
\pgfpathlineto{\pgfqpoint{4.259960in}{2.274877in}}%
\pgfpathlineto{\pgfqpoint{4.261200in}{2.274950in}}%
\pgfpathlineto{\pgfqpoint{4.264920in}{2.272020in}}%
\pgfpathlineto{\pgfqpoint{4.266160in}{2.272432in}}%
\pgfpathlineto{\pgfqpoint{4.268640in}{2.278134in}}%
\pgfpathlineto{\pgfqpoint{4.269880in}{2.279070in}}%
\pgfpathlineto{\pgfqpoint{4.272360in}{2.273525in}}%
\pgfpathlineto{\pgfqpoint{4.274840in}{2.278363in}}%
\pgfpathlineto{\pgfqpoint{4.277320in}{2.284406in}}%
\pgfpathlineto{\pgfqpoint{4.279800in}{2.285546in}}%
\pgfpathlineto{\pgfqpoint{4.281040in}{2.284909in}}%
\pgfpathlineto{\pgfqpoint{4.284760in}{2.277691in}}%
\pgfpathlineto{\pgfqpoint{4.287240in}{2.274785in}}%
\pgfpathlineto{\pgfqpoint{4.288480in}{2.275557in}}%
\pgfpathlineto{\pgfqpoint{4.290960in}{2.279999in}}%
\pgfpathlineto{\pgfqpoint{4.294680in}{2.283643in}}%
\pgfpathlineto{\pgfqpoint{4.297160in}{2.281338in}}%
\pgfpathlineto{\pgfqpoint{4.299640in}{2.284096in}}%
\pgfpathlineto{\pgfqpoint{4.305840in}{2.280608in}}%
\pgfpathlineto{\pgfqpoint{4.308320in}{2.283542in}}%
\pgfpathlineto{\pgfqpoint{4.309560in}{2.283120in}}%
\pgfpathlineto{\pgfqpoint{4.310800in}{2.284548in}}%
\pgfpathlineto{\pgfqpoint{4.314520in}{2.283048in}}%
\pgfpathlineto{\pgfqpoint{4.318240in}{2.289627in}}%
\pgfpathlineto{\pgfqpoint{4.319480in}{2.291014in}}%
\pgfpathlineto{\pgfqpoint{4.321960in}{2.296626in}}%
\pgfpathlineto{\pgfqpoint{4.323200in}{2.297795in}}%
\pgfpathlineto{\pgfqpoint{4.324440in}{2.297328in}}%
\pgfpathlineto{\pgfqpoint{4.326920in}{2.291748in}}%
\pgfpathlineto{\pgfqpoint{4.328160in}{2.290975in}}%
\pgfpathlineto{\pgfqpoint{4.330640in}{2.286360in}}%
\pgfpathlineto{\pgfqpoint{4.334360in}{2.281187in}}%
\pgfpathlineto{\pgfqpoint{4.339320in}{2.280968in}}%
\pgfpathlineto{\pgfqpoint{4.341800in}{2.279428in}}%
\pgfpathlineto{\pgfqpoint{4.344280in}{2.286383in}}%
\pgfpathlineto{\pgfqpoint{4.349240in}{2.279423in}}%
\pgfpathlineto{\pgfqpoint{4.351720in}{2.279560in}}%
\pgfpathlineto{\pgfqpoint{4.355440in}{2.275750in}}%
\pgfpathlineto{\pgfqpoint{4.359160in}{2.283249in}}%
\pgfpathlineto{\pgfqpoint{4.360400in}{2.281782in}}%
\pgfpathlineto{\pgfqpoint{4.362880in}{2.286040in}}%
\pgfpathlineto{\pgfqpoint{4.366600in}{2.279557in}}%
\pgfpathlineto{\pgfqpoint{4.367840in}{2.280997in}}%
\pgfpathlineto{\pgfqpoint{4.371560in}{2.279694in}}%
\pgfpathlineto{\pgfqpoint{4.374040in}{2.278169in}}%
\pgfpathlineto{\pgfqpoint{4.375280in}{2.275864in}}%
\pgfpathlineto{\pgfqpoint{4.376520in}{2.276246in}}%
\pgfpathlineto{\pgfqpoint{4.379000in}{2.278204in}}%
\pgfpathlineto{\pgfqpoint{4.381480in}{2.278550in}}%
\pgfpathlineto{\pgfqpoint{4.383960in}{2.281218in}}%
\pgfpathlineto{\pgfqpoint{4.385200in}{2.281442in}}%
\pgfpathlineto{\pgfqpoint{4.387680in}{2.279295in}}%
\pgfpathlineto{\pgfqpoint{4.390160in}{2.277888in}}%
\pgfpathlineto{\pgfqpoint{4.392640in}{2.283915in}}%
\pgfpathlineto{\pgfqpoint{4.393880in}{2.285046in}}%
\pgfpathlineto{\pgfqpoint{4.396360in}{2.279854in}}%
\pgfpathlineto{\pgfqpoint{4.398840in}{2.284901in}}%
\pgfpathlineto{\pgfqpoint{4.401320in}{2.291513in}}%
\pgfpathlineto{\pgfqpoint{4.403800in}{2.292146in}}%
\pgfpathlineto{\pgfqpoint{4.407520in}{2.290263in}}%
\pgfpathlineto{\pgfqpoint{4.410000in}{2.283192in}}%
\pgfpathlineto{\pgfqpoint{4.411240in}{2.282692in}}%
\pgfpathlineto{\pgfqpoint{4.412480in}{2.283471in}}%
\pgfpathlineto{\pgfqpoint{4.414960in}{2.287196in}}%
\pgfpathlineto{\pgfqpoint{4.418680in}{2.290937in}}%
\pgfpathlineto{\pgfqpoint{4.419920in}{2.288759in}}%
\pgfpathlineto{\pgfqpoint{4.421160in}{2.288872in}}%
\pgfpathlineto{\pgfqpoint{4.423640in}{2.292073in}}%
\pgfpathlineto{\pgfqpoint{4.429840in}{2.288929in}}%
\pgfpathlineto{\pgfqpoint{4.432320in}{2.291964in}}%
\pgfpathlineto{\pgfqpoint{4.433560in}{2.291506in}}%
\pgfpathlineto{\pgfqpoint{4.436040in}{2.292017in}}%
\pgfpathlineto{\pgfqpoint{4.438520in}{2.292449in}}%
\pgfpathlineto{\pgfqpoint{4.441000in}{2.296834in}}%
\pgfpathlineto{\pgfqpoint{4.448440in}{2.304682in}}%
\pgfpathlineto{\pgfqpoint{4.454640in}{2.293060in}}%
\pgfpathlineto{\pgfqpoint{4.458360in}{2.288174in}}%
\pgfpathlineto{\pgfqpoint{4.462080in}{2.289282in}}%
\pgfpathlineto{\pgfqpoint{4.465800in}{2.287374in}}%
\pgfpathlineto{\pgfqpoint{4.468280in}{2.293961in}}%
\pgfpathlineto{\pgfqpoint{4.472000in}{2.286974in}}%
\pgfpathlineto{\pgfqpoint{4.475720in}{2.289818in}}%
\pgfpathlineto{\pgfqpoint{4.478200in}{2.288701in}}%
\pgfpathlineto{\pgfqpoint{4.479440in}{2.287001in}}%
\pgfpathlineto{\pgfqpoint{4.483160in}{2.295484in}}%
\pgfpathlineto{\pgfqpoint{4.484400in}{2.293828in}}%
\pgfpathlineto{\pgfqpoint{4.486880in}{2.298645in}}%
\pgfpathlineto{\pgfqpoint{4.490600in}{2.293098in}}%
\pgfpathlineto{\pgfqpoint{4.491840in}{2.293771in}}%
\pgfpathlineto{\pgfqpoint{4.493080in}{2.292245in}}%
\pgfpathlineto{\pgfqpoint{4.495560in}{2.292761in}}%
\pgfpathlineto{\pgfqpoint{4.498040in}{2.290403in}}%
\pgfpathlineto{\pgfqpoint{4.499280in}{2.287984in}}%
\pgfpathlineto{\pgfqpoint{4.505480in}{2.289586in}}%
\pgfpathlineto{\pgfqpoint{4.507960in}{2.291653in}}%
\pgfpathlineto{\pgfqpoint{4.509200in}{2.291187in}}%
\pgfpathlineto{\pgfqpoint{4.511680in}{2.288413in}}%
\pgfpathlineto{\pgfqpoint{4.512920in}{2.287129in}}%
\pgfpathlineto{\pgfqpoint{4.514160in}{2.287715in}}%
\pgfpathlineto{\pgfqpoint{4.516640in}{2.294112in}}%
\pgfpathlineto{\pgfqpoint{4.517880in}{2.296289in}}%
\pgfpathlineto{\pgfqpoint{4.520360in}{2.290468in}}%
\pgfpathlineto{\pgfqpoint{4.522840in}{2.294410in}}%
\pgfpathlineto{\pgfqpoint{4.525320in}{2.301687in}}%
\pgfpathlineto{\pgfqpoint{4.529040in}{2.301364in}}%
\pgfpathlineto{\pgfqpoint{4.531520in}{2.299724in}}%
\pgfpathlineto{\pgfqpoint{4.534000in}{2.292578in}}%
\pgfpathlineto{\pgfqpoint{4.535240in}{2.291958in}}%
\pgfpathlineto{\pgfqpoint{4.538960in}{2.295215in}}%
\pgfpathlineto{\pgfqpoint{4.542680in}{2.299786in}}%
\pgfpathlineto{\pgfqpoint{4.545160in}{2.297254in}}%
\pgfpathlineto{\pgfqpoint{4.547640in}{2.300413in}}%
\pgfpathlineto{\pgfqpoint{4.552600in}{2.296633in}}%
\pgfpathlineto{\pgfqpoint{4.553840in}{2.296681in}}%
\pgfpathlineto{\pgfqpoint{4.556320in}{2.299832in}}%
\pgfpathlineto{\pgfqpoint{4.557560in}{2.299463in}}%
\pgfpathlineto{\pgfqpoint{4.560040in}{2.300926in}}%
\pgfpathlineto{\pgfqpoint{4.562520in}{2.299911in}}%
\pgfpathlineto{\pgfqpoint{4.565000in}{2.304993in}}%
\pgfpathlineto{\pgfqpoint{4.568720in}{2.311719in}}%
\pgfpathlineto{\pgfqpoint{4.571200in}{2.313896in}}%
\pgfpathlineto{\pgfqpoint{4.572440in}{2.313427in}}%
\pgfpathlineto{\pgfqpoint{4.578640in}{2.302601in}}%
\pgfpathlineto{\pgfqpoint{4.583600in}{2.297620in}}%
\pgfpathlineto{\pgfqpoint{4.586080in}{2.298575in}}%
\pgfpathlineto{\pgfqpoint{4.589800in}{2.297147in}}%
\pgfpathlineto{\pgfqpoint{4.592280in}{2.303009in}}%
\pgfpathlineto{\pgfqpoint{4.596000in}{2.295460in}}%
\pgfpathlineto{\pgfqpoint{4.597240in}{2.298611in}}%
\pgfpathlineto{\pgfqpoint{4.602200in}{2.297916in}}%
\pgfpathlineto{\pgfqpoint{4.603440in}{2.295558in}}%
\pgfpathlineto{\pgfqpoint{4.607160in}{2.305319in}}%
\pgfpathlineto{\pgfqpoint{4.608400in}{2.304273in}}%
\pgfpathlineto{\pgfqpoint{4.610880in}{2.309890in}}%
\pgfpathlineto{\pgfqpoint{4.614600in}{2.303874in}}%
\pgfpathlineto{\pgfqpoint{4.615840in}{2.303872in}}%
\pgfpathlineto{\pgfqpoint{4.618320in}{2.302711in}}%
\pgfpathlineto{\pgfqpoint{4.619560in}{2.302504in}}%
\pgfpathlineto{\pgfqpoint{4.624520in}{2.297135in}}%
\pgfpathlineto{\pgfqpoint{4.627000in}{2.297949in}}%
\pgfpathlineto{\pgfqpoint{4.629480in}{2.298928in}}%
\pgfpathlineto{\pgfqpoint{4.630720in}{2.300961in}}%
\pgfpathlineto{\pgfqpoint{4.638160in}{2.298368in}}%
\pgfpathlineto{\pgfqpoint{4.640640in}{2.305097in}}%
\pgfpathlineto{\pgfqpoint{4.641880in}{2.307108in}}%
\pgfpathlineto{\pgfqpoint{4.644360in}{2.300872in}}%
\pgfpathlineto{\pgfqpoint{4.646840in}{2.304848in}}%
\pgfpathlineto{\pgfqpoint{4.649320in}{2.312442in}}%
\pgfpathlineto{\pgfqpoint{4.651800in}{2.313032in}}%
\pgfpathlineto{\pgfqpoint{4.655520in}{2.309947in}}%
\pgfpathlineto{\pgfqpoint{4.658000in}{2.302308in}}%
\pgfpathlineto{\pgfqpoint{4.659240in}{2.300916in}}%
\pgfpathlineto{\pgfqpoint{4.660480in}{2.301397in}}%
\pgfpathlineto{\pgfqpoint{4.662960in}{2.303358in}}%
\pgfpathlineto{\pgfqpoint{4.666680in}{2.307784in}}%
\pgfpathlineto{\pgfqpoint{4.667920in}{2.306291in}}%
\pgfpathlineto{\pgfqpoint{4.669160in}{2.304711in}}%
\pgfpathlineto{\pgfqpoint{4.671640in}{2.307660in}}%
\pgfpathlineto{\pgfqpoint{4.675360in}{2.305417in}}%
\pgfpathlineto{\pgfqpoint{4.677840in}{2.306077in}}%
\pgfpathlineto{\pgfqpoint{4.680320in}{2.309815in}}%
\pgfpathlineto{\pgfqpoint{4.682800in}{2.309239in}}%
\pgfpathlineto{\pgfqpoint{4.684040in}{2.310087in}}%
\pgfpathlineto{\pgfqpoint{4.686520in}{2.307705in}}%
\pgfpathlineto{\pgfqpoint{4.690240in}{2.314534in}}%
\pgfpathlineto{\pgfqpoint{4.691480in}{2.316134in}}%
\pgfpathlineto{\pgfqpoint{4.693960in}{2.322626in}}%
\pgfpathlineto{\pgfqpoint{4.695200in}{2.323433in}}%
\pgfpathlineto{\pgfqpoint{4.697680in}{2.318395in}}%
\pgfpathlineto{\pgfqpoint{4.701400in}{2.311900in}}%
\pgfpathlineto{\pgfqpoint{4.707600in}{2.304492in}}%
\pgfpathlineto{\pgfqpoint{4.710080in}{2.304642in}}%
\pgfpathlineto{\pgfqpoint{4.713800in}{2.303245in}}%
\pgfpathlineto{\pgfqpoint{4.716280in}{2.308958in}}%
\pgfpathlineto{\pgfqpoint{4.720000in}{2.300940in}}%
\pgfpathlineto{\pgfqpoint{4.721240in}{2.301713in}}%
\pgfpathlineto{\pgfqpoint{4.723720in}{2.303853in}}%
\pgfpathlineto{\pgfqpoint{4.726200in}{2.302853in}}%
\pgfpathlineto{\pgfqpoint{4.727440in}{2.300620in}}%
\pgfpathlineto{\pgfqpoint{4.731160in}{2.310555in}}%
\pgfpathlineto{\pgfqpoint{4.732400in}{2.309900in}}%
\pgfpathlineto{\pgfqpoint{4.734880in}{2.316600in}}%
\pgfpathlineto{\pgfqpoint{4.739840in}{2.312053in}}%
\pgfpathlineto{\pgfqpoint{4.742320in}{2.311460in}}%
\pgfpathlineto{\pgfqpoint{4.743560in}{2.311577in}}%
\pgfpathlineto{\pgfqpoint{4.748520in}{2.305407in}}%
\pgfpathlineto{\pgfqpoint{4.751000in}{2.307449in}}%
\pgfpathlineto{\pgfqpoint{4.753480in}{2.308126in}}%
\pgfpathlineto{\pgfqpoint{4.754720in}{2.309897in}}%
\pgfpathlineto{\pgfqpoint{4.759680in}{2.308013in}}%
\pgfpathlineto{\pgfqpoint{4.760920in}{2.306297in}}%
\pgfpathlineto{\pgfqpoint{4.762160in}{2.306736in}}%
\pgfpathlineto{\pgfqpoint{4.764640in}{2.312825in}}%
\pgfpathlineto{\pgfqpoint{4.765880in}{2.314367in}}%
\pgfpathlineto{\pgfqpoint{4.768360in}{2.308441in}}%
\pgfpathlineto{\pgfqpoint{4.770840in}{2.313206in}}%
\pgfpathlineto{\pgfqpoint{4.772080in}{2.318901in}}%
\pgfpathlineto{\pgfqpoint{4.775800in}{2.319945in}}%
\pgfpathlineto{\pgfqpoint{4.777040in}{2.319393in}}%
\pgfpathlineto{\pgfqpoint{4.783240in}{2.307498in}}%
\pgfpathlineto{\pgfqpoint{4.790680in}{2.312232in}}%
\pgfpathlineto{\pgfqpoint{4.793160in}{2.309000in}}%
\pgfpathlineto{\pgfqpoint{4.796880in}{2.311538in}}%
\pgfpathlineto{\pgfqpoint{4.799360in}{2.310748in}}%
\pgfpathlineto{\pgfqpoint{4.805560in}{2.314099in}}%
\pgfpathlineto{\pgfqpoint{4.809280in}{2.313048in}}%
\pgfpathlineto{\pgfqpoint{4.810520in}{2.311394in}}%
\pgfpathlineto{\pgfqpoint{4.813000in}{2.314651in}}%
\pgfpathlineto{\pgfqpoint{4.815480in}{2.318106in}}%
\pgfpathlineto{\pgfqpoint{4.817960in}{2.325096in}}%
\pgfpathlineto{\pgfqpoint{4.819200in}{2.325946in}}%
\pgfpathlineto{\pgfqpoint{4.821680in}{2.321094in}}%
\pgfpathlineto{\pgfqpoint{4.822920in}{2.319059in}}%
\pgfpathlineto{\pgfqpoint{4.824160in}{2.319614in}}%
\pgfpathlineto{\pgfqpoint{4.829120in}{2.312478in}}%
\pgfpathlineto{\pgfqpoint{4.834080in}{2.309165in}}%
\pgfpathlineto{\pgfqpoint{4.835320in}{2.308153in}}%
\pgfpathlineto{\pgfqpoint{4.836560in}{2.305659in}}%
\pgfpathlineto{\pgfqpoint{4.837800in}{2.306050in}}%
\pgfpathlineto{\pgfqpoint{4.840280in}{2.312886in}}%
\pgfpathlineto{\pgfqpoint{4.844000in}{2.305210in}}%
\pgfpathlineto{\pgfqpoint{4.845240in}{2.302228in}}%
\pgfpathlineto{\pgfqpoint{4.847720in}{2.303313in}}%
\pgfpathlineto{\pgfqpoint{4.850200in}{2.301907in}}%
\pgfpathlineto{\pgfqpoint{4.851440in}{2.299740in}}%
\pgfpathlineto{\pgfqpoint{4.855160in}{2.308836in}}%
\pgfpathlineto{\pgfqpoint{4.856400in}{2.309160in}}%
\pgfpathlineto{\pgfqpoint{4.858880in}{2.316055in}}%
\pgfpathlineto{\pgfqpoint{4.860120in}{2.314511in}}%
\pgfpathlineto{\pgfqpoint{4.861360in}{2.314512in}}%
\pgfpathlineto{\pgfqpoint{4.863840in}{2.311624in}}%
\pgfpathlineto{\pgfqpoint{4.865080in}{2.311241in}}%
\pgfpathlineto{\pgfqpoint{4.867560in}{2.312647in}}%
\pgfpathlineto{\pgfqpoint{4.871280in}{2.307565in}}%
\pgfpathlineto{\pgfqpoint{4.872520in}{2.307818in}}%
\pgfpathlineto{\pgfqpoint{4.873760in}{2.309645in}}%
\pgfpathlineto{\pgfqpoint{4.877480in}{2.309091in}}%
\pgfpathlineto{\pgfqpoint{4.878720in}{2.310537in}}%
\pgfpathlineto{\pgfqpoint{4.883680in}{2.308666in}}%
\pgfpathlineto{\pgfqpoint{4.886160in}{2.307234in}}%
\pgfpathlineto{\pgfqpoint{4.888640in}{2.313320in}}%
\pgfpathlineto{\pgfqpoint{4.889880in}{2.315124in}}%
\pgfpathlineto{\pgfqpoint{4.892360in}{2.309713in}}%
\pgfpathlineto{\pgfqpoint{4.897320in}{2.320700in}}%
\pgfpathlineto{\pgfqpoint{4.899800in}{2.321317in}}%
\pgfpathlineto{\pgfqpoint{4.903520in}{2.317359in}}%
\pgfpathlineto{\pgfqpoint{4.907240in}{2.306722in}}%
\pgfpathlineto{\pgfqpoint{4.910960in}{2.307587in}}%
\pgfpathlineto{\pgfqpoint{4.913440in}{2.308870in}}%
\pgfpathlineto{\pgfqpoint{4.914680in}{2.310683in}}%
\pgfpathlineto{\pgfqpoint{4.917160in}{2.308467in}}%
\pgfpathlineto{\pgfqpoint{4.919640in}{2.311529in}}%
\pgfpathlineto{\pgfqpoint{4.922120in}{2.309767in}}%
\pgfpathlineto{\pgfqpoint{4.925840in}{2.311372in}}%
\pgfpathlineto{\pgfqpoint{4.928320in}{2.315431in}}%
\pgfpathlineto{\pgfqpoint{4.930800in}{2.313864in}}%
\pgfpathlineto{\pgfqpoint{4.932040in}{2.315000in}}%
\pgfpathlineto{\pgfqpoint{4.934520in}{2.312422in}}%
\pgfpathlineto{\pgfqpoint{4.937000in}{2.316722in}}%
\pgfpathlineto{\pgfqpoint{4.939480in}{2.318820in}}%
\pgfpathlineto{\pgfqpoint{4.941960in}{2.326467in}}%
\pgfpathlineto{\pgfqpoint{4.943200in}{2.327525in}}%
\pgfpathlineto{\pgfqpoint{4.944440in}{2.326030in}}%
\pgfpathlineto{\pgfqpoint{4.946920in}{2.320344in}}%
\pgfpathlineto{\pgfqpoint{4.948160in}{2.320908in}}%
\pgfpathlineto{\pgfqpoint{4.951880in}{2.315382in}}%
\pgfpathlineto{\pgfqpoint{4.955600in}{2.313183in}}%
\pgfpathlineto{\pgfqpoint{4.959320in}{2.310711in}}%
\pgfpathlineto{\pgfqpoint{4.960560in}{2.308036in}}%
\pgfpathlineto{\pgfqpoint{4.961800in}{2.308221in}}%
\pgfpathlineto{\pgfqpoint{4.964280in}{2.314577in}}%
\pgfpathlineto{\pgfqpoint{4.966760in}{2.309303in}}%
\pgfpathlineto{\pgfqpoint{4.969240in}{2.304282in}}%
\pgfpathlineto{\pgfqpoint{4.971720in}{2.305852in}}%
\pgfpathlineto{\pgfqpoint{4.974200in}{2.304095in}}%
\pgfpathlineto{\pgfqpoint{4.975440in}{2.301347in}}%
\pgfpathlineto{\pgfqpoint{4.979160in}{2.308982in}}%
\pgfpathlineto{\pgfqpoint{4.980400in}{2.309486in}}%
\pgfpathlineto{\pgfqpoint{4.982880in}{2.315295in}}%
\pgfpathlineto{\pgfqpoint{4.984120in}{2.313760in}}%
\pgfpathlineto{\pgfqpoint{4.985360in}{2.315143in}}%
\pgfpathlineto{\pgfqpoint{4.989080in}{2.312395in}}%
\pgfpathlineto{\pgfqpoint{4.991560in}{2.313218in}}%
\pgfpathlineto{\pgfqpoint{4.995280in}{2.307441in}}%
\pgfpathlineto{\pgfqpoint{5.001480in}{2.306488in}}%
\pgfpathlineto{\pgfqpoint{5.003960in}{2.308359in}}%
\pgfpathlineto{\pgfqpoint{5.005200in}{2.308867in}}%
\pgfpathlineto{\pgfqpoint{5.010160in}{2.306332in}}%
\pgfpathlineto{\pgfqpoint{5.013880in}{2.314283in}}%
\pgfpathlineto{\pgfqpoint{5.016360in}{2.308428in}}%
\pgfpathlineto{\pgfqpoint{5.020080in}{2.320338in}}%
\pgfpathlineto{\pgfqpoint{5.025040in}{2.318645in}}%
\pgfpathlineto{\pgfqpoint{5.028760in}{2.311274in}}%
\pgfpathlineto{\pgfqpoint{5.031240in}{2.305793in}}%
\pgfpathlineto{\pgfqpoint{5.033720in}{2.307268in}}%
\pgfpathlineto{\pgfqpoint{5.034960in}{2.306953in}}%
\pgfpathlineto{\pgfqpoint{5.038680in}{2.312697in}}%
\pgfpathlineto{\pgfqpoint{5.041160in}{2.310506in}}%
\pgfpathlineto{\pgfqpoint{5.043640in}{2.313978in}}%
\pgfpathlineto{\pgfqpoint{5.046120in}{2.311313in}}%
\pgfpathlineto{\pgfqpoint{5.047360in}{2.311061in}}%
\pgfpathlineto{\pgfqpoint{5.048600in}{2.312207in}}%
\pgfpathlineto{\pgfqpoint{5.049840in}{2.311638in}}%
\pgfpathlineto{\pgfqpoint{5.052320in}{2.316848in}}%
\pgfpathlineto{\pgfqpoint{5.054800in}{2.316343in}}%
\pgfpathlineto{\pgfqpoint{5.056040in}{2.317580in}}%
\pgfpathlineto{\pgfqpoint{5.058520in}{2.314984in}}%
\pgfpathlineto{\pgfqpoint{5.059760in}{2.317741in}}%
\pgfpathlineto{\pgfqpoint{5.062240in}{2.317775in}}%
\pgfpathlineto{\pgfqpoint{5.063480in}{2.319294in}}%
\pgfpathlineto{\pgfqpoint{5.067200in}{2.329065in}}%
\pgfpathlineto{\pgfqpoint{5.069680in}{2.324510in}}%
\pgfpathlineto{\pgfqpoint{5.070920in}{2.322613in}}%
\pgfpathlineto{\pgfqpoint{5.072160in}{2.323749in}}%
\pgfpathlineto{\pgfqpoint{5.075880in}{2.318572in}}%
\pgfpathlineto{\pgfqpoint{5.078360in}{2.317123in}}%
\pgfpathlineto{\pgfqpoint{5.079600in}{2.318038in}}%
\pgfpathlineto{\pgfqpoint{5.083320in}{2.315615in}}%
\pgfpathlineto{\pgfqpoint{5.085800in}{2.312313in}}%
\pgfpathlineto{\pgfqpoint{5.088280in}{2.317760in}}%
\pgfpathlineto{\pgfqpoint{5.090760in}{2.313266in}}%
\pgfpathlineto{\pgfqpoint{5.092000in}{2.308680in}}%
\pgfpathlineto{\pgfqpoint{5.093240in}{2.308370in}}%
\pgfpathlineto{\pgfqpoint{5.094480in}{2.309661in}}%
\pgfpathlineto{\pgfqpoint{5.098200in}{2.307505in}}%
\pgfpathlineto{\pgfqpoint{5.099440in}{2.304278in}}%
\pgfpathlineto{\pgfqpoint{5.103160in}{2.311501in}}%
\pgfpathlineto{\pgfqpoint{5.104400in}{2.311723in}}%
\pgfpathlineto{\pgfqpoint{5.106880in}{2.319400in}}%
\pgfpathlineto{\pgfqpoint{5.108120in}{2.318586in}}%
\pgfpathlineto{\pgfqpoint{5.109360in}{2.320527in}}%
\pgfpathlineto{\pgfqpoint{5.114320in}{2.317495in}}%
\pgfpathlineto{\pgfqpoint{5.115560in}{2.317652in}}%
\pgfpathlineto{\pgfqpoint{5.118040in}{2.314437in}}%
\pgfpathlineto{\pgfqpoint{5.120520in}{2.312163in}}%
\pgfpathlineto{\pgfqpoint{5.125480in}{2.312422in}}%
\pgfpathlineto{\pgfqpoint{5.127960in}{2.314202in}}%
\pgfpathlineto{\pgfqpoint{5.130440in}{2.315109in}}%
\pgfpathlineto{\pgfqpoint{5.134160in}{2.314958in}}%
\pgfpathlineto{\pgfqpoint{5.137880in}{2.322919in}}%
\pgfpathlineto{\pgfqpoint{5.140360in}{2.317513in}}%
\pgfpathlineto{\pgfqpoint{5.144080in}{2.326297in}}%
\pgfpathlineto{\pgfqpoint{5.149040in}{2.324523in}}%
\pgfpathlineto{\pgfqpoint{5.152760in}{2.317491in}}%
\pgfpathlineto{\pgfqpoint{5.155240in}{2.311241in}}%
\pgfpathlineto{\pgfqpoint{5.158960in}{2.312362in}}%
\pgfpathlineto{\pgfqpoint{5.162680in}{2.317528in}}%
\pgfpathlineto{\pgfqpoint{5.165160in}{2.315863in}}%
\pgfpathlineto{\pgfqpoint{5.167640in}{2.319013in}}%
\pgfpathlineto{\pgfqpoint{5.170120in}{2.315622in}}%
\pgfpathlineto{\pgfqpoint{5.171360in}{2.315130in}}%
\pgfpathlineto{\pgfqpoint{5.172600in}{2.316025in}}%
\pgfpathlineto{\pgfqpoint{5.173840in}{2.315472in}}%
\pgfpathlineto{\pgfqpoint{5.176320in}{2.321202in}}%
\pgfpathlineto{\pgfqpoint{5.177560in}{2.320278in}}%
\pgfpathlineto{\pgfqpoint{5.178800in}{2.320032in}}%
\pgfpathlineto{\pgfqpoint{5.180040in}{2.321878in}}%
\pgfpathlineto{\pgfqpoint{5.182520in}{2.318310in}}%
\pgfpathlineto{\pgfqpoint{5.183760in}{2.321385in}}%
\pgfpathlineto{\pgfqpoint{5.185000in}{2.321108in}}%
\pgfpathlineto{\pgfqpoint{5.187480in}{2.324209in}}%
\pgfpathlineto{\pgfqpoint{5.191200in}{2.333686in}}%
\pgfpathlineto{\pgfqpoint{5.194920in}{2.327298in}}%
\pgfpathlineto{\pgfqpoint{5.196160in}{2.329043in}}%
\pgfpathlineto{\pgfqpoint{5.198640in}{2.323525in}}%
\pgfpathlineto{\pgfqpoint{5.199880in}{2.323187in}}%
\pgfpathlineto{\pgfqpoint{5.202360in}{2.320829in}}%
\pgfpathlineto{\pgfqpoint{5.206080in}{2.320279in}}%
\pgfpathlineto{\pgfqpoint{5.209800in}{2.315055in}}%
\pgfpathlineto{\pgfqpoint{5.212280in}{2.320431in}}%
\pgfpathlineto{\pgfqpoint{5.213520in}{2.319577in}}%
\pgfpathlineto{\pgfqpoint{5.216000in}{2.311952in}}%
\pgfpathlineto{\pgfqpoint{5.218480in}{2.318938in}}%
\pgfpathlineto{\pgfqpoint{5.219720in}{2.318816in}}%
\pgfpathlineto{\pgfqpoint{5.222200in}{2.315528in}}%
\pgfpathlineto{\pgfqpoint{5.223440in}{2.311784in}}%
\pgfpathlineto{\pgfqpoint{5.227160in}{2.320583in}}%
\pgfpathlineto{\pgfqpoint{5.228400in}{2.319914in}}%
\pgfpathlineto{\pgfqpoint{5.233360in}{2.327448in}}%
\pgfpathlineto{\pgfqpoint{5.237080in}{2.324437in}}%
\pgfpathlineto{\pgfqpoint{5.239560in}{2.326406in}}%
\pgfpathlineto{\pgfqpoint{5.242040in}{2.322886in}}%
\pgfpathlineto{\pgfqpoint{5.244520in}{2.321340in}}%
\pgfpathlineto{\pgfqpoint{5.248240in}{2.321611in}}%
\pgfpathlineto{\pgfqpoint{5.251960in}{2.325179in}}%
\pgfpathlineto{\pgfqpoint{5.258160in}{2.326380in}}%
\pgfpathlineto{\pgfqpoint{5.261880in}{2.335427in}}%
\pgfpathlineto{\pgfqpoint{5.264360in}{2.328452in}}%
\pgfpathlineto{\pgfqpoint{5.268080in}{2.335164in}}%
\pgfpathlineto{\pgfqpoint{5.274280in}{2.332404in}}%
\pgfpathlineto{\pgfqpoint{5.275520in}{2.331048in}}%
\pgfpathlineto{\pgfqpoint{5.279240in}{2.319379in}}%
\pgfpathlineto{\pgfqpoint{5.282960in}{2.321391in}}%
\pgfpathlineto{\pgfqpoint{5.286680in}{2.326550in}}%
\pgfpathlineto{\pgfqpoint{5.289160in}{2.325257in}}%
\pgfpathlineto{\pgfqpoint{5.291640in}{2.328221in}}%
\pgfpathlineto{\pgfqpoint{5.294120in}{2.322556in}}%
\pgfpathlineto{\pgfqpoint{5.295360in}{2.321807in}}%
\pgfpathlineto{\pgfqpoint{5.296600in}{2.323124in}}%
\pgfpathlineto{\pgfqpoint{5.297840in}{2.322674in}}%
\pgfpathlineto{\pgfqpoint{5.300320in}{2.327607in}}%
\pgfpathlineto{\pgfqpoint{5.302800in}{2.325989in}}%
\pgfpathlineto{\pgfqpoint{5.304040in}{2.328049in}}%
\pgfpathlineto{\pgfqpoint{5.306520in}{2.323476in}}%
\pgfpathlineto{\pgfqpoint{5.307760in}{2.325923in}}%
\pgfpathlineto{\pgfqpoint{5.309000in}{2.325205in}}%
\pgfpathlineto{\pgfqpoint{5.311480in}{2.327707in}}%
\pgfpathlineto{\pgfqpoint{5.315200in}{2.337088in}}%
\pgfpathlineto{\pgfqpoint{5.318920in}{2.332608in}}%
\pgfpathlineto{\pgfqpoint{5.320160in}{2.335194in}}%
\pgfpathlineto{\pgfqpoint{5.322640in}{2.331169in}}%
\pgfpathlineto{\pgfqpoint{5.325120in}{2.328665in}}%
\pgfpathlineto{\pgfqpoint{5.328840in}{2.326490in}}%
\pgfpathlineto{\pgfqpoint{5.330080in}{2.326280in}}%
\pgfpathlineto{\pgfqpoint{5.333800in}{2.321617in}}%
\pgfpathlineto{\pgfqpoint{5.336280in}{2.325708in}}%
\pgfpathlineto{\pgfqpoint{5.337520in}{2.324669in}}%
\pgfpathlineto{\pgfqpoint{5.340000in}{2.318296in}}%
\pgfpathlineto{\pgfqpoint{5.342480in}{2.322384in}}%
\pgfpathlineto{\pgfqpoint{5.344960in}{2.322057in}}%
\pgfpathlineto{\pgfqpoint{5.347440in}{2.315164in}}%
\pgfpathlineto{\pgfqpoint{5.351160in}{2.323680in}}%
\pgfpathlineto{\pgfqpoint{5.352400in}{2.323384in}}%
\pgfpathlineto{\pgfqpoint{5.354880in}{2.330942in}}%
\pgfpathlineto{\pgfqpoint{5.356120in}{2.330407in}}%
\pgfpathlineto{\pgfqpoint{5.357360in}{2.331821in}}%
\pgfpathlineto{\pgfqpoint{5.361080in}{2.326651in}}%
\pgfpathlineto{\pgfqpoint{5.363560in}{2.327119in}}%
\pgfpathlineto{\pgfqpoint{5.364800in}{2.324819in}}%
\pgfpathlineto{\pgfqpoint{5.366040in}{2.325142in}}%
\pgfpathlineto{\pgfqpoint{5.367280in}{2.322961in}}%
\pgfpathlineto{\pgfqpoint{5.369760in}{2.322931in}}%
\pgfpathlineto{\pgfqpoint{5.372240in}{2.322664in}}%
\pgfpathlineto{\pgfqpoint{5.378440in}{2.328917in}}%
\pgfpathlineto{\pgfqpoint{5.380920in}{2.330116in}}%
\pgfpathlineto{\pgfqpoint{5.382160in}{2.330377in}}%
\pgfpathlineto{\pgfqpoint{5.385880in}{2.338079in}}%
\pgfpathlineto{\pgfqpoint{5.388360in}{2.332648in}}%
\pgfpathlineto{\pgfqpoint{5.389600in}{2.335692in}}%
\pgfpathlineto{\pgfqpoint{5.390840in}{2.335453in}}%
\pgfpathlineto{\pgfqpoint{5.392080in}{2.340016in}}%
\pgfpathlineto{\pgfqpoint{5.397040in}{2.339175in}}%
\pgfpathlineto{\pgfqpoint{5.399520in}{2.337206in}}%
\pgfpathlineto{\pgfqpoint{5.403240in}{2.325364in}}%
\pgfpathlineto{\pgfqpoint{5.405720in}{2.326587in}}%
\pgfpathlineto{\pgfqpoint{5.408200in}{2.328509in}}%
\pgfpathlineto{\pgfqpoint{5.410680in}{2.331158in}}%
\pgfpathlineto{\pgfqpoint{5.413160in}{2.327568in}}%
\pgfpathlineto{\pgfqpoint{5.415640in}{2.331524in}}%
\pgfpathlineto{\pgfqpoint{5.419360in}{2.324997in}}%
\pgfpathlineto{\pgfqpoint{5.420600in}{2.325781in}}%
\pgfpathlineto{\pgfqpoint{5.421840in}{2.324180in}}%
\pgfpathlineto{\pgfqpoint{5.425560in}{2.328913in}}%
\pgfpathlineto{\pgfqpoint{5.426800in}{2.328446in}}%
\pgfpathlineto{\pgfqpoint{5.428040in}{2.330680in}}%
\pgfpathlineto{\pgfqpoint{5.430520in}{2.325537in}}%
\pgfpathlineto{\pgfqpoint{5.433000in}{2.327709in}}%
\pgfpathlineto{\pgfqpoint{5.435480in}{2.330518in}}%
\pgfpathlineto{\pgfqpoint{5.439200in}{2.339112in}}%
\pgfpathlineto{\pgfqpoint{5.442920in}{2.334089in}}%
\pgfpathlineto{\pgfqpoint{5.444160in}{2.337037in}}%
\pgfpathlineto{\pgfqpoint{5.447880in}{2.331588in}}%
\pgfpathlineto{\pgfqpoint{5.450360in}{2.327999in}}%
\pgfpathlineto{\pgfqpoint{5.455320in}{2.326027in}}%
\pgfpathlineto{\pgfqpoint{5.457800in}{2.322332in}}%
\pgfpathlineto{\pgfqpoint{5.461520in}{2.325789in}}%
\pgfpathlineto{\pgfqpoint{5.464000in}{2.320947in}}%
\pgfpathlineto{\pgfqpoint{5.466480in}{2.326071in}}%
\pgfpathlineto{\pgfqpoint{5.467720in}{2.326674in}}%
\pgfpathlineto{\pgfqpoint{5.468960in}{2.325665in}}%
\pgfpathlineto{\pgfqpoint{5.471440in}{2.318271in}}%
\pgfpathlineto{\pgfqpoint{5.475160in}{2.326520in}}%
\pgfpathlineto{\pgfqpoint{5.476400in}{2.326805in}}%
\pgfpathlineto{\pgfqpoint{5.477640in}{2.328842in}}%
\pgfpathlineto{\pgfqpoint{5.478880in}{2.333922in}}%
\pgfpathlineto{\pgfqpoint{5.480120in}{2.333775in}}%
\pgfpathlineto{\pgfqpoint{5.481360in}{2.335378in}}%
\pgfpathlineto{\pgfqpoint{5.483840in}{2.330007in}}%
\pgfpathlineto{\pgfqpoint{5.485080in}{2.328284in}}%
\pgfpathlineto{\pgfqpoint{5.487560in}{2.328824in}}%
\pgfpathlineto{\pgfqpoint{5.488800in}{2.326558in}}%
\pgfpathlineto{\pgfqpoint{5.490040in}{2.326872in}}%
\pgfpathlineto{\pgfqpoint{5.491280in}{2.325037in}}%
\pgfpathlineto{\pgfqpoint{5.492520in}{2.326353in}}%
\pgfpathlineto{\pgfqpoint{5.496240in}{2.324603in}}%
\pgfpathlineto{\pgfqpoint{5.498720in}{2.328117in}}%
\pgfpathlineto{\pgfqpoint{5.499960in}{2.327257in}}%
\pgfpathlineto{\pgfqpoint{5.502440in}{2.328339in}}%
\pgfpathlineto{\pgfqpoint{5.506160in}{2.327248in}}%
\pgfpathlineto{\pgfqpoint{5.509880in}{2.333829in}}%
\pgfpathlineto{\pgfqpoint{5.512360in}{2.329290in}}%
\pgfpathlineto{\pgfqpoint{5.513600in}{2.332194in}}%
\pgfpathlineto{\pgfqpoint{5.514840in}{2.331450in}}%
\pgfpathlineto{\pgfqpoint{5.517320in}{2.336454in}}%
\pgfpathlineto{\pgfqpoint{5.518560in}{2.336080in}}%
\pgfpathlineto{\pgfqpoint{5.521040in}{2.333710in}}%
\pgfpathlineto{\pgfqpoint{5.523520in}{2.332878in}}%
\pgfpathlineto{\pgfqpoint{5.527240in}{2.322813in}}%
\pgfpathlineto{\pgfqpoint{5.528480in}{2.322614in}}%
\pgfpathlineto{\pgfqpoint{5.534680in}{2.330522in}}%
\pgfpathlineto{\pgfqpoint{5.537160in}{2.326294in}}%
\pgfpathlineto{\pgfqpoint{5.539640in}{2.329718in}}%
\pgfpathlineto{\pgfqpoint{5.543360in}{2.322974in}}%
\pgfpathlineto{\pgfqpoint{5.544600in}{2.323087in}}%
\pgfpathlineto{\pgfqpoint{5.545840in}{2.320908in}}%
\pgfpathlineto{\pgfqpoint{5.549560in}{2.326532in}}%
\pgfpathlineto{\pgfqpoint{5.550800in}{2.325570in}}%
\pgfpathlineto{\pgfqpoint{5.552040in}{2.327207in}}%
\pgfpathlineto{\pgfqpoint{5.554520in}{2.321302in}}%
\pgfpathlineto{\pgfqpoint{5.557000in}{2.323954in}}%
\pgfpathlineto{\pgfqpoint{5.558240in}{2.324094in}}%
\pgfpathlineto{\pgfqpoint{5.559480in}{2.325673in}}%
\pgfpathlineto{\pgfqpoint{5.561960in}{2.331899in}}%
\pgfpathlineto{\pgfqpoint{5.563200in}{2.332826in}}%
\pgfpathlineto{\pgfqpoint{5.566920in}{2.330000in}}%
\pgfpathlineto{\pgfqpoint{5.568160in}{2.332672in}}%
\pgfpathlineto{\pgfqpoint{5.570640in}{2.328192in}}%
\pgfpathlineto{\pgfqpoint{5.571880in}{2.327205in}}%
\pgfpathlineto{\pgfqpoint{5.574360in}{2.324503in}}%
\pgfpathlineto{\pgfqpoint{5.578080in}{2.324050in}}%
\pgfpathlineto{\pgfqpoint{5.579320in}{2.323066in}}%
\pgfpathlineto{\pgfqpoint{5.581800in}{2.320146in}}%
\pgfpathlineto{\pgfqpoint{5.584280in}{2.323540in}}%
\pgfpathlineto{\pgfqpoint{5.585520in}{2.322389in}}%
\pgfpathlineto{\pgfqpoint{5.588000in}{2.316306in}}%
\pgfpathlineto{\pgfqpoint{5.589240in}{2.316616in}}%
\pgfpathlineto{\pgfqpoint{5.591720in}{2.320034in}}%
\pgfpathlineto{\pgfqpoint{5.592960in}{2.318830in}}%
\pgfpathlineto{\pgfqpoint{5.595440in}{2.310968in}}%
\pgfpathlineto{\pgfqpoint{5.599160in}{2.318861in}}%
\pgfpathlineto{\pgfqpoint{5.601640in}{2.321544in}}%
\pgfpathlineto{\pgfqpoint{5.602880in}{2.326656in}}%
\pgfpathlineto{\pgfqpoint{5.604120in}{2.326571in}}%
\pgfpathlineto{\pgfqpoint{5.605360in}{2.328663in}}%
\pgfpathlineto{\pgfqpoint{5.607840in}{2.323189in}}%
\pgfpathlineto{\pgfqpoint{5.609080in}{2.322184in}}%
\pgfpathlineto{\pgfqpoint{5.611560in}{2.322490in}}%
\pgfpathlineto{\pgfqpoint{5.615280in}{2.317278in}}%
\pgfpathlineto{\pgfqpoint{5.616520in}{2.318058in}}%
\pgfpathlineto{\pgfqpoint{5.620240in}{2.316001in}}%
\pgfpathlineto{\pgfqpoint{5.622720in}{2.320871in}}%
\pgfpathlineto{\pgfqpoint{5.623960in}{2.321149in}}%
\pgfpathlineto{\pgfqpoint{5.626440in}{2.323233in}}%
\pgfpathlineto{\pgfqpoint{5.628920in}{2.322982in}}%
\pgfpathlineto{\pgfqpoint{5.633880in}{2.327724in}}%
\pgfpathlineto{\pgfqpoint{5.636360in}{2.322645in}}%
\pgfpathlineto{\pgfqpoint{5.637600in}{2.325750in}}%
\pgfpathlineto{\pgfqpoint{5.638840in}{2.325055in}}%
\pgfpathlineto{\pgfqpoint{5.640080in}{2.331068in}}%
\pgfpathlineto{\pgfqpoint{5.641320in}{2.331259in}}%
\pgfpathlineto{\pgfqpoint{5.645040in}{2.327402in}}%
\pgfpathlineto{\pgfqpoint{5.647520in}{2.326679in}}%
\pgfpathlineto{\pgfqpoint{5.651240in}{2.317274in}}%
\pgfpathlineto{\pgfqpoint{5.652480in}{2.316418in}}%
\pgfpathlineto{\pgfqpoint{5.658680in}{2.324428in}}%
\pgfpathlineto{\pgfqpoint{5.661160in}{2.319736in}}%
\pgfpathlineto{\pgfqpoint{5.663640in}{2.324038in}}%
\pgfpathlineto{\pgfqpoint{5.667360in}{2.316651in}}%
\pgfpathlineto{\pgfqpoint{5.668600in}{2.318029in}}%
\pgfpathlineto{\pgfqpoint{5.669840in}{2.316378in}}%
\pgfpathlineto{\pgfqpoint{5.673560in}{2.321223in}}%
\pgfpathlineto{\pgfqpoint{5.674800in}{2.318912in}}%
\pgfpathlineto{\pgfqpoint{5.676040in}{2.319958in}}%
\pgfpathlineto{\pgfqpoint{5.678520in}{2.313471in}}%
\pgfpathlineto{\pgfqpoint{5.681000in}{2.318118in}}%
\pgfpathlineto{\pgfqpoint{5.683480in}{2.319656in}}%
\pgfpathlineto{\pgfqpoint{5.685960in}{2.325914in}}%
\pgfpathlineto{\pgfqpoint{5.688440in}{2.325883in}}%
\pgfpathlineto{\pgfqpoint{5.690920in}{2.323660in}}%
\pgfpathlineto{\pgfqpoint{5.692160in}{2.325331in}}%
\pgfpathlineto{\pgfqpoint{5.694640in}{2.319165in}}%
\pgfpathlineto{\pgfqpoint{5.695880in}{2.318481in}}%
\pgfpathlineto{\pgfqpoint{5.698360in}{2.314760in}}%
\pgfpathlineto{\pgfqpoint{5.703320in}{2.312524in}}%
\pgfpathlineto{\pgfqpoint{5.704560in}{2.309628in}}%
\pgfpathlineto{\pgfqpoint{5.705800in}{2.310068in}}%
\pgfpathlineto{\pgfqpoint{5.708280in}{2.313297in}}%
\pgfpathlineto{\pgfqpoint{5.709520in}{2.313162in}}%
\pgfpathlineto{\pgfqpoint{5.712000in}{2.308526in}}%
\pgfpathlineto{\pgfqpoint{5.714480in}{2.315815in}}%
\pgfpathlineto{\pgfqpoint{5.715720in}{2.318325in}}%
\pgfpathlineto{\pgfqpoint{5.716960in}{2.317126in}}%
\pgfpathlineto{\pgfqpoint{5.719440in}{2.308245in}}%
\pgfpathlineto{\pgfqpoint{5.723160in}{2.316559in}}%
\pgfpathlineto{\pgfqpoint{5.724400in}{2.316124in}}%
\pgfpathlineto{\pgfqpoint{5.725640in}{2.317704in}}%
\pgfpathlineto{\pgfqpoint{5.726880in}{2.322628in}}%
\pgfpathlineto{\pgfqpoint{5.729360in}{2.321213in}}%
\pgfpathlineto{\pgfqpoint{5.731840in}{2.315642in}}%
\pgfpathlineto{\pgfqpoint{5.733080in}{2.315969in}}%
\pgfpathlineto{\pgfqpoint{5.738040in}{2.311794in}}%
\pgfpathlineto{\pgfqpoint{5.739280in}{2.309951in}}%
\pgfpathlineto{\pgfqpoint{5.740520in}{2.311256in}}%
\pgfpathlineto{\pgfqpoint{5.744240in}{2.310260in}}%
\pgfpathlineto{\pgfqpoint{5.747960in}{2.317960in}}%
\pgfpathlineto{\pgfqpoint{5.750440in}{2.319531in}}%
\pgfpathlineto{\pgfqpoint{5.751680in}{2.319261in}}%
\pgfpathlineto{\pgfqpoint{5.752920in}{2.317719in}}%
\pgfpathlineto{\pgfqpoint{5.757880in}{2.322533in}}%
\pgfpathlineto{\pgfqpoint{5.760360in}{2.317048in}}%
\pgfpathlineto{\pgfqpoint{5.761600in}{2.318264in}}%
\pgfpathlineto{\pgfqpoint{5.762840in}{2.316887in}}%
\pgfpathlineto{\pgfqpoint{5.765320in}{2.324688in}}%
\pgfpathlineto{\pgfqpoint{5.767800in}{2.321935in}}%
\pgfpathlineto{\pgfqpoint{5.771520in}{2.323766in}}%
\pgfpathlineto{\pgfqpoint{5.775240in}{2.313100in}}%
\pgfpathlineto{\pgfqpoint{5.776480in}{2.312832in}}%
\pgfpathlineto{\pgfqpoint{5.777720in}{2.315095in}}%
\pgfpathlineto{\pgfqpoint{5.778960in}{2.314451in}}%
\pgfpathlineto{\pgfqpoint{5.782680in}{2.320989in}}%
\pgfpathlineto{\pgfqpoint{5.785160in}{2.316528in}}%
\pgfpathlineto{\pgfqpoint{5.787640in}{2.321149in}}%
\pgfpathlineto{\pgfqpoint{5.791360in}{2.311299in}}%
\pgfpathlineto{\pgfqpoint{5.792600in}{2.312509in}}%
\pgfpathlineto{\pgfqpoint{5.793840in}{2.310123in}}%
\pgfpathlineto{\pgfqpoint{5.797560in}{2.314929in}}%
\pgfpathlineto{\pgfqpoint{5.798800in}{2.313216in}}%
\pgfpathlineto{\pgfqpoint{5.800040in}{2.313485in}}%
\pgfpathlineto{\pgfqpoint{5.802520in}{2.306715in}}%
\pgfpathlineto{\pgfqpoint{5.803760in}{2.310573in}}%
\pgfpathlineto{\pgfqpoint{5.807480in}{2.312105in}}%
\pgfpathlineto{\pgfqpoint{5.811200in}{2.319969in}}%
\pgfpathlineto{\pgfqpoint{5.814920in}{2.315857in}}%
\pgfpathlineto{\pgfqpoint{5.816160in}{2.318341in}}%
\pgfpathlineto{\pgfqpoint{5.821120in}{2.308794in}}%
\pgfpathlineto{\pgfqpoint{5.829800in}{2.304715in}}%
\pgfpathlineto{\pgfqpoint{5.832280in}{2.308821in}}%
\pgfpathlineto{\pgfqpoint{5.833520in}{2.309489in}}%
\pgfpathlineto{\pgfqpoint{5.836000in}{2.303884in}}%
\pgfpathlineto{\pgfqpoint{5.838480in}{2.312025in}}%
\pgfpathlineto{\pgfqpoint{5.839720in}{2.314525in}}%
\pgfpathlineto{\pgfqpoint{5.840960in}{2.312933in}}%
\pgfpathlineto{\pgfqpoint{5.843440in}{2.303750in}}%
\pgfpathlineto{\pgfqpoint{5.847160in}{2.312942in}}%
\pgfpathlineto{\pgfqpoint{5.848400in}{2.311512in}}%
\pgfpathlineto{\pgfqpoint{5.849640in}{2.312442in}}%
\pgfpathlineto{\pgfqpoint{5.850880in}{2.317587in}}%
\pgfpathlineto{\pgfqpoint{5.853360in}{2.315283in}}%
\pgfpathlineto{\pgfqpoint{5.855840in}{2.309778in}}%
\pgfpathlineto{\pgfqpoint{5.858320in}{2.310206in}}%
\pgfpathlineto{\pgfqpoint{5.859560in}{2.309468in}}%
\pgfpathlineto{\pgfqpoint{5.863280in}{2.303448in}}%
\pgfpathlineto{\pgfqpoint{5.864520in}{2.304457in}}%
\pgfpathlineto{\pgfqpoint{5.867000in}{2.302457in}}%
\pgfpathlineto{\pgfqpoint{5.868240in}{2.303993in}}%
\pgfpathlineto{\pgfqpoint{5.871960in}{2.312241in}}%
\pgfpathlineto{\pgfqpoint{5.874440in}{2.314144in}}%
\pgfpathlineto{\pgfqpoint{5.875680in}{2.313408in}}%
\pgfpathlineto{\pgfqpoint{5.878160in}{2.311170in}}%
\pgfpathlineto{\pgfqpoint{5.881880in}{2.316841in}}%
\pgfpathlineto{\pgfqpoint{5.884360in}{2.311788in}}%
\pgfpathlineto{\pgfqpoint{5.885600in}{2.312974in}}%
\pgfpathlineto{\pgfqpoint{5.886840in}{2.309687in}}%
\pgfpathlineto{\pgfqpoint{5.889320in}{2.315610in}}%
\pgfpathlineto{\pgfqpoint{5.891800in}{2.313220in}}%
\pgfpathlineto{\pgfqpoint{5.894280in}{2.313854in}}%
\pgfpathlineto{\pgfqpoint{5.895520in}{2.313357in}}%
\pgfpathlineto{\pgfqpoint{5.898000in}{2.304959in}}%
\pgfpathlineto{\pgfqpoint{5.900480in}{2.302658in}}%
\pgfpathlineto{\pgfqpoint{5.901720in}{2.304236in}}%
\pgfpathlineto{\pgfqpoint{5.902960in}{2.303978in}}%
\pgfpathlineto{\pgfqpoint{5.906680in}{2.310149in}}%
\pgfpathlineto{\pgfqpoint{5.909160in}{2.305675in}}%
\pgfpathlineto{\pgfqpoint{5.911640in}{2.308877in}}%
\pgfpathlineto{\pgfqpoint{5.914120in}{2.298860in}}%
\pgfpathlineto{\pgfqpoint{5.915360in}{2.297872in}}%
\pgfpathlineto{\pgfqpoint{5.916600in}{2.299630in}}%
\pgfpathlineto{\pgfqpoint{5.917840in}{2.297543in}}%
\pgfpathlineto{\pgfqpoint{5.921560in}{2.301793in}}%
\pgfpathlineto{\pgfqpoint{5.926520in}{2.291867in}}%
\pgfpathlineto{\pgfqpoint{5.929000in}{2.295064in}}%
\pgfpathlineto{\pgfqpoint{5.930240in}{2.295123in}}%
\pgfpathlineto{\pgfqpoint{5.931480in}{2.296302in}}%
\pgfpathlineto{\pgfqpoint{5.935200in}{2.305831in}}%
\pgfpathlineto{\pgfqpoint{5.937680in}{2.304093in}}%
\pgfpathlineto{\pgfqpoint{5.938920in}{2.302535in}}%
\pgfpathlineto{\pgfqpoint{5.940160in}{2.304674in}}%
\pgfpathlineto{\pgfqpoint{5.942640in}{2.298672in}}%
\pgfpathlineto{\pgfqpoint{5.943880in}{2.298964in}}%
\pgfpathlineto{\pgfqpoint{5.947600in}{2.293938in}}%
\pgfpathlineto{\pgfqpoint{5.948840in}{2.293970in}}%
\pgfpathlineto{\pgfqpoint{5.950080in}{2.296653in}}%
\pgfpathlineto{\pgfqpoint{5.953800in}{2.296307in}}%
\pgfpathlineto{\pgfqpoint{5.956280in}{2.301482in}}%
\pgfpathlineto{\pgfqpoint{5.957520in}{2.302473in}}%
\pgfpathlineto{\pgfqpoint{5.960000in}{2.295248in}}%
\pgfpathlineto{\pgfqpoint{5.963720in}{2.302003in}}%
\pgfpathlineto{\pgfqpoint{5.966200in}{2.294891in}}%
\pgfpathlineto{\pgfqpoint{5.967440in}{2.292102in}}%
\pgfpathlineto{\pgfqpoint{5.969920in}{2.297989in}}%
\pgfpathlineto{\pgfqpoint{5.971160in}{2.300737in}}%
\pgfpathlineto{\pgfqpoint{5.973640in}{2.299014in}}%
\pgfpathlineto{\pgfqpoint{5.974880in}{2.303905in}}%
\pgfpathlineto{\pgfqpoint{5.977360in}{2.302131in}}%
\pgfpathlineto{\pgfqpoint{5.979840in}{2.298318in}}%
\pgfpathlineto{\pgfqpoint{5.982320in}{2.299593in}}%
\pgfpathlineto{\pgfqpoint{5.986040in}{2.293136in}}%
\pgfpathlineto{\pgfqpoint{5.987280in}{2.291209in}}%
\pgfpathlineto{\pgfqpoint{5.988520in}{2.292505in}}%
\pgfpathlineto{\pgfqpoint{5.991000in}{2.288345in}}%
\pgfpathlineto{\pgfqpoint{5.993480in}{2.293768in}}%
\pgfpathlineto{\pgfqpoint{5.995960in}{2.297900in}}%
\pgfpathlineto{\pgfqpoint{5.998440in}{2.303391in}}%
\pgfpathlineto{\pgfqpoint{5.999680in}{2.303061in}}%
\pgfpathlineto{\pgfqpoint{6.002160in}{2.299573in}}%
\pgfpathlineto{\pgfqpoint{6.004640in}{2.302384in}}%
\pgfpathlineto{\pgfqpoint{6.005880in}{2.304932in}}%
\pgfpathlineto{\pgfqpoint{6.008360in}{2.299881in}}%
\pgfpathlineto{\pgfqpoint{6.009600in}{2.300939in}}%
\pgfpathlineto{\pgfqpoint{6.010840in}{2.296761in}}%
\pgfpathlineto{\pgfqpoint{6.013320in}{2.303723in}}%
\pgfpathlineto{\pgfqpoint{6.015800in}{2.300937in}}%
\pgfpathlineto{\pgfqpoint{6.018280in}{2.301357in}}%
\pgfpathlineto{\pgfqpoint{6.019520in}{2.299876in}}%
\pgfpathlineto{\pgfqpoint{6.022000in}{2.292358in}}%
\pgfpathlineto{\pgfqpoint{6.024480in}{2.289991in}}%
\pgfpathlineto{\pgfqpoint{6.026960in}{2.290467in}}%
\pgfpathlineto{\pgfqpoint{6.030680in}{2.297237in}}%
\pgfpathlineto{\pgfqpoint{6.034400in}{2.294444in}}%
\pgfpathlineto{\pgfqpoint{6.035640in}{2.293608in}}%
\pgfpathlineto{\pgfqpoint{6.038120in}{2.281395in}}%
\pgfpathlineto{\pgfqpoint{6.039360in}{2.280384in}}%
\pgfpathlineto{\pgfqpoint{6.040600in}{2.282716in}}%
\pgfpathlineto{\pgfqpoint{6.041840in}{2.280557in}}%
\pgfpathlineto{\pgfqpoint{6.045560in}{2.285352in}}%
\pgfpathlineto{\pgfqpoint{6.046800in}{2.283068in}}%
\pgfpathlineto{\pgfqpoint{6.048040in}{2.283586in}}%
\pgfpathlineto{\pgfqpoint{6.050520in}{2.276933in}}%
\pgfpathlineto{\pgfqpoint{6.051760in}{2.280381in}}%
\pgfpathlineto{\pgfqpoint{6.054240in}{2.281469in}}%
\pgfpathlineto{\pgfqpoint{6.055480in}{2.283539in}}%
\pgfpathlineto{\pgfqpoint{6.059200in}{2.295868in}}%
\pgfpathlineto{\pgfqpoint{6.060440in}{2.295899in}}%
\pgfpathlineto{\pgfqpoint{6.062920in}{2.291365in}}%
\pgfpathlineto{\pgfqpoint{6.064160in}{2.294597in}}%
\pgfpathlineto{\pgfqpoint{6.066640in}{2.289559in}}%
\pgfpathlineto{\pgfqpoint{6.067880in}{2.289262in}}%
\pgfpathlineto{\pgfqpoint{6.069120in}{2.285413in}}%
\pgfpathlineto{\pgfqpoint{6.070360in}{2.286005in}}%
\pgfpathlineto{\pgfqpoint{6.072840in}{2.283259in}}%
\pgfpathlineto{\pgfqpoint{6.074080in}{2.285473in}}%
\pgfpathlineto{\pgfqpoint{6.076560in}{2.284679in}}%
\pgfpathlineto{\pgfqpoint{6.081520in}{2.294103in}}%
\pgfpathlineto{\pgfqpoint{6.085240in}{2.285196in}}%
\pgfpathlineto{\pgfqpoint{6.087720in}{2.290040in}}%
\pgfpathlineto{\pgfqpoint{6.091440in}{2.283167in}}%
\pgfpathlineto{\pgfqpoint{6.093920in}{2.287477in}}%
\pgfpathlineto{\pgfqpoint{6.095160in}{2.289327in}}%
\pgfpathlineto{\pgfqpoint{6.097640in}{2.285861in}}%
\pgfpathlineto{\pgfqpoint{6.098880in}{2.290661in}}%
\pgfpathlineto{\pgfqpoint{6.101360in}{2.287853in}}%
\pgfpathlineto{\pgfqpoint{6.103840in}{2.283812in}}%
\pgfpathlineto{\pgfqpoint{6.106320in}{2.283628in}}%
\pgfpathlineto{\pgfqpoint{6.111280in}{2.277299in}}%
\pgfpathlineto{\pgfqpoint{6.112520in}{2.279020in}}%
\pgfpathlineto{\pgfqpoint{6.115000in}{2.276853in}}%
\pgfpathlineto{\pgfqpoint{6.116240in}{2.277165in}}%
\pgfpathlineto{\pgfqpoint{6.118720in}{2.283868in}}%
\pgfpathlineto{\pgfqpoint{6.119960in}{2.284229in}}%
\pgfpathlineto{\pgfqpoint{6.122440in}{2.293208in}}%
\pgfpathlineto{\pgfqpoint{6.123680in}{2.292688in}}%
\pgfpathlineto{\pgfqpoint{6.126160in}{2.289340in}}%
\pgfpathlineto{\pgfqpoint{6.127400in}{2.289648in}}%
\pgfpathlineto{\pgfqpoint{6.129880in}{2.294977in}}%
\pgfpathlineto{\pgfqpoint{6.132360in}{2.289126in}}%
\pgfpathlineto{\pgfqpoint{6.133600in}{2.289209in}}%
\pgfpathlineto{\pgfqpoint{6.134840in}{2.284759in}}%
\pgfpathlineto{\pgfqpoint{6.137320in}{2.291835in}}%
\pgfpathlineto{\pgfqpoint{6.139800in}{2.289172in}}%
\pgfpathlineto{\pgfqpoint{6.142280in}{2.290912in}}%
\pgfpathlineto{\pgfqpoint{6.143520in}{2.289769in}}%
\pgfpathlineto{\pgfqpoint{6.147240in}{2.279773in}}%
\pgfpathlineto{\pgfqpoint{6.149720in}{2.278594in}}%
\pgfpathlineto{\pgfqpoint{6.150960in}{2.279278in}}%
\pgfpathlineto{\pgfqpoint{6.153440in}{2.284937in}}%
\pgfpathlineto{\pgfqpoint{6.154680in}{2.287565in}}%
\pgfpathlineto{\pgfqpoint{6.157160in}{2.283367in}}%
\pgfpathlineto{\pgfqpoint{6.158400in}{2.284697in}}%
\pgfpathlineto{\pgfqpoint{6.159640in}{2.284457in}}%
\pgfpathlineto{\pgfqpoint{6.163360in}{2.267871in}}%
\pgfpathlineto{\pgfqpoint{6.164600in}{2.270124in}}%
\pgfpathlineto{\pgfqpoint{6.165840in}{2.268519in}}%
\pgfpathlineto{\pgfqpoint{6.168320in}{2.271405in}}%
\pgfpathlineto{\pgfqpoint{6.169560in}{2.274227in}}%
\pgfpathlineto{\pgfqpoint{6.172040in}{2.274200in}}%
\pgfpathlineto{\pgfqpoint{6.174520in}{2.265375in}}%
\pgfpathlineto{\pgfqpoint{6.175760in}{2.269239in}}%
\pgfpathlineto{\pgfqpoint{6.178240in}{2.270530in}}%
\pgfpathlineto{\pgfqpoint{6.180720in}{2.276164in}}%
\pgfpathlineto{\pgfqpoint{6.184440in}{2.284281in}}%
\pgfpathlineto{\pgfqpoint{6.185680in}{2.280868in}}%
\pgfpathlineto{\pgfqpoint{6.186920in}{2.281323in}}%
\pgfpathlineto{\pgfqpoint{6.188160in}{2.283514in}}%
\pgfpathlineto{\pgfqpoint{6.191880in}{2.280172in}}%
\pgfpathlineto{\pgfqpoint{6.193120in}{2.275427in}}%
\pgfpathlineto{\pgfqpoint{6.194360in}{2.276552in}}%
\pgfpathlineto{\pgfqpoint{6.196840in}{2.273084in}}%
\pgfpathlineto{\pgfqpoint{6.198080in}{2.274678in}}%
\pgfpathlineto{\pgfqpoint{6.200560in}{2.273390in}}%
\pgfpathlineto{\pgfqpoint{6.201800in}{2.275922in}}%
\pgfpathlineto{\pgfqpoint{6.204280in}{2.282961in}}%
\pgfpathlineto{\pgfqpoint{6.205520in}{2.286058in}}%
\pgfpathlineto{\pgfqpoint{6.209240in}{2.272579in}}%
\pgfpathlineto{\pgfqpoint{6.211720in}{2.278548in}}%
\pgfpathlineto{\pgfqpoint{6.215440in}{2.272044in}}%
\pgfpathlineto{\pgfqpoint{6.217920in}{2.275789in}}%
\pgfpathlineto{\pgfqpoint{6.219160in}{2.277327in}}%
\pgfpathlineto{\pgfqpoint{6.221640in}{2.273678in}}%
\pgfpathlineto{\pgfqpoint{6.222880in}{2.278121in}}%
\pgfpathlineto{\pgfqpoint{6.224120in}{2.277663in}}%
\pgfpathlineto{\pgfqpoint{6.229080in}{2.272662in}}%
\pgfpathlineto{\pgfqpoint{6.230320in}{2.274395in}}%
\pgfpathlineto{\pgfqpoint{6.234040in}{2.267032in}}%
\pgfpathlineto{\pgfqpoint{6.236520in}{2.268948in}}%
\pgfpathlineto{\pgfqpoint{6.237760in}{2.266751in}}%
\pgfpathlineto{\pgfqpoint{6.240240in}{2.267341in}}%
\pgfpathlineto{\pgfqpoint{6.242720in}{2.273547in}}%
\pgfpathlineto{\pgfqpoint{6.243960in}{2.272749in}}%
\pgfpathlineto{\pgfqpoint{6.246440in}{2.279046in}}%
\pgfpathlineto{\pgfqpoint{6.250160in}{2.273540in}}%
\pgfpathlineto{\pgfqpoint{6.251400in}{2.272841in}}%
\pgfpathlineto{\pgfqpoint{6.252640in}{2.273464in}}%
\pgfpathlineto{\pgfqpoint{6.253880in}{2.276118in}}%
\pgfpathlineto{\pgfqpoint{6.256360in}{2.268433in}}%
\pgfpathlineto{\pgfqpoint{6.257600in}{2.270133in}}%
\pgfpathlineto{\pgfqpoint{6.258840in}{2.266746in}}%
\pgfpathlineto{\pgfqpoint{6.261320in}{2.274421in}}%
\pgfpathlineto{\pgfqpoint{6.262560in}{2.271231in}}%
\pgfpathlineto{\pgfqpoint{6.266280in}{2.271656in}}%
\pgfpathlineto{\pgfqpoint{6.273720in}{2.262825in}}%
\pgfpathlineto{\pgfqpoint{6.274960in}{2.263692in}}%
\pgfpathlineto{\pgfqpoint{6.277440in}{2.271223in}}%
\pgfpathlineto{\pgfqpoint{6.279920in}{2.274193in}}%
\pgfpathlineto{\pgfqpoint{6.281160in}{2.271331in}}%
\pgfpathlineto{\pgfqpoint{6.283640in}{2.274153in}}%
\pgfpathlineto{\pgfqpoint{6.287360in}{2.256066in}}%
\pgfpathlineto{\pgfqpoint{6.288600in}{2.258270in}}%
\pgfpathlineto{\pgfqpoint{6.289840in}{2.256607in}}%
\pgfpathlineto{\pgfqpoint{6.291080in}{2.257400in}}%
\pgfpathlineto{\pgfqpoint{6.296040in}{2.265957in}}%
\pgfpathlineto{\pgfqpoint{6.298520in}{2.255716in}}%
\pgfpathlineto{\pgfqpoint{6.301000in}{2.258488in}}%
\pgfpathlineto{\pgfqpoint{6.302240in}{2.258818in}}%
\pgfpathlineto{\pgfqpoint{6.308440in}{2.270275in}}%
\pgfpathlineto{\pgfqpoint{6.309680in}{2.266557in}}%
\pgfpathlineto{\pgfqpoint{6.312160in}{2.267841in}}%
\pgfpathlineto{\pgfqpoint{6.313400in}{2.266207in}}%
\pgfpathlineto{\pgfqpoint{6.315880in}{2.266136in}}%
\pgfpathlineto{\pgfqpoint{6.317120in}{2.262497in}}%
\pgfpathlineto{\pgfqpoint{6.318360in}{2.265146in}}%
\pgfpathlineto{\pgfqpoint{6.320840in}{2.261159in}}%
\pgfpathlineto{\pgfqpoint{6.322080in}{2.262502in}}%
\pgfpathlineto{\pgfqpoint{6.324560in}{2.259462in}}%
\pgfpathlineto{\pgfqpoint{6.325800in}{2.262422in}}%
\pgfpathlineto{\pgfqpoint{6.328280in}{2.271813in}}%
\pgfpathlineto{\pgfqpoint{6.329520in}{2.273996in}}%
\pgfpathlineto{\pgfqpoint{6.333240in}{2.254136in}}%
\pgfpathlineto{\pgfqpoint{6.335720in}{2.261084in}}%
\pgfpathlineto{\pgfqpoint{6.338200in}{2.257136in}}%
\pgfpathlineto{\pgfqpoint{6.339440in}{2.255441in}}%
\pgfpathlineto{\pgfqpoint{6.343160in}{2.264343in}}%
\pgfpathlineto{\pgfqpoint{6.345640in}{2.259173in}}%
\pgfpathlineto{\pgfqpoint{6.346880in}{2.263949in}}%
\pgfpathlineto{\pgfqpoint{6.349360in}{2.262244in}}%
\pgfpathlineto{\pgfqpoint{6.351840in}{2.255592in}}%
\pgfpathlineto{\pgfqpoint{6.354320in}{2.259800in}}%
\pgfpathlineto{\pgfqpoint{6.356800in}{2.254571in}}%
\pgfpathlineto{\pgfqpoint{6.358040in}{2.254059in}}%
\pgfpathlineto{\pgfqpoint{6.359280in}{2.255346in}}%
\pgfpathlineto{\pgfqpoint{6.360520in}{2.254401in}}%
\pgfpathlineto{\pgfqpoint{6.361760in}{2.251418in}}%
\pgfpathlineto{\pgfqpoint{6.365480in}{2.253168in}}%
\pgfpathlineto{\pgfqpoint{6.366720in}{2.257448in}}%
\pgfpathlineto{\pgfqpoint{6.367960in}{2.257583in}}%
\pgfpathlineto{\pgfqpoint{6.370440in}{2.263454in}}%
\pgfpathlineto{\pgfqpoint{6.375400in}{2.256665in}}%
\pgfpathlineto{\pgfqpoint{6.377880in}{2.262146in}}%
\pgfpathlineto{\pgfqpoint{6.380360in}{2.256365in}}%
\pgfpathlineto{\pgfqpoint{6.381600in}{2.258626in}}%
\pgfpathlineto{\pgfqpoint{6.382840in}{2.255989in}}%
\pgfpathlineto{\pgfqpoint{6.385320in}{2.265975in}}%
\pgfpathlineto{\pgfqpoint{6.386560in}{2.263625in}}%
\pgfpathlineto{\pgfqpoint{6.387800in}{2.264330in}}%
\pgfpathlineto{\pgfqpoint{6.390280in}{2.261087in}}%
\pgfpathlineto{\pgfqpoint{6.395240in}{2.251012in}}%
\pgfpathlineto{\pgfqpoint{6.396480in}{2.251457in}}%
\pgfpathlineto{\pgfqpoint{6.398960in}{2.253180in}}%
\pgfpathlineto{\pgfqpoint{6.401440in}{2.263162in}}%
\pgfpathlineto{\pgfqpoint{6.403920in}{2.265425in}}%
\pgfpathlineto{\pgfqpoint{6.405160in}{2.262226in}}%
\pgfpathlineto{\pgfqpoint{6.407640in}{2.262799in}}%
\pgfpathlineto{\pgfqpoint{6.411360in}{2.243329in}}%
\pgfpathlineto{\pgfqpoint{6.412600in}{2.245081in}}%
\pgfpathlineto{\pgfqpoint{6.415080in}{2.244522in}}%
\pgfpathlineto{\pgfqpoint{6.416320in}{2.246439in}}%
\pgfpathlineto{\pgfqpoint{6.418800in}{2.253141in}}%
\pgfpathlineto{\pgfqpoint{6.420040in}{2.255837in}}%
\pgfpathlineto{\pgfqpoint{6.422520in}{2.245100in}}%
\pgfpathlineto{\pgfqpoint{6.425000in}{2.248738in}}%
\pgfpathlineto{\pgfqpoint{6.426240in}{2.248351in}}%
\pgfpathlineto{\pgfqpoint{6.431200in}{2.259048in}}%
\pgfpathlineto{\pgfqpoint{6.432440in}{2.259772in}}%
\pgfpathlineto{\pgfqpoint{6.433680in}{2.253112in}}%
\pgfpathlineto{\pgfqpoint{6.434920in}{2.253979in}}%
\pgfpathlineto{\pgfqpoint{6.436160in}{2.256265in}}%
\pgfpathlineto{\pgfqpoint{6.438640in}{2.253055in}}%
\pgfpathlineto{\pgfqpoint{6.439880in}{2.252598in}}%
\pgfpathlineto{\pgfqpoint{6.441120in}{2.246965in}}%
\pgfpathlineto{\pgfqpoint{6.442360in}{2.249581in}}%
\pgfpathlineto{\pgfqpoint{6.444840in}{2.246354in}}%
\pgfpathlineto{\pgfqpoint{6.447320in}{2.249126in}}%
\pgfpathlineto{\pgfqpoint{6.448560in}{2.247254in}}%
\pgfpathlineto{\pgfqpoint{6.449800in}{2.248899in}}%
\pgfpathlineto{\pgfqpoint{6.452280in}{2.257092in}}%
\pgfpathlineto{\pgfqpoint{6.453520in}{2.258827in}}%
\pgfpathlineto{\pgfqpoint{6.457240in}{2.245320in}}%
\pgfpathlineto{\pgfqpoint{6.459720in}{2.258614in}}%
\pgfpathlineto{\pgfqpoint{6.462200in}{2.253675in}}%
\pgfpathlineto{\pgfqpoint{6.463440in}{2.251384in}}%
\pgfpathlineto{\pgfqpoint{6.464680in}{2.251941in}}%
\pgfpathlineto{\pgfqpoint{6.467160in}{2.257029in}}%
\pgfpathlineto{\pgfqpoint{6.469640in}{2.250827in}}%
\pgfpathlineto{\pgfqpoint{6.472120in}{2.256475in}}%
\pgfpathlineto{\pgfqpoint{6.473360in}{2.255879in}}%
\pgfpathlineto{\pgfqpoint{6.475840in}{2.246538in}}%
\pgfpathlineto{\pgfqpoint{6.478320in}{2.250585in}}%
\pgfpathlineto{\pgfqpoint{6.480800in}{2.244868in}}%
\pgfpathlineto{\pgfqpoint{6.482040in}{2.243455in}}%
\pgfpathlineto{\pgfqpoint{6.484520in}{2.246959in}}%
\pgfpathlineto{\pgfqpoint{6.487000in}{2.247954in}}%
\pgfpathlineto{\pgfqpoint{6.489480in}{2.244238in}}%
\pgfpathlineto{\pgfqpoint{6.491960in}{2.250782in}}%
\pgfpathlineto{\pgfqpoint{6.494440in}{2.258045in}}%
\pgfpathlineto{\pgfqpoint{6.496920in}{2.250868in}}%
\pgfpathlineto{\pgfqpoint{6.500640in}{2.247058in}}%
\pgfpathlineto{\pgfqpoint{6.501880in}{2.248967in}}%
\pgfpathlineto{\pgfqpoint{6.504360in}{2.242949in}}%
\pgfpathlineto{\pgfqpoint{6.509320in}{2.252635in}}%
\pgfpathlineto{\pgfqpoint{6.510560in}{2.250830in}}%
\pgfpathlineto{\pgfqpoint{6.511800in}{2.251499in}}%
\pgfpathlineto{\pgfqpoint{6.515520in}{2.246649in}}%
\pgfpathlineto{\pgfqpoint{6.519240in}{2.234118in}}%
\pgfpathlineto{\pgfqpoint{6.520480in}{2.234929in}}%
\pgfpathlineto{\pgfqpoint{6.522960in}{2.232523in}}%
\pgfpathlineto{\pgfqpoint{6.525440in}{2.242453in}}%
\pgfpathlineto{\pgfqpoint{6.527920in}{2.244814in}}%
\pgfpathlineto{\pgfqpoint{6.530400in}{2.239198in}}%
\pgfpathlineto{\pgfqpoint{6.531640in}{2.240053in}}%
\pgfpathlineto{\pgfqpoint{6.534120in}{2.225404in}}%
\pgfpathlineto{\pgfqpoint{6.535360in}{2.225611in}}%
\pgfpathlineto{\pgfqpoint{6.536600in}{2.229302in}}%
\pgfpathlineto{\pgfqpoint{6.540320in}{2.229827in}}%
\pgfpathlineto{\pgfqpoint{6.542800in}{2.233533in}}%
\pgfpathlineto{\pgfqpoint{6.544040in}{2.236429in}}%
\pgfpathlineto{\pgfqpoint{6.546520in}{2.227726in}}%
\pgfpathlineto{\pgfqpoint{6.550240in}{2.232514in}}%
\pgfpathlineto{\pgfqpoint{6.553960in}{2.239866in}}%
\pgfpathlineto{\pgfqpoint{6.556440in}{2.243513in}}%
\pgfpathlineto{\pgfqpoint{6.557680in}{2.236982in}}%
\pgfpathlineto{\pgfqpoint{6.560160in}{2.242963in}}%
\pgfpathlineto{\pgfqpoint{6.561400in}{2.240090in}}%
\pgfpathlineto{\pgfqpoint{6.562640in}{2.241749in}}%
\pgfpathlineto{\pgfqpoint{6.563880in}{2.240403in}}%
\pgfpathlineto{\pgfqpoint{6.565120in}{2.234944in}}%
\pgfpathlineto{\pgfqpoint{6.566360in}{2.237158in}}%
\pgfpathlineto{\pgfqpoint{6.567600in}{2.234797in}}%
\pgfpathlineto{\pgfqpoint{6.568840in}{2.235278in}}%
\pgfpathlineto{\pgfqpoint{6.570080in}{2.238968in}}%
\pgfpathlineto{\pgfqpoint{6.572560in}{2.238012in}}%
\pgfpathlineto{\pgfqpoint{6.573800in}{2.238538in}}%
\pgfpathlineto{\pgfqpoint{6.575040in}{2.241993in}}%
\pgfpathlineto{\pgfqpoint{6.576280in}{2.242126in}}%
\pgfpathlineto{\pgfqpoint{6.577520in}{2.243572in}}%
\pgfpathlineto{\pgfqpoint{6.581240in}{2.228993in}}%
\pgfpathlineto{\pgfqpoint{6.583720in}{2.243289in}}%
\pgfpathlineto{\pgfqpoint{6.586200in}{2.234829in}}%
\pgfpathlineto{\pgfqpoint{6.588680in}{2.230633in}}%
\pgfpathlineto{\pgfqpoint{6.591160in}{2.235457in}}%
\pgfpathlineto{\pgfqpoint{6.593640in}{2.228115in}}%
\pgfpathlineto{\pgfqpoint{6.597360in}{2.231117in}}%
\pgfpathlineto{\pgfqpoint{6.599840in}{2.219652in}}%
\pgfpathlineto{\pgfqpoint{6.602320in}{2.224448in}}%
\pgfpathlineto{\pgfqpoint{6.603560in}{2.222633in}}%
\pgfpathlineto{\pgfqpoint{6.611000in}{2.231128in}}%
\pgfpathlineto{\pgfqpoint{6.613480in}{2.226402in}}%
\pgfpathlineto{\pgfqpoint{6.617200in}{2.239079in}}%
\pgfpathlineto{\pgfqpoint{6.618440in}{2.241980in}}%
\pgfpathlineto{\pgfqpoint{6.623400in}{2.235098in}}%
\pgfpathlineto{\pgfqpoint{6.625880in}{2.236278in}}%
\pgfpathlineto{\pgfqpoint{6.628360in}{2.232066in}}%
\pgfpathlineto{\pgfqpoint{6.629600in}{2.234347in}}%
\pgfpathlineto{\pgfqpoint{6.630840in}{2.231694in}}%
\pgfpathlineto{\pgfqpoint{6.633320in}{2.232668in}}%
\pgfpathlineto{\pgfqpoint{6.634560in}{2.228489in}}%
\pgfpathlineto{\pgfqpoint{6.635800in}{2.229878in}}%
\pgfpathlineto{\pgfqpoint{6.639520in}{2.228341in}}%
\pgfpathlineto{\pgfqpoint{6.643240in}{2.215710in}}%
\pgfpathlineto{\pgfqpoint{6.644480in}{2.216257in}}%
\pgfpathlineto{\pgfqpoint{6.645720in}{2.216279in}}%
\pgfpathlineto{\pgfqpoint{6.646960in}{2.217623in}}%
\pgfpathlineto{\pgfqpoint{6.649440in}{2.229800in}}%
\pgfpathlineto{\pgfqpoint{6.650680in}{2.229069in}}%
\pgfpathlineto{\pgfqpoint{6.651920in}{2.229886in}}%
\pgfpathlineto{\pgfqpoint{6.653160in}{2.228336in}}%
\pgfpathlineto{\pgfqpoint{6.654400in}{2.224717in}}%
\pgfpathlineto{\pgfqpoint{6.655640in}{2.226284in}}%
\pgfpathlineto{\pgfqpoint{6.658120in}{2.214439in}}%
\pgfpathlineto{\pgfqpoint{6.660600in}{2.218461in}}%
\pgfpathlineto{\pgfqpoint{6.661840in}{2.213373in}}%
\pgfpathlineto{\pgfqpoint{6.664320in}{2.214555in}}%
\pgfpathlineto{\pgfqpoint{6.666800in}{2.220320in}}%
\pgfpathlineto{\pgfqpoint{6.668040in}{2.224403in}}%
\pgfpathlineto{\pgfqpoint{6.670520in}{2.214308in}}%
\pgfpathlineto{\pgfqpoint{6.673000in}{2.212859in}}%
\pgfpathlineto{\pgfqpoint{6.680440in}{2.223164in}}%
\pgfpathlineto{\pgfqpoint{6.681680in}{2.217595in}}%
\pgfpathlineto{\pgfqpoint{6.684160in}{2.224992in}}%
\pgfpathlineto{\pgfqpoint{6.686640in}{2.224604in}}%
\pgfpathlineto{\pgfqpoint{6.687880in}{2.224014in}}%
\pgfpathlineto{\pgfqpoint{6.691600in}{2.214272in}}%
\pgfpathlineto{\pgfqpoint{6.692840in}{2.215831in}}%
\pgfpathlineto{\pgfqpoint{6.694080in}{2.220054in}}%
\pgfpathlineto{\pgfqpoint{6.695320in}{2.219963in}}%
\pgfpathlineto{\pgfqpoint{6.696560in}{2.217908in}}%
\pgfpathlineto{\pgfqpoint{6.697800in}{2.218517in}}%
\pgfpathlineto{\pgfqpoint{6.699040in}{2.223075in}}%
\pgfpathlineto{\pgfqpoint{6.700280in}{2.223254in}}%
\pgfpathlineto{\pgfqpoint{6.701520in}{2.224583in}}%
\pgfpathlineto{\pgfqpoint{6.702760in}{2.223296in}}%
\pgfpathlineto{\pgfqpoint{6.704000in}{2.217952in}}%
\pgfpathlineto{\pgfqpoint{6.705240in}{2.205239in}}%
\pgfpathlineto{\pgfqpoint{6.707720in}{2.218379in}}%
\pgfpathlineto{\pgfqpoint{6.711440in}{2.207391in}}%
\pgfpathlineto{\pgfqpoint{6.712680in}{2.206672in}}%
\pgfpathlineto{\pgfqpoint{6.715160in}{2.216014in}}%
\pgfpathlineto{\pgfqpoint{6.720120in}{2.203943in}}%
\pgfpathlineto{\pgfqpoint{6.721360in}{2.205609in}}%
\pgfpathlineto{\pgfqpoint{6.723840in}{2.189550in}}%
\pgfpathlineto{\pgfqpoint{6.727560in}{2.196977in}}%
\pgfpathlineto{\pgfqpoint{6.728800in}{2.199989in}}%
\pgfpathlineto{\pgfqpoint{6.731280in}{2.198336in}}%
\pgfpathlineto{\pgfqpoint{6.732520in}{2.198003in}}%
\pgfpathlineto{\pgfqpoint{6.735000in}{2.200343in}}%
\pgfpathlineto{\pgfqpoint{6.737480in}{2.198109in}}%
\pgfpathlineto{\pgfqpoint{6.741200in}{2.214785in}}%
\pgfpathlineto{\pgfqpoint{6.742440in}{2.215861in}}%
\pgfpathlineto{\pgfqpoint{6.743680in}{2.214271in}}%
\pgfpathlineto{\pgfqpoint{6.746160in}{2.216353in}}%
\pgfpathlineto{\pgfqpoint{6.748640in}{2.210112in}}%
\pgfpathlineto{\pgfqpoint{6.749880in}{2.212507in}}%
\pgfpathlineto{\pgfqpoint{6.751120in}{2.210160in}}%
\pgfpathlineto{\pgfqpoint{6.753600in}{2.209984in}}%
\pgfpathlineto{\pgfqpoint{6.761040in}{2.198430in}}%
\pgfpathlineto{\pgfqpoint{6.763520in}{2.201785in}}%
\pgfpathlineto{\pgfqpoint{6.767240in}{2.190468in}}%
\pgfpathlineto{\pgfqpoint{6.768480in}{2.191566in}}%
\pgfpathlineto{\pgfqpoint{6.769720in}{2.189521in}}%
\pgfpathlineto{\pgfqpoint{6.770960in}{2.190418in}}%
\pgfpathlineto{\pgfqpoint{6.773440in}{2.204996in}}%
\pgfpathlineto{\pgfqpoint{6.774680in}{2.205221in}}%
\pgfpathlineto{\pgfqpoint{6.775920in}{2.209676in}}%
\pgfpathlineto{\pgfqpoint{6.779640in}{2.204430in}}%
\pgfpathlineto{\pgfqpoint{6.782120in}{2.189556in}}%
\pgfpathlineto{\pgfqpoint{6.783360in}{2.190096in}}%
\pgfpathlineto{\pgfqpoint{6.784600in}{2.193436in}}%
\pgfpathlineto{\pgfqpoint{6.785840in}{2.187396in}}%
\pgfpathlineto{\pgfqpoint{6.787080in}{2.188380in}}%
\pgfpathlineto{\pgfqpoint{6.789560in}{2.197054in}}%
\pgfpathlineto{\pgfqpoint{6.792040in}{2.205677in}}%
\pgfpathlineto{\pgfqpoint{6.797000in}{2.181951in}}%
\pgfpathlineto{\pgfqpoint{6.799480in}{2.182631in}}%
\pgfpathlineto{\pgfqpoint{6.800720in}{2.180067in}}%
\pgfpathlineto{\pgfqpoint{6.804440in}{2.190724in}}%
\pgfpathlineto{\pgfqpoint{6.805680in}{2.184727in}}%
\pgfpathlineto{\pgfqpoint{6.809400in}{2.198029in}}%
\pgfpathlineto{\pgfqpoint{6.810640in}{2.195333in}}%
\pgfpathlineto{\pgfqpoint{6.811880in}{2.195691in}}%
\pgfpathlineto{\pgfqpoint{6.813120in}{2.191913in}}%
\pgfpathlineto{\pgfqpoint{6.814360in}{2.192516in}}%
\pgfpathlineto{\pgfqpoint{6.816840in}{2.188038in}}%
\pgfpathlineto{\pgfqpoint{6.818080in}{2.193194in}}%
\pgfpathlineto{\pgfqpoint{6.819320in}{2.192912in}}%
\pgfpathlineto{\pgfqpoint{6.821800in}{2.188239in}}%
\pgfpathlineto{\pgfqpoint{6.826760in}{2.198173in}}%
\pgfpathlineto{\pgfqpoint{6.829240in}{2.188717in}}%
\pgfpathlineto{\pgfqpoint{6.831720in}{2.199842in}}%
\pgfpathlineto{\pgfqpoint{6.834200in}{2.195199in}}%
\pgfpathlineto{\pgfqpoint{6.835440in}{2.194674in}}%
\pgfpathlineto{\pgfqpoint{6.837920in}{2.197547in}}%
\pgfpathlineto{\pgfqpoint{6.839160in}{2.202286in}}%
\pgfpathlineto{\pgfqpoint{6.841640in}{2.196591in}}%
\pgfpathlineto{\pgfqpoint{6.845360in}{2.193040in}}%
\pgfpathlineto{\pgfqpoint{6.847840in}{2.177525in}}%
\pgfpathlineto{\pgfqpoint{6.852800in}{2.199244in}}%
\pgfpathlineto{\pgfqpoint{6.855280in}{2.190642in}}%
\pgfpathlineto{\pgfqpoint{6.857760in}{2.191377in}}%
\pgfpathlineto{\pgfqpoint{6.859000in}{2.190571in}}%
\pgfpathlineto{\pgfqpoint{6.860240in}{2.187921in}}%
\pgfpathlineto{\pgfqpoint{6.861480in}{2.188103in}}%
\pgfpathlineto{\pgfqpoint{6.862720in}{2.192057in}}%
\pgfpathlineto{\pgfqpoint{6.863960in}{2.190702in}}%
\pgfpathlineto{\pgfqpoint{6.866440in}{2.198517in}}%
\pgfpathlineto{\pgfqpoint{6.868920in}{2.194296in}}%
\pgfpathlineto{\pgfqpoint{6.870160in}{2.194103in}}%
\pgfpathlineto{\pgfqpoint{6.871400in}{2.189115in}}%
\pgfpathlineto{\pgfqpoint{6.872640in}{2.190150in}}%
\pgfpathlineto{\pgfqpoint{6.873880in}{2.194665in}}%
\pgfpathlineto{\pgfqpoint{6.875120in}{2.191058in}}%
\pgfpathlineto{\pgfqpoint{6.876360in}{2.191151in}}%
\pgfpathlineto{\pgfqpoint{6.877600in}{2.193410in}}%
\pgfpathlineto{\pgfqpoint{6.882560in}{2.182543in}}%
\pgfpathlineto{\pgfqpoint{6.883800in}{2.187053in}}%
\pgfpathlineto{\pgfqpoint{6.885040in}{2.184524in}}%
\pgfpathlineto{\pgfqpoint{6.887520in}{2.188382in}}%
\pgfpathlineto{\pgfqpoint{6.890000in}{2.182477in}}%
\pgfpathlineto{\pgfqpoint{6.891240in}{2.177770in}}%
\pgfpathlineto{\pgfqpoint{6.893720in}{2.182630in}}%
\pgfpathlineto{\pgfqpoint{6.894960in}{2.181044in}}%
\pgfpathlineto{\pgfqpoint{6.897440in}{2.199463in}}%
\pgfpathlineto{\pgfqpoint{6.899920in}{2.203350in}}%
\pgfpathlineto{\pgfqpoint{6.904880in}{2.180386in}}%
\pgfpathlineto{\pgfqpoint{6.906120in}{2.169028in}}%
\pgfpathlineto{\pgfqpoint{6.907360in}{2.169582in}}%
\pgfpathlineto{\pgfqpoint{6.908600in}{2.173662in}}%
\pgfpathlineto{\pgfqpoint{6.909840in}{2.168427in}}%
\pgfpathlineto{\pgfqpoint{6.912320in}{2.176016in}}%
\pgfpathlineto{\pgfqpoint{6.916040in}{2.202886in}}%
\pgfpathlineto{\pgfqpoint{6.922240in}{2.174564in}}%
\pgfpathlineto{\pgfqpoint{6.923480in}{2.175224in}}%
\pgfpathlineto{\pgfqpoint{6.924720in}{2.174386in}}%
\pgfpathlineto{\pgfqpoint{6.925960in}{2.177173in}}%
\pgfpathlineto{\pgfqpoint{6.928440in}{2.192894in}}%
\pgfpathlineto{\pgfqpoint{6.929680in}{2.184636in}}%
\pgfpathlineto{\pgfqpoint{6.932160in}{2.201839in}}%
\pgfpathlineto{\pgfqpoint{6.937120in}{2.186907in}}%
\pgfpathlineto{\pgfqpoint{6.939600in}{2.189474in}}%
\pgfpathlineto{\pgfqpoint{6.940840in}{2.185779in}}%
\pgfpathlineto{\pgfqpoint{6.942080in}{2.188122in}}%
\pgfpathlineto{\pgfqpoint{6.944560in}{2.183245in}}%
\pgfpathlineto{\pgfqpoint{6.945800in}{2.184507in}}%
\pgfpathlineto{\pgfqpoint{6.950760in}{2.199336in}}%
\pgfpathlineto{\pgfqpoint{6.952000in}{2.198519in}}%
\pgfpathlineto{\pgfqpoint{6.954480in}{2.210922in}}%
\pgfpathlineto{\pgfqpoint{6.955720in}{2.215501in}}%
\pgfpathlineto{\pgfqpoint{6.956960in}{2.207190in}}%
\pgfpathlineto{\pgfqpoint{6.959440in}{2.212222in}}%
\pgfpathlineto{\pgfqpoint{6.961920in}{2.205106in}}%
\pgfpathlineto{\pgfqpoint{6.963160in}{2.207523in}}%
\pgfpathlineto{\pgfqpoint{6.965640in}{2.199983in}}%
\pgfpathlineto{\pgfqpoint{6.966880in}{2.199663in}}%
\pgfpathlineto{\pgfqpoint{6.969360in}{2.208847in}}%
\pgfpathlineto{\pgfqpoint{6.971840in}{2.196447in}}%
\pgfpathlineto{\pgfqpoint{6.976800in}{2.220238in}}%
\pgfpathlineto{\pgfqpoint{6.980520in}{2.204690in}}%
\pgfpathlineto{\pgfqpoint{6.981760in}{2.202435in}}%
\pgfpathlineto{\pgfqpoint{6.983000in}{2.204738in}}%
\pgfpathlineto{\pgfqpoint{6.985480in}{2.194495in}}%
\pgfpathlineto{\pgfqpoint{6.987960in}{2.206788in}}%
\pgfpathlineto{\pgfqpoint{6.990440in}{2.214391in}}%
\pgfpathlineto{\pgfqpoint{6.991680in}{2.210486in}}%
\pgfpathlineto{\pgfqpoint{6.992920in}{2.210423in}}%
\pgfpathlineto{\pgfqpoint{6.994160in}{2.213427in}}%
\pgfpathlineto{\pgfqpoint{6.995400in}{2.207530in}}%
\pgfpathlineto{\pgfqpoint{6.997880in}{2.221319in}}%
\pgfpathlineto{\pgfqpoint{6.999120in}{2.214416in}}%
\pgfpathlineto{\pgfqpoint{7.000360in}{2.214802in}}%
\pgfpathlineto{\pgfqpoint{7.005320in}{2.197459in}}%
\pgfpathlineto{\pgfqpoint{7.006560in}{2.189842in}}%
\pgfpathlineto{\pgfqpoint{7.007800in}{2.199607in}}%
\pgfpathlineto{\pgfqpoint{7.009040in}{2.197129in}}%
\pgfpathlineto{\pgfqpoint{7.010280in}{2.201069in}}%
\pgfpathlineto{\pgfqpoint{7.012760in}{2.192949in}}%
\pgfpathlineto{\pgfqpoint{7.015240in}{2.184450in}}%
\pgfpathlineto{\pgfqpoint{7.017720in}{2.193567in}}%
\pgfpathlineto{\pgfqpoint{7.018960in}{2.194362in}}%
\pgfpathlineto{\pgfqpoint{7.021440in}{2.218546in}}%
\pgfpathlineto{\pgfqpoint{7.022680in}{2.220360in}}%
\pgfpathlineto{\pgfqpoint{7.025160in}{2.210435in}}%
\pgfpathlineto{\pgfqpoint{7.027640in}{2.196101in}}%
\pgfpathlineto{\pgfqpoint{7.028880in}{2.185632in}}%
\pgfpathlineto{\pgfqpoint{7.030120in}{2.187046in}}%
\pgfpathlineto{\pgfqpoint{7.032600in}{2.197525in}}%
\pgfpathlineto{\pgfqpoint{7.033840in}{2.190261in}}%
\pgfpathlineto{\pgfqpoint{7.036320in}{2.195598in}}%
\pgfpathlineto{\pgfqpoint{7.040040in}{2.230910in}}%
\pgfpathlineto{\pgfqpoint{7.046240in}{2.173514in}}%
\pgfpathlineto{\pgfqpoint{7.047480in}{2.172969in}}%
\pgfpathlineto{\pgfqpoint{7.048720in}{2.174488in}}%
\pgfpathlineto{\pgfqpoint{7.052440in}{2.191022in}}%
\pgfpathlineto{\pgfqpoint{7.053680in}{2.184060in}}%
\pgfpathlineto{\pgfqpoint{7.056160in}{2.205795in}}%
\pgfpathlineto{\pgfqpoint{7.061120in}{2.187537in}}%
\pgfpathlineto{\pgfqpoint{7.062360in}{2.189876in}}%
\pgfpathlineto{\pgfqpoint{7.063600in}{2.189097in}}%
\pgfpathlineto{\pgfqpoint{7.064840in}{2.184756in}}%
\pgfpathlineto{\pgfqpoint{7.066080in}{2.190720in}}%
\pgfpathlineto{\pgfqpoint{7.068560in}{2.186895in}}%
\pgfpathlineto{\pgfqpoint{7.069800in}{2.188638in}}%
\pgfpathlineto{\pgfqpoint{7.071040in}{2.194172in}}%
\pgfpathlineto{\pgfqpoint{7.072280in}{2.192864in}}%
\pgfpathlineto{\pgfqpoint{7.074760in}{2.200748in}}%
\pgfpathlineto{\pgfqpoint{7.076000in}{2.195058in}}%
\pgfpathlineto{\pgfqpoint{7.079720in}{2.208375in}}%
\pgfpathlineto{\pgfqpoint{7.080960in}{2.201060in}}%
\pgfpathlineto{\pgfqpoint{7.082200in}{2.206624in}}%
\pgfpathlineto{\pgfqpoint{7.084680in}{2.203289in}}%
\pgfpathlineto{\pgfqpoint{7.085920in}{2.191100in}}%
\pgfpathlineto{\pgfqpoint{7.087160in}{2.195246in}}%
\pgfpathlineto{\pgfqpoint{7.088400in}{2.189949in}}%
\pgfpathlineto{\pgfqpoint{7.089640in}{2.190355in}}%
\pgfpathlineto{\pgfqpoint{7.093360in}{2.212066in}}%
\pgfpathlineto{\pgfqpoint{7.095840in}{2.200719in}}%
\pgfpathlineto{\pgfqpoint{7.099560in}{2.238381in}}%
\pgfpathlineto{\pgfqpoint{7.102040in}{2.258983in}}%
\pgfpathlineto{\pgfqpoint{7.105760in}{2.235137in}}%
\pgfpathlineto{\pgfqpoint{7.107000in}{2.240587in}}%
\pgfpathlineto{\pgfqpoint{7.109480in}{2.236620in}}%
\pgfpathlineto{\pgfqpoint{7.114440in}{2.275091in}}%
\pgfpathlineto{\pgfqpoint{7.116920in}{2.259153in}}%
\pgfpathlineto{\pgfqpoint{7.118160in}{2.268519in}}%
\pgfpathlineto{\pgfqpoint{7.119400in}{2.257728in}}%
\pgfpathlineto{\pgfqpoint{7.121880in}{2.278002in}}%
\pgfpathlineto{\pgfqpoint{7.126840in}{2.258769in}}%
\pgfpathlineto{\pgfqpoint{7.128080in}{2.251458in}}%
\pgfpathlineto{\pgfqpoint{7.129320in}{2.255039in}}%
\pgfpathlineto{\pgfqpoint{7.130560in}{2.240432in}}%
\pgfpathlineto{\pgfqpoint{7.131800in}{2.250745in}}%
\pgfpathlineto{\pgfqpoint{7.134280in}{2.243122in}}%
\pgfpathlineto{\pgfqpoint{7.136760in}{2.226641in}}%
\pgfpathlineto{\pgfqpoint{7.138000in}{2.225906in}}%
\pgfpathlineto{\pgfqpoint{7.141720in}{2.256427in}}%
\pgfpathlineto{\pgfqpoint{7.142960in}{2.250641in}}%
\pgfpathlineto{\pgfqpoint{7.144200in}{2.267524in}}%
\pgfpathlineto{\pgfqpoint{7.145440in}{2.268456in}}%
\pgfpathlineto{\pgfqpoint{7.146680in}{2.265809in}}%
\pgfpathlineto{\pgfqpoint{7.147920in}{2.267682in}}%
\pgfpathlineto{\pgfqpoint{7.149160in}{2.259976in}}%
\pgfpathlineto{\pgfqpoint{7.151640in}{2.227516in}}%
\pgfpathlineto{\pgfqpoint{7.152880in}{2.228663in}}%
\pgfpathlineto{\pgfqpoint{7.155360in}{2.256569in}}%
\pgfpathlineto{\pgfqpoint{7.156600in}{2.259276in}}%
\pgfpathlineto{\pgfqpoint{7.157840in}{2.252710in}}%
\pgfpathlineto{\pgfqpoint{7.164040in}{2.297110in}}%
\pgfpathlineto{\pgfqpoint{7.165280in}{2.284329in}}%
\pgfpathlineto{\pgfqpoint{7.169000in}{2.223400in}}%
\pgfpathlineto{\pgfqpoint{7.171480in}{2.205123in}}%
\pgfpathlineto{\pgfqpoint{7.175200in}{2.223434in}}%
\pgfpathlineto{\pgfqpoint{7.176440in}{2.234509in}}%
\pgfpathlineto{\pgfqpoint{7.177680in}{2.234860in}}%
\pgfpathlineto{\pgfqpoint{7.178920in}{2.244783in}}%
\pgfpathlineto{\pgfqpoint{7.180160in}{2.245194in}}%
\pgfpathlineto{\pgfqpoint{7.185120in}{2.206750in}}%
\pgfpathlineto{\pgfqpoint{7.186360in}{2.209371in}}%
\pgfpathlineto{\pgfqpoint{7.191320in}{2.246928in}}%
\pgfpathlineto{\pgfqpoint{7.193800in}{2.255428in}}%
\pgfpathlineto{\pgfqpoint{7.195040in}{2.248758in}}%
\pgfpathlineto{\pgfqpoint{7.196280in}{2.252611in}}%
\pgfpathlineto{\pgfqpoint{7.198760in}{2.233885in}}%
\pgfpathlineto{\pgfqpoint{7.200000in}{2.223781in}}%
\pgfpathlineto{\pgfqpoint{7.200000in}{2.223781in}}%
\pgfusepath{stroke}%
\end{pgfscope}%
\begin{pgfscope}%
\pgfpathrectangle{\pgfqpoint{1.000000in}{0.350000in}}{\pgfqpoint{6.200000in}{2.800000in}} %
\pgfusepath{clip}%
\pgfsetbuttcap%
\pgfsetroundjoin%
\pgfsetlinewidth{0.501875pt}%
\definecolor{currentstroke}{rgb}{0.000000,0.000000,0.000000}%
\pgfsetstrokecolor{currentstroke}%
\pgfsetdash{{1.000000pt}{3.000000pt}}{0.000000pt}%
\pgfpathmoveto{\pgfqpoint{1.000000in}{0.350000in}}%
\pgfpathlineto{\pgfqpoint{1.000000in}{3.150000in}}%
\pgfusepath{stroke}%
\end{pgfscope}%
\begin{pgfscope}%
\pgfsetbuttcap%
\pgfsetroundjoin%
\definecolor{currentfill}{rgb}{0.000000,0.000000,0.000000}%
\pgfsetfillcolor{currentfill}%
\pgfsetlinewidth{0.501875pt}%
\definecolor{currentstroke}{rgb}{0.000000,0.000000,0.000000}%
\pgfsetstrokecolor{currentstroke}%
\pgfsetdash{}{0pt}%
\pgfsys@defobject{currentmarker}{\pgfqpoint{0.000000in}{0.000000in}}{\pgfqpoint{0.000000in}{0.055556in}}{%
\pgfpathmoveto{\pgfqpoint{0.000000in}{0.000000in}}%
\pgfpathlineto{\pgfqpoint{0.000000in}{0.055556in}}%
\pgfusepath{stroke,fill}%
}%
\begin{pgfscope}%
\pgfsys@transformshift{1.000000in}{0.350000in}%
\pgfsys@useobject{currentmarker}{}%
\end{pgfscope}%
\end{pgfscope}%
\begin{pgfscope}%
\pgfsetbuttcap%
\pgfsetroundjoin%
\definecolor{currentfill}{rgb}{0.000000,0.000000,0.000000}%
\pgfsetfillcolor{currentfill}%
\pgfsetlinewidth{0.501875pt}%
\definecolor{currentstroke}{rgb}{0.000000,0.000000,0.000000}%
\pgfsetstrokecolor{currentstroke}%
\pgfsetdash{}{0pt}%
\pgfsys@defobject{currentmarker}{\pgfqpoint{0.000000in}{-0.055556in}}{\pgfqpoint{0.000000in}{0.000000in}}{%
\pgfpathmoveto{\pgfqpoint{0.000000in}{0.000000in}}%
\pgfpathlineto{\pgfqpoint{0.000000in}{-0.055556in}}%
\pgfusepath{stroke,fill}%
}%
\begin{pgfscope}%
\pgfsys@transformshift{1.000000in}{3.150000in}%
\pgfsys@useobject{currentmarker}{}%
\end{pgfscope}%
\end{pgfscope}%
\begin{pgfscope}%
\pgftext[left,bottom,x=0.946981in,y=0.168387in,rotate=0.000000]{{\sffamily\fontsize{12.000000}{14.400000}\selectfont 0}}
%
\end{pgfscope}%
\begin{pgfscope}%
\pgfpathrectangle{\pgfqpoint{1.000000in}{0.350000in}}{\pgfqpoint{6.200000in}{2.800000in}} %
\pgfusepath{clip}%
\pgfsetbuttcap%
\pgfsetroundjoin%
\pgfsetlinewidth{0.501875pt}%
\definecolor{currentstroke}{rgb}{0.000000,0.000000,0.000000}%
\pgfsetstrokecolor{currentstroke}%
\pgfsetdash{{1.000000pt}{3.000000pt}}{0.000000pt}%
\pgfpathmoveto{\pgfqpoint{2.240000in}{0.350000in}}%
\pgfpathlineto{\pgfqpoint{2.240000in}{3.150000in}}%
\pgfusepath{stroke}%
\end{pgfscope}%
\begin{pgfscope}%
\pgfsetbuttcap%
\pgfsetroundjoin%
\definecolor{currentfill}{rgb}{0.000000,0.000000,0.000000}%
\pgfsetfillcolor{currentfill}%
\pgfsetlinewidth{0.501875pt}%
\definecolor{currentstroke}{rgb}{0.000000,0.000000,0.000000}%
\pgfsetstrokecolor{currentstroke}%
\pgfsetdash{}{0pt}%
\pgfsys@defobject{currentmarker}{\pgfqpoint{0.000000in}{0.000000in}}{\pgfqpoint{0.000000in}{0.055556in}}{%
\pgfpathmoveto{\pgfqpoint{0.000000in}{0.000000in}}%
\pgfpathlineto{\pgfqpoint{0.000000in}{0.055556in}}%
\pgfusepath{stroke,fill}%
}%
\begin{pgfscope}%
\pgfsys@transformshift{2.240000in}{0.350000in}%
\pgfsys@useobject{currentmarker}{}%
\end{pgfscope}%
\end{pgfscope}%
\begin{pgfscope}%
\pgfsetbuttcap%
\pgfsetroundjoin%
\definecolor{currentfill}{rgb}{0.000000,0.000000,0.000000}%
\pgfsetfillcolor{currentfill}%
\pgfsetlinewidth{0.501875pt}%
\definecolor{currentstroke}{rgb}{0.000000,0.000000,0.000000}%
\pgfsetstrokecolor{currentstroke}%
\pgfsetdash{}{0pt}%
\pgfsys@defobject{currentmarker}{\pgfqpoint{0.000000in}{-0.055556in}}{\pgfqpoint{0.000000in}{0.000000in}}{%
\pgfpathmoveto{\pgfqpoint{0.000000in}{0.000000in}}%
\pgfpathlineto{\pgfqpoint{0.000000in}{-0.055556in}}%
\pgfusepath{stroke,fill}%
}%
\begin{pgfscope}%
\pgfsys@transformshift{2.240000in}{3.150000in}%
\pgfsys@useobject{currentmarker}{}%
\end{pgfscope}%
\end{pgfscope}%
\begin{pgfscope}%
\pgftext[left,bottom,x=2.080942in,y=0.168387in,rotate=0.000000]{{\sffamily\fontsize{12.000000}{14.400000}\selectfont 100}}
%
\end{pgfscope}%
\begin{pgfscope}%
\pgfpathrectangle{\pgfqpoint{1.000000in}{0.350000in}}{\pgfqpoint{6.200000in}{2.800000in}} %
\pgfusepath{clip}%
\pgfsetbuttcap%
\pgfsetroundjoin%
\pgfsetlinewidth{0.501875pt}%
\definecolor{currentstroke}{rgb}{0.000000,0.000000,0.000000}%
\pgfsetstrokecolor{currentstroke}%
\pgfsetdash{{1.000000pt}{3.000000pt}}{0.000000pt}%
\pgfpathmoveto{\pgfqpoint{3.480000in}{0.350000in}}%
\pgfpathlineto{\pgfqpoint{3.480000in}{3.150000in}}%
\pgfusepath{stroke}%
\end{pgfscope}%
\begin{pgfscope}%
\pgfsetbuttcap%
\pgfsetroundjoin%
\definecolor{currentfill}{rgb}{0.000000,0.000000,0.000000}%
\pgfsetfillcolor{currentfill}%
\pgfsetlinewidth{0.501875pt}%
\definecolor{currentstroke}{rgb}{0.000000,0.000000,0.000000}%
\pgfsetstrokecolor{currentstroke}%
\pgfsetdash{}{0pt}%
\pgfsys@defobject{currentmarker}{\pgfqpoint{0.000000in}{0.000000in}}{\pgfqpoint{0.000000in}{0.055556in}}{%
\pgfpathmoveto{\pgfqpoint{0.000000in}{0.000000in}}%
\pgfpathlineto{\pgfqpoint{0.000000in}{0.055556in}}%
\pgfusepath{stroke,fill}%
}%
\begin{pgfscope}%
\pgfsys@transformshift{3.480000in}{0.350000in}%
\pgfsys@useobject{currentmarker}{}%
\end{pgfscope}%
\end{pgfscope}%
\begin{pgfscope}%
\pgfsetbuttcap%
\pgfsetroundjoin%
\definecolor{currentfill}{rgb}{0.000000,0.000000,0.000000}%
\pgfsetfillcolor{currentfill}%
\pgfsetlinewidth{0.501875pt}%
\definecolor{currentstroke}{rgb}{0.000000,0.000000,0.000000}%
\pgfsetstrokecolor{currentstroke}%
\pgfsetdash{}{0pt}%
\pgfsys@defobject{currentmarker}{\pgfqpoint{0.000000in}{-0.055556in}}{\pgfqpoint{0.000000in}{0.000000in}}{%
\pgfpathmoveto{\pgfqpoint{0.000000in}{0.000000in}}%
\pgfpathlineto{\pgfqpoint{0.000000in}{-0.055556in}}%
\pgfusepath{stroke,fill}%
}%
\begin{pgfscope}%
\pgfsys@transformshift{3.480000in}{3.150000in}%
\pgfsys@useobject{currentmarker}{}%
\end{pgfscope}%
\end{pgfscope}%
\begin{pgfscope}%
\pgftext[left,bottom,x=3.320942in,y=0.168387in,rotate=0.000000]{{\sffamily\fontsize{12.000000}{14.400000}\selectfont 200}}
%
\end{pgfscope}%
\begin{pgfscope}%
\pgfpathrectangle{\pgfqpoint{1.000000in}{0.350000in}}{\pgfqpoint{6.200000in}{2.800000in}} %
\pgfusepath{clip}%
\pgfsetbuttcap%
\pgfsetroundjoin%
\pgfsetlinewidth{0.501875pt}%
\definecolor{currentstroke}{rgb}{0.000000,0.000000,0.000000}%
\pgfsetstrokecolor{currentstroke}%
\pgfsetdash{{1.000000pt}{3.000000pt}}{0.000000pt}%
\pgfpathmoveto{\pgfqpoint{4.720000in}{0.350000in}}%
\pgfpathlineto{\pgfqpoint{4.720000in}{3.150000in}}%
\pgfusepath{stroke}%
\end{pgfscope}%
\begin{pgfscope}%
\pgfsetbuttcap%
\pgfsetroundjoin%
\definecolor{currentfill}{rgb}{0.000000,0.000000,0.000000}%
\pgfsetfillcolor{currentfill}%
\pgfsetlinewidth{0.501875pt}%
\definecolor{currentstroke}{rgb}{0.000000,0.000000,0.000000}%
\pgfsetstrokecolor{currentstroke}%
\pgfsetdash{}{0pt}%
\pgfsys@defobject{currentmarker}{\pgfqpoint{0.000000in}{0.000000in}}{\pgfqpoint{0.000000in}{0.055556in}}{%
\pgfpathmoveto{\pgfqpoint{0.000000in}{0.000000in}}%
\pgfpathlineto{\pgfqpoint{0.000000in}{0.055556in}}%
\pgfusepath{stroke,fill}%
}%
\begin{pgfscope}%
\pgfsys@transformshift{4.720000in}{0.350000in}%
\pgfsys@useobject{currentmarker}{}%
\end{pgfscope}%
\end{pgfscope}%
\begin{pgfscope}%
\pgfsetbuttcap%
\pgfsetroundjoin%
\definecolor{currentfill}{rgb}{0.000000,0.000000,0.000000}%
\pgfsetfillcolor{currentfill}%
\pgfsetlinewidth{0.501875pt}%
\definecolor{currentstroke}{rgb}{0.000000,0.000000,0.000000}%
\pgfsetstrokecolor{currentstroke}%
\pgfsetdash{}{0pt}%
\pgfsys@defobject{currentmarker}{\pgfqpoint{0.000000in}{-0.055556in}}{\pgfqpoint{0.000000in}{0.000000in}}{%
\pgfpathmoveto{\pgfqpoint{0.000000in}{0.000000in}}%
\pgfpathlineto{\pgfqpoint{0.000000in}{-0.055556in}}%
\pgfusepath{stroke,fill}%
}%
\begin{pgfscope}%
\pgfsys@transformshift{4.720000in}{3.150000in}%
\pgfsys@useobject{currentmarker}{}%
\end{pgfscope}%
\end{pgfscope}%
\begin{pgfscope}%
\pgftext[left,bottom,x=4.560942in,y=0.168387in,rotate=0.000000]{{\sffamily\fontsize{12.000000}{14.400000}\selectfont 300}}
%
\end{pgfscope}%
\begin{pgfscope}%
\pgfpathrectangle{\pgfqpoint{1.000000in}{0.350000in}}{\pgfqpoint{6.200000in}{2.800000in}} %
\pgfusepath{clip}%
\pgfsetbuttcap%
\pgfsetroundjoin%
\pgfsetlinewidth{0.501875pt}%
\definecolor{currentstroke}{rgb}{0.000000,0.000000,0.000000}%
\pgfsetstrokecolor{currentstroke}%
\pgfsetdash{{1.000000pt}{3.000000pt}}{0.000000pt}%
\pgfpathmoveto{\pgfqpoint{5.960000in}{0.350000in}}%
\pgfpathlineto{\pgfqpoint{5.960000in}{3.150000in}}%
\pgfusepath{stroke}%
\end{pgfscope}%
\begin{pgfscope}%
\pgfsetbuttcap%
\pgfsetroundjoin%
\definecolor{currentfill}{rgb}{0.000000,0.000000,0.000000}%
\pgfsetfillcolor{currentfill}%
\pgfsetlinewidth{0.501875pt}%
\definecolor{currentstroke}{rgb}{0.000000,0.000000,0.000000}%
\pgfsetstrokecolor{currentstroke}%
\pgfsetdash{}{0pt}%
\pgfsys@defobject{currentmarker}{\pgfqpoint{0.000000in}{0.000000in}}{\pgfqpoint{0.000000in}{0.055556in}}{%
\pgfpathmoveto{\pgfqpoint{0.000000in}{0.000000in}}%
\pgfpathlineto{\pgfqpoint{0.000000in}{0.055556in}}%
\pgfusepath{stroke,fill}%
}%
\begin{pgfscope}%
\pgfsys@transformshift{5.960000in}{0.350000in}%
\pgfsys@useobject{currentmarker}{}%
\end{pgfscope}%
\end{pgfscope}%
\begin{pgfscope}%
\pgfsetbuttcap%
\pgfsetroundjoin%
\definecolor{currentfill}{rgb}{0.000000,0.000000,0.000000}%
\pgfsetfillcolor{currentfill}%
\pgfsetlinewidth{0.501875pt}%
\definecolor{currentstroke}{rgb}{0.000000,0.000000,0.000000}%
\pgfsetstrokecolor{currentstroke}%
\pgfsetdash{}{0pt}%
\pgfsys@defobject{currentmarker}{\pgfqpoint{0.000000in}{-0.055556in}}{\pgfqpoint{0.000000in}{0.000000in}}{%
\pgfpathmoveto{\pgfqpoint{0.000000in}{0.000000in}}%
\pgfpathlineto{\pgfqpoint{0.000000in}{-0.055556in}}%
\pgfusepath{stroke,fill}%
}%
\begin{pgfscope}%
\pgfsys@transformshift{5.960000in}{3.150000in}%
\pgfsys@useobject{currentmarker}{}%
\end{pgfscope}%
\end{pgfscope}%
\begin{pgfscope}%
\pgftext[left,bottom,x=5.800942in,y=0.168387in,rotate=0.000000]{{\sffamily\fontsize{12.000000}{14.400000}\selectfont 400}}
%
\end{pgfscope}%
\begin{pgfscope}%
\pgfpathrectangle{\pgfqpoint{1.000000in}{0.350000in}}{\pgfqpoint{6.200000in}{2.800000in}} %
\pgfusepath{clip}%
\pgfsetbuttcap%
\pgfsetroundjoin%
\pgfsetlinewidth{0.501875pt}%
\definecolor{currentstroke}{rgb}{0.000000,0.000000,0.000000}%
\pgfsetstrokecolor{currentstroke}%
\pgfsetdash{{1.000000pt}{3.000000pt}}{0.000000pt}%
\pgfpathmoveto{\pgfqpoint{7.200000in}{0.350000in}}%
\pgfpathlineto{\pgfqpoint{7.200000in}{3.150000in}}%
\pgfusepath{stroke}%
\end{pgfscope}%
\begin{pgfscope}%
\pgfsetbuttcap%
\pgfsetroundjoin%
\definecolor{currentfill}{rgb}{0.000000,0.000000,0.000000}%
\pgfsetfillcolor{currentfill}%
\pgfsetlinewidth{0.501875pt}%
\definecolor{currentstroke}{rgb}{0.000000,0.000000,0.000000}%
\pgfsetstrokecolor{currentstroke}%
\pgfsetdash{}{0pt}%
\pgfsys@defobject{currentmarker}{\pgfqpoint{0.000000in}{0.000000in}}{\pgfqpoint{0.000000in}{0.055556in}}{%
\pgfpathmoveto{\pgfqpoint{0.000000in}{0.000000in}}%
\pgfpathlineto{\pgfqpoint{0.000000in}{0.055556in}}%
\pgfusepath{stroke,fill}%
}%
\begin{pgfscope}%
\pgfsys@transformshift{7.200000in}{0.350000in}%
\pgfsys@useobject{currentmarker}{}%
\end{pgfscope}%
\end{pgfscope}%
\begin{pgfscope}%
\pgfsetbuttcap%
\pgfsetroundjoin%
\definecolor{currentfill}{rgb}{0.000000,0.000000,0.000000}%
\pgfsetfillcolor{currentfill}%
\pgfsetlinewidth{0.501875pt}%
\definecolor{currentstroke}{rgb}{0.000000,0.000000,0.000000}%
\pgfsetstrokecolor{currentstroke}%
\pgfsetdash{}{0pt}%
\pgfsys@defobject{currentmarker}{\pgfqpoint{0.000000in}{-0.055556in}}{\pgfqpoint{0.000000in}{0.000000in}}{%
\pgfpathmoveto{\pgfqpoint{0.000000in}{0.000000in}}%
\pgfpathlineto{\pgfqpoint{0.000000in}{-0.055556in}}%
\pgfusepath{stroke,fill}%
}%
\begin{pgfscope}%
\pgfsys@transformshift{7.200000in}{3.150000in}%
\pgfsys@useobject{currentmarker}{}%
\end{pgfscope}%
\end{pgfscope}%
\begin{pgfscope}%
\pgftext[left,bottom,x=7.040942in,y=0.168387in,rotate=0.000000]{{\sffamily\fontsize{12.000000}{14.400000}\selectfont 500}}
%
\end{pgfscope}%
\begin{pgfscope}%
\pgftext[left,bottom,x=3.911727in,y=-0.030045in,rotate=0.000000]{{\sffamily\fontsize{12.000000}{14.400000}\selectfont time}}
%
\end{pgfscope}%
\begin{pgfscope}%
\pgfpathrectangle{\pgfqpoint{1.000000in}{0.350000in}}{\pgfqpoint{6.200000in}{2.800000in}} %
\pgfusepath{clip}%
\pgfsetbuttcap%
\pgfsetroundjoin%
\pgfsetlinewidth{0.501875pt}%
\definecolor{currentstroke}{rgb}{0.000000,0.000000,0.000000}%
\pgfsetstrokecolor{currentstroke}%
\pgfsetdash{{1.000000pt}{3.000000pt}}{0.000000pt}%
\pgfpathmoveto{\pgfqpoint{1.000000in}{0.350000in}}%
\pgfpathlineto{\pgfqpoint{7.200000in}{0.350000in}}%
\pgfusepath{stroke}%
\end{pgfscope}%
\begin{pgfscope}%
\pgfsetbuttcap%
\pgfsetroundjoin%
\definecolor{currentfill}{rgb}{0.000000,0.000000,0.000000}%
\pgfsetfillcolor{currentfill}%
\pgfsetlinewidth{0.501875pt}%
\definecolor{currentstroke}{rgb}{0.000000,0.000000,0.000000}%
\pgfsetstrokecolor{currentstroke}%
\pgfsetdash{}{0pt}%
\pgfsys@defobject{currentmarker}{\pgfqpoint{0.000000in}{0.000000in}}{\pgfqpoint{0.055556in}{0.000000in}}{%
\pgfpathmoveto{\pgfqpoint{0.000000in}{0.000000in}}%
\pgfpathlineto{\pgfqpoint{0.055556in}{0.000000in}}%
\pgfusepath{stroke,fill}%
}%
\begin{pgfscope}%
\pgfsys@transformshift{1.000000in}{0.350000in}%
\pgfsys@useobject{currentmarker}{}%
\end{pgfscope}%
\end{pgfscope}%
\begin{pgfscope}%
\pgfsetbuttcap%
\pgfsetroundjoin%
\definecolor{currentfill}{rgb}{0.000000,0.000000,0.000000}%
\pgfsetfillcolor{currentfill}%
\pgfsetlinewidth{0.501875pt}%
\definecolor{currentstroke}{rgb}{0.000000,0.000000,0.000000}%
\pgfsetstrokecolor{currentstroke}%
\pgfsetdash{}{0pt}%
\pgfsys@defobject{currentmarker}{\pgfqpoint{-0.055556in}{0.000000in}}{\pgfqpoint{0.000000in}{0.000000in}}{%
\pgfpathmoveto{\pgfqpoint{0.000000in}{0.000000in}}%
\pgfpathlineto{\pgfqpoint{-0.055556in}{0.000000in}}%
\pgfusepath{stroke,fill}%
}%
\begin{pgfscope}%
\pgfsys@transformshift{7.200000in}{0.350000in}%
\pgfsys@useobject{currentmarker}{}%
\end{pgfscope}%
\end{pgfscope}%
\begin{pgfscope}%
\pgftext[left,bottom,x=0.361274in,y=0.286971in,rotate=0.000000]{{\sffamily\fontsize{12.000000}{14.400000}\selectfont 0.0020}}
%
\end{pgfscope}%
\begin{pgfscope}%
\pgfpathrectangle{\pgfqpoint{1.000000in}{0.350000in}}{\pgfqpoint{6.200000in}{2.800000in}} %
\pgfusepath{clip}%
\pgfsetbuttcap%
\pgfsetroundjoin%
\pgfsetlinewidth{0.501875pt}%
\definecolor{currentstroke}{rgb}{0.000000,0.000000,0.000000}%
\pgfsetstrokecolor{currentstroke}%
\pgfsetdash{{1.000000pt}{3.000000pt}}{0.000000pt}%
\pgfpathmoveto{\pgfqpoint{1.000000in}{0.661111in}}%
\pgfpathlineto{\pgfqpoint{7.200000in}{0.661111in}}%
\pgfusepath{stroke}%
\end{pgfscope}%
\begin{pgfscope}%
\pgfsetbuttcap%
\pgfsetroundjoin%
\definecolor{currentfill}{rgb}{0.000000,0.000000,0.000000}%
\pgfsetfillcolor{currentfill}%
\pgfsetlinewidth{0.501875pt}%
\definecolor{currentstroke}{rgb}{0.000000,0.000000,0.000000}%
\pgfsetstrokecolor{currentstroke}%
\pgfsetdash{}{0pt}%
\pgfsys@defobject{currentmarker}{\pgfqpoint{0.000000in}{0.000000in}}{\pgfqpoint{0.055556in}{0.000000in}}{%
\pgfpathmoveto{\pgfqpoint{0.000000in}{0.000000in}}%
\pgfpathlineto{\pgfqpoint{0.055556in}{0.000000in}}%
\pgfusepath{stroke,fill}%
}%
\begin{pgfscope}%
\pgfsys@transformshift{1.000000in}{0.661111in}%
\pgfsys@useobject{currentmarker}{}%
\end{pgfscope}%
\end{pgfscope}%
\begin{pgfscope}%
\pgfsetbuttcap%
\pgfsetroundjoin%
\definecolor{currentfill}{rgb}{0.000000,0.000000,0.000000}%
\pgfsetfillcolor{currentfill}%
\pgfsetlinewidth{0.501875pt}%
\definecolor{currentstroke}{rgb}{0.000000,0.000000,0.000000}%
\pgfsetstrokecolor{currentstroke}%
\pgfsetdash{}{0pt}%
\pgfsys@defobject{currentmarker}{\pgfqpoint{-0.055556in}{0.000000in}}{\pgfqpoint{0.000000in}{0.000000in}}{%
\pgfpathmoveto{\pgfqpoint{0.000000in}{0.000000in}}%
\pgfpathlineto{\pgfqpoint{-0.055556in}{0.000000in}}%
\pgfusepath{stroke,fill}%
}%
\begin{pgfscope}%
\pgfsys@transformshift{7.200000in}{0.661111in}%
\pgfsys@useobject{currentmarker}{}%
\end{pgfscope}%
\end{pgfscope}%
\begin{pgfscope}%
\pgftext[left,bottom,x=0.361274in,y=0.598082in,rotate=0.000000]{{\sffamily\fontsize{12.000000}{14.400000}\selectfont 0.0025}}
%
\end{pgfscope}%
\begin{pgfscope}%
\pgfpathrectangle{\pgfqpoint{1.000000in}{0.350000in}}{\pgfqpoint{6.200000in}{2.800000in}} %
\pgfusepath{clip}%
\pgfsetbuttcap%
\pgfsetroundjoin%
\pgfsetlinewidth{0.501875pt}%
\definecolor{currentstroke}{rgb}{0.000000,0.000000,0.000000}%
\pgfsetstrokecolor{currentstroke}%
\pgfsetdash{{1.000000pt}{3.000000pt}}{0.000000pt}%
\pgfpathmoveto{\pgfqpoint{1.000000in}{0.972222in}}%
\pgfpathlineto{\pgfqpoint{7.200000in}{0.972222in}}%
\pgfusepath{stroke}%
\end{pgfscope}%
\begin{pgfscope}%
\pgfsetbuttcap%
\pgfsetroundjoin%
\definecolor{currentfill}{rgb}{0.000000,0.000000,0.000000}%
\pgfsetfillcolor{currentfill}%
\pgfsetlinewidth{0.501875pt}%
\definecolor{currentstroke}{rgb}{0.000000,0.000000,0.000000}%
\pgfsetstrokecolor{currentstroke}%
\pgfsetdash{}{0pt}%
\pgfsys@defobject{currentmarker}{\pgfqpoint{0.000000in}{0.000000in}}{\pgfqpoint{0.055556in}{0.000000in}}{%
\pgfpathmoveto{\pgfqpoint{0.000000in}{0.000000in}}%
\pgfpathlineto{\pgfqpoint{0.055556in}{0.000000in}}%
\pgfusepath{stroke,fill}%
}%
\begin{pgfscope}%
\pgfsys@transformshift{1.000000in}{0.972222in}%
\pgfsys@useobject{currentmarker}{}%
\end{pgfscope}%
\end{pgfscope}%
\begin{pgfscope}%
\pgfsetbuttcap%
\pgfsetroundjoin%
\definecolor{currentfill}{rgb}{0.000000,0.000000,0.000000}%
\pgfsetfillcolor{currentfill}%
\pgfsetlinewidth{0.501875pt}%
\definecolor{currentstroke}{rgb}{0.000000,0.000000,0.000000}%
\pgfsetstrokecolor{currentstroke}%
\pgfsetdash{}{0pt}%
\pgfsys@defobject{currentmarker}{\pgfqpoint{-0.055556in}{0.000000in}}{\pgfqpoint{0.000000in}{0.000000in}}{%
\pgfpathmoveto{\pgfqpoint{0.000000in}{0.000000in}}%
\pgfpathlineto{\pgfqpoint{-0.055556in}{0.000000in}}%
\pgfusepath{stroke,fill}%
}%
\begin{pgfscope}%
\pgfsys@transformshift{7.200000in}{0.972222in}%
\pgfsys@useobject{currentmarker}{}%
\end{pgfscope}%
\end{pgfscope}%
\begin{pgfscope}%
\pgftext[left,bottom,x=0.361274in,y=0.909193in,rotate=0.000000]{{\sffamily\fontsize{12.000000}{14.400000}\selectfont 0.0030}}
%
\end{pgfscope}%
\begin{pgfscope}%
\pgfpathrectangle{\pgfqpoint{1.000000in}{0.350000in}}{\pgfqpoint{6.200000in}{2.800000in}} %
\pgfusepath{clip}%
\pgfsetbuttcap%
\pgfsetroundjoin%
\pgfsetlinewidth{0.501875pt}%
\definecolor{currentstroke}{rgb}{0.000000,0.000000,0.000000}%
\pgfsetstrokecolor{currentstroke}%
\pgfsetdash{{1.000000pt}{3.000000pt}}{0.000000pt}%
\pgfpathmoveto{\pgfqpoint{1.000000in}{1.283333in}}%
\pgfpathlineto{\pgfqpoint{7.200000in}{1.283333in}}%
\pgfusepath{stroke}%
\end{pgfscope}%
\begin{pgfscope}%
\pgfsetbuttcap%
\pgfsetroundjoin%
\definecolor{currentfill}{rgb}{0.000000,0.000000,0.000000}%
\pgfsetfillcolor{currentfill}%
\pgfsetlinewidth{0.501875pt}%
\definecolor{currentstroke}{rgb}{0.000000,0.000000,0.000000}%
\pgfsetstrokecolor{currentstroke}%
\pgfsetdash{}{0pt}%
\pgfsys@defobject{currentmarker}{\pgfqpoint{0.000000in}{0.000000in}}{\pgfqpoint{0.055556in}{0.000000in}}{%
\pgfpathmoveto{\pgfqpoint{0.000000in}{0.000000in}}%
\pgfpathlineto{\pgfqpoint{0.055556in}{0.000000in}}%
\pgfusepath{stroke,fill}%
}%
\begin{pgfscope}%
\pgfsys@transformshift{1.000000in}{1.283333in}%
\pgfsys@useobject{currentmarker}{}%
\end{pgfscope}%
\end{pgfscope}%
\begin{pgfscope}%
\pgfsetbuttcap%
\pgfsetroundjoin%
\definecolor{currentfill}{rgb}{0.000000,0.000000,0.000000}%
\pgfsetfillcolor{currentfill}%
\pgfsetlinewidth{0.501875pt}%
\definecolor{currentstroke}{rgb}{0.000000,0.000000,0.000000}%
\pgfsetstrokecolor{currentstroke}%
\pgfsetdash{}{0pt}%
\pgfsys@defobject{currentmarker}{\pgfqpoint{-0.055556in}{0.000000in}}{\pgfqpoint{0.000000in}{0.000000in}}{%
\pgfpathmoveto{\pgfqpoint{0.000000in}{0.000000in}}%
\pgfpathlineto{\pgfqpoint{-0.055556in}{0.000000in}}%
\pgfusepath{stroke,fill}%
}%
\begin{pgfscope}%
\pgfsys@transformshift{7.200000in}{1.283333in}%
\pgfsys@useobject{currentmarker}{}%
\end{pgfscope}%
\end{pgfscope}%
\begin{pgfscope}%
\pgftext[left,bottom,x=0.361274in,y=1.220305in,rotate=0.000000]{{\sffamily\fontsize{12.000000}{14.400000}\selectfont 0.0035}}
%
\end{pgfscope}%
\begin{pgfscope}%
\pgfpathrectangle{\pgfqpoint{1.000000in}{0.350000in}}{\pgfqpoint{6.200000in}{2.800000in}} %
\pgfusepath{clip}%
\pgfsetbuttcap%
\pgfsetroundjoin%
\pgfsetlinewidth{0.501875pt}%
\definecolor{currentstroke}{rgb}{0.000000,0.000000,0.000000}%
\pgfsetstrokecolor{currentstroke}%
\pgfsetdash{{1.000000pt}{3.000000pt}}{0.000000pt}%
\pgfpathmoveto{\pgfqpoint{1.000000in}{1.594444in}}%
\pgfpathlineto{\pgfqpoint{7.200000in}{1.594444in}}%
\pgfusepath{stroke}%
\end{pgfscope}%
\begin{pgfscope}%
\pgfsetbuttcap%
\pgfsetroundjoin%
\definecolor{currentfill}{rgb}{0.000000,0.000000,0.000000}%
\pgfsetfillcolor{currentfill}%
\pgfsetlinewidth{0.501875pt}%
\definecolor{currentstroke}{rgb}{0.000000,0.000000,0.000000}%
\pgfsetstrokecolor{currentstroke}%
\pgfsetdash{}{0pt}%
\pgfsys@defobject{currentmarker}{\pgfqpoint{0.000000in}{0.000000in}}{\pgfqpoint{0.055556in}{0.000000in}}{%
\pgfpathmoveto{\pgfqpoint{0.000000in}{0.000000in}}%
\pgfpathlineto{\pgfqpoint{0.055556in}{0.000000in}}%
\pgfusepath{stroke,fill}%
}%
\begin{pgfscope}%
\pgfsys@transformshift{1.000000in}{1.594444in}%
\pgfsys@useobject{currentmarker}{}%
\end{pgfscope}%
\end{pgfscope}%
\begin{pgfscope}%
\pgfsetbuttcap%
\pgfsetroundjoin%
\definecolor{currentfill}{rgb}{0.000000,0.000000,0.000000}%
\pgfsetfillcolor{currentfill}%
\pgfsetlinewidth{0.501875pt}%
\definecolor{currentstroke}{rgb}{0.000000,0.000000,0.000000}%
\pgfsetstrokecolor{currentstroke}%
\pgfsetdash{}{0pt}%
\pgfsys@defobject{currentmarker}{\pgfqpoint{-0.055556in}{0.000000in}}{\pgfqpoint{0.000000in}{0.000000in}}{%
\pgfpathmoveto{\pgfqpoint{0.000000in}{0.000000in}}%
\pgfpathlineto{\pgfqpoint{-0.055556in}{0.000000in}}%
\pgfusepath{stroke,fill}%
}%
\begin{pgfscope}%
\pgfsys@transformshift{7.200000in}{1.594444in}%
\pgfsys@useobject{currentmarker}{}%
\end{pgfscope}%
\end{pgfscope}%
\begin{pgfscope}%
\pgftext[left,bottom,x=0.361274in,y=1.531416in,rotate=0.000000]{{\sffamily\fontsize{12.000000}{14.400000}\selectfont 0.0040}}
%
\end{pgfscope}%
\begin{pgfscope}%
\pgfpathrectangle{\pgfqpoint{1.000000in}{0.350000in}}{\pgfqpoint{6.200000in}{2.800000in}} %
\pgfusepath{clip}%
\pgfsetbuttcap%
\pgfsetroundjoin%
\pgfsetlinewidth{0.501875pt}%
\definecolor{currentstroke}{rgb}{0.000000,0.000000,0.000000}%
\pgfsetstrokecolor{currentstroke}%
\pgfsetdash{{1.000000pt}{3.000000pt}}{0.000000pt}%
\pgfpathmoveto{\pgfqpoint{1.000000in}{1.905556in}}%
\pgfpathlineto{\pgfqpoint{7.200000in}{1.905556in}}%
\pgfusepath{stroke}%
\end{pgfscope}%
\begin{pgfscope}%
\pgfsetbuttcap%
\pgfsetroundjoin%
\definecolor{currentfill}{rgb}{0.000000,0.000000,0.000000}%
\pgfsetfillcolor{currentfill}%
\pgfsetlinewidth{0.501875pt}%
\definecolor{currentstroke}{rgb}{0.000000,0.000000,0.000000}%
\pgfsetstrokecolor{currentstroke}%
\pgfsetdash{}{0pt}%
\pgfsys@defobject{currentmarker}{\pgfqpoint{0.000000in}{0.000000in}}{\pgfqpoint{0.055556in}{0.000000in}}{%
\pgfpathmoveto{\pgfqpoint{0.000000in}{0.000000in}}%
\pgfpathlineto{\pgfqpoint{0.055556in}{0.000000in}}%
\pgfusepath{stroke,fill}%
}%
\begin{pgfscope}%
\pgfsys@transformshift{1.000000in}{1.905556in}%
\pgfsys@useobject{currentmarker}{}%
\end{pgfscope}%
\end{pgfscope}%
\begin{pgfscope}%
\pgfsetbuttcap%
\pgfsetroundjoin%
\definecolor{currentfill}{rgb}{0.000000,0.000000,0.000000}%
\pgfsetfillcolor{currentfill}%
\pgfsetlinewidth{0.501875pt}%
\definecolor{currentstroke}{rgb}{0.000000,0.000000,0.000000}%
\pgfsetstrokecolor{currentstroke}%
\pgfsetdash{}{0pt}%
\pgfsys@defobject{currentmarker}{\pgfqpoint{-0.055556in}{0.000000in}}{\pgfqpoint{0.000000in}{0.000000in}}{%
\pgfpathmoveto{\pgfqpoint{0.000000in}{0.000000in}}%
\pgfpathlineto{\pgfqpoint{-0.055556in}{0.000000in}}%
\pgfusepath{stroke,fill}%
}%
\begin{pgfscope}%
\pgfsys@transformshift{7.200000in}{1.905556in}%
\pgfsys@useobject{currentmarker}{}%
\end{pgfscope}%
\end{pgfscope}%
\begin{pgfscope}%
\pgftext[left,bottom,x=0.361274in,y=1.842527in,rotate=0.000000]{{\sffamily\fontsize{12.000000}{14.400000}\selectfont 0.0045}}
%
\end{pgfscope}%
\begin{pgfscope}%
\pgfpathrectangle{\pgfqpoint{1.000000in}{0.350000in}}{\pgfqpoint{6.200000in}{2.800000in}} %
\pgfusepath{clip}%
\pgfsetbuttcap%
\pgfsetroundjoin%
\pgfsetlinewidth{0.501875pt}%
\definecolor{currentstroke}{rgb}{0.000000,0.000000,0.000000}%
\pgfsetstrokecolor{currentstroke}%
\pgfsetdash{{1.000000pt}{3.000000pt}}{0.000000pt}%
\pgfpathmoveto{\pgfqpoint{1.000000in}{2.216667in}}%
\pgfpathlineto{\pgfqpoint{7.200000in}{2.216667in}}%
\pgfusepath{stroke}%
\end{pgfscope}%
\begin{pgfscope}%
\pgfsetbuttcap%
\pgfsetroundjoin%
\definecolor{currentfill}{rgb}{0.000000,0.000000,0.000000}%
\pgfsetfillcolor{currentfill}%
\pgfsetlinewidth{0.501875pt}%
\definecolor{currentstroke}{rgb}{0.000000,0.000000,0.000000}%
\pgfsetstrokecolor{currentstroke}%
\pgfsetdash{}{0pt}%
\pgfsys@defobject{currentmarker}{\pgfqpoint{0.000000in}{0.000000in}}{\pgfqpoint{0.055556in}{0.000000in}}{%
\pgfpathmoveto{\pgfqpoint{0.000000in}{0.000000in}}%
\pgfpathlineto{\pgfqpoint{0.055556in}{0.000000in}}%
\pgfusepath{stroke,fill}%
}%
\begin{pgfscope}%
\pgfsys@transformshift{1.000000in}{2.216667in}%
\pgfsys@useobject{currentmarker}{}%
\end{pgfscope}%
\end{pgfscope}%
\begin{pgfscope}%
\pgfsetbuttcap%
\pgfsetroundjoin%
\definecolor{currentfill}{rgb}{0.000000,0.000000,0.000000}%
\pgfsetfillcolor{currentfill}%
\pgfsetlinewidth{0.501875pt}%
\definecolor{currentstroke}{rgb}{0.000000,0.000000,0.000000}%
\pgfsetstrokecolor{currentstroke}%
\pgfsetdash{}{0pt}%
\pgfsys@defobject{currentmarker}{\pgfqpoint{-0.055556in}{0.000000in}}{\pgfqpoint{0.000000in}{0.000000in}}{%
\pgfpathmoveto{\pgfqpoint{0.000000in}{0.000000in}}%
\pgfpathlineto{\pgfqpoint{-0.055556in}{0.000000in}}%
\pgfusepath{stroke,fill}%
}%
\begin{pgfscope}%
\pgfsys@transformshift{7.200000in}{2.216667in}%
\pgfsys@useobject{currentmarker}{}%
\end{pgfscope}%
\end{pgfscope}%
\begin{pgfscope}%
\pgftext[left,bottom,x=0.361274in,y=2.153638in,rotate=0.000000]{{\sffamily\fontsize{12.000000}{14.400000}\selectfont 0.0050}}
%
\end{pgfscope}%
\begin{pgfscope}%
\pgfpathrectangle{\pgfqpoint{1.000000in}{0.350000in}}{\pgfqpoint{6.200000in}{2.800000in}} %
\pgfusepath{clip}%
\pgfsetbuttcap%
\pgfsetroundjoin%
\pgfsetlinewidth{0.501875pt}%
\definecolor{currentstroke}{rgb}{0.000000,0.000000,0.000000}%
\pgfsetstrokecolor{currentstroke}%
\pgfsetdash{{1.000000pt}{3.000000pt}}{0.000000pt}%
\pgfpathmoveto{\pgfqpoint{1.000000in}{2.527778in}}%
\pgfpathlineto{\pgfqpoint{7.200000in}{2.527778in}}%
\pgfusepath{stroke}%
\end{pgfscope}%
\begin{pgfscope}%
\pgfsetbuttcap%
\pgfsetroundjoin%
\definecolor{currentfill}{rgb}{0.000000,0.000000,0.000000}%
\pgfsetfillcolor{currentfill}%
\pgfsetlinewidth{0.501875pt}%
\definecolor{currentstroke}{rgb}{0.000000,0.000000,0.000000}%
\pgfsetstrokecolor{currentstroke}%
\pgfsetdash{}{0pt}%
\pgfsys@defobject{currentmarker}{\pgfqpoint{0.000000in}{0.000000in}}{\pgfqpoint{0.055556in}{0.000000in}}{%
\pgfpathmoveto{\pgfqpoint{0.000000in}{0.000000in}}%
\pgfpathlineto{\pgfqpoint{0.055556in}{0.000000in}}%
\pgfusepath{stroke,fill}%
}%
\begin{pgfscope}%
\pgfsys@transformshift{1.000000in}{2.527778in}%
\pgfsys@useobject{currentmarker}{}%
\end{pgfscope}%
\end{pgfscope}%
\begin{pgfscope}%
\pgfsetbuttcap%
\pgfsetroundjoin%
\definecolor{currentfill}{rgb}{0.000000,0.000000,0.000000}%
\pgfsetfillcolor{currentfill}%
\pgfsetlinewidth{0.501875pt}%
\definecolor{currentstroke}{rgb}{0.000000,0.000000,0.000000}%
\pgfsetstrokecolor{currentstroke}%
\pgfsetdash{}{0pt}%
\pgfsys@defobject{currentmarker}{\pgfqpoint{-0.055556in}{0.000000in}}{\pgfqpoint{0.000000in}{0.000000in}}{%
\pgfpathmoveto{\pgfqpoint{0.000000in}{0.000000in}}%
\pgfpathlineto{\pgfqpoint{-0.055556in}{0.000000in}}%
\pgfusepath{stroke,fill}%
}%
\begin{pgfscope}%
\pgfsys@transformshift{7.200000in}{2.527778in}%
\pgfsys@useobject{currentmarker}{}%
\end{pgfscope}%
\end{pgfscope}%
\begin{pgfscope}%
\pgftext[left,bottom,x=0.361274in,y=2.464749in,rotate=0.000000]{{\sffamily\fontsize{12.000000}{14.400000}\selectfont 0.0055}}
%
\end{pgfscope}%
\begin{pgfscope}%
\pgfpathrectangle{\pgfqpoint{1.000000in}{0.350000in}}{\pgfqpoint{6.200000in}{2.800000in}} %
\pgfusepath{clip}%
\pgfsetbuttcap%
\pgfsetroundjoin%
\pgfsetlinewidth{0.501875pt}%
\definecolor{currentstroke}{rgb}{0.000000,0.000000,0.000000}%
\pgfsetstrokecolor{currentstroke}%
\pgfsetdash{{1.000000pt}{3.000000pt}}{0.000000pt}%
\pgfpathmoveto{\pgfqpoint{1.000000in}{2.838889in}}%
\pgfpathlineto{\pgfqpoint{7.200000in}{2.838889in}}%
\pgfusepath{stroke}%
\end{pgfscope}%
\begin{pgfscope}%
\pgfsetbuttcap%
\pgfsetroundjoin%
\definecolor{currentfill}{rgb}{0.000000,0.000000,0.000000}%
\pgfsetfillcolor{currentfill}%
\pgfsetlinewidth{0.501875pt}%
\definecolor{currentstroke}{rgb}{0.000000,0.000000,0.000000}%
\pgfsetstrokecolor{currentstroke}%
\pgfsetdash{}{0pt}%
\pgfsys@defobject{currentmarker}{\pgfqpoint{0.000000in}{0.000000in}}{\pgfqpoint{0.055556in}{0.000000in}}{%
\pgfpathmoveto{\pgfqpoint{0.000000in}{0.000000in}}%
\pgfpathlineto{\pgfqpoint{0.055556in}{0.000000in}}%
\pgfusepath{stroke,fill}%
}%
\begin{pgfscope}%
\pgfsys@transformshift{1.000000in}{2.838889in}%
\pgfsys@useobject{currentmarker}{}%
\end{pgfscope}%
\end{pgfscope}%
\begin{pgfscope}%
\pgfsetbuttcap%
\pgfsetroundjoin%
\definecolor{currentfill}{rgb}{0.000000,0.000000,0.000000}%
\pgfsetfillcolor{currentfill}%
\pgfsetlinewidth{0.501875pt}%
\definecolor{currentstroke}{rgb}{0.000000,0.000000,0.000000}%
\pgfsetstrokecolor{currentstroke}%
\pgfsetdash{}{0pt}%
\pgfsys@defobject{currentmarker}{\pgfqpoint{-0.055556in}{0.000000in}}{\pgfqpoint{0.000000in}{0.000000in}}{%
\pgfpathmoveto{\pgfqpoint{0.000000in}{0.000000in}}%
\pgfpathlineto{\pgfqpoint{-0.055556in}{0.000000in}}%
\pgfusepath{stroke,fill}%
}%
\begin{pgfscope}%
\pgfsys@transformshift{7.200000in}{2.838889in}%
\pgfsys@useobject{currentmarker}{}%
\end{pgfscope}%
\end{pgfscope}%
\begin{pgfscope}%
\pgftext[left,bottom,x=0.361274in,y=2.775860in,rotate=0.000000]{{\sffamily\fontsize{12.000000}{14.400000}\selectfont 0.0060}}
%
\end{pgfscope}%
\begin{pgfscope}%
\pgfpathrectangle{\pgfqpoint{1.000000in}{0.350000in}}{\pgfqpoint{6.200000in}{2.800000in}} %
\pgfusepath{clip}%
\pgfsetbuttcap%
\pgfsetroundjoin%
\pgfsetlinewidth{0.501875pt}%
\definecolor{currentstroke}{rgb}{0.000000,0.000000,0.000000}%
\pgfsetstrokecolor{currentstroke}%
\pgfsetdash{{1.000000pt}{3.000000pt}}{0.000000pt}%
\pgfpathmoveto{\pgfqpoint{1.000000in}{3.150000in}}%
\pgfpathlineto{\pgfqpoint{7.200000in}{3.150000in}}%
\pgfusepath{stroke}%
\end{pgfscope}%
\begin{pgfscope}%
\pgfsetbuttcap%
\pgfsetroundjoin%
\definecolor{currentfill}{rgb}{0.000000,0.000000,0.000000}%
\pgfsetfillcolor{currentfill}%
\pgfsetlinewidth{0.501875pt}%
\definecolor{currentstroke}{rgb}{0.000000,0.000000,0.000000}%
\pgfsetstrokecolor{currentstroke}%
\pgfsetdash{}{0pt}%
\pgfsys@defobject{currentmarker}{\pgfqpoint{0.000000in}{0.000000in}}{\pgfqpoint{0.055556in}{0.000000in}}{%
\pgfpathmoveto{\pgfqpoint{0.000000in}{0.000000in}}%
\pgfpathlineto{\pgfqpoint{0.055556in}{0.000000in}}%
\pgfusepath{stroke,fill}%
}%
\begin{pgfscope}%
\pgfsys@transformshift{1.000000in}{3.150000in}%
\pgfsys@useobject{currentmarker}{}%
\end{pgfscope}%
\end{pgfscope}%
\begin{pgfscope}%
\pgfsetbuttcap%
\pgfsetroundjoin%
\definecolor{currentfill}{rgb}{0.000000,0.000000,0.000000}%
\pgfsetfillcolor{currentfill}%
\pgfsetlinewidth{0.501875pt}%
\definecolor{currentstroke}{rgb}{0.000000,0.000000,0.000000}%
\pgfsetstrokecolor{currentstroke}%
\pgfsetdash{}{0pt}%
\pgfsys@defobject{currentmarker}{\pgfqpoint{-0.055556in}{0.000000in}}{\pgfqpoint{0.000000in}{0.000000in}}{%
\pgfpathmoveto{\pgfqpoint{0.000000in}{0.000000in}}%
\pgfpathlineto{\pgfqpoint{-0.055556in}{0.000000in}}%
\pgfusepath{stroke,fill}%
}%
\begin{pgfscope}%
\pgfsys@transformshift{7.200000in}{3.150000in}%
\pgfsys@useobject{currentmarker}{}%
\end{pgfscope}%
\end{pgfscope}%
\begin{pgfscope}%
\pgftext[left,bottom,x=0.361274in,y=3.086971in,rotate=0.000000]{{\sffamily\fontsize{12.000000}{14.400000}\selectfont 0.0065}}
%
\end{pgfscope}%
\begin{pgfscope}%
\pgftext[left,bottom,x=0.291829in,y=0.932495in,rotate=90.000000]{{\sffamily\fontsize{12.000000}{14.400000}\selectfont diffusion coefficient}}
%
\end{pgfscope}%
\begin{pgfscope}%
\pgfsetrectcap%
\pgfsetroundjoin%
\pgfsetlinewidth{1.003750pt}%
\definecolor{currentstroke}{rgb}{0.000000,0.000000,0.000000}%
\pgfsetstrokecolor{currentstroke}%
\pgfsetdash{}{0pt}%
\pgfpathmoveto{\pgfqpoint{1.000000in}{3.150000in}}%
\pgfpathlineto{\pgfqpoint{7.200000in}{3.150000in}}%
\pgfusepath{stroke}%
\end{pgfscope}%
\begin{pgfscope}%
\pgfsetrectcap%
\pgfsetroundjoin%
\pgfsetlinewidth{1.003750pt}%
\definecolor{currentstroke}{rgb}{0.000000,0.000000,0.000000}%
\pgfsetstrokecolor{currentstroke}%
\pgfsetdash{}{0pt}%
\pgfpathmoveto{\pgfqpoint{7.200000in}{0.350000in}}%
\pgfpathlineto{\pgfqpoint{7.200000in}{3.150000in}}%
\pgfusepath{stroke}%
\end{pgfscope}%
\begin{pgfscope}%
\pgfsetrectcap%
\pgfsetroundjoin%
\pgfsetlinewidth{1.003750pt}%
\definecolor{currentstroke}{rgb}{0.000000,0.000000,0.000000}%
\pgfsetstrokecolor{currentstroke}%
\pgfsetdash{}{0pt}%
\pgfpathmoveto{\pgfqpoint{1.000000in}{0.350000in}}%
\pgfpathlineto{\pgfqpoint{7.200000in}{0.350000in}}%
\pgfusepath{stroke}%
\end{pgfscope}%
\begin{pgfscope}%
\pgfsetrectcap%
\pgfsetroundjoin%
\pgfsetlinewidth{1.003750pt}%
\definecolor{currentstroke}{rgb}{0.000000,0.000000,0.000000}%
\pgfsetstrokecolor{currentstroke}%
\pgfsetdash{}{0pt}%
\pgfpathmoveto{\pgfqpoint{1.000000in}{0.350000in}}%
\pgfpathlineto{\pgfqpoint{1.000000in}{3.150000in}}%
\pgfusepath{stroke}%
\end{pgfscope}%
\begin{pgfscope}%
\pgfsetrectcap%
\pgfsetroundjoin%
\definecolor{currentfill}{rgb}{1.000000,1.000000,1.000000}%
\pgfsetfillcolor{currentfill}%
\pgfsetlinewidth{1.003750pt}%
\definecolor{currentstroke}{rgb}{0.000000,0.000000,0.000000}%
\pgfsetstrokecolor{currentstroke}%
\pgfsetdash{}{0pt}%
\pgfpathmoveto{\pgfqpoint{1.069417in}{2.427606in}}%
\pgfpathlineto{\pgfqpoint{1.926808in}{2.427606in}}%
\pgfpathlineto{\pgfqpoint{1.926808in}{3.080583in}}%
\pgfpathlineto{\pgfqpoint{1.069417in}{3.080583in}}%
\pgfpathlineto{\pgfqpoint{1.069417in}{2.427606in}}%
\pgfpathclose%
\pgfusepath{stroke,fill}%
\end{pgfscope}%
\begin{pgfscope}%
\pgfsetrectcap%
\pgfsetroundjoin%
\pgfsetlinewidth{1.003750pt}%
\definecolor{currentstroke}{rgb}{0.000000,0.000000,1.000000}%
\pgfsetstrokecolor{currentstroke}%
\pgfsetdash{}{0pt}%
\pgfpathmoveto{\pgfqpoint{1.166600in}{2.968161in}}%
\pgfpathlineto{\pgfqpoint{1.360967in}{2.968161in}}%
\pgfusepath{stroke}%
\end{pgfscope}%
\begin{pgfscope}%
\pgftext[left,bottom,x=1.513683in,y=2.890691in,rotate=0.000000]{{\sffamily\fontsize{9.996000}{11.995200}\selectfont spc}}
%
\end{pgfscope}%
\begin{pgfscope}%
\pgfsetrectcap%
\pgfsetroundjoin%
\pgfsetlinewidth{1.003750pt}%
\definecolor{currentstroke}{rgb}{0.000000,0.500000,0.000000}%
\pgfsetstrokecolor{currentstroke}%
\pgfsetdash{}{0pt}%
\pgfpathmoveto{\pgfqpoint{1.166600in}{2.764385in}}%
\pgfpathlineto{\pgfqpoint{1.360967in}{2.764385in}}%
\pgfusepath{stroke}%
\end{pgfscope}%
\begin{pgfscope}%
\pgftext[left,bottom,x=1.513683in,y=2.686915in,rotate=0.000000]{{\sffamily\fontsize{9.996000}{11.995200}\selectfont spce}}
%
\end{pgfscope}%
\begin{pgfscope}%
\pgfsetrectcap%
\pgfsetroundjoin%
\pgfsetlinewidth{1.003750pt}%
\definecolor{currentstroke}{rgb}{1.000000,0.000000,0.000000}%
\pgfsetstrokecolor{currentstroke}%
\pgfsetdash{}{0pt}%
\pgfpathmoveto{\pgfqpoint{1.166600in}{2.560609in}}%
\pgfpathlineto{\pgfqpoint{1.360967in}{2.560609in}}%
\pgfusepath{stroke}%
\end{pgfscope}%
\begin{pgfscope}%
\pgftext[left,bottom,x=1.513683in,y=2.483139in,rotate=0.000000]{{\sffamily\fontsize{9.996000}{11.995200}\selectfont tip3p}}
%
\end{pgfscope}%
\end{pgfpicture}%
\makeatother%
\endgroup%
}
    \caption{Diffusion coefficient.} \label{fig:diffusion}
\end{figure}

The formula $< \Delta r(t)^2 > = 6 D t$ was used to to calculate the diffusion coefficient $D$. The average values were determined between $\unit[100]{ps}$ and $\unit[450]{ps}$.

\end{document}


% =============== Comments ============
\begin{comment}
\verb{x_init {}}

\begin{figure}[H]
	\resizebox{1\textwidth}{!{\input{../plots/GGA_mesh.pdf}}
	\caption{CAPTION}\label{fig:NAME}
\end{figure}
\end{comment}

\begin{figure}[H]
		\begin{subfigure}[a]{\textwidth}
			\resizebox{\linewidth}{!}{%% Creator: Matplotlib, PGF backend
%%
%% To include the figure in your LaTeX document, write
%%   \input{<filename>.pgf}
%%
%% Make sure the required packages are loaded in your preamble
%%   \usepackage{pgf}
%%
%% Figures using additional raster images can only be included by \input if
%% they are in the same directory as the main LaTeX file. For loading figures
%% from other directories you can use the `import` package
%%   \usepackage{import}
%% and then include the figures with
%%   \import{<path to file>}{<filename>.pgf}
%%
%% Matplotlib used the following preamble
%%   \usepackage{fontspec}
%%   \setmainfont{DejaVu Serif}
%%   \setsansfont{DejaVu Sans}
%%   \setmonofont{DejaVu Sans Mono}
%%
\begingroup%
\makeatletter%
\begin{pgfpicture}%
\pgfpathrectangle{\pgfpointorigin}{\pgfqpoint{8.000000in}{3.000000in}}%
\pgfusepath{use as bounding box}%
\begin{pgfscope}%
\pgfsetrectcap%
\pgfsetroundjoin%
\definecolor{currentfill}{rgb}{1.000000,1.000000,1.000000}%
\pgfsetfillcolor{currentfill}%
\pgfsetlinewidth{0.000000pt}%
\definecolor{currentstroke}{rgb}{1.000000,1.000000,1.000000}%
\pgfsetstrokecolor{currentstroke}%
\pgfsetdash{}{0pt}%
\pgfpathmoveto{\pgfqpoint{0.000000in}{0.000000in}}%
\pgfpathlineto{\pgfqpoint{8.000000in}{0.000000in}}%
\pgfpathlineto{\pgfqpoint{8.000000in}{3.000000in}}%
\pgfpathlineto{\pgfqpoint{0.000000in}{3.000000in}}%
\pgfpathclose%
\pgfusepath{fill}%
\end{pgfscope}%
\begin{pgfscope}%
\pgfsetrectcap%
\pgfsetroundjoin%
\definecolor{currentfill}{rgb}{1.000000,1.000000,1.000000}%
\pgfsetfillcolor{currentfill}%
\pgfsetlinewidth{0.000000pt}%
\definecolor{currentstroke}{rgb}{0.000000,0.000000,0.000000}%
\pgfsetstrokecolor{currentstroke}%
\pgfsetdash{}{0pt}%
\pgfpathmoveto{\pgfqpoint{1.000000in}{0.300000in}}%
\pgfpathlineto{\pgfqpoint{7.200000in}{0.300000in}}%
\pgfpathlineto{\pgfqpoint{7.200000in}{2.700000in}}%
\pgfpathlineto{\pgfqpoint{1.000000in}{2.700000in}}%
\pgfpathclose%
\pgfusepath{fill}%
\end{pgfscope}%
\begin{pgfscope}%
\pgfpathrectangle{\pgfqpoint{1.000000in}{0.300000in}}{\pgfqpoint{6.200000in}{2.400000in}} %
\pgfusepath{clip}%
\pgfsetrectcap%
\pgfsetroundjoin%
\pgfsetlinewidth{1.003750pt}%
\definecolor{currentstroke}{rgb}{0.000000,0.000000,1.000000}%
\pgfsetstrokecolor{currentstroke}%
\pgfsetdash{}{0pt}%
\pgfpathmoveto{\pgfqpoint{1.000000in}{0.490652in}}%
\pgfpathlineto{\pgfqpoint{1.466596in}{0.678595in}}%
\pgfpathlineto{\pgfqpoint{1.739538in}{0.776805in}}%
\pgfpathlineto{\pgfqpoint{1.933193in}{0.837675in}}%
\pgfpathlineto{\pgfqpoint{2.614159in}{1.018632in}}%
\pgfpathlineto{\pgfqpoint{2.822941in}{1.068905in}}%
\pgfpathlineto{\pgfqpoint{2.945672in}{1.103021in}}%
\pgfpathlineto{\pgfqpoint{3.311611in}{1.206745in}}%
\pgfpathlineto{\pgfqpoint{3.448665in}{1.243931in}}%
\pgfpathlineto{\pgfqpoint{3.562479in}{1.278696in}}%
\pgfpathlineto{\pgfqpoint{3.697562in}{1.318435in}}%
\pgfpathlineto{\pgfqpoint{3.810016in}{1.353491in}}%
\pgfpathlineto{\pgfqpoint{3.982633in}{1.403738in}}%
\pgfpathlineto{\pgfqpoint{4.006255in}{1.409618in}}%
\pgfpathlineto{\pgfqpoint{4.036514in}{1.418703in}}%
\pgfpathlineto{\pgfqpoint{4.188203in}{1.463418in}}%
\pgfpathlineto{\pgfqpoint{4.211417in}{1.470487in}}%
\pgfpathlineto{\pgfqpoint{4.244804in}{1.481321in}}%
\pgfpathlineto{\pgfqpoint{4.331290in}{1.507673in}}%
\pgfpathlineto{\pgfqpoint{4.345462in}{1.513228in}}%
\pgfpathlineto{\pgfqpoint{4.395014in}{1.527894in}}%
\pgfpathlineto{\pgfqpoint{4.453224in}{1.547018in}}%
\pgfpathlineto{\pgfqpoint{4.491922in}{1.557454in}}%
\pgfpathlineto{\pgfqpoint{4.950955in}{1.696155in}}%
\pgfpathlineto{\pgfqpoint{4.956636in}{1.697696in}}%
\pgfpathlineto{\pgfqpoint{4.969707in}{1.702707in}}%
\pgfpathlineto{\pgfqpoint{4.998665in}{1.710671in}}%
\pgfpathlineto{\pgfqpoint{5.017874in}{1.716260in}}%
\pgfpathlineto{\pgfqpoint{5.263288in}{1.791778in}}%
\pgfpathlineto{\pgfqpoint{5.270424in}{1.793495in}}%
\pgfpathlineto{\pgfqpoint{5.283313in}{1.797389in}}%
\pgfpathlineto{\pgfqpoint{5.300501in}{1.802135in}}%
\pgfpathlineto{\pgfqpoint{5.309492in}{1.805282in}}%
\pgfpathlineto{\pgfqpoint{5.347472in}{1.816202in}}%
\pgfpathlineto{\pgfqpoint{5.354818in}{1.818604in}}%
\pgfpathlineto{\pgfqpoint{5.380416in}{1.826285in}}%
\pgfpathlineto{\pgfqpoint{5.385421in}{1.827569in}}%
\pgfpathlineto{\pgfqpoint{5.394337in}{1.830562in}}%
\pgfpathlineto{\pgfqpoint{5.406044in}{1.833712in}}%
\pgfpathlineto{\pgfqpoint{5.432595in}{1.842333in}}%
\pgfpathlineto{\pgfqpoint{5.453647in}{1.847937in}}%
\pgfpathlineto{\pgfqpoint{5.461709in}{1.850274in}}%
\pgfpathlineto{\pgfqpoint{5.501470in}{1.862880in}}%
\pgfpathlineto{\pgfqpoint{5.505654in}{1.864186in}}%
\pgfpathlineto{\pgfqpoint{5.547689in}{1.876256in}}%
\pgfpathlineto{\pgfqpoint{5.553930in}{1.878220in}}%
\pgfpathlineto{\pgfqpoint{5.562416in}{1.880542in}}%
\pgfpathlineto{\pgfqpoint{5.574573in}{1.883765in}}%
\pgfpathlineto{\pgfqpoint{5.630190in}{1.900503in}}%
\pgfpathlineto{\pgfqpoint{5.633647in}{1.901690in}}%
\pgfpathlineto{\pgfqpoint{5.639142in}{1.903373in}}%
\pgfpathlineto{\pgfqpoint{5.642554in}{1.904314in}}%
\pgfpathlineto{\pgfqpoint{5.647977in}{1.905600in}}%
\pgfpathlineto{\pgfqpoint{5.651345in}{1.907215in}}%
\pgfpathlineto{\pgfqpoint{5.654696in}{1.908233in}}%
\pgfpathlineto{\pgfqpoint{5.680274in}{1.915580in}}%
\pgfpathlineto{\pgfqpoint{5.694915in}{1.920019in}}%
\pgfpathlineto{\pgfqpoint{5.698057in}{1.920621in}}%
\pgfpathlineto{\pgfqpoint{5.705536in}{1.922882in}}%
\pgfpathlineto{\pgfqpoint{5.714159in}{1.925454in}}%
\pgfpathlineto{\pgfqpoint{5.716602in}{1.926911in}}%
\pgfpathlineto{\pgfqpoint{5.727489in}{1.929980in}}%
\pgfpathlineto{\pgfqpoint{5.735244in}{1.932092in}}%
\pgfpathlineto{\pgfqpoint{5.741149in}{1.933727in}}%
\pgfpathlineto{\pgfqpoint{5.746419in}{1.935760in}}%
\pgfpathlineto{\pgfqpoint{5.749329in}{1.936639in}}%
\pgfpathlineto{\pgfqpoint{5.755112in}{1.938305in}}%
\pgfpathlineto{\pgfqpoint{5.758559in}{1.939157in}}%
\pgfpathlineto{\pgfqpoint{5.769920in}{1.942704in}}%
\pgfpathlineto{\pgfqpoint{5.784409in}{1.947343in}}%
\pgfpathlineto{\pgfqpoint{5.790447in}{1.949074in}}%
\pgfpathlineto{\pgfqpoint{5.796972in}{1.950721in}}%
\pgfpathlineto{\pgfqpoint{5.810366in}{1.954884in}}%
\pgfpathlineto{\pgfqpoint{5.813541in}{1.955664in}}%
\pgfpathlineto{\pgfqpoint{5.822978in}{1.958735in}}%
\pgfpathlineto{\pgfqpoint{5.829712in}{1.961101in}}%
\pgfpathlineto{\pgfqpoint{5.862408in}{1.970415in}}%
\pgfpathlineto{\pgfqpoint{5.866325in}{1.971851in}}%
\pgfpathlineto{\pgfqpoint{5.909800in}{1.984563in}}%
\pgfpathlineto{\pgfqpoint{5.914360in}{1.986203in}}%
\pgfpathlineto{\pgfqpoint{5.929640in}{1.991107in}}%
\pgfpathlineto{\pgfqpoint{5.932743in}{1.991880in}}%
\pgfpathlineto{\pgfqpoint{5.952786in}{1.997284in}}%
\pgfpathlineto{\pgfqpoint{5.957066in}{1.998419in}}%
\pgfpathlineto{\pgfqpoint{5.972250in}{2.003475in}}%
\pgfpathlineto{\pgfqpoint{5.975579in}{2.004147in}}%
\pgfpathlineto{\pgfqpoint{5.994807in}{2.009010in}}%
\pgfpathlineto{\pgfqpoint{5.999229in}{2.010413in}}%
\pgfpathlineto{\pgfqpoint{6.010356in}{2.013824in}}%
\pgfpathlineto{\pgfqpoint{6.025939in}{2.017674in}}%
\pgfpathlineto{\pgfqpoint{6.033217in}{2.019566in}}%
\pgfpathlineto{\pgfqpoint{6.036257in}{2.020037in}}%
\pgfpathlineto{\pgfqpoint{6.040039in}{2.021398in}}%
\pgfpathlineto{\pgfqpoint{6.041922in}{2.022084in}}%
\pgfpathlineto{\pgfqpoint{6.070274in}{2.029160in}}%
\pgfpathlineto{\pgfqpoint{6.072793in}{2.029274in}}%
\pgfpathlineto{\pgfqpoint{6.079228in}{2.031932in}}%
\pgfpathlineto{\pgfqpoint{6.103340in}{2.038358in}}%
\pgfpathlineto{\pgfqpoint{6.105397in}{2.038453in}}%
\pgfpathlineto{\pgfqpoint{6.108129in}{2.038930in}}%
\pgfpathlineto{\pgfqpoint{6.111868in}{2.040383in}}%
\pgfpathlineto{\pgfqpoint{6.113898in}{2.041364in}}%
\pgfpathlineto{\pgfqpoint{6.131245in}{2.045936in}}%
\pgfpathlineto{\pgfqpoint{6.133547in}{2.046757in}}%
\pgfpathlineto{\pgfqpoint{6.137474in}{2.048063in}}%
\pgfpathlineto{\pgfqpoint{6.142351in}{2.048772in}}%
\pgfpathlineto{\pgfqpoint{6.145260in}{2.049656in}}%
\pgfpathlineto{\pgfqpoint{6.159936in}{2.054396in}}%
\pgfpathlineto{\pgfqpoint{6.164340in}{2.055031in}}%
\pgfpathlineto{\pgfqpoint{6.177381in}{2.058923in}}%
\pgfpathlineto{\pgfqpoint{6.180143in}{2.060203in}}%
\pgfpathlineto{\pgfqpoint{6.187756in}{2.062394in}}%
\pgfpathlineto{\pgfqpoint{6.190174in}{2.063065in}}%
\pgfpathlineto{\pgfqpoint{6.192884in}{2.063385in}}%
\pgfpathlineto{\pgfqpoint{6.195284in}{2.063814in}}%
\pgfpathlineto{\pgfqpoint{6.202729in}{2.065390in}}%
\pgfpathlineto{\pgfqpoint{6.204504in}{2.066239in}}%
\pgfpathlineto{\pgfqpoint{6.218245in}{2.070677in}}%
\pgfpathlineto{\pgfqpoint{6.241008in}{2.077864in}}%
\pgfpathlineto{\pgfqpoint{6.244079in}{2.078182in}}%
\pgfpathlineto{\pgfqpoint{6.263568in}{2.082853in}}%
\pgfpathlineto{\pgfqpoint{6.266538in}{2.084775in}}%
\pgfpathlineto{\pgfqpoint{6.268152in}{2.085261in}}%
\pgfpathlineto{\pgfqpoint{6.271369in}{2.085694in}}%
\pgfpathlineto{\pgfqpoint{6.276962in}{2.087250in}}%
\pgfpathlineto{\pgfqpoint{6.278816in}{2.087385in}}%
\pgfpathlineto{\pgfqpoint{6.283560in}{2.088839in}}%
\pgfpathlineto{\pgfqpoint{6.286705in}{2.088934in}}%
\pgfpathlineto{\pgfqpoint{6.288532in}{2.089970in}}%
\pgfpathlineto{\pgfqpoint{6.290355in}{2.090515in}}%
\pgfpathlineto{\pgfqpoint{6.292950in}{2.091989in}}%
\pgfpathlineto{\pgfqpoint{6.294502in}{2.092319in}}%
\pgfpathlineto{\pgfqpoint{6.301954in}{2.094096in}}%
\pgfpathlineto{\pgfqpoint{6.302976in}{2.094429in}}%
\pgfpathlineto{\pgfqpoint{6.304505in}{2.094523in}}%
\pgfpathlineto{\pgfqpoint{6.308567in}{2.095866in}}%
\pgfpathlineto{\pgfqpoint{6.334871in}{2.102772in}}%
\pgfpathlineto{\pgfqpoint{6.337058in}{2.102965in}}%
\pgfpathlineto{\pgfqpoint{6.339237in}{2.104060in}}%
\pgfpathlineto{\pgfqpoint{6.342854in}{2.105757in}}%
\pgfpathlineto{\pgfqpoint{6.344536in}{2.106093in}}%
\pgfpathlineto{\pgfqpoint{6.347170in}{2.106529in}}%
\pgfpathlineto{\pgfqpoint{6.392002in}{2.119209in}}%
\pgfpathlineto{\pgfqpoint{6.393565in}{2.119220in}}%
\pgfpathlineto{\pgfqpoint{6.396902in}{2.120088in}}%
\pgfpathlineto{\pgfqpoint{6.399560in}{2.120643in}}%
\pgfpathlineto{\pgfqpoint{6.401327in}{2.121254in}}%
\pgfpathlineto{\pgfqpoint{6.410959in}{2.124137in}}%
\pgfpathlineto{\pgfqpoint{6.414860in}{2.125281in}}%
\pgfpathlineto{\pgfqpoint{6.416371in}{2.125686in}}%
\pgfpathlineto{\pgfqpoint{6.421954in}{2.126974in}}%
\pgfpathlineto{\pgfqpoint{6.426004in}{2.127602in}}%
\pgfpathlineto{\pgfqpoint{6.427491in}{2.127653in}}%
\pgfpathlineto{\pgfqpoint{6.430242in}{2.128678in}}%
\pgfpathlineto{\pgfqpoint{6.440929in}{2.132036in}}%
\pgfpathlineto{\pgfqpoint{6.444040in}{2.133084in}}%
\pgfpathlineto{\pgfqpoint{6.451244in}{2.134426in}}%
\pgfpathlineto{\pgfqpoint{6.453697in}{2.135367in}}%
\pgfpathlineto{\pgfqpoint{6.455733in}{2.136042in}}%
\pgfpathlineto{\pgfqpoint{6.458372in}{2.137109in}}%
\pgfpathlineto{\pgfqpoint{6.460395in}{2.137370in}}%
\pgfpathlineto{\pgfqpoint{6.465826in}{2.139498in}}%
\pgfpathlineto{\pgfqpoint{6.467227in}{2.139225in}}%
\pgfpathlineto{\pgfqpoint{6.468425in}{2.139067in}}%
\pgfpathlineto{\pgfqpoint{6.474782in}{2.141288in}}%
\pgfpathlineto{\pgfqpoint{6.476164in}{2.142283in}}%
\pgfpathlineto{\pgfqpoint{6.478724in}{2.142863in}}%
\pgfpathlineto{\pgfqpoint{6.480098in}{2.143210in}}%
\pgfpathlineto{\pgfqpoint{6.485179in}{2.145018in}}%
\pgfpathlineto{\pgfqpoint{6.486928in}{2.144984in}}%
\pgfpathlineto{\pgfqpoint{6.487705in}{2.144757in}}%
\pgfpathlineto{\pgfqpoint{6.488092in}{2.144965in}}%
\pgfpathlineto{\pgfqpoint{6.493305in}{2.147049in}}%
\pgfpathlineto{\pgfqpoint{6.494458in}{2.146772in}}%
\pgfpathlineto{\pgfqpoint{6.498288in}{2.148178in}}%
\pgfpathlineto{\pgfqpoint{6.500004in}{2.148699in}}%
\pgfpathlineto{\pgfqpoint{6.502664in}{2.149678in}}%
\pgfpathlineto{\pgfqpoint{6.503612in}{2.150240in}}%
\pgfpathlineto{\pgfqpoint{6.505881in}{2.150714in}}%
\pgfpathlineto{\pgfqpoint{6.507201in}{2.150395in}}%
\pgfpathlineto{\pgfqpoint{6.512269in}{2.152104in}}%
\pgfpathlineto{\pgfqpoint{6.513950in}{2.152363in}}%
\pgfpathlineto{\pgfqpoint{6.516184in}{2.153438in}}%
\pgfpathlineto{\pgfqpoint{6.539748in}{2.160542in}}%
\pgfpathlineto{\pgfqpoint{6.541362in}{2.161838in}}%
\pgfpathlineto{\pgfqpoint{6.542793in}{2.161362in}}%
\pgfpathlineto{\pgfqpoint{6.544043in}{2.161378in}}%
\pgfpathlineto{\pgfqpoint{6.549019in}{2.162800in}}%
\pgfpathlineto{\pgfqpoint{6.552551in}{2.164128in}}%
\pgfpathlineto{\pgfqpoint{6.555538in}{2.165406in}}%
\pgfpathlineto{\pgfqpoint{6.557114in}{2.165562in}}%
\pgfpathlineto{\pgfqpoint{6.560256in}{2.167340in}}%
\pgfpathlineto{\pgfqpoint{6.561126in}{2.166801in}}%
\pgfpathlineto{\pgfqpoint{6.561474in}{2.166970in}}%
\pgfpathlineto{\pgfqpoint{6.564595in}{2.167948in}}%
\pgfpathlineto{\pgfqpoint{6.565805in}{2.168142in}}%
\pgfpathlineto{\pgfqpoint{6.570966in}{2.169870in}}%
\pgfpathlineto{\pgfqpoint{6.572848in}{2.170470in}}%
\pgfpathlineto{\pgfqpoint{6.573702in}{2.170172in}}%
\pgfpathlineto{\pgfqpoint{6.574043in}{2.170764in}}%
\pgfpathlineto{\pgfqpoint{6.575066in}{2.171476in}}%
\pgfpathlineto{\pgfqpoint{6.576937in}{2.173129in}}%
\pgfpathlineto{\pgfqpoint{6.577107in}{2.172978in}}%
\pgfpathlineto{\pgfqpoint{6.578634in}{2.172310in}}%
\pgfpathlineto{\pgfqpoint{6.582856in}{2.173352in}}%
\pgfpathlineto{\pgfqpoint{6.583866in}{2.174314in}}%
\pgfpathlineto{\pgfqpoint{6.585713in}{2.175049in}}%
\pgfpathlineto{\pgfqpoint{6.586048in}{2.174278in}}%
\pgfpathlineto{\pgfqpoint{6.586885in}{2.174965in}}%
\pgfpathlineto{\pgfqpoint{6.588557in}{2.175851in}}%
\pgfpathlineto{\pgfqpoint{6.590058in}{2.175563in}}%
\pgfpathlineto{\pgfqpoint{6.596028in}{2.177899in}}%
\pgfpathlineto{\pgfqpoint{6.597183in}{2.178310in}}%
\pgfpathlineto{\pgfqpoint{6.598171in}{2.178648in}}%
\pgfpathlineto{\pgfqpoint{6.598336in}{2.178407in}}%
\pgfpathlineto{\pgfqpoint{6.599651in}{2.178726in}}%
\pgfpathlineto{\pgfqpoint{6.600471in}{2.180028in}}%
\pgfpathlineto{\pgfqpoint{6.602273in}{2.180906in}}%
\pgfpathlineto{\pgfqpoint{6.602437in}{2.180676in}}%
\pgfpathlineto{\pgfqpoint{6.602600in}{2.180206in}}%
\pgfpathlineto{\pgfqpoint{6.603417in}{2.180624in}}%
\pgfpathlineto{\pgfqpoint{6.605048in}{2.181798in}}%
\pgfpathlineto{\pgfqpoint{6.605211in}{2.181616in}}%
\pgfpathlineto{\pgfqpoint{6.607975in}{2.182094in}}%
\pgfpathlineto{\pgfqpoint{6.609595in}{2.183868in}}%
\pgfpathlineto{\pgfqpoint{6.612984in}{2.183317in}}%
\pgfpathlineto{\pgfqpoint{6.614432in}{2.184199in}}%
\pgfpathlineto{\pgfqpoint{6.615716in}{2.184056in}}%
\pgfpathlineto{\pgfqpoint{6.617797in}{2.185624in}}%
\pgfpathlineto{\pgfqpoint{6.618277in}{2.185843in}}%
\pgfpathlineto{\pgfqpoint{6.618596in}{2.185276in}}%
\pgfpathlineto{\pgfqpoint{6.619553in}{2.185081in}}%
\pgfpathlineto{\pgfqpoint{6.619713in}{2.185391in}}%
\pgfpathlineto{\pgfqpoint{6.620509in}{2.186252in}}%
\pgfpathlineto{\pgfqpoint{6.622258in}{2.187613in}}%
\pgfpathlineto{\pgfqpoint{6.622417in}{2.187389in}}%
\pgfpathlineto{\pgfqpoint{6.623052in}{2.186987in}}%
\pgfpathlineto{\pgfqpoint{6.623528in}{2.187721in}}%
\pgfpathlineto{\pgfqpoint{6.625111in}{2.188748in}}%
\pgfpathlineto{\pgfqpoint{6.626059in}{2.189038in}}%
\pgfpathlineto{\pgfqpoint{6.626217in}{2.188846in}}%
\pgfpathlineto{\pgfqpoint{6.627478in}{2.187994in}}%
\pgfpathlineto{\pgfqpoint{6.627636in}{2.188084in}}%
\pgfpathlineto{\pgfqpoint{6.629367in}{2.188924in}}%
\pgfpathlineto{\pgfqpoint{6.630779in}{2.189491in}}%
\pgfpathlineto{\pgfqpoint{6.632189in}{2.189305in}}%
\pgfpathlineto{\pgfqpoint{6.633908in}{2.190838in}}%
\pgfpathlineto{\pgfqpoint{6.634064in}{2.190760in}}%
\pgfpathlineto{\pgfqpoint{6.635311in}{2.189722in}}%
\pgfpathlineto{\pgfqpoint{6.635467in}{2.190004in}}%
\pgfpathlineto{\pgfqpoint{6.641667in}{2.193905in}}%
\pgfpathlineto{\pgfqpoint{6.644131in}{2.192785in}}%
\pgfpathlineto{\pgfqpoint{6.651622in}{2.195961in}}%
\pgfpathlineto{\pgfqpoint{6.653292in}{2.197653in}}%
\pgfpathlineto{\pgfqpoint{6.653444in}{2.197419in}}%
\pgfpathlineto{\pgfqpoint{6.654201in}{2.196954in}}%
\pgfpathlineto{\pgfqpoint{6.654655in}{2.197303in}}%
\pgfpathlineto{\pgfqpoint{6.657072in}{2.198880in}}%
\pgfpathlineto{\pgfqpoint{6.657223in}{2.198727in}}%
\pgfpathlineto{\pgfqpoint{6.659030in}{2.198228in}}%
\pgfpathlineto{\pgfqpoint{6.659481in}{2.198140in}}%
\pgfpathlineto{\pgfqpoint{6.659932in}{2.198542in}}%
\pgfpathlineto{\pgfqpoint{6.661581in}{2.198542in}}%
\pgfpathlineto{\pgfqpoint{6.662479in}{2.197958in}}%
\pgfpathlineto{\pgfqpoint{6.662779in}{2.198948in}}%
\pgfpathlineto{\pgfqpoint{6.663675in}{2.199790in}}%
\pgfpathlineto{\pgfqpoint{6.664123in}{2.199634in}}%
\pgfpathlineto{\pgfqpoint{6.664421in}{2.200219in}}%
\pgfpathlineto{\pgfqpoint{6.664869in}{2.199462in}}%
\pgfpathlineto{\pgfqpoint{6.664869in}{2.199462in}}%
\pgfpathlineto{\pgfqpoint{6.665018in}{2.199014in}}%
\pgfpathlineto{\pgfqpoint{6.665316in}{2.199485in}}%
\pgfpathlineto{\pgfqpoint{6.665316in}{2.199485in}}%
\pgfpathlineto{\pgfqpoint{6.667546in}{2.202355in}}%
\pgfpathlineto{\pgfqpoint{6.668881in}{2.203152in}}%
\pgfpathlineto{\pgfqpoint{6.669029in}{2.202673in}}%
\pgfpathlineto{\pgfqpoint{6.670213in}{2.203476in}}%
\pgfpathlineto{\pgfqpoint{6.671986in}{2.204608in}}%
\pgfpathlineto{\pgfqpoint{6.677275in}{2.202886in}}%
\pgfpathlineto{\pgfqpoint{6.678444in}{2.204860in}}%
\pgfpathlineto{\pgfqpoint{6.678590in}{2.204448in}}%
\pgfpathlineto{\pgfqpoint{6.679320in}{2.204816in}}%
\pgfpathlineto{\pgfqpoint{6.679612in}{2.204240in}}%
\pgfpathlineto{\pgfqpoint{6.679758in}{2.203927in}}%
\pgfpathlineto{\pgfqpoint{6.679904in}{2.204343in}}%
\pgfpathlineto{\pgfqpoint{6.679904in}{2.204343in}}%
\pgfpathlineto{\pgfqpoint{6.685998in}{2.209760in}}%
\pgfpathlineto{\pgfqpoint{6.686431in}{2.209102in}}%
\pgfpathlineto{\pgfqpoint{6.688881in}{2.207784in}}%
\pgfpathlineto{\pgfqpoint{6.689600in}{2.209137in}}%
\pgfpathlineto{\pgfqpoint{6.690461in}{2.208545in}}%
\pgfpathlineto{\pgfqpoint{6.691752in}{2.207458in}}%
\pgfpathlineto{\pgfqpoint{6.692038in}{2.208328in}}%
\pgfpathlineto{\pgfqpoint{6.693611in}{2.210186in}}%
\pgfpathlineto{\pgfqpoint{6.694895in}{2.209148in}}%
\pgfpathlineto{\pgfqpoint{6.695038in}{2.209693in}}%
\pgfpathlineto{\pgfqpoint{6.695465in}{2.210773in}}%
\pgfpathlineto{\pgfqpoint{6.700291in}{2.216004in}}%
\pgfpathlineto{\pgfqpoint{6.700432in}{2.215673in}}%
\pgfpathlineto{\pgfqpoint{6.701845in}{2.213630in}}%
\pgfpathlineto{\pgfqpoint{6.702268in}{2.214524in}}%
\pgfpathlineto{\pgfqpoint{6.702972in}{2.214140in}}%
\pgfpathlineto{\pgfqpoint{6.703113in}{2.213787in}}%
\pgfpathlineto{\pgfqpoint{6.703536in}{2.214169in}}%
\pgfpathlineto{\pgfqpoint{6.703536in}{2.214169in}}%
\pgfpathlineto{\pgfqpoint{6.703817in}{2.215381in}}%
\pgfpathlineto{\pgfqpoint{6.704661in}{2.214629in}}%
\pgfpathlineto{\pgfqpoint{6.706905in}{2.215050in}}%
\pgfpathlineto{\pgfqpoint{6.707045in}{2.215327in}}%
\pgfpathlineto{\pgfqpoint{6.707045in}{2.215327in}}%
\pgfpathlineto{\pgfqpoint{6.707045in}{2.215327in}}%
\pgfpathlineto{\pgfqpoint{6.707185in}{2.214886in}}%
\pgfpathlineto{\pgfqpoint{6.707605in}{2.215443in}}%
\pgfpathlineto{\pgfqpoint{6.707605in}{2.215443in}}%
\pgfpathlineto{\pgfqpoint{6.707745in}{2.215585in}}%
\pgfpathlineto{\pgfqpoint{6.707884in}{2.215429in}}%
\pgfpathlineto{\pgfqpoint{6.707884in}{2.215429in}}%
\pgfpathlineto{\pgfqpoint{6.708304in}{2.213859in}}%
\pgfpathlineto{\pgfqpoint{6.709002in}{2.214235in}}%
\pgfpathlineto{\pgfqpoint{6.711093in}{2.218737in}}%
\pgfpathlineto{\pgfqpoint{6.711927in}{2.218740in}}%
\pgfpathlineto{\pgfqpoint{6.712066in}{2.217961in}}%
\pgfpathlineto{\pgfqpoint{6.712205in}{2.218065in}}%
\pgfpathlineto{\pgfqpoint{6.712205in}{2.218065in}}%
\pgfpathlineto{\pgfqpoint{6.713455in}{2.219377in}}%
\pgfpathlineto{\pgfqpoint{6.714286in}{2.221357in}}%
\pgfpathlineto{\pgfqpoint{6.714840in}{2.219542in}}%
\pgfpathlineto{\pgfqpoint{6.717464in}{2.217770in}}%
\pgfpathlineto{\pgfqpoint{6.717602in}{2.218182in}}%
\pgfpathlineto{\pgfqpoint{6.718842in}{2.219703in}}%
\pgfpathlineto{\pgfqpoint{6.718979in}{2.219371in}}%
\pgfpathlineto{\pgfqpoint{6.719117in}{2.219549in}}%
\pgfpathlineto{\pgfqpoint{6.719117in}{2.219549in}}%
\pgfpathlineto{\pgfqpoint{6.719254in}{2.219912in}}%
\pgfpathlineto{\pgfqpoint{6.719254in}{2.219912in}}%
\pgfpathlineto{\pgfqpoint{6.719254in}{2.219912in}}%
\pgfpathlineto{\pgfqpoint{6.719529in}{2.219265in}}%
\pgfpathlineto{\pgfqpoint{6.719804in}{2.219508in}}%
\pgfpathlineto{\pgfqpoint{6.719804in}{2.219508in}}%
\pgfpathlineto{\pgfqpoint{6.720765in}{2.222614in}}%
\pgfpathlineto{\pgfqpoint{6.721039in}{2.221647in}}%
\pgfpathlineto{\pgfqpoint{6.722272in}{2.218901in}}%
\pgfpathlineto{\pgfqpoint{6.722819in}{2.219120in}}%
\pgfpathlineto{\pgfqpoint{6.725140in}{2.224938in}}%
\pgfpathlineto{\pgfqpoint{6.725277in}{2.223782in}}%
\pgfpathlineto{\pgfqpoint{6.725413in}{2.223461in}}%
\pgfpathlineto{\pgfqpoint{6.725413in}{2.223461in}}%
\pgfpathlineto{\pgfqpoint{6.725413in}{2.223461in}}%
\pgfpathlineto{\pgfqpoint{6.726774in}{2.225884in}}%
\pgfpathlineto{\pgfqpoint{6.727997in}{2.227680in}}%
\pgfpathlineto{\pgfqpoint{6.728132in}{2.226928in}}%
\pgfpathlineto{\pgfqpoint{6.729488in}{2.224938in}}%
\pgfpathlineto{\pgfqpoint{6.729623in}{2.224753in}}%
\pgfpathlineto{\pgfqpoint{6.729623in}{2.224753in}}%
\pgfpathlineto{\pgfqpoint{6.729623in}{2.224753in}}%
\pgfpathlineto{\pgfqpoint{6.729894in}{2.226158in}}%
\pgfpathlineto{\pgfqpoint{6.730570in}{2.225070in}}%
\pgfpathlineto{\pgfqpoint{6.730976in}{2.223766in}}%
\pgfpathlineto{\pgfqpoint{6.731246in}{2.224455in}}%
\pgfpathlineto{\pgfqpoint{6.731246in}{2.224455in}}%
\pgfpathlineto{\pgfqpoint{6.732460in}{2.226528in}}%
\pgfpathlineto{\pgfqpoint{6.732999in}{2.225536in}}%
\pgfpathlineto{\pgfqpoint{6.733404in}{2.226469in}}%
\pgfusepath{stroke}%
\end{pgfscope}%
\begin{pgfscope}%
\pgfpathrectangle{\pgfqpoint{1.000000in}{0.300000in}}{\pgfqpoint{6.200000in}{2.400000in}} %
\pgfusepath{clip}%
\pgfsetrectcap%
\pgfsetroundjoin%
\pgfsetlinewidth{1.003750pt}%
\definecolor{currentstroke}{rgb}{0.000000,0.500000,0.000000}%
\pgfsetstrokecolor{currentstroke}%
\pgfsetdash{}{0pt}%
\pgfpathmoveto{\pgfqpoint{1.000000in}{0.479596in}}%
\pgfpathlineto{\pgfqpoint{1.466596in}{0.659706in}}%
\pgfpathlineto{\pgfqpoint{1.739538in}{0.747768in}}%
\pgfpathlineto{\pgfqpoint{1.933193in}{0.797601in}}%
\pgfpathlineto{\pgfqpoint{2.550000in}{0.933805in}}%
\pgfpathlineto{\pgfqpoint{2.672731in}{0.960515in}}%
\pgfpathlineto{\pgfqpoint{2.907196in}{1.014980in}}%
\pgfpathlineto{\pgfqpoint{3.110678in}{1.066823in}}%
\pgfpathlineto{\pgfqpoint{3.193209in}{1.089701in}}%
\pgfpathlineto{\pgfqpoint{3.448665in}{1.163300in}}%
\pgfpathlineto{\pgfqpoint{3.516036in}{1.180984in}}%
\pgfpathlineto{\pgfqpoint{3.633404in}{1.213700in}}%
\pgfpathlineto{\pgfqpoint{3.685210in}{1.226580in}}%
\pgfpathlineto{\pgfqpoint{3.767261in}{1.251933in}}%
\pgfpathlineto{\pgfqpoint{3.820293in}{1.268300in}}%
\pgfpathlineto{\pgfqpoint{3.949789in}{1.302819in}}%
\pgfpathlineto{\pgfqpoint{4.199910in}{1.379119in}}%
\pgfpathlineto{\pgfqpoint{4.255574in}{1.393159in}}%
\pgfpathlineto{\pgfqpoint{4.271414in}{1.399346in}}%
\pgfpathlineto{\pgfqpoint{4.302017in}{1.408146in}}%
\pgfpathlineto{\pgfqpoint{4.331290in}{1.416301in}}%
\pgfpathlineto{\pgfqpoint{4.359342in}{1.425235in}}%
\pgfpathlineto{\pgfqpoint{4.372941in}{1.428946in}}%
\pgfpathlineto{\pgfqpoint{4.395014in}{1.434524in}}%
\pgfpathlineto{\pgfqpoint{4.453224in}{1.452799in}}%
\pgfpathlineto{\pgfqpoint{4.517745in}{1.471600in}}%
\pgfpathlineto{\pgfqpoint{4.528516in}{1.475431in}}%
\pgfpathlineto{\pgfqpoint{4.539117in}{1.479185in}}%
\pgfpathlineto{\pgfqpoint{4.573295in}{1.488764in}}%
\pgfpathlineto{\pgfqpoint{4.596227in}{1.496721in}}%
\pgfpathlineto{\pgfqpoint{4.686517in}{1.523112in}}%
\pgfpathlineto{\pgfqpoint{4.703207in}{1.528736in}}%
\pgfpathlineto{\pgfqpoint{4.722171in}{1.534127in}}%
\pgfpathlineto{\pgfqpoint{4.732772in}{1.536441in}}%
\pgfpathlineto{\pgfqpoint{4.761092in}{1.545700in}}%
\pgfpathlineto{\pgfqpoint{4.783409in}{1.552366in}}%
\pgfpathlineto{\pgfqpoint{4.835035in}{1.568195in}}%
\pgfpathlineto{\pgfqpoint{4.844011in}{1.570761in}}%
\pgfpathlineto{\pgfqpoint{4.885083in}{1.583742in}}%
\pgfpathlineto{\pgfqpoint{4.949051in}{1.603028in}}%
\pgfpathlineto{\pgfqpoint{4.956636in}{1.605310in}}%
\pgfpathlineto{\pgfqpoint{4.978891in}{1.612508in}}%
\pgfpathlineto{\pgfqpoint{4.986149in}{1.614374in}}%
\pgfpathlineto{\pgfqpoint{4.993329in}{1.617488in}}%
\pgfpathlineto{\pgfqpoint{5.000434in}{1.619183in}}%
\pgfpathlineto{\pgfqpoint{5.009210in}{1.622085in}}%
\pgfpathlineto{\pgfqpoint{5.043215in}{1.633270in}}%
\pgfpathlineto{\pgfqpoint{5.051455in}{1.635868in}}%
\pgfpathlineto{\pgfqpoint{5.062823in}{1.639198in}}%
\pgfpathlineto{\pgfqpoint{5.074003in}{1.641954in}}%
\pgfpathlineto{\pgfqpoint{5.088109in}{1.647035in}}%
\pgfpathlineto{\pgfqpoint{5.106469in}{1.652707in}}%
\pgfpathlineto{\pgfqpoint{5.147456in}{1.665077in}}%
\pgfpathlineto{\pgfqpoint{5.153113in}{1.667082in}}%
\pgfpathlineto{\pgfqpoint{5.161510in}{1.669145in}}%
\pgfpathlineto{\pgfqpoint{5.252440in}{1.697068in}}%
\pgfpathlineto{\pgfqpoint{5.259691in}{1.699665in}}%
\pgfpathlineto{\pgfqpoint{5.364146in}{1.731708in}}%
\pgfpathlineto{\pgfqpoint{5.369272in}{1.733604in}}%
\pgfpathlineto{\pgfqpoint{5.379410in}{1.736908in}}%
\pgfpathlineto{\pgfqpoint{5.382422in}{1.737538in}}%
\pgfpathlineto{\pgfqpoint{5.392366in}{1.741402in}}%
\pgfpathlineto{\pgfqpoint{5.398262in}{1.742668in}}%
\pgfpathlineto{\pgfqpoint{5.425115in}{1.751773in}}%
\pgfpathlineto{\pgfqpoint{5.430733in}{1.753310in}}%
\pgfpathlineto{\pgfqpoint{5.438150in}{1.755758in}}%
\pgfpathlineto{\pgfqpoint{5.459032in}{1.761687in}}%
\pgfpathlineto{\pgfqpoint{5.465261in}{1.763580in}}%
\pgfpathlineto{\pgfqpoint{5.485331in}{1.770267in}}%
\pgfpathlineto{\pgfqpoint{5.518870in}{1.780770in}}%
\pgfpathlineto{\pgfqpoint{5.522947in}{1.782075in}}%
\pgfpathlineto{\pgfqpoint{5.537424in}{1.786760in}}%
\pgfpathlineto{\pgfqpoint{5.542971in}{1.787668in}}%
\pgfpathlineto{\pgfqpoint{5.549255in}{1.789657in}}%
\pgfpathlineto{\pgfqpoint{5.553930in}{1.791037in}}%
\pgfpathlineto{\pgfqpoint{5.580570in}{1.799782in}}%
\pgfpathlineto{\pgfqpoint{5.587992in}{1.802328in}}%
\pgfpathlineto{\pgfqpoint{5.615472in}{1.810782in}}%
\pgfpathlineto{\pgfqpoint{5.617594in}{1.810897in}}%
\pgfpathlineto{\pgfqpoint{5.623221in}{1.812565in}}%
\pgfpathlineto{\pgfqpoint{5.627411in}{1.813759in}}%
\pgfpathlineto{\pgfqpoint{5.646626in}{1.819923in}}%
\pgfpathlineto{\pgfqpoint{5.652687in}{1.822527in}}%
\pgfpathlineto{\pgfqpoint{5.658695in}{1.824426in}}%
\pgfpathlineto{\pgfqpoint{5.665307in}{1.826128in}}%
\pgfpathlineto{\pgfqpoint{5.669244in}{1.827519in}}%
\pgfpathlineto{\pgfqpoint{5.678986in}{1.830939in}}%
\pgfpathlineto{\pgfqpoint{5.697430in}{1.835807in}}%
\pgfpathlineto{\pgfqpoint{5.701807in}{1.837529in}}%
\pgfpathlineto{\pgfqpoint{5.722068in}{1.844395in}}%
\pgfpathlineto{\pgfqpoint{5.728688in}{1.845831in}}%
\pgfpathlineto{\pgfqpoint{5.732272in}{1.847337in}}%
\pgfpathlineto{\pgfqpoint{5.738203in}{1.849341in}}%
\pgfpathlineto{\pgfqpoint{5.742910in}{1.850683in}}%
\pgfpathlineto{\pgfqpoint{5.745835in}{1.851165in}}%
\pgfpathlineto{\pgfqpoint{5.749329in}{1.852049in}}%
\pgfpathlineto{\pgfqpoint{5.776646in}{1.860705in}}%
\pgfpathlineto{\pgfqpoint{5.780539in}{1.862226in}}%
\pgfpathlineto{\pgfqpoint{5.792084in}{1.865512in}}%
\pgfpathlineto{\pgfqpoint{5.794804in}{1.866629in}}%
\pgfpathlineto{\pgfqpoint{5.806108in}{1.869476in}}%
\pgfpathlineto{\pgfqpoint{5.809835in}{1.870446in}}%
\pgfpathlineto{\pgfqpoint{5.813013in}{1.871819in}}%
\pgfpathlineto{\pgfqpoint{5.817226in}{1.873008in}}%
\pgfpathlineto{\pgfqpoint{5.820892in}{1.873906in}}%
\pgfpathlineto{\pgfqpoint{5.824018in}{1.874763in}}%
\pgfpathlineto{\pgfqpoint{5.829196in}{1.876692in}}%
\pgfpathlineto{\pgfqpoint{5.841970in}{1.880815in}}%
\pgfpathlineto{\pgfqpoint{5.845504in}{1.882184in}}%
\pgfpathlineto{\pgfqpoint{5.851019in}{1.883818in}}%
\pgfpathlineto{\pgfqpoint{5.853512in}{1.884259in}}%
\pgfpathlineto{\pgfqpoint{5.855995in}{1.884634in}}%
\pgfpathlineto{\pgfqpoint{5.870219in}{1.888887in}}%
\pgfpathlineto{\pgfqpoint{5.873124in}{1.889749in}}%
\pgfpathlineto{\pgfqpoint{5.881767in}{1.893095in}}%
\pgfpathlineto{\pgfqpoint{5.898261in}{1.898248in}}%
\pgfpathlineto{\pgfqpoint{5.900584in}{1.898506in}}%
\pgfpathlineto{\pgfqpoint{5.902438in}{1.898538in}}%
\pgfpathlineto{\pgfqpoint{5.932300in}{1.908398in}}%
\pgfpathlineto{\pgfqpoint{5.941967in}{1.911828in}}%
\pgfpathlineto{\pgfqpoint{5.945882in}{1.912883in}}%
\pgfpathlineto{\pgfqpoint{5.949774in}{1.913403in}}%
\pgfpathlineto{\pgfqpoint{5.953644in}{1.914457in}}%
\pgfpathlineto{\pgfqpoint{5.957492in}{1.916318in}}%
\pgfpathlineto{\pgfqpoint{5.963434in}{1.917881in}}%
\pgfpathlineto{\pgfqpoint{5.968067in}{1.919555in}}%
\pgfpathlineto{\pgfqpoint{5.981364in}{1.924081in}}%
\pgfpathlineto{\pgfqpoint{5.983828in}{1.924877in}}%
\pgfpathlineto{\pgfqpoint{5.987916in}{1.926036in}}%
\pgfpathlineto{\pgfqpoint{5.991573in}{1.926438in}}%
\pgfpathlineto{\pgfqpoint{5.997625in}{1.928497in}}%
\pgfpathlineto{\pgfqpoint{6.015069in}{1.934505in}}%
\pgfpathlineto{\pgfqpoint{6.018193in}{1.935223in}}%
\pgfpathlineto{\pgfqpoint{6.035878in}{1.940450in}}%
\pgfpathlineto{\pgfqpoint{6.038529in}{1.941391in}}%
\pgfpathlineto{\pgfqpoint{6.069913in}{1.950775in}}%
\pgfpathlineto{\pgfqpoint{6.072434in}{1.951633in}}%
\pgfpathlineto{\pgfqpoint{6.075660in}{1.953006in}}%
\pgfpathlineto{\pgfqpoint{6.078160in}{1.953508in}}%
\pgfpathlineto{\pgfqpoint{6.081004in}{1.954721in}}%
\pgfpathlineto{\pgfqpoint{6.084544in}{1.956512in}}%
\pgfpathlineto{\pgfqpoint{6.087713in}{1.957554in}}%
\pgfpathlineto{\pgfqpoint{6.089818in}{1.957861in}}%
\pgfpathlineto{\pgfqpoint{6.094356in}{1.959567in}}%
\pgfpathlineto{\pgfqpoint{6.097479in}{1.960809in}}%
\pgfpathlineto{\pgfqpoint{6.105397in}{1.961883in}}%
\pgfpathlineto{\pgfqpoint{6.112207in}{1.964551in}}%
\pgfpathlineto{\pgfqpoint{6.113560in}{1.965059in}}%
\pgfpathlineto{\pgfqpoint{6.119284in}{1.968015in}}%
\pgfpathlineto{\pgfqpoint{6.126950in}{1.970348in}}%
\pgfpathlineto{\pgfqpoint{6.134531in}{1.972318in}}%
\pgfpathlineto{\pgfqpoint{6.136821in}{1.972738in}}%
\pgfpathlineto{\pgfqpoint{6.169649in}{1.983884in}}%
\pgfpathlineto{\pgfqpoint{6.171202in}{1.984158in}}%
\pgfpathlineto{\pgfqpoint{6.173681in}{1.985355in}}%
\pgfpathlineto{\pgfqpoint{6.176766in}{1.986332in}}%
\pgfpathlineto{\pgfqpoint{6.177996in}{1.986749in}}%
\pgfpathlineto{\pgfqpoint{6.182589in}{1.989081in}}%
\pgfpathlineto{\pgfqpoint{6.186847in}{1.990530in}}%
\pgfpathlineto{\pgfqpoint{6.189873in}{1.991447in}}%
\pgfpathlineto{\pgfqpoint{6.194085in}{1.993247in}}%
\pgfpathlineto{\pgfqpoint{6.211263in}{1.998386in}}%
\pgfpathlineto{\pgfqpoint{6.215345in}{2.000194in}}%
\pgfpathlineto{\pgfqpoint{6.217376in}{2.000495in}}%
\pgfpathlineto{\pgfqpoint{6.219113in}{2.001067in}}%
\pgfpathlineto{\pgfqpoint{6.225155in}{2.003314in}}%
\pgfpathlineto{\pgfqpoint{6.227157in}{2.003341in}}%
\pgfpathlineto{\pgfqpoint{6.231144in}{2.003851in}}%
\pgfpathlineto{\pgfqpoint{6.233694in}{2.005412in}}%
\pgfpathlineto{\pgfqpoint{6.236798in}{2.006291in}}%
\pgfpathlineto{\pgfqpoint{6.240728in}{2.008142in}}%
\pgfpathlineto{\pgfqpoint{6.245748in}{2.009671in}}%
\pgfpathlineto{\pgfqpoint{6.247413in}{2.010169in}}%
\pgfpathlineto{\pgfqpoint{6.252383in}{2.012048in}}%
\pgfpathlineto{\pgfqpoint{6.260585in}{2.014068in}}%
\pgfpathlineto{\pgfqpoint{6.263027in}{2.014847in}}%
\pgfpathlineto{\pgfqpoint{6.265729in}{2.015551in}}%
\pgfpathlineto{\pgfqpoint{6.269226in}{2.017471in}}%
\pgfpathlineto{\pgfqpoint{6.272972in}{2.018469in}}%
\pgfpathlineto{\pgfqpoint{6.274571in}{2.018901in}}%
\pgfpathlineto{\pgfqpoint{6.280929in}{2.020776in}}%
\pgfpathlineto{\pgfqpoint{6.282772in}{2.020628in}}%
\pgfpathlineto{\pgfqpoint{6.307046in}{2.028435in}}%
\pgfpathlineto{\pgfqpoint{6.310588in}{2.028899in}}%
\pgfpathlineto{\pgfqpoint{6.314111in}{2.030325in}}%
\pgfpathlineto{\pgfqpoint{6.315615in}{2.030801in}}%
\pgfpathlineto{\pgfqpoint{6.318364in}{2.032047in}}%
\pgfpathlineto{\pgfqpoint{6.319610in}{2.032656in}}%
\pgfpathlineto{\pgfqpoint{6.322839in}{2.033888in}}%
\pgfpathlineto{\pgfqpoint{6.337785in}{2.037744in}}%
\pgfpathlineto{\pgfqpoint{6.351932in}{2.042716in}}%
\pgfpathlineto{\pgfqpoint{6.353117in}{2.043096in}}%
\pgfpathlineto{\pgfqpoint{6.355482in}{2.044065in}}%
\pgfpathlineto{\pgfqpoint{6.357132in}{2.043649in}}%
\pgfpathlineto{\pgfqpoint{6.368803in}{2.047886in}}%
\pgfpathlineto{\pgfqpoint{6.372035in}{2.049052in}}%
\pgfpathlineto{\pgfqpoint{6.373645in}{2.049603in}}%
\pgfpathlineto{\pgfqpoint{6.379136in}{2.050703in}}%
\pgfpathlineto{\pgfqpoint{6.381184in}{2.051064in}}%
\pgfpathlineto{\pgfqpoint{6.385035in}{2.052453in}}%
\pgfpathlineto{\pgfqpoint{6.388640in}{2.053733in}}%
\pgfpathlineto{\pgfqpoint{6.389987in}{2.053872in}}%
\pgfpathlineto{\pgfqpoint{6.394234in}{2.055730in}}%
\pgfpathlineto{\pgfqpoint{6.396236in}{2.056381in}}%
\pgfpathlineto{\pgfqpoint{6.404187in}{2.057541in}}%
\pgfpathlineto{\pgfqpoint{6.405722in}{2.058381in}}%
\pgfpathlineto{\pgfqpoint{6.408782in}{2.059130in}}%
\pgfpathlineto{\pgfqpoint{6.410741in}{2.060205in}}%
\pgfpathlineto{\pgfqpoint{6.412262in}{2.060617in}}%
\pgfpathlineto{\pgfqpoint{6.416802in}{2.062484in}}%
\pgfpathlineto{\pgfqpoint{6.418524in}{2.062961in}}%
\pgfpathlineto{\pgfqpoint{6.475572in}{2.079222in}}%
\pgfpathlineto{\pgfqpoint{6.477150in}{2.080135in}}%
\pgfpathlineto{\pgfqpoint{6.479510in}{2.080467in}}%
\pgfpathlineto{\pgfqpoint{6.514136in}{2.092148in}}%
\pgfpathlineto{\pgfqpoint{6.515440in}{2.092580in}}%
\pgfpathlineto{\pgfqpoint{6.517299in}{2.092479in}}%
\pgfpathlineto{\pgfqpoint{6.519152in}{2.092870in}}%
\pgfpathlineto{\pgfqpoint{6.520077in}{2.092963in}}%
\pgfpathlineto{\pgfqpoint{6.520262in}{2.093248in}}%
\pgfpathlineto{\pgfqpoint{6.521923in}{2.094245in}}%
\pgfpathlineto{\pgfqpoint{6.522291in}{2.093806in}}%
\pgfpathlineto{\pgfqpoint{6.525049in}{2.094114in}}%
\pgfpathlineto{\pgfqpoint{6.527064in}{2.095345in}}%
\pgfpathlineto{\pgfqpoint{6.530349in}{2.096927in}}%
\pgfpathlineto{\pgfqpoint{6.530713in}{2.096668in}}%
\pgfpathlineto{\pgfqpoint{6.531440in}{2.097211in}}%
\pgfpathlineto{\pgfqpoint{6.535788in}{2.098513in}}%
\pgfpathlineto{\pgfqpoint{6.537050in}{2.098731in}}%
\pgfpathlineto{\pgfqpoint{6.538671in}{2.099513in}}%
\pgfpathlineto{\pgfqpoint{6.539928in}{2.100138in}}%
\pgfpathlineto{\pgfqpoint{6.541362in}{2.100056in}}%
\pgfpathlineto{\pgfqpoint{6.550434in}{2.103519in}}%
\pgfpathlineto{\pgfqpoint{6.550964in}{2.104543in}}%
\pgfpathlineto{\pgfqpoint{6.551669in}{2.104323in}}%
\pgfpathlineto{\pgfqpoint{6.553431in}{2.104227in}}%
\pgfpathlineto{\pgfqpoint{6.558163in}{2.105831in}}%
\pgfpathlineto{\pgfqpoint{6.559559in}{2.105789in}}%
\pgfpathlineto{\pgfqpoint{6.561474in}{2.106198in}}%
\pgfpathlineto{\pgfqpoint{6.563036in}{2.106924in}}%
\pgfpathlineto{\pgfqpoint{6.564422in}{2.107370in}}%
\pgfpathlineto{\pgfqpoint{6.567874in}{2.109399in}}%
\pgfpathlineto{\pgfqpoint{6.568390in}{2.110566in}}%
\pgfpathlineto{\pgfqpoint{6.569078in}{2.110338in}}%
\pgfpathlineto{\pgfqpoint{6.570623in}{2.110155in}}%
\pgfpathlineto{\pgfqpoint{6.572506in}{2.110885in}}%
\pgfpathlineto{\pgfqpoint{6.573873in}{2.111213in}}%
\pgfpathlineto{\pgfqpoint{6.578634in}{2.111872in}}%
\pgfpathlineto{\pgfqpoint{6.580157in}{2.112620in}}%
\pgfpathlineto{\pgfqpoint{6.583361in}{2.113910in}}%
\pgfpathlineto{\pgfqpoint{6.587387in}{2.116166in}}%
\pgfpathlineto{\pgfqpoint{6.588557in}{2.116625in}}%
\pgfpathlineto{\pgfqpoint{6.590890in}{2.117380in}}%
\pgfpathlineto{\pgfqpoint{6.591056in}{2.117061in}}%
\pgfpathlineto{\pgfqpoint{6.591555in}{2.117607in}}%
\pgfpathlineto{\pgfqpoint{6.591555in}{2.117607in}}%
\pgfpathlineto{\pgfqpoint{6.592718in}{2.117558in}}%
\pgfpathlineto{\pgfqpoint{6.594375in}{2.117055in}}%
\pgfpathlineto{\pgfqpoint{6.596523in}{2.118472in}}%
\pgfpathlineto{\pgfqpoint{6.598994in}{2.119406in}}%
\pgfpathlineto{\pgfqpoint{6.601455in}{2.121031in}}%
\pgfpathlineto{\pgfqpoint{6.601946in}{2.122079in}}%
\pgfpathlineto{\pgfqpoint{6.602600in}{2.121474in}}%
\pgfpathlineto{\pgfqpoint{6.604233in}{2.121620in}}%
\pgfpathlineto{\pgfqpoint{6.607001in}{2.122978in}}%
\pgfpathlineto{\pgfqpoint{6.607650in}{2.122634in}}%
\pgfpathlineto{\pgfqpoint{6.608137in}{2.123397in}}%
\pgfpathlineto{\pgfqpoint{6.608299in}{2.123608in}}%
\pgfpathlineto{\pgfqpoint{6.608623in}{2.123156in}}%
\pgfpathlineto{\pgfqpoint{6.608623in}{2.123156in}}%
\pgfpathlineto{\pgfqpoint{6.609756in}{2.122990in}}%
\pgfpathlineto{\pgfqpoint{6.625111in}{2.128013in}}%
\pgfpathlineto{\pgfqpoint{6.627006in}{2.127666in}}%
\pgfpathlineto{\pgfqpoint{6.629681in}{2.129842in}}%
\pgfpathlineto{\pgfqpoint{6.629838in}{2.129599in}}%
\pgfpathlineto{\pgfqpoint{6.630779in}{2.129583in}}%
\pgfpathlineto{\pgfqpoint{6.630936in}{2.129777in}}%
\pgfpathlineto{\pgfqpoint{6.632658in}{2.131126in}}%
\pgfpathlineto{\pgfqpoint{6.633908in}{2.132723in}}%
\pgfpathlineto{\pgfqpoint{6.634376in}{2.131801in}}%
\pgfpathlineto{\pgfqpoint{6.640277in}{2.133117in}}%
\pgfpathlineto{\pgfqpoint{6.641050in}{2.132653in}}%
\pgfpathlineto{\pgfqpoint{6.641358in}{2.133099in}}%
\pgfpathlineto{\pgfqpoint{6.642900in}{2.132895in}}%
\pgfpathlineto{\pgfqpoint{6.643054in}{2.133103in}}%
\pgfpathlineto{\pgfqpoint{6.645206in}{2.134478in}}%
\pgfpathlineto{\pgfqpoint{6.645360in}{2.134181in}}%
\pgfpathlineto{\pgfqpoint{6.645667in}{2.133964in}}%
\pgfpathlineto{\pgfqpoint{6.646433in}{2.134449in}}%
\pgfpathlineto{\pgfqpoint{6.649185in}{2.136144in}}%
\pgfpathlineto{\pgfqpoint{6.650405in}{2.136674in}}%
\pgfpathlineto{\pgfqpoint{6.653292in}{2.136863in}}%
\pgfpathlineto{\pgfqpoint{6.654958in}{2.137517in}}%
\pgfpathlineto{\pgfqpoint{6.655260in}{2.138043in}}%
\pgfpathlineto{\pgfqpoint{6.655412in}{2.137449in}}%
\pgfpathlineto{\pgfqpoint{6.655412in}{2.137449in}}%
\pgfpathlineto{\pgfqpoint{6.656922in}{2.136413in}}%
\pgfpathlineto{\pgfqpoint{6.659781in}{2.138675in}}%
\pgfpathlineto{\pgfqpoint{6.664272in}{2.140814in}}%
\pgfpathlineto{\pgfqpoint{6.664421in}{2.141100in}}%
\pgfpathlineto{\pgfqpoint{6.664421in}{2.141100in}}%
\pgfpathlineto{\pgfqpoint{6.664421in}{2.141100in}}%
\pgfpathlineto{\pgfqpoint{6.664720in}{2.140339in}}%
\pgfpathlineto{\pgfqpoint{6.665465in}{2.140970in}}%
\pgfpathlineto{\pgfqpoint{6.670213in}{2.141699in}}%
\pgfpathlineto{\pgfqpoint{6.670361in}{2.140978in}}%
\pgfpathlineto{\pgfqpoint{6.671248in}{2.140211in}}%
\pgfpathlineto{\pgfqpoint{6.671543in}{2.140615in}}%
\pgfpathlineto{\pgfqpoint{6.673312in}{2.140832in}}%
\pgfpathlineto{\pgfqpoint{6.674782in}{2.142776in}}%
\pgfpathlineto{\pgfqpoint{6.677275in}{2.142176in}}%
\pgfpathlineto{\pgfqpoint{6.677421in}{2.142559in}}%
\pgfpathlineto{\pgfqpoint{6.682958in}{2.147154in}}%
\pgfpathlineto{\pgfqpoint{6.683248in}{2.146174in}}%
\pgfpathlineto{\pgfqpoint{6.686864in}{2.144579in}}%
\pgfpathlineto{\pgfqpoint{6.687009in}{2.144204in}}%
\pgfpathlineto{\pgfqpoint{6.687297in}{2.144712in}}%
\pgfpathlineto{\pgfqpoint{6.687297in}{2.144712in}}%
\pgfpathlineto{\pgfqpoint{6.689312in}{2.147332in}}%
\pgfpathlineto{\pgfqpoint{6.689744in}{2.146980in}}%
\pgfpathlineto{\pgfqpoint{6.690174in}{2.146246in}}%
\pgfpathlineto{\pgfqpoint{6.690892in}{2.146993in}}%
\pgfpathlineto{\pgfqpoint{6.692754in}{2.147797in}}%
\pgfpathlineto{\pgfqpoint{6.696035in}{2.151148in}}%
\pgfpathlineto{\pgfqpoint{6.696177in}{2.150874in}}%
\pgfpathlineto{\pgfqpoint{6.696319in}{2.150398in}}%
\pgfpathlineto{\pgfqpoint{6.696462in}{2.150930in}}%
\pgfpathlineto{\pgfqpoint{6.696462in}{2.150930in}}%
\pgfpathlineto{\pgfqpoint{6.697030in}{2.152528in}}%
\pgfpathlineto{\pgfqpoint{6.697598in}{2.150817in}}%
\pgfpathlineto{\pgfqpoint{6.702127in}{2.150412in}}%
\pgfpathlineto{\pgfqpoint{6.703395in}{2.151676in}}%
\pgfpathlineto{\pgfqpoint{6.703536in}{2.151358in}}%
\pgfpathlineto{\pgfqpoint{6.704942in}{2.150389in}}%
\pgfpathlineto{\pgfqpoint{6.707045in}{2.152276in}}%
\pgfpathlineto{\pgfqpoint{6.708444in}{2.153634in}}%
\pgfpathlineto{\pgfqpoint{6.708723in}{2.152846in}}%
\pgfpathlineto{\pgfqpoint{6.709002in}{2.153261in}}%
\pgfpathlineto{\pgfqpoint{6.709002in}{2.153261in}}%
\pgfpathlineto{\pgfqpoint{6.709281in}{2.154393in}}%
\pgfpathlineto{\pgfqpoint{6.709421in}{2.153753in}}%
\pgfpathlineto{\pgfqpoint{6.709421in}{2.153753in}}%
\pgfpathlineto{\pgfqpoint{6.709700in}{2.152917in}}%
\pgfpathlineto{\pgfqpoint{6.709979in}{2.153173in}}%
\pgfpathlineto{\pgfqpoint{6.709979in}{2.153173in}}%
\pgfpathlineto{\pgfqpoint{6.711371in}{2.155587in}}%
\pgfpathlineto{\pgfqpoint{6.712483in}{2.153423in}}%
\pgfpathlineto{\pgfqpoint{6.712761in}{2.153998in}}%
\pgfpathlineto{\pgfqpoint{6.713038in}{2.155324in}}%
\pgfpathlineto{\pgfqpoint{6.714009in}{2.154639in}}%
\pgfpathlineto{\pgfqpoint{6.714286in}{2.155186in}}%
\pgfpathlineto{\pgfqpoint{6.714563in}{2.154536in}}%
\pgfpathlineto{\pgfqpoint{6.714563in}{2.154536in}}%
\pgfpathlineto{\pgfqpoint{6.715532in}{2.154008in}}%
\pgfpathlineto{\pgfqpoint{6.715670in}{2.154439in}}%
\pgfpathlineto{\pgfqpoint{6.716775in}{2.157093in}}%
\pgfpathlineto{\pgfqpoint{6.717189in}{2.156769in}}%
\pgfpathlineto{\pgfqpoint{6.717327in}{2.157056in}}%
\pgfpathlineto{\pgfqpoint{6.717464in}{2.156639in}}%
\pgfpathlineto{\pgfqpoint{6.717464in}{2.156639in}}%
\pgfpathlineto{\pgfqpoint{6.718291in}{2.154857in}}%
\pgfpathlineto{\pgfqpoint{6.718979in}{2.155261in}}%
\pgfpathlineto{\pgfqpoint{6.719117in}{2.154968in}}%
\pgfpathlineto{\pgfqpoint{6.719254in}{2.155293in}}%
\pgfpathlineto{\pgfqpoint{6.719254in}{2.155293in}}%
\pgfpathlineto{\pgfqpoint{6.719941in}{2.157962in}}%
\pgfpathlineto{\pgfqpoint{6.720628in}{2.156795in}}%
\pgfpathlineto{\pgfqpoint{6.722546in}{2.160032in}}%
\pgfpathlineto{\pgfqpoint{6.722819in}{2.160107in}}%
\pgfpathlineto{\pgfqpoint{6.722956in}{2.161124in}}%
\pgfpathlineto{\pgfqpoint{6.723230in}{2.160185in}}%
\pgfpathlineto{\pgfqpoint{6.723230in}{2.160185in}}%
\pgfpathlineto{\pgfqpoint{6.723503in}{2.158690in}}%
\pgfpathlineto{\pgfqpoint{6.723776in}{2.159416in}}%
\pgfpathlineto{\pgfqpoint{6.723776in}{2.159416in}}%
\pgfpathlineto{\pgfqpoint{6.723913in}{2.160772in}}%
\pgfpathlineto{\pgfqpoint{6.724186in}{2.159644in}}%
\pgfpathlineto{\pgfqpoint{6.724186in}{2.159644in}}%
\pgfpathlineto{\pgfqpoint{6.724595in}{2.159205in}}%
\pgfpathlineto{\pgfqpoint{6.725004in}{2.159743in}}%
\pgfpathlineto{\pgfqpoint{6.725004in}{2.159743in}}%
\pgfpathlineto{\pgfqpoint{6.725140in}{2.160135in}}%
\pgfpathlineto{\pgfqpoint{6.725140in}{2.160135in}}%
\pgfpathlineto{\pgfqpoint{6.725140in}{2.160135in}}%
\pgfpathlineto{\pgfqpoint{6.725277in}{2.159035in}}%
\pgfpathlineto{\pgfqpoint{6.725413in}{2.159853in}}%
\pgfpathlineto{\pgfqpoint{6.725413in}{2.159853in}}%
\pgfpathlineto{\pgfqpoint{6.725822in}{2.160487in}}%
\pgfpathlineto{\pgfqpoint{6.726094in}{2.159883in}}%
\pgfpathlineto{\pgfqpoint{6.726094in}{2.159883in}}%
\pgfpathlineto{\pgfqpoint{6.726230in}{2.158965in}}%
\pgfpathlineto{\pgfqpoint{6.726366in}{2.159126in}}%
\pgfpathlineto{\pgfqpoint{6.726366in}{2.159126in}}%
\pgfpathlineto{\pgfqpoint{6.726774in}{2.162268in}}%
\pgfpathlineto{\pgfqpoint{6.727454in}{2.159415in}}%
\pgfpathlineto{\pgfqpoint{6.728404in}{2.157636in}}%
\pgfpathlineto{\pgfqpoint{6.728539in}{2.158889in}}%
\pgfpathlineto{\pgfqpoint{6.730705in}{2.164150in}}%
\pgfpathlineto{\pgfqpoint{6.731516in}{2.160899in}}%
\pgfpathlineto{\pgfqpoint{6.732056in}{2.162660in}}%
\pgfpathlineto{\pgfqpoint{6.732191in}{2.163222in}}%
\pgfpathlineto{\pgfqpoint{6.732595in}{2.162848in}}%
\pgfpathlineto{\pgfqpoint{6.732595in}{2.162848in}}%
\pgfpathlineto{\pgfqpoint{6.732730in}{2.162368in}}%
\pgfpathlineto{\pgfqpoint{6.732865in}{2.162859in}}%
\pgfpathlineto{\pgfqpoint{6.732865in}{2.162859in}}%
\pgfpathlineto{\pgfqpoint{6.733269in}{2.163804in}}%
\pgfpathlineto{\pgfqpoint{6.733404in}{2.163490in}}%
\pgfusepath{stroke}%
\end{pgfscope}%
\begin{pgfscope}%
\pgfpathrectangle{\pgfqpoint{1.000000in}{0.300000in}}{\pgfqpoint{6.200000in}{2.400000in}} %
\pgfusepath{clip}%
\pgfsetrectcap%
\pgfsetroundjoin%
\pgfsetlinewidth{1.003750pt}%
\definecolor{currentstroke}{rgb}{1.000000,0.000000,0.000000}%
\pgfsetstrokecolor{currentstroke}%
\pgfsetdash{}{0pt}%
\pgfpathmoveto{\pgfqpoint{1.000000in}{0.497039in}}%
\pgfpathlineto{\pgfqpoint{1.466596in}{0.692308in}}%
\pgfpathlineto{\pgfqpoint{1.739538in}{0.795276in}}%
\pgfpathlineto{\pgfqpoint{1.933193in}{0.858682in}}%
\pgfpathlineto{\pgfqpoint{2.309902in}{0.971959in}}%
\pgfpathlineto{\pgfqpoint{2.866386in}{1.132900in}}%
\pgfpathlineto{\pgfqpoint{2.945672in}{1.155466in}}%
\pgfpathlineto{\pgfqpoint{3.049440in}{1.187056in}}%
\pgfpathlineto{\pgfqpoint{3.218614in}{1.238008in}}%
\pgfpathlineto{\pgfqpoint{3.332982in}{1.269423in}}%
\pgfpathlineto{\pgfqpoint{3.393305in}{1.287605in}}%
\pgfpathlineto{\pgfqpoint{3.448665in}{1.304469in}}%
\pgfpathlineto{\pgfqpoint{3.799579in}{1.406377in}}%
\pgfpathlineto{\pgfqpoint{3.897309in}{1.435879in}}%
\pgfpathlineto{\pgfqpoint{3.941322in}{1.448895in}}%
\pgfpathlineto{\pgfqpoint{4.043871in}{1.479495in}}%
\pgfpathlineto{\pgfqpoint{4.086400in}{1.493359in}}%
\pgfpathlineto{\pgfqpoint{4.119898in}{1.502007in}}%
\pgfpathlineto{\pgfqpoint{4.524945in}{1.625851in}}%
\pgfpathlineto{\pgfqpoint{4.535602in}{1.628961in}}%
\pgfpathlineto{\pgfqpoint{4.552997in}{1.634833in}}%
\pgfpathlineto{\pgfqpoint{4.563222in}{1.637750in}}%
\pgfpathlineto{\pgfqpoint{4.579927in}{1.641933in}}%
\pgfpathlineto{\pgfqpoint{4.998665in}{1.769457in}}%
\pgfpathlineto{\pgfqpoint{5.007464in}{1.771740in}}%
\pgfpathlineto{\pgfqpoint{5.021308in}{1.775907in}}%
\pgfpathlineto{\pgfqpoint{5.049815in}{1.784149in}}%
\pgfpathlineto{\pgfqpoint{5.081876in}{1.794044in}}%
\pgfpathlineto{\pgfqpoint{5.091204in}{1.796726in}}%
\pgfpathlineto{\pgfqpoint{5.130195in}{1.809168in}}%
\pgfpathlineto{\pgfqpoint{5.241415in}{1.842110in}}%
\pgfpathlineto{\pgfqpoint{5.264483in}{1.849405in}}%
\pgfpathlineto{\pgfqpoint{5.301632in}{1.860818in}}%
\pgfpathlineto{\pgfqpoint{5.305012in}{1.861593in}}%
\pgfpathlineto{\pgfqpoint{5.312833in}{1.863141in}}%
\pgfpathlineto{\pgfqpoint{5.334690in}{1.869678in}}%
\pgfpathlineto{\pgfqpoint{5.358979in}{1.877564in}}%
\pgfpathlineto{\pgfqpoint{5.365174in}{1.879731in}}%
\pgfpathlineto{\pgfqpoint{5.373346in}{1.882373in}}%
\pgfpathlineto{\pgfqpoint{5.380416in}{1.884268in}}%
\pgfpathlineto{\pgfqpoint{5.391378in}{1.887810in}}%
\pgfpathlineto{\pgfqpoint{5.400216in}{1.890299in}}%
\pgfpathlineto{\pgfqpoint{5.407976in}{1.892807in}}%
\pgfpathlineto{\pgfqpoint{5.411823in}{1.893105in}}%
\pgfpathlineto{\pgfqpoint{5.456345in}{1.906656in}}%
\pgfpathlineto{\pgfqpoint{5.460818in}{1.908371in}}%
\pgfpathlineto{\pgfqpoint{5.467030in}{1.910269in}}%
\pgfpathlineto{\pgfqpoint{5.619005in}{1.957172in}}%
\pgfpathlineto{\pgfqpoint{5.624621in}{1.958803in}}%
\pgfpathlineto{\pgfqpoint{5.629496in}{1.960127in}}%
\pgfpathlineto{\pgfqpoint{5.649327in}{1.966088in}}%
\pgfpathlineto{\pgfqpoint{5.653357in}{1.966965in}}%
\pgfpathlineto{\pgfqpoint{5.673808in}{1.973736in}}%
\pgfpathlineto{\pgfqpoint{5.679630in}{1.976203in}}%
\pgfpathlineto{\pgfqpoint{5.712934in}{1.986445in}}%
\pgfpathlineto{\pgfqpoint{5.715992in}{1.986608in}}%
\pgfpathlineto{\pgfqpoint{5.735836in}{1.993214in}}%
\pgfpathlineto{\pgfqpoint{5.741149in}{1.995296in}}%
\pgfpathlineto{\pgfqpoint{5.769357in}{2.003805in}}%
\pgfpathlineto{\pgfqpoint{5.785511in}{2.008552in}}%
\pgfpathlineto{\pgfqpoint{5.793717in}{2.011597in}}%
\pgfpathlineto{\pgfqpoint{5.815123in}{2.018426in}}%
\pgfpathlineto{\pgfqpoint{5.818276in}{2.019400in}}%
\pgfpathlineto{\pgfqpoint{5.830227in}{2.022609in}}%
\pgfpathlineto{\pgfqpoint{5.841970in}{2.027038in}}%
\pgfpathlineto{\pgfqpoint{5.846007in}{2.028141in}}%
\pgfpathlineto{\pgfqpoint{5.876017in}{2.037078in}}%
\pgfpathlineto{\pgfqpoint{5.879377in}{2.037980in}}%
\pgfpathlineto{\pgfqpoint{5.888885in}{2.041443in}}%
\pgfpathlineto{\pgfqpoint{5.906129in}{2.047084in}}%
\pgfpathlineto{\pgfqpoint{5.911628in}{2.048694in}}%
\pgfpathlineto{\pgfqpoint{5.917082in}{2.050500in}}%
\pgfpathlineto{\pgfqpoint{5.924734in}{2.052293in}}%
\pgfpathlineto{\pgfqpoint{5.945448in}{2.059634in}}%
\pgfpathlineto{\pgfqpoint{6.043049in}{2.090130in}}%
\pgfpathlineto{\pgfqpoint{6.047540in}{2.091141in}}%
\pgfpathlineto{\pgfqpoint{6.049774in}{2.092381in}}%
\pgfpathlineto{\pgfqpoint{6.052371in}{2.093284in}}%
\pgfpathlineto{\pgfqpoint{6.056063in}{2.094414in}}%
\pgfpathlineto{\pgfqpoint{6.064116in}{2.096893in}}%
\pgfpathlineto{\pgfqpoint{6.066296in}{2.097719in}}%
\pgfpathlineto{\pgfqpoint{6.069191in}{2.098393in}}%
\pgfpathlineto{\pgfqpoint{6.072434in}{2.099375in}}%
\pgfpathlineto{\pgfqpoint{6.085602in}{2.103313in}}%
\pgfpathlineto{\pgfqpoint{6.089117in}{2.104423in}}%
\pgfpathlineto{\pgfqpoint{6.094008in}{2.106086in}}%
\pgfpathlineto{\pgfqpoint{6.097133in}{2.107807in}}%
\pgfpathlineto{\pgfqpoint{6.174299in}{2.132652in}}%
\pgfpathlineto{\pgfqpoint{6.177073in}{2.133243in}}%
\pgfpathlineto{\pgfqpoint{6.179530in}{2.134044in}}%
\pgfpathlineto{\pgfqpoint{6.181978in}{2.134561in}}%
\pgfpathlineto{\pgfqpoint{6.191681in}{2.138286in}}%
\pgfpathlineto{\pgfqpoint{6.194685in}{2.139681in}}%
\pgfpathlineto{\pgfqpoint{6.197377in}{2.139982in}}%
\pgfpathlineto{\pgfqpoint{6.218245in}{2.146838in}}%
\pgfpathlineto{\pgfqpoint{6.318364in}{2.180251in}}%
\pgfpathlineto{\pgfqpoint{6.320357in}{2.180704in}}%
\pgfpathlineto{\pgfqpoint{6.322839in}{2.181591in}}%
\pgfpathlineto{\pgfqpoint{6.365090in}{2.195834in}}%
\pgfpathlineto{\pgfqpoint{6.367645in}{2.196672in}}%
\pgfpathlineto{\pgfqpoint{6.372955in}{2.198341in}}%
\pgfpathlineto{\pgfqpoint{6.377082in}{2.200538in}}%
\pgfpathlineto{\pgfqpoint{6.383452in}{2.202164in}}%
\pgfpathlineto{\pgfqpoint{6.385938in}{2.203395in}}%
\pgfpathlineto{\pgfqpoint{6.388190in}{2.203301in}}%
\pgfpathlineto{\pgfqpoint{6.396680in}{2.206281in}}%
\pgfpathlineto{\pgfqpoint{6.398454in}{2.207027in}}%
\pgfpathlineto{\pgfqpoint{6.400444in}{2.207682in}}%
\pgfpathlineto{\pgfqpoint{6.401767in}{2.208038in}}%
\pgfpathlineto{\pgfqpoint{6.409871in}{2.210441in}}%
\pgfpathlineto{\pgfqpoint{6.413562in}{2.211562in}}%
\pgfpathlineto{\pgfqpoint{6.430242in}{2.217153in}}%
\pgfpathlineto{\pgfqpoint{6.434663in}{2.217938in}}%
\pgfpathlineto{\pgfqpoint{6.436968in}{2.219095in}}%
\pgfpathlineto{\pgfqpoint{6.438847in}{2.219358in}}%
\pgfpathlineto{\pgfqpoint{6.449399in}{2.223862in}}%
\pgfpathlineto{\pgfqpoint{6.451040in}{2.223801in}}%
\pgfpathlineto{\pgfqpoint{6.455123in}{2.224347in}}%
\pgfpathlineto{\pgfqpoint{6.458372in}{2.225655in}}%
\pgfpathlineto{\pgfqpoint{6.468824in}{2.229473in}}%
\pgfpathlineto{\pgfqpoint{6.472405in}{2.230411in}}%
\pgfpathlineto{\pgfqpoint{6.473792in}{2.230417in}}%
\pgfpathlineto{\pgfqpoint{6.475374in}{2.231303in}}%
\pgfpathlineto{\pgfqpoint{6.477740in}{2.232369in}}%
\pgfpathlineto{\pgfqpoint{6.481861in}{2.234155in}}%
\pgfpathlineto{\pgfqpoint{6.483229in}{2.234577in}}%
\pgfpathlineto{\pgfqpoint{6.484400in}{2.234315in}}%
\pgfpathlineto{\pgfqpoint{6.484594in}{2.234523in}}%
\pgfpathlineto{\pgfqpoint{6.487899in}{2.235700in}}%
\pgfpathlineto{\pgfqpoint{6.488867in}{2.236107in}}%
\pgfpathlineto{\pgfqpoint{6.490607in}{2.236870in}}%
\pgfpathlineto{\pgfqpoint{6.495801in}{2.238429in}}%
\pgfpathlineto{\pgfqpoint{6.497715in}{2.238562in}}%
\pgfpathlineto{\pgfqpoint{6.500194in}{2.240128in}}%
\pgfpathlineto{\pgfqpoint{6.502285in}{2.240955in}}%
\pgfpathlineto{\pgfqpoint{6.506259in}{2.241595in}}%
\pgfpathlineto{\pgfqpoint{6.514323in}{2.244063in}}%
\pgfpathlineto{\pgfqpoint{6.515626in}{2.244475in}}%
\pgfpathlineto{\pgfqpoint{6.538131in}{2.251191in}}%
\pgfpathlineto{\pgfqpoint{6.539389in}{2.251686in}}%
\pgfpathlineto{\pgfqpoint{6.540466in}{2.251502in}}%
\pgfpathlineto{\pgfqpoint{6.540645in}{2.251699in}}%
\pgfpathlineto{\pgfqpoint{6.543151in}{2.252497in}}%
\pgfpathlineto{\pgfqpoint{6.544578in}{2.252838in}}%
\pgfpathlineto{\pgfqpoint{6.545824in}{2.253592in}}%
\pgfpathlineto{\pgfqpoint{6.546180in}{2.253302in}}%
\pgfpathlineto{\pgfqpoint{6.550610in}{2.254579in}}%
\pgfpathlineto{\pgfqpoint{6.551846in}{2.254795in}}%
\pgfpathlineto{\pgfqpoint{6.554134in}{2.255789in}}%
\pgfpathlineto{\pgfqpoint{6.568046in}{2.259637in}}%
\pgfpathlineto{\pgfqpoint{6.569250in}{2.259864in}}%
\pgfpathlineto{\pgfqpoint{6.572164in}{2.260993in}}%
\pgfpathlineto{\pgfqpoint{6.574385in}{2.261659in}}%
\pgfpathlineto{\pgfqpoint{6.574555in}{2.261326in}}%
\pgfpathlineto{\pgfqpoint{6.575747in}{2.261692in}}%
\pgfpathlineto{\pgfqpoint{6.576598in}{2.262053in}}%
\pgfpathlineto{\pgfqpoint{6.576937in}{2.261506in}}%
\pgfpathlineto{\pgfqpoint{6.579480in}{2.262356in}}%
\pgfpathlineto{\pgfqpoint{6.580495in}{2.262987in}}%
\pgfpathlineto{\pgfqpoint{6.580833in}{2.262700in}}%
\pgfpathlineto{\pgfqpoint{6.587053in}{2.264223in}}%
\pgfpathlineto{\pgfqpoint{6.588056in}{2.264886in}}%
\pgfpathlineto{\pgfqpoint{6.589391in}{2.265758in}}%
\pgfpathlineto{\pgfqpoint{6.589724in}{2.265532in}}%
\pgfpathlineto{\pgfqpoint{6.591555in}{2.265608in}}%
\pgfpathlineto{\pgfqpoint{6.592552in}{2.266069in}}%
\pgfpathlineto{\pgfqpoint{6.593381in}{2.266255in}}%
\pgfpathlineto{\pgfqpoint{6.593547in}{2.265880in}}%
\pgfpathlineto{\pgfqpoint{6.594209in}{2.265656in}}%
\pgfpathlineto{\pgfqpoint{6.594540in}{2.265951in}}%
\pgfpathlineto{\pgfqpoint{6.597348in}{2.267362in}}%
\pgfpathlineto{\pgfqpoint{6.598994in}{2.267561in}}%
\pgfpathlineto{\pgfqpoint{6.600307in}{2.268166in}}%
\pgfpathlineto{\pgfqpoint{6.602437in}{2.268418in}}%
\pgfpathlineto{\pgfqpoint{6.609756in}{2.270727in}}%
\pgfpathlineto{\pgfqpoint{6.609918in}{2.270273in}}%
\pgfpathlineto{\pgfqpoint{6.610242in}{2.269786in}}%
\pgfpathlineto{\pgfqpoint{6.610888in}{2.270219in}}%
\pgfpathlineto{\pgfqpoint{6.612984in}{2.271715in}}%
\pgfpathlineto{\pgfqpoint{6.614592in}{2.271477in}}%
\pgfpathlineto{\pgfqpoint{6.616837in}{2.272204in}}%
\pgfpathlineto{\pgfqpoint{6.618756in}{2.272739in}}%
\pgfpathlineto{\pgfqpoint{6.621146in}{2.273749in}}%
\pgfpathlineto{\pgfqpoint{6.622576in}{2.273532in}}%
\pgfpathlineto{\pgfqpoint{6.623686in}{2.274094in}}%
\pgfpathlineto{\pgfqpoint{6.623844in}{2.273952in}}%
\pgfpathlineto{\pgfqpoint{6.624953in}{2.274563in}}%
\pgfpathlineto{\pgfqpoint{6.625743in}{2.275000in}}%
\pgfpathlineto{\pgfqpoint{6.625901in}{2.274494in}}%
\pgfpathlineto{\pgfqpoint{6.626217in}{2.273954in}}%
\pgfpathlineto{\pgfqpoint{6.626848in}{2.274423in}}%
\pgfpathlineto{\pgfqpoint{6.628108in}{2.274722in}}%
\pgfpathlineto{\pgfqpoint{6.630152in}{2.275772in}}%
\pgfpathlineto{\pgfqpoint{6.630936in}{2.275640in}}%
\pgfpathlineto{\pgfqpoint{6.631093in}{2.275884in}}%
\pgfpathlineto{\pgfqpoint{6.631563in}{2.276792in}}%
\pgfpathlineto{\pgfqpoint{6.631876in}{2.276250in}}%
\pgfpathlineto{\pgfqpoint{6.631876in}{2.276250in}}%
\pgfpathlineto{\pgfqpoint{6.632033in}{2.275627in}}%
\pgfpathlineto{\pgfqpoint{6.632971in}{2.276252in}}%
\pgfpathlineto{\pgfqpoint{6.634064in}{2.276792in}}%
\pgfpathlineto{\pgfqpoint{6.634220in}{2.276519in}}%
\pgfpathlineto{\pgfqpoint{6.636090in}{2.276819in}}%
\pgfpathlineto{\pgfqpoint{6.637644in}{2.277894in}}%
\pgfpathlineto{\pgfqpoint{6.638264in}{2.277680in}}%
\pgfpathlineto{\pgfqpoint{6.638729in}{2.278328in}}%
\pgfpathlineto{\pgfqpoint{6.641358in}{2.279088in}}%
\pgfpathlineto{\pgfqpoint{6.641821in}{2.277942in}}%
\pgfpathlineto{\pgfqpoint{6.642592in}{2.278738in}}%
\pgfpathlineto{\pgfqpoint{6.645206in}{2.279635in}}%
\pgfpathlineto{\pgfqpoint{6.651622in}{2.281431in}}%
\pgfpathlineto{\pgfqpoint{6.654655in}{2.282137in}}%
\pgfpathlineto{\pgfqpoint{6.655563in}{2.281478in}}%
\pgfpathlineto{\pgfqpoint{6.655865in}{2.282233in}}%
\pgfpathlineto{\pgfqpoint{6.656620in}{2.282307in}}%
\pgfpathlineto{\pgfqpoint{6.656771in}{2.281878in}}%
\pgfpathlineto{\pgfqpoint{6.656922in}{2.281424in}}%
\pgfpathlineto{\pgfqpoint{6.657826in}{2.282217in}}%
\pgfpathlineto{\pgfqpoint{6.663376in}{2.283856in}}%
\pgfpathlineto{\pgfqpoint{6.665762in}{2.284350in}}%
\pgfpathlineto{\pgfqpoint{6.669178in}{2.285518in}}%
\pgfpathlineto{\pgfqpoint{6.672870in}{2.286023in}}%
\pgfpathlineto{\pgfqpoint{6.673017in}{2.286341in}}%
\pgfpathlineto{\pgfqpoint{6.673165in}{2.285980in}}%
\pgfpathlineto{\pgfqpoint{6.673165in}{2.285980in}}%
\pgfpathlineto{\pgfqpoint{6.673606in}{2.285751in}}%
\pgfpathlineto{\pgfqpoint{6.674194in}{2.286356in}}%
\pgfpathlineto{\pgfqpoint{6.676249in}{2.287045in}}%
\pgfpathlineto{\pgfqpoint{6.678298in}{2.286788in}}%
\pgfpathlineto{\pgfqpoint{6.678590in}{2.287506in}}%
\pgfpathlineto{\pgfqpoint{6.679320in}{2.287033in}}%
\pgfpathlineto{\pgfqpoint{6.679612in}{2.286041in}}%
\pgfpathlineto{\pgfqpoint{6.680486in}{2.286905in}}%
\pgfpathlineto{\pgfqpoint{6.681505in}{2.287784in}}%
\pgfpathlineto{\pgfqpoint{6.682232in}{2.288656in}}%
\pgfpathlineto{\pgfqpoint{6.682667in}{2.288533in}}%
\pgfpathlineto{\pgfqpoint{6.685131in}{2.287809in}}%
\pgfpathlineto{\pgfqpoint{6.685709in}{2.289285in}}%
\pgfpathlineto{\pgfqpoint{6.686287in}{2.288631in}}%
\pgfpathlineto{\pgfqpoint{6.686864in}{2.288141in}}%
\pgfpathlineto{\pgfqpoint{6.687297in}{2.288927in}}%
\pgfpathlineto{\pgfqpoint{6.687585in}{2.289597in}}%
\pgfpathlineto{\pgfqpoint{6.687873in}{2.288742in}}%
\pgfpathlineto{\pgfqpoint{6.687873in}{2.288742in}}%
\pgfpathlineto{\pgfqpoint{6.688593in}{2.288180in}}%
\pgfpathlineto{\pgfqpoint{6.688881in}{2.288912in}}%
\pgfpathlineto{\pgfqpoint{6.691608in}{2.290338in}}%
\pgfpathlineto{\pgfqpoint{6.691752in}{2.290147in}}%
\pgfpathlineto{\pgfqpoint{6.691895in}{2.289788in}}%
\pgfpathlineto{\pgfqpoint{6.692038in}{2.290214in}}%
\pgfpathlineto{\pgfqpoint{6.692038in}{2.290214in}}%
\pgfpathlineto{\pgfqpoint{6.693039in}{2.291058in}}%
\pgfpathlineto{\pgfqpoint{6.693325in}{2.290763in}}%
\pgfpathlineto{\pgfqpoint{6.694039in}{2.289694in}}%
\pgfpathlineto{\pgfqpoint{6.694467in}{2.291079in}}%
\pgfpathlineto{\pgfqpoint{6.695750in}{2.291208in}}%
\pgfpathlineto{\pgfqpoint{6.697314in}{2.291631in}}%
\pgfpathlineto{\pgfqpoint{6.698592in}{2.291839in}}%
\pgfpathlineto{\pgfqpoint{6.698733in}{2.291732in}}%
\pgfpathlineto{\pgfqpoint{6.699017in}{2.291252in}}%
\pgfpathlineto{\pgfqpoint{6.699442in}{2.291606in}}%
\pgfpathlineto{\pgfqpoint{6.700008in}{2.293288in}}%
\pgfpathlineto{\pgfqpoint{6.700574in}{2.291912in}}%
\pgfpathlineto{\pgfqpoint{6.700715in}{2.291184in}}%
\pgfpathlineto{\pgfqpoint{6.701421in}{2.291878in}}%
\pgfpathlineto{\pgfqpoint{6.701421in}{2.291878in}}%
\pgfpathlineto{\pgfqpoint{6.701845in}{2.293825in}}%
\pgfpathlineto{\pgfqpoint{6.702550in}{2.292129in}}%
\pgfpathlineto{\pgfqpoint{6.703536in}{2.293723in}}%
\pgfpathlineto{\pgfqpoint{6.703676in}{2.294322in}}%
\pgfpathlineto{\pgfqpoint{6.704098in}{2.293846in}}%
\pgfpathlineto{\pgfqpoint{6.704098in}{2.293846in}}%
\pgfpathlineto{\pgfqpoint{6.705082in}{2.293502in}}%
\pgfpathlineto{\pgfqpoint{6.705222in}{2.293631in}}%
\pgfpathlineto{\pgfqpoint{6.707185in}{2.295791in}}%
\pgfpathlineto{\pgfqpoint{6.707605in}{2.295392in}}%
\pgfpathlineto{\pgfqpoint{6.707745in}{2.295855in}}%
\pgfpathlineto{\pgfqpoint{6.707745in}{2.295855in}}%
\pgfpathlineto{\pgfqpoint{6.707884in}{2.296096in}}%
\pgfpathlineto{\pgfqpoint{6.708024in}{2.295766in}}%
\pgfpathlineto{\pgfqpoint{6.708024in}{2.295766in}}%
\pgfpathlineto{\pgfqpoint{6.708164in}{2.295348in}}%
\pgfpathlineto{\pgfqpoint{6.708164in}{2.295348in}}%
\pgfpathlineto{\pgfqpoint{6.708164in}{2.295348in}}%
\pgfpathlineto{\pgfqpoint{6.708723in}{2.297120in}}%
\pgfpathlineto{\pgfqpoint{6.709421in}{2.296296in}}%
\pgfpathlineto{\pgfqpoint{6.709700in}{2.295692in}}%
\pgfpathlineto{\pgfqpoint{6.709979in}{2.296607in}}%
\pgfpathlineto{\pgfqpoint{6.709979in}{2.296607in}}%
\pgfpathlineto{\pgfqpoint{6.711093in}{2.297926in}}%
\pgfpathlineto{\pgfqpoint{6.711371in}{2.297576in}}%
\pgfpathlineto{\pgfqpoint{6.713038in}{2.296047in}}%
\pgfpathlineto{\pgfqpoint{6.713177in}{2.296442in}}%
\pgfpathlineto{\pgfqpoint{6.713871in}{2.298722in}}%
\pgfpathlineto{\pgfqpoint{6.714425in}{2.297264in}}%
\pgfpathlineto{\pgfqpoint{6.714563in}{2.296599in}}%
\pgfpathlineto{\pgfqpoint{6.714840in}{2.297346in}}%
\pgfpathlineto{\pgfqpoint{6.714840in}{2.297346in}}%
\pgfpathlineto{\pgfqpoint{6.714978in}{2.297531in}}%
\pgfpathlineto{\pgfqpoint{6.714978in}{2.297531in}}%
\pgfpathlineto{\pgfqpoint{6.714978in}{2.297531in}}%
\pgfpathlineto{\pgfqpoint{6.715117in}{2.297084in}}%
\pgfpathlineto{\pgfqpoint{6.715393in}{2.297530in}}%
\pgfpathlineto{\pgfqpoint{6.715393in}{2.297530in}}%
\pgfpathlineto{\pgfqpoint{6.715808in}{2.300027in}}%
\pgfpathlineto{\pgfqpoint{6.716223in}{2.297609in}}%
\pgfpathlineto{\pgfqpoint{6.716637in}{2.296383in}}%
\pgfpathlineto{\pgfqpoint{6.717051in}{2.297414in}}%
\pgfpathlineto{\pgfqpoint{6.717051in}{2.297414in}}%
\pgfpathlineto{\pgfqpoint{6.717602in}{2.298901in}}%
\pgfpathlineto{\pgfqpoint{6.718429in}{2.298031in}}%
\pgfpathlineto{\pgfqpoint{6.718566in}{2.297780in}}%
\pgfpathlineto{\pgfqpoint{6.718566in}{2.297780in}}%
\pgfpathlineto{\pgfqpoint{6.718566in}{2.297780in}}%
\pgfpathlineto{\pgfqpoint{6.720216in}{2.299884in}}%
\pgfpathlineto{\pgfqpoint{6.721176in}{2.298939in}}%
\pgfpathlineto{\pgfqpoint{6.721313in}{2.299009in}}%
\pgfpathlineto{\pgfqpoint{6.722683in}{2.304020in}}%
\pgfpathlineto{\pgfqpoint{6.723230in}{2.302970in}}%
\pgfpathlineto{\pgfqpoint{6.723503in}{2.302791in}}%
\pgfpathlineto{\pgfqpoint{6.723503in}{2.302791in}}%
\pgfpathlineto{\pgfqpoint{6.723503in}{2.302791in}}%
\pgfpathlineto{\pgfqpoint{6.724868in}{2.305950in}}%
\pgfpathlineto{\pgfqpoint{6.725140in}{2.305648in}}%
\pgfpathlineto{\pgfqpoint{6.726638in}{2.303047in}}%
\pgfpathlineto{\pgfqpoint{6.727725in}{2.306156in}}%
\pgfpathlineto{\pgfqpoint{6.727861in}{2.305689in}}%
\pgfpathlineto{\pgfqpoint{6.728132in}{2.303617in}}%
\pgfpathlineto{\pgfqpoint{6.728539in}{2.305674in}}%
\pgfpathlineto{\pgfqpoint{6.728539in}{2.305674in}}%
\pgfpathlineto{\pgfqpoint{6.728675in}{2.305895in}}%
\pgfpathlineto{\pgfqpoint{6.728675in}{2.305895in}}%
\pgfpathlineto{\pgfqpoint{6.728675in}{2.305895in}}%
\pgfpathlineto{\pgfqpoint{6.728810in}{2.305503in}}%
\pgfpathlineto{\pgfqpoint{6.728810in}{2.305503in}}%
\pgfpathlineto{\pgfqpoint{6.728810in}{2.305503in}}%
\pgfpathlineto{\pgfqpoint{6.729488in}{2.308633in}}%
\pgfpathlineto{\pgfqpoint{6.729894in}{2.305845in}}%
\pgfpathlineto{\pgfqpoint{6.730300in}{2.302788in}}%
\pgfpathlineto{\pgfqpoint{6.730976in}{2.304987in}}%
\pgfpathlineto{\pgfqpoint{6.731246in}{2.305758in}}%
\pgfpathlineto{\pgfqpoint{6.731381in}{2.305057in}}%
\pgfpathlineto{\pgfqpoint{6.731381in}{2.305057in}}%
\pgfpathlineto{\pgfqpoint{6.731786in}{2.303357in}}%
\pgfpathlineto{\pgfqpoint{6.732191in}{2.304877in}}%
\pgfpathlineto{\pgfqpoint{6.732191in}{2.304877in}}%
\pgfpathlineto{\pgfqpoint{6.732730in}{2.306896in}}%
\pgfpathlineto{\pgfqpoint{6.733269in}{2.305633in}}%
\pgfpathlineto{\pgfqpoint{6.733404in}{2.305000in}}%
\pgfpathlineto{\pgfqpoint{6.733404in}{2.305000in}}%
\pgfusepath{stroke}%
\end{pgfscope}%
\begin{pgfscope}%
\pgfpathrectangle{\pgfqpoint{1.000000in}{0.300000in}}{\pgfqpoint{6.200000in}{2.400000in}} %
\pgfusepath{clip}%
\pgfsetbuttcap%
\pgfsetroundjoin%
\pgfsetlinewidth{0.501875pt}%
\definecolor{currentstroke}{rgb}{0.000000,0.000000,0.000000}%
\pgfsetstrokecolor{currentstroke}%
\pgfsetdash{{1.000000pt}{3.000000pt}}{0.000000pt}%
\pgfpathmoveto{\pgfqpoint{1.000000in}{0.300000in}}%
\pgfpathlineto{\pgfqpoint{1.000000in}{2.700000in}}%
\pgfusepath{stroke}%
\end{pgfscope}%
\begin{pgfscope}%
\pgfsetbuttcap%
\pgfsetroundjoin%
\definecolor{currentfill}{rgb}{0.000000,0.000000,0.000000}%
\pgfsetfillcolor{currentfill}%
\pgfsetlinewidth{0.501875pt}%
\definecolor{currentstroke}{rgb}{0.000000,0.000000,0.000000}%
\pgfsetstrokecolor{currentstroke}%
\pgfsetdash{}{0pt}%
\pgfsys@defobject{currentmarker}{\pgfqpoint{0.000000in}{0.000000in}}{\pgfqpoint{0.000000in}{0.055556in}}{%
\pgfpathmoveto{\pgfqpoint{0.000000in}{0.000000in}}%
\pgfpathlineto{\pgfqpoint{0.000000in}{0.055556in}}%
\pgfusepath{stroke,fill}%
}%
\begin{pgfscope}%
\pgfsys@transformshift{1.000000in}{0.300000in}%
\pgfsys@useobject{currentmarker}{}%
\end{pgfscope}%
\end{pgfscope}%
\begin{pgfscope}%
\pgfsetbuttcap%
\pgfsetroundjoin%
\definecolor{currentfill}{rgb}{0.000000,0.000000,0.000000}%
\pgfsetfillcolor{currentfill}%
\pgfsetlinewidth{0.501875pt}%
\definecolor{currentstroke}{rgb}{0.000000,0.000000,0.000000}%
\pgfsetstrokecolor{currentstroke}%
\pgfsetdash{}{0pt}%
\pgfsys@defobject{currentmarker}{\pgfqpoint{0.000000in}{-0.055556in}}{\pgfqpoint{0.000000in}{0.000000in}}{%
\pgfpathmoveto{\pgfqpoint{0.000000in}{0.000000in}}%
\pgfpathlineto{\pgfqpoint{0.000000in}{-0.055556in}}%
\pgfusepath{stroke,fill}%
}%
\begin{pgfscope}%
\pgfsys@transformshift{1.000000in}{2.700000in}%
\pgfsys@useobject{currentmarker}{}%
\end{pgfscope}%
\end{pgfscope}%
\begin{pgfscope}%
\pgftext[left,bottom,x=0.839506in,y=0.104024in,rotate=0.000000]{{\sffamily\fontsize{12.000000}{14.400000}\selectfont \(\displaystyle {10^{-1}}\)}}
%
\end{pgfscope}%
\begin{pgfscope}%
\pgfpathrectangle{\pgfqpoint{1.000000in}{0.300000in}}{\pgfqpoint{6.200000in}{2.400000in}} %
\pgfusepath{clip}%
\pgfsetbuttcap%
\pgfsetroundjoin%
\pgfsetlinewidth{0.501875pt}%
\definecolor{currentstroke}{rgb}{0.000000,0.000000,0.000000}%
\pgfsetstrokecolor{currentstroke}%
\pgfsetdash{{1.000000pt}{3.000000pt}}{0.000000pt}%
\pgfpathmoveto{\pgfqpoint{2.550000in}{0.300000in}}%
\pgfpathlineto{\pgfqpoint{2.550000in}{2.700000in}}%
\pgfusepath{stroke}%
\end{pgfscope}%
\begin{pgfscope}%
\pgfsetbuttcap%
\pgfsetroundjoin%
\definecolor{currentfill}{rgb}{0.000000,0.000000,0.000000}%
\pgfsetfillcolor{currentfill}%
\pgfsetlinewidth{0.501875pt}%
\definecolor{currentstroke}{rgb}{0.000000,0.000000,0.000000}%
\pgfsetstrokecolor{currentstroke}%
\pgfsetdash{}{0pt}%
\pgfsys@defobject{currentmarker}{\pgfqpoint{0.000000in}{0.000000in}}{\pgfqpoint{0.000000in}{0.055556in}}{%
\pgfpathmoveto{\pgfqpoint{0.000000in}{0.000000in}}%
\pgfpathlineto{\pgfqpoint{0.000000in}{0.055556in}}%
\pgfusepath{stroke,fill}%
}%
\begin{pgfscope}%
\pgfsys@transformshift{2.550000in}{0.300000in}%
\pgfsys@useobject{currentmarker}{}%
\end{pgfscope}%
\end{pgfscope}%
\begin{pgfscope}%
\pgfsetbuttcap%
\pgfsetroundjoin%
\definecolor{currentfill}{rgb}{0.000000,0.000000,0.000000}%
\pgfsetfillcolor{currentfill}%
\pgfsetlinewidth{0.501875pt}%
\definecolor{currentstroke}{rgb}{0.000000,0.000000,0.000000}%
\pgfsetstrokecolor{currentstroke}%
\pgfsetdash{}{0pt}%
\pgfsys@defobject{currentmarker}{\pgfqpoint{0.000000in}{-0.055556in}}{\pgfqpoint{0.000000in}{0.000000in}}{%
\pgfpathmoveto{\pgfqpoint{0.000000in}{0.000000in}}%
\pgfpathlineto{\pgfqpoint{0.000000in}{-0.055556in}}%
\pgfusepath{stroke,fill}%
}%
\begin{pgfscope}%
\pgfsys@transformshift{2.550000in}{2.700000in}%
\pgfsys@useobject{currentmarker}{}%
\end{pgfscope}%
\end{pgfscope}%
\begin{pgfscope}%
\pgftext[left,bottom,x=2.435417in,y=0.104024in,rotate=0.000000]{{\sffamily\fontsize{12.000000}{14.400000}\selectfont \(\displaystyle {10^{0}}\)}}
%
\end{pgfscope}%
\begin{pgfscope}%
\pgfpathrectangle{\pgfqpoint{1.000000in}{0.300000in}}{\pgfqpoint{6.200000in}{2.400000in}} %
\pgfusepath{clip}%
\pgfsetbuttcap%
\pgfsetroundjoin%
\pgfsetlinewidth{0.501875pt}%
\definecolor{currentstroke}{rgb}{0.000000,0.000000,0.000000}%
\pgfsetstrokecolor{currentstroke}%
\pgfsetdash{{1.000000pt}{3.000000pt}}{0.000000pt}%
\pgfpathmoveto{\pgfqpoint{4.100000in}{0.300000in}}%
\pgfpathlineto{\pgfqpoint{4.100000in}{2.700000in}}%
\pgfusepath{stroke}%
\end{pgfscope}%
\begin{pgfscope}%
\pgfsetbuttcap%
\pgfsetroundjoin%
\definecolor{currentfill}{rgb}{0.000000,0.000000,0.000000}%
\pgfsetfillcolor{currentfill}%
\pgfsetlinewidth{0.501875pt}%
\definecolor{currentstroke}{rgb}{0.000000,0.000000,0.000000}%
\pgfsetstrokecolor{currentstroke}%
\pgfsetdash{}{0pt}%
\pgfsys@defobject{currentmarker}{\pgfqpoint{0.000000in}{0.000000in}}{\pgfqpoint{0.000000in}{0.055556in}}{%
\pgfpathmoveto{\pgfqpoint{0.000000in}{0.000000in}}%
\pgfpathlineto{\pgfqpoint{0.000000in}{0.055556in}}%
\pgfusepath{stroke,fill}%
}%
\begin{pgfscope}%
\pgfsys@transformshift{4.100000in}{0.300000in}%
\pgfsys@useobject{currentmarker}{}%
\end{pgfscope}%
\end{pgfscope}%
\begin{pgfscope}%
\pgfsetbuttcap%
\pgfsetroundjoin%
\definecolor{currentfill}{rgb}{0.000000,0.000000,0.000000}%
\pgfsetfillcolor{currentfill}%
\pgfsetlinewidth{0.501875pt}%
\definecolor{currentstroke}{rgb}{0.000000,0.000000,0.000000}%
\pgfsetstrokecolor{currentstroke}%
\pgfsetdash{}{0pt}%
\pgfsys@defobject{currentmarker}{\pgfqpoint{0.000000in}{-0.055556in}}{\pgfqpoint{0.000000in}{0.000000in}}{%
\pgfpathmoveto{\pgfqpoint{0.000000in}{0.000000in}}%
\pgfpathlineto{\pgfqpoint{0.000000in}{-0.055556in}}%
\pgfusepath{stroke,fill}%
}%
\begin{pgfscope}%
\pgfsys@transformshift{4.100000in}{2.700000in}%
\pgfsys@useobject{currentmarker}{}%
\end{pgfscope}%
\end{pgfscope}%
\begin{pgfscope}%
\pgftext[left,bottom,x=3.985417in,y=0.104024in,rotate=0.000000]{{\sffamily\fontsize{12.000000}{14.400000}\selectfont \(\displaystyle {10^{1}}\)}}
%
\end{pgfscope}%
\begin{pgfscope}%
\pgfpathrectangle{\pgfqpoint{1.000000in}{0.300000in}}{\pgfqpoint{6.200000in}{2.400000in}} %
\pgfusepath{clip}%
\pgfsetbuttcap%
\pgfsetroundjoin%
\pgfsetlinewidth{0.501875pt}%
\definecolor{currentstroke}{rgb}{0.000000,0.000000,0.000000}%
\pgfsetstrokecolor{currentstroke}%
\pgfsetdash{{1.000000pt}{3.000000pt}}{0.000000pt}%
\pgfpathmoveto{\pgfqpoint{5.650000in}{0.300000in}}%
\pgfpathlineto{\pgfqpoint{5.650000in}{2.700000in}}%
\pgfusepath{stroke}%
\end{pgfscope}%
\begin{pgfscope}%
\pgfsetbuttcap%
\pgfsetroundjoin%
\definecolor{currentfill}{rgb}{0.000000,0.000000,0.000000}%
\pgfsetfillcolor{currentfill}%
\pgfsetlinewidth{0.501875pt}%
\definecolor{currentstroke}{rgb}{0.000000,0.000000,0.000000}%
\pgfsetstrokecolor{currentstroke}%
\pgfsetdash{}{0pt}%
\pgfsys@defobject{currentmarker}{\pgfqpoint{0.000000in}{0.000000in}}{\pgfqpoint{0.000000in}{0.055556in}}{%
\pgfpathmoveto{\pgfqpoint{0.000000in}{0.000000in}}%
\pgfpathlineto{\pgfqpoint{0.000000in}{0.055556in}}%
\pgfusepath{stroke,fill}%
}%
\begin{pgfscope}%
\pgfsys@transformshift{5.650000in}{0.300000in}%
\pgfsys@useobject{currentmarker}{}%
\end{pgfscope}%
\end{pgfscope}%
\begin{pgfscope}%
\pgfsetbuttcap%
\pgfsetroundjoin%
\definecolor{currentfill}{rgb}{0.000000,0.000000,0.000000}%
\pgfsetfillcolor{currentfill}%
\pgfsetlinewidth{0.501875pt}%
\definecolor{currentstroke}{rgb}{0.000000,0.000000,0.000000}%
\pgfsetstrokecolor{currentstroke}%
\pgfsetdash{}{0pt}%
\pgfsys@defobject{currentmarker}{\pgfqpoint{0.000000in}{-0.055556in}}{\pgfqpoint{0.000000in}{0.000000in}}{%
\pgfpathmoveto{\pgfqpoint{0.000000in}{0.000000in}}%
\pgfpathlineto{\pgfqpoint{0.000000in}{-0.055556in}}%
\pgfusepath{stroke,fill}%
}%
\begin{pgfscope}%
\pgfsys@transformshift{5.650000in}{2.700000in}%
\pgfsys@useobject{currentmarker}{}%
\end{pgfscope}%
\end{pgfscope}%
\begin{pgfscope}%
\pgftext[left,bottom,x=5.535417in,y=0.104024in,rotate=0.000000]{{\sffamily\fontsize{12.000000}{14.400000}\selectfont \(\displaystyle {10^{2}}\)}}
%
\end{pgfscope}%
\begin{pgfscope}%
\pgfpathrectangle{\pgfqpoint{1.000000in}{0.300000in}}{\pgfqpoint{6.200000in}{2.400000in}} %
\pgfusepath{clip}%
\pgfsetbuttcap%
\pgfsetroundjoin%
\pgfsetlinewidth{0.501875pt}%
\definecolor{currentstroke}{rgb}{0.000000,0.000000,0.000000}%
\pgfsetstrokecolor{currentstroke}%
\pgfsetdash{{1.000000pt}{3.000000pt}}{0.000000pt}%
\pgfpathmoveto{\pgfqpoint{7.200000in}{0.300000in}}%
\pgfpathlineto{\pgfqpoint{7.200000in}{2.700000in}}%
\pgfusepath{stroke}%
\end{pgfscope}%
\begin{pgfscope}%
\pgfsetbuttcap%
\pgfsetroundjoin%
\definecolor{currentfill}{rgb}{0.000000,0.000000,0.000000}%
\pgfsetfillcolor{currentfill}%
\pgfsetlinewidth{0.501875pt}%
\definecolor{currentstroke}{rgb}{0.000000,0.000000,0.000000}%
\pgfsetstrokecolor{currentstroke}%
\pgfsetdash{}{0pt}%
\pgfsys@defobject{currentmarker}{\pgfqpoint{0.000000in}{0.000000in}}{\pgfqpoint{0.000000in}{0.055556in}}{%
\pgfpathmoveto{\pgfqpoint{0.000000in}{0.000000in}}%
\pgfpathlineto{\pgfqpoint{0.000000in}{0.055556in}}%
\pgfusepath{stroke,fill}%
}%
\begin{pgfscope}%
\pgfsys@transformshift{7.200000in}{0.300000in}%
\pgfsys@useobject{currentmarker}{}%
\end{pgfscope}%
\end{pgfscope}%
\begin{pgfscope}%
\pgfsetbuttcap%
\pgfsetroundjoin%
\definecolor{currentfill}{rgb}{0.000000,0.000000,0.000000}%
\pgfsetfillcolor{currentfill}%
\pgfsetlinewidth{0.501875pt}%
\definecolor{currentstroke}{rgb}{0.000000,0.000000,0.000000}%
\pgfsetstrokecolor{currentstroke}%
\pgfsetdash{}{0pt}%
\pgfsys@defobject{currentmarker}{\pgfqpoint{0.000000in}{-0.055556in}}{\pgfqpoint{0.000000in}{0.000000in}}{%
\pgfpathmoveto{\pgfqpoint{0.000000in}{0.000000in}}%
\pgfpathlineto{\pgfqpoint{0.000000in}{-0.055556in}}%
\pgfusepath{stroke,fill}%
}%
\begin{pgfscope}%
\pgfsys@transformshift{7.200000in}{2.700000in}%
\pgfsys@useobject{currentmarker}{}%
\end{pgfscope}%
\end{pgfscope}%
\begin{pgfscope}%
\pgftext[left,bottom,x=7.085417in,y=0.104024in,rotate=0.000000]{{\sffamily\fontsize{12.000000}{14.400000}\selectfont \(\displaystyle {10^{3}}\)}}
%
\end{pgfscope}%
\begin{pgfscope}%
\pgfsetbuttcap%
\pgfsetroundjoin%
\definecolor{currentfill}{rgb}{0.000000,0.000000,0.000000}%
\pgfsetfillcolor{currentfill}%
\pgfsetlinewidth{0.501875pt}%
\definecolor{currentstroke}{rgb}{0.000000,0.000000,0.000000}%
\pgfsetstrokecolor{currentstroke}%
\pgfsetdash{}{0pt}%
\pgfsys@defobject{currentmarker}{\pgfqpoint{0.000000in}{0.000000in}}{\pgfqpoint{0.000000in}{0.027778in}}{%
\pgfpathmoveto{\pgfqpoint{0.000000in}{0.000000in}}%
\pgfpathlineto{\pgfqpoint{0.000000in}{0.027778in}}%
\pgfusepath{stroke,fill}%
}%
\begin{pgfscope}%
\pgfsys@transformshift{1.466596in}{0.300000in}%
\pgfsys@useobject{currentmarker}{}%
\end{pgfscope}%
\end{pgfscope}%
\begin{pgfscope}%
\pgfsetbuttcap%
\pgfsetroundjoin%
\definecolor{currentfill}{rgb}{0.000000,0.000000,0.000000}%
\pgfsetfillcolor{currentfill}%
\pgfsetlinewidth{0.501875pt}%
\definecolor{currentstroke}{rgb}{0.000000,0.000000,0.000000}%
\pgfsetstrokecolor{currentstroke}%
\pgfsetdash{}{0pt}%
\pgfsys@defobject{currentmarker}{\pgfqpoint{0.000000in}{-0.027778in}}{\pgfqpoint{0.000000in}{0.000000in}}{%
\pgfpathmoveto{\pgfqpoint{0.000000in}{0.000000in}}%
\pgfpathlineto{\pgfqpoint{0.000000in}{-0.027778in}}%
\pgfusepath{stroke,fill}%
}%
\begin{pgfscope}%
\pgfsys@transformshift{1.466596in}{2.700000in}%
\pgfsys@useobject{currentmarker}{}%
\end{pgfscope}%
\end{pgfscope}%
\begin{pgfscope}%
\pgfsetbuttcap%
\pgfsetroundjoin%
\definecolor{currentfill}{rgb}{0.000000,0.000000,0.000000}%
\pgfsetfillcolor{currentfill}%
\pgfsetlinewidth{0.501875pt}%
\definecolor{currentstroke}{rgb}{0.000000,0.000000,0.000000}%
\pgfsetstrokecolor{currentstroke}%
\pgfsetdash{}{0pt}%
\pgfsys@defobject{currentmarker}{\pgfqpoint{0.000000in}{0.000000in}}{\pgfqpoint{0.000000in}{0.027778in}}{%
\pgfpathmoveto{\pgfqpoint{0.000000in}{0.000000in}}%
\pgfpathlineto{\pgfqpoint{0.000000in}{0.027778in}}%
\pgfusepath{stroke,fill}%
}%
\begin{pgfscope}%
\pgfsys@transformshift{1.739538in}{0.300000in}%
\pgfsys@useobject{currentmarker}{}%
\end{pgfscope}%
\end{pgfscope}%
\begin{pgfscope}%
\pgfsetbuttcap%
\pgfsetroundjoin%
\definecolor{currentfill}{rgb}{0.000000,0.000000,0.000000}%
\pgfsetfillcolor{currentfill}%
\pgfsetlinewidth{0.501875pt}%
\definecolor{currentstroke}{rgb}{0.000000,0.000000,0.000000}%
\pgfsetstrokecolor{currentstroke}%
\pgfsetdash{}{0pt}%
\pgfsys@defobject{currentmarker}{\pgfqpoint{0.000000in}{-0.027778in}}{\pgfqpoint{0.000000in}{0.000000in}}{%
\pgfpathmoveto{\pgfqpoint{0.000000in}{0.000000in}}%
\pgfpathlineto{\pgfqpoint{0.000000in}{-0.027778in}}%
\pgfusepath{stroke,fill}%
}%
\begin{pgfscope}%
\pgfsys@transformshift{1.739538in}{2.700000in}%
\pgfsys@useobject{currentmarker}{}%
\end{pgfscope}%
\end{pgfscope}%
\begin{pgfscope}%
\pgfsetbuttcap%
\pgfsetroundjoin%
\definecolor{currentfill}{rgb}{0.000000,0.000000,0.000000}%
\pgfsetfillcolor{currentfill}%
\pgfsetlinewidth{0.501875pt}%
\definecolor{currentstroke}{rgb}{0.000000,0.000000,0.000000}%
\pgfsetstrokecolor{currentstroke}%
\pgfsetdash{}{0pt}%
\pgfsys@defobject{currentmarker}{\pgfqpoint{0.000000in}{0.000000in}}{\pgfqpoint{0.000000in}{0.027778in}}{%
\pgfpathmoveto{\pgfqpoint{0.000000in}{0.000000in}}%
\pgfpathlineto{\pgfqpoint{0.000000in}{0.027778in}}%
\pgfusepath{stroke,fill}%
}%
\begin{pgfscope}%
\pgfsys@transformshift{1.933193in}{0.300000in}%
\pgfsys@useobject{currentmarker}{}%
\end{pgfscope}%
\end{pgfscope}%
\begin{pgfscope}%
\pgfsetbuttcap%
\pgfsetroundjoin%
\definecolor{currentfill}{rgb}{0.000000,0.000000,0.000000}%
\pgfsetfillcolor{currentfill}%
\pgfsetlinewidth{0.501875pt}%
\definecolor{currentstroke}{rgb}{0.000000,0.000000,0.000000}%
\pgfsetstrokecolor{currentstroke}%
\pgfsetdash{}{0pt}%
\pgfsys@defobject{currentmarker}{\pgfqpoint{0.000000in}{-0.027778in}}{\pgfqpoint{0.000000in}{0.000000in}}{%
\pgfpathmoveto{\pgfqpoint{0.000000in}{0.000000in}}%
\pgfpathlineto{\pgfqpoint{0.000000in}{-0.027778in}}%
\pgfusepath{stroke,fill}%
}%
\begin{pgfscope}%
\pgfsys@transformshift{1.933193in}{2.700000in}%
\pgfsys@useobject{currentmarker}{}%
\end{pgfscope}%
\end{pgfscope}%
\begin{pgfscope}%
\pgfsetbuttcap%
\pgfsetroundjoin%
\definecolor{currentfill}{rgb}{0.000000,0.000000,0.000000}%
\pgfsetfillcolor{currentfill}%
\pgfsetlinewidth{0.501875pt}%
\definecolor{currentstroke}{rgb}{0.000000,0.000000,0.000000}%
\pgfsetstrokecolor{currentstroke}%
\pgfsetdash{}{0pt}%
\pgfsys@defobject{currentmarker}{\pgfqpoint{0.000000in}{0.000000in}}{\pgfqpoint{0.000000in}{0.027778in}}{%
\pgfpathmoveto{\pgfqpoint{0.000000in}{0.000000in}}%
\pgfpathlineto{\pgfqpoint{0.000000in}{0.027778in}}%
\pgfusepath{stroke,fill}%
}%
\begin{pgfscope}%
\pgfsys@transformshift{2.083404in}{0.300000in}%
\pgfsys@useobject{currentmarker}{}%
\end{pgfscope}%
\end{pgfscope}%
\begin{pgfscope}%
\pgfsetbuttcap%
\pgfsetroundjoin%
\definecolor{currentfill}{rgb}{0.000000,0.000000,0.000000}%
\pgfsetfillcolor{currentfill}%
\pgfsetlinewidth{0.501875pt}%
\definecolor{currentstroke}{rgb}{0.000000,0.000000,0.000000}%
\pgfsetstrokecolor{currentstroke}%
\pgfsetdash{}{0pt}%
\pgfsys@defobject{currentmarker}{\pgfqpoint{0.000000in}{-0.027778in}}{\pgfqpoint{0.000000in}{0.000000in}}{%
\pgfpathmoveto{\pgfqpoint{0.000000in}{0.000000in}}%
\pgfpathlineto{\pgfqpoint{0.000000in}{-0.027778in}}%
\pgfusepath{stroke,fill}%
}%
\begin{pgfscope}%
\pgfsys@transformshift{2.083404in}{2.700000in}%
\pgfsys@useobject{currentmarker}{}%
\end{pgfscope}%
\end{pgfscope}%
\begin{pgfscope}%
\pgfsetbuttcap%
\pgfsetroundjoin%
\definecolor{currentfill}{rgb}{0.000000,0.000000,0.000000}%
\pgfsetfillcolor{currentfill}%
\pgfsetlinewidth{0.501875pt}%
\definecolor{currentstroke}{rgb}{0.000000,0.000000,0.000000}%
\pgfsetstrokecolor{currentstroke}%
\pgfsetdash{}{0pt}%
\pgfsys@defobject{currentmarker}{\pgfqpoint{0.000000in}{0.000000in}}{\pgfqpoint{0.000000in}{0.027778in}}{%
\pgfpathmoveto{\pgfqpoint{0.000000in}{0.000000in}}%
\pgfpathlineto{\pgfqpoint{0.000000in}{0.027778in}}%
\pgfusepath{stroke,fill}%
}%
\begin{pgfscope}%
\pgfsys@transformshift{2.206134in}{0.300000in}%
\pgfsys@useobject{currentmarker}{}%
\end{pgfscope}%
\end{pgfscope}%
\begin{pgfscope}%
\pgfsetbuttcap%
\pgfsetroundjoin%
\definecolor{currentfill}{rgb}{0.000000,0.000000,0.000000}%
\pgfsetfillcolor{currentfill}%
\pgfsetlinewidth{0.501875pt}%
\definecolor{currentstroke}{rgb}{0.000000,0.000000,0.000000}%
\pgfsetstrokecolor{currentstroke}%
\pgfsetdash{}{0pt}%
\pgfsys@defobject{currentmarker}{\pgfqpoint{0.000000in}{-0.027778in}}{\pgfqpoint{0.000000in}{0.000000in}}{%
\pgfpathmoveto{\pgfqpoint{0.000000in}{0.000000in}}%
\pgfpathlineto{\pgfqpoint{0.000000in}{-0.027778in}}%
\pgfusepath{stroke,fill}%
}%
\begin{pgfscope}%
\pgfsys@transformshift{2.206134in}{2.700000in}%
\pgfsys@useobject{currentmarker}{}%
\end{pgfscope}%
\end{pgfscope}%
\begin{pgfscope}%
\pgfsetbuttcap%
\pgfsetroundjoin%
\definecolor{currentfill}{rgb}{0.000000,0.000000,0.000000}%
\pgfsetfillcolor{currentfill}%
\pgfsetlinewidth{0.501875pt}%
\definecolor{currentstroke}{rgb}{0.000000,0.000000,0.000000}%
\pgfsetstrokecolor{currentstroke}%
\pgfsetdash{}{0pt}%
\pgfsys@defobject{currentmarker}{\pgfqpoint{0.000000in}{0.000000in}}{\pgfqpoint{0.000000in}{0.027778in}}{%
\pgfpathmoveto{\pgfqpoint{0.000000in}{0.000000in}}%
\pgfpathlineto{\pgfqpoint{0.000000in}{0.027778in}}%
\pgfusepath{stroke,fill}%
}%
\begin{pgfscope}%
\pgfsys@transformshift{2.309902in}{0.300000in}%
\pgfsys@useobject{currentmarker}{}%
\end{pgfscope}%
\end{pgfscope}%
\begin{pgfscope}%
\pgfsetbuttcap%
\pgfsetroundjoin%
\definecolor{currentfill}{rgb}{0.000000,0.000000,0.000000}%
\pgfsetfillcolor{currentfill}%
\pgfsetlinewidth{0.501875pt}%
\definecolor{currentstroke}{rgb}{0.000000,0.000000,0.000000}%
\pgfsetstrokecolor{currentstroke}%
\pgfsetdash{}{0pt}%
\pgfsys@defobject{currentmarker}{\pgfqpoint{0.000000in}{-0.027778in}}{\pgfqpoint{0.000000in}{0.000000in}}{%
\pgfpathmoveto{\pgfqpoint{0.000000in}{0.000000in}}%
\pgfpathlineto{\pgfqpoint{0.000000in}{-0.027778in}}%
\pgfusepath{stroke,fill}%
}%
\begin{pgfscope}%
\pgfsys@transformshift{2.309902in}{2.700000in}%
\pgfsys@useobject{currentmarker}{}%
\end{pgfscope}%
\end{pgfscope}%
\begin{pgfscope}%
\pgfsetbuttcap%
\pgfsetroundjoin%
\definecolor{currentfill}{rgb}{0.000000,0.000000,0.000000}%
\pgfsetfillcolor{currentfill}%
\pgfsetlinewidth{0.501875pt}%
\definecolor{currentstroke}{rgb}{0.000000,0.000000,0.000000}%
\pgfsetstrokecolor{currentstroke}%
\pgfsetdash{}{0pt}%
\pgfsys@defobject{currentmarker}{\pgfqpoint{0.000000in}{0.000000in}}{\pgfqpoint{0.000000in}{0.027778in}}{%
\pgfpathmoveto{\pgfqpoint{0.000000in}{0.000000in}}%
\pgfpathlineto{\pgfqpoint{0.000000in}{0.027778in}}%
\pgfusepath{stroke,fill}%
}%
\begin{pgfscope}%
\pgfsys@transformshift{2.399789in}{0.300000in}%
\pgfsys@useobject{currentmarker}{}%
\end{pgfscope}%
\end{pgfscope}%
\begin{pgfscope}%
\pgfsetbuttcap%
\pgfsetroundjoin%
\definecolor{currentfill}{rgb}{0.000000,0.000000,0.000000}%
\pgfsetfillcolor{currentfill}%
\pgfsetlinewidth{0.501875pt}%
\definecolor{currentstroke}{rgb}{0.000000,0.000000,0.000000}%
\pgfsetstrokecolor{currentstroke}%
\pgfsetdash{}{0pt}%
\pgfsys@defobject{currentmarker}{\pgfqpoint{0.000000in}{-0.027778in}}{\pgfqpoint{0.000000in}{0.000000in}}{%
\pgfpathmoveto{\pgfqpoint{0.000000in}{0.000000in}}%
\pgfpathlineto{\pgfqpoint{0.000000in}{-0.027778in}}%
\pgfusepath{stroke,fill}%
}%
\begin{pgfscope}%
\pgfsys@transformshift{2.399789in}{2.700000in}%
\pgfsys@useobject{currentmarker}{}%
\end{pgfscope}%
\end{pgfscope}%
\begin{pgfscope}%
\pgfsetbuttcap%
\pgfsetroundjoin%
\definecolor{currentfill}{rgb}{0.000000,0.000000,0.000000}%
\pgfsetfillcolor{currentfill}%
\pgfsetlinewidth{0.501875pt}%
\definecolor{currentstroke}{rgb}{0.000000,0.000000,0.000000}%
\pgfsetstrokecolor{currentstroke}%
\pgfsetdash{}{0pt}%
\pgfsys@defobject{currentmarker}{\pgfqpoint{0.000000in}{0.000000in}}{\pgfqpoint{0.000000in}{0.027778in}}{%
\pgfpathmoveto{\pgfqpoint{0.000000in}{0.000000in}}%
\pgfpathlineto{\pgfqpoint{0.000000in}{0.027778in}}%
\pgfusepath{stroke,fill}%
}%
\begin{pgfscope}%
\pgfsys@transformshift{2.479076in}{0.300000in}%
\pgfsys@useobject{currentmarker}{}%
\end{pgfscope}%
\end{pgfscope}%
\begin{pgfscope}%
\pgfsetbuttcap%
\pgfsetroundjoin%
\definecolor{currentfill}{rgb}{0.000000,0.000000,0.000000}%
\pgfsetfillcolor{currentfill}%
\pgfsetlinewidth{0.501875pt}%
\definecolor{currentstroke}{rgb}{0.000000,0.000000,0.000000}%
\pgfsetstrokecolor{currentstroke}%
\pgfsetdash{}{0pt}%
\pgfsys@defobject{currentmarker}{\pgfqpoint{0.000000in}{-0.027778in}}{\pgfqpoint{0.000000in}{0.000000in}}{%
\pgfpathmoveto{\pgfqpoint{0.000000in}{0.000000in}}%
\pgfpathlineto{\pgfqpoint{0.000000in}{-0.027778in}}%
\pgfusepath{stroke,fill}%
}%
\begin{pgfscope}%
\pgfsys@transformshift{2.479076in}{2.700000in}%
\pgfsys@useobject{currentmarker}{}%
\end{pgfscope}%
\end{pgfscope}%
\begin{pgfscope}%
\pgfsetbuttcap%
\pgfsetroundjoin%
\definecolor{currentfill}{rgb}{0.000000,0.000000,0.000000}%
\pgfsetfillcolor{currentfill}%
\pgfsetlinewidth{0.501875pt}%
\definecolor{currentstroke}{rgb}{0.000000,0.000000,0.000000}%
\pgfsetstrokecolor{currentstroke}%
\pgfsetdash{}{0pt}%
\pgfsys@defobject{currentmarker}{\pgfqpoint{0.000000in}{0.000000in}}{\pgfqpoint{0.000000in}{0.027778in}}{%
\pgfpathmoveto{\pgfqpoint{0.000000in}{0.000000in}}%
\pgfpathlineto{\pgfqpoint{0.000000in}{0.027778in}}%
\pgfusepath{stroke,fill}%
}%
\begin{pgfscope}%
\pgfsys@transformshift{3.016596in}{0.300000in}%
\pgfsys@useobject{currentmarker}{}%
\end{pgfscope}%
\end{pgfscope}%
\begin{pgfscope}%
\pgfsetbuttcap%
\pgfsetroundjoin%
\definecolor{currentfill}{rgb}{0.000000,0.000000,0.000000}%
\pgfsetfillcolor{currentfill}%
\pgfsetlinewidth{0.501875pt}%
\definecolor{currentstroke}{rgb}{0.000000,0.000000,0.000000}%
\pgfsetstrokecolor{currentstroke}%
\pgfsetdash{}{0pt}%
\pgfsys@defobject{currentmarker}{\pgfqpoint{0.000000in}{-0.027778in}}{\pgfqpoint{0.000000in}{0.000000in}}{%
\pgfpathmoveto{\pgfqpoint{0.000000in}{0.000000in}}%
\pgfpathlineto{\pgfqpoint{0.000000in}{-0.027778in}}%
\pgfusepath{stroke,fill}%
}%
\begin{pgfscope}%
\pgfsys@transformshift{3.016596in}{2.700000in}%
\pgfsys@useobject{currentmarker}{}%
\end{pgfscope}%
\end{pgfscope}%
\begin{pgfscope}%
\pgfsetbuttcap%
\pgfsetroundjoin%
\definecolor{currentfill}{rgb}{0.000000,0.000000,0.000000}%
\pgfsetfillcolor{currentfill}%
\pgfsetlinewidth{0.501875pt}%
\definecolor{currentstroke}{rgb}{0.000000,0.000000,0.000000}%
\pgfsetstrokecolor{currentstroke}%
\pgfsetdash{}{0pt}%
\pgfsys@defobject{currentmarker}{\pgfqpoint{0.000000in}{0.000000in}}{\pgfqpoint{0.000000in}{0.027778in}}{%
\pgfpathmoveto{\pgfqpoint{0.000000in}{0.000000in}}%
\pgfpathlineto{\pgfqpoint{0.000000in}{0.027778in}}%
\pgfusepath{stroke,fill}%
}%
\begin{pgfscope}%
\pgfsys@transformshift{3.289538in}{0.300000in}%
\pgfsys@useobject{currentmarker}{}%
\end{pgfscope}%
\end{pgfscope}%
\begin{pgfscope}%
\pgfsetbuttcap%
\pgfsetroundjoin%
\definecolor{currentfill}{rgb}{0.000000,0.000000,0.000000}%
\pgfsetfillcolor{currentfill}%
\pgfsetlinewidth{0.501875pt}%
\definecolor{currentstroke}{rgb}{0.000000,0.000000,0.000000}%
\pgfsetstrokecolor{currentstroke}%
\pgfsetdash{}{0pt}%
\pgfsys@defobject{currentmarker}{\pgfqpoint{0.000000in}{-0.027778in}}{\pgfqpoint{0.000000in}{0.000000in}}{%
\pgfpathmoveto{\pgfqpoint{0.000000in}{0.000000in}}%
\pgfpathlineto{\pgfqpoint{0.000000in}{-0.027778in}}%
\pgfusepath{stroke,fill}%
}%
\begin{pgfscope}%
\pgfsys@transformshift{3.289538in}{2.700000in}%
\pgfsys@useobject{currentmarker}{}%
\end{pgfscope}%
\end{pgfscope}%
\begin{pgfscope}%
\pgfsetbuttcap%
\pgfsetroundjoin%
\definecolor{currentfill}{rgb}{0.000000,0.000000,0.000000}%
\pgfsetfillcolor{currentfill}%
\pgfsetlinewidth{0.501875pt}%
\definecolor{currentstroke}{rgb}{0.000000,0.000000,0.000000}%
\pgfsetstrokecolor{currentstroke}%
\pgfsetdash{}{0pt}%
\pgfsys@defobject{currentmarker}{\pgfqpoint{0.000000in}{0.000000in}}{\pgfqpoint{0.000000in}{0.027778in}}{%
\pgfpathmoveto{\pgfqpoint{0.000000in}{0.000000in}}%
\pgfpathlineto{\pgfqpoint{0.000000in}{0.027778in}}%
\pgfusepath{stroke,fill}%
}%
\begin{pgfscope}%
\pgfsys@transformshift{3.483193in}{0.300000in}%
\pgfsys@useobject{currentmarker}{}%
\end{pgfscope}%
\end{pgfscope}%
\begin{pgfscope}%
\pgfsetbuttcap%
\pgfsetroundjoin%
\definecolor{currentfill}{rgb}{0.000000,0.000000,0.000000}%
\pgfsetfillcolor{currentfill}%
\pgfsetlinewidth{0.501875pt}%
\definecolor{currentstroke}{rgb}{0.000000,0.000000,0.000000}%
\pgfsetstrokecolor{currentstroke}%
\pgfsetdash{}{0pt}%
\pgfsys@defobject{currentmarker}{\pgfqpoint{0.000000in}{-0.027778in}}{\pgfqpoint{0.000000in}{0.000000in}}{%
\pgfpathmoveto{\pgfqpoint{0.000000in}{0.000000in}}%
\pgfpathlineto{\pgfqpoint{0.000000in}{-0.027778in}}%
\pgfusepath{stroke,fill}%
}%
\begin{pgfscope}%
\pgfsys@transformshift{3.483193in}{2.700000in}%
\pgfsys@useobject{currentmarker}{}%
\end{pgfscope}%
\end{pgfscope}%
\begin{pgfscope}%
\pgfsetbuttcap%
\pgfsetroundjoin%
\definecolor{currentfill}{rgb}{0.000000,0.000000,0.000000}%
\pgfsetfillcolor{currentfill}%
\pgfsetlinewidth{0.501875pt}%
\definecolor{currentstroke}{rgb}{0.000000,0.000000,0.000000}%
\pgfsetstrokecolor{currentstroke}%
\pgfsetdash{}{0pt}%
\pgfsys@defobject{currentmarker}{\pgfqpoint{0.000000in}{0.000000in}}{\pgfqpoint{0.000000in}{0.027778in}}{%
\pgfpathmoveto{\pgfqpoint{0.000000in}{0.000000in}}%
\pgfpathlineto{\pgfqpoint{0.000000in}{0.027778in}}%
\pgfusepath{stroke,fill}%
}%
\begin{pgfscope}%
\pgfsys@transformshift{3.633404in}{0.300000in}%
\pgfsys@useobject{currentmarker}{}%
\end{pgfscope}%
\end{pgfscope}%
\begin{pgfscope}%
\pgfsetbuttcap%
\pgfsetroundjoin%
\definecolor{currentfill}{rgb}{0.000000,0.000000,0.000000}%
\pgfsetfillcolor{currentfill}%
\pgfsetlinewidth{0.501875pt}%
\definecolor{currentstroke}{rgb}{0.000000,0.000000,0.000000}%
\pgfsetstrokecolor{currentstroke}%
\pgfsetdash{}{0pt}%
\pgfsys@defobject{currentmarker}{\pgfqpoint{0.000000in}{-0.027778in}}{\pgfqpoint{0.000000in}{0.000000in}}{%
\pgfpathmoveto{\pgfqpoint{0.000000in}{0.000000in}}%
\pgfpathlineto{\pgfqpoint{0.000000in}{-0.027778in}}%
\pgfusepath{stroke,fill}%
}%
\begin{pgfscope}%
\pgfsys@transformshift{3.633404in}{2.700000in}%
\pgfsys@useobject{currentmarker}{}%
\end{pgfscope}%
\end{pgfscope}%
\begin{pgfscope}%
\pgfsetbuttcap%
\pgfsetroundjoin%
\definecolor{currentfill}{rgb}{0.000000,0.000000,0.000000}%
\pgfsetfillcolor{currentfill}%
\pgfsetlinewidth{0.501875pt}%
\definecolor{currentstroke}{rgb}{0.000000,0.000000,0.000000}%
\pgfsetstrokecolor{currentstroke}%
\pgfsetdash{}{0pt}%
\pgfsys@defobject{currentmarker}{\pgfqpoint{0.000000in}{0.000000in}}{\pgfqpoint{0.000000in}{0.027778in}}{%
\pgfpathmoveto{\pgfqpoint{0.000000in}{0.000000in}}%
\pgfpathlineto{\pgfqpoint{0.000000in}{0.027778in}}%
\pgfusepath{stroke,fill}%
}%
\begin{pgfscope}%
\pgfsys@transformshift{3.756134in}{0.300000in}%
\pgfsys@useobject{currentmarker}{}%
\end{pgfscope}%
\end{pgfscope}%
\begin{pgfscope}%
\pgfsetbuttcap%
\pgfsetroundjoin%
\definecolor{currentfill}{rgb}{0.000000,0.000000,0.000000}%
\pgfsetfillcolor{currentfill}%
\pgfsetlinewidth{0.501875pt}%
\definecolor{currentstroke}{rgb}{0.000000,0.000000,0.000000}%
\pgfsetstrokecolor{currentstroke}%
\pgfsetdash{}{0pt}%
\pgfsys@defobject{currentmarker}{\pgfqpoint{0.000000in}{-0.027778in}}{\pgfqpoint{0.000000in}{0.000000in}}{%
\pgfpathmoveto{\pgfqpoint{0.000000in}{0.000000in}}%
\pgfpathlineto{\pgfqpoint{0.000000in}{-0.027778in}}%
\pgfusepath{stroke,fill}%
}%
\begin{pgfscope}%
\pgfsys@transformshift{3.756134in}{2.700000in}%
\pgfsys@useobject{currentmarker}{}%
\end{pgfscope}%
\end{pgfscope}%
\begin{pgfscope}%
\pgfsetbuttcap%
\pgfsetroundjoin%
\definecolor{currentfill}{rgb}{0.000000,0.000000,0.000000}%
\pgfsetfillcolor{currentfill}%
\pgfsetlinewidth{0.501875pt}%
\definecolor{currentstroke}{rgb}{0.000000,0.000000,0.000000}%
\pgfsetstrokecolor{currentstroke}%
\pgfsetdash{}{0pt}%
\pgfsys@defobject{currentmarker}{\pgfqpoint{0.000000in}{0.000000in}}{\pgfqpoint{0.000000in}{0.027778in}}{%
\pgfpathmoveto{\pgfqpoint{0.000000in}{0.000000in}}%
\pgfpathlineto{\pgfqpoint{0.000000in}{0.027778in}}%
\pgfusepath{stroke,fill}%
}%
\begin{pgfscope}%
\pgfsys@transformshift{3.859902in}{0.300000in}%
\pgfsys@useobject{currentmarker}{}%
\end{pgfscope}%
\end{pgfscope}%
\begin{pgfscope}%
\pgfsetbuttcap%
\pgfsetroundjoin%
\definecolor{currentfill}{rgb}{0.000000,0.000000,0.000000}%
\pgfsetfillcolor{currentfill}%
\pgfsetlinewidth{0.501875pt}%
\definecolor{currentstroke}{rgb}{0.000000,0.000000,0.000000}%
\pgfsetstrokecolor{currentstroke}%
\pgfsetdash{}{0pt}%
\pgfsys@defobject{currentmarker}{\pgfqpoint{0.000000in}{-0.027778in}}{\pgfqpoint{0.000000in}{0.000000in}}{%
\pgfpathmoveto{\pgfqpoint{0.000000in}{0.000000in}}%
\pgfpathlineto{\pgfqpoint{0.000000in}{-0.027778in}}%
\pgfusepath{stroke,fill}%
}%
\begin{pgfscope}%
\pgfsys@transformshift{3.859902in}{2.700000in}%
\pgfsys@useobject{currentmarker}{}%
\end{pgfscope}%
\end{pgfscope}%
\begin{pgfscope}%
\pgfsetbuttcap%
\pgfsetroundjoin%
\definecolor{currentfill}{rgb}{0.000000,0.000000,0.000000}%
\pgfsetfillcolor{currentfill}%
\pgfsetlinewidth{0.501875pt}%
\definecolor{currentstroke}{rgb}{0.000000,0.000000,0.000000}%
\pgfsetstrokecolor{currentstroke}%
\pgfsetdash{}{0pt}%
\pgfsys@defobject{currentmarker}{\pgfqpoint{0.000000in}{0.000000in}}{\pgfqpoint{0.000000in}{0.027778in}}{%
\pgfpathmoveto{\pgfqpoint{0.000000in}{0.000000in}}%
\pgfpathlineto{\pgfqpoint{0.000000in}{0.027778in}}%
\pgfusepath{stroke,fill}%
}%
\begin{pgfscope}%
\pgfsys@transformshift{3.949789in}{0.300000in}%
\pgfsys@useobject{currentmarker}{}%
\end{pgfscope}%
\end{pgfscope}%
\begin{pgfscope}%
\pgfsetbuttcap%
\pgfsetroundjoin%
\definecolor{currentfill}{rgb}{0.000000,0.000000,0.000000}%
\pgfsetfillcolor{currentfill}%
\pgfsetlinewidth{0.501875pt}%
\definecolor{currentstroke}{rgb}{0.000000,0.000000,0.000000}%
\pgfsetstrokecolor{currentstroke}%
\pgfsetdash{}{0pt}%
\pgfsys@defobject{currentmarker}{\pgfqpoint{0.000000in}{-0.027778in}}{\pgfqpoint{0.000000in}{0.000000in}}{%
\pgfpathmoveto{\pgfqpoint{0.000000in}{0.000000in}}%
\pgfpathlineto{\pgfqpoint{0.000000in}{-0.027778in}}%
\pgfusepath{stroke,fill}%
}%
\begin{pgfscope}%
\pgfsys@transformshift{3.949789in}{2.700000in}%
\pgfsys@useobject{currentmarker}{}%
\end{pgfscope}%
\end{pgfscope}%
\begin{pgfscope}%
\pgfsetbuttcap%
\pgfsetroundjoin%
\definecolor{currentfill}{rgb}{0.000000,0.000000,0.000000}%
\pgfsetfillcolor{currentfill}%
\pgfsetlinewidth{0.501875pt}%
\definecolor{currentstroke}{rgb}{0.000000,0.000000,0.000000}%
\pgfsetstrokecolor{currentstroke}%
\pgfsetdash{}{0pt}%
\pgfsys@defobject{currentmarker}{\pgfqpoint{0.000000in}{0.000000in}}{\pgfqpoint{0.000000in}{0.027778in}}{%
\pgfpathmoveto{\pgfqpoint{0.000000in}{0.000000in}}%
\pgfpathlineto{\pgfqpoint{0.000000in}{0.027778in}}%
\pgfusepath{stroke,fill}%
}%
\begin{pgfscope}%
\pgfsys@transformshift{4.029076in}{0.300000in}%
\pgfsys@useobject{currentmarker}{}%
\end{pgfscope}%
\end{pgfscope}%
\begin{pgfscope}%
\pgfsetbuttcap%
\pgfsetroundjoin%
\definecolor{currentfill}{rgb}{0.000000,0.000000,0.000000}%
\pgfsetfillcolor{currentfill}%
\pgfsetlinewidth{0.501875pt}%
\definecolor{currentstroke}{rgb}{0.000000,0.000000,0.000000}%
\pgfsetstrokecolor{currentstroke}%
\pgfsetdash{}{0pt}%
\pgfsys@defobject{currentmarker}{\pgfqpoint{0.000000in}{-0.027778in}}{\pgfqpoint{0.000000in}{0.000000in}}{%
\pgfpathmoveto{\pgfqpoint{0.000000in}{0.000000in}}%
\pgfpathlineto{\pgfqpoint{0.000000in}{-0.027778in}}%
\pgfusepath{stroke,fill}%
}%
\begin{pgfscope}%
\pgfsys@transformshift{4.029076in}{2.700000in}%
\pgfsys@useobject{currentmarker}{}%
\end{pgfscope}%
\end{pgfscope}%
\begin{pgfscope}%
\pgfsetbuttcap%
\pgfsetroundjoin%
\definecolor{currentfill}{rgb}{0.000000,0.000000,0.000000}%
\pgfsetfillcolor{currentfill}%
\pgfsetlinewidth{0.501875pt}%
\definecolor{currentstroke}{rgb}{0.000000,0.000000,0.000000}%
\pgfsetstrokecolor{currentstroke}%
\pgfsetdash{}{0pt}%
\pgfsys@defobject{currentmarker}{\pgfqpoint{0.000000in}{0.000000in}}{\pgfqpoint{0.000000in}{0.027778in}}{%
\pgfpathmoveto{\pgfqpoint{0.000000in}{0.000000in}}%
\pgfpathlineto{\pgfqpoint{0.000000in}{0.027778in}}%
\pgfusepath{stroke,fill}%
}%
\begin{pgfscope}%
\pgfsys@transformshift{4.566596in}{0.300000in}%
\pgfsys@useobject{currentmarker}{}%
\end{pgfscope}%
\end{pgfscope}%
\begin{pgfscope}%
\pgfsetbuttcap%
\pgfsetroundjoin%
\definecolor{currentfill}{rgb}{0.000000,0.000000,0.000000}%
\pgfsetfillcolor{currentfill}%
\pgfsetlinewidth{0.501875pt}%
\definecolor{currentstroke}{rgb}{0.000000,0.000000,0.000000}%
\pgfsetstrokecolor{currentstroke}%
\pgfsetdash{}{0pt}%
\pgfsys@defobject{currentmarker}{\pgfqpoint{0.000000in}{-0.027778in}}{\pgfqpoint{0.000000in}{0.000000in}}{%
\pgfpathmoveto{\pgfqpoint{0.000000in}{0.000000in}}%
\pgfpathlineto{\pgfqpoint{0.000000in}{-0.027778in}}%
\pgfusepath{stroke,fill}%
}%
\begin{pgfscope}%
\pgfsys@transformshift{4.566596in}{2.700000in}%
\pgfsys@useobject{currentmarker}{}%
\end{pgfscope}%
\end{pgfscope}%
\begin{pgfscope}%
\pgfsetbuttcap%
\pgfsetroundjoin%
\definecolor{currentfill}{rgb}{0.000000,0.000000,0.000000}%
\pgfsetfillcolor{currentfill}%
\pgfsetlinewidth{0.501875pt}%
\definecolor{currentstroke}{rgb}{0.000000,0.000000,0.000000}%
\pgfsetstrokecolor{currentstroke}%
\pgfsetdash{}{0pt}%
\pgfsys@defobject{currentmarker}{\pgfqpoint{0.000000in}{0.000000in}}{\pgfqpoint{0.000000in}{0.027778in}}{%
\pgfpathmoveto{\pgfqpoint{0.000000in}{0.000000in}}%
\pgfpathlineto{\pgfqpoint{0.000000in}{0.027778in}}%
\pgfusepath{stroke,fill}%
}%
\begin{pgfscope}%
\pgfsys@transformshift{4.839538in}{0.300000in}%
\pgfsys@useobject{currentmarker}{}%
\end{pgfscope}%
\end{pgfscope}%
\begin{pgfscope}%
\pgfsetbuttcap%
\pgfsetroundjoin%
\definecolor{currentfill}{rgb}{0.000000,0.000000,0.000000}%
\pgfsetfillcolor{currentfill}%
\pgfsetlinewidth{0.501875pt}%
\definecolor{currentstroke}{rgb}{0.000000,0.000000,0.000000}%
\pgfsetstrokecolor{currentstroke}%
\pgfsetdash{}{0pt}%
\pgfsys@defobject{currentmarker}{\pgfqpoint{0.000000in}{-0.027778in}}{\pgfqpoint{0.000000in}{0.000000in}}{%
\pgfpathmoveto{\pgfqpoint{0.000000in}{0.000000in}}%
\pgfpathlineto{\pgfqpoint{0.000000in}{-0.027778in}}%
\pgfusepath{stroke,fill}%
}%
\begin{pgfscope}%
\pgfsys@transformshift{4.839538in}{2.700000in}%
\pgfsys@useobject{currentmarker}{}%
\end{pgfscope}%
\end{pgfscope}%
\begin{pgfscope}%
\pgfsetbuttcap%
\pgfsetroundjoin%
\definecolor{currentfill}{rgb}{0.000000,0.000000,0.000000}%
\pgfsetfillcolor{currentfill}%
\pgfsetlinewidth{0.501875pt}%
\definecolor{currentstroke}{rgb}{0.000000,0.000000,0.000000}%
\pgfsetstrokecolor{currentstroke}%
\pgfsetdash{}{0pt}%
\pgfsys@defobject{currentmarker}{\pgfqpoint{0.000000in}{0.000000in}}{\pgfqpoint{0.000000in}{0.027778in}}{%
\pgfpathmoveto{\pgfqpoint{0.000000in}{0.000000in}}%
\pgfpathlineto{\pgfqpoint{0.000000in}{0.027778in}}%
\pgfusepath{stroke,fill}%
}%
\begin{pgfscope}%
\pgfsys@transformshift{5.033193in}{0.300000in}%
\pgfsys@useobject{currentmarker}{}%
\end{pgfscope}%
\end{pgfscope}%
\begin{pgfscope}%
\pgfsetbuttcap%
\pgfsetroundjoin%
\definecolor{currentfill}{rgb}{0.000000,0.000000,0.000000}%
\pgfsetfillcolor{currentfill}%
\pgfsetlinewidth{0.501875pt}%
\definecolor{currentstroke}{rgb}{0.000000,0.000000,0.000000}%
\pgfsetstrokecolor{currentstroke}%
\pgfsetdash{}{0pt}%
\pgfsys@defobject{currentmarker}{\pgfqpoint{0.000000in}{-0.027778in}}{\pgfqpoint{0.000000in}{0.000000in}}{%
\pgfpathmoveto{\pgfqpoint{0.000000in}{0.000000in}}%
\pgfpathlineto{\pgfqpoint{0.000000in}{-0.027778in}}%
\pgfusepath{stroke,fill}%
}%
\begin{pgfscope}%
\pgfsys@transformshift{5.033193in}{2.700000in}%
\pgfsys@useobject{currentmarker}{}%
\end{pgfscope}%
\end{pgfscope}%
\begin{pgfscope}%
\pgfsetbuttcap%
\pgfsetroundjoin%
\definecolor{currentfill}{rgb}{0.000000,0.000000,0.000000}%
\pgfsetfillcolor{currentfill}%
\pgfsetlinewidth{0.501875pt}%
\definecolor{currentstroke}{rgb}{0.000000,0.000000,0.000000}%
\pgfsetstrokecolor{currentstroke}%
\pgfsetdash{}{0pt}%
\pgfsys@defobject{currentmarker}{\pgfqpoint{0.000000in}{0.000000in}}{\pgfqpoint{0.000000in}{0.027778in}}{%
\pgfpathmoveto{\pgfqpoint{0.000000in}{0.000000in}}%
\pgfpathlineto{\pgfqpoint{0.000000in}{0.027778in}}%
\pgfusepath{stroke,fill}%
}%
\begin{pgfscope}%
\pgfsys@transformshift{5.183404in}{0.300000in}%
\pgfsys@useobject{currentmarker}{}%
\end{pgfscope}%
\end{pgfscope}%
\begin{pgfscope}%
\pgfsetbuttcap%
\pgfsetroundjoin%
\definecolor{currentfill}{rgb}{0.000000,0.000000,0.000000}%
\pgfsetfillcolor{currentfill}%
\pgfsetlinewidth{0.501875pt}%
\definecolor{currentstroke}{rgb}{0.000000,0.000000,0.000000}%
\pgfsetstrokecolor{currentstroke}%
\pgfsetdash{}{0pt}%
\pgfsys@defobject{currentmarker}{\pgfqpoint{0.000000in}{-0.027778in}}{\pgfqpoint{0.000000in}{0.000000in}}{%
\pgfpathmoveto{\pgfqpoint{0.000000in}{0.000000in}}%
\pgfpathlineto{\pgfqpoint{0.000000in}{-0.027778in}}%
\pgfusepath{stroke,fill}%
}%
\begin{pgfscope}%
\pgfsys@transformshift{5.183404in}{2.700000in}%
\pgfsys@useobject{currentmarker}{}%
\end{pgfscope}%
\end{pgfscope}%
\begin{pgfscope}%
\pgfsetbuttcap%
\pgfsetroundjoin%
\definecolor{currentfill}{rgb}{0.000000,0.000000,0.000000}%
\pgfsetfillcolor{currentfill}%
\pgfsetlinewidth{0.501875pt}%
\definecolor{currentstroke}{rgb}{0.000000,0.000000,0.000000}%
\pgfsetstrokecolor{currentstroke}%
\pgfsetdash{}{0pt}%
\pgfsys@defobject{currentmarker}{\pgfqpoint{0.000000in}{0.000000in}}{\pgfqpoint{0.000000in}{0.027778in}}{%
\pgfpathmoveto{\pgfqpoint{0.000000in}{0.000000in}}%
\pgfpathlineto{\pgfqpoint{0.000000in}{0.027778in}}%
\pgfusepath{stroke,fill}%
}%
\begin{pgfscope}%
\pgfsys@transformshift{5.306134in}{0.300000in}%
\pgfsys@useobject{currentmarker}{}%
\end{pgfscope}%
\end{pgfscope}%
\begin{pgfscope}%
\pgfsetbuttcap%
\pgfsetroundjoin%
\definecolor{currentfill}{rgb}{0.000000,0.000000,0.000000}%
\pgfsetfillcolor{currentfill}%
\pgfsetlinewidth{0.501875pt}%
\definecolor{currentstroke}{rgb}{0.000000,0.000000,0.000000}%
\pgfsetstrokecolor{currentstroke}%
\pgfsetdash{}{0pt}%
\pgfsys@defobject{currentmarker}{\pgfqpoint{0.000000in}{-0.027778in}}{\pgfqpoint{0.000000in}{0.000000in}}{%
\pgfpathmoveto{\pgfqpoint{0.000000in}{0.000000in}}%
\pgfpathlineto{\pgfqpoint{0.000000in}{-0.027778in}}%
\pgfusepath{stroke,fill}%
}%
\begin{pgfscope}%
\pgfsys@transformshift{5.306134in}{2.700000in}%
\pgfsys@useobject{currentmarker}{}%
\end{pgfscope}%
\end{pgfscope}%
\begin{pgfscope}%
\pgfsetbuttcap%
\pgfsetroundjoin%
\definecolor{currentfill}{rgb}{0.000000,0.000000,0.000000}%
\pgfsetfillcolor{currentfill}%
\pgfsetlinewidth{0.501875pt}%
\definecolor{currentstroke}{rgb}{0.000000,0.000000,0.000000}%
\pgfsetstrokecolor{currentstroke}%
\pgfsetdash{}{0pt}%
\pgfsys@defobject{currentmarker}{\pgfqpoint{0.000000in}{0.000000in}}{\pgfqpoint{0.000000in}{0.027778in}}{%
\pgfpathmoveto{\pgfqpoint{0.000000in}{0.000000in}}%
\pgfpathlineto{\pgfqpoint{0.000000in}{0.027778in}}%
\pgfusepath{stroke,fill}%
}%
\begin{pgfscope}%
\pgfsys@transformshift{5.409902in}{0.300000in}%
\pgfsys@useobject{currentmarker}{}%
\end{pgfscope}%
\end{pgfscope}%
\begin{pgfscope}%
\pgfsetbuttcap%
\pgfsetroundjoin%
\definecolor{currentfill}{rgb}{0.000000,0.000000,0.000000}%
\pgfsetfillcolor{currentfill}%
\pgfsetlinewidth{0.501875pt}%
\definecolor{currentstroke}{rgb}{0.000000,0.000000,0.000000}%
\pgfsetstrokecolor{currentstroke}%
\pgfsetdash{}{0pt}%
\pgfsys@defobject{currentmarker}{\pgfqpoint{0.000000in}{-0.027778in}}{\pgfqpoint{0.000000in}{0.000000in}}{%
\pgfpathmoveto{\pgfqpoint{0.000000in}{0.000000in}}%
\pgfpathlineto{\pgfqpoint{0.000000in}{-0.027778in}}%
\pgfusepath{stroke,fill}%
}%
\begin{pgfscope}%
\pgfsys@transformshift{5.409902in}{2.700000in}%
\pgfsys@useobject{currentmarker}{}%
\end{pgfscope}%
\end{pgfscope}%
\begin{pgfscope}%
\pgfsetbuttcap%
\pgfsetroundjoin%
\definecolor{currentfill}{rgb}{0.000000,0.000000,0.000000}%
\pgfsetfillcolor{currentfill}%
\pgfsetlinewidth{0.501875pt}%
\definecolor{currentstroke}{rgb}{0.000000,0.000000,0.000000}%
\pgfsetstrokecolor{currentstroke}%
\pgfsetdash{}{0pt}%
\pgfsys@defobject{currentmarker}{\pgfqpoint{0.000000in}{0.000000in}}{\pgfqpoint{0.000000in}{0.027778in}}{%
\pgfpathmoveto{\pgfqpoint{0.000000in}{0.000000in}}%
\pgfpathlineto{\pgfqpoint{0.000000in}{0.027778in}}%
\pgfusepath{stroke,fill}%
}%
\begin{pgfscope}%
\pgfsys@transformshift{5.499789in}{0.300000in}%
\pgfsys@useobject{currentmarker}{}%
\end{pgfscope}%
\end{pgfscope}%
\begin{pgfscope}%
\pgfsetbuttcap%
\pgfsetroundjoin%
\definecolor{currentfill}{rgb}{0.000000,0.000000,0.000000}%
\pgfsetfillcolor{currentfill}%
\pgfsetlinewidth{0.501875pt}%
\definecolor{currentstroke}{rgb}{0.000000,0.000000,0.000000}%
\pgfsetstrokecolor{currentstroke}%
\pgfsetdash{}{0pt}%
\pgfsys@defobject{currentmarker}{\pgfqpoint{0.000000in}{-0.027778in}}{\pgfqpoint{0.000000in}{0.000000in}}{%
\pgfpathmoveto{\pgfqpoint{0.000000in}{0.000000in}}%
\pgfpathlineto{\pgfqpoint{0.000000in}{-0.027778in}}%
\pgfusepath{stroke,fill}%
}%
\begin{pgfscope}%
\pgfsys@transformshift{5.499789in}{2.700000in}%
\pgfsys@useobject{currentmarker}{}%
\end{pgfscope}%
\end{pgfscope}%
\begin{pgfscope}%
\pgfsetbuttcap%
\pgfsetroundjoin%
\definecolor{currentfill}{rgb}{0.000000,0.000000,0.000000}%
\pgfsetfillcolor{currentfill}%
\pgfsetlinewidth{0.501875pt}%
\definecolor{currentstroke}{rgb}{0.000000,0.000000,0.000000}%
\pgfsetstrokecolor{currentstroke}%
\pgfsetdash{}{0pt}%
\pgfsys@defobject{currentmarker}{\pgfqpoint{0.000000in}{0.000000in}}{\pgfqpoint{0.000000in}{0.027778in}}{%
\pgfpathmoveto{\pgfqpoint{0.000000in}{0.000000in}}%
\pgfpathlineto{\pgfqpoint{0.000000in}{0.027778in}}%
\pgfusepath{stroke,fill}%
}%
\begin{pgfscope}%
\pgfsys@transformshift{5.579076in}{0.300000in}%
\pgfsys@useobject{currentmarker}{}%
\end{pgfscope}%
\end{pgfscope}%
\begin{pgfscope}%
\pgfsetbuttcap%
\pgfsetroundjoin%
\definecolor{currentfill}{rgb}{0.000000,0.000000,0.000000}%
\pgfsetfillcolor{currentfill}%
\pgfsetlinewidth{0.501875pt}%
\definecolor{currentstroke}{rgb}{0.000000,0.000000,0.000000}%
\pgfsetstrokecolor{currentstroke}%
\pgfsetdash{}{0pt}%
\pgfsys@defobject{currentmarker}{\pgfqpoint{0.000000in}{-0.027778in}}{\pgfqpoint{0.000000in}{0.000000in}}{%
\pgfpathmoveto{\pgfqpoint{0.000000in}{0.000000in}}%
\pgfpathlineto{\pgfqpoint{0.000000in}{-0.027778in}}%
\pgfusepath{stroke,fill}%
}%
\begin{pgfscope}%
\pgfsys@transformshift{5.579076in}{2.700000in}%
\pgfsys@useobject{currentmarker}{}%
\end{pgfscope}%
\end{pgfscope}%
\begin{pgfscope}%
\pgfsetbuttcap%
\pgfsetroundjoin%
\definecolor{currentfill}{rgb}{0.000000,0.000000,0.000000}%
\pgfsetfillcolor{currentfill}%
\pgfsetlinewidth{0.501875pt}%
\definecolor{currentstroke}{rgb}{0.000000,0.000000,0.000000}%
\pgfsetstrokecolor{currentstroke}%
\pgfsetdash{}{0pt}%
\pgfsys@defobject{currentmarker}{\pgfqpoint{0.000000in}{0.000000in}}{\pgfqpoint{0.000000in}{0.027778in}}{%
\pgfpathmoveto{\pgfqpoint{0.000000in}{0.000000in}}%
\pgfpathlineto{\pgfqpoint{0.000000in}{0.027778in}}%
\pgfusepath{stroke,fill}%
}%
\begin{pgfscope}%
\pgfsys@transformshift{6.116596in}{0.300000in}%
\pgfsys@useobject{currentmarker}{}%
\end{pgfscope}%
\end{pgfscope}%
\begin{pgfscope}%
\pgfsetbuttcap%
\pgfsetroundjoin%
\definecolor{currentfill}{rgb}{0.000000,0.000000,0.000000}%
\pgfsetfillcolor{currentfill}%
\pgfsetlinewidth{0.501875pt}%
\definecolor{currentstroke}{rgb}{0.000000,0.000000,0.000000}%
\pgfsetstrokecolor{currentstroke}%
\pgfsetdash{}{0pt}%
\pgfsys@defobject{currentmarker}{\pgfqpoint{0.000000in}{-0.027778in}}{\pgfqpoint{0.000000in}{0.000000in}}{%
\pgfpathmoveto{\pgfqpoint{0.000000in}{0.000000in}}%
\pgfpathlineto{\pgfqpoint{0.000000in}{-0.027778in}}%
\pgfusepath{stroke,fill}%
}%
\begin{pgfscope}%
\pgfsys@transformshift{6.116596in}{2.700000in}%
\pgfsys@useobject{currentmarker}{}%
\end{pgfscope}%
\end{pgfscope}%
\begin{pgfscope}%
\pgfsetbuttcap%
\pgfsetroundjoin%
\definecolor{currentfill}{rgb}{0.000000,0.000000,0.000000}%
\pgfsetfillcolor{currentfill}%
\pgfsetlinewidth{0.501875pt}%
\definecolor{currentstroke}{rgb}{0.000000,0.000000,0.000000}%
\pgfsetstrokecolor{currentstroke}%
\pgfsetdash{}{0pt}%
\pgfsys@defobject{currentmarker}{\pgfqpoint{0.000000in}{0.000000in}}{\pgfqpoint{0.000000in}{0.027778in}}{%
\pgfpathmoveto{\pgfqpoint{0.000000in}{0.000000in}}%
\pgfpathlineto{\pgfqpoint{0.000000in}{0.027778in}}%
\pgfusepath{stroke,fill}%
}%
\begin{pgfscope}%
\pgfsys@transformshift{6.389538in}{0.300000in}%
\pgfsys@useobject{currentmarker}{}%
\end{pgfscope}%
\end{pgfscope}%
\begin{pgfscope}%
\pgfsetbuttcap%
\pgfsetroundjoin%
\definecolor{currentfill}{rgb}{0.000000,0.000000,0.000000}%
\pgfsetfillcolor{currentfill}%
\pgfsetlinewidth{0.501875pt}%
\definecolor{currentstroke}{rgb}{0.000000,0.000000,0.000000}%
\pgfsetstrokecolor{currentstroke}%
\pgfsetdash{}{0pt}%
\pgfsys@defobject{currentmarker}{\pgfqpoint{0.000000in}{-0.027778in}}{\pgfqpoint{0.000000in}{0.000000in}}{%
\pgfpathmoveto{\pgfqpoint{0.000000in}{0.000000in}}%
\pgfpathlineto{\pgfqpoint{0.000000in}{-0.027778in}}%
\pgfusepath{stroke,fill}%
}%
\begin{pgfscope}%
\pgfsys@transformshift{6.389538in}{2.700000in}%
\pgfsys@useobject{currentmarker}{}%
\end{pgfscope}%
\end{pgfscope}%
\begin{pgfscope}%
\pgfsetbuttcap%
\pgfsetroundjoin%
\definecolor{currentfill}{rgb}{0.000000,0.000000,0.000000}%
\pgfsetfillcolor{currentfill}%
\pgfsetlinewidth{0.501875pt}%
\definecolor{currentstroke}{rgb}{0.000000,0.000000,0.000000}%
\pgfsetstrokecolor{currentstroke}%
\pgfsetdash{}{0pt}%
\pgfsys@defobject{currentmarker}{\pgfqpoint{0.000000in}{0.000000in}}{\pgfqpoint{0.000000in}{0.027778in}}{%
\pgfpathmoveto{\pgfqpoint{0.000000in}{0.000000in}}%
\pgfpathlineto{\pgfqpoint{0.000000in}{0.027778in}}%
\pgfusepath{stroke,fill}%
}%
\begin{pgfscope}%
\pgfsys@transformshift{6.583193in}{0.300000in}%
\pgfsys@useobject{currentmarker}{}%
\end{pgfscope}%
\end{pgfscope}%
\begin{pgfscope}%
\pgfsetbuttcap%
\pgfsetroundjoin%
\definecolor{currentfill}{rgb}{0.000000,0.000000,0.000000}%
\pgfsetfillcolor{currentfill}%
\pgfsetlinewidth{0.501875pt}%
\definecolor{currentstroke}{rgb}{0.000000,0.000000,0.000000}%
\pgfsetstrokecolor{currentstroke}%
\pgfsetdash{}{0pt}%
\pgfsys@defobject{currentmarker}{\pgfqpoint{0.000000in}{-0.027778in}}{\pgfqpoint{0.000000in}{0.000000in}}{%
\pgfpathmoveto{\pgfqpoint{0.000000in}{0.000000in}}%
\pgfpathlineto{\pgfqpoint{0.000000in}{-0.027778in}}%
\pgfusepath{stroke,fill}%
}%
\begin{pgfscope}%
\pgfsys@transformshift{6.583193in}{2.700000in}%
\pgfsys@useobject{currentmarker}{}%
\end{pgfscope}%
\end{pgfscope}%
\begin{pgfscope}%
\pgfsetbuttcap%
\pgfsetroundjoin%
\definecolor{currentfill}{rgb}{0.000000,0.000000,0.000000}%
\pgfsetfillcolor{currentfill}%
\pgfsetlinewidth{0.501875pt}%
\definecolor{currentstroke}{rgb}{0.000000,0.000000,0.000000}%
\pgfsetstrokecolor{currentstroke}%
\pgfsetdash{}{0pt}%
\pgfsys@defobject{currentmarker}{\pgfqpoint{0.000000in}{0.000000in}}{\pgfqpoint{0.000000in}{0.027778in}}{%
\pgfpathmoveto{\pgfqpoint{0.000000in}{0.000000in}}%
\pgfpathlineto{\pgfqpoint{0.000000in}{0.027778in}}%
\pgfusepath{stroke,fill}%
}%
\begin{pgfscope}%
\pgfsys@transformshift{6.733404in}{0.300000in}%
\pgfsys@useobject{currentmarker}{}%
\end{pgfscope}%
\end{pgfscope}%
\begin{pgfscope}%
\pgfsetbuttcap%
\pgfsetroundjoin%
\definecolor{currentfill}{rgb}{0.000000,0.000000,0.000000}%
\pgfsetfillcolor{currentfill}%
\pgfsetlinewidth{0.501875pt}%
\definecolor{currentstroke}{rgb}{0.000000,0.000000,0.000000}%
\pgfsetstrokecolor{currentstroke}%
\pgfsetdash{}{0pt}%
\pgfsys@defobject{currentmarker}{\pgfqpoint{0.000000in}{-0.027778in}}{\pgfqpoint{0.000000in}{0.000000in}}{%
\pgfpathmoveto{\pgfqpoint{0.000000in}{0.000000in}}%
\pgfpathlineto{\pgfqpoint{0.000000in}{-0.027778in}}%
\pgfusepath{stroke,fill}%
}%
\begin{pgfscope}%
\pgfsys@transformshift{6.733404in}{2.700000in}%
\pgfsys@useobject{currentmarker}{}%
\end{pgfscope}%
\end{pgfscope}%
\begin{pgfscope}%
\pgfsetbuttcap%
\pgfsetroundjoin%
\definecolor{currentfill}{rgb}{0.000000,0.000000,0.000000}%
\pgfsetfillcolor{currentfill}%
\pgfsetlinewidth{0.501875pt}%
\definecolor{currentstroke}{rgb}{0.000000,0.000000,0.000000}%
\pgfsetstrokecolor{currentstroke}%
\pgfsetdash{}{0pt}%
\pgfsys@defobject{currentmarker}{\pgfqpoint{0.000000in}{0.000000in}}{\pgfqpoint{0.000000in}{0.027778in}}{%
\pgfpathmoveto{\pgfqpoint{0.000000in}{0.000000in}}%
\pgfpathlineto{\pgfqpoint{0.000000in}{0.027778in}}%
\pgfusepath{stroke,fill}%
}%
\begin{pgfscope}%
\pgfsys@transformshift{6.856134in}{0.300000in}%
\pgfsys@useobject{currentmarker}{}%
\end{pgfscope}%
\end{pgfscope}%
\begin{pgfscope}%
\pgfsetbuttcap%
\pgfsetroundjoin%
\definecolor{currentfill}{rgb}{0.000000,0.000000,0.000000}%
\pgfsetfillcolor{currentfill}%
\pgfsetlinewidth{0.501875pt}%
\definecolor{currentstroke}{rgb}{0.000000,0.000000,0.000000}%
\pgfsetstrokecolor{currentstroke}%
\pgfsetdash{}{0pt}%
\pgfsys@defobject{currentmarker}{\pgfqpoint{0.000000in}{-0.027778in}}{\pgfqpoint{0.000000in}{0.000000in}}{%
\pgfpathmoveto{\pgfqpoint{0.000000in}{0.000000in}}%
\pgfpathlineto{\pgfqpoint{0.000000in}{-0.027778in}}%
\pgfusepath{stroke,fill}%
}%
\begin{pgfscope}%
\pgfsys@transformshift{6.856134in}{2.700000in}%
\pgfsys@useobject{currentmarker}{}%
\end{pgfscope}%
\end{pgfscope}%
\begin{pgfscope}%
\pgfsetbuttcap%
\pgfsetroundjoin%
\definecolor{currentfill}{rgb}{0.000000,0.000000,0.000000}%
\pgfsetfillcolor{currentfill}%
\pgfsetlinewidth{0.501875pt}%
\definecolor{currentstroke}{rgb}{0.000000,0.000000,0.000000}%
\pgfsetstrokecolor{currentstroke}%
\pgfsetdash{}{0pt}%
\pgfsys@defobject{currentmarker}{\pgfqpoint{0.000000in}{0.000000in}}{\pgfqpoint{0.000000in}{0.027778in}}{%
\pgfpathmoveto{\pgfqpoint{0.000000in}{0.000000in}}%
\pgfpathlineto{\pgfqpoint{0.000000in}{0.027778in}}%
\pgfusepath{stroke,fill}%
}%
\begin{pgfscope}%
\pgfsys@transformshift{6.959902in}{0.300000in}%
\pgfsys@useobject{currentmarker}{}%
\end{pgfscope}%
\end{pgfscope}%
\begin{pgfscope}%
\pgfsetbuttcap%
\pgfsetroundjoin%
\definecolor{currentfill}{rgb}{0.000000,0.000000,0.000000}%
\pgfsetfillcolor{currentfill}%
\pgfsetlinewidth{0.501875pt}%
\definecolor{currentstroke}{rgb}{0.000000,0.000000,0.000000}%
\pgfsetstrokecolor{currentstroke}%
\pgfsetdash{}{0pt}%
\pgfsys@defobject{currentmarker}{\pgfqpoint{0.000000in}{-0.027778in}}{\pgfqpoint{0.000000in}{0.000000in}}{%
\pgfpathmoveto{\pgfqpoint{0.000000in}{0.000000in}}%
\pgfpathlineto{\pgfqpoint{0.000000in}{-0.027778in}}%
\pgfusepath{stroke,fill}%
}%
\begin{pgfscope}%
\pgfsys@transformshift{6.959902in}{2.700000in}%
\pgfsys@useobject{currentmarker}{}%
\end{pgfscope}%
\end{pgfscope}%
\begin{pgfscope}%
\pgfsetbuttcap%
\pgfsetroundjoin%
\definecolor{currentfill}{rgb}{0.000000,0.000000,0.000000}%
\pgfsetfillcolor{currentfill}%
\pgfsetlinewidth{0.501875pt}%
\definecolor{currentstroke}{rgb}{0.000000,0.000000,0.000000}%
\pgfsetstrokecolor{currentstroke}%
\pgfsetdash{}{0pt}%
\pgfsys@defobject{currentmarker}{\pgfqpoint{0.000000in}{0.000000in}}{\pgfqpoint{0.000000in}{0.027778in}}{%
\pgfpathmoveto{\pgfqpoint{0.000000in}{0.000000in}}%
\pgfpathlineto{\pgfqpoint{0.000000in}{0.027778in}}%
\pgfusepath{stroke,fill}%
}%
\begin{pgfscope}%
\pgfsys@transformshift{7.049789in}{0.300000in}%
\pgfsys@useobject{currentmarker}{}%
\end{pgfscope}%
\end{pgfscope}%
\begin{pgfscope}%
\pgfsetbuttcap%
\pgfsetroundjoin%
\definecolor{currentfill}{rgb}{0.000000,0.000000,0.000000}%
\pgfsetfillcolor{currentfill}%
\pgfsetlinewidth{0.501875pt}%
\definecolor{currentstroke}{rgb}{0.000000,0.000000,0.000000}%
\pgfsetstrokecolor{currentstroke}%
\pgfsetdash{}{0pt}%
\pgfsys@defobject{currentmarker}{\pgfqpoint{0.000000in}{-0.027778in}}{\pgfqpoint{0.000000in}{0.000000in}}{%
\pgfpathmoveto{\pgfqpoint{0.000000in}{0.000000in}}%
\pgfpathlineto{\pgfqpoint{0.000000in}{-0.027778in}}%
\pgfusepath{stroke,fill}%
}%
\begin{pgfscope}%
\pgfsys@transformshift{7.049789in}{2.700000in}%
\pgfsys@useobject{currentmarker}{}%
\end{pgfscope}%
\end{pgfscope}%
\begin{pgfscope}%
\pgfsetbuttcap%
\pgfsetroundjoin%
\definecolor{currentfill}{rgb}{0.000000,0.000000,0.000000}%
\pgfsetfillcolor{currentfill}%
\pgfsetlinewidth{0.501875pt}%
\definecolor{currentstroke}{rgb}{0.000000,0.000000,0.000000}%
\pgfsetstrokecolor{currentstroke}%
\pgfsetdash{}{0pt}%
\pgfsys@defobject{currentmarker}{\pgfqpoint{0.000000in}{0.000000in}}{\pgfqpoint{0.000000in}{0.027778in}}{%
\pgfpathmoveto{\pgfqpoint{0.000000in}{0.000000in}}%
\pgfpathlineto{\pgfqpoint{0.000000in}{0.027778in}}%
\pgfusepath{stroke,fill}%
}%
\begin{pgfscope}%
\pgfsys@transformshift{7.129076in}{0.300000in}%
\pgfsys@useobject{currentmarker}{}%
\end{pgfscope}%
\end{pgfscope}%
\begin{pgfscope}%
\pgfsetbuttcap%
\pgfsetroundjoin%
\definecolor{currentfill}{rgb}{0.000000,0.000000,0.000000}%
\pgfsetfillcolor{currentfill}%
\pgfsetlinewidth{0.501875pt}%
\definecolor{currentstroke}{rgb}{0.000000,0.000000,0.000000}%
\pgfsetstrokecolor{currentstroke}%
\pgfsetdash{}{0pt}%
\pgfsys@defobject{currentmarker}{\pgfqpoint{0.000000in}{-0.027778in}}{\pgfqpoint{0.000000in}{0.000000in}}{%
\pgfpathmoveto{\pgfqpoint{0.000000in}{0.000000in}}%
\pgfpathlineto{\pgfqpoint{0.000000in}{-0.027778in}}%
\pgfusepath{stroke,fill}%
}%
\begin{pgfscope}%
\pgfsys@transformshift{7.129076in}{2.700000in}%
\pgfsys@useobject{currentmarker}{}%
\end{pgfscope}%
\end{pgfscope}%
\begin{pgfscope}%
\pgftext[left,bottom,x=3.723901in,y=-0.126716in,rotate=0.000000]{{\sffamily\fontsize{12.000000}{14.400000}\selectfont time [ps]}}
%
\end{pgfscope}%
\begin{pgfscope}%
\pgfpathrectangle{\pgfqpoint{1.000000in}{0.300000in}}{\pgfqpoint{6.200000in}{2.400000in}} %
\pgfusepath{clip}%
\pgfsetbuttcap%
\pgfsetroundjoin%
\pgfsetlinewidth{0.501875pt}%
\definecolor{currentstroke}{rgb}{0.000000,0.000000,0.000000}%
\pgfsetstrokecolor{currentstroke}%
\pgfsetdash{{1.000000pt}{3.000000pt}}{0.000000pt}%
\pgfpathmoveto{\pgfqpoint{1.000000in}{0.300000in}}%
\pgfpathlineto{\pgfqpoint{7.200000in}{0.300000in}}%
\pgfusepath{stroke}%
\end{pgfscope}%
\begin{pgfscope}%
\pgfsetbuttcap%
\pgfsetroundjoin%
\definecolor{currentfill}{rgb}{0.000000,0.000000,0.000000}%
\pgfsetfillcolor{currentfill}%
\pgfsetlinewidth{0.501875pt}%
\definecolor{currentstroke}{rgb}{0.000000,0.000000,0.000000}%
\pgfsetstrokecolor{currentstroke}%
\pgfsetdash{}{0pt}%
\pgfsys@defobject{currentmarker}{\pgfqpoint{0.000000in}{0.000000in}}{\pgfqpoint{0.055556in}{0.000000in}}{%
\pgfpathmoveto{\pgfqpoint{0.000000in}{0.000000in}}%
\pgfpathlineto{\pgfqpoint{0.055556in}{0.000000in}}%
\pgfusepath{stroke,fill}%
}%
\begin{pgfscope}%
\pgfsys@transformshift{1.000000in}{0.300000in}%
\pgfsys@useobject{currentmarker}{}%
\end{pgfscope}%
\end{pgfscope}%
\begin{pgfscope}%
\pgfsetbuttcap%
\pgfsetroundjoin%
\definecolor{currentfill}{rgb}{0.000000,0.000000,0.000000}%
\pgfsetfillcolor{currentfill}%
\pgfsetlinewidth{0.501875pt}%
\definecolor{currentstroke}{rgb}{0.000000,0.000000,0.000000}%
\pgfsetstrokecolor{currentstroke}%
\pgfsetdash{}{0pt}%
\pgfsys@defobject{currentmarker}{\pgfqpoint{-0.055556in}{0.000000in}}{\pgfqpoint{0.000000in}{0.000000in}}{%
\pgfpathmoveto{\pgfqpoint{0.000000in}{0.000000in}}%
\pgfpathlineto{\pgfqpoint{-0.055556in}{0.000000in}}%
\pgfusepath{stroke,fill}%
}%
\begin{pgfscope}%
\pgfsys@transformshift{7.200000in}{0.300000in}%
\pgfsys@useobject{currentmarker}{}%
\end{pgfscope}%
\end{pgfscope}%
\begin{pgfscope}%
\pgftext[left,bottom,x=0.623456in,y=0.229790in,rotate=0.000000]{{\sffamily\fontsize{12.000000}{14.400000}\selectfont \(\displaystyle {10^{-3}}\)}}
%
\end{pgfscope}%
\begin{pgfscope}%
\pgfpathrectangle{\pgfqpoint{1.000000in}{0.300000in}}{\pgfqpoint{6.200000in}{2.400000in}} %
\pgfusepath{clip}%
\pgfsetbuttcap%
\pgfsetroundjoin%
\pgfsetlinewidth{0.501875pt}%
\definecolor{currentstroke}{rgb}{0.000000,0.000000,0.000000}%
\pgfsetstrokecolor{currentstroke}%
\pgfsetdash{{1.000000pt}{3.000000pt}}{0.000000pt}%
\pgfpathmoveto{\pgfqpoint{1.000000in}{0.780000in}}%
\pgfpathlineto{\pgfqpoint{7.200000in}{0.780000in}}%
\pgfusepath{stroke}%
\end{pgfscope}%
\begin{pgfscope}%
\pgfsetbuttcap%
\pgfsetroundjoin%
\definecolor{currentfill}{rgb}{0.000000,0.000000,0.000000}%
\pgfsetfillcolor{currentfill}%
\pgfsetlinewidth{0.501875pt}%
\definecolor{currentstroke}{rgb}{0.000000,0.000000,0.000000}%
\pgfsetstrokecolor{currentstroke}%
\pgfsetdash{}{0pt}%
\pgfsys@defobject{currentmarker}{\pgfqpoint{0.000000in}{0.000000in}}{\pgfqpoint{0.055556in}{0.000000in}}{%
\pgfpathmoveto{\pgfqpoint{0.000000in}{0.000000in}}%
\pgfpathlineto{\pgfqpoint{0.055556in}{0.000000in}}%
\pgfusepath{stroke,fill}%
}%
\begin{pgfscope}%
\pgfsys@transformshift{1.000000in}{0.780000in}%
\pgfsys@useobject{currentmarker}{}%
\end{pgfscope}%
\end{pgfscope}%
\begin{pgfscope}%
\pgfsetbuttcap%
\pgfsetroundjoin%
\definecolor{currentfill}{rgb}{0.000000,0.000000,0.000000}%
\pgfsetfillcolor{currentfill}%
\pgfsetlinewidth{0.501875pt}%
\definecolor{currentstroke}{rgb}{0.000000,0.000000,0.000000}%
\pgfsetstrokecolor{currentstroke}%
\pgfsetdash{}{0pt}%
\pgfsys@defobject{currentmarker}{\pgfqpoint{-0.055556in}{0.000000in}}{\pgfqpoint{0.000000in}{0.000000in}}{%
\pgfpathmoveto{\pgfqpoint{0.000000in}{0.000000in}}%
\pgfpathlineto{\pgfqpoint{-0.055556in}{0.000000in}}%
\pgfusepath{stroke,fill}%
}%
\begin{pgfscope}%
\pgfsys@transformshift{7.200000in}{0.780000in}%
\pgfsys@useobject{currentmarker}{}%
\end{pgfscope}%
\end{pgfscope}%
\begin{pgfscope}%
\pgftext[left,bottom,x=0.623456in,y=0.709790in,rotate=0.000000]{{\sffamily\fontsize{12.000000}{14.400000}\selectfont \(\displaystyle {10^{-2}}\)}}
%
\end{pgfscope}%
\begin{pgfscope}%
\pgfpathrectangle{\pgfqpoint{1.000000in}{0.300000in}}{\pgfqpoint{6.200000in}{2.400000in}} %
\pgfusepath{clip}%
\pgfsetbuttcap%
\pgfsetroundjoin%
\pgfsetlinewidth{0.501875pt}%
\definecolor{currentstroke}{rgb}{0.000000,0.000000,0.000000}%
\pgfsetstrokecolor{currentstroke}%
\pgfsetdash{{1.000000pt}{3.000000pt}}{0.000000pt}%
\pgfpathmoveto{\pgfqpoint{1.000000in}{1.260000in}}%
\pgfpathlineto{\pgfqpoint{7.200000in}{1.260000in}}%
\pgfusepath{stroke}%
\end{pgfscope}%
\begin{pgfscope}%
\pgfsetbuttcap%
\pgfsetroundjoin%
\definecolor{currentfill}{rgb}{0.000000,0.000000,0.000000}%
\pgfsetfillcolor{currentfill}%
\pgfsetlinewidth{0.501875pt}%
\definecolor{currentstroke}{rgb}{0.000000,0.000000,0.000000}%
\pgfsetstrokecolor{currentstroke}%
\pgfsetdash{}{0pt}%
\pgfsys@defobject{currentmarker}{\pgfqpoint{0.000000in}{0.000000in}}{\pgfqpoint{0.055556in}{0.000000in}}{%
\pgfpathmoveto{\pgfqpoint{0.000000in}{0.000000in}}%
\pgfpathlineto{\pgfqpoint{0.055556in}{0.000000in}}%
\pgfusepath{stroke,fill}%
}%
\begin{pgfscope}%
\pgfsys@transformshift{1.000000in}{1.260000in}%
\pgfsys@useobject{currentmarker}{}%
\end{pgfscope}%
\end{pgfscope}%
\begin{pgfscope}%
\pgfsetbuttcap%
\pgfsetroundjoin%
\definecolor{currentfill}{rgb}{0.000000,0.000000,0.000000}%
\pgfsetfillcolor{currentfill}%
\pgfsetlinewidth{0.501875pt}%
\definecolor{currentstroke}{rgb}{0.000000,0.000000,0.000000}%
\pgfsetstrokecolor{currentstroke}%
\pgfsetdash{}{0pt}%
\pgfsys@defobject{currentmarker}{\pgfqpoint{-0.055556in}{0.000000in}}{\pgfqpoint{0.000000in}{0.000000in}}{%
\pgfpathmoveto{\pgfqpoint{0.000000in}{0.000000in}}%
\pgfpathlineto{\pgfqpoint{-0.055556in}{0.000000in}}%
\pgfusepath{stroke,fill}%
}%
\begin{pgfscope}%
\pgfsys@transformshift{7.200000in}{1.260000in}%
\pgfsys@useobject{currentmarker}{}%
\end{pgfscope}%
\end{pgfscope}%
\begin{pgfscope}%
\pgftext[left,bottom,x=0.623456in,y=1.189790in,rotate=0.000000]{{\sffamily\fontsize{12.000000}{14.400000}\selectfont \(\displaystyle {10^{-1}}\)}}
%
\end{pgfscope}%
\begin{pgfscope}%
\pgfpathrectangle{\pgfqpoint{1.000000in}{0.300000in}}{\pgfqpoint{6.200000in}{2.400000in}} %
\pgfusepath{clip}%
\pgfsetbuttcap%
\pgfsetroundjoin%
\pgfsetlinewidth{0.501875pt}%
\definecolor{currentstroke}{rgb}{0.000000,0.000000,0.000000}%
\pgfsetstrokecolor{currentstroke}%
\pgfsetdash{{1.000000pt}{3.000000pt}}{0.000000pt}%
\pgfpathmoveto{\pgfqpoint{1.000000in}{1.740000in}}%
\pgfpathlineto{\pgfqpoint{7.200000in}{1.740000in}}%
\pgfusepath{stroke}%
\end{pgfscope}%
\begin{pgfscope}%
\pgfsetbuttcap%
\pgfsetroundjoin%
\definecolor{currentfill}{rgb}{0.000000,0.000000,0.000000}%
\pgfsetfillcolor{currentfill}%
\pgfsetlinewidth{0.501875pt}%
\definecolor{currentstroke}{rgb}{0.000000,0.000000,0.000000}%
\pgfsetstrokecolor{currentstroke}%
\pgfsetdash{}{0pt}%
\pgfsys@defobject{currentmarker}{\pgfqpoint{0.000000in}{0.000000in}}{\pgfqpoint{0.055556in}{0.000000in}}{%
\pgfpathmoveto{\pgfqpoint{0.000000in}{0.000000in}}%
\pgfpathlineto{\pgfqpoint{0.055556in}{0.000000in}}%
\pgfusepath{stroke,fill}%
}%
\begin{pgfscope}%
\pgfsys@transformshift{1.000000in}{1.740000in}%
\pgfsys@useobject{currentmarker}{}%
\end{pgfscope}%
\end{pgfscope}%
\begin{pgfscope}%
\pgfsetbuttcap%
\pgfsetroundjoin%
\definecolor{currentfill}{rgb}{0.000000,0.000000,0.000000}%
\pgfsetfillcolor{currentfill}%
\pgfsetlinewidth{0.501875pt}%
\definecolor{currentstroke}{rgb}{0.000000,0.000000,0.000000}%
\pgfsetstrokecolor{currentstroke}%
\pgfsetdash{}{0pt}%
\pgfsys@defobject{currentmarker}{\pgfqpoint{-0.055556in}{0.000000in}}{\pgfqpoint{0.000000in}{0.000000in}}{%
\pgfpathmoveto{\pgfqpoint{0.000000in}{0.000000in}}%
\pgfpathlineto{\pgfqpoint{-0.055556in}{0.000000in}}%
\pgfusepath{stroke,fill}%
}%
\begin{pgfscope}%
\pgfsys@transformshift{7.200000in}{1.740000in}%
\pgfsys@useobject{currentmarker}{}%
\end{pgfscope}%
\end{pgfscope}%
\begin{pgfscope}%
\pgftext[left,bottom,x=0.715279in,y=1.669790in,rotate=0.000000]{{\sffamily\fontsize{12.000000}{14.400000}\selectfont \(\displaystyle {10^{0}}\)}}
%
\end{pgfscope}%
\begin{pgfscope}%
\pgfpathrectangle{\pgfqpoint{1.000000in}{0.300000in}}{\pgfqpoint{6.200000in}{2.400000in}} %
\pgfusepath{clip}%
\pgfsetbuttcap%
\pgfsetroundjoin%
\pgfsetlinewidth{0.501875pt}%
\definecolor{currentstroke}{rgb}{0.000000,0.000000,0.000000}%
\pgfsetstrokecolor{currentstroke}%
\pgfsetdash{{1.000000pt}{3.000000pt}}{0.000000pt}%
\pgfpathmoveto{\pgfqpoint{1.000000in}{2.220000in}}%
\pgfpathlineto{\pgfqpoint{7.200000in}{2.220000in}}%
\pgfusepath{stroke}%
\end{pgfscope}%
\begin{pgfscope}%
\pgfsetbuttcap%
\pgfsetroundjoin%
\definecolor{currentfill}{rgb}{0.000000,0.000000,0.000000}%
\pgfsetfillcolor{currentfill}%
\pgfsetlinewidth{0.501875pt}%
\definecolor{currentstroke}{rgb}{0.000000,0.000000,0.000000}%
\pgfsetstrokecolor{currentstroke}%
\pgfsetdash{}{0pt}%
\pgfsys@defobject{currentmarker}{\pgfqpoint{0.000000in}{0.000000in}}{\pgfqpoint{0.055556in}{0.000000in}}{%
\pgfpathmoveto{\pgfqpoint{0.000000in}{0.000000in}}%
\pgfpathlineto{\pgfqpoint{0.055556in}{0.000000in}}%
\pgfusepath{stroke,fill}%
}%
\begin{pgfscope}%
\pgfsys@transformshift{1.000000in}{2.220000in}%
\pgfsys@useobject{currentmarker}{}%
\end{pgfscope}%
\end{pgfscope}%
\begin{pgfscope}%
\pgfsetbuttcap%
\pgfsetroundjoin%
\definecolor{currentfill}{rgb}{0.000000,0.000000,0.000000}%
\pgfsetfillcolor{currentfill}%
\pgfsetlinewidth{0.501875pt}%
\definecolor{currentstroke}{rgb}{0.000000,0.000000,0.000000}%
\pgfsetstrokecolor{currentstroke}%
\pgfsetdash{}{0pt}%
\pgfsys@defobject{currentmarker}{\pgfqpoint{-0.055556in}{0.000000in}}{\pgfqpoint{0.000000in}{0.000000in}}{%
\pgfpathmoveto{\pgfqpoint{0.000000in}{0.000000in}}%
\pgfpathlineto{\pgfqpoint{-0.055556in}{0.000000in}}%
\pgfusepath{stroke,fill}%
}%
\begin{pgfscope}%
\pgfsys@transformshift{7.200000in}{2.220000in}%
\pgfsys@useobject{currentmarker}{}%
\end{pgfscope}%
\end{pgfscope}%
\begin{pgfscope}%
\pgftext[left,bottom,x=0.715279in,y=2.149790in,rotate=0.000000]{{\sffamily\fontsize{12.000000}{14.400000}\selectfont \(\displaystyle {10^{1}}\)}}
%
\end{pgfscope}%
\begin{pgfscope}%
\pgfpathrectangle{\pgfqpoint{1.000000in}{0.300000in}}{\pgfqpoint{6.200000in}{2.400000in}} %
\pgfusepath{clip}%
\pgfsetbuttcap%
\pgfsetroundjoin%
\pgfsetlinewidth{0.501875pt}%
\definecolor{currentstroke}{rgb}{0.000000,0.000000,0.000000}%
\pgfsetstrokecolor{currentstroke}%
\pgfsetdash{{1.000000pt}{3.000000pt}}{0.000000pt}%
\pgfpathmoveto{\pgfqpoint{1.000000in}{2.700000in}}%
\pgfpathlineto{\pgfqpoint{7.200000in}{2.700000in}}%
\pgfusepath{stroke}%
\end{pgfscope}%
\begin{pgfscope}%
\pgfsetbuttcap%
\pgfsetroundjoin%
\definecolor{currentfill}{rgb}{0.000000,0.000000,0.000000}%
\pgfsetfillcolor{currentfill}%
\pgfsetlinewidth{0.501875pt}%
\definecolor{currentstroke}{rgb}{0.000000,0.000000,0.000000}%
\pgfsetstrokecolor{currentstroke}%
\pgfsetdash{}{0pt}%
\pgfsys@defobject{currentmarker}{\pgfqpoint{0.000000in}{0.000000in}}{\pgfqpoint{0.055556in}{0.000000in}}{%
\pgfpathmoveto{\pgfqpoint{0.000000in}{0.000000in}}%
\pgfpathlineto{\pgfqpoint{0.055556in}{0.000000in}}%
\pgfusepath{stroke,fill}%
}%
\begin{pgfscope}%
\pgfsys@transformshift{1.000000in}{2.700000in}%
\pgfsys@useobject{currentmarker}{}%
\end{pgfscope}%
\end{pgfscope}%
\begin{pgfscope}%
\pgfsetbuttcap%
\pgfsetroundjoin%
\definecolor{currentfill}{rgb}{0.000000,0.000000,0.000000}%
\pgfsetfillcolor{currentfill}%
\pgfsetlinewidth{0.501875pt}%
\definecolor{currentstroke}{rgb}{0.000000,0.000000,0.000000}%
\pgfsetstrokecolor{currentstroke}%
\pgfsetdash{}{0pt}%
\pgfsys@defobject{currentmarker}{\pgfqpoint{-0.055556in}{0.000000in}}{\pgfqpoint{0.000000in}{0.000000in}}{%
\pgfpathmoveto{\pgfqpoint{0.000000in}{0.000000in}}%
\pgfpathlineto{\pgfqpoint{-0.055556in}{0.000000in}}%
\pgfusepath{stroke,fill}%
}%
\begin{pgfscope}%
\pgfsys@transformshift{7.200000in}{2.700000in}%
\pgfsys@useobject{currentmarker}{}%
\end{pgfscope}%
\end{pgfscope}%
\begin{pgfscope}%
\pgftext[left,bottom,x=0.715279in,y=2.629790in,rotate=0.000000]{{\sffamily\fontsize{12.000000}{14.400000}\selectfont \(\displaystyle {10^{2}}\)}}
%
\end{pgfscope}%
\begin{pgfscope}%
\pgfsetbuttcap%
\pgfsetroundjoin%
\definecolor{currentfill}{rgb}{0.000000,0.000000,0.000000}%
\pgfsetfillcolor{currentfill}%
\pgfsetlinewidth{0.501875pt}%
\definecolor{currentstroke}{rgb}{0.000000,0.000000,0.000000}%
\pgfsetstrokecolor{currentstroke}%
\pgfsetdash{}{0pt}%
\pgfsys@defobject{currentmarker}{\pgfqpoint{0.000000in}{0.000000in}}{\pgfqpoint{0.027778in}{0.000000in}}{%
\pgfpathmoveto{\pgfqpoint{0.000000in}{0.000000in}}%
\pgfpathlineto{\pgfqpoint{0.027778in}{0.000000in}}%
\pgfusepath{stroke,fill}%
}%
\begin{pgfscope}%
\pgfsys@transformshift{1.000000in}{0.444494in}%
\pgfsys@useobject{currentmarker}{}%
\end{pgfscope}%
\end{pgfscope}%
\begin{pgfscope}%
\pgfsetbuttcap%
\pgfsetroundjoin%
\definecolor{currentfill}{rgb}{0.000000,0.000000,0.000000}%
\pgfsetfillcolor{currentfill}%
\pgfsetlinewidth{0.501875pt}%
\definecolor{currentstroke}{rgb}{0.000000,0.000000,0.000000}%
\pgfsetstrokecolor{currentstroke}%
\pgfsetdash{}{0pt}%
\pgfsys@defobject{currentmarker}{\pgfqpoint{-0.027778in}{0.000000in}}{\pgfqpoint{0.000000in}{0.000000in}}{%
\pgfpathmoveto{\pgfqpoint{0.000000in}{0.000000in}}%
\pgfpathlineto{\pgfqpoint{-0.027778in}{0.000000in}}%
\pgfusepath{stroke,fill}%
}%
\begin{pgfscope}%
\pgfsys@transformshift{7.200000in}{0.444494in}%
\pgfsys@useobject{currentmarker}{}%
\end{pgfscope}%
\end{pgfscope}%
\begin{pgfscope}%
\pgfsetbuttcap%
\pgfsetroundjoin%
\definecolor{currentfill}{rgb}{0.000000,0.000000,0.000000}%
\pgfsetfillcolor{currentfill}%
\pgfsetlinewidth{0.501875pt}%
\definecolor{currentstroke}{rgb}{0.000000,0.000000,0.000000}%
\pgfsetstrokecolor{currentstroke}%
\pgfsetdash{}{0pt}%
\pgfsys@defobject{currentmarker}{\pgfqpoint{0.000000in}{0.000000in}}{\pgfqpoint{0.027778in}{0.000000in}}{%
\pgfpathmoveto{\pgfqpoint{0.000000in}{0.000000in}}%
\pgfpathlineto{\pgfqpoint{0.027778in}{0.000000in}}%
\pgfusepath{stroke,fill}%
}%
\begin{pgfscope}%
\pgfsys@transformshift{1.000000in}{0.529018in}%
\pgfsys@useobject{currentmarker}{}%
\end{pgfscope}%
\end{pgfscope}%
\begin{pgfscope}%
\pgfsetbuttcap%
\pgfsetroundjoin%
\definecolor{currentfill}{rgb}{0.000000,0.000000,0.000000}%
\pgfsetfillcolor{currentfill}%
\pgfsetlinewidth{0.501875pt}%
\definecolor{currentstroke}{rgb}{0.000000,0.000000,0.000000}%
\pgfsetstrokecolor{currentstroke}%
\pgfsetdash{}{0pt}%
\pgfsys@defobject{currentmarker}{\pgfqpoint{-0.027778in}{0.000000in}}{\pgfqpoint{0.000000in}{0.000000in}}{%
\pgfpathmoveto{\pgfqpoint{0.000000in}{0.000000in}}%
\pgfpathlineto{\pgfqpoint{-0.027778in}{0.000000in}}%
\pgfusepath{stroke,fill}%
}%
\begin{pgfscope}%
\pgfsys@transformshift{7.200000in}{0.529018in}%
\pgfsys@useobject{currentmarker}{}%
\end{pgfscope}%
\end{pgfscope}%
\begin{pgfscope}%
\pgfsetbuttcap%
\pgfsetroundjoin%
\definecolor{currentfill}{rgb}{0.000000,0.000000,0.000000}%
\pgfsetfillcolor{currentfill}%
\pgfsetlinewidth{0.501875pt}%
\definecolor{currentstroke}{rgb}{0.000000,0.000000,0.000000}%
\pgfsetstrokecolor{currentstroke}%
\pgfsetdash{}{0pt}%
\pgfsys@defobject{currentmarker}{\pgfqpoint{0.000000in}{0.000000in}}{\pgfqpoint{0.027778in}{0.000000in}}{%
\pgfpathmoveto{\pgfqpoint{0.000000in}{0.000000in}}%
\pgfpathlineto{\pgfqpoint{0.027778in}{0.000000in}}%
\pgfusepath{stroke,fill}%
}%
\begin{pgfscope}%
\pgfsys@transformshift{1.000000in}{0.588989in}%
\pgfsys@useobject{currentmarker}{}%
\end{pgfscope}%
\end{pgfscope}%
\begin{pgfscope}%
\pgfsetbuttcap%
\pgfsetroundjoin%
\definecolor{currentfill}{rgb}{0.000000,0.000000,0.000000}%
\pgfsetfillcolor{currentfill}%
\pgfsetlinewidth{0.501875pt}%
\definecolor{currentstroke}{rgb}{0.000000,0.000000,0.000000}%
\pgfsetstrokecolor{currentstroke}%
\pgfsetdash{}{0pt}%
\pgfsys@defobject{currentmarker}{\pgfqpoint{-0.027778in}{0.000000in}}{\pgfqpoint{0.000000in}{0.000000in}}{%
\pgfpathmoveto{\pgfqpoint{0.000000in}{0.000000in}}%
\pgfpathlineto{\pgfqpoint{-0.027778in}{0.000000in}}%
\pgfusepath{stroke,fill}%
}%
\begin{pgfscope}%
\pgfsys@transformshift{7.200000in}{0.588989in}%
\pgfsys@useobject{currentmarker}{}%
\end{pgfscope}%
\end{pgfscope}%
\begin{pgfscope}%
\pgfsetbuttcap%
\pgfsetroundjoin%
\definecolor{currentfill}{rgb}{0.000000,0.000000,0.000000}%
\pgfsetfillcolor{currentfill}%
\pgfsetlinewidth{0.501875pt}%
\definecolor{currentstroke}{rgb}{0.000000,0.000000,0.000000}%
\pgfsetstrokecolor{currentstroke}%
\pgfsetdash{}{0pt}%
\pgfsys@defobject{currentmarker}{\pgfqpoint{0.000000in}{0.000000in}}{\pgfqpoint{0.027778in}{0.000000in}}{%
\pgfpathmoveto{\pgfqpoint{0.000000in}{0.000000in}}%
\pgfpathlineto{\pgfqpoint{0.027778in}{0.000000in}}%
\pgfusepath{stroke,fill}%
}%
\begin{pgfscope}%
\pgfsys@transformshift{1.000000in}{0.635506in}%
\pgfsys@useobject{currentmarker}{}%
\end{pgfscope}%
\end{pgfscope}%
\begin{pgfscope}%
\pgfsetbuttcap%
\pgfsetroundjoin%
\definecolor{currentfill}{rgb}{0.000000,0.000000,0.000000}%
\pgfsetfillcolor{currentfill}%
\pgfsetlinewidth{0.501875pt}%
\definecolor{currentstroke}{rgb}{0.000000,0.000000,0.000000}%
\pgfsetstrokecolor{currentstroke}%
\pgfsetdash{}{0pt}%
\pgfsys@defobject{currentmarker}{\pgfqpoint{-0.027778in}{0.000000in}}{\pgfqpoint{0.000000in}{0.000000in}}{%
\pgfpathmoveto{\pgfqpoint{0.000000in}{0.000000in}}%
\pgfpathlineto{\pgfqpoint{-0.027778in}{0.000000in}}%
\pgfusepath{stroke,fill}%
}%
\begin{pgfscope}%
\pgfsys@transformshift{7.200000in}{0.635506in}%
\pgfsys@useobject{currentmarker}{}%
\end{pgfscope}%
\end{pgfscope}%
\begin{pgfscope}%
\pgfsetbuttcap%
\pgfsetroundjoin%
\definecolor{currentfill}{rgb}{0.000000,0.000000,0.000000}%
\pgfsetfillcolor{currentfill}%
\pgfsetlinewidth{0.501875pt}%
\definecolor{currentstroke}{rgb}{0.000000,0.000000,0.000000}%
\pgfsetstrokecolor{currentstroke}%
\pgfsetdash{}{0pt}%
\pgfsys@defobject{currentmarker}{\pgfqpoint{0.000000in}{0.000000in}}{\pgfqpoint{0.027778in}{0.000000in}}{%
\pgfpathmoveto{\pgfqpoint{0.000000in}{0.000000in}}%
\pgfpathlineto{\pgfqpoint{0.027778in}{0.000000in}}%
\pgfusepath{stroke,fill}%
}%
\begin{pgfscope}%
\pgfsys@transformshift{1.000000in}{0.673513in}%
\pgfsys@useobject{currentmarker}{}%
\end{pgfscope}%
\end{pgfscope}%
\begin{pgfscope}%
\pgfsetbuttcap%
\pgfsetroundjoin%
\definecolor{currentfill}{rgb}{0.000000,0.000000,0.000000}%
\pgfsetfillcolor{currentfill}%
\pgfsetlinewidth{0.501875pt}%
\definecolor{currentstroke}{rgb}{0.000000,0.000000,0.000000}%
\pgfsetstrokecolor{currentstroke}%
\pgfsetdash{}{0pt}%
\pgfsys@defobject{currentmarker}{\pgfqpoint{-0.027778in}{0.000000in}}{\pgfqpoint{0.000000in}{0.000000in}}{%
\pgfpathmoveto{\pgfqpoint{0.000000in}{0.000000in}}%
\pgfpathlineto{\pgfqpoint{-0.027778in}{0.000000in}}%
\pgfusepath{stroke,fill}%
}%
\begin{pgfscope}%
\pgfsys@transformshift{7.200000in}{0.673513in}%
\pgfsys@useobject{currentmarker}{}%
\end{pgfscope}%
\end{pgfscope}%
\begin{pgfscope}%
\pgfsetbuttcap%
\pgfsetroundjoin%
\definecolor{currentfill}{rgb}{0.000000,0.000000,0.000000}%
\pgfsetfillcolor{currentfill}%
\pgfsetlinewidth{0.501875pt}%
\definecolor{currentstroke}{rgb}{0.000000,0.000000,0.000000}%
\pgfsetstrokecolor{currentstroke}%
\pgfsetdash{}{0pt}%
\pgfsys@defobject{currentmarker}{\pgfqpoint{0.000000in}{0.000000in}}{\pgfqpoint{0.027778in}{0.000000in}}{%
\pgfpathmoveto{\pgfqpoint{0.000000in}{0.000000in}}%
\pgfpathlineto{\pgfqpoint{0.027778in}{0.000000in}}%
\pgfusepath{stroke,fill}%
}%
\begin{pgfscope}%
\pgfsys@transformshift{1.000000in}{0.705647in}%
\pgfsys@useobject{currentmarker}{}%
\end{pgfscope}%
\end{pgfscope}%
\begin{pgfscope}%
\pgfsetbuttcap%
\pgfsetroundjoin%
\definecolor{currentfill}{rgb}{0.000000,0.000000,0.000000}%
\pgfsetfillcolor{currentfill}%
\pgfsetlinewidth{0.501875pt}%
\definecolor{currentstroke}{rgb}{0.000000,0.000000,0.000000}%
\pgfsetstrokecolor{currentstroke}%
\pgfsetdash{}{0pt}%
\pgfsys@defobject{currentmarker}{\pgfqpoint{-0.027778in}{0.000000in}}{\pgfqpoint{0.000000in}{0.000000in}}{%
\pgfpathmoveto{\pgfqpoint{0.000000in}{0.000000in}}%
\pgfpathlineto{\pgfqpoint{-0.027778in}{0.000000in}}%
\pgfusepath{stroke,fill}%
}%
\begin{pgfscope}%
\pgfsys@transformshift{7.200000in}{0.705647in}%
\pgfsys@useobject{currentmarker}{}%
\end{pgfscope}%
\end{pgfscope}%
\begin{pgfscope}%
\pgfsetbuttcap%
\pgfsetroundjoin%
\definecolor{currentfill}{rgb}{0.000000,0.000000,0.000000}%
\pgfsetfillcolor{currentfill}%
\pgfsetlinewidth{0.501875pt}%
\definecolor{currentstroke}{rgb}{0.000000,0.000000,0.000000}%
\pgfsetstrokecolor{currentstroke}%
\pgfsetdash{}{0pt}%
\pgfsys@defobject{currentmarker}{\pgfqpoint{0.000000in}{0.000000in}}{\pgfqpoint{0.027778in}{0.000000in}}{%
\pgfpathmoveto{\pgfqpoint{0.000000in}{0.000000in}}%
\pgfpathlineto{\pgfqpoint{0.027778in}{0.000000in}}%
\pgfusepath{stroke,fill}%
}%
\begin{pgfscope}%
\pgfsys@transformshift{1.000000in}{0.733483in}%
\pgfsys@useobject{currentmarker}{}%
\end{pgfscope}%
\end{pgfscope}%
\begin{pgfscope}%
\pgfsetbuttcap%
\pgfsetroundjoin%
\definecolor{currentfill}{rgb}{0.000000,0.000000,0.000000}%
\pgfsetfillcolor{currentfill}%
\pgfsetlinewidth{0.501875pt}%
\definecolor{currentstroke}{rgb}{0.000000,0.000000,0.000000}%
\pgfsetstrokecolor{currentstroke}%
\pgfsetdash{}{0pt}%
\pgfsys@defobject{currentmarker}{\pgfqpoint{-0.027778in}{0.000000in}}{\pgfqpoint{0.000000in}{0.000000in}}{%
\pgfpathmoveto{\pgfqpoint{0.000000in}{0.000000in}}%
\pgfpathlineto{\pgfqpoint{-0.027778in}{0.000000in}}%
\pgfusepath{stroke,fill}%
}%
\begin{pgfscope}%
\pgfsys@transformshift{7.200000in}{0.733483in}%
\pgfsys@useobject{currentmarker}{}%
\end{pgfscope}%
\end{pgfscope}%
\begin{pgfscope}%
\pgfsetbuttcap%
\pgfsetroundjoin%
\definecolor{currentfill}{rgb}{0.000000,0.000000,0.000000}%
\pgfsetfillcolor{currentfill}%
\pgfsetlinewidth{0.501875pt}%
\definecolor{currentstroke}{rgb}{0.000000,0.000000,0.000000}%
\pgfsetstrokecolor{currentstroke}%
\pgfsetdash{}{0pt}%
\pgfsys@defobject{currentmarker}{\pgfqpoint{0.000000in}{0.000000in}}{\pgfqpoint{0.027778in}{0.000000in}}{%
\pgfpathmoveto{\pgfqpoint{0.000000in}{0.000000in}}%
\pgfpathlineto{\pgfqpoint{0.027778in}{0.000000in}}%
\pgfusepath{stroke,fill}%
}%
\begin{pgfscope}%
\pgfsys@transformshift{1.000000in}{0.758036in}%
\pgfsys@useobject{currentmarker}{}%
\end{pgfscope}%
\end{pgfscope}%
\begin{pgfscope}%
\pgfsetbuttcap%
\pgfsetroundjoin%
\definecolor{currentfill}{rgb}{0.000000,0.000000,0.000000}%
\pgfsetfillcolor{currentfill}%
\pgfsetlinewidth{0.501875pt}%
\definecolor{currentstroke}{rgb}{0.000000,0.000000,0.000000}%
\pgfsetstrokecolor{currentstroke}%
\pgfsetdash{}{0pt}%
\pgfsys@defobject{currentmarker}{\pgfqpoint{-0.027778in}{0.000000in}}{\pgfqpoint{0.000000in}{0.000000in}}{%
\pgfpathmoveto{\pgfqpoint{0.000000in}{0.000000in}}%
\pgfpathlineto{\pgfqpoint{-0.027778in}{0.000000in}}%
\pgfusepath{stroke,fill}%
}%
\begin{pgfscope}%
\pgfsys@transformshift{7.200000in}{0.758036in}%
\pgfsys@useobject{currentmarker}{}%
\end{pgfscope}%
\end{pgfscope}%
\begin{pgfscope}%
\pgfsetbuttcap%
\pgfsetroundjoin%
\definecolor{currentfill}{rgb}{0.000000,0.000000,0.000000}%
\pgfsetfillcolor{currentfill}%
\pgfsetlinewidth{0.501875pt}%
\definecolor{currentstroke}{rgb}{0.000000,0.000000,0.000000}%
\pgfsetstrokecolor{currentstroke}%
\pgfsetdash{}{0pt}%
\pgfsys@defobject{currentmarker}{\pgfqpoint{0.000000in}{0.000000in}}{\pgfqpoint{0.027778in}{0.000000in}}{%
\pgfpathmoveto{\pgfqpoint{0.000000in}{0.000000in}}%
\pgfpathlineto{\pgfqpoint{0.027778in}{0.000000in}}%
\pgfusepath{stroke,fill}%
}%
\begin{pgfscope}%
\pgfsys@transformshift{1.000000in}{0.924494in}%
\pgfsys@useobject{currentmarker}{}%
\end{pgfscope}%
\end{pgfscope}%
\begin{pgfscope}%
\pgfsetbuttcap%
\pgfsetroundjoin%
\definecolor{currentfill}{rgb}{0.000000,0.000000,0.000000}%
\pgfsetfillcolor{currentfill}%
\pgfsetlinewidth{0.501875pt}%
\definecolor{currentstroke}{rgb}{0.000000,0.000000,0.000000}%
\pgfsetstrokecolor{currentstroke}%
\pgfsetdash{}{0pt}%
\pgfsys@defobject{currentmarker}{\pgfqpoint{-0.027778in}{0.000000in}}{\pgfqpoint{0.000000in}{0.000000in}}{%
\pgfpathmoveto{\pgfqpoint{0.000000in}{0.000000in}}%
\pgfpathlineto{\pgfqpoint{-0.027778in}{0.000000in}}%
\pgfusepath{stroke,fill}%
}%
\begin{pgfscope}%
\pgfsys@transformshift{7.200000in}{0.924494in}%
\pgfsys@useobject{currentmarker}{}%
\end{pgfscope}%
\end{pgfscope}%
\begin{pgfscope}%
\pgfsetbuttcap%
\pgfsetroundjoin%
\definecolor{currentfill}{rgb}{0.000000,0.000000,0.000000}%
\pgfsetfillcolor{currentfill}%
\pgfsetlinewidth{0.501875pt}%
\definecolor{currentstroke}{rgb}{0.000000,0.000000,0.000000}%
\pgfsetstrokecolor{currentstroke}%
\pgfsetdash{}{0pt}%
\pgfsys@defobject{currentmarker}{\pgfqpoint{0.000000in}{0.000000in}}{\pgfqpoint{0.027778in}{0.000000in}}{%
\pgfpathmoveto{\pgfqpoint{0.000000in}{0.000000in}}%
\pgfpathlineto{\pgfqpoint{0.027778in}{0.000000in}}%
\pgfusepath{stroke,fill}%
}%
\begin{pgfscope}%
\pgfsys@transformshift{1.000000in}{1.009018in}%
\pgfsys@useobject{currentmarker}{}%
\end{pgfscope}%
\end{pgfscope}%
\begin{pgfscope}%
\pgfsetbuttcap%
\pgfsetroundjoin%
\definecolor{currentfill}{rgb}{0.000000,0.000000,0.000000}%
\pgfsetfillcolor{currentfill}%
\pgfsetlinewidth{0.501875pt}%
\definecolor{currentstroke}{rgb}{0.000000,0.000000,0.000000}%
\pgfsetstrokecolor{currentstroke}%
\pgfsetdash{}{0pt}%
\pgfsys@defobject{currentmarker}{\pgfqpoint{-0.027778in}{0.000000in}}{\pgfqpoint{0.000000in}{0.000000in}}{%
\pgfpathmoveto{\pgfqpoint{0.000000in}{0.000000in}}%
\pgfpathlineto{\pgfqpoint{-0.027778in}{0.000000in}}%
\pgfusepath{stroke,fill}%
}%
\begin{pgfscope}%
\pgfsys@transformshift{7.200000in}{1.009018in}%
\pgfsys@useobject{currentmarker}{}%
\end{pgfscope}%
\end{pgfscope}%
\begin{pgfscope}%
\pgfsetbuttcap%
\pgfsetroundjoin%
\definecolor{currentfill}{rgb}{0.000000,0.000000,0.000000}%
\pgfsetfillcolor{currentfill}%
\pgfsetlinewidth{0.501875pt}%
\definecolor{currentstroke}{rgb}{0.000000,0.000000,0.000000}%
\pgfsetstrokecolor{currentstroke}%
\pgfsetdash{}{0pt}%
\pgfsys@defobject{currentmarker}{\pgfqpoint{0.000000in}{0.000000in}}{\pgfqpoint{0.027778in}{0.000000in}}{%
\pgfpathmoveto{\pgfqpoint{0.000000in}{0.000000in}}%
\pgfpathlineto{\pgfqpoint{0.027778in}{0.000000in}}%
\pgfusepath{stroke,fill}%
}%
\begin{pgfscope}%
\pgfsys@transformshift{1.000000in}{1.068989in}%
\pgfsys@useobject{currentmarker}{}%
\end{pgfscope}%
\end{pgfscope}%
\begin{pgfscope}%
\pgfsetbuttcap%
\pgfsetroundjoin%
\definecolor{currentfill}{rgb}{0.000000,0.000000,0.000000}%
\pgfsetfillcolor{currentfill}%
\pgfsetlinewidth{0.501875pt}%
\definecolor{currentstroke}{rgb}{0.000000,0.000000,0.000000}%
\pgfsetstrokecolor{currentstroke}%
\pgfsetdash{}{0pt}%
\pgfsys@defobject{currentmarker}{\pgfqpoint{-0.027778in}{0.000000in}}{\pgfqpoint{0.000000in}{0.000000in}}{%
\pgfpathmoveto{\pgfqpoint{0.000000in}{0.000000in}}%
\pgfpathlineto{\pgfqpoint{-0.027778in}{0.000000in}}%
\pgfusepath{stroke,fill}%
}%
\begin{pgfscope}%
\pgfsys@transformshift{7.200000in}{1.068989in}%
\pgfsys@useobject{currentmarker}{}%
\end{pgfscope}%
\end{pgfscope}%
\begin{pgfscope}%
\pgfsetbuttcap%
\pgfsetroundjoin%
\definecolor{currentfill}{rgb}{0.000000,0.000000,0.000000}%
\pgfsetfillcolor{currentfill}%
\pgfsetlinewidth{0.501875pt}%
\definecolor{currentstroke}{rgb}{0.000000,0.000000,0.000000}%
\pgfsetstrokecolor{currentstroke}%
\pgfsetdash{}{0pt}%
\pgfsys@defobject{currentmarker}{\pgfqpoint{0.000000in}{0.000000in}}{\pgfqpoint{0.027778in}{0.000000in}}{%
\pgfpathmoveto{\pgfqpoint{0.000000in}{0.000000in}}%
\pgfpathlineto{\pgfqpoint{0.027778in}{0.000000in}}%
\pgfusepath{stroke,fill}%
}%
\begin{pgfscope}%
\pgfsys@transformshift{1.000000in}{1.115506in}%
\pgfsys@useobject{currentmarker}{}%
\end{pgfscope}%
\end{pgfscope}%
\begin{pgfscope}%
\pgfsetbuttcap%
\pgfsetroundjoin%
\definecolor{currentfill}{rgb}{0.000000,0.000000,0.000000}%
\pgfsetfillcolor{currentfill}%
\pgfsetlinewidth{0.501875pt}%
\definecolor{currentstroke}{rgb}{0.000000,0.000000,0.000000}%
\pgfsetstrokecolor{currentstroke}%
\pgfsetdash{}{0pt}%
\pgfsys@defobject{currentmarker}{\pgfqpoint{-0.027778in}{0.000000in}}{\pgfqpoint{0.000000in}{0.000000in}}{%
\pgfpathmoveto{\pgfqpoint{0.000000in}{0.000000in}}%
\pgfpathlineto{\pgfqpoint{-0.027778in}{0.000000in}}%
\pgfusepath{stroke,fill}%
}%
\begin{pgfscope}%
\pgfsys@transformshift{7.200000in}{1.115506in}%
\pgfsys@useobject{currentmarker}{}%
\end{pgfscope}%
\end{pgfscope}%
\begin{pgfscope}%
\pgfsetbuttcap%
\pgfsetroundjoin%
\definecolor{currentfill}{rgb}{0.000000,0.000000,0.000000}%
\pgfsetfillcolor{currentfill}%
\pgfsetlinewidth{0.501875pt}%
\definecolor{currentstroke}{rgb}{0.000000,0.000000,0.000000}%
\pgfsetstrokecolor{currentstroke}%
\pgfsetdash{}{0pt}%
\pgfsys@defobject{currentmarker}{\pgfqpoint{0.000000in}{0.000000in}}{\pgfqpoint{0.027778in}{0.000000in}}{%
\pgfpathmoveto{\pgfqpoint{0.000000in}{0.000000in}}%
\pgfpathlineto{\pgfqpoint{0.027778in}{0.000000in}}%
\pgfusepath{stroke,fill}%
}%
\begin{pgfscope}%
\pgfsys@transformshift{1.000000in}{1.153513in}%
\pgfsys@useobject{currentmarker}{}%
\end{pgfscope}%
\end{pgfscope}%
\begin{pgfscope}%
\pgfsetbuttcap%
\pgfsetroundjoin%
\definecolor{currentfill}{rgb}{0.000000,0.000000,0.000000}%
\pgfsetfillcolor{currentfill}%
\pgfsetlinewidth{0.501875pt}%
\definecolor{currentstroke}{rgb}{0.000000,0.000000,0.000000}%
\pgfsetstrokecolor{currentstroke}%
\pgfsetdash{}{0pt}%
\pgfsys@defobject{currentmarker}{\pgfqpoint{-0.027778in}{0.000000in}}{\pgfqpoint{0.000000in}{0.000000in}}{%
\pgfpathmoveto{\pgfqpoint{0.000000in}{0.000000in}}%
\pgfpathlineto{\pgfqpoint{-0.027778in}{0.000000in}}%
\pgfusepath{stroke,fill}%
}%
\begin{pgfscope}%
\pgfsys@transformshift{7.200000in}{1.153513in}%
\pgfsys@useobject{currentmarker}{}%
\end{pgfscope}%
\end{pgfscope}%
\begin{pgfscope}%
\pgfsetbuttcap%
\pgfsetroundjoin%
\definecolor{currentfill}{rgb}{0.000000,0.000000,0.000000}%
\pgfsetfillcolor{currentfill}%
\pgfsetlinewidth{0.501875pt}%
\definecolor{currentstroke}{rgb}{0.000000,0.000000,0.000000}%
\pgfsetstrokecolor{currentstroke}%
\pgfsetdash{}{0pt}%
\pgfsys@defobject{currentmarker}{\pgfqpoint{0.000000in}{0.000000in}}{\pgfqpoint{0.027778in}{0.000000in}}{%
\pgfpathmoveto{\pgfqpoint{0.000000in}{0.000000in}}%
\pgfpathlineto{\pgfqpoint{0.027778in}{0.000000in}}%
\pgfusepath{stroke,fill}%
}%
\begin{pgfscope}%
\pgfsys@transformshift{1.000000in}{1.185647in}%
\pgfsys@useobject{currentmarker}{}%
\end{pgfscope}%
\end{pgfscope}%
\begin{pgfscope}%
\pgfsetbuttcap%
\pgfsetroundjoin%
\definecolor{currentfill}{rgb}{0.000000,0.000000,0.000000}%
\pgfsetfillcolor{currentfill}%
\pgfsetlinewidth{0.501875pt}%
\definecolor{currentstroke}{rgb}{0.000000,0.000000,0.000000}%
\pgfsetstrokecolor{currentstroke}%
\pgfsetdash{}{0pt}%
\pgfsys@defobject{currentmarker}{\pgfqpoint{-0.027778in}{0.000000in}}{\pgfqpoint{0.000000in}{0.000000in}}{%
\pgfpathmoveto{\pgfqpoint{0.000000in}{0.000000in}}%
\pgfpathlineto{\pgfqpoint{-0.027778in}{0.000000in}}%
\pgfusepath{stroke,fill}%
}%
\begin{pgfscope}%
\pgfsys@transformshift{7.200000in}{1.185647in}%
\pgfsys@useobject{currentmarker}{}%
\end{pgfscope}%
\end{pgfscope}%
\begin{pgfscope}%
\pgfsetbuttcap%
\pgfsetroundjoin%
\definecolor{currentfill}{rgb}{0.000000,0.000000,0.000000}%
\pgfsetfillcolor{currentfill}%
\pgfsetlinewidth{0.501875pt}%
\definecolor{currentstroke}{rgb}{0.000000,0.000000,0.000000}%
\pgfsetstrokecolor{currentstroke}%
\pgfsetdash{}{0pt}%
\pgfsys@defobject{currentmarker}{\pgfqpoint{0.000000in}{0.000000in}}{\pgfqpoint{0.027778in}{0.000000in}}{%
\pgfpathmoveto{\pgfqpoint{0.000000in}{0.000000in}}%
\pgfpathlineto{\pgfqpoint{0.027778in}{0.000000in}}%
\pgfusepath{stroke,fill}%
}%
\begin{pgfscope}%
\pgfsys@transformshift{1.000000in}{1.213483in}%
\pgfsys@useobject{currentmarker}{}%
\end{pgfscope}%
\end{pgfscope}%
\begin{pgfscope}%
\pgfsetbuttcap%
\pgfsetroundjoin%
\definecolor{currentfill}{rgb}{0.000000,0.000000,0.000000}%
\pgfsetfillcolor{currentfill}%
\pgfsetlinewidth{0.501875pt}%
\definecolor{currentstroke}{rgb}{0.000000,0.000000,0.000000}%
\pgfsetstrokecolor{currentstroke}%
\pgfsetdash{}{0pt}%
\pgfsys@defobject{currentmarker}{\pgfqpoint{-0.027778in}{0.000000in}}{\pgfqpoint{0.000000in}{0.000000in}}{%
\pgfpathmoveto{\pgfqpoint{0.000000in}{0.000000in}}%
\pgfpathlineto{\pgfqpoint{-0.027778in}{0.000000in}}%
\pgfusepath{stroke,fill}%
}%
\begin{pgfscope}%
\pgfsys@transformshift{7.200000in}{1.213483in}%
\pgfsys@useobject{currentmarker}{}%
\end{pgfscope}%
\end{pgfscope}%
\begin{pgfscope}%
\pgfsetbuttcap%
\pgfsetroundjoin%
\definecolor{currentfill}{rgb}{0.000000,0.000000,0.000000}%
\pgfsetfillcolor{currentfill}%
\pgfsetlinewidth{0.501875pt}%
\definecolor{currentstroke}{rgb}{0.000000,0.000000,0.000000}%
\pgfsetstrokecolor{currentstroke}%
\pgfsetdash{}{0pt}%
\pgfsys@defobject{currentmarker}{\pgfqpoint{0.000000in}{0.000000in}}{\pgfqpoint{0.027778in}{0.000000in}}{%
\pgfpathmoveto{\pgfqpoint{0.000000in}{0.000000in}}%
\pgfpathlineto{\pgfqpoint{0.027778in}{0.000000in}}%
\pgfusepath{stroke,fill}%
}%
\begin{pgfscope}%
\pgfsys@transformshift{1.000000in}{1.238036in}%
\pgfsys@useobject{currentmarker}{}%
\end{pgfscope}%
\end{pgfscope}%
\begin{pgfscope}%
\pgfsetbuttcap%
\pgfsetroundjoin%
\definecolor{currentfill}{rgb}{0.000000,0.000000,0.000000}%
\pgfsetfillcolor{currentfill}%
\pgfsetlinewidth{0.501875pt}%
\definecolor{currentstroke}{rgb}{0.000000,0.000000,0.000000}%
\pgfsetstrokecolor{currentstroke}%
\pgfsetdash{}{0pt}%
\pgfsys@defobject{currentmarker}{\pgfqpoint{-0.027778in}{0.000000in}}{\pgfqpoint{0.000000in}{0.000000in}}{%
\pgfpathmoveto{\pgfqpoint{0.000000in}{0.000000in}}%
\pgfpathlineto{\pgfqpoint{-0.027778in}{0.000000in}}%
\pgfusepath{stroke,fill}%
}%
\begin{pgfscope}%
\pgfsys@transformshift{7.200000in}{1.238036in}%
\pgfsys@useobject{currentmarker}{}%
\end{pgfscope}%
\end{pgfscope}%
\begin{pgfscope}%
\pgfsetbuttcap%
\pgfsetroundjoin%
\definecolor{currentfill}{rgb}{0.000000,0.000000,0.000000}%
\pgfsetfillcolor{currentfill}%
\pgfsetlinewidth{0.501875pt}%
\definecolor{currentstroke}{rgb}{0.000000,0.000000,0.000000}%
\pgfsetstrokecolor{currentstroke}%
\pgfsetdash{}{0pt}%
\pgfsys@defobject{currentmarker}{\pgfqpoint{0.000000in}{0.000000in}}{\pgfqpoint{0.027778in}{0.000000in}}{%
\pgfpathmoveto{\pgfqpoint{0.000000in}{0.000000in}}%
\pgfpathlineto{\pgfqpoint{0.027778in}{0.000000in}}%
\pgfusepath{stroke,fill}%
}%
\begin{pgfscope}%
\pgfsys@transformshift{1.000000in}{1.404494in}%
\pgfsys@useobject{currentmarker}{}%
\end{pgfscope}%
\end{pgfscope}%
\begin{pgfscope}%
\pgfsetbuttcap%
\pgfsetroundjoin%
\definecolor{currentfill}{rgb}{0.000000,0.000000,0.000000}%
\pgfsetfillcolor{currentfill}%
\pgfsetlinewidth{0.501875pt}%
\definecolor{currentstroke}{rgb}{0.000000,0.000000,0.000000}%
\pgfsetstrokecolor{currentstroke}%
\pgfsetdash{}{0pt}%
\pgfsys@defobject{currentmarker}{\pgfqpoint{-0.027778in}{0.000000in}}{\pgfqpoint{0.000000in}{0.000000in}}{%
\pgfpathmoveto{\pgfqpoint{0.000000in}{0.000000in}}%
\pgfpathlineto{\pgfqpoint{-0.027778in}{0.000000in}}%
\pgfusepath{stroke,fill}%
}%
\begin{pgfscope}%
\pgfsys@transformshift{7.200000in}{1.404494in}%
\pgfsys@useobject{currentmarker}{}%
\end{pgfscope}%
\end{pgfscope}%
\begin{pgfscope}%
\pgfsetbuttcap%
\pgfsetroundjoin%
\definecolor{currentfill}{rgb}{0.000000,0.000000,0.000000}%
\pgfsetfillcolor{currentfill}%
\pgfsetlinewidth{0.501875pt}%
\definecolor{currentstroke}{rgb}{0.000000,0.000000,0.000000}%
\pgfsetstrokecolor{currentstroke}%
\pgfsetdash{}{0pt}%
\pgfsys@defobject{currentmarker}{\pgfqpoint{0.000000in}{0.000000in}}{\pgfqpoint{0.027778in}{0.000000in}}{%
\pgfpathmoveto{\pgfqpoint{0.000000in}{0.000000in}}%
\pgfpathlineto{\pgfqpoint{0.027778in}{0.000000in}}%
\pgfusepath{stroke,fill}%
}%
\begin{pgfscope}%
\pgfsys@transformshift{1.000000in}{1.489018in}%
\pgfsys@useobject{currentmarker}{}%
\end{pgfscope}%
\end{pgfscope}%
\begin{pgfscope}%
\pgfsetbuttcap%
\pgfsetroundjoin%
\definecolor{currentfill}{rgb}{0.000000,0.000000,0.000000}%
\pgfsetfillcolor{currentfill}%
\pgfsetlinewidth{0.501875pt}%
\definecolor{currentstroke}{rgb}{0.000000,0.000000,0.000000}%
\pgfsetstrokecolor{currentstroke}%
\pgfsetdash{}{0pt}%
\pgfsys@defobject{currentmarker}{\pgfqpoint{-0.027778in}{0.000000in}}{\pgfqpoint{0.000000in}{0.000000in}}{%
\pgfpathmoveto{\pgfqpoint{0.000000in}{0.000000in}}%
\pgfpathlineto{\pgfqpoint{-0.027778in}{0.000000in}}%
\pgfusepath{stroke,fill}%
}%
\begin{pgfscope}%
\pgfsys@transformshift{7.200000in}{1.489018in}%
\pgfsys@useobject{currentmarker}{}%
\end{pgfscope}%
\end{pgfscope}%
\begin{pgfscope}%
\pgfsetbuttcap%
\pgfsetroundjoin%
\definecolor{currentfill}{rgb}{0.000000,0.000000,0.000000}%
\pgfsetfillcolor{currentfill}%
\pgfsetlinewidth{0.501875pt}%
\definecolor{currentstroke}{rgb}{0.000000,0.000000,0.000000}%
\pgfsetstrokecolor{currentstroke}%
\pgfsetdash{}{0pt}%
\pgfsys@defobject{currentmarker}{\pgfqpoint{0.000000in}{0.000000in}}{\pgfqpoint{0.027778in}{0.000000in}}{%
\pgfpathmoveto{\pgfqpoint{0.000000in}{0.000000in}}%
\pgfpathlineto{\pgfqpoint{0.027778in}{0.000000in}}%
\pgfusepath{stroke,fill}%
}%
\begin{pgfscope}%
\pgfsys@transformshift{1.000000in}{1.548989in}%
\pgfsys@useobject{currentmarker}{}%
\end{pgfscope}%
\end{pgfscope}%
\begin{pgfscope}%
\pgfsetbuttcap%
\pgfsetroundjoin%
\definecolor{currentfill}{rgb}{0.000000,0.000000,0.000000}%
\pgfsetfillcolor{currentfill}%
\pgfsetlinewidth{0.501875pt}%
\definecolor{currentstroke}{rgb}{0.000000,0.000000,0.000000}%
\pgfsetstrokecolor{currentstroke}%
\pgfsetdash{}{0pt}%
\pgfsys@defobject{currentmarker}{\pgfqpoint{-0.027778in}{0.000000in}}{\pgfqpoint{0.000000in}{0.000000in}}{%
\pgfpathmoveto{\pgfqpoint{0.000000in}{0.000000in}}%
\pgfpathlineto{\pgfqpoint{-0.027778in}{0.000000in}}%
\pgfusepath{stroke,fill}%
}%
\begin{pgfscope}%
\pgfsys@transformshift{7.200000in}{1.548989in}%
\pgfsys@useobject{currentmarker}{}%
\end{pgfscope}%
\end{pgfscope}%
\begin{pgfscope}%
\pgfsetbuttcap%
\pgfsetroundjoin%
\definecolor{currentfill}{rgb}{0.000000,0.000000,0.000000}%
\pgfsetfillcolor{currentfill}%
\pgfsetlinewidth{0.501875pt}%
\definecolor{currentstroke}{rgb}{0.000000,0.000000,0.000000}%
\pgfsetstrokecolor{currentstroke}%
\pgfsetdash{}{0pt}%
\pgfsys@defobject{currentmarker}{\pgfqpoint{0.000000in}{0.000000in}}{\pgfqpoint{0.027778in}{0.000000in}}{%
\pgfpathmoveto{\pgfqpoint{0.000000in}{0.000000in}}%
\pgfpathlineto{\pgfqpoint{0.027778in}{0.000000in}}%
\pgfusepath{stroke,fill}%
}%
\begin{pgfscope}%
\pgfsys@transformshift{1.000000in}{1.595506in}%
\pgfsys@useobject{currentmarker}{}%
\end{pgfscope}%
\end{pgfscope}%
\begin{pgfscope}%
\pgfsetbuttcap%
\pgfsetroundjoin%
\definecolor{currentfill}{rgb}{0.000000,0.000000,0.000000}%
\pgfsetfillcolor{currentfill}%
\pgfsetlinewidth{0.501875pt}%
\definecolor{currentstroke}{rgb}{0.000000,0.000000,0.000000}%
\pgfsetstrokecolor{currentstroke}%
\pgfsetdash{}{0pt}%
\pgfsys@defobject{currentmarker}{\pgfqpoint{-0.027778in}{0.000000in}}{\pgfqpoint{0.000000in}{0.000000in}}{%
\pgfpathmoveto{\pgfqpoint{0.000000in}{0.000000in}}%
\pgfpathlineto{\pgfqpoint{-0.027778in}{0.000000in}}%
\pgfusepath{stroke,fill}%
}%
\begin{pgfscope}%
\pgfsys@transformshift{7.200000in}{1.595506in}%
\pgfsys@useobject{currentmarker}{}%
\end{pgfscope}%
\end{pgfscope}%
\begin{pgfscope}%
\pgfsetbuttcap%
\pgfsetroundjoin%
\definecolor{currentfill}{rgb}{0.000000,0.000000,0.000000}%
\pgfsetfillcolor{currentfill}%
\pgfsetlinewidth{0.501875pt}%
\definecolor{currentstroke}{rgb}{0.000000,0.000000,0.000000}%
\pgfsetstrokecolor{currentstroke}%
\pgfsetdash{}{0pt}%
\pgfsys@defobject{currentmarker}{\pgfqpoint{0.000000in}{0.000000in}}{\pgfqpoint{0.027778in}{0.000000in}}{%
\pgfpathmoveto{\pgfqpoint{0.000000in}{0.000000in}}%
\pgfpathlineto{\pgfqpoint{0.027778in}{0.000000in}}%
\pgfusepath{stroke,fill}%
}%
\begin{pgfscope}%
\pgfsys@transformshift{1.000000in}{1.633513in}%
\pgfsys@useobject{currentmarker}{}%
\end{pgfscope}%
\end{pgfscope}%
\begin{pgfscope}%
\pgfsetbuttcap%
\pgfsetroundjoin%
\definecolor{currentfill}{rgb}{0.000000,0.000000,0.000000}%
\pgfsetfillcolor{currentfill}%
\pgfsetlinewidth{0.501875pt}%
\definecolor{currentstroke}{rgb}{0.000000,0.000000,0.000000}%
\pgfsetstrokecolor{currentstroke}%
\pgfsetdash{}{0pt}%
\pgfsys@defobject{currentmarker}{\pgfqpoint{-0.027778in}{0.000000in}}{\pgfqpoint{0.000000in}{0.000000in}}{%
\pgfpathmoveto{\pgfqpoint{0.000000in}{0.000000in}}%
\pgfpathlineto{\pgfqpoint{-0.027778in}{0.000000in}}%
\pgfusepath{stroke,fill}%
}%
\begin{pgfscope}%
\pgfsys@transformshift{7.200000in}{1.633513in}%
\pgfsys@useobject{currentmarker}{}%
\end{pgfscope}%
\end{pgfscope}%
\begin{pgfscope}%
\pgfsetbuttcap%
\pgfsetroundjoin%
\definecolor{currentfill}{rgb}{0.000000,0.000000,0.000000}%
\pgfsetfillcolor{currentfill}%
\pgfsetlinewidth{0.501875pt}%
\definecolor{currentstroke}{rgb}{0.000000,0.000000,0.000000}%
\pgfsetstrokecolor{currentstroke}%
\pgfsetdash{}{0pt}%
\pgfsys@defobject{currentmarker}{\pgfqpoint{0.000000in}{0.000000in}}{\pgfqpoint{0.027778in}{0.000000in}}{%
\pgfpathmoveto{\pgfqpoint{0.000000in}{0.000000in}}%
\pgfpathlineto{\pgfqpoint{0.027778in}{0.000000in}}%
\pgfusepath{stroke,fill}%
}%
\begin{pgfscope}%
\pgfsys@transformshift{1.000000in}{1.665647in}%
\pgfsys@useobject{currentmarker}{}%
\end{pgfscope}%
\end{pgfscope}%
\begin{pgfscope}%
\pgfsetbuttcap%
\pgfsetroundjoin%
\definecolor{currentfill}{rgb}{0.000000,0.000000,0.000000}%
\pgfsetfillcolor{currentfill}%
\pgfsetlinewidth{0.501875pt}%
\definecolor{currentstroke}{rgb}{0.000000,0.000000,0.000000}%
\pgfsetstrokecolor{currentstroke}%
\pgfsetdash{}{0pt}%
\pgfsys@defobject{currentmarker}{\pgfqpoint{-0.027778in}{0.000000in}}{\pgfqpoint{0.000000in}{0.000000in}}{%
\pgfpathmoveto{\pgfqpoint{0.000000in}{0.000000in}}%
\pgfpathlineto{\pgfqpoint{-0.027778in}{0.000000in}}%
\pgfusepath{stroke,fill}%
}%
\begin{pgfscope}%
\pgfsys@transformshift{7.200000in}{1.665647in}%
\pgfsys@useobject{currentmarker}{}%
\end{pgfscope}%
\end{pgfscope}%
\begin{pgfscope}%
\pgfsetbuttcap%
\pgfsetroundjoin%
\definecolor{currentfill}{rgb}{0.000000,0.000000,0.000000}%
\pgfsetfillcolor{currentfill}%
\pgfsetlinewidth{0.501875pt}%
\definecolor{currentstroke}{rgb}{0.000000,0.000000,0.000000}%
\pgfsetstrokecolor{currentstroke}%
\pgfsetdash{}{0pt}%
\pgfsys@defobject{currentmarker}{\pgfqpoint{0.000000in}{0.000000in}}{\pgfqpoint{0.027778in}{0.000000in}}{%
\pgfpathmoveto{\pgfqpoint{0.000000in}{0.000000in}}%
\pgfpathlineto{\pgfqpoint{0.027778in}{0.000000in}}%
\pgfusepath{stroke,fill}%
}%
\begin{pgfscope}%
\pgfsys@transformshift{1.000000in}{1.693483in}%
\pgfsys@useobject{currentmarker}{}%
\end{pgfscope}%
\end{pgfscope}%
\begin{pgfscope}%
\pgfsetbuttcap%
\pgfsetroundjoin%
\definecolor{currentfill}{rgb}{0.000000,0.000000,0.000000}%
\pgfsetfillcolor{currentfill}%
\pgfsetlinewidth{0.501875pt}%
\definecolor{currentstroke}{rgb}{0.000000,0.000000,0.000000}%
\pgfsetstrokecolor{currentstroke}%
\pgfsetdash{}{0pt}%
\pgfsys@defobject{currentmarker}{\pgfqpoint{-0.027778in}{0.000000in}}{\pgfqpoint{0.000000in}{0.000000in}}{%
\pgfpathmoveto{\pgfqpoint{0.000000in}{0.000000in}}%
\pgfpathlineto{\pgfqpoint{-0.027778in}{0.000000in}}%
\pgfusepath{stroke,fill}%
}%
\begin{pgfscope}%
\pgfsys@transformshift{7.200000in}{1.693483in}%
\pgfsys@useobject{currentmarker}{}%
\end{pgfscope}%
\end{pgfscope}%
\begin{pgfscope}%
\pgfsetbuttcap%
\pgfsetroundjoin%
\definecolor{currentfill}{rgb}{0.000000,0.000000,0.000000}%
\pgfsetfillcolor{currentfill}%
\pgfsetlinewidth{0.501875pt}%
\definecolor{currentstroke}{rgb}{0.000000,0.000000,0.000000}%
\pgfsetstrokecolor{currentstroke}%
\pgfsetdash{}{0pt}%
\pgfsys@defobject{currentmarker}{\pgfqpoint{0.000000in}{0.000000in}}{\pgfqpoint{0.027778in}{0.000000in}}{%
\pgfpathmoveto{\pgfqpoint{0.000000in}{0.000000in}}%
\pgfpathlineto{\pgfqpoint{0.027778in}{0.000000in}}%
\pgfusepath{stroke,fill}%
}%
\begin{pgfscope}%
\pgfsys@transformshift{1.000000in}{1.718036in}%
\pgfsys@useobject{currentmarker}{}%
\end{pgfscope}%
\end{pgfscope}%
\begin{pgfscope}%
\pgfsetbuttcap%
\pgfsetroundjoin%
\definecolor{currentfill}{rgb}{0.000000,0.000000,0.000000}%
\pgfsetfillcolor{currentfill}%
\pgfsetlinewidth{0.501875pt}%
\definecolor{currentstroke}{rgb}{0.000000,0.000000,0.000000}%
\pgfsetstrokecolor{currentstroke}%
\pgfsetdash{}{0pt}%
\pgfsys@defobject{currentmarker}{\pgfqpoint{-0.027778in}{0.000000in}}{\pgfqpoint{0.000000in}{0.000000in}}{%
\pgfpathmoveto{\pgfqpoint{0.000000in}{0.000000in}}%
\pgfpathlineto{\pgfqpoint{-0.027778in}{0.000000in}}%
\pgfusepath{stroke,fill}%
}%
\begin{pgfscope}%
\pgfsys@transformshift{7.200000in}{1.718036in}%
\pgfsys@useobject{currentmarker}{}%
\end{pgfscope}%
\end{pgfscope}%
\begin{pgfscope}%
\pgfsetbuttcap%
\pgfsetroundjoin%
\definecolor{currentfill}{rgb}{0.000000,0.000000,0.000000}%
\pgfsetfillcolor{currentfill}%
\pgfsetlinewidth{0.501875pt}%
\definecolor{currentstroke}{rgb}{0.000000,0.000000,0.000000}%
\pgfsetstrokecolor{currentstroke}%
\pgfsetdash{}{0pt}%
\pgfsys@defobject{currentmarker}{\pgfqpoint{0.000000in}{0.000000in}}{\pgfqpoint{0.027778in}{0.000000in}}{%
\pgfpathmoveto{\pgfqpoint{0.000000in}{0.000000in}}%
\pgfpathlineto{\pgfqpoint{0.027778in}{0.000000in}}%
\pgfusepath{stroke,fill}%
}%
\begin{pgfscope}%
\pgfsys@transformshift{1.000000in}{1.884494in}%
\pgfsys@useobject{currentmarker}{}%
\end{pgfscope}%
\end{pgfscope}%
\begin{pgfscope}%
\pgfsetbuttcap%
\pgfsetroundjoin%
\definecolor{currentfill}{rgb}{0.000000,0.000000,0.000000}%
\pgfsetfillcolor{currentfill}%
\pgfsetlinewidth{0.501875pt}%
\definecolor{currentstroke}{rgb}{0.000000,0.000000,0.000000}%
\pgfsetstrokecolor{currentstroke}%
\pgfsetdash{}{0pt}%
\pgfsys@defobject{currentmarker}{\pgfqpoint{-0.027778in}{0.000000in}}{\pgfqpoint{0.000000in}{0.000000in}}{%
\pgfpathmoveto{\pgfqpoint{0.000000in}{0.000000in}}%
\pgfpathlineto{\pgfqpoint{-0.027778in}{0.000000in}}%
\pgfusepath{stroke,fill}%
}%
\begin{pgfscope}%
\pgfsys@transformshift{7.200000in}{1.884494in}%
\pgfsys@useobject{currentmarker}{}%
\end{pgfscope}%
\end{pgfscope}%
\begin{pgfscope}%
\pgfsetbuttcap%
\pgfsetroundjoin%
\definecolor{currentfill}{rgb}{0.000000,0.000000,0.000000}%
\pgfsetfillcolor{currentfill}%
\pgfsetlinewidth{0.501875pt}%
\definecolor{currentstroke}{rgb}{0.000000,0.000000,0.000000}%
\pgfsetstrokecolor{currentstroke}%
\pgfsetdash{}{0pt}%
\pgfsys@defobject{currentmarker}{\pgfqpoint{0.000000in}{0.000000in}}{\pgfqpoint{0.027778in}{0.000000in}}{%
\pgfpathmoveto{\pgfqpoint{0.000000in}{0.000000in}}%
\pgfpathlineto{\pgfqpoint{0.027778in}{0.000000in}}%
\pgfusepath{stroke,fill}%
}%
\begin{pgfscope}%
\pgfsys@transformshift{1.000000in}{1.969018in}%
\pgfsys@useobject{currentmarker}{}%
\end{pgfscope}%
\end{pgfscope}%
\begin{pgfscope}%
\pgfsetbuttcap%
\pgfsetroundjoin%
\definecolor{currentfill}{rgb}{0.000000,0.000000,0.000000}%
\pgfsetfillcolor{currentfill}%
\pgfsetlinewidth{0.501875pt}%
\definecolor{currentstroke}{rgb}{0.000000,0.000000,0.000000}%
\pgfsetstrokecolor{currentstroke}%
\pgfsetdash{}{0pt}%
\pgfsys@defobject{currentmarker}{\pgfqpoint{-0.027778in}{0.000000in}}{\pgfqpoint{0.000000in}{0.000000in}}{%
\pgfpathmoveto{\pgfqpoint{0.000000in}{0.000000in}}%
\pgfpathlineto{\pgfqpoint{-0.027778in}{0.000000in}}%
\pgfusepath{stroke,fill}%
}%
\begin{pgfscope}%
\pgfsys@transformshift{7.200000in}{1.969018in}%
\pgfsys@useobject{currentmarker}{}%
\end{pgfscope}%
\end{pgfscope}%
\begin{pgfscope}%
\pgfsetbuttcap%
\pgfsetroundjoin%
\definecolor{currentfill}{rgb}{0.000000,0.000000,0.000000}%
\pgfsetfillcolor{currentfill}%
\pgfsetlinewidth{0.501875pt}%
\definecolor{currentstroke}{rgb}{0.000000,0.000000,0.000000}%
\pgfsetstrokecolor{currentstroke}%
\pgfsetdash{}{0pt}%
\pgfsys@defobject{currentmarker}{\pgfqpoint{0.000000in}{0.000000in}}{\pgfqpoint{0.027778in}{0.000000in}}{%
\pgfpathmoveto{\pgfqpoint{0.000000in}{0.000000in}}%
\pgfpathlineto{\pgfqpoint{0.027778in}{0.000000in}}%
\pgfusepath{stroke,fill}%
}%
\begin{pgfscope}%
\pgfsys@transformshift{1.000000in}{2.028989in}%
\pgfsys@useobject{currentmarker}{}%
\end{pgfscope}%
\end{pgfscope}%
\begin{pgfscope}%
\pgfsetbuttcap%
\pgfsetroundjoin%
\definecolor{currentfill}{rgb}{0.000000,0.000000,0.000000}%
\pgfsetfillcolor{currentfill}%
\pgfsetlinewidth{0.501875pt}%
\definecolor{currentstroke}{rgb}{0.000000,0.000000,0.000000}%
\pgfsetstrokecolor{currentstroke}%
\pgfsetdash{}{0pt}%
\pgfsys@defobject{currentmarker}{\pgfqpoint{-0.027778in}{0.000000in}}{\pgfqpoint{0.000000in}{0.000000in}}{%
\pgfpathmoveto{\pgfqpoint{0.000000in}{0.000000in}}%
\pgfpathlineto{\pgfqpoint{-0.027778in}{0.000000in}}%
\pgfusepath{stroke,fill}%
}%
\begin{pgfscope}%
\pgfsys@transformshift{7.200000in}{2.028989in}%
\pgfsys@useobject{currentmarker}{}%
\end{pgfscope}%
\end{pgfscope}%
\begin{pgfscope}%
\pgfsetbuttcap%
\pgfsetroundjoin%
\definecolor{currentfill}{rgb}{0.000000,0.000000,0.000000}%
\pgfsetfillcolor{currentfill}%
\pgfsetlinewidth{0.501875pt}%
\definecolor{currentstroke}{rgb}{0.000000,0.000000,0.000000}%
\pgfsetstrokecolor{currentstroke}%
\pgfsetdash{}{0pt}%
\pgfsys@defobject{currentmarker}{\pgfqpoint{0.000000in}{0.000000in}}{\pgfqpoint{0.027778in}{0.000000in}}{%
\pgfpathmoveto{\pgfqpoint{0.000000in}{0.000000in}}%
\pgfpathlineto{\pgfqpoint{0.027778in}{0.000000in}}%
\pgfusepath{stroke,fill}%
}%
\begin{pgfscope}%
\pgfsys@transformshift{1.000000in}{2.075506in}%
\pgfsys@useobject{currentmarker}{}%
\end{pgfscope}%
\end{pgfscope}%
\begin{pgfscope}%
\pgfsetbuttcap%
\pgfsetroundjoin%
\definecolor{currentfill}{rgb}{0.000000,0.000000,0.000000}%
\pgfsetfillcolor{currentfill}%
\pgfsetlinewidth{0.501875pt}%
\definecolor{currentstroke}{rgb}{0.000000,0.000000,0.000000}%
\pgfsetstrokecolor{currentstroke}%
\pgfsetdash{}{0pt}%
\pgfsys@defobject{currentmarker}{\pgfqpoint{-0.027778in}{0.000000in}}{\pgfqpoint{0.000000in}{0.000000in}}{%
\pgfpathmoveto{\pgfqpoint{0.000000in}{0.000000in}}%
\pgfpathlineto{\pgfqpoint{-0.027778in}{0.000000in}}%
\pgfusepath{stroke,fill}%
}%
\begin{pgfscope}%
\pgfsys@transformshift{7.200000in}{2.075506in}%
\pgfsys@useobject{currentmarker}{}%
\end{pgfscope}%
\end{pgfscope}%
\begin{pgfscope}%
\pgfsetbuttcap%
\pgfsetroundjoin%
\definecolor{currentfill}{rgb}{0.000000,0.000000,0.000000}%
\pgfsetfillcolor{currentfill}%
\pgfsetlinewidth{0.501875pt}%
\definecolor{currentstroke}{rgb}{0.000000,0.000000,0.000000}%
\pgfsetstrokecolor{currentstroke}%
\pgfsetdash{}{0pt}%
\pgfsys@defobject{currentmarker}{\pgfqpoint{0.000000in}{0.000000in}}{\pgfqpoint{0.027778in}{0.000000in}}{%
\pgfpathmoveto{\pgfqpoint{0.000000in}{0.000000in}}%
\pgfpathlineto{\pgfqpoint{0.027778in}{0.000000in}}%
\pgfusepath{stroke,fill}%
}%
\begin{pgfscope}%
\pgfsys@transformshift{1.000000in}{2.113513in}%
\pgfsys@useobject{currentmarker}{}%
\end{pgfscope}%
\end{pgfscope}%
\begin{pgfscope}%
\pgfsetbuttcap%
\pgfsetroundjoin%
\definecolor{currentfill}{rgb}{0.000000,0.000000,0.000000}%
\pgfsetfillcolor{currentfill}%
\pgfsetlinewidth{0.501875pt}%
\definecolor{currentstroke}{rgb}{0.000000,0.000000,0.000000}%
\pgfsetstrokecolor{currentstroke}%
\pgfsetdash{}{0pt}%
\pgfsys@defobject{currentmarker}{\pgfqpoint{-0.027778in}{0.000000in}}{\pgfqpoint{0.000000in}{0.000000in}}{%
\pgfpathmoveto{\pgfqpoint{0.000000in}{0.000000in}}%
\pgfpathlineto{\pgfqpoint{-0.027778in}{0.000000in}}%
\pgfusepath{stroke,fill}%
}%
\begin{pgfscope}%
\pgfsys@transformshift{7.200000in}{2.113513in}%
\pgfsys@useobject{currentmarker}{}%
\end{pgfscope}%
\end{pgfscope}%
\begin{pgfscope}%
\pgfsetbuttcap%
\pgfsetroundjoin%
\definecolor{currentfill}{rgb}{0.000000,0.000000,0.000000}%
\pgfsetfillcolor{currentfill}%
\pgfsetlinewidth{0.501875pt}%
\definecolor{currentstroke}{rgb}{0.000000,0.000000,0.000000}%
\pgfsetstrokecolor{currentstroke}%
\pgfsetdash{}{0pt}%
\pgfsys@defobject{currentmarker}{\pgfqpoint{0.000000in}{0.000000in}}{\pgfqpoint{0.027778in}{0.000000in}}{%
\pgfpathmoveto{\pgfqpoint{0.000000in}{0.000000in}}%
\pgfpathlineto{\pgfqpoint{0.027778in}{0.000000in}}%
\pgfusepath{stroke,fill}%
}%
\begin{pgfscope}%
\pgfsys@transformshift{1.000000in}{2.145647in}%
\pgfsys@useobject{currentmarker}{}%
\end{pgfscope}%
\end{pgfscope}%
\begin{pgfscope}%
\pgfsetbuttcap%
\pgfsetroundjoin%
\definecolor{currentfill}{rgb}{0.000000,0.000000,0.000000}%
\pgfsetfillcolor{currentfill}%
\pgfsetlinewidth{0.501875pt}%
\definecolor{currentstroke}{rgb}{0.000000,0.000000,0.000000}%
\pgfsetstrokecolor{currentstroke}%
\pgfsetdash{}{0pt}%
\pgfsys@defobject{currentmarker}{\pgfqpoint{-0.027778in}{0.000000in}}{\pgfqpoint{0.000000in}{0.000000in}}{%
\pgfpathmoveto{\pgfqpoint{0.000000in}{0.000000in}}%
\pgfpathlineto{\pgfqpoint{-0.027778in}{0.000000in}}%
\pgfusepath{stroke,fill}%
}%
\begin{pgfscope}%
\pgfsys@transformshift{7.200000in}{2.145647in}%
\pgfsys@useobject{currentmarker}{}%
\end{pgfscope}%
\end{pgfscope}%
\begin{pgfscope}%
\pgfsetbuttcap%
\pgfsetroundjoin%
\definecolor{currentfill}{rgb}{0.000000,0.000000,0.000000}%
\pgfsetfillcolor{currentfill}%
\pgfsetlinewidth{0.501875pt}%
\definecolor{currentstroke}{rgb}{0.000000,0.000000,0.000000}%
\pgfsetstrokecolor{currentstroke}%
\pgfsetdash{}{0pt}%
\pgfsys@defobject{currentmarker}{\pgfqpoint{0.000000in}{0.000000in}}{\pgfqpoint{0.027778in}{0.000000in}}{%
\pgfpathmoveto{\pgfqpoint{0.000000in}{0.000000in}}%
\pgfpathlineto{\pgfqpoint{0.027778in}{0.000000in}}%
\pgfusepath{stroke,fill}%
}%
\begin{pgfscope}%
\pgfsys@transformshift{1.000000in}{2.173483in}%
\pgfsys@useobject{currentmarker}{}%
\end{pgfscope}%
\end{pgfscope}%
\begin{pgfscope}%
\pgfsetbuttcap%
\pgfsetroundjoin%
\definecolor{currentfill}{rgb}{0.000000,0.000000,0.000000}%
\pgfsetfillcolor{currentfill}%
\pgfsetlinewidth{0.501875pt}%
\definecolor{currentstroke}{rgb}{0.000000,0.000000,0.000000}%
\pgfsetstrokecolor{currentstroke}%
\pgfsetdash{}{0pt}%
\pgfsys@defobject{currentmarker}{\pgfqpoint{-0.027778in}{0.000000in}}{\pgfqpoint{0.000000in}{0.000000in}}{%
\pgfpathmoveto{\pgfqpoint{0.000000in}{0.000000in}}%
\pgfpathlineto{\pgfqpoint{-0.027778in}{0.000000in}}%
\pgfusepath{stroke,fill}%
}%
\begin{pgfscope}%
\pgfsys@transformshift{7.200000in}{2.173483in}%
\pgfsys@useobject{currentmarker}{}%
\end{pgfscope}%
\end{pgfscope}%
\begin{pgfscope}%
\pgfsetbuttcap%
\pgfsetroundjoin%
\definecolor{currentfill}{rgb}{0.000000,0.000000,0.000000}%
\pgfsetfillcolor{currentfill}%
\pgfsetlinewidth{0.501875pt}%
\definecolor{currentstroke}{rgb}{0.000000,0.000000,0.000000}%
\pgfsetstrokecolor{currentstroke}%
\pgfsetdash{}{0pt}%
\pgfsys@defobject{currentmarker}{\pgfqpoint{0.000000in}{0.000000in}}{\pgfqpoint{0.027778in}{0.000000in}}{%
\pgfpathmoveto{\pgfqpoint{0.000000in}{0.000000in}}%
\pgfpathlineto{\pgfqpoint{0.027778in}{0.000000in}}%
\pgfusepath{stroke,fill}%
}%
\begin{pgfscope}%
\pgfsys@transformshift{1.000000in}{2.198036in}%
\pgfsys@useobject{currentmarker}{}%
\end{pgfscope}%
\end{pgfscope}%
\begin{pgfscope}%
\pgfsetbuttcap%
\pgfsetroundjoin%
\definecolor{currentfill}{rgb}{0.000000,0.000000,0.000000}%
\pgfsetfillcolor{currentfill}%
\pgfsetlinewidth{0.501875pt}%
\definecolor{currentstroke}{rgb}{0.000000,0.000000,0.000000}%
\pgfsetstrokecolor{currentstroke}%
\pgfsetdash{}{0pt}%
\pgfsys@defobject{currentmarker}{\pgfqpoint{-0.027778in}{0.000000in}}{\pgfqpoint{0.000000in}{0.000000in}}{%
\pgfpathmoveto{\pgfqpoint{0.000000in}{0.000000in}}%
\pgfpathlineto{\pgfqpoint{-0.027778in}{0.000000in}}%
\pgfusepath{stroke,fill}%
}%
\begin{pgfscope}%
\pgfsys@transformshift{7.200000in}{2.198036in}%
\pgfsys@useobject{currentmarker}{}%
\end{pgfscope}%
\end{pgfscope}%
\begin{pgfscope}%
\pgfsetbuttcap%
\pgfsetroundjoin%
\definecolor{currentfill}{rgb}{0.000000,0.000000,0.000000}%
\pgfsetfillcolor{currentfill}%
\pgfsetlinewidth{0.501875pt}%
\definecolor{currentstroke}{rgb}{0.000000,0.000000,0.000000}%
\pgfsetstrokecolor{currentstroke}%
\pgfsetdash{}{0pt}%
\pgfsys@defobject{currentmarker}{\pgfqpoint{0.000000in}{0.000000in}}{\pgfqpoint{0.027778in}{0.000000in}}{%
\pgfpathmoveto{\pgfqpoint{0.000000in}{0.000000in}}%
\pgfpathlineto{\pgfqpoint{0.027778in}{0.000000in}}%
\pgfusepath{stroke,fill}%
}%
\begin{pgfscope}%
\pgfsys@transformshift{1.000000in}{2.364494in}%
\pgfsys@useobject{currentmarker}{}%
\end{pgfscope}%
\end{pgfscope}%
\begin{pgfscope}%
\pgfsetbuttcap%
\pgfsetroundjoin%
\definecolor{currentfill}{rgb}{0.000000,0.000000,0.000000}%
\pgfsetfillcolor{currentfill}%
\pgfsetlinewidth{0.501875pt}%
\definecolor{currentstroke}{rgb}{0.000000,0.000000,0.000000}%
\pgfsetstrokecolor{currentstroke}%
\pgfsetdash{}{0pt}%
\pgfsys@defobject{currentmarker}{\pgfqpoint{-0.027778in}{0.000000in}}{\pgfqpoint{0.000000in}{0.000000in}}{%
\pgfpathmoveto{\pgfqpoint{0.000000in}{0.000000in}}%
\pgfpathlineto{\pgfqpoint{-0.027778in}{0.000000in}}%
\pgfusepath{stroke,fill}%
}%
\begin{pgfscope}%
\pgfsys@transformshift{7.200000in}{2.364494in}%
\pgfsys@useobject{currentmarker}{}%
\end{pgfscope}%
\end{pgfscope}%
\begin{pgfscope}%
\pgfsetbuttcap%
\pgfsetroundjoin%
\definecolor{currentfill}{rgb}{0.000000,0.000000,0.000000}%
\pgfsetfillcolor{currentfill}%
\pgfsetlinewidth{0.501875pt}%
\definecolor{currentstroke}{rgb}{0.000000,0.000000,0.000000}%
\pgfsetstrokecolor{currentstroke}%
\pgfsetdash{}{0pt}%
\pgfsys@defobject{currentmarker}{\pgfqpoint{0.000000in}{0.000000in}}{\pgfqpoint{0.027778in}{0.000000in}}{%
\pgfpathmoveto{\pgfqpoint{0.000000in}{0.000000in}}%
\pgfpathlineto{\pgfqpoint{0.027778in}{0.000000in}}%
\pgfusepath{stroke,fill}%
}%
\begin{pgfscope}%
\pgfsys@transformshift{1.000000in}{2.449018in}%
\pgfsys@useobject{currentmarker}{}%
\end{pgfscope}%
\end{pgfscope}%
\begin{pgfscope}%
\pgfsetbuttcap%
\pgfsetroundjoin%
\definecolor{currentfill}{rgb}{0.000000,0.000000,0.000000}%
\pgfsetfillcolor{currentfill}%
\pgfsetlinewidth{0.501875pt}%
\definecolor{currentstroke}{rgb}{0.000000,0.000000,0.000000}%
\pgfsetstrokecolor{currentstroke}%
\pgfsetdash{}{0pt}%
\pgfsys@defobject{currentmarker}{\pgfqpoint{-0.027778in}{0.000000in}}{\pgfqpoint{0.000000in}{0.000000in}}{%
\pgfpathmoveto{\pgfqpoint{0.000000in}{0.000000in}}%
\pgfpathlineto{\pgfqpoint{-0.027778in}{0.000000in}}%
\pgfusepath{stroke,fill}%
}%
\begin{pgfscope}%
\pgfsys@transformshift{7.200000in}{2.449018in}%
\pgfsys@useobject{currentmarker}{}%
\end{pgfscope}%
\end{pgfscope}%
\begin{pgfscope}%
\pgfsetbuttcap%
\pgfsetroundjoin%
\definecolor{currentfill}{rgb}{0.000000,0.000000,0.000000}%
\pgfsetfillcolor{currentfill}%
\pgfsetlinewidth{0.501875pt}%
\definecolor{currentstroke}{rgb}{0.000000,0.000000,0.000000}%
\pgfsetstrokecolor{currentstroke}%
\pgfsetdash{}{0pt}%
\pgfsys@defobject{currentmarker}{\pgfqpoint{0.000000in}{0.000000in}}{\pgfqpoint{0.027778in}{0.000000in}}{%
\pgfpathmoveto{\pgfqpoint{0.000000in}{0.000000in}}%
\pgfpathlineto{\pgfqpoint{0.027778in}{0.000000in}}%
\pgfusepath{stroke,fill}%
}%
\begin{pgfscope}%
\pgfsys@transformshift{1.000000in}{2.508989in}%
\pgfsys@useobject{currentmarker}{}%
\end{pgfscope}%
\end{pgfscope}%
\begin{pgfscope}%
\pgfsetbuttcap%
\pgfsetroundjoin%
\definecolor{currentfill}{rgb}{0.000000,0.000000,0.000000}%
\pgfsetfillcolor{currentfill}%
\pgfsetlinewidth{0.501875pt}%
\definecolor{currentstroke}{rgb}{0.000000,0.000000,0.000000}%
\pgfsetstrokecolor{currentstroke}%
\pgfsetdash{}{0pt}%
\pgfsys@defobject{currentmarker}{\pgfqpoint{-0.027778in}{0.000000in}}{\pgfqpoint{0.000000in}{0.000000in}}{%
\pgfpathmoveto{\pgfqpoint{0.000000in}{0.000000in}}%
\pgfpathlineto{\pgfqpoint{-0.027778in}{0.000000in}}%
\pgfusepath{stroke,fill}%
}%
\begin{pgfscope}%
\pgfsys@transformshift{7.200000in}{2.508989in}%
\pgfsys@useobject{currentmarker}{}%
\end{pgfscope}%
\end{pgfscope}%
\begin{pgfscope}%
\pgfsetbuttcap%
\pgfsetroundjoin%
\definecolor{currentfill}{rgb}{0.000000,0.000000,0.000000}%
\pgfsetfillcolor{currentfill}%
\pgfsetlinewidth{0.501875pt}%
\definecolor{currentstroke}{rgb}{0.000000,0.000000,0.000000}%
\pgfsetstrokecolor{currentstroke}%
\pgfsetdash{}{0pt}%
\pgfsys@defobject{currentmarker}{\pgfqpoint{0.000000in}{0.000000in}}{\pgfqpoint{0.027778in}{0.000000in}}{%
\pgfpathmoveto{\pgfqpoint{0.000000in}{0.000000in}}%
\pgfpathlineto{\pgfqpoint{0.027778in}{0.000000in}}%
\pgfusepath{stroke,fill}%
}%
\begin{pgfscope}%
\pgfsys@transformshift{1.000000in}{2.555506in}%
\pgfsys@useobject{currentmarker}{}%
\end{pgfscope}%
\end{pgfscope}%
\begin{pgfscope}%
\pgfsetbuttcap%
\pgfsetroundjoin%
\definecolor{currentfill}{rgb}{0.000000,0.000000,0.000000}%
\pgfsetfillcolor{currentfill}%
\pgfsetlinewidth{0.501875pt}%
\definecolor{currentstroke}{rgb}{0.000000,0.000000,0.000000}%
\pgfsetstrokecolor{currentstroke}%
\pgfsetdash{}{0pt}%
\pgfsys@defobject{currentmarker}{\pgfqpoint{-0.027778in}{0.000000in}}{\pgfqpoint{0.000000in}{0.000000in}}{%
\pgfpathmoveto{\pgfqpoint{0.000000in}{0.000000in}}%
\pgfpathlineto{\pgfqpoint{-0.027778in}{0.000000in}}%
\pgfusepath{stroke,fill}%
}%
\begin{pgfscope}%
\pgfsys@transformshift{7.200000in}{2.555506in}%
\pgfsys@useobject{currentmarker}{}%
\end{pgfscope}%
\end{pgfscope}%
\begin{pgfscope}%
\pgfsetbuttcap%
\pgfsetroundjoin%
\definecolor{currentfill}{rgb}{0.000000,0.000000,0.000000}%
\pgfsetfillcolor{currentfill}%
\pgfsetlinewidth{0.501875pt}%
\definecolor{currentstroke}{rgb}{0.000000,0.000000,0.000000}%
\pgfsetstrokecolor{currentstroke}%
\pgfsetdash{}{0pt}%
\pgfsys@defobject{currentmarker}{\pgfqpoint{0.000000in}{0.000000in}}{\pgfqpoint{0.027778in}{0.000000in}}{%
\pgfpathmoveto{\pgfqpoint{0.000000in}{0.000000in}}%
\pgfpathlineto{\pgfqpoint{0.027778in}{0.000000in}}%
\pgfusepath{stroke,fill}%
}%
\begin{pgfscope}%
\pgfsys@transformshift{1.000000in}{2.593513in}%
\pgfsys@useobject{currentmarker}{}%
\end{pgfscope}%
\end{pgfscope}%
\begin{pgfscope}%
\pgfsetbuttcap%
\pgfsetroundjoin%
\definecolor{currentfill}{rgb}{0.000000,0.000000,0.000000}%
\pgfsetfillcolor{currentfill}%
\pgfsetlinewidth{0.501875pt}%
\definecolor{currentstroke}{rgb}{0.000000,0.000000,0.000000}%
\pgfsetstrokecolor{currentstroke}%
\pgfsetdash{}{0pt}%
\pgfsys@defobject{currentmarker}{\pgfqpoint{-0.027778in}{0.000000in}}{\pgfqpoint{0.000000in}{0.000000in}}{%
\pgfpathmoveto{\pgfqpoint{0.000000in}{0.000000in}}%
\pgfpathlineto{\pgfqpoint{-0.027778in}{0.000000in}}%
\pgfusepath{stroke,fill}%
}%
\begin{pgfscope}%
\pgfsys@transformshift{7.200000in}{2.593513in}%
\pgfsys@useobject{currentmarker}{}%
\end{pgfscope}%
\end{pgfscope}%
\begin{pgfscope}%
\pgfsetbuttcap%
\pgfsetroundjoin%
\definecolor{currentfill}{rgb}{0.000000,0.000000,0.000000}%
\pgfsetfillcolor{currentfill}%
\pgfsetlinewidth{0.501875pt}%
\definecolor{currentstroke}{rgb}{0.000000,0.000000,0.000000}%
\pgfsetstrokecolor{currentstroke}%
\pgfsetdash{}{0pt}%
\pgfsys@defobject{currentmarker}{\pgfqpoint{0.000000in}{0.000000in}}{\pgfqpoint{0.027778in}{0.000000in}}{%
\pgfpathmoveto{\pgfqpoint{0.000000in}{0.000000in}}%
\pgfpathlineto{\pgfqpoint{0.027778in}{0.000000in}}%
\pgfusepath{stroke,fill}%
}%
\begin{pgfscope}%
\pgfsys@transformshift{1.000000in}{2.625647in}%
\pgfsys@useobject{currentmarker}{}%
\end{pgfscope}%
\end{pgfscope}%
\begin{pgfscope}%
\pgfsetbuttcap%
\pgfsetroundjoin%
\definecolor{currentfill}{rgb}{0.000000,0.000000,0.000000}%
\pgfsetfillcolor{currentfill}%
\pgfsetlinewidth{0.501875pt}%
\definecolor{currentstroke}{rgb}{0.000000,0.000000,0.000000}%
\pgfsetstrokecolor{currentstroke}%
\pgfsetdash{}{0pt}%
\pgfsys@defobject{currentmarker}{\pgfqpoint{-0.027778in}{0.000000in}}{\pgfqpoint{0.000000in}{0.000000in}}{%
\pgfpathmoveto{\pgfqpoint{0.000000in}{0.000000in}}%
\pgfpathlineto{\pgfqpoint{-0.027778in}{0.000000in}}%
\pgfusepath{stroke,fill}%
}%
\begin{pgfscope}%
\pgfsys@transformshift{7.200000in}{2.625647in}%
\pgfsys@useobject{currentmarker}{}%
\end{pgfscope}%
\end{pgfscope}%
\begin{pgfscope}%
\pgfsetbuttcap%
\pgfsetroundjoin%
\definecolor{currentfill}{rgb}{0.000000,0.000000,0.000000}%
\pgfsetfillcolor{currentfill}%
\pgfsetlinewidth{0.501875pt}%
\definecolor{currentstroke}{rgb}{0.000000,0.000000,0.000000}%
\pgfsetstrokecolor{currentstroke}%
\pgfsetdash{}{0pt}%
\pgfsys@defobject{currentmarker}{\pgfqpoint{0.000000in}{0.000000in}}{\pgfqpoint{0.027778in}{0.000000in}}{%
\pgfpathmoveto{\pgfqpoint{0.000000in}{0.000000in}}%
\pgfpathlineto{\pgfqpoint{0.027778in}{0.000000in}}%
\pgfusepath{stroke,fill}%
}%
\begin{pgfscope}%
\pgfsys@transformshift{1.000000in}{2.653483in}%
\pgfsys@useobject{currentmarker}{}%
\end{pgfscope}%
\end{pgfscope}%
\begin{pgfscope}%
\pgfsetbuttcap%
\pgfsetroundjoin%
\definecolor{currentfill}{rgb}{0.000000,0.000000,0.000000}%
\pgfsetfillcolor{currentfill}%
\pgfsetlinewidth{0.501875pt}%
\definecolor{currentstroke}{rgb}{0.000000,0.000000,0.000000}%
\pgfsetstrokecolor{currentstroke}%
\pgfsetdash{}{0pt}%
\pgfsys@defobject{currentmarker}{\pgfqpoint{-0.027778in}{0.000000in}}{\pgfqpoint{0.000000in}{0.000000in}}{%
\pgfpathmoveto{\pgfqpoint{0.000000in}{0.000000in}}%
\pgfpathlineto{\pgfqpoint{-0.027778in}{0.000000in}}%
\pgfusepath{stroke,fill}%
}%
\begin{pgfscope}%
\pgfsys@transformshift{7.200000in}{2.653483in}%
\pgfsys@useobject{currentmarker}{}%
\end{pgfscope}%
\end{pgfscope}%
\begin{pgfscope}%
\pgfsetbuttcap%
\pgfsetroundjoin%
\definecolor{currentfill}{rgb}{0.000000,0.000000,0.000000}%
\pgfsetfillcolor{currentfill}%
\pgfsetlinewidth{0.501875pt}%
\definecolor{currentstroke}{rgb}{0.000000,0.000000,0.000000}%
\pgfsetstrokecolor{currentstroke}%
\pgfsetdash{}{0pt}%
\pgfsys@defobject{currentmarker}{\pgfqpoint{0.000000in}{0.000000in}}{\pgfqpoint{0.027778in}{0.000000in}}{%
\pgfpathmoveto{\pgfqpoint{0.000000in}{0.000000in}}%
\pgfpathlineto{\pgfqpoint{0.027778in}{0.000000in}}%
\pgfusepath{stroke,fill}%
}%
\begin{pgfscope}%
\pgfsys@transformshift{1.000000in}{2.678036in}%
\pgfsys@useobject{currentmarker}{}%
\end{pgfscope}%
\end{pgfscope}%
\begin{pgfscope}%
\pgfsetbuttcap%
\pgfsetroundjoin%
\definecolor{currentfill}{rgb}{0.000000,0.000000,0.000000}%
\pgfsetfillcolor{currentfill}%
\pgfsetlinewidth{0.501875pt}%
\definecolor{currentstroke}{rgb}{0.000000,0.000000,0.000000}%
\pgfsetstrokecolor{currentstroke}%
\pgfsetdash{}{0pt}%
\pgfsys@defobject{currentmarker}{\pgfqpoint{-0.027778in}{0.000000in}}{\pgfqpoint{0.000000in}{0.000000in}}{%
\pgfpathmoveto{\pgfqpoint{0.000000in}{0.000000in}}%
\pgfpathlineto{\pgfqpoint{-0.027778in}{0.000000in}}%
\pgfusepath{stroke,fill}%
}%
\begin{pgfscope}%
\pgfsys@transformshift{7.200000in}{2.678036in}%
\pgfsys@useobject{currentmarker}{}%
\end{pgfscope}%
\end{pgfscope}%
\begin{pgfscope}%
\pgftext[left,bottom,x=0.554012in,y=0.106703in,rotate=90.000000]{{\sffamily\fontsize{12.000000}{14.400000}\selectfont mean square displacement [nm\(\displaystyle ^2\)]}}
%
\end{pgfscope}%
\begin{pgfscope}%
\pgfsetrectcap%
\pgfsetroundjoin%
\pgfsetlinewidth{1.003750pt}%
\definecolor{currentstroke}{rgb}{0.000000,0.000000,0.000000}%
\pgfsetstrokecolor{currentstroke}%
\pgfsetdash{}{0pt}%
\pgfpathmoveto{\pgfqpoint{1.000000in}{2.700000in}}%
\pgfpathlineto{\pgfqpoint{7.200000in}{2.700000in}}%
\pgfusepath{stroke}%
\end{pgfscope}%
\begin{pgfscope}%
\pgfsetrectcap%
\pgfsetroundjoin%
\pgfsetlinewidth{1.003750pt}%
\definecolor{currentstroke}{rgb}{0.000000,0.000000,0.000000}%
\pgfsetstrokecolor{currentstroke}%
\pgfsetdash{}{0pt}%
\pgfpathmoveto{\pgfqpoint{7.200000in}{0.300000in}}%
\pgfpathlineto{\pgfqpoint{7.200000in}{2.700000in}}%
\pgfusepath{stroke}%
\end{pgfscope}%
\begin{pgfscope}%
\pgfsetrectcap%
\pgfsetroundjoin%
\pgfsetlinewidth{1.003750pt}%
\definecolor{currentstroke}{rgb}{0.000000,0.000000,0.000000}%
\pgfsetstrokecolor{currentstroke}%
\pgfsetdash{}{0pt}%
\pgfpathmoveto{\pgfqpoint{1.000000in}{0.300000in}}%
\pgfpathlineto{\pgfqpoint{7.200000in}{0.300000in}}%
\pgfusepath{stroke}%
\end{pgfscope}%
\begin{pgfscope}%
\pgfsetrectcap%
\pgfsetroundjoin%
\pgfsetlinewidth{1.003750pt}%
\definecolor{currentstroke}{rgb}{0.000000,0.000000,0.000000}%
\pgfsetstrokecolor{currentstroke}%
\pgfsetdash{}{0pt}%
\pgfpathmoveto{\pgfqpoint{1.000000in}{0.300000in}}%
\pgfpathlineto{\pgfqpoint{1.000000in}{2.700000in}}%
\pgfusepath{stroke}%
\end{pgfscope}%
\begin{pgfscope}%
\pgfsetrectcap%
\pgfsetroundjoin%
\definecolor{currentfill}{rgb}{1.000000,1.000000,1.000000}%
\pgfsetfillcolor{currentfill}%
\pgfsetlinewidth{1.003750pt}%
\definecolor{currentstroke}{rgb}{0.000000,0.000000,0.000000}%
\pgfsetstrokecolor{currentstroke}%
\pgfsetdash{}{0pt}%
\pgfpathmoveto{\pgfqpoint{1.069417in}{1.977606in}}%
\pgfpathlineto{\pgfqpoint{1.926808in}{1.977606in}}%
\pgfpathlineto{\pgfqpoint{1.926808in}{2.630583in}}%
\pgfpathlineto{\pgfqpoint{1.069417in}{2.630583in}}%
\pgfpathlineto{\pgfqpoint{1.069417in}{1.977606in}}%
\pgfpathclose%
\pgfusepath{stroke,fill}%
\end{pgfscope}%
\begin{pgfscope}%
\pgfsetrectcap%
\pgfsetroundjoin%
\pgfsetlinewidth{1.003750pt}%
\definecolor{currentstroke}{rgb}{0.000000,0.000000,1.000000}%
\pgfsetstrokecolor{currentstroke}%
\pgfsetdash{}{0pt}%
\pgfpathmoveto{\pgfqpoint{1.166600in}{2.518161in}}%
\pgfpathlineto{\pgfqpoint{1.360967in}{2.518161in}}%
\pgfusepath{stroke}%
\end{pgfscope}%
\begin{pgfscope}%
\pgftext[left,bottom,x=1.513683in,y=2.440691in,rotate=0.000000]{{\sffamily\fontsize{9.996000}{11.995200}\selectfont spc}}
%
\end{pgfscope}%
\begin{pgfscope}%
\pgfsetrectcap%
\pgfsetroundjoin%
\pgfsetlinewidth{1.003750pt}%
\definecolor{currentstroke}{rgb}{0.000000,0.500000,0.000000}%
\pgfsetstrokecolor{currentstroke}%
\pgfsetdash{}{0pt}%
\pgfpathmoveto{\pgfqpoint{1.166600in}{2.314385in}}%
\pgfpathlineto{\pgfqpoint{1.360967in}{2.314385in}}%
\pgfusepath{stroke}%
\end{pgfscope}%
\begin{pgfscope}%
\pgftext[left,bottom,x=1.513683in,y=2.236915in,rotate=0.000000]{{\sffamily\fontsize{9.996000}{11.995200}\selectfont spce}}
%
\end{pgfscope}%
\begin{pgfscope}%
\pgfsetrectcap%
\pgfsetroundjoin%
\pgfsetlinewidth{1.003750pt}%
\definecolor{currentstroke}{rgb}{1.000000,0.000000,0.000000}%
\pgfsetstrokecolor{currentstroke}%
\pgfsetdash{}{0pt}%
\pgfpathmoveto{\pgfqpoint{1.166600in}{2.110609in}}%
\pgfpathlineto{\pgfqpoint{1.360967in}{2.110609in}}%
\pgfusepath{stroke}%
\end{pgfscope}%
\begin{pgfscope}%
\pgftext[left,bottom,x=1.513683in,y=2.033139in,rotate=0.000000]{{\sffamily\fontsize{9.996000}{11.995200}\selectfont tip3p}}
%
\end{pgfscope}%
\end{pgfpicture}%
\makeatother%
\endgroup%
}
    		\caption{LONG}
		\end{subfigure}
		\begin{subfigure}[a]{\textwidth}
			\resizebox{\linewidth}{!}{%% Creator: Matplotlib, PGF backend
%%
%% To include the figure in your LaTeX document, write
%%   \input{<filename>.pgf}
%%
%% Make sure the required packages are loaded in your preamble
%%   \usepackage{pgf}
%%
%% Figures using additional raster images can only be included by \input if
%% they are in the same directory as the main LaTeX file. For loading figures
%% from other directories you can use the `import` package
%%   \usepackage{import}
%% and then include the figures with
%%   \import{<path to file>}{<filename>.pgf}
%%
%% Matplotlib used the following preamble
%%   \usepackage{fontspec}
%%   \setmainfont{DejaVu Serif}
%%   \setsansfont{DejaVu Sans}
%%   \setmonofont{DejaVu Sans Mono}
%%
\begingroup%
\makeatletter%
\begin{pgfpicture}%
\pgfpathrectangle{\pgfpointorigin}{\pgfqpoint{8.000000in}{3.500000in}}%
\pgfusepath{use as bounding box}%
\begin{pgfscope}%
\pgfsetrectcap%
\pgfsetroundjoin%
\definecolor{currentfill}{rgb}{1.000000,1.000000,1.000000}%
\pgfsetfillcolor{currentfill}%
\pgfsetlinewidth{0.000000pt}%
\definecolor{currentstroke}{rgb}{1.000000,1.000000,1.000000}%
\pgfsetstrokecolor{currentstroke}%
\pgfsetdash{}{0pt}%
\pgfpathmoveto{\pgfqpoint{0.000000in}{0.000000in}}%
\pgfpathlineto{\pgfqpoint{8.000000in}{0.000000in}}%
\pgfpathlineto{\pgfqpoint{8.000000in}{3.500000in}}%
\pgfpathlineto{\pgfqpoint{0.000000in}{3.500000in}}%
\pgfpathclose%
\pgfusepath{fill}%
\end{pgfscope}%
\begin{pgfscope}%
\pgfsetrectcap%
\pgfsetroundjoin%
\definecolor{currentfill}{rgb}{1.000000,1.000000,1.000000}%
\pgfsetfillcolor{currentfill}%
\pgfsetlinewidth{0.000000pt}%
\definecolor{currentstroke}{rgb}{0.000000,0.000000,0.000000}%
\pgfsetstrokecolor{currentstroke}%
\pgfsetdash{}{0pt}%
\pgfpathmoveto{\pgfqpoint{1.000000in}{0.350000in}}%
\pgfpathlineto{\pgfqpoint{7.200000in}{0.350000in}}%
\pgfpathlineto{\pgfqpoint{7.200000in}{3.150000in}}%
\pgfpathlineto{\pgfqpoint{1.000000in}{3.150000in}}%
\pgfpathclose%
\pgfusepath{fill}%
\end{pgfscope}%
\begin{pgfscope}%
\pgfpathrectangle{\pgfqpoint{1.000000in}{0.350000in}}{\pgfqpoint{6.200000in}{2.800000in}} %
\pgfusepath{clip}%
\pgfsetrectcap%
\pgfsetroundjoin%
\pgfsetlinewidth{1.003750pt}%
\definecolor{currentstroke}{rgb}{0.000000,0.000000,1.000000}%
\pgfsetstrokecolor{currentstroke}%
\pgfsetdash{}{0pt}%
\pgfpathmoveto{\pgfqpoint{1.001240in}{1.693679in}}%
\pgfpathlineto{\pgfqpoint{1.002480in}{2.293444in}}%
\pgfpathlineto{\pgfqpoint{1.003720in}{2.509775in}}%
\pgfpathlineto{\pgfqpoint{1.004960in}{2.524485in}}%
\pgfpathlineto{\pgfqpoint{1.009920in}{2.188602in}}%
\pgfpathlineto{\pgfqpoint{1.019840in}{1.847099in}}%
\pgfpathlineto{\pgfqpoint{1.021080in}{1.840417in}}%
\pgfpathlineto{\pgfqpoint{1.023560in}{1.807506in}}%
\pgfpathlineto{\pgfqpoint{1.026040in}{1.781259in}}%
\pgfpathlineto{\pgfqpoint{1.027280in}{1.766197in}}%
\pgfpathlineto{\pgfqpoint{1.028520in}{1.765199in}}%
\pgfpathlineto{\pgfqpoint{1.029760in}{1.760003in}}%
\pgfpathlineto{\pgfqpoint{1.043400in}{1.648181in}}%
\pgfpathlineto{\pgfqpoint{1.044640in}{1.648676in}}%
\pgfpathlineto{\pgfqpoint{1.047120in}{1.632141in}}%
\pgfpathlineto{\pgfqpoint{1.049600in}{1.632737in}}%
\pgfpathlineto{\pgfqpoint{1.050840in}{1.622928in}}%
\pgfpathlineto{\pgfqpoint{1.054560in}{1.629162in}}%
\pgfpathlineto{\pgfqpoint{1.058280in}{1.618009in}}%
\pgfpathlineto{\pgfqpoint{1.059520in}{1.619420in}}%
\pgfpathlineto{\pgfqpoint{1.060760in}{1.611418in}}%
\pgfpathlineto{\pgfqpoint{1.063240in}{1.612318in}}%
\pgfpathlineto{\pgfqpoint{1.064480in}{1.613311in}}%
\pgfpathlineto{\pgfqpoint{1.068200in}{1.601138in}}%
\pgfpathlineto{\pgfqpoint{1.069440in}{1.602630in}}%
\pgfpathlineto{\pgfqpoint{1.070680in}{1.601795in}}%
\pgfpathlineto{\pgfqpoint{1.073160in}{1.609824in}}%
\pgfpathlineto{\pgfqpoint{1.075640in}{1.608793in}}%
\pgfpathlineto{\pgfqpoint{1.076880in}{1.609381in}}%
\pgfpathlineto{\pgfqpoint{1.081840in}{1.602175in}}%
\pgfpathlineto{\pgfqpoint{1.083080in}{1.601224in}}%
\pgfpathlineto{\pgfqpoint{1.086800in}{1.591837in}}%
\pgfpathlineto{\pgfqpoint{1.090520in}{1.583108in}}%
\pgfpathlineto{\pgfqpoint{1.091760in}{1.582659in}}%
\pgfpathlineto{\pgfqpoint{1.093000in}{1.584019in}}%
\pgfpathlineto{\pgfqpoint{1.096720in}{1.572188in}}%
\pgfpathlineto{\pgfqpoint{1.099200in}{1.574119in}}%
\pgfpathlineto{\pgfqpoint{1.100440in}{1.573435in}}%
\pgfpathlineto{\pgfqpoint{1.101680in}{1.574096in}}%
\pgfpathlineto{\pgfqpoint{1.107880in}{1.548874in}}%
\pgfpathlineto{\pgfqpoint{1.109120in}{1.550960in}}%
\pgfpathlineto{\pgfqpoint{1.110360in}{1.550784in}}%
\pgfpathlineto{\pgfqpoint{1.111600in}{1.549114in}}%
\pgfpathlineto{\pgfqpoint{1.114080in}{1.544892in}}%
\pgfpathlineto{\pgfqpoint{1.116560in}{1.546476in}}%
\pgfpathlineto{\pgfqpoint{1.117800in}{1.545955in}}%
\pgfpathlineto{\pgfqpoint{1.120280in}{1.537643in}}%
\pgfpathlineto{\pgfqpoint{1.121520in}{1.537492in}}%
\pgfpathlineto{\pgfqpoint{1.122760in}{1.538539in}}%
\pgfpathlineto{\pgfqpoint{1.125240in}{1.534532in}}%
\pgfpathlineto{\pgfqpoint{1.126480in}{1.535028in}}%
\pgfpathlineto{\pgfqpoint{1.128960in}{1.539462in}}%
\pgfpathlineto{\pgfqpoint{1.131440in}{1.535744in}}%
\pgfpathlineto{\pgfqpoint{1.133920in}{1.532136in}}%
\pgfpathlineto{\pgfqpoint{1.135160in}{1.529958in}}%
\pgfpathlineto{\pgfqpoint{1.137640in}{1.534380in}}%
\pgfpathlineto{\pgfqpoint{1.142600in}{1.515521in}}%
\pgfpathlineto{\pgfqpoint{1.143840in}{1.515880in}}%
\pgfpathlineto{\pgfqpoint{1.145080in}{1.518475in}}%
\pgfpathlineto{\pgfqpoint{1.146320in}{1.517836in}}%
\pgfpathlineto{\pgfqpoint{1.148800in}{1.524427in}}%
\pgfpathlineto{\pgfqpoint{1.150040in}{1.522966in}}%
\pgfpathlineto{\pgfqpoint{1.152520in}{1.525275in}}%
\pgfpathlineto{\pgfqpoint{1.157480in}{1.518839in}}%
\pgfpathlineto{\pgfqpoint{1.158720in}{1.520394in}}%
\pgfpathlineto{\pgfqpoint{1.161200in}{1.519172in}}%
\pgfpathlineto{\pgfqpoint{1.163680in}{1.518301in}}%
\pgfpathlineto{\pgfqpoint{1.164920in}{1.514000in}}%
\pgfpathlineto{\pgfqpoint{1.168640in}{1.515942in}}%
\pgfpathlineto{\pgfqpoint{1.172360in}{1.521128in}}%
\pgfpathlineto{\pgfqpoint{1.174840in}{1.518572in}}%
\pgfpathlineto{\pgfqpoint{1.177320in}{1.530670in}}%
\pgfpathlineto{\pgfqpoint{1.178560in}{1.532114in}}%
\pgfpathlineto{\pgfqpoint{1.181040in}{1.528841in}}%
\pgfpathlineto{\pgfqpoint{1.183520in}{1.530337in}}%
\pgfpathlineto{\pgfqpoint{1.186000in}{1.527023in}}%
\pgfpathlineto{\pgfqpoint{1.188480in}{1.530053in}}%
\pgfpathlineto{\pgfqpoint{1.190960in}{1.527852in}}%
\pgfpathlineto{\pgfqpoint{1.192200in}{1.524214in}}%
\pgfpathlineto{\pgfqpoint{1.194680in}{1.526192in}}%
\pgfpathlineto{\pgfqpoint{1.197160in}{1.536220in}}%
\pgfpathlineto{\pgfqpoint{1.198400in}{1.537803in}}%
\pgfpathlineto{\pgfqpoint{1.200880in}{1.543002in}}%
\pgfpathlineto{\pgfqpoint{1.202120in}{1.540678in}}%
\pgfpathlineto{\pgfqpoint{1.204600in}{1.541254in}}%
\pgfpathlineto{\pgfqpoint{1.205840in}{1.536776in}}%
\pgfpathlineto{\pgfqpoint{1.207080in}{1.538382in}}%
\pgfpathlineto{\pgfqpoint{1.209560in}{1.536990in}}%
\pgfpathlineto{\pgfqpoint{1.214520in}{1.525407in}}%
\pgfpathlineto{\pgfqpoint{1.217000in}{1.522427in}}%
\pgfpathlineto{\pgfqpoint{1.219480in}{1.517510in}}%
\pgfpathlineto{\pgfqpoint{1.220720in}{1.517156in}}%
\pgfpathlineto{\pgfqpoint{1.221960in}{1.518996in}}%
\pgfpathlineto{\pgfqpoint{1.224440in}{1.519279in}}%
\pgfpathlineto{\pgfqpoint{1.225680in}{1.522023in}}%
\pgfpathlineto{\pgfqpoint{1.230640in}{1.518557in}}%
\pgfpathlineto{\pgfqpoint{1.231880in}{1.515070in}}%
\pgfpathlineto{\pgfqpoint{1.234360in}{1.515761in}}%
\pgfpathlineto{\pgfqpoint{1.235600in}{1.517191in}}%
\pgfpathlineto{\pgfqpoint{1.238080in}{1.515630in}}%
\pgfpathlineto{\pgfqpoint{1.239320in}{1.514867in}}%
\pgfpathlineto{\pgfqpoint{1.241800in}{1.509732in}}%
\pgfpathlineto{\pgfqpoint{1.244280in}{1.506802in}}%
\pgfpathlineto{\pgfqpoint{1.246760in}{1.510794in}}%
\pgfpathlineto{\pgfqpoint{1.248000in}{1.509366in}}%
\pgfpathlineto{\pgfqpoint{1.250480in}{1.514577in}}%
\pgfpathlineto{\pgfqpoint{1.251720in}{1.516427in}}%
\pgfpathlineto{\pgfqpoint{1.252960in}{1.515594in}}%
\pgfpathlineto{\pgfqpoint{1.255440in}{1.512800in}}%
\pgfpathlineto{\pgfqpoint{1.256680in}{1.513200in}}%
\pgfpathlineto{\pgfqpoint{1.259160in}{1.509780in}}%
\pgfpathlineto{\pgfqpoint{1.261640in}{1.512464in}}%
\pgfpathlineto{\pgfqpoint{1.265360in}{1.505327in}}%
\pgfpathlineto{\pgfqpoint{1.266600in}{1.504232in}}%
\pgfpathlineto{\pgfqpoint{1.270320in}{1.506606in}}%
\pgfpathlineto{\pgfqpoint{1.272800in}{1.509860in}}%
\pgfpathlineto{\pgfqpoint{1.274040in}{1.508872in}}%
\pgfpathlineto{\pgfqpoint{1.277760in}{1.514472in}}%
\pgfpathlineto{\pgfqpoint{1.279000in}{1.513592in}}%
\pgfpathlineto{\pgfqpoint{1.281480in}{1.507955in}}%
\pgfpathlineto{\pgfqpoint{1.283960in}{1.513913in}}%
\pgfpathlineto{\pgfqpoint{1.285200in}{1.514227in}}%
\pgfpathlineto{\pgfqpoint{1.287680in}{1.509993in}}%
\pgfpathlineto{\pgfqpoint{1.290160in}{1.506199in}}%
\pgfpathlineto{\pgfqpoint{1.293880in}{1.502726in}}%
\pgfpathlineto{\pgfqpoint{1.295120in}{1.504118in}}%
\pgfpathlineto{\pgfqpoint{1.296360in}{1.508019in}}%
\pgfpathlineto{\pgfqpoint{1.297600in}{1.506318in}}%
\pgfpathlineto{\pgfqpoint{1.298840in}{1.502682in}}%
\pgfpathlineto{\pgfqpoint{1.300080in}{1.506755in}}%
\pgfpathlineto{\pgfqpoint{1.302560in}{1.507359in}}%
\pgfpathlineto{\pgfqpoint{1.305040in}{1.504281in}}%
\pgfpathlineto{\pgfqpoint{1.306280in}{1.503790in}}%
\pgfpathlineto{\pgfqpoint{1.307520in}{1.505519in}}%
\pgfpathlineto{\pgfqpoint{1.310000in}{1.501808in}}%
\pgfpathlineto{\pgfqpoint{1.312480in}{1.504877in}}%
\pgfpathlineto{\pgfqpoint{1.317440in}{1.498498in}}%
\pgfpathlineto{\pgfqpoint{1.318680in}{1.499089in}}%
\pgfpathlineto{\pgfqpoint{1.323640in}{1.508291in}}%
\pgfpathlineto{\pgfqpoint{1.324880in}{1.509095in}}%
\pgfpathlineto{\pgfqpoint{1.326120in}{1.507850in}}%
\pgfpathlineto{\pgfqpoint{1.327360in}{1.509517in}}%
\pgfpathlineto{\pgfqpoint{1.328600in}{1.508163in}}%
\pgfpathlineto{\pgfqpoint{1.329840in}{1.503520in}}%
\pgfpathlineto{\pgfqpoint{1.331080in}{1.503776in}}%
\pgfpathlineto{\pgfqpoint{1.333560in}{1.500147in}}%
\pgfpathlineto{\pgfqpoint{1.337280in}{1.489643in}}%
\pgfpathlineto{\pgfqpoint{1.339760in}{1.491581in}}%
\pgfpathlineto{\pgfqpoint{1.341000in}{1.491367in}}%
\pgfpathlineto{\pgfqpoint{1.343480in}{1.489045in}}%
\pgfpathlineto{\pgfqpoint{1.350920in}{1.491818in}}%
\pgfpathlineto{\pgfqpoint{1.352160in}{1.491533in}}%
\pgfpathlineto{\pgfqpoint{1.355880in}{1.486560in}}%
\pgfpathlineto{\pgfqpoint{1.358360in}{1.488821in}}%
\pgfpathlineto{\pgfqpoint{1.360840in}{1.489672in}}%
\pgfpathlineto{\pgfqpoint{1.362080in}{1.488929in}}%
\pgfpathlineto{\pgfqpoint{1.363320in}{1.489792in}}%
\pgfpathlineto{\pgfqpoint{1.365800in}{1.487897in}}%
\pgfpathlineto{\pgfqpoint{1.368280in}{1.487427in}}%
\pgfpathlineto{\pgfqpoint{1.370760in}{1.491369in}}%
\pgfpathlineto{\pgfqpoint{1.372000in}{1.490927in}}%
\pgfpathlineto{\pgfqpoint{1.373240in}{1.495233in}}%
\pgfpathlineto{\pgfqpoint{1.374480in}{1.495054in}}%
\pgfpathlineto{\pgfqpoint{1.375720in}{1.496662in}}%
\pgfpathlineto{\pgfqpoint{1.379440in}{1.492642in}}%
\pgfpathlineto{\pgfqpoint{1.381920in}{1.493606in}}%
\pgfpathlineto{\pgfqpoint{1.383160in}{1.493973in}}%
\pgfpathlineto{\pgfqpoint{1.385640in}{1.496806in}}%
\pgfpathlineto{\pgfqpoint{1.386880in}{1.494244in}}%
\pgfpathlineto{\pgfqpoint{1.388120in}{1.494485in}}%
\pgfpathlineto{\pgfqpoint{1.391840in}{1.490872in}}%
\pgfpathlineto{\pgfqpoint{1.395560in}{1.494307in}}%
\pgfpathlineto{\pgfqpoint{1.396800in}{1.496428in}}%
\pgfpathlineto{\pgfqpoint{1.399280in}{1.497027in}}%
\pgfpathlineto{\pgfqpoint{1.400520in}{1.499409in}}%
\pgfpathlineto{\pgfqpoint{1.403000in}{1.497777in}}%
\pgfpathlineto{\pgfqpoint{1.405480in}{1.492612in}}%
\pgfpathlineto{\pgfqpoint{1.407960in}{1.496723in}}%
\pgfpathlineto{\pgfqpoint{1.409200in}{1.495747in}}%
\pgfpathlineto{\pgfqpoint{1.410440in}{1.496975in}}%
\pgfpathlineto{\pgfqpoint{1.419120in}{1.485599in}}%
\pgfpathlineto{\pgfqpoint{1.420360in}{1.488413in}}%
\pgfpathlineto{\pgfqpoint{1.422840in}{1.486428in}}%
\pgfpathlineto{\pgfqpoint{1.425320in}{1.490154in}}%
\pgfpathlineto{\pgfqpoint{1.426560in}{1.489535in}}%
\pgfpathlineto{\pgfqpoint{1.430280in}{1.485657in}}%
\pgfpathlineto{\pgfqpoint{1.431520in}{1.487168in}}%
\pgfpathlineto{\pgfqpoint{1.434000in}{1.481582in}}%
\pgfpathlineto{\pgfqpoint{1.436480in}{1.483361in}}%
\pgfpathlineto{\pgfqpoint{1.438960in}{1.479371in}}%
\pgfpathlineto{\pgfqpoint{1.440200in}{1.475305in}}%
\pgfpathlineto{\pgfqpoint{1.442680in}{1.476893in}}%
\pgfpathlineto{\pgfqpoint{1.445160in}{1.481113in}}%
\pgfpathlineto{\pgfqpoint{1.447640in}{1.485128in}}%
\pgfpathlineto{\pgfqpoint{1.450120in}{1.486958in}}%
\pgfpathlineto{\pgfqpoint{1.451360in}{1.487866in}}%
\pgfpathlineto{\pgfqpoint{1.452600in}{1.486377in}}%
\pgfpathlineto{\pgfqpoint{1.453840in}{1.481699in}}%
\pgfpathlineto{\pgfqpoint{1.456320in}{1.481157in}}%
\pgfpathlineto{\pgfqpoint{1.457560in}{1.481081in}}%
\pgfpathlineto{\pgfqpoint{1.461280in}{1.471759in}}%
\pgfpathlineto{\pgfqpoint{1.463760in}{1.472171in}}%
\pgfpathlineto{\pgfqpoint{1.466240in}{1.472358in}}%
\pgfpathlineto{\pgfqpoint{1.469960in}{1.475371in}}%
\pgfpathlineto{\pgfqpoint{1.473680in}{1.478603in}}%
\pgfpathlineto{\pgfqpoint{1.477400in}{1.473049in}}%
\pgfpathlineto{\pgfqpoint{1.479880in}{1.469839in}}%
\pgfpathlineto{\pgfqpoint{1.483600in}{1.472965in}}%
\pgfpathlineto{\pgfqpoint{1.486080in}{1.472222in}}%
\pgfpathlineto{\pgfqpoint{1.487320in}{1.472486in}}%
\pgfpathlineto{\pgfqpoint{1.488560in}{1.470134in}}%
\pgfpathlineto{\pgfqpoint{1.493520in}{1.471381in}}%
\pgfpathlineto{\pgfqpoint{1.494760in}{1.473524in}}%
\pgfpathlineto{\pgfqpoint{1.496000in}{1.472494in}}%
\pgfpathlineto{\pgfqpoint{1.497240in}{1.475296in}}%
\pgfpathlineto{\pgfqpoint{1.498480in}{1.474638in}}%
\pgfpathlineto{\pgfqpoint{1.499720in}{1.476613in}}%
\pgfpathlineto{\pgfqpoint{1.503440in}{1.473962in}}%
\pgfpathlineto{\pgfqpoint{1.504680in}{1.474153in}}%
\pgfpathlineto{\pgfqpoint{1.507160in}{1.472214in}}%
\pgfpathlineto{\pgfqpoint{1.509640in}{1.475296in}}%
\pgfpathlineto{\pgfqpoint{1.510880in}{1.473302in}}%
\pgfpathlineto{\pgfqpoint{1.513360in}{1.473324in}}%
\pgfpathlineto{\pgfqpoint{1.514600in}{1.471387in}}%
\pgfpathlineto{\pgfqpoint{1.519560in}{1.473534in}}%
\pgfpathlineto{\pgfqpoint{1.524520in}{1.478995in}}%
\pgfpathlineto{\pgfqpoint{1.527000in}{1.476984in}}%
\pgfpathlineto{\pgfqpoint{1.528240in}{1.472692in}}%
\pgfpathlineto{\pgfqpoint{1.530720in}{1.473736in}}%
\pgfpathlineto{\pgfqpoint{1.531960in}{1.474598in}}%
\pgfpathlineto{\pgfqpoint{1.536920in}{1.468625in}}%
\pgfpathlineto{\pgfqpoint{1.541880in}{1.463887in}}%
\pgfpathlineto{\pgfqpoint{1.545600in}{1.467525in}}%
\pgfpathlineto{\pgfqpoint{1.546840in}{1.466567in}}%
\pgfpathlineto{\pgfqpoint{1.549320in}{1.471545in}}%
\pgfpathlineto{\pgfqpoint{1.550560in}{1.471074in}}%
\pgfpathlineto{\pgfqpoint{1.554280in}{1.466636in}}%
\pgfpathlineto{\pgfqpoint{1.555520in}{1.469097in}}%
\pgfpathlineto{\pgfqpoint{1.558000in}{1.465460in}}%
\pgfpathlineto{\pgfqpoint{1.561720in}{1.468947in}}%
\pgfpathlineto{\pgfqpoint{1.565440in}{1.465156in}}%
\pgfpathlineto{\pgfqpoint{1.569160in}{1.470565in}}%
\pgfpathlineto{\pgfqpoint{1.572880in}{1.475320in}}%
\pgfpathlineto{\pgfqpoint{1.574120in}{1.474704in}}%
\pgfpathlineto{\pgfqpoint{1.575360in}{1.476012in}}%
\pgfpathlineto{\pgfqpoint{1.576600in}{1.474773in}}%
\pgfpathlineto{\pgfqpoint{1.579080in}{1.469956in}}%
\pgfpathlineto{\pgfqpoint{1.584040in}{1.464870in}}%
\pgfpathlineto{\pgfqpoint{1.586520in}{1.462787in}}%
\pgfpathlineto{\pgfqpoint{1.587760in}{1.462255in}}%
\pgfpathlineto{\pgfqpoint{1.593960in}{1.466536in}}%
\pgfpathlineto{\pgfqpoint{1.597680in}{1.467741in}}%
\pgfpathlineto{\pgfqpoint{1.598920in}{1.466178in}}%
\pgfpathlineto{\pgfqpoint{1.600160in}{1.466422in}}%
\pgfpathlineto{\pgfqpoint{1.603880in}{1.461823in}}%
\pgfpathlineto{\pgfqpoint{1.606360in}{1.463532in}}%
\pgfpathlineto{\pgfqpoint{1.607600in}{1.463841in}}%
\pgfpathlineto{\pgfqpoint{1.610080in}{1.462391in}}%
\pgfpathlineto{\pgfqpoint{1.611320in}{1.462448in}}%
\pgfpathlineto{\pgfqpoint{1.615040in}{1.458751in}}%
\pgfpathlineto{\pgfqpoint{1.617520in}{1.459088in}}%
\pgfpathlineto{\pgfqpoint{1.618760in}{1.461936in}}%
\pgfpathlineto{\pgfqpoint{1.620000in}{1.461849in}}%
\pgfpathlineto{\pgfqpoint{1.621240in}{1.463418in}}%
\pgfpathlineto{\pgfqpoint{1.622480in}{1.462873in}}%
\pgfpathlineto{\pgfqpoint{1.623720in}{1.465258in}}%
\pgfpathlineto{\pgfqpoint{1.626200in}{1.465072in}}%
\pgfpathlineto{\pgfqpoint{1.628680in}{1.465214in}}%
\pgfpathlineto{\pgfqpoint{1.631160in}{1.463827in}}%
\pgfpathlineto{\pgfqpoint{1.632400in}{1.466299in}}%
\pgfpathlineto{\pgfqpoint{1.634880in}{1.465422in}}%
\pgfpathlineto{\pgfqpoint{1.636120in}{1.466361in}}%
\pgfpathlineto{\pgfqpoint{1.637360in}{1.464896in}}%
\pgfpathlineto{\pgfqpoint{1.638600in}{1.461220in}}%
\pgfpathlineto{\pgfqpoint{1.642320in}{1.461692in}}%
\pgfpathlineto{\pgfqpoint{1.647280in}{1.466186in}}%
\pgfpathlineto{\pgfqpoint{1.648520in}{1.468136in}}%
\pgfpathlineto{\pgfqpoint{1.651000in}{1.466148in}}%
\pgfpathlineto{\pgfqpoint{1.652240in}{1.462291in}}%
\pgfpathlineto{\pgfqpoint{1.654720in}{1.463263in}}%
\pgfpathlineto{\pgfqpoint{1.655960in}{1.463903in}}%
\pgfpathlineto{\pgfqpoint{1.665880in}{1.455902in}}%
\pgfpathlineto{\pgfqpoint{1.668360in}{1.459899in}}%
\pgfpathlineto{\pgfqpoint{1.670840in}{1.456409in}}%
\pgfpathlineto{\pgfqpoint{1.673320in}{1.461539in}}%
\pgfpathlineto{\pgfqpoint{1.674560in}{1.461517in}}%
\pgfpathlineto{\pgfqpoint{1.678280in}{1.459156in}}%
\pgfpathlineto{\pgfqpoint{1.679520in}{1.460727in}}%
\pgfpathlineto{\pgfqpoint{1.682000in}{1.459611in}}%
\pgfpathlineto{\pgfqpoint{1.684480in}{1.463819in}}%
\pgfpathlineto{\pgfqpoint{1.689440in}{1.459592in}}%
\pgfpathlineto{\pgfqpoint{1.699360in}{1.468180in}}%
\pgfpathlineto{\pgfqpoint{1.705560in}{1.461306in}}%
\pgfpathlineto{\pgfqpoint{1.706800in}{1.460121in}}%
\pgfpathlineto{\pgfqpoint{1.709280in}{1.455804in}}%
\pgfpathlineto{\pgfqpoint{1.711760in}{1.454625in}}%
\pgfpathlineto{\pgfqpoint{1.715480in}{1.458174in}}%
\pgfpathlineto{\pgfqpoint{1.716720in}{1.457835in}}%
\pgfpathlineto{\pgfqpoint{1.721680in}{1.461411in}}%
\pgfpathlineto{\pgfqpoint{1.724160in}{1.459239in}}%
\pgfpathlineto{\pgfqpoint{1.729120in}{1.456067in}}%
\pgfpathlineto{\pgfqpoint{1.732840in}{1.454787in}}%
\pgfpathlineto{\pgfqpoint{1.735320in}{1.455800in}}%
\pgfpathlineto{\pgfqpoint{1.736560in}{1.453537in}}%
\pgfpathlineto{\pgfqpoint{1.737800in}{1.453704in}}%
\pgfpathlineto{\pgfqpoint{1.739040in}{1.452652in}}%
\pgfpathlineto{\pgfqpoint{1.741520in}{1.454601in}}%
\pgfpathlineto{\pgfqpoint{1.742760in}{1.457431in}}%
\pgfpathlineto{\pgfqpoint{1.745240in}{1.456092in}}%
\pgfpathlineto{\pgfqpoint{1.746480in}{1.455719in}}%
\pgfpathlineto{\pgfqpoint{1.747720in}{1.457789in}}%
\pgfpathlineto{\pgfqpoint{1.751440in}{1.456652in}}%
\pgfpathlineto{\pgfqpoint{1.752680in}{1.456674in}}%
\pgfpathlineto{\pgfqpoint{1.755160in}{1.455253in}}%
\pgfpathlineto{\pgfqpoint{1.756400in}{1.456961in}}%
\pgfpathlineto{\pgfqpoint{1.758880in}{1.455851in}}%
\pgfpathlineto{\pgfqpoint{1.760120in}{1.455992in}}%
\pgfpathlineto{\pgfqpoint{1.761360in}{1.454647in}}%
\pgfpathlineto{\pgfqpoint{1.763840in}{1.450354in}}%
\pgfpathlineto{\pgfqpoint{1.767560in}{1.452191in}}%
\pgfpathlineto{\pgfqpoint{1.772520in}{1.455002in}}%
\pgfpathlineto{\pgfqpoint{1.777480in}{1.449011in}}%
\pgfpathlineto{\pgfqpoint{1.781200in}{1.451267in}}%
\pgfpathlineto{\pgfqpoint{1.783680in}{1.449095in}}%
\pgfpathlineto{\pgfqpoint{1.788640in}{1.445884in}}%
\pgfpathlineto{\pgfqpoint{1.789880in}{1.446199in}}%
\pgfpathlineto{\pgfqpoint{1.792360in}{1.450395in}}%
\pgfpathlineto{\pgfqpoint{1.794840in}{1.446493in}}%
\pgfpathlineto{\pgfqpoint{1.797320in}{1.450291in}}%
\pgfpathlineto{\pgfqpoint{1.802280in}{1.449500in}}%
\pgfpathlineto{\pgfqpoint{1.803520in}{1.451244in}}%
\pgfpathlineto{\pgfqpoint{1.806000in}{1.449850in}}%
\pgfpathlineto{\pgfqpoint{1.808480in}{1.452997in}}%
\pgfpathlineto{\pgfqpoint{1.814680in}{1.448044in}}%
\pgfpathlineto{\pgfqpoint{1.819640in}{1.452664in}}%
\pgfpathlineto{\pgfqpoint{1.820880in}{1.454162in}}%
\pgfpathlineto{\pgfqpoint{1.822120in}{1.453889in}}%
\pgfpathlineto{\pgfqpoint{1.823360in}{1.454929in}}%
\pgfpathlineto{\pgfqpoint{1.824600in}{1.453860in}}%
\pgfpathlineto{\pgfqpoint{1.827080in}{1.450037in}}%
\pgfpathlineto{\pgfqpoint{1.830800in}{1.446984in}}%
\pgfpathlineto{\pgfqpoint{1.834520in}{1.441894in}}%
\pgfpathlineto{\pgfqpoint{1.838240in}{1.445808in}}%
\pgfpathlineto{\pgfqpoint{1.839480in}{1.447207in}}%
\pgfpathlineto{\pgfqpoint{1.841960in}{1.446495in}}%
\pgfpathlineto{\pgfqpoint{1.845680in}{1.448879in}}%
\pgfpathlineto{\pgfqpoint{1.850640in}{1.442880in}}%
\pgfpathlineto{\pgfqpoint{1.851880in}{1.442014in}}%
\pgfpathlineto{\pgfqpoint{1.854360in}{1.444112in}}%
\pgfpathlineto{\pgfqpoint{1.859320in}{1.444920in}}%
\pgfpathlineto{\pgfqpoint{1.861800in}{1.442470in}}%
\pgfpathlineto{\pgfqpoint{1.863040in}{1.441258in}}%
\pgfpathlineto{\pgfqpoint{1.865520in}{1.442871in}}%
\pgfpathlineto{\pgfqpoint{1.866760in}{1.445343in}}%
\pgfpathlineto{\pgfqpoint{1.868000in}{1.444504in}}%
\pgfpathlineto{\pgfqpoint{1.869240in}{1.447676in}}%
\pgfpathlineto{\pgfqpoint{1.870480in}{1.447235in}}%
\pgfpathlineto{\pgfqpoint{1.871720in}{1.448964in}}%
\pgfpathlineto{\pgfqpoint{1.879160in}{1.447212in}}%
\pgfpathlineto{\pgfqpoint{1.880400in}{1.448953in}}%
\pgfpathlineto{\pgfqpoint{1.884120in}{1.447470in}}%
\pgfpathlineto{\pgfqpoint{1.885360in}{1.446354in}}%
\pgfpathlineto{\pgfqpoint{1.887840in}{1.442655in}}%
\pgfpathlineto{\pgfqpoint{1.894040in}{1.443837in}}%
\pgfpathlineto{\pgfqpoint{1.896520in}{1.446964in}}%
\pgfpathlineto{\pgfqpoint{1.899000in}{1.444782in}}%
\pgfpathlineto{\pgfqpoint{1.901480in}{1.441057in}}%
\pgfpathlineto{\pgfqpoint{1.905200in}{1.441616in}}%
\pgfpathlineto{\pgfqpoint{1.907680in}{1.439682in}}%
\pgfpathlineto{\pgfqpoint{1.910160in}{1.436445in}}%
\pgfpathlineto{\pgfqpoint{1.911400in}{1.436617in}}%
\pgfpathlineto{\pgfqpoint{1.913880in}{1.434865in}}%
\pgfpathlineto{\pgfqpoint{1.916360in}{1.436756in}}%
\pgfpathlineto{\pgfqpoint{1.918840in}{1.432283in}}%
\pgfpathlineto{\pgfqpoint{1.921320in}{1.436488in}}%
\pgfpathlineto{\pgfqpoint{1.926280in}{1.435488in}}%
\pgfpathlineto{\pgfqpoint{1.928760in}{1.436577in}}%
\pgfpathlineto{\pgfqpoint{1.930000in}{1.435515in}}%
\pgfpathlineto{\pgfqpoint{1.932480in}{1.439441in}}%
\pgfpathlineto{\pgfqpoint{1.938680in}{1.433532in}}%
\pgfpathlineto{\pgfqpoint{1.947360in}{1.441183in}}%
\pgfpathlineto{\pgfqpoint{1.953560in}{1.436165in}}%
\pgfpathlineto{\pgfqpoint{1.957280in}{1.431057in}}%
\pgfpathlineto{\pgfqpoint{1.958520in}{1.430558in}}%
\pgfpathlineto{\pgfqpoint{1.969680in}{1.439234in}}%
\pgfpathlineto{\pgfqpoint{1.974640in}{1.430893in}}%
\pgfpathlineto{\pgfqpoint{1.975880in}{1.430060in}}%
\pgfpathlineto{\pgfqpoint{1.978360in}{1.433329in}}%
\pgfpathlineto{\pgfqpoint{1.979600in}{1.434058in}}%
\pgfpathlineto{\pgfqpoint{1.982080in}{1.432957in}}%
\pgfpathlineto{\pgfqpoint{1.983320in}{1.433148in}}%
\pgfpathlineto{\pgfqpoint{1.987040in}{1.429991in}}%
\pgfpathlineto{\pgfqpoint{1.989520in}{1.430819in}}%
\pgfpathlineto{\pgfqpoint{1.990760in}{1.432996in}}%
\pgfpathlineto{\pgfqpoint{1.992000in}{1.431383in}}%
\pgfpathlineto{\pgfqpoint{1.993240in}{1.437206in}}%
\pgfpathlineto{\pgfqpoint{1.994480in}{1.436975in}}%
\pgfpathlineto{\pgfqpoint{1.995720in}{1.439328in}}%
\pgfpathlineto{\pgfqpoint{2.000680in}{1.437090in}}%
\pgfpathlineto{\pgfqpoint{2.003160in}{1.437017in}}%
\pgfpathlineto{\pgfqpoint{2.005640in}{1.438339in}}%
\pgfpathlineto{\pgfqpoint{2.011840in}{1.433869in}}%
\pgfpathlineto{\pgfqpoint{2.014320in}{1.434578in}}%
\pgfpathlineto{\pgfqpoint{2.015560in}{1.435356in}}%
\pgfpathlineto{\pgfqpoint{2.018040in}{1.433925in}}%
\pgfpathlineto{\pgfqpoint{2.020520in}{1.437427in}}%
\pgfpathlineto{\pgfqpoint{2.024240in}{1.431419in}}%
\pgfpathlineto{\pgfqpoint{2.029200in}{1.432867in}}%
\pgfpathlineto{\pgfqpoint{2.031680in}{1.430750in}}%
\pgfpathlineto{\pgfqpoint{2.034160in}{1.427735in}}%
\pgfpathlineto{\pgfqpoint{2.035400in}{1.428581in}}%
\pgfpathlineto{\pgfqpoint{2.037880in}{1.426970in}}%
\pgfpathlineto{\pgfqpoint{2.040360in}{1.428469in}}%
\pgfpathlineto{\pgfqpoint{2.042840in}{1.424141in}}%
\pgfpathlineto{\pgfqpoint{2.045320in}{1.427670in}}%
\pgfpathlineto{\pgfqpoint{2.047800in}{1.427058in}}%
\pgfpathlineto{\pgfqpoint{2.052760in}{1.428555in}}%
\pgfpathlineto{\pgfqpoint{2.054000in}{1.426688in}}%
\pgfpathlineto{\pgfqpoint{2.056480in}{1.429358in}}%
\pgfpathlineto{\pgfqpoint{2.062680in}{1.425239in}}%
\pgfpathlineto{\pgfqpoint{2.071360in}{1.431003in}}%
\pgfpathlineto{\pgfqpoint{2.072600in}{1.430760in}}%
\pgfpathlineto{\pgfqpoint{2.078800in}{1.422356in}}%
\pgfpathlineto{\pgfqpoint{2.083760in}{1.420094in}}%
\pgfpathlineto{\pgfqpoint{2.089960in}{1.423906in}}%
\pgfpathlineto{\pgfqpoint{2.092440in}{1.425399in}}%
\pgfpathlineto{\pgfqpoint{2.093680in}{1.427072in}}%
\pgfpathlineto{\pgfqpoint{2.098640in}{1.417832in}}%
\pgfpathlineto{\pgfqpoint{2.099880in}{1.417131in}}%
\pgfpathlineto{\pgfqpoint{2.103600in}{1.420851in}}%
\pgfpathlineto{\pgfqpoint{2.106080in}{1.419253in}}%
\pgfpathlineto{\pgfqpoint{2.107320in}{1.419089in}}%
\pgfpathlineto{\pgfqpoint{2.109800in}{1.417522in}}%
\pgfpathlineto{\pgfqpoint{2.112280in}{1.417142in}}%
\pgfpathlineto{\pgfqpoint{2.113520in}{1.416912in}}%
\pgfpathlineto{\pgfqpoint{2.114760in}{1.418840in}}%
\pgfpathlineto{\pgfqpoint{2.116000in}{1.417180in}}%
\pgfpathlineto{\pgfqpoint{2.117240in}{1.422925in}}%
\pgfpathlineto{\pgfqpoint{2.118480in}{1.423023in}}%
\pgfpathlineto{\pgfqpoint{2.119720in}{1.425705in}}%
\pgfpathlineto{\pgfqpoint{2.127160in}{1.423681in}}%
\pgfpathlineto{\pgfqpoint{2.129640in}{1.425103in}}%
\pgfpathlineto{\pgfqpoint{2.132120in}{1.423578in}}%
\pgfpathlineto{\pgfqpoint{2.133360in}{1.423299in}}%
\pgfpathlineto{\pgfqpoint{2.135840in}{1.420628in}}%
\pgfpathlineto{\pgfqpoint{2.138320in}{1.421616in}}%
\pgfpathlineto{\pgfqpoint{2.139560in}{1.423102in}}%
\pgfpathlineto{\pgfqpoint{2.142040in}{1.421785in}}%
\pgfpathlineto{\pgfqpoint{2.144520in}{1.424609in}}%
\pgfpathlineto{\pgfqpoint{2.148240in}{1.419022in}}%
\pgfpathlineto{\pgfqpoint{2.151960in}{1.421743in}}%
\pgfpathlineto{\pgfqpoint{2.155680in}{1.419606in}}%
\pgfpathlineto{\pgfqpoint{2.158160in}{1.416760in}}%
\pgfpathlineto{\pgfqpoint{2.159400in}{1.417327in}}%
\pgfpathlineto{\pgfqpoint{2.161880in}{1.416344in}}%
\pgfpathlineto{\pgfqpoint{2.164360in}{1.417960in}}%
\pgfpathlineto{\pgfqpoint{2.166840in}{1.413707in}}%
\pgfpathlineto{\pgfqpoint{2.169320in}{1.417598in}}%
\pgfpathlineto{\pgfqpoint{2.170560in}{1.416775in}}%
\pgfpathlineto{\pgfqpoint{2.174280in}{1.418586in}}%
\pgfpathlineto{\pgfqpoint{2.175520in}{1.420314in}}%
\pgfpathlineto{\pgfqpoint{2.176760in}{1.419711in}}%
\pgfpathlineto{\pgfqpoint{2.178000in}{1.417744in}}%
\pgfpathlineto{\pgfqpoint{2.180480in}{1.421089in}}%
\pgfpathlineto{\pgfqpoint{2.185440in}{1.417302in}}%
\pgfpathlineto{\pgfqpoint{2.189160in}{1.418256in}}%
\pgfpathlineto{\pgfqpoint{2.194120in}{1.421573in}}%
\pgfpathlineto{\pgfqpoint{2.196600in}{1.422135in}}%
\pgfpathlineto{\pgfqpoint{2.200320in}{1.416777in}}%
\pgfpathlineto{\pgfqpoint{2.201560in}{1.415901in}}%
\pgfpathlineto{\pgfqpoint{2.204040in}{1.412071in}}%
\pgfpathlineto{\pgfqpoint{2.206520in}{1.409568in}}%
\pgfpathlineto{\pgfqpoint{2.211480in}{1.414836in}}%
\pgfpathlineto{\pgfqpoint{2.213960in}{1.414271in}}%
\pgfpathlineto{\pgfqpoint{2.217680in}{1.417008in}}%
\pgfpathlineto{\pgfqpoint{2.222640in}{1.408293in}}%
\pgfpathlineto{\pgfqpoint{2.223880in}{1.408146in}}%
\pgfpathlineto{\pgfqpoint{2.226360in}{1.411867in}}%
\pgfpathlineto{\pgfqpoint{2.227600in}{1.412795in}}%
\pgfpathlineto{\pgfqpoint{2.230080in}{1.410914in}}%
\pgfpathlineto{\pgfqpoint{2.231320in}{1.410681in}}%
\pgfpathlineto{\pgfqpoint{2.235040in}{1.407570in}}%
\pgfpathlineto{\pgfqpoint{2.237520in}{1.406614in}}%
\pgfpathlineto{\pgfqpoint{2.238760in}{1.408587in}}%
\pgfpathlineto{\pgfqpoint{2.240000in}{1.407176in}}%
\pgfpathlineto{\pgfqpoint{2.241240in}{1.413517in}}%
\pgfpathlineto{\pgfqpoint{2.242480in}{1.413853in}}%
\pgfpathlineto{\pgfqpoint{2.244960in}{1.416878in}}%
\pgfpathlineto{\pgfqpoint{2.249920in}{1.413148in}}%
\pgfpathlineto{\pgfqpoint{2.251160in}{1.413533in}}%
\pgfpathlineto{\pgfqpoint{2.252400in}{1.415345in}}%
\pgfpathlineto{\pgfqpoint{2.253640in}{1.415122in}}%
\pgfpathlineto{\pgfqpoint{2.254880in}{1.413609in}}%
\pgfpathlineto{\pgfqpoint{2.257360in}{1.413914in}}%
\pgfpathlineto{\pgfqpoint{2.259840in}{1.411126in}}%
\pgfpathlineto{\pgfqpoint{2.261080in}{1.410449in}}%
\pgfpathlineto{\pgfqpoint{2.264800in}{1.411590in}}%
\pgfpathlineto{\pgfqpoint{2.266040in}{1.411211in}}%
\pgfpathlineto{\pgfqpoint{2.268520in}{1.413789in}}%
\pgfpathlineto{\pgfqpoint{2.272240in}{1.408536in}}%
\pgfpathlineto{\pgfqpoint{2.277200in}{1.410949in}}%
\pgfpathlineto{\pgfqpoint{2.279680in}{1.410129in}}%
\pgfpathlineto{\pgfqpoint{2.282160in}{1.408189in}}%
\pgfpathlineto{\pgfqpoint{2.283400in}{1.408209in}}%
\pgfpathlineto{\pgfqpoint{2.284640in}{1.406827in}}%
\pgfpathlineto{\pgfqpoint{2.288360in}{1.408017in}}%
\pgfpathlineto{\pgfqpoint{2.290840in}{1.403165in}}%
\pgfpathlineto{\pgfqpoint{2.293320in}{1.405789in}}%
\pgfpathlineto{\pgfqpoint{2.294560in}{1.404967in}}%
\pgfpathlineto{\pgfqpoint{2.300760in}{1.408233in}}%
\pgfpathlineto{\pgfqpoint{2.302000in}{1.406000in}}%
\pgfpathlineto{\pgfqpoint{2.304480in}{1.409818in}}%
\pgfpathlineto{\pgfqpoint{2.309440in}{1.406452in}}%
\pgfpathlineto{\pgfqpoint{2.313160in}{1.407592in}}%
\pgfpathlineto{\pgfqpoint{2.319360in}{1.412729in}}%
\pgfpathlineto{\pgfqpoint{2.320600in}{1.412228in}}%
\pgfpathlineto{\pgfqpoint{2.324320in}{1.406768in}}%
\pgfpathlineto{\pgfqpoint{2.325560in}{1.406235in}}%
\pgfpathlineto{\pgfqpoint{2.328040in}{1.402656in}}%
\pgfpathlineto{\pgfqpoint{2.330520in}{1.400703in}}%
\pgfpathlineto{\pgfqpoint{2.335480in}{1.405476in}}%
\pgfpathlineto{\pgfqpoint{2.336720in}{1.404353in}}%
\pgfpathlineto{\pgfqpoint{2.339200in}{1.405953in}}%
\pgfpathlineto{\pgfqpoint{2.341680in}{1.408123in}}%
\pgfpathlineto{\pgfqpoint{2.346640in}{1.401542in}}%
\pgfpathlineto{\pgfqpoint{2.347880in}{1.400985in}}%
\pgfpathlineto{\pgfqpoint{2.351600in}{1.406046in}}%
\pgfpathlineto{\pgfqpoint{2.359040in}{1.400871in}}%
\pgfpathlineto{\pgfqpoint{2.364000in}{1.400453in}}%
\pgfpathlineto{\pgfqpoint{2.365240in}{1.405800in}}%
\pgfpathlineto{\pgfqpoint{2.366480in}{1.405962in}}%
\pgfpathlineto{\pgfqpoint{2.367720in}{1.408398in}}%
\pgfpathlineto{\pgfqpoint{2.372680in}{1.406569in}}%
\pgfpathlineto{\pgfqpoint{2.375160in}{1.406833in}}%
\pgfpathlineto{\pgfqpoint{2.377640in}{1.408385in}}%
\pgfpathlineto{\pgfqpoint{2.378880in}{1.406882in}}%
\pgfpathlineto{\pgfqpoint{2.381360in}{1.406968in}}%
\pgfpathlineto{\pgfqpoint{2.383840in}{1.403856in}}%
\pgfpathlineto{\pgfqpoint{2.385080in}{1.402959in}}%
\pgfpathlineto{\pgfqpoint{2.388800in}{1.404148in}}%
\pgfpathlineto{\pgfqpoint{2.390040in}{1.403735in}}%
\pgfpathlineto{\pgfqpoint{2.392520in}{1.405478in}}%
\pgfpathlineto{\pgfqpoint{2.397480in}{1.400858in}}%
\pgfpathlineto{\pgfqpoint{2.399960in}{1.403966in}}%
\pgfpathlineto{\pgfqpoint{2.403680in}{1.402932in}}%
\pgfpathlineto{\pgfqpoint{2.406160in}{1.401431in}}%
\pgfpathlineto{\pgfqpoint{2.407400in}{1.401656in}}%
\pgfpathlineto{\pgfqpoint{2.408640in}{1.400493in}}%
\pgfpathlineto{\pgfqpoint{2.412360in}{1.402406in}}%
\pgfpathlineto{\pgfqpoint{2.414840in}{1.397253in}}%
\pgfpathlineto{\pgfqpoint{2.423520in}{1.403279in}}%
\pgfpathlineto{\pgfqpoint{2.426000in}{1.400537in}}%
\pgfpathlineto{\pgfqpoint{2.428480in}{1.403952in}}%
\pgfpathlineto{\pgfqpoint{2.434680in}{1.402140in}}%
\pgfpathlineto{\pgfqpoint{2.437160in}{1.403689in}}%
\pgfpathlineto{\pgfqpoint{2.443360in}{1.408197in}}%
\pgfpathlineto{\pgfqpoint{2.445840in}{1.405421in}}%
\pgfpathlineto{\pgfqpoint{2.448320in}{1.402637in}}%
\pgfpathlineto{\pgfqpoint{2.449560in}{1.402313in}}%
\pgfpathlineto{\pgfqpoint{2.452040in}{1.399400in}}%
\pgfpathlineto{\pgfqpoint{2.454520in}{1.397964in}}%
\pgfpathlineto{\pgfqpoint{2.459480in}{1.401866in}}%
\pgfpathlineto{\pgfqpoint{2.460720in}{1.400656in}}%
\pgfpathlineto{\pgfqpoint{2.465680in}{1.404619in}}%
\pgfpathlineto{\pgfqpoint{2.468160in}{1.402409in}}%
\pgfpathlineto{\pgfqpoint{2.471880in}{1.398028in}}%
\pgfpathlineto{\pgfqpoint{2.475600in}{1.402889in}}%
\pgfpathlineto{\pgfqpoint{2.480560in}{1.400343in}}%
\pgfpathlineto{\pgfqpoint{2.488000in}{1.399896in}}%
\pgfpathlineto{\pgfqpoint{2.489240in}{1.404212in}}%
\pgfpathlineto{\pgfqpoint{2.490480in}{1.404163in}}%
\pgfpathlineto{\pgfqpoint{2.491720in}{1.405942in}}%
\pgfpathlineto{\pgfqpoint{2.499160in}{1.404209in}}%
\pgfpathlineto{\pgfqpoint{2.501640in}{1.405389in}}%
\pgfpathlineto{\pgfqpoint{2.502880in}{1.404646in}}%
\pgfpathlineto{\pgfqpoint{2.505360in}{1.405394in}}%
\pgfpathlineto{\pgfqpoint{2.509080in}{1.400935in}}%
\pgfpathlineto{\pgfqpoint{2.516520in}{1.403436in}}%
\pgfpathlineto{\pgfqpoint{2.520240in}{1.397856in}}%
\pgfpathlineto{\pgfqpoint{2.521480in}{1.398379in}}%
\pgfpathlineto{\pgfqpoint{2.523960in}{1.401306in}}%
\pgfpathlineto{\pgfqpoint{2.527680in}{1.400404in}}%
\pgfpathlineto{\pgfqpoint{2.530160in}{1.398601in}}%
\pgfpathlineto{\pgfqpoint{2.533880in}{1.399017in}}%
\pgfpathlineto{\pgfqpoint{2.536360in}{1.400814in}}%
\pgfpathlineto{\pgfqpoint{2.538840in}{1.395611in}}%
\pgfpathlineto{\pgfqpoint{2.545040in}{1.397915in}}%
\pgfpathlineto{\pgfqpoint{2.546280in}{1.397374in}}%
\pgfpathlineto{\pgfqpoint{2.547520in}{1.398970in}}%
\pgfpathlineto{\pgfqpoint{2.548760in}{1.398395in}}%
\pgfpathlineto{\pgfqpoint{2.550000in}{1.396304in}}%
\pgfpathlineto{\pgfqpoint{2.552480in}{1.399843in}}%
\pgfpathlineto{\pgfqpoint{2.558680in}{1.397977in}}%
\pgfpathlineto{\pgfqpoint{2.561160in}{1.399969in}}%
\pgfpathlineto{\pgfqpoint{2.566120in}{1.403590in}}%
\pgfpathlineto{\pgfqpoint{2.568600in}{1.402326in}}%
\pgfpathlineto{\pgfqpoint{2.576040in}{1.393777in}}%
\pgfpathlineto{\pgfqpoint{2.578520in}{1.392430in}}%
\pgfpathlineto{\pgfqpoint{2.584720in}{1.395549in}}%
\pgfpathlineto{\pgfqpoint{2.588440in}{1.397181in}}%
\pgfpathlineto{\pgfqpoint{2.589680in}{1.398152in}}%
\pgfpathlineto{\pgfqpoint{2.595880in}{1.391614in}}%
\pgfpathlineto{\pgfqpoint{2.599600in}{1.396829in}}%
\pgfpathlineto{\pgfqpoint{2.602080in}{1.396059in}}%
\pgfpathlineto{\pgfqpoint{2.612000in}{1.394312in}}%
\pgfpathlineto{\pgfqpoint{2.613240in}{1.396833in}}%
\pgfpathlineto{\pgfqpoint{2.614480in}{1.396611in}}%
\pgfpathlineto{\pgfqpoint{2.616960in}{1.398910in}}%
\pgfpathlineto{\pgfqpoint{2.619440in}{1.398948in}}%
\pgfpathlineto{\pgfqpoint{2.626880in}{1.399731in}}%
\pgfpathlineto{\pgfqpoint{2.629360in}{1.400588in}}%
\pgfpathlineto{\pgfqpoint{2.633080in}{1.395014in}}%
\pgfpathlineto{\pgfqpoint{2.636800in}{1.395805in}}%
\pgfpathlineto{\pgfqpoint{2.641760in}{1.394086in}}%
\pgfpathlineto{\pgfqpoint{2.644240in}{1.390838in}}%
\pgfpathlineto{\pgfqpoint{2.650440in}{1.395335in}}%
\pgfpathlineto{\pgfqpoint{2.652920in}{1.393813in}}%
\pgfpathlineto{\pgfqpoint{2.656640in}{1.391609in}}%
\pgfpathlineto{\pgfqpoint{2.660360in}{1.393853in}}%
\pgfpathlineto{\pgfqpoint{2.662840in}{1.388754in}}%
\pgfpathlineto{\pgfqpoint{2.666560in}{1.389545in}}%
\pgfpathlineto{\pgfqpoint{2.667800in}{1.389790in}}%
\pgfpathlineto{\pgfqpoint{2.669040in}{1.391182in}}%
\pgfpathlineto{\pgfqpoint{2.670280in}{1.390594in}}%
\pgfpathlineto{\pgfqpoint{2.671520in}{1.392676in}}%
\pgfpathlineto{\pgfqpoint{2.672760in}{1.392495in}}%
\pgfpathlineto{\pgfqpoint{2.674000in}{1.390870in}}%
\pgfpathlineto{\pgfqpoint{2.676480in}{1.394853in}}%
\pgfpathlineto{\pgfqpoint{2.682680in}{1.393113in}}%
\pgfpathlineto{\pgfqpoint{2.685160in}{1.395355in}}%
\pgfpathlineto{\pgfqpoint{2.690120in}{1.399203in}}%
\pgfpathlineto{\pgfqpoint{2.691360in}{1.399460in}}%
\pgfpathlineto{\pgfqpoint{2.695080in}{1.395147in}}%
\pgfpathlineto{\pgfqpoint{2.698800in}{1.390459in}}%
\pgfpathlineto{\pgfqpoint{2.703760in}{1.388584in}}%
\pgfpathlineto{\pgfqpoint{2.708720in}{1.392147in}}%
\pgfpathlineto{\pgfqpoint{2.714920in}{1.393028in}}%
\pgfpathlineto{\pgfqpoint{2.717400in}{1.392361in}}%
\pgfpathlineto{\pgfqpoint{2.719880in}{1.388338in}}%
\pgfpathlineto{\pgfqpoint{2.723600in}{1.393364in}}%
\pgfpathlineto{\pgfqpoint{2.726080in}{1.392952in}}%
\pgfpathlineto{\pgfqpoint{2.729800in}{1.391750in}}%
\pgfpathlineto{\pgfqpoint{2.732280in}{1.391461in}}%
\pgfpathlineto{\pgfqpoint{2.733520in}{1.391510in}}%
\pgfpathlineto{\pgfqpoint{2.734760in}{1.392945in}}%
\pgfpathlineto{\pgfqpoint{2.736000in}{1.391067in}}%
\pgfpathlineto{\pgfqpoint{2.739720in}{1.392706in}}%
\pgfpathlineto{\pgfqpoint{2.744680in}{1.394179in}}%
\pgfpathlineto{\pgfqpoint{2.747160in}{1.393543in}}%
\pgfpathlineto{\pgfqpoint{2.752120in}{1.396351in}}%
\pgfpathlineto{\pgfqpoint{2.753360in}{1.396593in}}%
\pgfpathlineto{\pgfqpoint{2.757080in}{1.390843in}}%
\pgfpathlineto{\pgfqpoint{2.760800in}{1.392230in}}%
\pgfpathlineto{\pgfqpoint{2.762040in}{1.391744in}}%
\pgfpathlineto{\pgfqpoint{2.764520in}{1.392670in}}%
\pgfpathlineto{\pgfqpoint{2.768240in}{1.387684in}}%
\pgfpathlineto{\pgfqpoint{2.769480in}{1.388309in}}%
\pgfpathlineto{\pgfqpoint{2.771960in}{1.391573in}}%
\pgfpathlineto{\pgfqpoint{2.775680in}{1.391266in}}%
\pgfpathlineto{\pgfqpoint{2.778160in}{1.389286in}}%
\pgfpathlineto{\pgfqpoint{2.779400in}{1.389306in}}%
\pgfpathlineto{\pgfqpoint{2.780640in}{1.387947in}}%
\pgfpathlineto{\pgfqpoint{2.784360in}{1.390330in}}%
\pgfpathlineto{\pgfqpoint{2.786840in}{1.385274in}}%
\pgfpathlineto{\pgfqpoint{2.789320in}{1.387396in}}%
\pgfpathlineto{\pgfqpoint{2.791800in}{1.386649in}}%
\pgfpathlineto{\pgfqpoint{2.793040in}{1.387596in}}%
\pgfpathlineto{\pgfqpoint{2.794280in}{1.386313in}}%
\pgfpathlineto{\pgfqpoint{2.796760in}{1.388225in}}%
\pgfpathlineto{\pgfqpoint{2.798000in}{1.387109in}}%
\pgfpathlineto{\pgfqpoint{2.800480in}{1.391101in}}%
\pgfpathlineto{\pgfqpoint{2.802960in}{1.390104in}}%
\pgfpathlineto{\pgfqpoint{2.804200in}{1.388049in}}%
\pgfpathlineto{\pgfqpoint{2.806680in}{1.388610in}}%
\pgfpathlineto{\pgfqpoint{2.807920in}{1.390800in}}%
\pgfpathlineto{\pgfqpoint{2.809160in}{1.390534in}}%
\pgfpathlineto{\pgfqpoint{2.814120in}{1.394636in}}%
\pgfpathlineto{\pgfqpoint{2.816600in}{1.392884in}}%
\pgfpathlineto{\pgfqpoint{2.824040in}{1.384259in}}%
\pgfpathlineto{\pgfqpoint{2.827760in}{1.382387in}}%
\pgfpathlineto{\pgfqpoint{2.833960in}{1.386960in}}%
\pgfpathlineto{\pgfqpoint{2.836440in}{1.386750in}}%
\pgfpathlineto{\pgfqpoint{2.837680in}{1.387163in}}%
\pgfpathlineto{\pgfqpoint{2.840160in}{1.385716in}}%
\pgfpathlineto{\pgfqpoint{2.841400in}{1.385354in}}%
\pgfpathlineto{\pgfqpoint{2.843880in}{1.381395in}}%
\pgfpathlineto{\pgfqpoint{2.848840in}{1.385465in}}%
\pgfpathlineto{\pgfqpoint{2.853800in}{1.384512in}}%
\pgfpathlineto{\pgfqpoint{2.856280in}{1.384855in}}%
\pgfpathlineto{\pgfqpoint{2.857520in}{1.384974in}}%
\pgfpathlineto{\pgfqpoint{2.858760in}{1.386394in}}%
\pgfpathlineto{\pgfqpoint{2.860000in}{1.384749in}}%
\pgfpathlineto{\pgfqpoint{2.861240in}{1.386042in}}%
\pgfpathlineto{\pgfqpoint{2.862480in}{1.385373in}}%
\pgfpathlineto{\pgfqpoint{2.864960in}{1.388257in}}%
\pgfpathlineto{\pgfqpoint{2.868680in}{1.389349in}}%
\pgfpathlineto{\pgfqpoint{2.871160in}{1.388563in}}%
\pgfpathlineto{\pgfqpoint{2.874880in}{1.390233in}}%
\pgfpathlineto{\pgfqpoint{2.877360in}{1.390400in}}%
\pgfpathlineto{\pgfqpoint{2.882320in}{1.385035in}}%
\pgfpathlineto{\pgfqpoint{2.884800in}{1.386471in}}%
\pgfpathlineto{\pgfqpoint{2.887280in}{1.386581in}}%
\pgfpathlineto{\pgfqpoint{2.888520in}{1.386642in}}%
\pgfpathlineto{\pgfqpoint{2.891000in}{1.383093in}}%
\pgfpathlineto{\pgfqpoint{2.892240in}{1.382287in}}%
\pgfpathlineto{\pgfqpoint{2.897200in}{1.386407in}}%
\pgfpathlineto{\pgfqpoint{2.900920in}{1.384581in}}%
\pgfpathlineto{\pgfqpoint{2.905880in}{1.382530in}}%
\pgfpathlineto{\pgfqpoint{2.908360in}{1.384470in}}%
\pgfpathlineto{\pgfqpoint{2.910840in}{1.379877in}}%
\pgfpathlineto{\pgfqpoint{2.914560in}{1.380535in}}%
\pgfpathlineto{\pgfqpoint{2.915800in}{1.380405in}}%
\pgfpathlineto{\pgfqpoint{2.917040in}{1.381496in}}%
\pgfpathlineto{\pgfqpoint{2.918280in}{1.380159in}}%
\pgfpathlineto{\pgfqpoint{2.920760in}{1.381607in}}%
\pgfpathlineto{\pgfqpoint{2.922000in}{1.380728in}}%
\pgfpathlineto{\pgfqpoint{2.924480in}{1.384277in}}%
\pgfpathlineto{\pgfqpoint{2.926960in}{1.383493in}}%
\pgfpathlineto{\pgfqpoint{2.929440in}{1.381912in}}%
\pgfpathlineto{\pgfqpoint{2.930680in}{1.382035in}}%
\pgfpathlineto{\pgfqpoint{2.931920in}{1.384068in}}%
\pgfpathlineto{\pgfqpoint{2.933160in}{1.383551in}}%
\pgfpathlineto{\pgfqpoint{2.938120in}{1.387044in}}%
\pgfpathlineto{\pgfqpoint{2.940600in}{1.384625in}}%
\pgfpathlineto{\pgfqpoint{2.946800in}{1.374989in}}%
\pgfpathlineto{\pgfqpoint{2.948040in}{1.375122in}}%
\pgfpathlineto{\pgfqpoint{2.951760in}{1.372292in}}%
\pgfpathlineto{\pgfqpoint{2.956720in}{1.375740in}}%
\pgfpathlineto{\pgfqpoint{2.957960in}{1.376181in}}%
\pgfpathlineto{\pgfqpoint{2.960440in}{1.375906in}}%
\pgfpathlineto{\pgfqpoint{2.965400in}{1.375927in}}%
\pgfpathlineto{\pgfqpoint{2.967880in}{1.372014in}}%
\pgfpathlineto{\pgfqpoint{2.970360in}{1.375649in}}%
\pgfpathlineto{\pgfqpoint{2.972840in}{1.376901in}}%
\pgfpathlineto{\pgfqpoint{2.981520in}{1.376433in}}%
\pgfpathlineto{\pgfqpoint{2.982760in}{1.377153in}}%
\pgfpathlineto{\pgfqpoint{2.985240in}{1.374710in}}%
\pgfpathlineto{\pgfqpoint{2.986480in}{1.374272in}}%
\pgfpathlineto{\pgfqpoint{2.988960in}{1.376971in}}%
\pgfpathlineto{\pgfqpoint{2.993920in}{1.377486in}}%
\pgfpathlineto{\pgfqpoint{2.995160in}{1.377343in}}%
\pgfpathlineto{\pgfqpoint{2.998880in}{1.379613in}}%
\pgfpathlineto{\pgfqpoint{3.001360in}{1.379590in}}%
\pgfpathlineto{\pgfqpoint{3.006320in}{1.373641in}}%
\pgfpathlineto{\pgfqpoint{3.008800in}{1.374765in}}%
\pgfpathlineto{\pgfqpoint{3.012520in}{1.375695in}}%
\pgfpathlineto{\pgfqpoint{3.015000in}{1.371906in}}%
\pgfpathlineto{\pgfqpoint{3.016240in}{1.371246in}}%
\pgfpathlineto{\pgfqpoint{3.021200in}{1.374427in}}%
\pgfpathlineto{\pgfqpoint{3.026160in}{1.371424in}}%
\pgfpathlineto{\pgfqpoint{3.029880in}{1.369071in}}%
\pgfpathlineto{\pgfqpoint{3.032360in}{1.370409in}}%
\pgfpathlineto{\pgfqpoint{3.034840in}{1.365746in}}%
\pgfpathlineto{\pgfqpoint{3.037320in}{1.367647in}}%
\pgfpathlineto{\pgfqpoint{3.039800in}{1.365899in}}%
\pgfpathlineto{\pgfqpoint{3.041040in}{1.367178in}}%
\pgfpathlineto{\pgfqpoint{3.042280in}{1.366661in}}%
\pgfpathlineto{\pgfqpoint{3.044760in}{1.368403in}}%
\pgfpathlineto{\pgfqpoint{3.046000in}{1.367811in}}%
\pgfpathlineto{\pgfqpoint{3.048480in}{1.371293in}}%
\pgfpathlineto{\pgfqpoint{3.050960in}{1.370597in}}%
\pgfpathlineto{\pgfqpoint{3.053440in}{1.369427in}}%
\pgfpathlineto{\pgfqpoint{3.054680in}{1.369601in}}%
\pgfpathlineto{\pgfqpoint{3.055920in}{1.371769in}}%
\pgfpathlineto{\pgfqpoint{3.057160in}{1.370816in}}%
\pgfpathlineto{\pgfqpoint{3.063360in}{1.374043in}}%
\pgfpathlineto{\pgfqpoint{3.065840in}{1.370044in}}%
\pgfpathlineto{\pgfqpoint{3.069560in}{1.363823in}}%
\pgfpathlineto{\pgfqpoint{3.070800in}{1.362850in}}%
\pgfpathlineto{\pgfqpoint{3.072040in}{1.363454in}}%
\pgfpathlineto{\pgfqpoint{3.075760in}{1.360498in}}%
\pgfpathlineto{\pgfqpoint{3.078240in}{1.362813in}}%
\pgfpathlineto{\pgfqpoint{3.079480in}{1.364936in}}%
\pgfpathlineto{\pgfqpoint{3.081960in}{1.364697in}}%
\pgfpathlineto{\pgfqpoint{3.084440in}{1.364612in}}%
\pgfpathlineto{\pgfqpoint{3.085680in}{1.364687in}}%
\pgfpathlineto{\pgfqpoint{3.088160in}{1.363827in}}%
\pgfpathlineto{\pgfqpoint{3.089400in}{1.364173in}}%
\pgfpathlineto{\pgfqpoint{3.091880in}{1.360555in}}%
\pgfpathlineto{\pgfqpoint{3.093120in}{1.361222in}}%
\pgfpathlineto{\pgfqpoint{3.095600in}{1.365456in}}%
\pgfpathlineto{\pgfqpoint{3.104280in}{1.365209in}}%
\pgfpathlineto{\pgfqpoint{3.106760in}{1.366064in}}%
\pgfpathlineto{\pgfqpoint{3.110480in}{1.361812in}}%
\pgfpathlineto{\pgfqpoint{3.114200in}{1.364313in}}%
\pgfpathlineto{\pgfqpoint{3.121640in}{1.365339in}}%
\pgfpathlineto{\pgfqpoint{3.125360in}{1.365728in}}%
\pgfpathlineto{\pgfqpoint{3.130320in}{1.360785in}}%
\pgfpathlineto{\pgfqpoint{3.132800in}{1.362305in}}%
\pgfpathlineto{\pgfqpoint{3.136520in}{1.362510in}}%
\pgfpathlineto{\pgfqpoint{3.139000in}{1.358601in}}%
\pgfpathlineto{\pgfqpoint{3.140240in}{1.357957in}}%
\pgfpathlineto{\pgfqpoint{3.143960in}{1.359975in}}%
\pgfpathlineto{\pgfqpoint{3.150160in}{1.357200in}}%
\pgfpathlineto{\pgfqpoint{3.153880in}{1.354935in}}%
\pgfpathlineto{\pgfqpoint{3.156360in}{1.356481in}}%
\pgfpathlineto{\pgfqpoint{3.158840in}{1.352620in}}%
\pgfpathlineto{\pgfqpoint{3.161320in}{1.353951in}}%
\pgfpathlineto{\pgfqpoint{3.163800in}{1.351944in}}%
\pgfpathlineto{\pgfqpoint{3.165040in}{1.353200in}}%
\pgfpathlineto{\pgfqpoint{3.166280in}{1.352459in}}%
\pgfpathlineto{\pgfqpoint{3.168760in}{1.353661in}}%
\pgfpathlineto{\pgfqpoint{3.170000in}{1.353153in}}%
\pgfpathlineto{\pgfqpoint{3.172480in}{1.356553in}}%
\pgfpathlineto{\pgfqpoint{3.174960in}{1.355648in}}%
\pgfpathlineto{\pgfqpoint{3.177440in}{1.353989in}}%
\pgfpathlineto{\pgfqpoint{3.178680in}{1.353884in}}%
\pgfpathlineto{\pgfqpoint{3.179920in}{1.356126in}}%
\pgfpathlineto{\pgfqpoint{3.181160in}{1.355224in}}%
\pgfpathlineto{\pgfqpoint{3.186120in}{1.358208in}}%
\pgfpathlineto{\pgfqpoint{3.188600in}{1.355015in}}%
\pgfpathlineto{\pgfqpoint{3.192320in}{1.347526in}}%
\pgfpathlineto{\pgfqpoint{3.199760in}{1.343404in}}%
\pgfpathlineto{\pgfqpoint{3.205960in}{1.346634in}}%
\pgfpathlineto{\pgfqpoint{3.209680in}{1.346830in}}%
\pgfpathlineto{\pgfqpoint{3.212160in}{1.345782in}}%
\pgfpathlineto{\pgfqpoint{3.213400in}{1.346346in}}%
\pgfpathlineto{\pgfqpoint{3.215880in}{1.342782in}}%
\pgfpathlineto{\pgfqpoint{3.217120in}{1.343560in}}%
\pgfpathlineto{\pgfqpoint{3.219600in}{1.347456in}}%
\pgfpathlineto{\pgfqpoint{3.229520in}{1.347228in}}%
\pgfpathlineto{\pgfqpoint{3.230760in}{1.348565in}}%
\pgfpathlineto{\pgfqpoint{3.234480in}{1.343605in}}%
\pgfpathlineto{\pgfqpoint{3.238200in}{1.347337in}}%
\pgfpathlineto{\pgfqpoint{3.240680in}{1.347564in}}%
\pgfpathlineto{\pgfqpoint{3.243160in}{1.345664in}}%
\pgfpathlineto{\pgfqpoint{3.248120in}{1.347998in}}%
\pgfpathlineto{\pgfqpoint{3.254320in}{1.341888in}}%
\pgfpathlineto{\pgfqpoint{3.256800in}{1.343128in}}%
\pgfpathlineto{\pgfqpoint{3.260520in}{1.344213in}}%
\pgfpathlineto{\pgfqpoint{3.263000in}{1.340140in}}%
\pgfpathlineto{\pgfqpoint{3.264240in}{1.339036in}}%
\pgfpathlineto{\pgfqpoint{3.269200in}{1.340444in}}%
\pgfpathlineto{\pgfqpoint{3.277880in}{1.336427in}}%
\pgfpathlineto{\pgfqpoint{3.279120in}{1.337871in}}%
\pgfpathlineto{\pgfqpoint{3.280360in}{1.337328in}}%
\pgfpathlineto{\pgfqpoint{3.282840in}{1.332622in}}%
\pgfpathlineto{\pgfqpoint{3.285320in}{1.333441in}}%
\pgfpathlineto{\pgfqpoint{3.287800in}{1.331279in}}%
\pgfpathlineto{\pgfqpoint{3.289040in}{1.332753in}}%
\pgfpathlineto{\pgfqpoint{3.291520in}{1.332312in}}%
\pgfpathlineto{\pgfqpoint{3.295240in}{1.334631in}}%
\pgfpathlineto{\pgfqpoint{3.297720in}{1.335627in}}%
\pgfpathlineto{\pgfqpoint{3.298960in}{1.334861in}}%
\pgfpathlineto{\pgfqpoint{3.301440in}{1.332626in}}%
\pgfpathlineto{\pgfqpoint{3.302680in}{1.332728in}}%
\pgfpathlineto{\pgfqpoint{3.303920in}{1.334834in}}%
\pgfpathlineto{\pgfqpoint{3.305160in}{1.334393in}}%
\pgfpathlineto{\pgfqpoint{3.310120in}{1.336321in}}%
\pgfpathlineto{\pgfqpoint{3.312600in}{1.333945in}}%
\pgfpathlineto{\pgfqpoint{3.318800in}{1.324404in}}%
\pgfpathlineto{\pgfqpoint{3.321280in}{1.324061in}}%
\pgfpathlineto{\pgfqpoint{3.323760in}{1.322109in}}%
\pgfpathlineto{\pgfqpoint{3.329960in}{1.326780in}}%
\pgfpathlineto{\pgfqpoint{3.338640in}{1.325557in}}%
\pgfpathlineto{\pgfqpoint{3.339880in}{1.324188in}}%
\pgfpathlineto{\pgfqpoint{3.341120in}{1.324963in}}%
\pgfpathlineto{\pgfqpoint{3.343600in}{1.328919in}}%
\pgfpathlineto{\pgfqpoint{3.347320in}{1.328863in}}%
\pgfpathlineto{\pgfqpoint{3.352280in}{1.329237in}}%
\pgfpathlineto{\pgfqpoint{3.354760in}{1.330947in}}%
\pgfpathlineto{\pgfqpoint{3.358480in}{1.325431in}}%
\pgfpathlineto{\pgfqpoint{3.363440in}{1.330092in}}%
\pgfpathlineto{\pgfqpoint{3.365920in}{1.328956in}}%
\pgfpathlineto{\pgfqpoint{3.368400in}{1.328615in}}%
\pgfpathlineto{\pgfqpoint{3.372120in}{1.329824in}}%
\pgfpathlineto{\pgfqpoint{3.378320in}{1.324339in}}%
\pgfpathlineto{\pgfqpoint{3.380800in}{1.325927in}}%
\pgfpathlineto{\pgfqpoint{3.383280in}{1.325905in}}%
\pgfpathlineto{\pgfqpoint{3.384520in}{1.326174in}}%
\pgfpathlineto{\pgfqpoint{3.387000in}{1.323204in}}%
\pgfpathlineto{\pgfqpoint{3.388240in}{1.322080in}}%
\pgfpathlineto{\pgfqpoint{3.391960in}{1.323992in}}%
\pgfpathlineto{\pgfqpoint{3.394440in}{1.323547in}}%
\pgfpathlineto{\pgfqpoint{3.395680in}{1.324369in}}%
\pgfpathlineto{\pgfqpoint{3.399400in}{1.322258in}}%
\pgfpathlineto{\pgfqpoint{3.401880in}{1.320505in}}%
\pgfpathlineto{\pgfqpoint{3.404360in}{1.321494in}}%
\pgfpathlineto{\pgfqpoint{3.406840in}{1.317169in}}%
\pgfpathlineto{\pgfqpoint{3.409320in}{1.318938in}}%
\pgfpathlineto{\pgfqpoint{3.411800in}{1.316417in}}%
\pgfpathlineto{\pgfqpoint{3.413040in}{1.317503in}}%
\pgfpathlineto{\pgfqpoint{3.414280in}{1.316436in}}%
\pgfpathlineto{\pgfqpoint{3.419240in}{1.319712in}}%
\pgfpathlineto{\pgfqpoint{3.421720in}{1.320859in}}%
\pgfpathlineto{\pgfqpoint{3.426680in}{1.316829in}}%
\pgfpathlineto{\pgfqpoint{3.427920in}{1.318788in}}%
\pgfpathlineto{\pgfqpoint{3.431640in}{1.318115in}}%
\pgfpathlineto{\pgfqpoint{3.434120in}{1.319204in}}%
\pgfpathlineto{\pgfqpoint{3.437840in}{1.314555in}}%
\pgfpathlineto{\pgfqpoint{3.441560in}{1.310020in}}%
\pgfpathlineto{\pgfqpoint{3.442800in}{1.308454in}}%
\pgfpathlineto{\pgfqpoint{3.445280in}{1.308781in}}%
\pgfpathlineto{\pgfqpoint{3.447760in}{1.307111in}}%
\pgfpathlineto{\pgfqpoint{3.453960in}{1.312799in}}%
\pgfpathlineto{\pgfqpoint{3.457680in}{1.314057in}}%
\pgfpathlineto{\pgfqpoint{3.458920in}{1.312729in}}%
\pgfpathlineto{\pgfqpoint{3.461400in}{1.313812in}}%
\pgfpathlineto{\pgfqpoint{3.463880in}{1.310306in}}%
\pgfpathlineto{\pgfqpoint{3.465120in}{1.311054in}}%
\pgfpathlineto{\pgfqpoint{3.467600in}{1.315226in}}%
\pgfpathlineto{\pgfqpoint{3.477520in}{1.316713in}}%
\pgfpathlineto{\pgfqpoint{3.478760in}{1.317848in}}%
\pgfpathlineto{\pgfqpoint{3.480000in}{1.315844in}}%
\pgfpathlineto{\pgfqpoint{3.482480in}{1.316900in}}%
\pgfpathlineto{\pgfqpoint{3.488680in}{1.320914in}}%
\pgfpathlineto{\pgfqpoint{3.492400in}{1.318959in}}%
\pgfpathlineto{\pgfqpoint{3.496120in}{1.320328in}}%
\pgfpathlineto{\pgfqpoint{3.502320in}{1.313833in}}%
\pgfpathlineto{\pgfqpoint{3.504800in}{1.314593in}}%
\pgfpathlineto{\pgfqpoint{3.511000in}{1.311587in}}%
\pgfpathlineto{\pgfqpoint{3.512240in}{1.310641in}}%
\pgfpathlineto{\pgfqpoint{3.515960in}{1.312707in}}%
\pgfpathlineto{\pgfqpoint{3.518440in}{1.312739in}}%
\pgfpathlineto{\pgfqpoint{3.519680in}{1.313510in}}%
\pgfpathlineto{\pgfqpoint{3.523400in}{1.309965in}}%
\pgfpathlineto{\pgfqpoint{3.524640in}{1.308169in}}%
\pgfpathlineto{\pgfqpoint{3.528360in}{1.310569in}}%
\pgfpathlineto{\pgfqpoint{3.530840in}{1.306442in}}%
\pgfpathlineto{\pgfqpoint{3.533320in}{1.307775in}}%
\pgfpathlineto{\pgfqpoint{3.538280in}{1.305477in}}%
\pgfpathlineto{\pgfqpoint{3.542000in}{1.305379in}}%
\pgfpathlineto{\pgfqpoint{3.544480in}{1.309249in}}%
\pgfpathlineto{\pgfqpoint{3.546960in}{1.308865in}}%
\pgfpathlineto{\pgfqpoint{3.549440in}{1.306142in}}%
\pgfpathlineto{\pgfqpoint{3.550680in}{1.305713in}}%
\pgfpathlineto{\pgfqpoint{3.553160in}{1.308249in}}%
\pgfpathlineto{\pgfqpoint{3.554400in}{1.308821in}}%
\pgfpathlineto{\pgfqpoint{3.555640in}{1.308165in}}%
\pgfpathlineto{\pgfqpoint{3.558120in}{1.308955in}}%
\pgfpathlineto{\pgfqpoint{3.561840in}{1.304009in}}%
\pgfpathlineto{\pgfqpoint{3.566800in}{1.298208in}}%
\pgfpathlineto{\pgfqpoint{3.569280in}{1.298684in}}%
\pgfpathlineto{\pgfqpoint{3.571760in}{1.296894in}}%
\pgfpathlineto{\pgfqpoint{3.574240in}{1.298884in}}%
\pgfpathlineto{\pgfqpoint{3.575480in}{1.300794in}}%
\pgfpathlineto{\pgfqpoint{3.576720in}{1.300506in}}%
\pgfpathlineto{\pgfqpoint{3.579200in}{1.302474in}}%
\pgfpathlineto{\pgfqpoint{3.581680in}{1.303217in}}%
\pgfpathlineto{\pgfqpoint{3.582920in}{1.301874in}}%
\pgfpathlineto{\pgfqpoint{3.585400in}{1.303542in}}%
\pgfpathlineto{\pgfqpoint{3.587880in}{1.300322in}}%
\pgfpathlineto{\pgfqpoint{3.589120in}{1.300627in}}%
\pgfpathlineto{\pgfqpoint{3.591600in}{1.305210in}}%
\pgfpathlineto{\pgfqpoint{3.601520in}{1.306867in}}%
\pgfpathlineto{\pgfqpoint{3.602760in}{1.307725in}}%
\pgfpathlineto{\pgfqpoint{3.604000in}{1.305862in}}%
\pgfpathlineto{\pgfqpoint{3.608960in}{1.308964in}}%
\pgfpathlineto{\pgfqpoint{3.611440in}{1.310382in}}%
\pgfpathlineto{\pgfqpoint{3.613920in}{1.309451in}}%
\pgfpathlineto{\pgfqpoint{3.616400in}{1.309136in}}%
\pgfpathlineto{\pgfqpoint{3.620120in}{1.310739in}}%
\pgfpathlineto{\pgfqpoint{3.626320in}{1.303638in}}%
\pgfpathlineto{\pgfqpoint{3.632520in}{1.304690in}}%
\pgfpathlineto{\pgfqpoint{3.635000in}{1.302254in}}%
\pgfpathlineto{\pgfqpoint{3.637480in}{1.301197in}}%
\pgfpathlineto{\pgfqpoint{3.641200in}{1.302131in}}%
\pgfpathlineto{\pgfqpoint{3.644920in}{1.302382in}}%
\pgfpathlineto{\pgfqpoint{3.649880in}{1.298042in}}%
\pgfpathlineto{\pgfqpoint{3.652360in}{1.299623in}}%
\pgfpathlineto{\pgfqpoint{3.654840in}{1.295917in}}%
\pgfpathlineto{\pgfqpoint{3.657320in}{1.296849in}}%
\pgfpathlineto{\pgfqpoint{3.659800in}{1.294458in}}%
\pgfpathlineto{\pgfqpoint{3.661040in}{1.295569in}}%
\pgfpathlineto{\pgfqpoint{3.663520in}{1.294389in}}%
\pgfpathlineto{\pgfqpoint{3.664760in}{1.294707in}}%
\pgfpathlineto{\pgfqpoint{3.666000in}{1.293448in}}%
\pgfpathlineto{\pgfqpoint{3.668480in}{1.296769in}}%
\pgfpathlineto{\pgfqpoint{3.670960in}{1.296978in}}%
\pgfpathlineto{\pgfqpoint{3.673440in}{1.294329in}}%
\pgfpathlineto{\pgfqpoint{3.674680in}{1.294204in}}%
\pgfpathlineto{\pgfqpoint{3.677160in}{1.297239in}}%
\pgfpathlineto{\pgfqpoint{3.682120in}{1.299395in}}%
\pgfpathlineto{\pgfqpoint{3.685840in}{1.295423in}}%
\pgfpathlineto{\pgfqpoint{3.690800in}{1.289823in}}%
\pgfpathlineto{\pgfqpoint{3.693280in}{1.290404in}}%
\pgfpathlineto{\pgfqpoint{3.695760in}{1.288337in}}%
\pgfpathlineto{\pgfqpoint{3.698240in}{1.290649in}}%
\pgfpathlineto{\pgfqpoint{3.700720in}{1.292313in}}%
\pgfpathlineto{\pgfqpoint{3.704440in}{1.293247in}}%
\pgfpathlineto{\pgfqpoint{3.710640in}{1.291597in}}%
\pgfpathlineto{\pgfqpoint{3.713120in}{1.291054in}}%
\pgfpathlineto{\pgfqpoint{3.716840in}{1.295498in}}%
\pgfpathlineto{\pgfqpoint{3.721800in}{1.296968in}}%
\pgfpathlineto{\pgfqpoint{3.725520in}{1.297639in}}%
\pgfpathlineto{\pgfqpoint{3.726760in}{1.298014in}}%
\pgfpathlineto{\pgfqpoint{3.728000in}{1.296263in}}%
\pgfpathlineto{\pgfqpoint{3.729240in}{1.297826in}}%
\pgfpathlineto{\pgfqpoint{3.730480in}{1.297410in}}%
\pgfpathlineto{\pgfqpoint{3.736680in}{1.301621in}}%
\pgfpathlineto{\pgfqpoint{3.740400in}{1.299555in}}%
\pgfpathlineto{\pgfqpoint{3.742880in}{1.300572in}}%
\pgfpathlineto{\pgfqpoint{3.744120in}{1.301370in}}%
\pgfpathlineto{\pgfqpoint{3.750320in}{1.293769in}}%
\pgfpathlineto{\pgfqpoint{3.754040in}{1.296113in}}%
\pgfpathlineto{\pgfqpoint{3.761480in}{1.293313in}}%
\pgfpathlineto{\pgfqpoint{3.767680in}{1.295850in}}%
\pgfpathlineto{\pgfqpoint{3.771400in}{1.291407in}}%
\pgfpathlineto{\pgfqpoint{3.772640in}{1.289928in}}%
\pgfpathlineto{\pgfqpoint{3.776360in}{1.292036in}}%
\pgfpathlineto{\pgfqpoint{3.778840in}{1.287938in}}%
\pgfpathlineto{\pgfqpoint{3.781320in}{1.288137in}}%
\pgfpathlineto{\pgfqpoint{3.783800in}{1.285897in}}%
\pgfpathlineto{\pgfqpoint{3.785040in}{1.287429in}}%
\pgfpathlineto{\pgfqpoint{3.787520in}{1.286332in}}%
\pgfpathlineto{\pgfqpoint{3.788760in}{1.286520in}}%
\pgfpathlineto{\pgfqpoint{3.790000in}{1.284970in}}%
\pgfpathlineto{\pgfqpoint{3.792480in}{1.288053in}}%
\pgfpathlineto{\pgfqpoint{3.794960in}{1.287373in}}%
\pgfpathlineto{\pgfqpoint{3.798680in}{1.284459in}}%
\pgfpathlineto{\pgfqpoint{3.801160in}{1.287341in}}%
\pgfpathlineto{\pgfqpoint{3.806120in}{1.289845in}}%
\pgfpathlineto{\pgfqpoint{3.811080in}{1.284820in}}%
\pgfpathlineto{\pgfqpoint{3.813560in}{1.281299in}}%
\pgfpathlineto{\pgfqpoint{3.814800in}{1.280140in}}%
\pgfpathlineto{\pgfqpoint{3.817280in}{1.280252in}}%
\pgfpathlineto{\pgfqpoint{3.819760in}{1.277323in}}%
\pgfpathlineto{\pgfqpoint{3.822240in}{1.279579in}}%
\pgfpathlineto{\pgfqpoint{3.823480in}{1.281361in}}%
\pgfpathlineto{\pgfqpoint{3.825960in}{1.281836in}}%
\pgfpathlineto{\pgfqpoint{3.829680in}{1.282179in}}%
\pgfpathlineto{\pgfqpoint{3.830920in}{1.280953in}}%
\pgfpathlineto{\pgfqpoint{3.833400in}{1.282879in}}%
\pgfpathlineto{\pgfqpoint{3.835880in}{1.279529in}}%
\pgfpathlineto{\pgfqpoint{3.837120in}{1.280165in}}%
\pgfpathlineto{\pgfqpoint{3.840840in}{1.285316in}}%
\pgfpathlineto{\pgfqpoint{3.844560in}{1.286046in}}%
\pgfpathlineto{\pgfqpoint{3.849520in}{1.288437in}}%
\pgfpathlineto{\pgfqpoint{3.850760in}{1.288575in}}%
\pgfpathlineto{\pgfqpoint{3.852000in}{1.286859in}}%
\pgfpathlineto{\pgfqpoint{3.853240in}{1.289003in}}%
\pgfpathlineto{\pgfqpoint{3.854480in}{1.288334in}}%
\pgfpathlineto{\pgfqpoint{3.860680in}{1.293065in}}%
\pgfpathlineto{\pgfqpoint{3.864400in}{1.291719in}}%
\pgfpathlineto{\pgfqpoint{3.869360in}{1.291337in}}%
\pgfpathlineto{\pgfqpoint{3.873080in}{1.285647in}}%
\pgfpathlineto{\pgfqpoint{3.875560in}{1.285525in}}%
\pgfpathlineto{\pgfqpoint{3.878040in}{1.286662in}}%
\pgfpathlineto{\pgfqpoint{3.886720in}{1.284224in}}%
\pgfpathlineto{\pgfqpoint{3.892920in}{1.286187in}}%
\pgfpathlineto{\pgfqpoint{3.896640in}{1.281904in}}%
\pgfpathlineto{\pgfqpoint{3.900360in}{1.283622in}}%
\pgfpathlineto{\pgfqpoint{3.902840in}{1.278912in}}%
\pgfpathlineto{\pgfqpoint{3.905320in}{1.279460in}}%
\pgfpathlineto{\pgfqpoint{3.907800in}{1.276903in}}%
\pgfpathlineto{\pgfqpoint{3.909040in}{1.278192in}}%
\pgfpathlineto{\pgfqpoint{3.911520in}{1.277084in}}%
\pgfpathlineto{\pgfqpoint{3.912760in}{1.277011in}}%
\pgfpathlineto{\pgfqpoint{3.914000in}{1.275050in}}%
\pgfpathlineto{\pgfqpoint{3.916480in}{1.277897in}}%
\pgfpathlineto{\pgfqpoint{3.918960in}{1.276857in}}%
\pgfpathlineto{\pgfqpoint{3.922680in}{1.274230in}}%
\pgfpathlineto{\pgfqpoint{3.925160in}{1.277517in}}%
\pgfpathlineto{\pgfqpoint{3.928880in}{1.279680in}}%
\pgfpathlineto{\pgfqpoint{3.930120in}{1.279906in}}%
\pgfpathlineto{\pgfqpoint{3.935080in}{1.275482in}}%
\pgfpathlineto{\pgfqpoint{3.937560in}{1.271947in}}%
\pgfpathlineto{\pgfqpoint{3.938800in}{1.271090in}}%
\pgfpathlineto{\pgfqpoint{3.941280in}{1.271949in}}%
\pgfpathlineto{\pgfqpoint{3.943760in}{1.269280in}}%
\pgfpathlineto{\pgfqpoint{3.945000in}{1.270007in}}%
\pgfpathlineto{\pgfqpoint{3.947480in}{1.274219in}}%
\pgfpathlineto{\pgfqpoint{3.951200in}{1.274763in}}%
\pgfpathlineto{\pgfqpoint{3.953680in}{1.274887in}}%
\pgfpathlineto{\pgfqpoint{3.954920in}{1.273585in}}%
\pgfpathlineto{\pgfqpoint{3.957400in}{1.275159in}}%
\pgfpathlineto{\pgfqpoint{3.959880in}{1.272259in}}%
\pgfpathlineto{\pgfqpoint{3.961120in}{1.273054in}}%
\pgfpathlineto{\pgfqpoint{3.964840in}{1.278484in}}%
\pgfpathlineto{\pgfqpoint{3.966080in}{1.278257in}}%
\pgfpathlineto{\pgfqpoint{3.967320in}{1.279325in}}%
\pgfpathlineto{\pgfqpoint{3.968560in}{1.278919in}}%
\pgfpathlineto{\pgfqpoint{3.974760in}{1.282305in}}%
\pgfpathlineto{\pgfqpoint{3.976000in}{1.280192in}}%
\pgfpathlineto{\pgfqpoint{3.977240in}{1.281239in}}%
\pgfpathlineto{\pgfqpoint{3.978480in}{1.280419in}}%
\pgfpathlineto{\pgfqpoint{3.984680in}{1.285720in}}%
\pgfpathlineto{\pgfqpoint{3.988400in}{1.284035in}}%
\pgfpathlineto{\pgfqpoint{3.989640in}{1.284340in}}%
\pgfpathlineto{\pgfqpoint{3.992120in}{1.286484in}}%
\pgfpathlineto{\pgfqpoint{3.998320in}{1.278230in}}%
\pgfpathlineto{\pgfqpoint{4.005760in}{1.278469in}}%
\pgfpathlineto{\pgfqpoint{4.007000in}{1.278253in}}%
\pgfpathlineto{\pgfqpoint{4.009480in}{1.276492in}}%
\pgfpathlineto{\pgfqpoint{4.016920in}{1.278334in}}%
\pgfpathlineto{\pgfqpoint{4.020640in}{1.275194in}}%
\pgfpathlineto{\pgfqpoint{4.024360in}{1.277763in}}%
\pgfpathlineto{\pgfqpoint{4.026840in}{1.274195in}}%
\pgfpathlineto{\pgfqpoint{4.029320in}{1.273693in}}%
\pgfpathlineto{\pgfqpoint{4.031800in}{1.271377in}}%
\pgfpathlineto{\pgfqpoint{4.033040in}{1.272081in}}%
\pgfpathlineto{\pgfqpoint{4.035520in}{1.270523in}}%
\pgfpathlineto{\pgfqpoint{4.036760in}{1.270304in}}%
\pgfpathlineto{\pgfqpoint{4.038000in}{1.268408in}}%
\pgfpathlineto{\pgfqpoint{4.039240in}{1.270737in}}%
\pgfpathlineto{\pgfqpoint{4.044200in}{1.269575in}}%
\pgfpathlineto{\pgfqpoint{4.046680in}{1.267092in}}%
\pgfpathlineto{\pgfqpoint{4.049160in}{1.269876in}}%
\pgfpathlineto{\pgfqpoint{4.051640in}{1.270679in}}%
\pgfpathlineto{\pgfqpoint{4.054120in}{1.270955in}}%
\pgfpathlineto{\pgfqpoint{4.057840in}{1.266680in}}%
\pgfpathlineto{\pgfqpoint{4.059080in}{1.266443in}}%
\pgfpathlineto{\pgfqpoint{4.061560in}{1.263466in}}%
\pgfpathlineto{\pgfqpoint{4.064040in}{1.263285in}}%
\pgfpathlineto{\pgfqpoint{4.065280in}{1.263394in}}%
\pgfpathlineto{\pgfqpoint{4.067760in}{1.260639in}}%
\pgfpathlineto{\pgfqpoint{4.069000in}{1.261520in}}%
\pgfpathlineto{\pgfqpoint{4.071480in}{1.266658in}}%
\pgfpathlineto{\pgfqpoint{4.080160in}{1.266266in}}%
\pgfpathlineto{\pgfqpoint{4.081400in}{1.266707in}}%
\pgfpathlineto{\pgfqpoint{4.083880in}{1.264023in}}%
\pgfpathlineto{\pgfqpoint{4.085120in}{1.264393in}}%
\pgfpathlineto{\pgfqpoint{4.088840in}{1.269595in}}%
\pgfpathlineto{\pgfqpoint{4.090080in}{1.269430in}}%
\pgfpathlineto{\pgfqpoint{4.091320in}{1.270904in}}%
\pgfpathlineto{\pgfqpoint{4.092560in}{1.270634in}}%
\pgfpathlineto{\pgfqpoint{4.098760in}{1.274805in}}%
\pgfpathlineto{\pgfqpoint{4.100000in}{1.272697in}}%
\pgfpathlineto{\pgfqpoint{4.101240in}{1.273324in}}%
\pgfpathlineto{\pgfqpoint{4.102480in}{1.272039in}}%
\pgfpathlineto{\pgfqpoint{4.104960in}{1.273874in}}%
\pgfpathlineto{\pgfqpoint{4.108680in}{1.275685in}}%
\pgfpathlineto{\pgfqpoint{4.111160in}{1.272802in}}%
\pgfpathlineto{\pgfqpoint{4.113640in}{1.274132in}}%
\pgfpathlineto{\pgfqpoint{4.116120in}{1.276367in}}%
\pgfpathlineto{\pgfqpoint{4.121080in}{1.268810in}}%
\pgfpathlineto{\pgfqpoint{4.123560in}{1.268056in}}%
\pgfpathlineto{\pgfqpoint{4.126040in}{1.269528in}}%
\pgfpathlineto{\pgfqpoint{4.127280in}{1.269233in}}%
\pgfpathlineto{\pgfqpoint{4.128520in}{1.270127in}}%
\pgfpathlineto{\pgfqpoint{4.131000in}{1.268523in}}%
\pgfpathlineto{\pgfqpoint{4.133480in}{1.266783in}}%
\pgfpathlineto{\pgfqpoint{4.140920in}{1.268328in}}%
\pgfpathlineto{\pgfqpoint{4.142160in}{1.267520in}}%
\pgfpathlineto{\pgfqpoint{4.144640in}{1.264367in}}%
\pgfpathlineto{\pgfqpoint{4.148360in}{1.268441in}}%
\pgfpathlineto{\pgfqpoint{4.150840in}{1.264678in}}%
\pgfpathlineto{\pgfqpoint{4.153320in}{1.264762in}}%
\pgfpathlineto{\pgfqpoint{4.155800in}{1.262434in}}%
\pgfpathlineto{\pgfqpoint{4.157040in}{1.263081in}}%
\pgfpathlineto{\pgfqpoint{4.159520in}{1.261412in}}%
\pgfpathlineto{\pgfqpoint{4.160760in}{1.261250in}}%
\pgfpathlineto{\pgfqpoint{4.162000in}{1.259689in}}%
\pgfpathlineto{\pgfqpoint{4.164480in}{1.262625in}}%
\pgfpathlineto{\pgfqpoint{4.166960in}{1.261886in}}%
\pgfpathlineto{\pgfqpoint{4.169440in}{1.260503in}}%
\pgfpathlineto{\pgfqpoint{4.170680in}{1.259032in}}%
\pgfpathlineto{\pgfqpoint{4.173160in}{1.261624in}}%
\pgfpathlineto{\pgfqpoint{4.176880in}{1.262487in}}%
\pgfpathlineto{\pgfqpoint{4.178120in}{1.262273in}}%
\pgfpathlineto{\pgfqpoint{4.181840in}{1.258394in}}%
\pgfpathlineto{\pgfqpoint{4.183080in}{1.257886in}}%
\pgfpathlineto{\pgfqpoint{4.185560in}{1.254410in}}%
\pgfpathlineto{\pgfqpoint{4.188040in}{1.254392in}}%
\pgfpathlineto{\pgfqpoint{4.189280in}{1.254992in}}%
\pgfpathlineto{\pgfqpoint{4.191760in}{1.252355in}}%
\pgfpathlineto{\pgfqpoint{4.193000in}{1.253201in}}%
\pgfpathlineto{\pgfqpoint{4.195480in}{1.258641in}}%
\pgfpathlineto{\pgfqpoint{4.204160in}{1.257672in}}%
\pgfpathlineto{\pgfqpoint{4.205400in}{1.258396in}}%
\pgfpathlineto{\pgfqpoint{4.207880in}{1.255750in}}%
\pgfpathlineto{\pgfqpoint{4.209120in}{1.255809in}}%
\pgfpathlineto{\pgfqpoint{4.211600in}{1.259930in}}%
\pgfpathlineto{\pgfqpoint{4.214080in}{1.258984in}}%
\pgfpathlineto{\pgfqpoint{4.215320in}{1.260617in}}%
\pgfpathlineto{\pgfqpoint{4.216560in}{1.260018in}}%
\pgfpathlineto{\pgfqpoint{4.221520in}{1.264253in}}%
\pgfpathlineto{\pgfqpoint{4.222760in}{1.264783in}}%
\pgfpathlineto{\pgfqpoint{4.225240in}{1.262656in}}%
\pgfpathlineto{\pgfqpoint{4.226480in}{1.260955in}}%
\pgfpathlineto{\pgfqpoint{4.232680in}{1.264033in}}%
\pgfpathlineto{\pgfqpoint{4.235160in}{1.262124in}}%
\pgfpathlineto{\pgfqpoint{4.240120in}{1.266502in}}%
\pgfpathlineto{\pgfqpoint{4.245080in}{1.259411in}}%
\pgfpathlineto{\pgfqpoint{4.247560in}{1.259076in}}%
\pgfpathlineto{\pgfqpoint{4.250040in}{1.261105in}}%
\pgfpathlineto{\pgfqpoint{4.258720in}{1.258252in}}%
\pgfpathlineto{\pgfqpoint{4.261200in}{1.258827in}}%
\pgfpathlineto{\pgfqpoint{4.264920in}{1.258445in}}%
\pgfpathlineto{\pgfqpoint{4.266160in}{1.257730in}}%
\pgfpathlineto{\pgfqpoint{4.268640in}{1.254850in}}%
\pgfpathlineto{\pgfqpoint{4.272360in}{1.259669in}}%
\pgfpathlineto{\pgfqpoint{4.274840in}{1.255646in}}%
\pgfpathlineto{\pgfqpoint{4.276080in}{1.256454in}}%
\pgfpathlineto{\pgfqpoint{4.279800in}{1.253332in}}%
\pgfpathlineto{\pgfqpoint{4.281040in}{1.254907in}}%
\pgfpathlineto{\pgfqpoint{4.286000in}{1.252175in}}%
\pgfpathlineto{\pgfqpoint{4.287240in}{1.254738in}}%
\pgfpathlineto{\pgfqpoint{4.293440in}{1.253308in}}%
\pgfpathlineto{\pgfqpoint{4.294680in}{1.251902in}}%
\pgfpathlineto{\pgfqpoint{4.297160in}{1.254239in}}%
\pgfpathlineto{\pgfqpoint{4.300880in}{1.255397in}}%
\pgfpathlineto{\pgfqpoint{4.302120in}{1.255240in}}%
\pgfpathlineto{\pgfqpoint{4.308320in}{1.246734in}}%
\pgfpathlineto{\pgfqpoint{4.310800in}{1.245301in}}%
\pgfpathlineto{\pgfqpoint{4.313280in}{1.246164in}}%
\pgfpathlineto{\pgfqpoint{4.315760in}{1.243888in}}%
\pgfpathlineto{\pgfqpoint{4.317000in}{1.244655in}}%
\pgfpathlineto{\pgfqpoint{4.320720in}{1.249020in}}%
\pgfpathlineto{\pgfqpoint{4.326920in}{1.246857in}}%
\pgfpathlineto{\pgfqpoint{4.329400in}{1.248738in}}%
\pgfpathlineto{\pgfqpoint{4.331880in}{1.246356in}}%
\pgfpathlineto{\pgfqpoint{4.333120in}{1.246586in}}%
\pgfpathlineto{\pgfqpoint{4.335600in}{1.250318in}}%
\pgfpathlineto{\pgfqpoint{4.338080in}{1.249083in}}%
\pgfpathlineto{\pgfqpoint{4.339320in}{1.251213in}}%
\pgfpathlineto{\pgfqpoint{4.340560in}{1.250544in}}%
\pgfpathlineto{\pgfqpoint{4.345520in}{1.254559in}}%
\pgfpathlineto{\pgfqpoint{4.346760in}{1.255042in}}%
\pgfpathlineto{\pgfqpoint{4.348000in}{1.252510in}}%
\pgfpathlineto{\pgfqpoint{4.349240in}{1.254864in}}%
\pgfpathlineto{\pgfqpoint{4.350480in}{1.252441in}}%
\pgfpathlineto{\pgfqpoint{4.351720in}{1.252802in}}%
\pgfpathlineto{\pgfqpoint{4.354200in}{1.255481in}}%
\pgfpathlineto{\pgfqpoint{4.361640in}{1.256392in}}%
\pgfpathlineto{\pgfqpoint{4.364120in}{1.258896in}}%
\pgfpathlineto{\pgfqpoint{4.369080in}{1.251215in}}%
\pgfpathlineto{\pgfqpoint{4.371560in}{1.251913in}}%
\pgfpathlineto{\pgfqpoint{4.374040in}{1.253434in}}%
\pgfpathlineto{\pgfqpoint{4.376520in}{1.253407in}}%
\pgfpathlineto{\pgfqpoint{4.377760in}{1.251575in}}%
\pgfpathlineto{\pgfqpoint{4.379000in}{1.251777in}}%
\pgfpathlineto{\pgfqpoint{4.381480in}{1.249648in}}%
\pgfpathlineto{\pgfqpoint{4.383960in}{1.250543in}}%
\pgfpathlineto{\pgfqpoint{4.386440in}{1.248850in}}%
\pgfpathlineto{\pgfqpoint{4.388920in}{1.248428in}}%
\pgfpathlineto{\pgfqpoint{4.390160in}{1.247947in}}%
\pgfpathlineto{\pgfqpoint{4.391400in}{1.245867in}}%
\pgfpathlineto{\pgfqpoint{4.392640in}{1.246400in}}%
\pgfpathlineto{\pgfqpoint{4.396360in}{1.251548in}}%
\pgfpathlineto{\pgfqpoint{4.398840in}{1.247671in}}%
\pgfpathlineto{\pgfqpoint{4.400080in}{1.248667in}}%
\pgfpathlineto{\pgfqpoint{4.401320in}{1.248226in}}%
\pgfpathlineto{\pgfqpoint{4.403800in}{1.245433in}}%
\pgfpathlineto{\pgfqpoint{4.405040in}{1.247505in}}%
\pgfpathlineto{\pgfqpoint{4.410000in}{1.243485in}}%
\pgfpathlineto{\pgfqpoint{4.411240in}{1.246002in}}%
\pgfpathlineto{\pgfqpoint{4.413720in}{1.245069in}}%
\pgfpathlineto{\pgfqpoint{4.417440in}{1.244429in}}%
\pgfpathlineto{\pgfqpoint{4.418680in}{1.243450in}}%
\pgfpathlineto{\pgfqpoint{4.421160in}{1.246073in}}%
\pgfpathlineto{\pgfqpoint{4.424880in}{1.246342in}}%
\pgfpathlineto{\pgfqpoint{4.426120in}{1.246179in}}%
\pgfpathlineto{\pgfqpoint{4.433560in}{1.236863in}}%
\pgfpathlineto{\pgfqpoint{4.434800in}{1.236468in}}%
\pgfpathlineto{\pgfqpoint{4.437280in}{1.237763in}}%
\pgfpathlineto{\pgfqpoint{4.439760in}{1.236401in}}%
\pgfpathlineto{\pgfqpoint{4.441000in}{1.237169in}}%
\pgfpathlineto{\pgfqpoint{4.444720in}{1.242083in}}%
\pgfpathlineto{\pgfqpoint{4.450920in}{1.240354in}}%
\pgfpathlineto{\pgfqpoint{4.453400in}{1.242693in}}%
\pgfpathlineto{\pgfqpoint{4.455880in}{1.239727in}}%
\pgfpathlineto{\pgfqpoint{4.457120in}{1.240011in}}%
\pgfpathlineto{\pgfqpoint{4.459600in}{1.243439in}}%
\pgfpathlineto{\pgfqpoint{4.462080in}{1.242680in}}%
\pgfpathlineto{\pgfqpoint{4.463320in}{1.245149in}}%
\pgfpathlineto{\pgfqpoint{4.464560in}{1.244146in}}%
\pgfpathlineto{\pgfqpoint{4.469520in}{1.247133in}}%
\pgfpathlineto{\pgfqpoint{4.470760in}{1.247383in}}%
\pgfpathlineto{\pgfqpoint{4.472000in}{1.245326in}}%
\pgfpathlineto{\pgfqpoint{4.473240in}{1.246302in}}%
\pgfpathlineto{\pgfqpoint{4.475720in}{1.243657in}}%
\pgfpathlineto{\pgfqpoint{4.478200in}{1.245475in}}%
\pgfpathlineto{\pgfqpoint{4.485640in}{1.246626in}}%
\pgfpathlineto{\pgfqpoint{4.488120in}{1.248384in}}%
\pgfpathlineto{\pgfqpoint{4.493080in}{1.241594in}}%
\pgfpathlineto{\pgfqpoint{4.494320in}{1.241473in}}%
\pgfpathlineto{\pgfqpoint{4.499280in}{1.244806in}}%
\pgfpathlineto{\pgfqpoint{4.500520in}{1.244787in}}%
\pgfpathlineto{\pgfqpoint{4.501760in}{1.242920in}}%
\pgfpathlineto{\pgfqpoint{4.503000in}{1.243551in}}%
\pgfpathlineto{\pgfqpoint{4.505480in}{1.241173in}}%
\pgfpathlineto{\pgfqpoint{4.507960in}{1.242567in}}%
\pgfpathlineto{\pgfqpoint{4.510440in}{1.240738in}}%
\pgfpathlineto{\pgfqpoint{4.512920in}{1.240783in}}%
\pgfpathlineto{\pgfqpoint{4.516640in}{1.238426in}}%
\pgfpathlineto{\pgfqpoint{4.519120in}{1.242020in}}%
\pgfpathlineto{\pgfqpoint{4.520360in}{1.243032in}}%
\pgfpathlineto{\pgfqpoint{4.522840in}{1.239129in}}%
\pgfpathlineto{\pgfqpoint{4.525320in}{1.239463in}}%
\pgfpathlineto{\pgfqpoint{4.527800in}{1.236136in}}%
\pgfpathlineto{\pgfqpoint{4.529040in}{1.238230in}}%
\pgfpathlineto{\pgfqpoint{4.534000in}{1.234371in}}%
\pgfpathlineto{\pgfqpoint{4.535240in}{1.237407in}}%
\pgfpathlineto{\pgfqpoint{4.538960in}{1.236823in}}%
\pgfpathlineto{\pgfqpoint{4.541440in}{1.236532in}}%
\pgfpathlineto{\pgfqpoint{4.542680in}{1.235547in}}%
\pgfpathlineto{\pgfqpoint{4.545160in}{1.237927in}}%
\pgfpathlineto{\pgfqpoint{4.547640in}{1.238292in}}%
\pgfpathlineto{\pgfqpoint{4.552600in}{1.235043in}}%
\pgfpathlineto{\pgfqpoint{4.557560in}{1.229837in}}%
\pgfpathlineto{\pgfqpoint{4.562520in}{1.229150in}}%
\pgfpathlineto{\pgfqpoint{4.563760in}{1.228782in}}%
\pgfpathlineto{\pgfqpoint{4.566240in}{1.231774in}}%
\pgfpathlineto{\pgfqpoint{4.567480in}{1.234246in}}%
\pgfpathlineto{\pgfqpoint{4.571200in}{1.233192in}}%
\pgfpathlineto{\pgfqpoint{4.573680in}{1.234443in}}%
\pgfpathlineto{\pgfqpoint{4.574920in}{1.232133in}}%
\pgfpathlineto{\pgfqpoint{4.577400in}{1.233728in}}%
\pgfpathlineto{\pgfqpoint{4.581120in}{1.230634in}}%
\pgfpathlineto{\pgfqpoint{4.583600in}{1.233638in}}%
\pgfpathlineto{\pgfqpoint{4.586080in}{1.232862in}}%
\pgfpathlineto{\pgfqpoint{4.587320in}{1.235457in}}%
\pgfpathlineto{\pgfqpoint{4.588560in}{1.234337in}}%
\pgfpathlineto{\pgfqpoint{4.594760in}{1.237831in}}%
\pgfpathlineto{\pgfqpoint{4.596000in}{1.235683in}}%
\pgfpathlineto{\pgfqpoint{4.597240in}{1.236851in}}%
\pgfpathlineto{\pgfqpoint{4.599720in}{1.234679in}}%
\pgfpathlineto{\pgfqpoint{4.602200in}{1.236444in}}%
\pgfpathlineto{\pgfqpoint{4.607160in}{1.235524in}}%
\pgfpathlineto{\pgfqpoint{4.610880in}{1.238636in}}%
\pgfpathlineto{\pgfqpoint{4.612120in}{1.239477in}}%
\pgfpathlineto{\pgfqpoint{4.618320in}{1.231727in}}%
\pgfpathlineto{\pgfqpoint{4.623280in}{1.235233in}}%
\pgfpathlineto{\pgfqpoint{4.624520in}{1.235072in}}%
\pgfpathlineto{\pgfqpoint{4.625760in}{1.232616in}}%
\pgfpathlineto{\pgfqpoint{4.627000in}{1.233240in}}%
\pgfpathlineto{\pgfqpoint{4.629480in}{1.231308in}}%
\pgfpathlineto{\pgfqpoint{4.631960in}{1.232901in}}%
\pgfpathlineto{\pgfqpoint{4.634440in}{1.230777in}}%
\pgfpathlineto{\pgfqpoint{4.638160in}{1.230294in}}%
\pgfpathlineto{\pgfqpoint{4.639400in}{1.228439in}}%
\pgfpathlineto{\pgfqpoint{4.640640in}{1.229154in}}%
\pgfpathlineto{\pgfqpoint{4.643120in}{1.232304in}}%
\pgfpathlineto{\pgfqpoint{4.644360in}{1.233083in}}%
\pgfpathlineto{\pgfqpoint{4.646840in}{1.229165in}}%
\pgfpathlineto{\pgfqpoint{4.649320in}{1.229994in}}%
\pgfpathlineto{\pgfqpoint{4.651800in}{1.227252in}}%
\pgfpathlineto{\pgfqpoint{4.653040in}{1.229222in}}%
\pgfpathlineto{\pgfqpoint{4.655520in}{1.227067in}}%
\pgfpathlineto{\pgfqpoint{4.656760in}{1.226917in}}%
\pgfpathlineto{\pgfqpoint{4.658000in}{1.225298in}}%
\pgfpathlineto{\pgfqpoint{4.659240in}{1.228333in}}%
\pgfpathlineto{\pgfqpoint{4.667920in}{1.228852in}}%
\pgfpathlineto{\pgfqpoint{4.670400in}{1.229719in}}%
\pgfpathlineto{\pgfqpoint{4.675360in}{1.226478in}}%
\pgfpathlineto{\pgfqpoint{4.681560in}{1.219319in}}%
\pgfpathlineto{\pgfqpoint{4.685280in}{1.219653in}}%
\pgfpathlineto{\pgfqpoint{4.687760in}{1.218865in}}%
\pgfpathlineto{\pgfqpoint{4.690240in}{1.222484in}}%
\pgfpathlineto{\pgfqpoint{4.691480in}{1.224717in}}%
\pgfpathlineto{\pgfqpoint{4.693960in}{1.224032in}}%
\pgfpathlineto{\pgfqpoint{4.695200in}{1.223947in}}%
\pgfpathlineto{\pgfqpoint{4.697680in}{1.226091in}}%
\pgfpathlineto{\pgfqpoint{4.698920in}{1.224278in}}%
\pgfpathlineto{\pgfqpoint{4.701400in}{1.226166in}}%
\pgfpathlineto{\pgfqpoint{4.703880in}{1.223087in}}%
\pgfpathlineto{\pgfqpoint{4.705120in}{1.223069in}}%
\pgfpathlineto{\pgfqpoint{4.708840in}{1.225702in}}%
\pgfpathlineto{\pgfqpoint{4.710080in}{1.224955in}}%
\pgfpathlineto{\pgfqpoint{4.711320in}{1.227279in}}%
\pgfpathlineto{\pgfqpoint{4.712560in}{1.226300in}}%
\pgfpathlineto{\pgfqpoint{4.718760in}{1.230254in}}%
\pgfpathlineto{\pgfqpoint{4.722480in}{1.225315in}}%
\pgfpathlineto{\pgfqpoint{4.723720in}{1.225190in}}%
\pgfpathlineto{\pgfqpoint{4.726200in}{1.227061in}}%
\pgfpathlineto{\pgfqpoint{4.728680in}{1.227612in}}%
\pgfpathlineto{\pgfqpoint{4.731160in}{1.225940in}}%
\pgfpathlineto{\pgfqpoint{4.734880in}{1.230731in}}%
\pgfpathlineto{\pgfqpoint{4.736120in}{1.231829in}}%
\pgfpathlineto{\pgfqpoint{4.742320in}{1.224482in}}%
\pgfpathlineto{\pgfqpoint{4.748520in}{1.226915in}}%
\pgfpathlineto{\pgfqpoint{4.749760in}{1.224616in}}%
\pgfpathlineto{\pgfqpoint{4.751000in}{1.225012in}}%
\pgfpathlineto{\pgfqpoint{4.753480in}{1.223198in}}%
\pgfpathlineto{\pgfqpoint{4.755960in}{1.225466in}}%
\pgfpathlineto{\pgfqpoint{4.759680in}{1.222746in}}%
\pgfpathlineto{\pgfqpoint{4.762160in}{1.221873in}}%
\pgfpathlineto{\pgfqpoint{4.763400in}{1.219529in}}%
\pgfpathlineto{\pgfqpoint{4.764640in}{1.219885in}}%
\pgfpathlineto{\pgfqpoint{4.767120in}{1.222842in}}%
\pgfpathlineto{\pgfqpoint{4.768360in}{1.223404in}}%
\pgfpathlineto{\pgfqpoint{4.770840in}{1.220259in}}%
\pgfpathlineto{\pgfqpoint{4.773320in}{1.222556in}}%
\pgfpathlineto{\pgfqpoint{4.775800in}{1.220079in}}%
\pgfpathlineto{\pgfqpoint{4.777040in}{1.222507in}}%
\pgfpathlineto{\pgfqpoint{4.782000in}{1.218190in}}%
\pgfpathlineto{\pgfqpoint{4.783240in}{1.220775in}}%
\pgfpathlineto{\pgfqpoint{4.790680in}{1.219900in}}%
\pgfpathlineto{\pgfqpoint{4.793160in}{1.222643in}}%
\pgfpathlineto{\pgfqpoint{4.795640in}{1.222934in}}%
\pgfpathlineto{\pgfqpoint{4.800600in}{1.218617in}}%
\pgfpathlineto{\pgfqpoint{4.805560in}{1.211869in}}%
\pgfpathlineto{\pgfqpoint{4.806800in}{1.212092in}}%
\pgfpathlineto{\pgfqpoint{4.810520in}{1.210572in}}%
\pgfpathlineto{\pgfqpoint{4.811760in}{1.210542in}}%
\pgfpathlineto{\pgfqpoint{4.817960in}{1.215719in}}%
\pgfpathlineto{\pgfqpoint{4.820440in}{1.216432in}}%
\pgfpathlineto{\pgfqpoint{4.821680in}{1.217238in}}%
\pgfpathlineto{\pgfqpoint{4.822920in}{1.215393in}}%
\pgfpathlineto{\pgfqpoint{4.825400in}{1.216771in}}%
\pgfpathlineto{\pgfqpoint{4.829120in}{1.213102in}}%
\pgfpathlineto{\pgfqpoint{4.832840in}{1.215871in}}%
\pgfpathlineto{\pgfqpoint{4.834080in}{1.214866in}}%
\pgfpathlineto{\pgfqpoint{4.835320in}{1.217098in}}%
\pgfpathlineto{\pgfqpoint{4.836560in}{1.216147in}}%
\pgfpathlineto{\pgfqpoint{4.842760in}{1.220425in}}%
\pgfpathlineto{\pgfqpoint{4.844000in}{1.217856in}}%
\pgfpathlineto{\pgfqpoint{4.845240in}{1.217937in}}%
\pgfpathlineto{\pgfqpoint{4.846480in}{1.215431in}}%
\pgfpathlineto{\pgfqpoint{4.848960in}{1.216494in}}%
\pgfpathlineto{\pgfqpoint{4.852680in}{1.217129in}}%
\pgfpathlineto{\pgfqpoint{4.855160in}{1.216214in}}%
\pgfpathlineto{\pgfqpoint{4.858880in}{1.221218in}}%
\pgfpathlineto{\pgfqpoint{4.860120in}{1.222210in}}%
\pgfpathlineto{\pgfqpoint{4.866320in}{1.215819in}}%
\pgfpathlineto{\pgfqpoint{4.867560in}{1.216227in}}%
\pgfpathlineto{\pgfqpoint{4.870040in}{1.218450in}}%
\pgfpathlineto{\pgfqpoint{4.872520in}{1.217824in}}%
\pgfpathlineto{\pgfqpoint{4.873760in}{1.215634in}}%
\pgfpathlineto{\pgfqpoint{4.875000in}{1.216040in}}%
\pgfpathlineto{\pgfqpoint{4.877480in}{1.214638in}}%
\pgfpathlineto{\pgfqpoint{4.879960in}{1.217197in}}%
\pgfpathlineto{\pgfqpoint{4.886160in}{1.212994in}}%
\pgfpathlineto{\pgfqpoint{4.887400in}{1.210420in}}%
\pgfpathlineto{\pgfqpoint{4.888640in}{1.210969in}}%
\pgfpathlineto{\pgfqpoint{4.892360in}{1.215799in}}%
\pgfpathlineto{\pgfqpoint{4.894840in}{1.213174in}}%
\pgfpathlineto{\pgfqpoint{4.897320in}{1.215825in}}%
\pgfpathlineto{\pgfqpoint{4.899800in}{1.212343in}}%
\pgfpathlineto{\pgfqpoint{4.901040in}{1.214696in}}%
\pgfpathlineto{\pgfqpoint{4.907240in}{1.213456in}}%
\pgfpathlineto{\pgfqpoint{4.913440in}{1.212377in}}%
\pgfpathlineto{\pgfqpoint{4.914680in}{1.211677in}}%
\pgfpathlineto{\pgfqpoint{4.918400in}{1.215509in}}%
\pgfpathlineto{\pgfqpoint{4.920880in}{1.215185in}}%
\pgfpathlineto{\pgfqpoint{4.922120in}{1.214714in}}%
\pgfpathlineto{\pgfqpoint{4.925840in}{1.208838in}}%
\pgfpathlineto{\pgfqpoint{4.932040in}{1.203346in}}%
\pgfpathlineto{\pgfqpoint{4.937000in}{1.204777in}}%
\pgfpathlineto{\pgfqpoint{4.940720in}{1.207964in}}%
\pgfpathlineto{\pgfqpoint{4.944440in}{1.207272in}}%
\pgfpathlineto{\pgfqpoint{4.945680in}{1.207856in}}%
\pgfpathlineto{\pgfqpoint{4.946920in}{1.206398in}}%
\pgfpathlineto{\pgfqpoint{4.949400in}{1.208354in}}%
\pgfpathlineto{\pgfqpoint{4.951880in}{1.205261in}}%
\pgfpathlineto{\pgfqpoint{4.953120in}{1.205055in}}%
\pgfpathlineto{\pgfqpoint{4.955600in}{1.208039in}}%
\pgfpathlineto{\pgfqpoint{4.958080in}{1.204782in}}%
\pgfpathlineto{\pgfqpoint{4.959320in}{1.206882in}}%
\pgfpathlineto{\pgfqpoint{4.961800in}{1.206560in}}%
\pgfpathlineto{\pgfqpoint{4.964280in}{1.209391in}}%
\pgfpathlineto{\pgfqpoint{4.966760in}{1.210193in}}%
\pgfpathlineto{\pgfqpoint{4.970480in}{1.204030in}}%
\pgfpathlineto{\pgfqpoint{4.975440in}{1.205012in}}%
\pgfpathlineto{\pgfqpoint{4.976680in}{1.205654in}}%
\pgfpathlineto{\pgfqpoint{4.979160in}{1.205108in}}%
\pgfpathlineto{\pgfqpoint{4.982880in}{1.209840in}}%
\pgfpathlineto{\pgfqpoint{4.984120in}{1.211005in}}%
\pgfpathlineto{\pgfqpoint{4.989080in}{1.203359in}}%
\pgfpathlineto{\pgfqpoint{4.990320in}{1.204502in}}%
\pgfpathlineto{\pgfqpoint{4.996520in}{1.207761in}}%
\pgfpathlineto{\pgfqpoint{4.997760in}{1.205964in}}%
\pgfpathlineto{\pgfqpoint{4.999000in}{1.206560in}}%
\pgfpathlineto{\pgfqpoint{5.001480in}{1.204841in}}%
\pgfpathlineto{\pgfqpoint{5.003960in}{1.208072in}}%
\pgfpathlineto{\pgfqpoint{5.010160in}{1.204279in}}%
\pgfpathlineto{\pgfqpoint{5.011400in}{1.201605in}}%
\pgfpathlineto{\pgfqpoint{5.013880in}{1.204151in}}%
\pgfpathlineto{\pgfqpoint{5.016360in}{1.206665in}}%
\pgfpathlineto{\pgfqpoint{5.018840in}{1.202860in}}%
\pgfpathlineto{\pgfqpoint{5.021320in}{1.206382in}}%
\pgfpathlineto{\pgfqpoint{5.023800in}{1.203256in}}%
\pgfpathlineto{\pgfqpoint{5.025040in}{1.205821in}}%
\pgfpathlineto{\pgfqpoint{5.027520in}{1.206268in}}%
\pgfpathlineto{\pgfqpoint{5.028760in}{1.206480in}}%
\pgfpathlineto{\pgfqpoint{5.030000in}{1.204043in}}%
\pgfpathlineto{\pgfqpoint{5.032480in}{1.206429in}}%
\pgfpathlineto{\pgfqpoint{5.034960in}{1.206069in}}%
\pgfpathlineto{\pgfqpoint{5.038680in}{1.204561in}}%
\pgfpathlineto{\pgfqpoint{5.043640in}{1.209125in}}%
\pgfpathlineto{\pgfqpoint{5.046120in}{1.208545in}}%
\pgfpathlineto{\pgfqpoint{5.049840in}{1.201936in}}%
\pgfpathlineto{\pgfqpoint{5.053560in}{1.196687in}}%
\pgfpathlineto{\pgfqpoint{5.054800in}{1.197691in}}%
\pgfpathlineto{\pgfqpoint{5.056040in}{1.195992in}}%
\pgfpathlineto{\pgfqpoint{5.057280in}{1.196510in}}%
\pgfpathlineto{\pgfqpoint{5.058520in}{1.195760in}}%
\pgfpathlineto{\pgfqpoint{5.062240in}{1.198721in}}%
\pgfpathlineto{\pgfqpoint{5.064720in}{1.200826in}}%
\pgfpathlineto{\pgfqpoint{5.068440in}{1.200709in}}%
\pgfpathlineto{\pgfqpoint{5.069680in}{1.201043in}}%
\pgfpathlineto{\pgfqpoint{5.070920in}{1.199622in}}%
\pgfpathlineto{\pgfqpoint{5.073400in}{1.200557in}}%
\pgfpathlineto{\pgfqpoint{5.077120in}{1.197658in}}%
\pgfpathlineto{\pgfqpoint{5.079600in}{1.200364in}}%
\pgfpathlineto{\pgfqpoint{5.082080in}{1.197208in}}%
\pgfpathlineto{\pgfqpoint{5.083320in}{1.198534in}}%
\pgfpathlineto{\pgfqpoint{5.085800in}{1.197887in}}%
\pgfpathlineto{\pgfqpoint{5.089520in}{1.201675in}}%
\pgfpathlineto{\pgfqpoint{5.090760in}{1.201540in}}%
\pgfpathlineto{\pgfqpoint{5.094480in}{1.196674in}}%
\pgfpathlineto{\pgfqpoint{5.098200in}{1.197057in}}%
\pgfpathlineto{\pgfqpoint{5.103160in}{1.197497in}}%
\pgfpathlineto{\pgfqpoint{5.106880in}{1.202844in}}%
\pgfpathlineto{\pgfqpoint{5.108120in}{1.204011in}}%
\pgfpathlineto{\pgfqpoint{5.110600in}{1.201694in}}%
\pgfpathlineto{\pgfqpoint{5.113080in}{1.196722in}}%
\pgfpathlineto{\pgfqpoint{5.116800in}{1.201139in}}%
\pgfpathlineto{\pgfqpoint{5.120520in}{1.202439in}}%
\pgfpathlineto{\pgfqpoint{5.121760in}{1.200913in}}%
\pgfpathlineto{\pgfqpoint{5.123000in}{1.201287in}}%
\pgfpathlineto{\pgfqpoint{5.125480in}{1.198382in}}%
\pgfpathlineto{\pgfqpoint{5.127960in}{1.200570in}}%
\pgfpathlineto{\pgfqpoint{5.134160in}{1.198513in}}%
\pgfpathlineto{\pgfqpoint{5.135400in}{1.195799in}}%
\pgfpathlineto{\pgfqpoint{5.139120in}{1.199435in}}%
\pgfpathlineto{\pgfqpoint{5.140360in}{1.200681in}}%
\pgfpathlineto{\pgfqpoint{5.142840in}{1.197052in}}%
\pgfpathlineto{\pgfqpoint{5.145320in}{1.200287in}}%
\pgfpathlineto{\pgfqpoint{5.147800in}{1.197323in}}%
\pgfpathlineto{\pgfqpoint{5.149040in}{1.199989in}}%
\pgfpathlineto{\pgfqpoint{5.151520in}{1.199035in}}%
\pgfpathlineto{\pgfqpoint{5.152760in}{1.199321in}}%
\pgfpathlineto{\pgfqpoint{5.154000in}{1.197451in}}%
\pgfpathlineto{\pgfqpoint{5.155240in}{1.200627in}}%
\pgfpathlineto{\pgfqpoint{5.158960in}{1.201394in}}%
\pgfpathlineto{\pgfqpoint{5.161440in}{1.200772in}}%
\pgfpathlineto{\pgfqpoint{5.162680in}{1.200343in}}%
\pgfpathlineto{\pgfqpoint{5.167640in}{1.203777in}}%
\pgfpathlineto{\pgfqpoint{5.170120in}{1.204274in}}%
\pgfpathlineto{\pgfqpoint{5.176320in}{1.191998in}}%
\pgfpathlineto{\pgfqpoint{5.177560in}{1.191410in}}%
\pgfpathlineto{\pgfqpoint{5.178800in}{1.192249in}}%
\pgfpathlineto{\pgfqpoint{5.180040in}{1.190520in}}%
\pgfpathlineto{\pgfqpoint{5.181280in}{1.191369in}}%
\pgfpathlineto{\pgfqpoint{5.183760in}{1.191322in}}%
\pgfpathlineto{\pgfqpoint{5.186240in}{1.192525in}}%
\pgfpathlineto{\pgfqpoint{5.187480in}{1.195248in}}%
\pgfpathlineto{\pgfqpoint{5.192440in}{1.193711in}}%
\pgfpathlineto{\pgfqpoint{5.193680in}{1.194596in}}%
\pgfpathlineto{\pgfqpoint{5.194920in}{1.193473in}}%
\pgfpathlineto{\pgfqpoint{5.197400in}{1.194577in}}%
\pgfpathlineto{\pgfqpoint{5.199880in}{1.192152in}}%
\pgfpathlineto{\pgfqpoint{5.202360in}{1.193580in}}%
\pgfpathlineto{\pgfqpoint{5.203600in}{1.195026in}}%
\pgfpathlineto{\pgfqpoint{5.208560in}{1.192206in}}%
\pgfpathlineto{\pgfqpoint{5.209800in}{1.192935in}}%
\pgfpathlineto{\pgfqpoint{5.213520in}{1.197068in}}%
\pgfpathlineto{\pgfqpoint{5.214760in}{1.197203in}}%
\pgfpathlineto{\pgfqpoint{5.217240in}{1.194476in}}%
\pgfpathlineto{\pgfqpoint{5.218480in}{1.192341in}}%
\pgfpathlineto{\pgfqpoint{5.222200in}{1.193381in}}%
\pgfpathlineto{\pgfqpoint{5.225920in}{1.192346in}}%
\pgfpathlineto{\pgfqpoint{5.227160in}{1.193295in}}%
\pgfpathlineto{\pgfqpoint{5.230880in}{1.199063in}}%
\pgfpathlineto{\pgfqpoint{5.232120in}{1.199957in}}%
\pgfpathlineto{\pgfqpoint{5.234600in}{1.197088in}}%
\pgfpathlineto{\pgfqpoint{5.237080in}{1.191026in}}%
\pgfpathlineto{\pgfqpoint{5.242040in}{1.197306in}}%
\pgfpathlineto{\pgfqpoint{5.243280in}{1.196513in}}%
\pgfpathlineto{\pgfqpoint{5.244520in}{1.197341in}}%
\pgfpathlineto{\pgfqpoint{5.245760in}{1.196318in}}%
\pgfpathlineto{\pgfqpoint{5.247000in}{1.196931in}}%
\pgfpathlineto{\pgfqpoint{5.249480in}{1.194013in}}%
\pgfpathlineto{\pgfqpoint{5.251960in}{1.197512in}}%
\pgfpathlineto{\pgfqpoint{5.253200in}{1.197320in}}%
\pgfpathlineto{\pgfqpoint{5.255680in}{1.196137in}}%
\pgfpathlineto{\pgfqpoint{5.258160in}{1.194983in}}%
\pgfpathlineto{\pgfqpoint{5.259400in}{1.191929in}}%
\pgfpathlineto{\pgfqpoint{5.264360in}{1.197014in}}%
\pgfpathlineto{\pgfqpoint{5.266840in}{1.192369in}}%
\pgfpathlineto{\pgfqpoint{5.269320in}{1.194819in}}%
\pgfpathlineto{\pgfqpoint{5.271800in}{1.192628in}}%
\pgfpathlineto{\pgfqpoint{5.273040in}{1.195447in}}%
\pgfpathlineto{\pgfqpoint{5.275520in}{1.194689in}}%
\pgfpathlineto{\pgfqpoint{5.276760in}{1.195087in}}%
\pgfpathlineto{\pgfqpoint{5.278000in}{1.192994in}}%
\pgfpathlineto{\pgfqpoint{5.279240in}{1.196797in}}%
\pgfpathlineto{\pgfqpoint{5.282960in}{1.197620in}}%
\pgfpathlineto{\pgfqpoint{5.284200in}{1.198668in}}%
\pgfpathlineto{\pgfqpoint{5.286680in}{1.197905in}}%
\pgfpathlineto{\pgfqpoint{5.294120in}{1.200568in}}%
\pgfpathlineto{\pgfqpoint{5.299080in}{1.192722in}}%
\pgfpathlineto{\pgfqpoint{5.300320in}{1.188146in}}%
\pgfpathlineto{\pgfqpoint{5.310240in}{1.190844in}}%
\pgfpathlineto{\pgfqpoint{5.312720in}{1.193173in}}%
\pgfpathlineto{\pgfqpoint{5.321400in}{1.194318in}}%
\pgfpathlineto{\pgfqpoint{5.323880in}{1.192305in}}%
\pgfpathlineto{\pgfqpoint{5.328840in}{1.193603in}}%
\pgfpathlineto{\pgfqpoint{5.330080in}{1.190790in}}%
\pgfpathlineto{\pgfqpoint{5.331320in}{1.192185in}}%
\pgfpathlineto{\pgfqpoint{5.332560in}{1.191790in}}%
\pgfpathlineto{\pgfqpoint{5.337520in}{1.195031in}}%
\pgfpathlineto{\pgfqpoint{5.340000in}{1.193627in}}%
\pgfpathlineto{\pgfqpoint{5.342480in}{1.186982in}}%
\pgfpathlineto{\pgfqpoint{5.351160in}{1.189190in}}%
\pgfpathlineto{\pgfqpoint{5.354880in}{1.194482in}}%
\pgfpathlineto{\pgfqpoint{5.357360in}{1.194586in}}%
\pgfpathlineto{\pgfqpoint{5.359840in}{1.189853in}}%
\pgfpathlineto{\pgfqpoint{5.361080in}{1.186465in}}%
\pgfpathlineto{\pgfqpoint{5.364800in}{1.191541in}}%
\pgfpathlineto{\pgfqpoint{5.371000in}{1.191968in}}%
\pgfpathlineto{\pgfqpoint{5.373480in}{1.188362in}}%
\pgfpathlineto{\pgfqpoint{5.375960in}{1.192439in}}%
\pgfpathlineto{\pgfqpoint{5.379680in}{1.191950in}}%
\pgfpathlineto{\pgfqpoint{5.382160in}{1.191790in}}%
\pgfpathlineto{\pgfqpoint{5.383400in}{1.189378in}}%
\pgfpathlineto{\pgfqpoint{5.388360in}{1.193698in}}%
\pgfpathlineto{\pgfqpoint{5.390840in}{1.189069in}}%
\pgfpathlineto{\pgfqpoint{5.393320in}{1.191493in}}%
\pgfpathlineto{\pgfqpoint{5.395800in}{1.188964in}}%
\pgfpathlineto{\pgfqpoint{5.397040in}{1.191489in}}%
\pgfpathlineto{\pgfqpoint{5.398280in}{1.189997in}}%
\pgfpathlineto{\pgfqpoint{5.400760in}{1.190935in}}%
\pgfpathlineto{\pgfqpoint{5.402000in}{1.189159in}}%
\pgfpathlineto{\pgfqpoint{5.403240in}{1.193385in}}%
\pgfpathlineto{\pgfqpoint{5.405720in}{1.193648in}}%
\pgfpathlineto{\pgfqpoint{5.409440in}{1.194398in}}%
\pgfpathlineto{\pgfqpoint{5.414400in}{1.195858in}}%
\pgfpathlineto{\pgfqpoint{5.419360in}{1.194807in}}%
\pgfpathlineto{\pgfqpoint{5.421840in}{1.191329in}}%
\pgfpathlineto{\pgfqpoint{5.425560in}{1.183846in}}%
\pgfpathlineto{\pgfqpoint{5.426800in}{1.184571in}}%
\pgfpathlineto{\pgfqpoint{5.428040in}{1.183239in}}%
\pgfpathlineto{\pgfqpoint{5.430520in}{1.184059in}}%
\pgfpathlineto{\pgfqpoint{5.441680in}{1.188515in}}%
\pgfpathlineto{\pgfqpoint{5.442920in}{1.187291in}}%
\pgfpathlineto{\pgfqpoint{5.445400in}{1.188285in}}%
\pgfpathlineto{\pgfqpoint{5.447880in}{1.185394in}}%
\pgfpathlineto{\pgfqpoint{5.451600in}{1.187007in}}%
\pgfpathlineto{\pgfqpoint{5.452840in}{1.185535in}}%
\pgfpathlineto{\pgfqpoint{5.454080in}{1.182089in}}%
\pgfpathlineto{\pgfqpoint{5.456560in}{1.183707in}}%
\pgfpathlineto{\pgfqpoint{5.461520in}{1.186827in}}%
\pgfpathlineto{\pgfqpoint{5.464000in}{1.185449in}}%
\pgfpathlineto{\pgfqpoint{5.467720in}{1.182021in}}%
\pgfpathlineto{\pgfqpoint{5.468960in}{1.182798in}}%
\pgfpathlineto{\pgfqpoint{5.471440in}{1.182112in}}%
\pgfpathlineto{\pgfqpoint{5.475160in}{1.182848in}}%
\pgfpathlineto{\pgfqpoint{5.480120in}{1.189923in}}%
\pgfpathlineto{\pgfqpoint{5.482600in}{1.187834in}}%
\pgfpathlineto{\pgfqpoint{5.485080in}{1.181700in}}%
\pgfpathlineto{\pgfqpoint{5.487560in}{1.185303in}}%
\pgfpathlineto{\pgfqpoint{5.488800in}{1.187021in}}%
\pgfpathlineto{\pgfqpoint{5.491280in}{1.186590in}}%
\pgfpathlineto{\pgfqpoint{5.492520in}{1.188363in}}%
\pgfpathlineto{\pgfqpoint{5.493760in}{1.186984in}}%
\pgfpathlineto{\pgfqpoint{5.495000in}{1.187640in}}%
\pgfpathlineto{\pgfqpoint{5.497480in}{1.184773in}}%
\pgfpathlineto{\pgfqpoint{5.499960in}{1.189434in}}%
\pgfpathlineto{\pgfqpoint{5.501200in}{1.189846in}}%
\pgfpathlineto{\pgfqpoint{5.502440in}{1.188766in}}%
\pgfpathlineto{\pgfqpoint{5.503680in}{1.189309in}}%
\pgfpathlineto{\pgfqpoint{5.504920in}{1.188159in}}%
\pgfpathlineto{\pgfqpoint{5.506160in}{1.188916in}}%
\pgfpathlineto{\pgfqpoint{5.507400in}{1.186890in}}%
\pgfpathlineto{\pgfqpoint{5.513600in}{1.188860in}}%
\pgfpathlineto{\pgfqpoint{5.514840in}{1.185762in}}%
\pgfpathlineto{\pgfqpoint{5.517320in}{1.188517in}}%
\pgfpathlineto{\pgfqpoint{5.519800in}{1.185348in}}%
\pgfpathlineto{\pgfqpoint{5.521040in}{1.187454in}}%
\pgfpathlineto{\pgfqpoint{5.523520in}{1.186918in}}%
\pgfpathlineto{\pgfqpoint{5.524760in}{1.187982in}}%
\pgfpathlineto{\pgfqpoint{5.526000in}{1.185838in}}%
\pgfpathlineto{\pgfqpoint{5.527240in}{1.189707in}}%
\pgfpathlineto{\pgfqpoint{5.530960in}{1.189038in}}%
\pgfpathlineto{\pgfqpoint{5.533440in}{1.189688in}}%
\pgfpathlineto{\pgfqpoint{5.538400in}{1.192420in}}%
\pgfpathlineto{\pgfqpoint{5.540880in}{1.192583in}}%
\pgfpathlineto{\pgfqpoint{5.542120in}{1.193494in}}%
\pgfpathlineto{\pgfqpoint{5.545840in}{1.187808in}}%
\pgfpathlineto{\pgfqpoint{5.548320in}{1.180153in}}%
\pgfpathlineto{\pgfqpoint{5.550800in}{1.180607in}}%
\pgfpathlineto{\pgfqpoint{5.552040in}{1.178734in}}%
\pgfpathlineto{\pgfqpoint{5.554520in}{1.179850in}}%
\pgfpathlineto{\pgfqpoint{5.558240in}{1.180233in}}%
\pgfpathlineto{\pgfqpoint{5.559480in}{1.182061in}}%
\pgfpathlineto{\pgfqpoint{5.561960in}{1.181747in}}%
\pgfpathlineto{\pgfqpoint{5.564440in}{1.183492in}}%
\pgfpathlineto{\pgfqpoint{5.565680in}{1.184404in}}%
\pgfpathlineto{\pgfqpoint{5.566920in}{1.182516in}}%
\pgfpathlineto{\pgfqpoint{5.569400in}{1.184467in}}%
\pgfpathlineto{\pgfqpoint{5.570640in}{1.182615in}}%
\pgfpathlineto{\pgfqpoint{5.575600in}{1.185945in}}%
\pgfpathlineto{\pgfqpoint{5.578080in}{1.181860in}}%
\pgfpathlineto{\pgfqpoint{5.579320in}{1.183367in}}%
\pgfpathlineto{\pgfqpoint{5.580560in}{1.182774in}}%
\pgfpathlineto{\pgfqpoint{5.585520in}{1.185255in}}%
\pgfpathlineto{\pgfqpoint{5.588000in}{1.183621in}}%
\pgfpathlineto{\pgfqpoint{5.589240in}{1.186923in}}%
\pgfpathlineto{\pgfqpoint{5.590480in}{1.184227in}}%
\pgfpathlineto{\pgfqpoint{5.594200in}{1.185091in}}%
\pgfpathlineto{\pgfqpoint{5.597920in}{1.185819in}}%
\pgfpathlineto{\pgfqpoint{5.600400in}{1.187557in}}%
\pgfpathlineto{\pgfqpoint{5.604120in}{1.190419in}}%
\pgfpathlineto{\pgfqpoint{5.606600in}{1.188467in}}%
\pgfpathlineto{\pgfqpoint{5.609080in}{1.181775in}}%
\pgfpathlineto{\pgfqpoint{5.610320in}{1.184843in}}%
\pgfpathlineto{\pgfqpoint{5.611560in}{1.184627in}}%
\pgfpathlineto{\pgfqpoint{5.612800in}{1.186262in}}%
\pgfpathlineto{\pgfqpoint{5.615280in}{1.185770in}}%
\pgfpathlineto{\pgfqpoint{5.616520in}{1.186671in}}%
\pgfpathlineto{\pgfqpoint{5.617760in}{1.184976in}}%
\pgfpathlineto{\pgfqpoint{5.619000in}{1.186186in}}%
\pgfpathlineto{\pgfqpoint{5.621480in}{1.183862in}}%
\pgfpathlineto{\pgfqpoint{5.623960in}{1.188184in}}%
\pgfpathlineto{\pgfqpoint{5.625200in}{1.188621in}}%
\pgfpathlineto{\pgfqpoint{5.626440in}{1.186929in}}%
\pgfpathlineto{\pgfqpoint{5.627680in}{1.188594in}}%
\pgfpathlineto{\pgfqpoint{5.628920in}{1.188097in}}%
\pgfpathlineto{\pgfqpoint{5.630160in}{1.188903in}}%
\pgfpathlineto{\pgfqpoint{5.631400in}{1.187451in}}%
\pgfpathlineto{\pgfqpoint{5.636360in}{1.188843in}}%
\pgfpathlineto{\pgfqpoint{5.640080in}{1.185305in}}%
\pgfpathlineto{\pgfqpoint{5.641320in}{1.187063in}}%
\pgfpathlineto{\pgfqpoint{5.642560in}{1.183746in}}%
\pgfpathlineto{\pgfqpoint{5.643800in}{1.184417in}}%
\pgfpathlineto{\pgfqpoint{5.645040in}{1.186999in}}%
\pgfpathlineto{\pgfqpoint{5.648760in}{1.186863in}}%
\pgfpathlineto{\pgfqpoint{5.650000in}{1.184732in}}%
\pgfpathlineto{\pgfqpoint{5.651240in}{1.189195in}}%
\pgfpathlineto{\pgfqpoint{5.652480in}{1.188693in}}%
\pgfpathlineto{\pgfqpoint{5.656200in}{1.190784in}}%
\pgfpathlineto{\pgfqpoint{5.658680in}{1.191328in}}%
\pgfpathlineto{\pgfqpoint{5.661160in}{1.192690in}}%
\pgfpathlineto{\pgfqpoint{5.667360in}{1.192471in}}%
\pgfpathlineto{\pgfqpoint{5.671080in}{1.183511in}}%
\pgfpathlineto{\pgfqpoint{5.672320in}{1.180400in}}%
\pgfpathlineto{\pgfqpoint{5.674800in}{1.180356in}}%
\pgfpathlineto{\pgfqpoint{5.676040in}{1.178395in}}%
\pgfpathlineto{\pgfqpoint{5.679760in}{1.179811in}}%
\pgfpathlineto{\pgfqpoint{5.689680in}{1.182756in}}%
\pgfpathlineto{\pgfqpoint{5.692160in}{1.180334in}}%
\pgfpathlineto{\pgfqpoint{5.693400in}{1.182422in}}%
\pgfpathlineto{\pgfqpoint{5.694640in}{1.180397in}}%
\pgfpathlineto{\pgfqpoint{5.697120in}{1.182296in}}%
\pgfpathlineto{\pgfqpoint{5.698360in}{1.183169in}}%
\pgfpathlineto{\pgfqpoint{5.700840in}{1.180486in}}%
\pgfpathlineto{\pgfqpoint{5.702080in}{1.177691in}}%
\pgfpathlineto{\pgfqpoint{5.703320in}{1.178870in}}%
\pgfpathlineto{\pgfqpoint{5.704560in}{1.177407in}}%
\pgfpathlineto{\pgfqpoint{5.708280in}{1.179220in}}%
\pgfpathlineto{\pgfqpoint{5.713240in}{1.181228in}}%
\pgfpathlineto{\pgfqpoint{5.715720in}{1.178223in}}%
\pgfpathlineto{\pgfqpoint{5.718200in}{1.179621in}}%
\pgfpathlineto{\pgfqpoint{5.725640in}{1.182277in}}%
\pgfpathlineto{\pgfqpoint{5.728120in}{1.183252in}}%
\pgfpathlineto{\pgfqpoint{5.729360in}{1.183676in}}%
\pgfpathlineto{\pgfqpoint{5.730600in}{1.182729in}}%
\pgfpathlineto{\pgfqpoint{5.733080in}{1.175598in}}%
\pgfpathlineto{\pgfqpoint{5.735560in}{1.179741in}}%
\pgfpathlineto{\pgfqpoint{5.736800in}{1.181622in}}%
\pgfpathlineto{\pgfqpoint{5.739280in}{1.180967in}}%
\pgfpathlineto{\pgfqpoint{5.740520in}{1.181916in}}%
\pgfpathlineto{\pgfqpoint{5.741760in}{1.180164in}}%
\pgfpathlineto{\pgfqpoint{5.743000in}{1.181311in}}%
\pgfpathlineto{\pgfqpoint{5.745480in}{1.179470in}}%
\pgfpathlineto{\pgfqpoint{5.747960in}{1.182928in}}%
\pgfpathlineto{\pgfqpoint{5.749200in}{1.183301in}}%
\pgfpathlineto{\pgfqpoint{5.750440in}{1.181484in}}%
\pgfpathlineto{\pgfqpoint{5.751680in}{1.182785in}}%
\pgfpathlineto{\pgfqpoint{5.752920in}{1.182151in}}%
\pgfpathlineto{\pgfqpoint{5.754160in}{1.183606in}}%
\pgfpathlineto{\pgfqpoint{5.756640in}{1.183687in}}%
\pgfpathlineto{\pgfqpoint{5.760360in}{1.184479in}}%
\pgfpathlineto{\pgfqpoint{5.764080in}{1.180113in}}%
\pgfpathlineto{\pgfqpoint{5.765320in}{1.181411in}}%
\pgfpathlineto{\pgfqpoint{5.766560in}{1.178321in}}%
\pgfpathlineto{\pgfqpoint{5.769040in}{1.183024in}}%
\pgfpathlineto{\pgfqpoint{5.770280in}{1.182120in}}%
\pgfpathlineto{\pgfqpoint{5.771520in}{1.181171in}}%
\pgfpathlineto{\pgfqpoint{5.772760in}{1.182181in}}%
\pgfpathlineto{\pgfqpoint{5.774000in}{1.180077in}}%
\pgfpathlineto{\pgfqpoint{5.776480in}{1.184774in}}%
\pgfpathlineto{\pgfqpoint{5.780200in}{1.188098in}}%
\pgfpathlineto{\pgfqpoint{5.782680in}{1.190436in}}%
\pgfpathlineto{\pgfqpoint{5.785160in}{1.192429in}}%
\pgfpathlineto{\pgfqpoint{5.786400in}{1.193377in}}%
\pgfpathlineto{\pgfqpoint{5.787640in}{1.192831in}}%
\pgfpathlineto{\pgfqpoint{5.790120in}{1.195036in}}%
\pgfpathlineto{\pgfqpoint{5.792600in}{1.191489in}}%
\pgfpathlineto{\pgfqpoint{5.795080in}{1.184757in}}%
\pgfpathlineto{\pgfqpoint{5.796320in}{1.182104in}}%
\pgfpathlineto{\pgfqpoint{5.812440in}{1.182123in}}%
\pgfpathlineto{\pgfqpoint{5.813680in}{1.183551in}}%
\pgfpathlineto{\pgfqpoint{5.816160in}{1.181488in}}%
\pgfpathlineto{\pgfqpoint{5.817400in}{1.183700in}}%
\pgfpathlineto{\pgfqpoint{5.818640in}{1.181223in}}%
\pgfpathlineto{\pgfqpoint{5.819880in}{1.181321in}}%
\pgfpathlineto{\pgfqpoint{5.822360in}{1.184019in}}%
\pgfpathlineto{\pgfqpoint{5.823600in}{1.184100in}}%
\pgfpathlineto{\pgfqpoint{5.826080in}{1.179329in}}%
\pgfpathlineto{\pgfqpoint{5.827320in}{1.180847in}}%
\pgfpathlineto{\pgfqpoint{5.828560in}{1.179603in}}%
\pgfpathlineto{\pgfqpoint{5.831040in}{1.180814in}}%
\pgfpathlineto{\pgfqpoint{5.836000in}{1.179289in}}%
\pgfpathlineto{\pgfqpoint{5.837240in}{1.182214in}}%
\pgfpathlineto{\pgfqpoint{5.838480in}{1.179919in}}%
\pgfpathlineto{\pgfqpoint{5.839720in}{1.180305in}}%
\pgfpathlineto{\pgfqpoint{5.842200in}{1.182636in}}%
\pgfpathlineto{\pgfqpoint{5.844680in}{1.182910in}}%
\pgfpathlineto{\pgfqpoint{5.848400in}{1.183258in}}%
\pgfpathlineto{\pgfqpoint{5.853360in}{1.185435in}}%
\pgfpathlineto{\pgfqpoint{5.855840in}{1.181012in}}%
\pgfpathlineto{\pgfqpoint{5.857080in}{1.176529in}}%
\pgfpathlineto{\pgfqpoint{5.859560in}{1.179421in}}%
\pgfpathlineto{\pgfqpoint{5.860800in}{1.181622in}}%
\pgfpathlineto{\pgfqpoint{5.863280in}{1.181407in}}%
\pgfpathlineto{\pgfqpoint{5.864520in}{1.183077in}}%
\pgfpathlineto{\pgfqpoint{5.865760in}{1.181218in}}%
\pgfpathlineto{\pgfqpoint{5.867000in}{1.181912in}}%
\pgfpathlineto{\pgfqpoint{5.869480in}{1.179857in}}%
\pgfpathlineto{\pgfqpoint{5.873200in}{1.183575in}}%
\pgfpathlineto{\pgfqpoint{5.874440in}{1.182500in}}%
\pgfpathlineto{\pgfqpoint{5.875680in}{1.183259in}}%
\pgfpathlineto{\pgfqpoint{5.876920in}{1.181578in}}%
\pgfpathlineto{\pgfqpoint{5.879400in}{1.182484in}}%
\pgfpathlineto{\pgfqpoint{5.883120in}{1.180952in}}%
\pgfpathlineto{\pgfqpoint{5.884360in}{1.181525in}}%
\pgfpathlineto{\pgfqpoint{5.886840in}{1.176293in}}%
\pgfpathlineto{\pgfqpoint{5.889320in}{1.178738in}}%
\pgfpathlineto{\pgfqpoint{5.890560in}{1.175930in}}%
\pgfpathlineto{\pgfqpoint{5.893040in}{1.180765in}}%
\pgfpathlineto{\pgfqpoint{5.895520in}{1.179301in}}%
\pgfpathlineto{\pgfqpoint{5.896760in}{1.180906in}}%
\pgfpathlineto{\pgfqpoint{5.898000in}{1.178590in}}%
\pgfpathlineto{\pgfqpoint{5.900480in}{1.184705in}}%
\pgfpathlineto{\pgfqpoint{5.904200in}{1.188532in}}%
\pgfpathlineto{\pgfqpoint{5.905440in}{1.188335in}}%
\pgfpathlineto{\pgfqpoint{5.910400in}{1.193983in}}%
\pgfpathlineto{\pgfqpoint{5.912880in}{1.193879in}}%
\pgfpathlineto{\pgfqpoint{5.914120in}{1.195443in}}%
\pgfpathlineto{\pgfqpoint{5.917840in}{1.187916in}}%
\pgfpathlineto{\pgfqpoint{5.920320in}{1.181830in}}%
\pgfpathlineto{\pgfqpoint{5.922800in}{1.181576in}}%
\pgfpathlineto{\pgfqpoint{5.931480in}{1.181974in}}%
\pgfpathlineto{\pgfqpoint{5.933960in}{1.180344in}}%
\pgfpathlineto{\pgfqpoint{5.938920in}{1.182046in}}%
\pgfpathlineto{\pgfqpoint{5.940160in}{1.182048in}}%
\pgfpathlineto{\pgfqpoint{5.941400in}{1.184402in}}%
\pgfpathlineto{\pgfqpoint{5.943880in}{1.181625in}}%
\pgfpathlineto{\pgfqpoint{5.947600in}{1.184045in}}%
\pgfpathlineto{\pgfqpoint{5.950080in}{1.179559in}}%
\pgfpathlineto{\pgfqpoint{5.951320in}{1.181640in}}%
\pgfpathlineto{\pgfqpoint{5.952560in}{1.179970in}}%
\pgfpathlineto{\pgfqpoint{5.955040in}{1.180631in}}%
\pgfpathlineto{\pgfqpoint{5.960000in}{1.178893in}}%
\pgfpathlineto{\pgfqpoint{5.961240in}{1.184692in}}%
\pgfpathlineto{\pgfqpoint{5.962480in}{1.182998in}}%
\pgfpathlineto{\pgfqpoint{5.963720in}{1.183611in}}%
\pgfpathlineto{\pgfqpoint{5.967440in}{1.188090in}}%
\pgfpathlineto{\pgfqpoint{5.972400in}{1.186747in}}%
\pgfpathlineto{\pgfqpoint{5.977360in}{1.188920in}}%
\pgfpathlineto{\pgfqpoint{5.978600in}{1.187464in}}%
\pgfpathlineto{\pgfqpoint{5.981080in}{1.178742in}}%
\pgfpathlineto{\pgfqpoint{5.984800in}{1.183765in}}%
\pgfpathlineto{\pgfqpoint{5.987280in}{1.182997in}}%
\pgfpathlineto{\pgfqpoint{5.988520in}{1.185113in}}%
\pgfpathlineto{\pgfqpoint{5.989760in}{1.183725in}}%
\pgfpathlineto{\pgfqpoint{5.991000in}{1.185313in}}%
\pgfpathlineto{\pgfqpoint{5.993480in}{1.183495in}}%
\pgfpathlineto{\pgfqpoint{5.997200in}{1.187447in}}%
\pgfpathlineto{\pgfqpoint{5.998440in}{1.186046in}}%
\pgfpathlineto{\pgfqpoint{5.999680in}{1.186677in}}%
\pgfpathlineto{\pgfqpoint{6.000920in}{1.184163in}}%
\pgfpathlineto{\pgfqpoint{6.003400in}{1.184970in}}%
\pgfpathlineto{\pgfqpoint{6.005880in}{1.183388in}}%
\pgfpathlineto{\pgfqpoint{6.008360in}{1.184167in}}%
\pgfpathlineto{\pgfqpoint{6.010840in}{1.179173in}}%
\pgfpathlineto{\pgfqpoint{6.012080in}{1.179607in}}%
\pgfpathlineto{\pgfqpoint{6.013320in}{1.182249in}}%
\pgfpathlineto{\pgfqpoint{6.014560in}{1.179699in}}%
\pgfpathlineto{\pgfqpoint{6.019520in}{1.184313in}}%
\pgfpathlineto{\pgfqpoint{6.020760in}{1.185830in}}%
\pgfpathlineto{\pgfqpoint{6.022000in}{1.182907in}}%
\pgfpathlineto{\pgfqpoint{6.024480in}{1.188014in}}%
\pgfpathlineto{\pgfqpoint{6.030680in}{1.194253in}}%
\pgfpathlineto{\pgfqpoint{6.031920in}{1.196989in}}%
\pgfpathlineto{\pgfqpoint{6.034400in}{1.195752in}}%
\pgfpathlineto{\pgfqpoint{6.036880in}{1.195272in}}%
\pgfpathlineto{\pgfqpoint{6.038120in}{1.197004in}}%
\pgfpathlineto{\pgfqpoint{6.043080in}{1.185948in}}%
\pgfpathlineto{\pgfqpoint{6.044320in}{1.182630in}}%
\pgfpathlineto{\pgfqpoint{6.055480in}{1.184025in}}%
\pgfpathlineto{\pgfqpoint{6.057960in}{1.181257in}}%
\pgfpathlineto{\pgfqpoint{6.060440in}{1.182910in}}%
\pgfpathlineto{\pgfqpoint{6.061680in}{1.185237in}}%
\pgfpathlineto{\pgfqpoint{6.064160in}{1.184551in}}%
\pgfpathlineto{\pgfqpoint{6.065400in}{1.186718in}}%
\pgfpathlineto{\pgfqpoint{6.066640in}{1.184178in}}%
\pgfpathlineto{\pgfqpoint{6.071600in}{1.184871in}}%
\pgfpathlineto{\pgfqpoint{6.074080in}{1.180175in}}%
\pgfpathlineto{\pgfqpoint{6.075320in}{1.183159in}}%
\pgfpathlineto{\pgfqpoint{6.076560in}{1.182072in}}%
\pgfpathlineto{\pgfqpoint{6.079040in}{1.183043in}}%
\pgfpathlineto{\pgfqpoint{6.082760in}{1.181087in}}%
\pgfpathlineto{\pgfqpoint{6.084000in}{1.180930in}}%
\pgfpathlineto{\pgfqpoint{6.085240in}{1.189077in}}%
\pgfpathlineto{\pgfqpoint{6.086480in}{1.187495in}}%
\pgfpathlineto{\pgfqpoint{6.092680in}{1.194288in}}%
\pgfpathlineto{\pgfqpoint{6.097640in}{1.195091in}}%
\pgfpathlineto{\pgfqpoint{6.100120in}{1.197430in}}%
\pgfpathlineto{\pgfqpoint{6.102600in}{1.194753in}}%
\pgfpathlineto{\pgfqpoint{6.105080in}{1.186730in}}%
\pgfpathlineto{\pgfqpoint{6.106320in}{1.188492in}}%
\pgfpathlineto{\pgfqpoint{6.107560in}{1.188119in}}%
\pgfpathlineto{\pgfqpoint{6.108800in}{1.189776in}}%
\pgfpathlineto{\pgfqpoint{6.111280in}{1.188380in}}%
\pgfpathlineto{\pgfqpoint{6.112520in}{1.190662in}}%
\pgfpathlineto{\pgfqpoint{6.113760in}{1.188861in}}%
\pgfpathlineto{\pgfqpoint{6.115000in}{1.190923in}}%
\pgfpathlineto{\pgfqpoint{6.117480in}{1.188344in}}%
\pgfpathlineto{\pgfqpoint{6.121200in}{1.194625in}}%
\pgfpathlineto{\pgfqpoint{6.122440in}{1.193798in}}%
\pgfpathlineto{\pgfqpoint{6.123680in}{1.195072in}}%
\pgfpathlineto{\pgfqpoint{6.126160in}{1.192112in}}%
\pgfpathlineto{\pgfqpoint{6.132360in}{1.191416in}}%
\pgfpathlineto{\pgfqpoint{6.134840in}{1.186168in}}%
\pgfpathlineto{\pgfqpoint{6.136080in}{1.187304in}}%
\pgfpathlineto{\pgfqpoint{6.137320in}{1.190488in}}%
\pgfpathlineto{\pgfqpoint{6.138560in}{1.186459in}}%
\pgfpathlineto{\pgfqpoint{6.142280in}{1.190620in}}%
\pgfpathlineto{\pgfqpoint{6.143520in}{1.190043in}}%
\pgfpathlineto{\pgfqpoint{6.144760in}{1.191375in}}%
\pgfpathlineto{\pgfqpoint{6.146000in}{1.188968in}}%
\pgfpathlineto{\pgfqpoint{6.148480in}{1.193866in}}%
\pgfpathlineto{\pgfqpoint{6.155920in}{1.203695in}}%
\pgfpathlineto{\pgfqpoint{6.158400in}{1.201724in}}%
\pgfpathlineto{\pgfqpoint{6.159640in}{1.200565in}}%
\pgfpathlineto{\pgfqpoint{6.160880in}{1.201240in}}%
\pgfpathlineto{\pgfqpoint{6.162120in}{1.203223in}}%
\pgfpathlineto{\pgfqpoint{6.170800in}{1.188309in}}%
\pgfpathlineto{\pgfqpoint{6.175760in}{1.186944in}}%
\pgfpathlineto{\pgfqpoint{6.177000in}{1.185047in}}%
\pgfpathlineto{\pgfqpoint{6.179480in}{1.185233in}}%
\pgfpathlineto{\pgfqpoint{6.181960in}{1.183000in}}%
\pgfpathlineto{\pgfqpoint{6.184440in}{1.185689in}}%
\pgfpathlineto{\pgfqpoint{6.185680in}{1.189301in}}%
\pgfpathlineto{\pgfqpoint{6.186920in}{1.188354in}}%
\pgfpathlineto{\pgfqpoint{6.188160in}{1.188895in}}%
\pgfpathlineto{\pgfqpoint{6.189400in}{1.191162in}}%
\pgfpathlineto{\pgfqpoint{6.190640in}{1.188442in}}%
\pgfpathlineto{\pgfqpoint{6.191880in}{1.189029in}}%
\pgfpathlineto{\pgfqpoint{6.193120in}{1.191072in}}%
\pgfpathlineto{\pgfqpoint{6.195600in}{1.190022in}}%
\pgfpathlineto{\pgfqpoint{6.198080in}{1.184335in}}%
\pgfpathlineto{\pgfqpoint{6.199320in}{1.187180in}}%
\pgfpathlineto{\pgfqpoint{6.201800in}{1.185039in}}%
\pgfpathlineto{\pgfqpoint{6.206760in}{1.184569in}}%
\pgfpathlineto{\pgfqpoint{6.208000in}{1.183030in}}%
\pgfpathlineto{\pgfqpoint{6.209240in}{1.188228in}}%
\pgfpathlineto{\pgfqpoint{6.210480in}{1.186947in}}%
\pgfpathlineto{\pgfqpoint{6.215440in}{1.193691in}}%
\pgfpathlineto{\pgfqpoint{6.217920in}{1.194165in}}%
\pgfpathlineto{\pgfqpoint{6.219160in}{1.192459in}}%
\pgfpathlineto{\pgfqpoint{6.221640in}{1.193857in}}%
\pgfpathlineto{\pgfqpoint{6.222880in}{1.195503in}}%
\pgfpathlineto{\pgfqpoint{6.225360in}{1.194558in}}%
\pgfpathlineto{\pgfqpoint{6.227840in}{1.187902in}}%
\pgfpathlineto{\pgfqpoint{6.229080in}{1.182800in}}%
\pgfpathlineto{\pgfqpoint{6.232800in}{1.186279in}}%
\pgfpathlineto{\pgfqpoint{6.235280in}{1.182992in}}%
\pgfpathlineto{\pgfqpoint{6.236520in}{1.185592in}}%
\pgfpathlineto{\pgfqpoint{6.237760in}{1.185100in}}%
\pgfpathlineto{\pgfqpoint{6.239000in}{1.188910in}}%
\pgfpathlineto{\pgfqpoint{6.241480in}{1.189568in}}%
\pgfpathlineto{\pgfqpoint{6.245200in}{1.197068in}}%
\pgfpathlineto{\pgfqpoint{6.246440in}{1.195640in}}%
\pgfpathlineto{\pgfqpoint{6.247680in}{1.196491in}}%
\pgfpathlineto{\pgfqpoint{6.248920in}{1.193023in}}%
\pgfpathlineto{\pgfqpoint{6.251400in}{1.196330in}}%
\pgfpathlineto{\pgfqpoint{6.252640in}{1.195002in}}%
\pgfpathlineto{\pgfqpoint{6.256360in}{1.199952in}}%
\pgfpathlineto{\pgfqpoint{6.258840in}{1.194294in}}%
\pgfpathlineto{\pgfqpoint{6.260080in}{1.194554in}}%
\pgfpathlineto{\pgfqpoint{6.261320in}{1.196223in}}%
\pgfpathlineto{\pgfqpoint{6.262560in}{1.191205in}}%
\pgfpathlineto{\pgfqpoint{6.266280in}{1.197084in}}%
\pgfpathlineto{\pgfqpoint{6.267520in}{1.196618in}}%
\pgfpathlineto{\pgfqpoint{6.268760in}{1.197725in}}%
\pgfpathlineto{\pgfqpoint{6.270000in}{1.194553in}}%
\pgfpathlineto{\pgfqpoint{6.272480in}{1.198127in}}%
\pgfpathlineto{\pgfqpoint{6.276200in}{1.201959in}}%
\pgfpathlineto{\pgfqpoint{6.277440in}{1.201075in}}%
\pgfpathlineto{\pgfqpoint{6.279920in}{1.205176in}}%
\pgfpathlineto{\pgfqpoint{6.283640in}{1.200857in}}%
\pgfpathlineto{\pgfqpoint{6.284880in}{1.200833in}}%
\pgfpathlineto{\pgfqpoint{6.286120in}{1.202438in}}%
\pgfpathlineto{\pgfqpoint{6.287360in}{1.200023in}}%
\pgfpathlineto{\pgfqpoint{6.289840in}{1.193240in}}%
\pgfpathlineto{\pgfqpoint{6.292320in}{1.188059in}}%
\pgfpathlineto{\pgfqpoint{6.296040in}{1.188599in}}%
\pgfpathlineto{\pgfqpoint{6.298520in}{1.187978in}}%
\pgfpathlineto{\pgfqpoint{6.302240in}{1.186517in}}%
\pgfpathlineto{\pgfqpoint{6.303480in}{1.186644in}}%
\pgfpathlineto{\pgfqpoint{6.305960in}{1.183934in}}%
\pgfpathlineto{\pgfqpoint{6.308440in}{1.186199in}}%
\pgfpathlineto{\pgfqpoint{6.309680in}{1.190429in}}%
\pgfpathlineto{\pgfqpoint{6.312160in}{1.191026in}}%
\pgfpathlineto{\pgfqpoint{6.313400in}{1.193425in}}%
\pgfpathlineto{\pgfqpoint{6.314640in}{1.191307in}}%
\pgfpathlineto{\pgfqpoint{6.319600in}{1.193636in}}%
\pgfpathlineto{\pgfqpoint{6.322080in}{1.190135in}}%
\pgfpathlineto{\pgfqpoint{6.323320in}{1.192317in}}%
\pgfpathlineto{\pgfqpoint{6.325800in}{1.189884in}}%
\pgfpathlineto{\pgfqpoint{6.329520in}{1.189206in}}%
\pgfpathlineto{\pgfqpoint{6.333240in}{1.186533in}}%
\pgfpathlineto{\pgfqpoint{6.334480in}{1.186092in}}%
\pgfpathlineto{\pgfqpoint{6.336960in}{1.190699in}}%
\pgfpathlineto{\pgfqpoint{6.341920in}{1.193731in}}%
\pgfpathlineto{\pgfqpoint{6.343160in}{1.191348in}}%
\pgfpathlineto{\pgfqpoint{6.348120in}{1.196107in}}%
\pgfpathlineto{\pgfqpoint{6.349360in}{1.194839in}}%
\pgfpathlineto{\pgfqpoint{6.353080in}{1.182236in}}%
\pgfpathlineto{\pgfqpoint{6.354320in}{1.184534in}}%
\pgfpathlineto{\pgfqpoint{6.355560in}{1.183284in}}%
\pgfpathlineto{\pgfqpoint{6.356800in}{1.184611in}}%
\pgfpathlineto{\pgfqpoint{6.359280in}{1.180613in}}%
\pgfpathlineto{\pgfqpoint{6.364240in}{1.189154in}}%
\pgfpathlineto{\pgfqpoint{6.366720in}{1.190982in}}%
\pgfpathlineto{\pgfqpoint{6.367960in}{1.192685in}}%
\pgfpathlineto{\pgfqpoint{6.369200in}{1.196438in}}%
\pgfpathlineto{\pgfqpoint{6.370440in}{1.195495in}}%
\pgfpathlineto{\pgfqpoint{6.371680in}{1.198376in}}%
\pgfpathlineto{\pgfqpoint{6.372920in}{1.194976in}}%
\pgfpathlineto{\pgfqpoint{6.380360in}{1.201745in}}%
\pgfpathlineto{\pgfqpoint{6.382840in}{1.197303in}}%
\pgfpathlineto{\pgfqpoint{6.385320in}{1.198591in}}%
\pgfpathlineto{\pgfqpoint{6.386560in}{1.194166in}}%
\pgfpathlineto{\pgfqpoint{6.387800in}{1.194898in}}%
\pgfpathlineto{\pgfqpoint{6.389040in}{1.198543in}}%
\pgfpathlineto{\pgfqpoint{6.390280in}{1.198457in}}%
\pgfpathlineto{\pgfqpoint{6.391520in}{1.196721in}}%
\pgfpathlineto{\pgfqpoint{6.392760in}{1.197468in}}%
\pgfpathlineto{\pgfqpoint{6.394000in}{1.194313in}}%
\pgfpathlineto{\pgfqpoint{6.396480in}{1.198135in}}%
\pgfpathlineto{\pgfqpoint{6.397720in}{1.197793in}}%
\pgfpathlineto{\pgfqpoint{6.400200in}{1.201266in}}%
\pgfpathlineto{\pgfqpoint{6.401440in}{1.200030in}}%
\pgfpathlineto{\pgfqpoint{6.403920in}{1.203930in}}%
\pgfpathlineto{\pgfqpoint{6.406400in}{1.202019in}}%
\pgfpathlineto{\pgfqpoint{6.410120in}{1.202781in}}%
\pgfpathlineto{\pgfqpoint{6.417560in}{1.191424in}}%
\pgfpathlineto{\pgfqpoint{6.420040in}{1.189046in}}%
\pgfpathlineto{\pgfqpoint{6.421280in}{1.187875in}}%
\pgfpathlineto{\pgfqpoint{6.422520in}{1.188961in}}%
\pgfpathlineto{\pgfqpoint{6.426240in}{1.187846in}}%
\pgfpathlineto{\pgfqpoint{6.427480in}{1.187505in}}%
\pgfpathlineto{\pgfqpoint{6.429960in}{1.183916in}}%
\pgfpathlineto{\pgfqpoint{6.432440in}{1.184229in}}%
\pgfpathlineto{\pgfqpoint{6.433680in}{1.187548in}}%
\pgfpathlineto{\pgfqpoint{6.434920in}{1.186583in}}%
\pgfpathlineto{\pgfqpoint{6.437400in}{1.188985in}}%
\pgfpathlineto{\pgfqpoint{6.438640in}{1.186127in}}%
\pgfpathlineto{\pgfqpoint{6.443600in}{1.185597in}}%
\pgfpathlineto{\pgfqpoint{6.449800in}{1.180496in}}%
\pgfpathlineto{\pgfqpoint{6.452280in}{1.182427in}}%
\pgfpathlineto{\pgfqpoint{6.454760in}{1.181207in}}%
\pgfpathlineto{\pgfqpoint{6.456000in}{1.178538in}}%
\pgfpathlineto{\pgfqpoint{6.457240in}{1.184536in}}%
\pgfpathlineto{\pgfqpoint{6.458480in}{1.183854in}}%
\pgfpathlineto{\pgfqpoint{6.460960in}{1.188159in}}%
\pgfpathlineto{\pgfqpoint{6.464680in}{1.188506in}}%
\pgfpathlineto{\pgfqpoint{6.465920in}{1.190047in}}%
\pgfpathlineto{\pgfqpoint{6.467160in}{1.187737in}}%
\pgfpathlineto{\pgfqpoint{6.468400in}{1.189814in}}%
\pgfpathlineto{\pgfqpoint{6.469640in}{1.188773in}}%
\pgfpathlineto{\pgfqpoint{6.472120in}{1.190475in}}%
\pgfpathlineto{\pgfqpoint{6.474600in}{1.187426in}}%
\pgfpathlineto{\pgfqpoint{6.477080in}{1.180482in}}%
\pgfpathlineto{\pgfqpoint{6.480800in}{1.186725in}}%
\pgfpathlineto{\pgfqpoint{6.483280in}{1.181065in}}%
\pgfpathlineto{\pgfqpoint{6.488240in}{1.190752in}}%
\pgfpathlineto{\pgfqpoint{6.489480in}{1.190752in}}%
\pgfpathlineto{\pgfqpoint{6.491960in}{1.195015in}}%
\pgfpathlineto{\pgfqpoint{6.494440in}{1.199556in}}%
\pgfpathlineto{\pgfqpoint{6.495680in}{1.202390in}}%
\pgfpathlineto{\pgfqpoint{6.496920in}{1.199961in}}%
\pgfpathlineto{\pgfqpoint{6.499400in}{1.202045in}}%
\pgfpathlineto{\pgfqpoint{6.501880in}{1.202771in}}%
\pgfpathlineto{\pgfqpoint{6.504360in}{1.204267in}}%
\pgfpathlineto{\pgfqpoint{6.506840in}{1.198570in}}%
\pgfpathlineto{\pgfqpoint{6.509320in}{1.199808in}}%
\pgfpathlineto{\pgfqpoint{6.510560in}{1.193923in}}%
\pgfpathlineto{\pgfqpoint{6.511800in}{1.194414in}}%
\pgfpathlineto{\pgfqpoint{6.513040in}{1.198469in}}%
\pgfpathlineto{\pgfqpoint{6.515520in}{1.196512in}}%
\pgfpathlineto{\pgfqpoint{6.516760in}{1.197895in}}%
\pgfpathlineto{\pgfqpoint{6.518000in}{1.195255in}}%
\pgfpathlineto{\pgfqpoint{6.520480in}{1.200405in}}%
\pgfpathlineto{\pgfqpoint{6.522960in}{1.202568in}}%
\pgfpathlineto{\pgfqpoint{6.524200in}{1.205373in}}%
\pgfpathlineto{\pgfqpoint{6.525440in}{1.203312in}}%
\pgfpathlineto{\pgfqpoint{6.527920in}{1.207230in}}%
\pgfpathlineto{\pgfqpoint{6.529160in}{1.206638in}}%
\pgfpathlineto{\pgfqpoint{6.530400in}{1.208399in}}%
\pgfpathlineto{\pgfqpoint{6.532880in}{1.205291in}}%
\pgfpathlineto{\pgfqpoint{6.534120in}{1.206321in}}%
\pgfpathlineto{\pgfqpoint{6.545280in}{1.194218in}}%
\pgfpathlineto{\pgfqpoint{6.546520in}{1.195590in}}%
\pgfpathlineto{\pgfqpoint{6.549000in}{1.191400in}}%
\pgfpathlineto{\pgfqpoint{6.550240in}{1.193075in}}%
\pgfpathlineto{\pgfqpoint{6.551480in}{1.192203in}}%
\pgfpathlineto{\pgfqpoint{6.552720in}{1.189363in}}%
\pgfpathlineto{\pgfqpoint{6.556440in}{1.190481in}}%
\pgfpathlineto{\pgfqpoint{6.557680in}{1.193274in}}%
\pgfpathlineto{\pgfqpoint{6.558920in}{1.192554in}}%
\pgfpathlineto{\pgfqpoint{6.560160in}{1.193439in}}%
\pgfpathlineto{\pgfqpoint{6.561400in}{1.195908in}}%
\pgfpathlineto{\pgfqpoint{6.562640in}{1.192647in}}%
\pgfpathlineto{\pgfqpoint{6.565120in}{1.193228in}}%
\pgfpathlineto{\pgfqpoint{6.566360in}{1.193126in}}%
\pgfpathlineto{\pgfqpoint{6.570080in}{1.189238in}}%
\pgfpathlineto{\pgfqpoint{6.571320in}{1.189467in}}%
\pgfpathlineto{\pgfqpoint{6.573800in}{1.186357in}}%
\pgfpathlineto{\pgfqpoint{6.576280in}{1.187025in}}%
\pgfpathlineto{\pgfqpoint{6.578760in}{1.183214in}}%
\pgfpathlineto{\pgfqpoint{6.580000in}{1.178851in}}%
\pgfpathlineto{\pgfqpoint{6.581240in}{1.188247in}}%
\pgfpathlineto{\pgfqpoint{6.582480in}{1.187796in}}%
\pgfpathlineto{\pgfqpoint{6.584960in}{1.194004in}}%
\pgfpathlineto{\pgfqpoint{6.588680in}{1.192234in}}%
\pgfpathlineto{\pgfqpoint{6.589920in}{1.193441in}}%
\pgfpathlineto{\pgfqpoint{6.591160in}{1.188452in}}%
\pgfpathlineto{\pgfqpoint{6.594880in}{1.192397in}}%
\pgfpathlineto{\pgfqpoint{6.596120in}{1.195423in}}%
\pgfpathlineto{\pgfqpoint{6.598600in}{1.191038in}}%
\pgfpathlineto{\pgfqpoint{6.601080in}{1.181537in}}%
\pgfpathlineto{\pgfqpoint{6.604800in}{1.188798in}}%
\pgfpathlineto{\pgfqpoint{6.607280in}{1.183767in}}%
\pgfpathlineto{\pgfqpoint{6.613480in}{1.197637in}}%
\pgfpathlineto{\pgfqpoint{6.615960in}{1.200899in}}%
\pgfpathlineto{\pgfqpoint{6.617200in}{1.204734in}}%
\pgfpathlineto{\pgfqpoint{6.618440in}{1.204623in}}%
\pgfpathlineto{\pgfqpoint{6.619680in}{1.208702in}}%
\pgfpathlineto{\pgfqpoint{6.620920in}{1.204491in}}%
\pgfpathlineto{\pgfqpoint{6.623400in}{1.208953in}}%
\pgfpathlineto{\pgfqpoint{6.627120in}{1.209534in}}%
\pgfpathlineto{\pgfqpoint{6.628360in}{1.211911in}}%
\pgfpathlineto{\pgfqpoint{6.630840in}{1.208813in}}%
\pgfpathlineto{\pgfqpoint{6.633320in}{1.211040in}}%
\pgfpathlineto{\pgfqpoint{6.635800in}{1.204966in}}%
\pgfpathlineto{\pgfqpoint{6.637040in}{1.210034in}}%
\pgfpathlineto{\pgfqpoint{6.639520in}{1.207047in}}%
\pgfpathlineto{\pgfqpoint{6.640760in}{1.208436in}}%
\pgfpathlineto{\pgfqpoint{6.642000in}{1.204658in}}%
\pgfpathlineto{\pgfqpoint{6.645720in}{1.212552in}}%
\pgfpathlineto{\pgfqpoint{6.648200in}{1.216244in}}%
\pgfpathlineto{\pgfqpoint{6.649440in}{1.213311in}}%
\pgfpathlineto{\pgfqpoint{6.651920in}{1.219020in}}%
\pgfpathlineto{\pgfqpoint{6.653160in}{1.217970in}}%
\pgfpathlineto{\pgfqpoint{6.654400in}{1.220579in}}%
\pgfpathlineto{\pgfqpoint{6.660600in}{1.215023in}}%
\pgfpathlineto{\pgfqpoint{6.661840in}{1.216987in}}%
\pgfpathlineto{\pgfqpoint{6.665560in}{1.208078in}}%
\pgfpathlineto{\pgfqpoint{6.669280in}{1.204553in}}%
\pgfpathlineto{\pgfqpoint{6.670520in}{1.204740in}}%
\pgfpathlineto{\pgfqpoint{6.673000in}{1.198435in}}%
\pgfpathlineto{\pgfqpoint{6.675480in}{1.199811in}}%
\pgfpathlineto{\pgfqpoint{6.677960in}{1.195904in}}%
\pgfpathlineto{\pgfqpoint{6.679200in}{1.194325in}}%
\pgfpathlineto{\pgfqpoint{6.682920in}{1.196381in}}%
\pgfpathlineto{\pgfqpoint{6.685400in}{1.203759in}}%
\pgfpathlineto{\pgfqpoint{6.686640in}{1.200271in}}%
\pgfpathlineto{\pgfqpoint{6.687880in}{1.200251in}}%
\pgfpathlineto{\pgfqpoint{6.691600in}{1.192802in}}%
\pgfpathlineto{\pgfqpoint{6.695320in}{1.190239in}}%
\pgfpathlineto{\pgfqpoint{6.697800in}{1.186524in}}%
\pgfpathlineto{\pgfqpoint{6.700280in}{1.187340in}}%
\pgfpathlineto{\pgfqpoint{6.701520in}{1.185049in}}%
\pgfpathlineto{\pgfqpoint{6.702760in}{1.185481in}}%
\pgfpathlineto{\pgfqpoint{6.704000in}{1.182301in}}%
\pgfpathlineto{\pgfqpoint{6.706480in}{1.193328in}}%
\pgfpathlineto{\pgfqpoint{6.708960in}{1.199316in}}%
\pgfpathlineto{\pgfqpoint{6.710200in}{1.196589in}}%
\pgfpathlineto{\pgfqpoint{6.711440in}{1.199019in}}%
\pgfpathlineto{\pgfqpoint{6.713920in}{1.198417in}}%
\pgfpathlineto{\pgfqpoint{6.715160in}{1.193827in}}%
\pgfpathlineto{\pgfqpoint{6.716400in}{1.194688in}}%
\pgfpathlineto{\pgfqpoint{6.718880in}{1.193818in}}%
\pgfpathlineto{\pgfqpoint{6.720120in}{1.196789in}}%
\pgfpathlineto{\pgfqpoint{6.721360in}{1.195253in}}%
\pgfpathlineto{\pgfqpoint{6.725080in}{1.185009in}}%
\pgfpathlineto{\pgfqpoint{6.727560in}{1.189623in}}%
\pgfpathlineto{\pgfqpoint{6.728800in}{1.192007in}}%
\pgfpathlineto{\pgfqpoint{6.731280in}{1.186109in}}%
\pgfpathlineto{\pgfqpoint{6.736240in}{1.197734in}}%
\pgfpathlineto{\pgfqpoint{6.737480in}{1.198858in}}%
\pgfpathlineto{\pgfqpoint{6.741200in}{1.209513in}}%
\pgfpathlineto{\pgfqpoint{6.742440in}{1.207769in}}%
\pgfpathlineto{\pgfqpoint{6.743680in}{1.211332in}}%
\pgfpathlineto{\pgfqpoint{6.744920in}{1.204590in}}%
\pgfpathlineto{\pgfqpoint{6.749880in}{1.210895in}}%
\pgfpathlineto{\pgfqpoint{6.751120in}{1.210050in}}%
\pgfpathlineto{\pgfqpoint{6.752360in}{1.210654in}}%
\pgfpathlineto{\pgfqpoint{6.754840in}{1.207049in}}%
\pgfpathlineto{\pgfqpoint{6.757320in}{1.210969in}}%
\pgfpathlineto{\pgfqpoint{6.759800in}{1.203762in}}%
\pgfpathlineto{\pgfqpoint{6.761040in}{1.209564in}}%
\pgfpathlineto{\pgfqpoint{6.763520in}{1.205995in}}%
\pgfpathlineto{\pgfqpoint{6.764760in}{1.206931in}}%
\pgfpathlineto{\pgfqpoint{6.766000in}{1.204494in}}%
\pgfpathlineto{\pgfqpoint{6.769720in}{1.213812in}}%
\pgfpathlineto{\pgfqpoint{6.772200in}{1.216301in}}%
\pgfpathlineto{\pgfqpoint{6.773440in}{1.212222in}}%
\pgfpathlineto{\pgfqpoint{6.778400in}{1.224282in}}%
\pgfpathlineto{\pgfqpoint{6.784600in}{1.212326in}}%
\pgfpathlineto{\pgfqpoint{6.785840in}{1.214266in}}%
\pgfpathlineto{\pgfqpoint{6.788320in}{1.206695in}}%
\pgfpathlineto{\pgfqpoint{6.790800in}{1.204487in}}%
\pgfpathlineto{\pgfqpoint{6.792040in}{1.205181in}}%
\pgfpathlineto{\pgfqpoint{6.794520in}{1.208220in}}%
\pgfpathlineto{\pgfqpoint{6.798240in}{1.201832in}}%
\pgfpathlineto{\pgfqpoint{6.799480in}{1.201231in}}%
\pgfpathlineto{\pgfqpoint{6.800720in}{1.197336in}}%
\pgfpathlineto{\pgfqpoint{6.801960in}{1.198025in}}%
\pgfpathlineto{\pgfqpoint{6.803200in}{1.195327in}}%
\pgfpathlineto{\pgfqpoint{6.804440in}{1.197654in}}%
\pgfpathlineto{\pgfqpoint{6.805680in}{1.196583in}}%
\pgfpathlineto{\pgfqpoint{6.806920in}{1.197554in}}%
\pgfpathlineto{\pgfqpoint{6.809400in}{1.206692in}}%
\pgfpathlineto{\pgfqpoint{6.811880in}{1.203422in}}%
\pgfpathlineto{\pgfqpoint{6.814360in}{1.201531in}}%
\pgfpathlineto{\pgfqpoint{6.816840in}{1.198054in}}%
\pgfpathlineto{\pgfqpoint{6.818080in}{1.196076in}}%
\pgfpathlineto{\pgfqpoint{6.819320in}{1.196399in}}%
\pgfpathlineto{\pgfqpoint{6.821800in}{1.189450in}}%
\pgfpathlineto{\pgfqpoint{6.824280in}{1.186178in}}%
\pgfpathlineto{\pgfqpoint{6.825520in}{1.183700in}}%
\pgfpathlineto{\pgfqpoint{6.826760in}{1.186032in}}%
\pgfpathlineto{\pgfqpoint{6.828000in}{1.183182in}}%
\pgfpathlineto{\pgfqpoint{6.832960in}{1.197414in}}%
\pgfpathlineto{\pgfqpoint{6.834200in}{1.195149in}}%
\pgfpathlineto{\pgfqpoint{6.836680in}{1.202659in}}%
\pgfpathlineto{\pgfqpoint{6.837920in}{1.204014in}}%
\pgfpathlineto{\pgfqpoint{6.839160in}{1.200509in}}%
\pgfpathlineto{\pgfqpoint{6.844120in}{1.204745in}}%
\pgfpathlineto{\pgfqpoint{6.847840in}{1.193936in}}%
\pgfpathlineto{\pgfqpoint{6.849080in}{1.189305in}}%
\pgfpathlineto{\pgfqpoint{6.851560in}{1.194000in}}%
\pgfpathlineto{\pgfqpoint{6.852800in}{1.195616in}}%
\pgfpathlineto{\pgfqpoint{6.855280in}{1.190336in}}%
\pgfpathlineto{\pgfqpoint{6.856520in}{1.195353in}}%
\pgfpathlineto{\pgfqpoint{6.857760in}{1.194832in}}%
\pgfpathlineto{\pgfqpoint{6.862720in}{1.210793in}}%
\pgfpathlineto{\pgfqpoint{6.865200in}{1.216461in}}%
\pgfpathlineto{\pgfqpoint{6.866440in}{1.212681in}}%
\pgfpathlineto{\pgfqpoint{6.867680in}{1.218477in}}%
\pgfpathlineto{\pgfqpoint{6.868920in}{1.212239in}}%
\pgfpathlineto{\pgfqpoint{6.870160in}{1.214038in}}%
\pgfpathlineto{\pgfqpoint{6.873880in}{1.227896in}}%
\pgfpathlineto{\pgfqpoint{6.875120in}{1.226824in}}%
\pgfpathlineto{\pgfqpoint{6.876360in}{1.228571in}}%
\pgfpathlineto{\pgfqpoint{6.878840in}{1.217515in}}%
\pgfpathlineto{\pgfqpoint{6.881320in}{1.224098in}}%
\pgfpathlineto{\pgfqpoint{6.883800in}{1.218499in}}%
\pgfpathlineto{\pgfqpoint{6.885040in}{1.223316in}}%
\pgfpathlineto{\pgfqpoint{6.887520in}{1.219300in}}%
\pgfpathlineto{\pgfqpoint{6.890000in}{1.224984in}}%
\pgfpathlineto{\pgfqpoint{6.893720in}{1.230847in}}%
\pgfpathlineto{\pgfqpoint{6.894960in}{1.230426in}}%
\pgfpathlineto{\pgfqpoint{6.896200in}{1.232238in}}%
\pgfpathlineto{\pgfqpoint{6.897440in}{1.229002in}}%
\pgfpathlineto{\pgfqpoint{6.901160in}{1.243591in}}%
\pgfpathlineto{\pgfqpoint{6.902400in}{1.242839in}}%
\pgfpathlineto{\pgfqpoint{6.907360in}{1.230038in}}%
\pgfpathlineto{\pgfqpoint{6.909840in}{1.229362in}}%
\pgfpathlineto{\pgfqpoint{6.913560in}{1.216592in}}%
\pgfpathlineto{\pgfqpoint{6.916040in}{1.213771in}}%
\pgfpathlineto{\pgfqpoint{6.918520in}{1.221678in}}%
\pgfpathlineto{\pgfqpoint{6.919760in}{1.221496in}}%
\pgfpathlineto{\pgfqpoint{6.921000in}{1.218260in}}%
\pgfpathlineto{\pgfqpoint{6.922240in}{1.219485in}}%
\pgfpathlineto{\pgfqpoint{6.923480in}{1.218919in}}%
\pgfpathlineto{\pgfqpoint{6.927200in}{1.211385in}}%
\pgfpathlineto{\pgfqpoint{6.928440in}{1.213613in}}%
\pgfpathlineto{\pgfqpoint{6.930920in}{1.213926in}}%
\pgfpathlineto{\pgfqpoint{6.933400in}{1.225329in}}%
\pgfpathlineto{\pgfqpoint{6.939600in}{1.213941in}}%
\pgfpathlineto{\pgfqpoint{6.940840in}{1.215046in}}%
\pgfpathlineto{\pgfqpoint{6.942080in}{1.214158in}}%
\pgfpathlineto{\pgfqpoint{6.943320in}{1.214817in}}%
\pgfpathlineto{\pgfqpoint{6.945800in}{1.207923in}}%
\pgfpathlineto{\pgfqpoint{6.948280in}{1.204285in}}%
\pgfpathlineto{\pgfqpoint{6.949520in}{1.201081in}}%
\pgfpathlineto{\pgfqpoint{6.950760in}{1.204621in}}%
\pgfpathlineto{\pgfqpoint{6.952000in}{1.201559in}}%
\pgfpathlineto{\pgfqpoint{6.954480in}{1.205424in}}%
\pgfpathlineto{\pgfqpoint{6.955720in}{1.211417in}}%
\pgfpathlineto{\pgfqpoint{6.958200in}{1.204322in}}%
\pgfpathlineto{\pgfqpoint{6.961920in}{1.214640in}}%
\pgfpathlineto{\pgfqpoint{6.963160in}{1.209751in}}%
\pgfpathlineto{\pgfqpoint{6.964400in}{1.210497in}}%
\pgfpathlineto{\pgfqpoint{6.968120in}{1.215057in}}%
\pgfpathlineto{\pgfqpoint{6.969360in}{1.213041in}}%
\pgfpathlineto{\pgfqpoint{6.973080in}{1.195927in}}%
\pgfpathlineto{\pgfqpoint{6.974320in}{1.198207in}}%
\pgfpathlineto{\pgfqpoint{6.976800in}{1.204768in}}%
\pgfpathlineto{\pgfqpoint{6.978040in}{1.197568in}}%
\pgfpathlineto{\pgfqpoint{6.979280in}{1.197529in}}%
\pgfpathlineto{\pgfqpoint{6.986720in}{1.231054in}}%
\pgfpathlineto{\pgfqpoint{6.987960in}{1.228741in}}%
\pgfpathlineto{\pgfqpoint{6.989200in}{1.231816in}}%
\pgfpathlineto{\pgfqpoint{6.990440in}{1.230561in}}%
\pgfpathlineto{\pgfqpoint{6.991680in}{1.238523in}}%
\pgfpathlineto{\pgfqpoint{6.994160in}{1.225599in}}%
\pgfpathlineto{\pgfqpoint{6.997880in}{1.236573in}}%
\pgfpathlineto{\pgfqpoint{6.999120in}{1.237010in}}%
\pgfpathlineto{\pgfqpoint{7.000360in}{1.243496in}}%
\pgfpathlineto{\pgfqpoint{7.002840in}{1.226946in}}%
\pgfpathlineto{\pgfqpoint{7.005320in}{1.233965in}}%
\pgfpathlineto{\pgfqpoint{7.006560in}{1.225590in}}%
\pgfpathlineto{\pgfqpoint{7.007800in}{1.226207in}}%
\pgfpathlineto{\pgfqpoint{7.009040in}{1.234169in}}%
\pgfpathlineto{\pgfqpoint{7.011520in}{1.224474in}}%
\pgfpathlineto{\pgfqpoint{7.012760in}{1.226586in}}%
\pgfpathlineto{\pgfqpoint{7.014000in}{1.223566in}}%
\pgfpathlineto{\pgfqpoint{7.016480in}{1.227835in}}%
\pgfpathlineto{\pgfqpoint{7.017720in}{1.235225in}}%
\pgfpathlineto{\pgfqpoint{7.020200in}{1.237054in}}%
\pgfpathlineto{\pgfqpoint{7.021440in}{1.233448in}}%
\pgfpathlineto{\pgfqpoint{7.026400in}{1.253310in}}%
\pgfpathlineto{\pgfqpoint{7.027640in}{1.247642in}}%
\pgfpathlineto{\pgfqpoint{7.028880in}{1.233781in}}%
\pgfpathlineto{\pgfqpoint{7.031360in}{1.232944in}}%
\pgfpathlineto{\pgfqpoint{7.032600in}{1.231309in}}%
\pgfpathlineto{\pgfqpoint{7.033840in}{1.233638in}}%
\pgfpathlineto{\pgfqpoint{7.036320in}{1.223256in}}%
\pgfpathlineto{\pgfqpoint{7.037560in}{1.223509in}}%
\pgfpathlineto{\pgfqpoint{7.041280in}{1.228287in}}%
\pgfpathlineto{\pgfqpoint{7.042520in}{1.228202in}}%
\pgfpathlineto{\pgfqpoint{7.043760in}{1.229550in}}%
\pgfpathlineto{\pgfqpoint{7.045000in}{1.228837in}}%
\pgfpathlineto{\pgfqpoint{7.051200in}{1.207672in}}%
\pgfpathlineto{\pgfqpoint{7.052440in}{1.208361in}}%
\pgfpathlineto{\pgfqpoint{7.054920in}{1.206729in}}%
\pgfpathlineto{\pgfqpoint{7.056160in}{1.210456in}}%
\pgfpathlineto{\pgfqpoint{7.057400in}{1.218553in}}%
\pgfpathlineto{\pgfqpoint{7.058640in}{1.217664in}}%
\pgfpathlineto{\pgfqpoint{7.059880in}{1.219532in}}%
\pgfpathlineto{\pgfqpoint{7.061120in}{1.218614in}}%
\pgfpathlineto{\pgfqpoint{7.062360in}{1.215980in}}%
\pgfpathlineto{\pgfqpoint{7.064840in}{1.217437in}}%
\pgfpathlineto{\pgfqpoint{7.066080in}{1.216666in}}%
\pgfpathlineto{\pgfqpoint{7.067320in}{1.221969in}}%
\pgfpathlineto{\pgfqpoint{7.068560in}{1.218170in}}%
\pgfpathlineto{\pgfqpoint{7.069800in}{1.219539in}}%
\pgfpathlineto{\pgfqpoint{7.071040in}{1.222795in}}%
\pgfpathlineto{\pgfqpoint{7.073520in}{1.215367in}}%
\pgfpathlineto{\pgfqpoint{7.074760in}{1.218675in}}%
\pgfpathlineto{\pgfqpoint{7.076000in}{1.216967in}}%
\pgfpathlineto{\pgfqpoint{7.077240in}{1.233312in}}%
\pgfpathlineto{\pgfqpoint{7.078480in}{1.234612in}}%
\pgfpathlineto{\pgfqpoint{7.079720in}{1.240101in}}%
\pgfpathlineto{\pgfqpoint{7.080960in}{1.238523in}}%
\pgfpathlineto{\pgfqpoint{7.082200in}{1.233374in}}%
\pgfpathlineto{\pgfqpoint{7.083440in}{1.235899in}}%
\pgfpathlineto{\pgfqpoint{7.084680in}{1.245609in}}%
\pgfpathlineto{\pgfqpoint{7.087160in}{1.234830in}}%
\pgfpathlineto{\pgfqpoint{7.089640in}{1.229592in}}%
\pgfpathlineto{\pgfqpoint{7.090880in}{1.226309in}}%
\pgfpathlineto{\pgfqpoint{7.093360in}{1.217595in}}%
\pgfpathlineto{\pgfqpoint{7.097080in}{1.203863in}}%
\pgfpathlineto{\pgfqpoint{7.098320in}{1.203122in}}%
\pgfpathlineto{\pgfqpoint{7.100800in}{1.214429in}}%
\pgfpathlineto{\pgfqpoint{7.103280in}{1.203619in}}%
\pgfpathlineto{\pgfqpoint{7.107000in}{1.218500in}}%
\pgfpathlineto{\pgfqpoint{7.110720in}{1.253262in}}%
\pgfpathlineto{\pgfqpoint{7.113200in}{1.242714in}}%
\pgfpathlineto{\pgfqpoint{7.114440in}{1.243564in}}%
\pgfpathlineto{\pgfqpoint{7.115680in}{1.250553in}}%
\pgfpathlineto{\pgfqpoint{7.118160in}{1.237787in}}%
\pgfpathlineto{\pgfqpoint{7.119400in}{1.244542in}}%
\pgfpathlineto{\pgfqpoint{7.120640in}{1.245012in}}%
\pgfpathlineto{\pgfqpoint{7.121880in}{1.249473in}}%
\pgfpathlineto{\pgfqpoint{7.123120in}{1.249626in}}%
\pgfpathlineto{\pgfqpoint{7.124360in}{1.255575in}}%
\pgfpathlineto{\pgfqpoint{7.126840in}{1.239531in}}%
\pgfpathlineto{\pgfqpoint{7.129320in}{1.256143in}}%
\pgfpathlineto{\pgfqpoint{7.130560in}{1.250842in}}%
\pgfpathlineto{\pgfqpoint{7.131800in}{1.253596in}}%
\pgfpathlineto{\pgfqpoint{7.133040in}{1.263226in}}%
\pgfpathlineto{\pgfqpoint{7.134280in}{1.261595in}}%
\pgfpathlineto{\pgfqpoint{7.136760in}{1.254730in}}%
\pgfpathlineto{\pgfqpoint{7.139240in}{1.260125in}}%
\pgfpathlineto{\pgfqpoint{7.140480in}{1.261282in}}%
\pgfpathlineto{\pgfqpoint{7.142960in}{1.275065in}}%
\pgfpathlineto{\pgfqpoint{7.144200in}{1.271194in}}%
\pgfpathlineto{\pgfqpoint{7.146680in}{1.258689in}}%
\pgfpathlineto{\pgfqpoint{7.150400in}{1.274824in}}%
\pgfpathlineto{\pgfqpoint{7.152880in}{1.254408in}}%
\pgfpathlineto{\pgfqpoint{7.154120in}{1.263169in}}%
\pgfpathlineto{\pgfqpoint{7.156600in}{1.257851in}}%
\pgfpathlineto{\pgfqpoint{7.157840in}{1.263077in}}%
\pgfpathlineto{\pgfqpoint{7.165280in}{1.239407in}}%
\pgfpathlineto{\pgfqpoint{7.167760in}{1.252977in}}%
\pgfpathlineto{\pgfqpoint{7.170240in}{1.250655in}}%
\pgfpathlineto{\pgfqpoint{7.172720in}{1.236231in}}%
\pgfpathlineto{\pgfqpoint{7.173960in}{1.239657in}}%
\pgfpathlineto{\pgfqpoint{7.176440in}{1.226953in}}%
\pgfpathlineto{\pgfqpoint{7.177680in}{1.225070in}}%
\pgfpathlineto{\pgfqpoint{7.180160in}{1.231232in}}%
\pgfpathlineto{\pgfqpoint{7.181400in}{1.243662in}}%
\pgfpathlineto{\pgfqpoint{7.182640in}{1.241964in}}%
\pgfpathlineto{\pgfqpoint{7.185120in}{1.246430in}}%
\pgfpathlineto{\pgfqpoint{7.186360in}{1.242218in}}%
\pgfpathlineto{\pgfqpoint{7.187600in}{1.242766in}}%
\pgfpathlineto{\pgfqpoint{7.190080in}{1.246210in}}%
\pgfpathlineto{\pgfqpoint{7.191320in}{1.248606in}}%
\pgfpathlineto{\pgfqpoint{7.193800in}{1.240855in}}%
\pgfpathlineto{\pgfqpoint{7.195040in}{1.241134in}}%
\pgfpathlineto{\pgfqpoint{7.196280in}{1.236722in}}%
\pgfpathlineto{\pgfqpoint{7.197520in}{1.237976in}}%
\pgfpathlineto{\pgfqpoint{7.200000in}{1.245004in}}%
\pgfpathlineto{\pgfqpoint{7.200000in}{1.245004in}}%
\pgfusepath{stroke}%
\end{pgfscope}%
\begin{pgfscope}%
\pgfpathrectangle{\pgfqpoint{1.000000in}{0.350000in}}{\pgfqpoint{6.200000in}{2.800000in}} %
\pgfusepath{clip}%
\pgfsetrectcap%
\pgfsetroundjoin%
\pgfsetlinewidth{1.003750pt}%
\definecolor{currentstroke}{rgb}{0.000000,0.500000,0.000000}%
\pgfsetstrokecolor{currentstroke}%
\pgfsetdash{}{0pt}%
\pgfpathmoveto{\pgfqpoint{1.001240in}{1.559994in}}%
\pgfpathlineto{\pgfqpoint{1.002480in}{2.017291in}}%
\pgfpathlineto{\pgfqpoint{1.003720in}{2.067136in}}%
\pgfpathlineto{\pgfqpoint{1.011160in}{1.344012in}}%
\pgfpathlineto{\pgfqpoint{1.016120in}{1.117958in}}%
\pgfpathlineto{\pgfqpoint{1.021080in}{0.988717in}}%
\pgfpathlineto{\pgfqpoint{1.027280in}{0.901845in}}%
\pgfpathlineto{\pgfqpoint{1.031000in}{0.867336in}}%
\pgfpathlineto{\pgfqpoint{1.032240in}{0.867657in}}%
\pgfpathlineto{\pgfqpoint{1.035960in}{0.856464in}}%
\pgfpathlineto{\pgfqpoint{1.037200in}{0.856154in}}%
\pgfpathlineto{\pgfqpoint{1.038440in}{0.853431in}}%
\pgfpathlineto{\pgfqpoint{1.042160in}{0.837417in}}%
\pgfpathlineto{\pgfqpoint{1.044640in}{0.840081in}}%
\pgfpathlineto{\pgfqpoint{1.047120in}{0.821691in}}%
\pgfpathlineto{\pgfqpoint{1.050840in}{0.798015in}}%
\pgfpathlineto{\pgfqpoint{1.053320in}{0.792598in}}%
\pgfpathlineto{\pgfqpoint{1.055800in}{0.782629in}}%
\pgfpathlineto{\pgfqpoint{1.059520in}{0.777147in}}%
\pgfpathlineto{\pgfqpoint{1.062000in}{0.766540in}}%
\pgfpathlineto{\pgfqpoint{1.066960in}{0.741524in}}%
\pgfpathlineto{\pgfqpoint{1.068200in}{0.740294in}}%
\pgfpathlineto{\pgfqpoint{1.073160in}{0.744642in}}%
\pgfpathlineto{\pgfqpoint{1.078120in}{0.739982in}}%
\pgfpathlineto{\pgfqpoint{1.079360in}{0.740315in}}%
\pgfpathlineto{\pgfqpoint{1.080600in}{0.738251in}}%
\pgfpathlineto{\pgfqpoint{1.081840in}{0.740649in}}%
\pgfpathlineto{\pgfqpoint{1.086800in}{0.723733in}}%
\pgfpathlineto{\pgfqpoint{1.089280in}{0.713006in}}%
\pgfpathlineto{\pgfqpoint{1.090520in}{0.711869in}}%
\pgfpathlineto{\pgfqpoint{1.091760in}{0.714224in}}%
\pgfpathlineto{\pgfqpoint{1.097960in}{0.697867in}}%
\pgfpathlineto{\pgfqpoint{1.100440in}{0.697369in}}%
\pgfpathlineto{\pgfqpoint{1.101680in}{0.702340in}}%
\pgfpathlineto{\pgfqpoint{1.109120in}{0.697254in}}%
\pgfpathlineto{\pgfqpoint{1.111600in}{0.700945in}}%
\pgfpathlineto{\pgfqpoint{1.112840in}{0.699345in}}%
\pgfpathlineto{\pgfqpoint{1.114080in}{0.700180in}}%
\pgfpathlineto{\pgfqpoint{1.115320in}{0.702782in}}%
\pgfpathlineto{\pgfqpoint{1.116560in}{0.702703in}}%
\pgfpathlineto{\pgfqpoint{1.117800in}{0.700879in}}%
\pgfpathlineto{\pgfqpoint{1.120280in}{0.703394in}}%
\pgfpathlineto{\pgfqpoint{1.122760in}{0.699764in}}%
\pgfpathlineto{\pgfqpoint{1.124000in}{0.700601in}}%
\pgfpathlineto{\pgfqpoint{1.126480in}{0.693595in}}%
\pgfpathlineto{\pgfqpoint{1.127720in}{0.694266in}}%
\pgfpathlineto{\pgfqpoint{1.131440in}{0.701272in}}%
\pgfpathlineto{\pgfqpoint{1.132680in}{0.701042in}}%
\pgfpathlineto{\pgfqpoint{1.133920in}{0.702689in}}%
\pgfpathlineto{\pgfqpoint{1.142600in}{0.688868in}}%
\pgfpathlineto{\pgfqpoint{1.143840in}{0.688628in}}%
\pgfpathlineto{\pgfqpoint{1.155000in}{0.662231in}}%
\pgfpathlineto{\pgfqpoint{1.157480in}{0.668991in}}%
\pgfpathlineto{\pgfqpoint{1.159960in}{0.674142in}}%
\pgfpathlineto{\pgfqpoint{1.164920in}{0.666663in}}%
\pgfpathlineto{\pgfqpoint{1.166160in}{0.666157in}}%
\pgfpathlineto{\pgfqpoint{1.168640in}{0.669331in}}%
\pgfpathlineto{\pgfqpoint{1.171120in}{0.663877in}}%
\pgfpathlineto{\pgfqpoint{1.172360in}{0.662394in}}%
\pgfpathlineto{\pgfqpoint{1.173600in}{0.663852in}}%
\pgfpathlineto{\pgfqpoint{1.176080in}{0.661162in}}%
\pgfpathlineto{\pgfqpoint{1.181040in}{0.661729in}}%
\pgfpathlineto{\pgfqpoint{1.183520in}{0.665645in}}%
\pgfpathlineto{\pgfqpoint{1.184760in}{0.665795in}}%
\pgfpathlineto{\pgfqpoint{1.187240in}{0.656819in}}%
\pgfpathlineto{\pgfqpoint{1.188480in}{0.657884in}}%
\pgfpathlineto{\pgfqpoint{1.192200in}{0.651002in}}%
\pgfpathlineto{\pgfqpoint{1.195920in}{0.652729in}}%
\pgfpathlineto{\pgfqpoint{1.197160in}{0.655554in}}%
\pgfpathlineto{\pgfqpoint{1.199640in}{0.655134in}}%
\pgfpathlineto{\pgfqpoint{1.200880in}{0.655491in}}%
\pgfpathlineto{\pgfqpoint{1.202120in}{0.658108in}}%
\pgfpathlineto{\pgfqpoint{1.204600in}{0.656152in}}%
\pgfpathlineto{\pgfqpoint{1.205840in}{0.658119in}}%
\pgfpathlineto{\pgfqpoint{1.208320in}{0.653679in}}%
\pgfpathlineto{\pgfqpoint{1.210800in}{0.651796in}}%
\pgfpathlineto{\pgfqpoint{1.213280in}{0.648088in}}%
\pgfpathlineto{\pgfqpoint{1.214520in}{0.647720in}}%
\pgfpathlineto{\pgfqpoint{1.217000in}{0.650237in}}%
\pgfpathlineto{\pgfqpoint{1.221960in}{0.644594in}}%
\pgfpathlineto{\pgfqpoint{1.224440in}{0.647120in}}%
\pgfpathlineto{\pgfqpoint{1.225680in}{0.647846in}}%
\pgfpathlineto{\pgfqpoint{1.228160in}{0.642659in}}%
\pgfpathlineto{\pgfqpoint{1.229400in}{0.641957in}}%
\pgfpathlineto{\pgfqpoint{1.230640in}{0.644117in}}%
\pgfpathlineto{\pgfqpoint{1.231880in}{0.643542in}}%
\pgfpathlineto{\pgfqpoint{1.233120in}{0.644519in}}%
\pgfpathlineto{\pgfqpoint{1.235600in}{0.651166in}}%
\pgfpathlineto{\pgfqpoint{1.241800in}{0.649587in}}%
\pgfpathlineto{\pgfqpoint{1.244280in}{0.651825in}}%
\pgfpathlineto{\pgfqpoint{1.249240in}{0.642512in}}%
\pgfpathlineto{\pgfqpoint{1.251720in}{0.645745in}}%
\pgfpathlineto{\pgfqpoint{1.254200in}{0.648626in}}%
\pgfpathlineto{\pgfqpoint{1.256680in}{0.648772in}}%
\pgfpathlineto{\pgfqpoint{1.257920in}{0.651558in}}%
\pgfpathlineto{\pgfqpoint{1.262880in}{0.646617in}}%
\pgfpathlineto{\pgfqpoint{1.265360in}{0.647756in}}%
\pgfpathlineto{\pgfqpoint{1.267840in}{0.651543in}}%
\pgfpathlineto{\pgfqpoint{1.270320in}{0.648619in}}%
\pgfpathlineto{\pgfqpoint{1.275280in}{0.643935in}}%
\pgfpathlineto{\pgfqpoint{1.279000in}{0.635950in}}%
\pgfpathlineto{\pgfqpoint{1.283960in}{0.641294in}}%
\pgfpathlineto{\pgfqpoint{1.286440in}{0.638121in}}%
\pgfpathlineto{\pgfqpoint{1.287680in}{0.637550in}}%
\pgfpathlineto{\pgfqpoint{1.292640in}{0.640858in}}%
\pgfpathlineto{\pgfqpoint{1.295120in}{0.638549in}}%
\pgfpathlineto{\pgfqpoint{1.296360in}{0.638574in}}%
\pgfpathlineto{\pgfqpoint{1.297600in}{0.640556in}}%
\pgfpathlineto{\pgfqpoint{1.300080in}{0.638592in}}%
\pgfpathlineto{\pgfqpoint{1.302560in}{0.640825in}}%
\pgfpathlineto{\pgfqpoint{1.307520in}{0.644305in}}%
\pgfpathlineto{\pgfqpoint{1.308760in}{0.644031in}}%
\pgfpathlineto{\pgfqpoint{1.311240in}{0.637941in}}%
\pgfpathlineto{\pgfqpoint{1.312480in}{0.638374in}}%
\pgfpathlineto{\pgfqpoint{1.316200in}{0.630033in}}%
\pgfpathlineto{\pgfqpoint{1.321160in}{0.634112in}}%
\pgfpathlineto{\pgfqpoint{1.323640in}{0.632007in}}%
\pgfpathlineto{\pgfqpoint{1.327360in}{0.636399in}}%
\pgfpathlineto{\pgfqpoint{1.328600in}{0.634766in}}%
\pgfpathlineto{\pgfqpoint{1.329840in}{0.636476in}}%
\pgfpathlineto{\pgfqpoint{1.336040in}{0.629480in}}%
\pgfpathlineto{\pgfqpoint{1.338520in}{0.629392in}}%
\pgfpathlineto{\pgfqpoint{1.341000in}{0.634672in}}%
\pgfpathlineto{\pgfqpoint{1.342240in}{0.633059in}}%
\pgfpathlineto{\pgfqpoint{1.343480in}{0.633373in}}%
\pgfpathlineto{\pgfqpoint{1.344720in}{0.632096in}}%
\pgfpathlineto{\pgfqpoint{1.345960in}{0.628409in}}%
\pgfpathlineto{\pgfqpoint{1.349680in}{0.631423in}}%
\pgfpathlineto{\pgfqpoint{1.352160in}{0.628401in}}%
\pgfpathlineto{\pgfqpoint{1.353400in}{0.628519in}}%
\pgfpathlineto{\pgfqpoint{1.355880in}{0.629606in}}%
\pgfpathlineto{\pgfqpoint{1.357120in}{0.629201in}}%
\pgfpathlineto{\pgfqpoint{1.362080in}{0.635309in}}%
\pgfpathlineto{\pgfqpoint{1.363320in}{0.634102in}}%
\pgfpathlineto{\pgfqpoint{1.364560in}{0.634656in}}%
\pgfpathlineto{\pgfqpoint{1.365800in}{0.633751in}}%
\pgfpathlineto{\pgfqpoint{1.367040in}{0.635255in}}%
\pgfpathlineto{\pgfqpoint{1.368280in}{0.634923in}}%
\pgfpathlineto{\pgfqpoint{1.372000in}{0.628773in}}%
\pgfpathlineto{\pgfqpoint{1.373240in}{0.628171in}}%
\pgfpathlineto{\pgfqpoint{1.376960in}{0.632893in}}%
\pgfpathlineto{\pgfqpoint{1.379440in}{0.632457in}}%
\pgfpathlineto{\pgfqpoint{1.381920in}{0.636502in}}%
\pgfpathlineto{\pgfqpoint{1.386880in}{0.631137in}}%
\pgfpathlineto{\pgfqpoint{1.391840in}{0.636294in}}%
\pgfpathlineto{\pgfqpoint{1.395560in}{0.633953in}}%
\pgfpathlineto{\pgfqpoint{1.396800in}{0.634333in}}%
\pgfpathlineto{\pgfqpoint{1.398040in}{0.632252in}}%
\pgfpathlineto{\pgfqpoint{1.400520in}{0.631477in}}%
\pgfpathlineto{\pgfqpoint{1.404240in}{0.628384in}}%
\pgfpathlineto{\pgfqpoint{1.407960in}{0.633001in}}%
\pgfpathlineto{\pgfqpoint{1.412920in}{0.630087in}}%
\pgfpathlineto{\pgfqpoint{1.415400in}{0.630458in}}%
\pgfpathlineto{\pgfqpoint{1.416640in}{0.633725in}}%
\pgfpathlineto{\pgfqpoint{1.420360in}{0.631903in}}%
\pgfpathlineto{\pgfqpoint{1.421600in}{0.632885in}}%
\pgfpathlineto{\pgfqpoint{1.424080in}{0.631998in}}%
\pgfpathlineto{\pgfqpoint{1.425320in}{0.633141in}}%
\pgfpathlineto{\pgfqpoint{1.426560in}{0.632309in}}%
\pgfpathlineto{\pgfqpoint{1.432760in}{0.635976in}}%
\pgfpathlineto{\pgfqpoint{1.435240in}{0.629672in}}%
\pgfpathlineto{\pgfqpoint{1.436480in}{0.630713in}}%
\pgfpathlineto{\pgfqpoint{1.440200in}{0.624529in}}%
\pgfpathlineto{\pgfqpoint{1.442680in}{0.627929in}}%
\pgfpathlineto{\pgfqpoint{1.446400in}{0.629476in}}%
\pgfpathlineto{\pgfqpoint{1.447640in}{0.628995in}}%
\pgfpathlineto{\pgfqpoint{1.451360in}{0.632111in}}%
\pgfpathlineto{\pgfqpoint{1.452600in}{0.631239in}}%
\pgfpathlineto{\pgfqpoint{1.453840in}{0.632468in}}%
\pgfpathlineto{\pgfqpoint{1.456320in}{0.630335in}}%
\pgfpathlineto{\pgfqpoint{1.457560in}{0.630169in}}%
\pgfpathlineto{\pgfqpoint{1.460040in}{0.625592in}}%
\pgfpathlineto{\pgfqpoint{1.462520in}{0.627381in}}%
\pgfpathlineto{\pgfqpoint{1.465000in}{0.633700in}}%
\pgfpathlineto{\pgfqpoint{1.466240in}{0.633031in}}%
\pgfpathlineto{\pgfqpoint{1.467480in}{0.633895in}}%
\pgfpathlineto{\pgfqpoint{1.469960in}{0.629590in}}%
\pgfpathlineto{\pgfqpoint{1.473680in}{0.632085in}}%
\pgfpathlineto{\pgfqpoint{1.474920in}{0.630084in}}%
\pgfpathlineto{\pgfqpoint{1.482360in}{0.632786in}}%
\pgfpathlineto{\pgfqpoint{1.484840in}{0.636002in}}%
\pgfpathlineto{\pgfqpoint{1.486080in}{0.636148in}}%
\pgfpathlineto{\pgfqpoint{1.489800in}{0.633547in}}%
\pgfpathlineto{\pgfqpoint{1.492280in}{0.635762in}}%
\pgfpathlineto{\pgfqpoint{1.496000in}{0.629017in}}%
\pgfpathlineto{\pgfqpoint{1.497240in}{0.629506in}}%
\pgfpathlineto{\pgfqpoint{1.500960in}{0.635034in}}%
\pgfpathlineto{\pgfqpoint{1.503440in}{0.636340in}}%
\pgfpathlineto{\pgfqpoint{1.505920in}{0.640004in}}%
\pgfpathlineto{\pgfqpoint{1.510880in}{0.635424in}}%
\pgfpathlineto{\pgfqpoint{1.515840in}{0.637189in}}%
\pgfpathlineto{\pgfqpoint{1.519560in}{0.635712in}}%
\pgfpathlineto{\pgfqpoint{1.520800in}{0.636388in}}%
\pgfpathlineto{\pgfqpoint{1.522040in}{0.633742in}}%
\pgfpathlineto{\pgfqpoint{1.524520in}{0.632780in}}%
\pgfpathlineto{\pgfqpoint{1.525760in}{0.632209in}}%
\pgfpathlineto{\pgfqpoint{1.528240in}{0.629849in}}%
\pgfpathlineto{\pgfqpoint{1.531960in}{0.634201in}}%
\pgfpathlineto{\pgfqpoint{1.534440in}{0.634697in}}%
\pgfpathlineto{\pgfqpoint{1.538160in}{0.635333in}}%
\pgfpathlineto{\pgfqpoint{1.541880in}{0.636260in}}%
\pgfpathlineto{\pgfqpoint{1.544360in}{0.634006in}}%
\pgfpathlineto{\pgfqpoint{1.546840in}{0.633972in}}%
\pgfpathlineto{\pgfqpoint{1.548080in}{0.633397in}}%
\pgfpathlineto{\pgfqpoint{1.549320in}{0.634523in}}%
\pgfpathlineto{\pgfqpoint{1.550560in}{0.633919in}}%
\pgfpathlineto{\pgfqpoint{1.553040in}{0.635227in}}%
\pgfpathlineto{\pgfqpoint{1.555520in}{0.635294in}}%
\pgfpathlineto{\pgfqpoint{1.556760in}{0.635892in}}%
\pgfpathlineto{\pgfqpoint{1.559240in}{0.629765in}}%
\pgfpathlineto{\pgfqpoint{1.560480in}{0.630858in}}%
\pgfpathlineto{\pgfqpoint{1.564200in}{0.626720in}}%
\pgfpathlineto{\pgfqpoint{1.566680in}{0.629923in}}%
\pgfpathlineto{\pgfqpoint{1.569160in}{0.630386in}}%
\pgfpathlineto{\pgfqpoint{1.572880in}{0.631167in}}%
\pgfpathlineto{\pgfqpoint{1.575360in}{0.633475in}}%
\pgfpathlineto{\pgfqpoint{1.576600in}{0.631722in}}%
\pgfpathlineto{\pgfqpoint{1.577840in}{0.633142in}}%
\pgfpathlineto{\pgfqpoint{1.580320in}{0.630993in}}%
\pgfpathlineto{\pgfqpoint{1.581560in}{0.631132in}}%
\pgfpathlineto{\pgfqpoint{1.584040in}{0.627380in}}%
\pgfpathlineto{\pgfqpoint{1.586520in}{0.628908in}}%
\pgfpathlineto{\pgfqpoint{1.589000in}{0.635323in}}%
\pgfpathlineto{\pgfqpoint{1.590240in}{0.634558in}}%
\pgfpathlineto{\pgfqpoint{1.592720in}{0.634717in}}%
\pgfpathlineto{\pgfqpoint{1.593960in}{0.632715in}}%
\pgfpathlineto{\pgfqpoint{1.595200in}{0.633539in}}%
\pgfpathlineto{\pgfqpoint{1.600160in}{0.630780in}}%
\pgfpathlineto{\pgfqpoint{1.606360in}{0.631325in}}%
\pgfpathlineto{\pgfqpoint{1.608840in}{0.634735in}}%
\pgfpathlineto{\pgfqpoint{1.610080in}{0.636043in}}%
\pgfpathlineto{\pgfqpoint{1.613800in}{0.634207in}}%
\pgfpathlineto{\pgfqpoint{1.616280in}{0.635966in}}%
\pgfpathlineto{\pgfqpoint{1.621240in}{0.630929in}}%
\pgfpathlineto{\pgfqpoint{1.624960in}{0.635609in}}%
\pgfpathlineto{\pgfqpoint{1.627440in}{0.635683in}}%
\pgfpathlineto{\pgfqpoint{1.629920in}{0.640446in}}%
\pgfpathlineto{\pgfqpoint{1.634880in}{0.636313in}}%
\pgfpathlineto{\pgfqpoint{1.638600in}{0.636625in}}%
\pgfpathlineto{\pgfqpoint{1.639840in}{0.636606in}}%
\pgfpathlineto{\pgfqpoint{1.643560in}{0.634418in}}%
\pgfpathlineto{\pgfqpoint{1.644800in}{0.635432in}}%
\pgfpathlineto{\pgfqpoint{1.647280in}{0.633957in}}%
\pgfpathlineto{\pgfqpoint{1.652240in}{0.630313in}}%
\pgfpathlineto{\pgfqpoint{1.655960in}{0.633648in}}%
\pgfpathlineto{\pgfqpoint{1.663400in}{0.635292in}}%
\pgfpathlineto{\pgfqpoint{1.664640in}{0.636803in}}%
\pgfpathlineto{\pgfqpoint{1.668360in}{0.633958in}}%
\pgfpathlineto{\pgfqpoint{1.669600in}{0.634319in}}%
\pgfpathlineto{\pgfqpoint{1.670840in}{0.633243in}}%
\pgfpathlineto{\pgfqpoint{1.674560in}{0.634594in}}%
\pgfpathlineto{\pgfqpoint{1.680760in}{0.635151in}}%
\pgfpathlineto{\pgfqpoint{1.683240in}{0.629187in}}%
\pgfpathlineto{\pgfqpoint{1.684480in}{0.630445in}}%
\pgfpathlineto{\pgfqpoint{1.688200in}{0.628182in}}%
\pgfpathlineto{\pgfqpoint{1.690680in}{0.631044in}}%
\pgfpathlineto{\pgfqpoint{1.696880in}{0.632489in}}%
\pgfpathlineto{\pgfqpoint{1.699360in}{0.634450in}}%
\pgfpathlineto{\pgfqpoint{1.700600in}{0.633150in}}%
\pgfpathlineto{\pgfqpoint{1.701840in}{0.634929in}}%
\pgfpathlineto{\pgfqpoint{1.704320in}{0.632812in}}%
\pgfpathlineto{\pgfqpoint{1.705560in}{0.632422in}}%
\pgfpathlineto{\pgfqpoint{1.708040in}{0.628362in}}%
\pgfpathlineto{\pgfqpoint{1.710520in}{0.630264in}}%
\pgfpathlineto{\pgfqpoint{1.713000in}{0.635732in}}%
\pgfpathlineto{\pgfqpoint{1.714240in}{0.634510in}}%
\pgfpathlineto{\pgfqpoint{1.716720in}{0.635686in}}%
\pgfpathlineto{\pgfqpoint{1.717960in}{0.633991in}}%
\pgfpathlineto{\pgfqpoint{1.719200in}{0.635407in}}%
\pgfpathlineto{\pgfqpoint{1.724160in}{0.633520in}}%
\pgfpathlineto{\pgfqpoint{1.725400in}{0.633148in}}%
\pgfpathlineto{\pgfqpoint{1.726640in}{0.634392in}}%
\pgfpathlineto{\pgfqpoint{1.729120in}{0.632298in}}%
\pgfpathlineto{\pgfqpoint{1.730360in}{0.633162in}}%
\pgfpathlineto{\pgfqpoint{1.732840in}{0.636149in}}%
\pgfpathlineto{\pgfqpoint{1.734080in}{0.637554in}}%
\pgfpathlineto{\pgfqpoint{1.737800in}{0.634959in}}%
\pgfpathlineto{\pgfqpoint{1.740280in}{0.636278in}}%
\pgfpathlineto{\pgfqpoint{1.744000in}{0.631747in}}%
\pgfpathlineto{\pgfqpoint{1.753920in}{0.639731in}}%
\pgfpathlineto{\pgfqpoint{1.756400in}{0.636470in}}%
\pgfpathlineto{\pgfqpoint{1.757640in}{0.636353in}}%
\pgfpathlineto{\pgfqpoint{1.758880in}{0.635090in}}%
\pgfpathlineto{\pgfqpoint{1.763840in}{0.635823in}}%
\pgfpathlineto{\pgfqpoint{1.766320in}{0.633665in}}%
\pgfpathlineto{\pgfqpoint{1.768800in}{0.635919in}}%
\pgfpathlineto{\pgfqpoint{1.776240in}{0.631897in}}%
\pgfpathlineto{\pgfqpoint{1.778720in}{0.634353in}}%
\pgfpathlineto{\pgfqpoint{1.787400in}{0.635399in}}%
\pgfpathlineto{\pgfqpoint{1.788640in}{0.637812in}}%
\pgfpathlineto{\pgfqpoint{1.797320in}{0.635369in}}%
\pgfpathlineto{\pgfqpoint{1.801040in}{0.636166in}}%
\pgfpathlineto{\pgfqpoint{1.803520in}{0.635267in}}%
\pgfpathlineto{\pgfqpoint{1.804760in}{0.636162in}}%
\pgfpathlineto{\pgfqpoint{1.807240in}{0.630263in}}%
\pgfpathlineto{\pgfqpoint{1.808480in}{0.631039in}}%
\pgfpathlineto{\pgfqpoint{1.812200in}{0.628478in}}%
\pgfpathlineto{\pgfqpoint{1.814680in}{0.631834in}}%
\pgfpathlineto{\pgfqpoint{1.819640in}{0.632709in}}%
\pgfpathlineto{\pgfqpoint{1.823360in}{0.636033in}}%
\pgfpathlineto{\pgfqpoint{1.824600in}{0.634371in}}%
\pgfpathlineto{\pgfqpoint{1.825840in}{0.635842in}}%
\pgfpathlineto{\pgfqpoint{1.828320in}{0.633582in}}%
\pgfpathlineto{\pgfqpoint{1.829560in}{0.632865in}}%
\pgfpathlineto{\pgfqpoint{1.832040in}{0.629255in}}%
\pgfpathlineto{\pgfqpoint{1.834520in}{0.632122in}}%
\pgfpathlineto{\pgfqpoint{1.837000in}{0.637504in}}%
\pgfpathlineto{\pgfqpoint{1.838240in}{0.636164in}}%
\pgfpathlineto{\pgfqpoint{1.840720in}{0.637525in}}%
\pgfpathlineto{\pgfqpoint{1.841960in}{0.636506in}}%
\pgfpathlineto{\pgfqpoint{1.843200in}{0.637488in}}%
\pgfpathlineto{\pgfqpoint{1.854360in}{0.632800in}}%
\pgfpathlineto{\pgfqpoint{1.856840in}{0.636063in}}%
\pgfpathlineto{\pgfqpoint{1.858080in}{0.637238in}}%
\pgfpathlineto{\pgfqpoint{1.861800in}{0.635148in}}%
\pgfpathlineto{\pgfqpoint{1.864280in}{0.636680in}}%
\pgfpathlineto{\pgfqpoint{1.868000in}{0.632756in}}%
\pgfpathlineto{\pgfqpoint{1.871720in}{0.638025in}}%
\pgfpathlineto{\pgfqpoint{1.872960in}{0.639059in}}%
\pgfpathlineto{\pgfqpoint{1.875440in}{0.638079in}}%
\pgfpathlineto{\pgfqpoint{1.877920in}{0.641601in}}%
\pgfpathlineto{\pgfqpoint{1.882880in}{0.638375in}}%
\pgfpathlineto{\pgfqpoint{1.887840in}{0.638082in}}%
\pgfpathlineto{\pgfqpoint{1.890320in}{0.635055in}}%
\pgfpathlineto{\pgfqpoint{1.891560in}{0.635394in}}%
\pgfpathlineto{\pgfqpoint{1.894040in}{0.637292in}}%
\pgfpathlineto{\pgfqpoint{1.900240in}{0.634528in}}%
\pgfpathlineto{\pgfqpoint{1.902720in}{0.637023in}}%
\pgfpathlineto{\pgfqpoint{1.906440in}{0.637888in}}%
\pgfpathlineto{\pgfqpoint{1.911400in}{0.637675in}}%
\pgfpathlineto{\pgfqpoint{1.912640in}{0.639934in}}%
\pgfpathlineto{\pgfqpoint{1.916360in}{0.638686in}}%
\pgfpathlineto{\pgfqpoint{1.917600in}{0.639908in}}%
\pgfpathlineto{\pgfqpoint{1.920080in}{0.638120in}}%
\pgfpathlineto{\pgfqpoint{1.922560in}{0.638112in}}%
\pgfpathlineto{\pgfqpoint{1.925040in}{0.639300in}}%
\pgfpathlineto{\pgfqpoint{1.926280in}{0.637524in}}%
\pgfpathlineto{\pgfqpoint{1.928760in}{0.639429in}}%
\pgfpathlineto{\pgfqpoint{1.930000in}{0.637633in}}%
\pgfpathlineto{\pgfqpoint{1.931240in}{0.633535in}}%
\pgfpathlineto{\pgfqpoint{1.933720in}{0.633760in}}%
\pgfpathlineto{\pgfqpoint{1.936200in}{0.630838in}}%
\pgfpathlineto{\pgfqpoint{1.938680in}{0.632877in}}%
\pgfpathlineto{\pgfqpoint{1.944880in}{0.635389in}}%
\pgfpathlineto{\pgfqpoint{1.947360in}{0.637656in}}%
\pgfpathlineto{\pgfqpoint{1.948600in}{0.636426in}}%
\pgfpathlineto{\pgfqpoint{1.949840in}{0.637554in}}%
\pgfpathlineto{\pgfqpoint{1.952320in}{0.635685in}}%
\pgfpathlineto{\pgfqpoint{1.953560in}{0.635098in}}%
\pgfpathlineto{\pgfqpoint{1.956040in}{0.631116in}}%
\pgfpathlineto{\pgfqpoint{1.961000in}{0.639179in}}%
\pgfpathlineto{\pgfqpoint{1.962240in}{0.637470in}}%
\pgfpathlineto{\pgfqpoint{1.967200in}{0.638575in}}%
\pgfpathlineto{\pgfqpoint{1.969680in}{0.638447in}}%
\pgfpathlineto{\pgfqpoint{1.972160in}{0.637196in}}%
\pgfpathlineto{\pgfqpoint{1.977120in}{0.634159in}}%
\pgfpathlineto{\pgfqpoint{1.978360in}{0.635008in}}%
\pgfpathlineto{\pgfqpoint{1.980840in}{0.638640in}}%
\pgfpathlineto{\pgfqpoint{1.982080in}{0.639375in}}%
\pgfpathlineto{\pgfqpoint{1.984560in}{0.637536in}}%
\pgfpathlineto{\pgfqpoint{1.985800in}{0.636992in}}%
\pgfpathlineto{\pgfqpoint{1.988280in}{0.639759in}}%
\pgfpathlineto{\pgfqpoint{1.992000in}{0.635717in}}%
\pgfpathlineto{\pgfqpoint{1.994480in}{0.638586in}}%
\pgfpathlineto{\pgfqpoint{1.998200in}{0.640383in}}%
\pgfpathlineto{\pgfqpoint{1.999440in}{0.640306in}}%
\pgfpathlineto{\pgfqpoint{2.001920in}{0.643900in}}%
\pgfpathlineto{\pgfqpoint{2.006880in}{0.640383in}}%
\pgfpathlineto{\pgfqpoint{2.010600in}{0.640218in}}%
\pgfpathlineto{\pgfqpoint{2.013080in}{0.638149in}}%
\pgfpathlineto{\pgfqpoint{2.014320in}{0.636783in}}%
\pgfpathlineto{\pgfqpoint{2.015560in}{0.637116in}}%
\pgfpathlineto{\pgfqpoint{2.018040in}{0.638804in}}%
\pgfpathlineto{\pgfqpoint{2.024240in}{0.636453in}}%
\pgfpathlineto{\pgfqpoint{2.026720in}{0.638128in}}%
\pgfpathlineto{\pgfqpoint{2.031680in}{0.638937in}}%
\pgfpathlineto{\pgfqpoint{2.035400in}{0.638582in}}%
\pgfpathlineto{\pgfqpoint{2.036640in}{0.640457in}}%
\pgfpathlineto{\pgfqpoint{2.040360in}{0.638665in}}%
\pgfpathlineto{\pgfqpoint{2.041600in}{0.639790in}}%
\pgfpathlineto{\pgfqpoint{2.044080in}{0.638350in}}%
\pgfpathlineto{\pgfqpoint{2.046560in}{0.638871in}}%
\pgfpathlineto{\pgfqpoint{2.049040in}{0.639610in}}%
\pgfpathlineto{\pgfqpoint{2.050280in}{0.637665in}}%
\pgfpathlineto{\pgfqpoint{2.052760in}{0.639405in}}%
\pgfpathlineto{\pgfqpoint{2.054000in}{0.637601in}}%
\pgfpathlineto{\pgfqpoint{2.055240in}{0.633571in}}%
\pgfpathlineto{\pgfqpoint{2.057720in}{0.633659in}}%
\pgfpathlineto{\pgfqpoint{2.060200in}{0.631164in}}%
\pgfpathlineto{\pgfqpoint{2.063920in}{0.633590in}}%
\pgfpathlineto{\pgfqpoint{2.065160in}{0.632801in}}%
\pgfpathlineto{\pgfqpoint{2.071360in}{0.636650in}}%
\pgfpathlineto{\pgfqpoint{2.072600in}{0.634712in}}%
\pgfpathlineto{\pgfqpoint{2.073840in}{0.635245in}}%
\pgfpathlineto{\pgfqpoint{2.076320in}{0.633369in}}%
\pgfpathlineto{\pgfqpoint{2.077560in}{0.633127in}}%
\pgfpathlineto{\pgfqpoint{2.080040in}{0.629381in}}%
\pgfpathlineto{\pgfqpoint{2.085000in}{0.637289in}}%
\pgfpathlineto{\pgfqpoint{2.086240in}{0.635446in}}%
\pgfpathlineto{\pgfqpoint{2.089960in}{0.635864in}}%
\pgfpathlineto{\pgfqpoint{2.093680in}{0.636008in}}%
\pgfpathlineto{\pgfqpoint{2.096160in}{0.634634in}}%
\pgfpathlineto{\pgfqpoint{2.101120in}{0.631869in}}%
\pgfpathlineto{\pgfqpoint{2.106080in}{0.636801in}}%
\pgfpathlineto{\pgfqpoint{2.108560in}{0.635452in}}%
\pgfpathlineto{\pgfqpoint{2.109800in}{0.635052in}}%
\pgfpathlineto{\pgfqpoint{2.112280in}{0.638151in}}%
\pgfpathlineto{\pgfqpoint{2.116000in}{0.634494in}}%
\pgfpathlineto{\pgfqpoint{2.120960in}{0.639361in}}%
\pgfpathlineto{\pgfqpoint{2.123440in}{0.638459in}}%
\pgfpathlineto{\pgfqpoint{2.125920in}{0.641787in}}%
\pgfpathlineto{\pgfqpoint{2.128400in}{0.639675in}}%
\pgfpathlineto{\pgfqpoint{2.129640in}{0.640029in}}%
\pgfpathlineto{\pgfqpoint{2.132120in}{0.639075in}}%
\pgfpathlineto{\pgfqpoint{2.134600in}{0.639634in}}%
\pgfpathlineto{\pgfqpoint{2.135840in}{0.639317in}}%
\pgfpathlineto{\pgfqpoint{2.138320in}{0.636417in}}%
\pgfpathlineto{\pgfqpoint{2.148240in}{0.635712in}}%
\pgfpathlineto{\pgfqpoint{2.150720in}{0.637275in}}%
\pgfpathlineto{\pgfqpoint{2.154440in}{0.639134in}}%
\pgfpathlineto{\pgfqpoint{2.156920in}{0.638347in}}%
\pgfpathlineto{\pgfqpoint{2.159400in}{0.638574in}}%
\pgfpathlineto{\pgfqpoint{2.160640in}{0.640714in}}%
\pgfpathlineto{\pgfqpoint{2.169320in}{0.640106in}}%
\pgfpathlineto{\pgfqpoint{2.171800in}{0.639953in}}%
\pgfpathlineto{\pgfqpoint{2.173040in}{0.639713in}}%
\pgfpathlineto{\pgfqpoint{2.174280in}{0.637687in}}%
\pgfpathlineto{\pgfqpoint{2.176760in}{0.639868in}}%
\pgfpathlineto{\pgfqpoint{2.178000in}{0.638526in}}%
\pgfpathlineto{\pgfqpoint{2.179240in}{0.634558in}}%
\pgfpathlineto{\pgfqpoint{2.181720in}{0.634538in}}%
\pgfpathlineto{\pgfqpoint{2.184200in}{0.632769in}}%
\pgfpathlineto{\pgfqpoint{2.187920in}{0.634027in}}%
\pgfpathlineto{\pgfqpoint{2.190400in}{0.633662in}}%
\pgfpathlineto{\pgfqpoint{2.194120in}{0.635654in}}%
\pgfpathlineto{\pgfqpoint{2.195360in}{0.636293in}}%
\pgfpathlineto{\pgfqpoint{2.196600in}{0.634460in}}%
\pgfpathlineto{\pgfqpoint{2.197840in}{0.634756in}}%
\pgfpathlineto{\pgfqpoint{2.200320in}{0.633032in}}%
\pgfpathlineto{\pgfqpoint{2.201560in}{0.632729in}}%
\pgfpathlineto{\pgfqpoint{2.204040in}{0.630139in}}%
\pgfpathlineto{\pgfqpoint{2.209000in}{0.637679in}}%
\pgfpathlineto{\pgfqpoint{2.210240in}{0.636322in}}%
\pgfpathlineto{\pgfqpoint{2.213960in}{0.636970in}}%
\pgfpathlineto{\pgfqpoint{2.220160in}{0.635944in}}%
\pgfpathlineto{\pgfqpoint{2.225120in}{0.632876in}}%
\pgfpathlineto{\pgfqpoint{2.230080in}{0.636523in}}%
\pgfpathlineto{\pgfqpoint{2.232560in}{0.634778in}}%
\pgfpathlineto{\pgfqpoint{2.233800in}{0.634805in}}%
\pgfpathlineto{\pgfqpoint{2.236280in}{0.637406in}}%
\pgfpathlineto{\pgfqpoint{2.240000in}{0.634304in}}%
\pgfpathlineto{\pgfqpoint{2.243720in}{0.638819in}}%
\pgfpathlineto{\pgfqpoint{2.244960in}{0.640143in}}%
\pgfpathlineto{\pgfqpoint{2.247440in}{0.639051in}}%
\pgfpathlineto{\pgfqpoint{2.249920in}{0.642171in}}%
\pgfpathlineto{\pgfqpoint{2.254880in}{0.640442in}}%
\pgfpathlineto{\pgfqpoint{2.259840in}{0.640452in}}%
\pgfpathlineto{\pgfqpoint{2.262320in}{0.638088in}}%
\pgfpathlineto{\pgfqpoint{2.268520in}{0.637887in}}%
\pgfpathlineto{\pgfqpoint{2.271000in}{0.636273in}}%
\pgfpathlineto{\pgfqpoint{2.282160in}{0.640230in}}%
\pgfpathlineto{\pgfqpoint{2.283400in}{0.639849in}}%
\pgfpathlineto{\pgfqpoint{2.284640in}{0.642062in}}%
\pgfpathlineto{\pgfqpoint{2.292080in}{0.640699in}}%
\pgfpathlineto{\pgfqpoint{2.297040in}{0.641848in}}%
\pgfpathlineto{\pgfqpoint{2.298280in}{0.639786in}}%
\pgfpathlineto{\pgfqpoint{2.300760in}{0.641675in}}%
\pgfpathlineto{\pgfqpoint{2.302000in}{0.640262in}}%
\pgfpathlineto{\pgfqpoint{2.303240in}{0.636808in}}%
\pgfpathlineto{\pgfqpoint{2.304480in}{0.637581in}}%
\pgfpathlineto{\pgfqpoint{2.308200in}{0.634669in}}%
\pgfpathlineto{\pgfqpoint{2.311920in}{0.636381in}}%
\pgfpathlineto{\pgfqpoint{2.315640in}{0.635669in}}%
\pgfpathlineto{\pgfqpoint{2.319360in}{0.638567in}}%
\pgfpathlineto{\pgfqpoint{2.320600in}{0.636427in}}%
\pgfpathlineto{\pgfqpoint{2.321840in}{0.636528in}}%
\pgfpathlineto{\pgfqpoint{2.324320in}{0.635185in}}%
\pgfpathlineto{\pgfqpoint{2.328040in}{0.632518in}}%
\pgfpathlineto{\pgfqpoint{2.333000in}{0.638653in}}%
\pgfpathlineto{\pgfqpoint{2.334240in}{0.637556in}}%
\pgfpathlineto{\pgfqpoint{2.337960in}{0.638823in}}%
\pgfpathlineto{\pgfqpoint{2.346640in}{0.636799in}}%
\pgfpathlineto{\pgfqpoint{2.349120in}{0.634413in}}%
\pgfpathlineto{\pgfqpoint{2.354080in}{0.638253in}}%
\pgfpathlineto{\pgfqpoint{2.356560in}{0.636882in}}%
\pgfpathlineto{\pgfqpoint{2.357800in}{0.636781in}}%
\pgfpathlineto{\pgfqpoint{2.360280in}{0.639434in}}%
\pgfpathlineto{\pgfqpoint{2.365240in}{0.637978in}}%
\pgfpathlineto{\pgfqpoint{2.370200in}{0.640530in}}%
\pgfpathlineto{\pgfqpoint{2.371440in}{0.639986in}}%
\pgfpathlineto{\pgfqpoint{2.375160in}{0.642764in}}%
\pgfpathlineto{\pgfqpoint{2.377640in}{0.643292in}}%
\pgfpathlineto{\pgfqpoint{2.383840in}{0.642331in}}%
\pgfpathlineto{\pgfqpoint{2.386320in}{0.640064in}}%
\pgfpathlineto{\pgfqpoint{2.390040in}{0.640352in}}%
\pgfpathlineto{\pgfqpoint{2.393760in}{0.638442in}}%
\pgfpathlineto{\pgfqpoint{2.396240in}{0.638059in}}%
\pgfpathlineto{\pgfqpoint{2.398720in}{0.640260in}}%
\pgfpathlineto{\pgfqpoint{2.404920in}{0.642073in}}%
\pgfpathlineto{\pgfqpoint{2.407400in}{0.641403in}}%
\pgfpathlineto{\pgfqpoint{2.408640in}{0.643584in}}%
\pgfpathlineto{\pgfqpoint{2.412360in}{0.641681in}}%
\pgfpathlineto{\pgfqpoint{2.413600in}{0.642590in}}%
\pgfpathlineto{\pgfqpoint{2.414840in}{0.641670in}}%
\pgfpathlineto{\pgfqpoint{2.419800in}{0.642662in}}%
\pgfpathlineto{\pgfqpoint{2.421040in}{0.642768in}}%
\pgfpathlineto{\pgfqpoint{2.422280in}{0.640624in}}%
\pgfpathlineto{\pgfqpoint{2.424760in}{0.642672in}}%
\pgfpathlineto{\pgfqpoint{2.426000in}{0.641732in}}%
\pgfpathlineto{\pgfqpoint{2.427240in}{0.638568in}}%
\pgfpathlineto{\pgfqpoint{2.428480in}{0.639587in}}%
\pgfpathlineto{\pgfqpoint{2.433440in}{0.636755in}}%
\pgfpathlineto{\pgfqpoint{2.437160in}{0.637158in}}%
\pgfpathlineto{\pgfqpoint{2.439640in}{0.637056in}}%
\pgfpathlineto{\pgfqpoint{2.443360in}{0.638847in}}%
\pgfpathlineto{\pgfqpoint{2.444600in}{0.637210in}}%
\pgfpathlineto{\pgfqpoint{2.445840in}{0.637658in}}%
\pgfpathlineto{\pgfqpoint{2.448320in}{0.636446in}}%
\pgfpathlineto{\pgfqpoint{2.452040in}{0.634242in}}%
\pgfpathlineto{\pgfqpoint{2.457000in}{0.639488in}}%
\pgfpathlineto{\pgfqpoint{2.458240in}{0.637857in}}%
\pgfpathlineto{\pgfqpoint{2.463200in}{0.638454in}}%
\pgfpathlineto{\pgfqpoint{2.465680in}{0.638519in}}%
\pgfpathlineto{\pgfqpoint{2.466920in}{0.637030in}}%
\pgfpathlineto{\pgfqpoint{2.469400in}{0.637080in}}%
\pgfpathlineto{\pgfqpoint{2.470640in}{0.637196in}}%
\pgfpathlineto{\pgfqpoint{2.473120in}{0.635185in}}%
\pgfpathlineto{\pgfqpoint{2.475600in}{0.637965in}}%
\pgfpathlineto{\pgfqpoint{2.476840in}{0.637105in}}%
\pgfpathlineto{\pgfqpoint{2.478080in}{0.638004in}}%
\pgfpathlineto{\pgfqpoint{2.480560in}{0.636201in}}%
\pgfpathlineto{\pgfqpoint{2.483040in}{0.636980in}}%
\pgfpathlineto{\pgfqpoint{2.484280in}{0.638534in}}%
\pgfpathlineto{\pgfqpoint{2.486760in}{0.637585in}}%
\pgfpathlineto{\pgfqpoint{2.489240in}{0.635215in}}%
\pgfpathlineto{\pgfqpoint{2.499160in}{0.639598in}}%
\pgfpathlineto{\pgfqpoint{2.500400in}{0.639076in}}%
\pgfpathlineto{\pgfqpoint{2.504120in}{0.640336in}}%
\pgfpathlineto{\pgfqpoint{2.506600in}{0.640507in}}%
\pgfpathlineto{\pgfqpoint{2.511560in}{0.637580in}}%
\pgfpathlineto{\pgfqpoint{2.515280in}{0.637527in}}%
\pgfpathlineto{\pgfqpoint{2.520240in}{0.636040in}}%
\pgfpathlineto{\pgfqpoint{2.522720in}{0.637448in}}%
\pgfpathlineto{\pgfqpoint{2.528920in}{0.639655in}}%
\pgfpathlineto{\pgfqpoint{2.531400in}{0.638800in}}%
\pgfpathlineto{\pgfqpoint{2.532640in}{0.641201in}}%
\pgfpathlineto{\pgfqpoint{2.536360in}{0.640128in}}%
\pgfpathlineto{\pgfqpoint{2.538840in}{0.640253in}}%
\pgfpathlineto{\pgfqpoint{2.542560in}{0.642154in}}%
\pgfpathlineto{\pgfqpoint{2.546280in}{0.639031in}}%
\pgfpathlineto{\pgfqpoint{2.548760in}{0.641043in}}%
\pgfpathlineto{\pgfqpoint{2.550000in}{0.640296in}}%
\pgfpathlineto{\pgfqpoint{2.551240in}{0.637370in}}%
\pgfpathlineto{\pgfqpoint{2.552480in}{0.638772in}}%
\pgfpathlineto{\pgfqpoint{2.554960in}{0.636161in}}%
\pgfpathlineto{\pgfqpoint{2.556200in}{0.634644in}}%
\pgfpathlineto{\pgfqpoint{2.561160in}{0.635832in}}%
\pgfpathlineto{\pgfqpoint{2.569840in}{0.635644in}}%
\pgfpathlineto{\pgfqpoint{2.574800in}{0.633307in}}%
\pgfpathlineto{\pgfqpoint{2.576040in}{0.632611in}}%
\pgfpathlineto{\pgfqpoint{2.581000in}{0.638296in}}%
\pgfpathlineto{\pgfqpoint{2.583480in}{0.636554in}}%
\pgfpathlineto{\pgfqpoint{2.585960in}{0.637005in}}%
\pgfpathlineto{\pgfqpoint{2.589680in}{0.636666in}}%
\pgfpathlineto{\pgfqpoint{2.590920in}{0.635310in}}%
\pgfpathlineto{\pgfqpoint{2.592160in}{0.636114in}}%
\pgfpathlineto{\pgfqpoint{2.597120in}{0.633631in}}%
\pgfpathlineto{\pgfqpoint{2.599600in}{0.636455in}}%
\pgfpathlineto{\pgfqpoint{2.600840in}{0.635318in}}%
\pgfpathlineto{\pgfqpoint{2.602080in}{0.636004in}}%
\pgfpathlineto{\pgfqpoint{2.604560in}{0.634400in}}%
\pgfpathlineto{\pgfqpoint{2.605800in}{0.634108in}}%
\pgfpathlineto{\pgfqpoint{2.609520in}{0.636703in}}%
\pgfpathlineto{\pgfqpoint{2.614480in}{0.634835in}}%
\pgfpathlineto{\pgfqpoint{2.616960in}{0.637150in}}%
\pgfpathlineto{\pgfqpoint{2.619440in}{0.636225in}}%
\pgfpathlineto{\pgfqpoint{2.623160in}{0.638334in}}%
\pgfpathlineto{\pgfqpoint{2.624400in}{0.637829in}}%
\pgfpathlineto{\pgfqpoint{2.626880in}{0.639074in}}%
\pgfpathlineto{\pgfqpoint{2.631840in}{0.638566in}}%
\pgfpathlineto{\pgfqpoint{2.634320in}{0.635846in}}%
\pgfpathlineto{\pgfqpoint{2.641760in}{0.636023in}}%
\pgfpathlineto{\pgfqpoint{2.644240in}{0.635216in}}%
\pgfpathlineto{\pgfqpoint{2.646720in}{0.636458in}}%
\pgfpathlineto{\pgfqpoint{2.649200in}{0.637727in}}%
\pgfpathlineto{\pgfqpoint{2.650440in}{0.639155in}}%
\pgfpathlineto{\pgfqpoint{2.651680in}{0.638642in}}%
\pgfpathlineto{\pgfqpoint{2.654160in}{0.639033in}}%
\pgfpathlineto{\pgfqpoint{2.655400in}{0.638809in}}%
\pgfpathlineto{\pgfqpoint{2.656640in}{0.640758in}}%
\pgfpathlineto{\pgfqpoint{2.660360in}{0.639069in}}%
\pgfpathlineto{\pgfqpoint{2.666560in}{0.641574in}}%
\pgfpathlineto{\pgfqpoint{2.671520in}{0.639201in}}%
\pgfpathlineto{\pgfqpoint{2.672760in}{0.640094in}}%
\pgfpathlineto{\pgfqpoint{2.674000in}{0.639318in}}%
\pgfpathlineto{\pgfqpoint{2.675240in}{0.636740in}}%
\pgfpathlineto{\pgfqpoint{2.676480in}{0.638008in}}%
\pgfpathlineto{\pgfqpoint{2.681440in}{0.633633in}}%
\pgfpathlineto{\pgfqpoint{2.686400in}{0.634468in}}%
\pgfpathlineto{\pgfqpoint{2.693840in}{0.632968in}}%
\pgfpathlineto{\pgfqpoint{2.696320in}{0.631925in}}%
\pgfpathlineto{\pgfqpoint{2.700040in}{0.630272in}}%
\pgfpathlineto{\pgfqpoint{2.705000in}{0.635626in}}%
\pgfpathlineto{\pgfqpoint{2.706240in}{0.634213in}}%
\pgfpathlineto{\pgfqpoint{2.711200in}{0.635133in}}%
\pgfpathlineto{\pgfqpoint{2.712440in}{0.634220in}}%
\pgfpathlineto{\pgfqpoint{2.713680in}{0.634757in}}%
\pgfpathlineto{\pgfqpoint{2.714920in}{0.633569in}}%
\pgfpathlineto{\pgfqpoint{2.716160in}{0.634346in}}%
\pgfpathlineto{\pgfqpoint{2.721120in}{0.631427in}}%
\pgfpathlineto{\pgfqpoint{2.724840in}{0.633183in}}%
\pgfpathlineto{\pgfqpoint{2.726080in}{0.633740in}}%
\pgfpathlineto{\pgfqpoint{2.728560in}{0.632299in}}%
\pgfpathlineto{\pgfqpoint{2.729800in}{0.631970in}}%
\pgfpathlineto{\pgfqpoint{2.733520in}{0.634495in}}%
\pgfpathlineto{\pgfqpoint{2.736000in}{0.632874in}}%
\pgfpathlineto{\pgfqpoint{2.737240in}{0.631732in}}%
\pgfpathlineto{\pgfqpoint{2.739720in}{0.633437in}}%
\pgfpathlineto{\pgfqpoint{2.742200in}{0.634672in}}%
\pgfpathlineto{\pgfqpoint{2.744680in}{0.635874in}}%
\pgfpathlineto{\pgfqpoint{2.745920in}{0.637667in}}%
\pgfpathlineto{\pgfqpoint{2.749640in}{0.637401in}}%
\pgfpathlineto{\pgfqpoint{2.754600in}{0.637673in}}%
\pgfpathlineto{\pgfqpoint{2.755840in}{0.637119in}}%
\pgfpathlineto{\pgfqpoint{2.758320in}{0.634044in}}%
\pgfpathlineto{\pgfqpoint{2.765760in}{0.634086in}}%
\pgfpathlineto{\pgfqpoint{2.767000in}{0.632991in}}%
\pgfpathlineto{\pgfqpoint{2.773200in}{0.635881in}}%
\pgfpathlineto{\pgfqpoint{2.776920in}{0.637244in}}%
\pgfpathlineto{\pgfqpoint{2.779400in}{0.636237in}}%
\pgfpathlineto{\pgfqpoint{2.780640in}{0.638139in}}%
\pgfpathlineto{\pgfqpoint{2.783120in}{0.636382in}}%
\pgfpathlineto{\pgfqpoint{2.788080in}{0.637271in}}%
\pgfpathlineto{\pgfqpoint{2.791800in}{0.637471in}}%
\pgfpathlineto{\pgfqpoint{2.793040in}{0.637731in}}%
\pgfpathlineto{\pgfqpoint{2.794280in}{0.635640in}}%
\pgfpathlineto{\pgfqpoint{2.798000in}{0.636709in}}%
\pgfpathlineto{\pgfqpoint{2.799240in}{0.634345in}}%
\pgfpathlineto{\pgfqpoint{2.800480in}{0.635757in}}%
\pgfpathlineto{\pgfqpoint{2.805440in}{0.630577in}}%
\pgfpathlineto{\pgfqpoint{2.807920in}{0.630811in}}%
\pgfpathlineto{\pgfqpoint{2.811640in}{0.631079in}}%
\pgfpathlineto{\pgfqpoint{2.814120in}{0.632212in}}%
\pgfpathlineto{\pgfqpoint{2.815360in}{0.632479in}}%
\pgfpathlineto{\pgfqpoint{2.819080in}{0.629654in}}%
\pgfpathlineto{\pgfqpoint{2.822800in}{0.629845in}}%
\pgfpathlineto{\pgfqpoint{2.824040in}{0.629239in}}%
\pgfpathlineto{\pgfqpoint{2.829000in}{0.634408in}}%
\pgfpathlineto{\pgfqpoint{2.830240in}{0.633218in}}%
\pgfpathlineto{\pgfqpoint{2.835200in}{0.635150in}}%
\pgfpathlineto{\pgfqpoint{2.836440in}{0.634061in}}%
\pgfpathlineto{\pgfqpoint{2.837680in}{0.634576in}}%
\pgfpathlineto{\pgfqpoint{2.838920in}{0.633378in}}%
\pgfpathlineto{\pgfqpoint{2.840160in}{0.634598in}}%
\pgfpathlineto{\pgfqpoint{2.841400in}{0.633799in}}%
\pgfpathlineto{\pgfqpoint{2.842640in}{0.634376in}}%
\pgfpathlineto{\pgfqpoint{2.845120in}{0.631645in}}%
\pgfpathlineto{\pgfqpoint{2.847600in}{0.634127in}}%
\pgfpathlineto{\pgfqpoint{2.848840in}{0.633192in}}%
\pgfpathlineto{\pgfqpoint{2.850080in}{0.633795in}}%
\pgfpathlineto{\pgfqpoint{2.852560in}{0.632041in}}%
\pgfpathlineto{\pgfqpoint{2.855040in}{0.632600in}}%
\pgfpathlineto{\pgfqpoint{2.857520in}{0.634285in}}%
\pgfpathlineto{\pgfqpoint{2.860000in}{0.632931in}}%
\pgfpathlineto{\pgfqpoint{2.861240in}{0.631409in}}%
\pgfpathlineto{\pgfqpoint{2.866200in}{0.634465in}}%
\pgfpathlineto{\pgfqpoint{2.868680in}{0.635615in}}%
\pgfpathlineto{\pgfqpoint{2.869920in}{0.637214in}}%
\pgfpathlineto{\pgfqpoint{2.872400in}{0.636126in}}%
\pgfpathlineto{\pgfqpoint{2.874880in}{0.637229in}}%
\pgfpathlineto{\pgfqpoint{2.879840in}{0.637436in}}%
\pgfpathlineto{\pgfqpoint{2.882320in}{0.634413in}}%
\pgfpathlineto{\pgfqpoint{2.888520in}{0.634848in}}%
\pgfpathlineto{\pgfqpoint{2.892240in}{0.633670in}}%
\pgfpathlineto{\pgfqpoint{2.894720in}{0.634914in}}%
\pgfpathlineto{\pgfqpoint{2.897200in}{0.635565in}}%
\pgfpathlineto{\pgfqpoint{2.898440in}{0.636679in}}%
\pgfpathlineto{\pgfqpoint{2.899680in}{0.636173in}}%
\pgfpathlineto{\pgfqpoint{2.902160in}{0.636612in}}%
\pgfpathlineto{\pgfqpoint{2.903400in}{0.636297in}}%
\pgfpathlineto{\pgfqpoint{2.904640in}{0.638183in}}%
\pgfpathlineto{\pgfqpoint{2.907120in}{0.635758in}}%
\pgfpathlineto{\pgfqpoint{2.913320in}{0.638065in}}%
\pgfpathlineto{\pgfqpoint{2.915800in}{0.638451in}}%
\pgfpathlineto{\pgfqpoint{2.917040in}{0.638861in}}%
\pgfpathlineto{\pgfqpoint{2.918280in}{0.636757in}}%
\pgfpathlineto{\pgfqpoint{2.922000in}{0.637473in}}%
\pgfpathlineto{\pgfqpoint{2.923240in}{0.635356in}}%
\pgfpathlineto{\pgfqpoint{2.924480in}{0.636909in}}%
\pgfpathlineto{\pgfqpoint{2.929440in}{0.632259in}}%
\pgfpathlineto{\pgfqpoint{2.931920in}{0.632283in}}%
\pgfpathlineto{\pgfqpoint{2.935640in}{0.631880in}}%
\pgfpathlineto{\pgfqpoint{2.939360in}{0.633123in}}%
\pgfpathlineto{\pgfqpoint{2.943080in}{0.630483in}}%
\pgfpathlineto{\pgfqpoint{2.949280in}{0.632025in}}%
\pgfpathlineto{\pgfqpoint{2.953000in}{0.635241in}}%
\pgfpathlineto{\pgfqpoint{2.954240in}{0.633830in}}%
\pgfpathlineto{\pgfqpoint{2.959200in}{0.636183in}}%
\pgfpathlineto{\pgfqpoint{2.960440in}{0.635084in}}%
\pgfpathlineto{\pgfqpoint{2.961680in}{0.635598in}}%
\pgfpathlineto{\pgfqpoint{2.962920in}{0.634612in}}%
\pgfpathlineto{\pgfqpoint{2.964160in}{0.635833in}}%
\pgfpathlineto{\pgfqpoint{2.969120in}{0.632730in}}%
\pgfpathlineto{\pgfqpoint{2.971600in}{0.635120in}}%
\pgfpathlineto{\pgfqpoint{2.972840in}{0.634198in}}%
\pgfpathlineto{\pgfqpoint{2.974080in}{0.634892in}}%
\pgfpathlineto{\pgfqpoint{2.977800in}{0.632809in}}%
\pgfpathlineto{\pgfqpoint{2.984000in}{0.634168in}}%
\pgfpathlineto{\pgfqpoint{2.985240in}{0.632772in}}%
\pgfpathlineto{\pgfqpoint{2.992680in}{0.636860in}}%
\pgfpathlineto{\pgfqpoint{2.993920in}{0.638532in}}%
\pgfpathlineto{\pgfqpoint{2.996400in}{0.637285in}}%
\pgfpathlineto{\pgfqpoint{2.998880in}{0.638994in}}%
\pgfpathlineto{\pgfqpoint{3.003840in}{0.638997in}}%
\pgfpathlineto{\pgfqpoint{3.006320in}{0.636268in}}%
\pgfpathlineto{\pgfqpoint{3.008800in}{0.637413in}}%
\pgfpathlineto{\pgfqpoint{3.013760in}{0.634732in}}%
\pgfpathlineto{\pgfqpoint{3.015000in}{0.634040in}}%
\pgfpathlineto{\pgfqpoint{3.024920in}{0.637760in}}%
\pgfpathlineto{\pgfqpoint{3.027400in}{0.636964in}}%
\pgfpathlineto{\pgfqpoint{3.028640in}{0.638450in}}%
\pgfpathlineto{\pgfqpoint{3.031120in}{0.636337in}}%
\pgfpathlineto{\pgfqpoint{3.036080in}{0.638697in}}%
\pgfpathlineto{\pgfqpoint{3.038560in}{0.639859in}}%
\pgfpathlineto{\pgfqpoint{3.043520in}{0.637501in}}%
\pgfpathlineto{\pgfqpoint{3.046000in}{0.638139in}}%
\pgfpathlineto{\pgfqpoint{3.047240in}{0.636275in}}%
\pgfpathlineto{\pgfqpoint{3.048480in}{0.637910in}}%
\pgfpathlineto{\pgfqpoint{3.053440in}{0.633125in}}%
\pgfpathlineto{\pgfqpoint{3.055920in}{0.633090in}}%
\pgfpathlineto{\pgfqpoint{3.059640in}{0.632547in}}%
\pgfpathlineto{\pgfqpoint{3.063360in}{0.633820in}}%
\pgfpathlineto{\pgfqpoint{3.067080in}{0.631176in}}%
\pgfpathlineto{\pgfqpoint{3.070800in}{0.631577in}}%
\pgfpathlineto{\pgfqpoint{3.072040in}{0.630782in}}%
\pgfpathlineto{\pgfqpoint{3.077000in}{0.635343in}}%
\pgfpathlineto{\pgfqpoint{3.078240in}{0.633904in}}%
\pgfpathlineto{\pgfqpoint{3.085680in}{0.634532in}}%
\pgfpathlineto{\pgfqpoint{3.086920in}{0.633672in}}%
\pgfpathlineto{\pgfqpoint{3.088160in}{0.634865in}}%
\pgfpathlineto{\pgfqpoint{3.089400in}{0.633828in}}%
\pgfpathlineto{\pgfqpoint{3.090640in}{0.634250in}}%
\pgfpathlineto{\pgfqpoint{3.093120in}{0.631585in}}%
\pgfpathlineto{\pgfqpoint{3.098080in}{0.634811in}}%
\pgfpathlineto{\pgfqpoint{3.101800in}{0.633077in}}%
\pgfpathlineto{\pgfqpoint{3.106760in}{0.634871in}}%
\pgfpathlineto{\pgfqpoint{3.110480in}{0.634210in}}%
\pgfpathlineto{\pgfqpoint{3.117920in}{0.639290in}}%
\pgfpathlineto{\pgfqpoint{3.120400in}{0.638314in}}%
\pgfpathlineto{\pgfqpoint{3.122880in}{0.639686in}}%
\pgfpathlineto{\pgfqpoint{3.127840in}{0.639845in}}%
\pgfpathlineto{\pgfqpoint{3.130320in}{0.637008in}}%
\pgfpathlineto{\pgfqpoint{3.132800in}{0.638351in}}%
\pgfpathlineto{\pgfqpoint{3.137760in}{0.634806in}}%
\pgfpathlineto{\pgfqpoint{3.139000in}{0.633992in}}%
\pgfpathlineto{\pgfqpoint{3.143960in}{0.636316in}}%
\pgfpathlineto{\pgfqpoint{3.152640in}{0.638883in}}%
\pgfpathlineto{\pgfqpoint{3.155120in}{0.636635in}}%
\pgfpathlineto{\pgfqpoint{3.157600in}{0.637986in}}%
\pgfpathlineto{\pgfqpoint{3.160080in}{0.639150in}}%
\pgfpathlineto{\pgfqpoint{3.162560in}{0.640031in}}%
\pgfpathlineto{\pgfqpoint{3.167520in}{0.637641in}}%
\pgfpathlineto{\pgfqpoint{3.170000in}{0.638385in}}%
\pgfpathlineto{\pgfqpoint{3.171240in}{0.636657in}}%
\pgfpathlineto{\pgfqpoint{3.172480in}{0.638429in}}%
\pgfpathlineto{\pgfqpoint{3.177440in}{0.634146in}}%
\pgfpathlineto{\pgfqpoint{3.179920in}{0.633646in}}%
\pgfpathlineto{\pgfqpoint{3.183640in}{0.632520in}}%
\pgfpathlineto{\pgfqpoint{3.187360in}{0.633469in}}%
\pgfpathlineto{\pgfqpoint{3.191080in}{0.631414in}}%
\pgfpathlineto{\pgfqpoint{3.193560in}{0.631706in}}%
\pgfpathlineto{\pgfqpoint{3.194800in}{0.631758in}}%
\pgfpathlineto{\pgfqpoint{3.196040in}{0.630627in}}%
\pgfpathlineto{\pgfqpoint{3.201000in}{0.636240in}}%
\pgfpathlineto{\pgfqpoint{3.202240in}{0.634917in}}%
\pgfpathlineto{\pgfqpoint{3.207200in}{0.636898in}}%
\pgfpathlineto{\pgfqpoint{3.209680in}{0.635639in}}%
\pgfpathlineto{\pgfqpoint{3.210920in}{0.634548in}}%
\pgfpathlineto{\pgfqpoint{3.212160in}{0.635685in}}%
\pgfpathlineto{\pgfqpoint{3.213400in}{0.634735in}}%
\pgfpathlineto{\pgfqpoint{3.214640in}{0.635394in}}%
\pgfpathlineto{\pgfqpoint{3.217120in}{0.632633in}}%
\pgfpathlineto{\pgfqpoint{3.222080in}{0.635509in}}%
\pgfpathlineto{\pgfqpoint{3.224560in}{0.634399in}}%
\pgfpathlineto{\pgfqpoint{3.227040in}{0.634469in}}%
\pgfpathlineto{\pgfqpoint{3.229520in}{0.636339in}}%
\pgfpathlineto{\pgfqpoint{3.233240in}{0.634537in}}%
\pgfpathlineto{\pgfqpoint{3.235720in}{0.636338in}}%
\pgfpathlineto{\pgfqpoint{3.238200in}{0.638273in}}%
\pgfpathlineto{\pgfqpoint{3.248120in}{0.640862in}}%
\pgfpathlineto{\pgfqpoint{3.251840in}{0.640233in}}%
\pgfpathlineto{\pgfqpoint{3.254320in}{0.637581in}}%
\pgfpathlineto{\pgfqpoint{3.256800in}{0.639071in}}%
\pgfpathlineto{\pgfqpoint{3.263000in}{0.634932in}}%
\pgfpathlineto{\pgfqpoint{3.270440in}{0.639011in}}%
\pgfpathlineto{\pgfqpoint{3.272920in}{0.639522in}}%
\pgfpathlineto{\pgfqpoint{3.277880in}{0.639174in}}%
\pgfpathlineto{\pgfqpoint{3.279120in}{0.637916in}}%
\pgfpathlineto{\pgfqpoint{3.282840in}{0.640010in}}%
\pgfpathlineto{\pgfqpoint{3.286560in}{0.641675in}}%
\pgfpathlineto{\pgfqpoint{3.291520in}{0.638737in}}%
\pgfpathlineto{\pgfqpoint{3.294000in}{0.639502in}}%
\pgfpathlineto{\pgfqpoint{3.295240in}{0.637608in}}%
\pgfpathlineto{\pgfqpoint{3.296480in}{0.639301in}}%
\pgfpathlineto{\pgfqpoint{3.301440in}{0.635682in}}%
\pgfpathlineto{\pgfqpoint{3.303920in}{0.635141in}}%
\pgfpathlineto{\pgfqpoint{3.307640in}{0.633984in}}%
\pgfpathlineto{\pgfqpoint{3.311360in}{0.635035in}}%
\pgfpathlineto{\pgfqpoint{3.313840in}{0.633112in}}%
\pgfpathlineto{\pgfqpoint{3.315080in}{0.632316in}}%
\pgfpathlineto{\pgfqpoint{3.317560in}{0.632841in}}%
\pgfpathlineto{\pgfqpoint{3.318800in}{0.632712in}}%
\pgfpathlineto{\pgfqpoint{3.320040in}{0.631275in}}%
\pgfpathlineto{\pgfqpoint{3.325000in}{0.636582in}}%
\pgfpathlineto{\pgfqpoint{3.326240in}{0.635296in}}%
\pgfpathlineto{\pgfqpoint{3.331200in}{0.637458in}}%
\pgfpathlineto{\pgfqpoint{3.334920in}{0.635287in}}%
\pgfpathlineto{\pgfqpoint{3.336160in}{0.636308in}}%
\pgfpathlineto{\pgfqpoint{3.341120in}{0.633279in}}%
\pgfpathlineto{\pgfqpoint{3.344840in}{0.635544in}}%
\pgfpathlineto{\pgfqpoint{3.346080in}{0.636336in}}%
\pgfpathlineto{\pgfqpoint{3.349800in}{0.634455in}}%
\pgfpathlineto{\pgfqpoint{3.351040in}{0.634841in}}%
\pgfpathlineto{\pgfqpoint{3.352280in}{0.636861in}}%
\pgfpathlineto{\pgfqpoint{3.357240in}{0.636274in}}%
\pgfpathlineto{\pgfqpoint{3.359720in}{0.638572in}}%
\pgfpathlineto{\pgfqpoint{3.364680in}{0.641980in}}%
\pgfpathlineto{\pgfqpoint{3.365920in}{0.642952in}}%
\pgfpathlineto{\pgfqpoint{3.368400in}{0.642748in}}%
\pgfpathlineto{\pgfqpoint{3.370880in}{0.643429in}}%
\pgfpathlineto{\pgfqpoint{3.375840in}{0.642422in}}%
\pgfpathlineto{\pgfqpoint{3.378320in}{0.639889in}}%
\pgfpathlineto{\pgfqpoint{3.380800in}{0.641537in}}%
\pgfpathlineto{\pgfqpoint{3.387000in}{0.638113in}}%
\pgfpathlineto{\pgfqpoint{3.391960in}{0.640397in}}%
\pgfpathlineto{\pgfqpoint{3.395680in}{0.641573in}}%
\pgfpathlineto{\pgfqpoint{3.398160in}{0.642354in}}%
\pgfpathlineto{\pgfqpoint{3.399400in}{0.642101in}}%
\pgfpathlineto{\pgfqpoint{3.400640in}{0.643381in}}%
\pgfpathlineto{\pgfqpoint{3.404360in}{0.640852in}}%
\pgfpathlineto{\pgfqpoint{3.408080in}{0.643355in}}%
\pgfpathlineto{\pgfqpoint{3.410560in}{0.644121in}}%
\pgfpathlineto{\pgfqpoint{3.415520in}{0.641196in}}%
\pgfpathlineto{\pgfqpoint{3.418000in}{0.641939in}}%
\pgfpathlineto{\pgfqpoint{3.419240in}{0.640105in}}%
\pgfpathlineto{\pgfqpoint{3.420480in}{0.642054in}}%
\pgfpathlineto{\pgfqpoint{3.425440in}{0.637497in}}%
\pgfpathlineto{\pgfqpoint{3.430400in}{0.636212in}}%
\pgfpathlineto{\pgfqpoint{3.437840in}{0.634858in}}%
\pgfpathlineto{\pgfqpoint{3.439080in}{0.633991in}}%
\pgfpathlineto{\pgfqpoint{3.441560in}{0.634661in}}%
\pgfpathlineto{\pgfqpoint{3.445280in}{0.635017in}}%
\pgfpathlineto{\pgfqpoint{3.447760in}{0.637134in}}%
\pgfpathlineto{\pgfqpoint{3.449000in}{0.638890in}}%
\pgfpathlineto{\pgfqpoint{3.450240in}{0.637961in}}%
\pgfpathlineto{\pgfqpoint{3.456440in}{0.639507in}}%
\pgfpathlineto{\pgfqpoint{3.463880in}{0.638090in}}%
\pgfpathlineto{\pgfqpoint{3.465120in}{0.636635in}}%
\pgfpathlineto{\pgfqpoint{3.468840in}{0.638745in}}%
\pgfpathlineto{\pgfqpoint{3.470080in}{0.639850in}}%
\pgfpathlineto{\pgfqpoint{3.473800in}{0.638104in}}%
\pgfpathlineto{\pgfqpoint{3.475040in}{0.638417in}}%
\pgfpathlineto{\pgfqpoint{3.476280in}{0.640204in}}%
\pgfpathlineto{\pgfqpoint{3.480000in}{0.638664in}}%
\pgfpathlineto{\pgfqpoint{3.486200in}{0.645124in}}%
\pgfpathlineto{\pgfqpoint{3.488680in}{0.646411in}}%
\pgfpathlineto{\pgfqpoint{3.491160in}{0.647107in}}%
\pgfpathlineto{\pgfqpoint{3.497360in}{0.648830in}}%
\pgfpathlineto{\pgfqpoint{3.511000in}{0.643085in}}%
\pgfpathlineto{\pgfqpoint{3.512240in}{0.643728in}}%
\pgfpathlineto{\pgfqpoint{3.514720in}{0.645867in}}%
\pgfpathlineto{\pgfqpoint{3.517200in}{0.646051in}}%
\pgfpathlineto{\pgfqpoint{3.520920in}{0.648399in}}%
\pgfpathlineto{\pgfqpoint{3.525880in}{0.647829in}}%
\pgfpathlineto{\pgfqpoint{3.528360in}{0.646474in}}%
\pgfpathlineto{\pgfqpoint{3.532080in}{0.648714in}}%
\pgfpathlineto{\pgfqpoint{3.534560in}{0.649899in}}%
\pgfpathlineto{\pgfqpoint{3.543240in}{0.646276in}}%
\pgfpathlineto{\pgfqpoint{3.544480in}{0.647941in}}%
\pgfpathlineto{\pgfqpoint{3.549440in}{0.643104in}}%
\pgfpathlineto{\pgfqpoint{3.561840in}{0.641028in}}%
\pgfpathlineto{\pgfqpoint{3.563080in}{0.640150in}}%
\pgfpathlineto{\pgfqpoint{3.565560in}{0.641493in}}%
\pgfpathlineto{\pgfqpoint{3.568040in}{0.639830in}}%
\pgfpathlineto{\pgfqpoint{3.573000in}{0.645178in}}%
\pgfpathlineto{\pgfqpoint{3.574240in}{0.644187in}}%
\pgfpathlineto{\pgfqpoint{3.580440in}{0.645648in}}%
\pgfpathlineto{\pgfqpoint{3.587880in}{0.644534in}}%
\pgfpathlineto{\pgfqpoint{3.589120in}{0.643231in}}%
\pgfpathlineto{\pgfqpoint{3.591600in}{0.645799in}}%
\pgfpathlineto{\pgfqpoint{3.592840in}{0.645107in}}%
\pgfpathlineto{\pgfqpoint{3.594080in}{0.646180in}}%
\pgfpathlineto{\pgfqpoint{3.597800in}{0.644548in}}%
\pgfpathlineto{\pgfqpoint{3.599040in}{0.644779in}}%
\pgfpathlineto{\pgfqpoint{3.601520in}{0.646747in}}%
\pgfpathlineto{\pgfqpoint{3.604000in}{0.645654in}}%
\pgfpathlineto{\pgfqpoint{3.607720in}{0.649607in}}%
\pgfpathlineto{\pgfqpoint{3.611440in}{0.651721in}}%
\pgfpathlineto{\pgfqpoint{3.620120in}{0.655177in}}%
\pgfpathlineto{\pgfqpoint{3.622600in}{0.654796in}}%
\pgfpathlineto{\pgfqpoint{3.625080in}{0.652436in}}%
\pgfpathlineto{\pgfqpoint{3.626320in}{0.651397in}}%
\pgfpathlineto{\pgfqpoint{3.628800in}{0.652614in}}%
\pgfpathlineto{\pgfqpoint{3.635000in}{0.649121in}}%
\pgfpathlineto{\pgfqpoint{3.644920in}{0.654693in}}%
\pgfpathlineto{\pgfqpoint{3.647400in}{0.654568in}}%
\pgfpathlineto{\pgfqpoint{3.648640in}{0.655737in}}%
\pgfpathlineto{\pgfqpoint{3.652360in}{0.652729in}}%
\pgfpathlineto{\pgfqpoint{3.656080in}{0.655248in}}%
\pgfpathlineto{\pgfqpoint{3.661040in}{0.656390in}}%
\pgfpathlineto{\pgfqpoint{3.663520in}{0.653700in}}%
\pgfpathlineto{\pgfqpoint{3.666000in}{0.654831in}}%
\pgfpathlineto{\pgfqpoint{3.667240in}{0.652930in}}%
\pgfpathlineto{\pgfqpoint{3.668480in}{0.654485in}}%
\pgfpathlineto{\pgfqpoint{3.673440in}{0.650063in}}%
\pgfpathlineto{\pgfqpoint{3.677160in}{0.649257in}}%
\pgfpathlineto{\pgfqpoint{3.682120in}{0.648900in}}%
\pgfpathlineto{\pgfqpoint{3.683360in}{0.649490in}}%
\pgfpathlineto{\pgfqpoint{3.685840in}{0.647317in}}%
\pgfpathlineto{\pgfqpoint{3.687080in}{0.646505in}}%
\pgfpathlineto{\pgfqpoint{3.689560in}{0.647958in}}%
\pgfpathlineto{\pgfqpoint{3.692040in}{0.646208in}}%
\pgfpathlineto{\pgfqpoint{3.695760in}{0.649915in}}%
\pgfpathlineto{\pgfqpoint{3.697000in}{0.651547in}}%
\pgfpathlineto{\pgfqpoint{3.698240in}{0.650588in}}%
\pgfpathlineto{\pgfqpoint{3.703200in}{0.652648in}}%
\pgfpathlineto{\pgfqpoint{3.706920in}{0.650532in}}%
\pgfpathlineto{\pgfqpoint{3.708160in}{0.651947in}}%
\pgfpathlineto{\pgfqpoint{3.709400in}{0.651481in}}%
\pgfpathlineto{\pgfqpoint{3.710640in}{0.652264in}}%
\pgfpathlineto{\pgfqpoint{3.713120in}{0.649774in}}%
\pgfpathlineto{\pgfqpoint{3.715600in}{0.652071in}}%
\pgfpathlineto{\pgfqpoint{3.716840in}{0.651569in}}%
\pgfpathlineto{\pgfqpoint{3.718080in}{0.652732in}}%
\pgfpathlineto{\pgfqpoint{3.719320in}{0.651530in}}%
\pgfpathlineto{\pgfqpoint{3.720560in}{0.652093in}}%
\pgfpathlineto{\pgfqpoint{3.723040in}{0.651232in}}%
\pgfpathlineto{\pgfqpoint{3.725520in}{0.653388in}}%
\pgfpathlineto{\pgfqpoint{3.728000in}{0.651938in}}%
\pgfpathlineto{\pgfqpoint{3.731720in}{0.655759in}}%
\pgfpathlineto{\pgfqpoint{3.737920in}{0.659820in}}%
\pgfpathlineto{\pgfqpoint{3.740400in}{0.660238in}}%
\pgfpathlineto{\pgfqpoint{3.742880in}{0.661566in}}%
\pgfpathlineto{\pgfqpoint{3.746600in}{0.661434in}}%
\pgfpathlineto{\pgfqpoint{3.751560in}{0.658929in}}%
\pgfpathlineto{\pgfqpoint{3.754040in}{0.658604in}}%
\pgfpathlineto{\pgfqpoint{3.760240in}{0.656396in}}%
\pgfpathlineto{\pgfqpoint{3.762720in}{0.658844in}}%
\pgfpathlineto{\pgfqpoint{3.765200in}{0.659149in}}%
\pgfpathlineto{\pgfqpoint{3.768920in}{0.660795in}}%
\pgfpathlineto{\pgfqpoint{3.771400in}{0.660605in}}%
\pgfpathlineto{\pgfqpoint{3.772640in}{0.661835in}}%
\pgfpathlineto{\pgfqpoint{3.776360in}{0.658138in}}%
\pgfpathlineto{\pgfqpoint{3.780080in}{0.661379in}}%
\pgfpathlineto{\pgfqpoint{3.782560in}{0.662844in}}%
\pgfpathlineto{\pgfqpoint{3.785040in}{0.663120in}}%
\pgfpathlineto{\pgfqpoint{3.787520in}{0.660165in}}%
\pgfpathlineto{\pgfqpoint{3.790000in}{0.661821in}}%
\pgfpathlineto{\pgfqpoint{3.791240in}{0.659844in}}%
\pgfpathlineto{\pgfqpoint{3.792480in}{0.661498in}}%
\pgfpathlineto{\pgfqpoint{3.796200in}{0.656696in}}%
\pgfpathlineto{\pgfqpoint{3.798680in}{0.656732in}}%
\pgfpathlineto{\pgfqpoint{3.802400in}{0.654760in}}%
\pgfpathlineto{\pgfqpoint{3.806120in}{0.655310in}}%
\pgfpathlineto{\pgfqpoint{3.807360in}{0.655669in}}%
\pgfpathlineto{\pgfqpoint{3.809840in}{0.653148in}}%
\pgfpathlineto{\pgfqpoint{3.811080in}{0.652227in}}%
\pgfpathlineto{\pgfqpoint{3.813560in}{0.654018in}}%
\pgfpathlineto{\pgfqpoint{3.816040in}{0.652417in}}%
\pgfpathlineto{\pgfqpoint{3.818520in}{0.655080in}}%
\pgfpathlineto{\pgfqpoint{3.824720in}{0.658298in}}%
\pgfpathlineto{\pgfqpoint{3.825960in}{0.657780in}}%
\pgfpathlineto{\pgfqpoint{3.828440in}{0.658615in}}%
\pgfpathlineto{\pgfqpoint{3.833400in}{0.658279in}}%
\pgfpathlineto{\pgfqpoint{3.834640in}{0.659224in}}%
\pgfpathlineto{\pgfqpoint{3.837120in}{0.656751in}}%
\pgfpathlineto{\pgfqpoint{3.839600in}{0.659141in}}%
\pgfpathlineto{\pgfqpoint{3.840840in}{0.658929in}}%
\pgfpathlineto{\pgfqpoint{3.842080in}{0.659980in}}%
\pgfpathlineto{\pgfqpoint{3.843320in}{0.658922in}}%
\pgfpathlineto{\pgfqpoint{3.844560in}{0.659814in}}%
\pgfpathlineto{\pgfqpoint{3.847040in}{0.659313in}}%
\pgfpathlineto{\pgfqpoint{3.849520in}{0.661120in}}%
\pgfpathlineto{\pgfqpoint{3.852000in}{0.659542in}}%
\pgfpathlineto{\pgfqpoint{3.859440in}{0.664385in}}%
\pgfpathlineto{\pgfqpoint{3.868120in}{0.667581in}}%
\pgfpathlineto{\pgfqpoint{3.870600in}{0.666876in}}%
\pgfpathlineto{\pgfqpoint{3.874320in}{0.663039in}}%
\pgfpathlineto{\pgfqpoint{3.876800in}{0.664598in}}%
\pgfpathlineto{\pgfqpoint{3.883000in}{0.660848in}}%
\pgfpathlineto{\pgfqpoint{3.890440in}{0.664955in}}%
\pgfpathlineto{\pgfqpoint{3.891680in}{0.664473in}}%
\pgfpathlineto{\pgfqpoint{3.894160in}{0.665297in}}%
\pgfpathlineto{\pgfqpoint{3.895400in}{0.665144in}}%
\pgfpathlineto{\pgfqpoint{3.896640in}{0.666434in}}%
\pgfpathlineto{\pgfqpoint{3.900360in}{0.662978in}}%
\pgfpathlineto{\pgfqpoint{3.902840in}{0.664939in}}%
\pgfpathlineto{\pgfqpoint{3.906560in}{0.667243in}}%
\pgfpathlineto{\pgfqpoint{3.909040in}{0.667888in}}%
\pgfpathlineto{\pgfqpoint{3.911520in}{0.664949in}}%
\pgfpathlineto{\pgfqpoint{3.912760in}{0.666457in}}%
\pgfpathlineto{\pgfqpoint{3.914000in}{0.666177in}}%
\pgfpathlineto{\pgfqpoint{3.915240in}{0.664256in}}%
\pgfpathlineto{\pgfqpoint{3.916480in}{0.665904in}}%
\pgfpathlineto{\pgfqpoint{3.920200in}{0.661472in}}%
\pgfpathlineto{\pgfqpoint{3.922680in}{0.661740in}}%
\pgfpathlineto{\pgfqpoint{3.926400in}{0.659832in}}%
\pgfpathlineto{\pgfqpoint{3.930120in}{0.659597in}}%
\pgfpathlineto{\pgfqpoint{3.931360in}{0.659769in}}%
\pgfpathlineto{\pgfqpoint{3.932600in}{0.657366in}}%
\pgfpathlineto{\pgfqpoint{3.938800in}{0.657501in}}%
\pgfpathlineto{\pgfqpoint{3.940040in}{0.656344in}}%
\pgfpathlineto{\pgfqpoint{3.942520in}{0.659131in}}%
\pgfpathlineto{\pgfqpoint{3.948720in}{0.661373in}}%
\pgfpathlineto{\pgfqpoint{3.954920in}{0.659801in}}%
\pgfpathlineto{\pgfqpoint{3.956160in}{0.661564in}}%
\pgfpathlineto{\pgfqpoint{3.961120in}{0.659270in}}%
\pgfpathlineto{\pgfqpoint{3.963600in}{0.661710in}}%
\pgfpathlineto{\pgfqpoint{3.964840in}{0.661475in}}%
\pgfpathlineto{\pgfqpoint{3.966080in}{0.662542in}}%
\pgfpathlineto{\pgfqpoint{3.967320in}{0.661432in}}%
\pgfpathlineto{\pgfqpoint{3.968560in}{0.662398in}}%
\pgfpathlineto{\pgfqpoint{3.971040in}{0.661306in}}%
\pgfpathlineto{\pgfqpoint{3.972280in}{0.662967in}}%
\pgfpathlineto{\pgfqpoint{3.977240in}{0.662860in}}%
\pgfpathlineto{\pgfqpoint{3.984680in}{0.667444in}}%
\pgfpathlineto{\pgfqpoint{3.990880in}{0.669865in}}%
\pgfpathlineto{\pgfqpoint{3.994600in}{0.669214in}}%
\pgfpathlineto{\pgfqpoint{3.998320in}{0.666116in}}%
\pgfpathlineto{\pgfqpoint{4.000800in}{0.667535in}}%
\pgfpathlineto{\pgfqpoint{4.008240in}{0.664044in}}%
\pgfpathlineto{\pgfqpoint{4.010720in}{0.666049in}}%
\pgfpathlineto{\pgfqpoint{4.013200in}{0.666531in}}%
\pgfpathlineto{\pgfqpoint{4.014440in}{0.667536in}}%
\pgfpathlineto{\pgfqpoint{4.016920in}{0.667692in}}%
\pgfpathlineto{\pgfqpoint{4.021880in}{0.667511in}}%
\pgfpathlineto{\pgfqpoint{4.023120in}{0.665743in}}%
\pgfpathlineto{\pgfqpoint{4.026840in}{0.667339in}}%
\pgfpathlineto{\pgfqpoint{4.030560in}{0.669716in}}%
\pgfpathlineto{\pgfqpoint{4.034280in}{0.668443in}}%
\pgfpathlineto{\pgfqpoint{4.035520in}{0.668059in}}%
\pgfpathlineto{\pgfqpoint{4.038000in}{0.669780in}}%
\pgfpathlineto{\pgfqpoint{4.039240in}{0.667906in}}%
\pgfpathlineto{\pgfqpoint{4.040480in}{0.669396in}}%
\pgfpathlineto{\pgfqpoint{4.044200in}{0.664140in}}%
\pgfpathlineto{\pgfqpoint{4.046680in}{0.664623in}}%
\pgfpathlineto{\pgfqpoint{4.050400in}{0.662254in}}%
\pgfpathlineto{\pgfqpoint{4.054120in}{0.662585in}}%
\pgfpathlineto{\pgfqpoint{4.055360in}{0.662896in}}%
\pgfpathlineto{\pgfqpoint{4.057840in}{0.660627in}}%
\pgfpathlineto{\pgfqpoint{4.059080in}{0.659383in}}%
\pgfpathlineto{\pgfqpoint{4.062800in}{0.661048in}}%
\pgfpathlineto{\pgfqpoint{4.064040in}{0.659902in}}%
\pgfpathlineto{\pgfqpoint{4.067760in}{0.663354in}}%
\pgfpathlineto{\pgfqpoint{4.069000in}{0.664895in}}%
\pgfpathlineto{\pgfqpoint{4.071480in}{0.664548in}}%
\pgfpathlineto{\pgfqpoint{4.072720in}{0.665564in}}%
\pgfpathlineto{\pgfqpoint{4.078920in}{0.662306in}}%
\pgfpathlineto{\pgfqpoint{4.080160in}{0.664242in}}%
\pgfpathlineto{\pgfqpoint{4.083880in}{0.663780in}}%
\pgfpathlineto{\pgfqpoint{4.085120in}{0.662249in}}%
\pgfpathlineto{\pgfqpoint{4.090080in}{0.665506in}}%
\pgfpathlineto{\pgfqpoint{4.091320in}{0.664127in}}%
\pgfpathlineto{\pgfqpoint{4.092560in}{0.665053in}}%
\pgfpathlineto{\pgfqpoint{4.095040in}{0.663724in}}%
\pgfpathlineto{\pgfqpoint{4.097520in}{0.664827in}}%
\pgfpathlineto{\pgfqpoint{4.101240in}{0.665100in}}%
\pgfpathlineto{\pgfqpoint{4.107440in}{0.669077in}}%
\pgfpathlineto{\pgfqpoint{4.117360in}{0.672636in}}%
\pgfpathlineto{\pgfqpoint{4.119840in}{0.671218in}}%
\pgfpathlineto{\pgfqpoint{4.122320in}{0.669369in}}%
\pgfpathlineto{\pgfqpoint{4.124800in}{0.670424in}}%
\pgfpathlineto{\pgfqpoint{4.132240in}{0.666452in}}%
\pgfpathlineto{\pgfqpoint{4.134720in}{0.668528in}}%
\pgfpathlineto{\pgfqpoint{4.137200in}{0.668620in}}%
\pgfpathlineto{\pgfqpoint{4.138440in}{0.669449in}}%
\pgfpathlineto{\pgfqpoint{4.142160in}{0.668298in}}%
\pgfpathlineto{\pgfqpoint{4.143400in}{0.667882in}}%
\pgfpathlineto{\pgfqpoint{4.144640in}{0.669007in}}%
\pgfpathlineto{\pgfqpoint{4.147120in}{0.665602in}}%
\pgfpathlineto{\pgfqpoint{4.149600in}{0.666541in}}%
\pgfpathlineto{\pgfqpoint{4.157040in}{0.669319in}}%
\pgfpathlineto{\pgfqpoint{4.159520in}{0.667436in}}%
\pgfpathlineto{\pgfqpoint{4.160760in}{0.669956in}}%
\pgfpathlineto{\pgfqpoint{4.162000in}{0.669867in}}%
\pgfpathlineto{\pgfqpoint{4.163240in}{0.668099in}}%
\pgfpathlineto{\pgfqpoint{4.164480in}{0.669568in}}%
\pgfpathlineto{\pgfqpoint{4.168200in}{0.664297in}}%
\pgfpathlineto{\pgfqpoint{4.170680in}{0.664591in}}%
\pgfpathlineto{\pgfqpoint{4.174400in}{0.662221in}}%
\pgfpathlineto{\pgfqpoint{4.179360in}{0.662515in}}%
\pgfpathlineto{\pgfqpoint{4.181840in}{0.660161in}}%
\pgfpathlineto{\pgfqpoint{4.183080in}{0.659212in}}%
\pgfpathlineto{\pgfqpoint{4.186800in}{0.660853in}}%
\pgfpathlineto{\pgfqpoint{4.188040in}{0.659893in}}%
\pgfpathlineto{\pgfqpoint{4.191760in}{0.663150in}}%
\pgfpathlineto{\pgfqpoint{4.193000in}{0.664401in}}%
\pgfpathlineto{\pgfqpoint{4.194240in}{0.663406in}}%
\pgfpathlineto{\pgfqpoint{4.200440in}{0.665001in}}%
\pgfpathlineto{\pgfqpoint{4.202920in}{0.662472in}}%
\pgfpathlineto{\pgfqpoint{4.204160in}{0.664667in}}%
\pgfpathlineto{\pgfqpoint{4.210360in}{0.664171in}}%
\pgfpathlineto{\pgfqpoint{4.214080in}{0.666756in}}%
\pgfpathlineto{\pgfqpoint{4.215320in}{0.665314in}}%
\pgfpathlineto{\pgfqpoint{4.216560in}{0.666488in}}%
\pgfpathlineto{\pgfqpoint{4.219040in}{0.664902in}}%
\pgfpathlineto{\pgfqpoint{4.220280in}{0.666019in}}%
\pgfpathlineto{\pgfqpoint{4.225240in}{0.664592in}}%
\pgfpathlineto{\pgfqpoint{4.233920in}{0.669247in}}%
\pgfpathlineto{\pgfqpoint{4.236400in}{0.668974in}}%
\pgfpathlineto{\pgfqpoint{4.240120in}{0.671279in}}%
\pgfpathlineto{\pgfqpoint{4.242600in}{0.670899in}}%
\pgfpathlineto{\pgfqpoint{4.246320in}{0.668622in}}%
\pgfpathlineto{\pgfqpoint{4.250040in}{0.668807in}}%
\pgfpathlineto{\pgfqpoint{4.256240in}{0.665416in}}%
\pgfpathlineto{\pgfqpoint{4.261200in}{0.668126in}}%
\pgfpathlineto{\pgfqpoint{4.266160in}{0.667332in}}%
\pgfpathlineto{\pgfqpoint{4.267400in}{0.666900in}}%
\pgfpathlineto{\pgfqpoint{4.268640in}{0.667858in}}%
\pgfpathlineto{\pgfqpoint{4.271120in}{0.664590in}}%
\pgfpathlineto{\pgfqpoint{4.281040in}{0.668468in}}%
\pgfpathlineto{\pgfqpoint{4.283520in}{0.667248in}}%
\pgfpathlineto{\pgfqpoint{4.286000in}{0.669799in}}%
\pgfpathlineto{\pgfqpoint{4.287240in}{0.668070in}}%
\pgfpathlineto{\pgfqpoint{4.288480in}{0.669350in}}%
\pgfpathlineto{\pgfqpoint{4.292200in}{0.663451in}}%
\pgfpathlineto{\pgfqpoint{4.294680in}{0.663562in}}%
\pgfpathlineto{\pgfqpoint{4.298400in}{0.661419in}}%
\pgfpathlineto{\pgfqpoint{4.303360in}{0.663042in}}%
\pgfpathlineto{\pgfqpoint{4.305840in}{0.660345in}}%
\pgfpathlineto{\pgfqpoint{4.307080in}{0.659175in}}%
\pgfpathlineto{\pgfqpoint{4.310800in}{0.660696in}}%
\pgfpathlineto{\pgfqpoint{4.312040in}{0.659651in}}%
\pgfpathlineto{\pgfqpoint{4.314520in}{0.661643in}}%
\pgfpathlineto{\pgfqpoint{4.323200in}{0.664892in}}%
\pgfpathlineto{\pgfqpoint{4.326920in}{0.661057in}}%
\pgfpathlineto{\pgfqpoint{4.329400in}{0.663193in}}%
\pgfpathlineto{\pgfqpoint{4.331880in}{0.663010in}}%
\pgfpathlineto{\pgfqpoint{4.333120in}{0.662238in}}%
\pgfpathlineto{\pgfqpoint{4.338080in}{0.666219in}}%
\pgfpathlineto{\pgfqpoint{4.339320in}{0.664843in}}%
\pgfpathlineto{\pgfqpoint{4.340560in}{0.666046in}}%
\pgfpathlineto{\pgfqpoint{4.343040in}{0.665154in}}%
\pgfpathlineto{\pgfqpoint{4.344280in}{0.666444in}}%
\pgfpathlineto{\pgfqpoint{4.349240in}{0.664398in}}%
\pgfpathlineto{\pgfqpoint{4.355440in}{0.667653in}}%
\pgfpathlineto{\pgfqpoint{4.356680in}{0.667160in}}%
\pgfpathlineto{\pgfqpoint{4.359160in}{0.667558in}}%
\pgfpathlineto{\pgfqpoint{4.360400in}{0.667904in}}%
\pgfpathlineto{\pgfqpoint{4.362880in}{0.670522in}}%
\pgfpathlineto{\pgfqpoint{4.366600in}{0.670553in}}%
\pgfpathlineto{\pgfqpoint{4.371560in}{0.668487in}}%
\pgfpathlineto{\pgfqpoint{4.374040in}{0.668612in}}%
\pgfpathlineto{\pgfqpoint{4.377760in}{0.667092in}}%
\pgfpathlineto{\pgfqpoint{4.380240in}{0.665444in}}%
\pgfpathlineto{\pgfqpoint{4.386440in}{0.668315in}}%
\pgfpathlineto{\pgfqpoint{4.388920in}{0.667604in}}%
\pgfpathlineto{\pgfqpoint{4.393880in}{0.665499in}}%
\pgfpathlineto{\pgfqpoint{4.395120in}{0.664255in}}%
\pgfpathlineto{\pgfqpoint{4.405040in}{0.666938in}}%
\pgfpathlineto{\pgfqpoint{4.406280in}{0.665536in}}%
\pgfpathlineto{\pgfqpoint{4.407520in}{0.666013in}}%
\pgfpathlineto{\pgfqpoint{4.410000in}{0.668714in}}%
\pgfpathlineto{\pgfqpoint{4.411240in}{0.667233in}}%
\pgfpathlineto{\pgfqpoint{4.412480in}{0.668116in}}%
\pgfpathlineto{\pgfqpoint{4.416200in}{0.662944in}}%
\pgfpathlineto{\pgfqpoint{4.418680in}{0.663105in}}%
\pgfpathlineto{\pgfqpoint{4.421160in}{0.660976in}}%
\pgfpathlineto{\pgfqpoint{4.423640in}{0.661502in}}%
\pgfpathlineto{\pgfqpoint{4.427360in}{0.662300in}}%
\pgfpathlineto{\pgfqpoint{4.429840in}{0.659319in}}%
\pgfpathlineto{\pgfqpoint{4.431080in}{0.658072in}}%
\pgfpathlineto{\pgfqpoint{4.434800in}{0.660471in}}%
\pgfpathlineto{\pgfqpoint{4.436040in}{0.659356in}}%
\pgfpathlineto{\pgfqpoint{4.438520in}{0.661339in}}%
\pgfpathlineto{\pgfqpoint{4.441000in}{0.662778in}}%
\pgfpathlineto{\pgfqpoint{4.442240in}{0.661346in}}%
\pgfpathlineto{\pgfqpoint{4.448440in}{0.662666in}}%
\pgfpathlineto{\pgfqpoint{4.450920in}{0.659956in}}%
\pgfpathlineto{\pgfqpoint{4.453400in}{0.661964in}}%
\pgfpathlineto{\pgfqpoint{4.455880in}{0.661893in}}%
\pgfpathlineto{\pgfqpoint{4.457120in}{0.661472in}}%
\pgfpathlineto{\pgfqpoint{4.462080in}{0.664669in}}%
\pgfpathlineto{\pgfqpoint{4.463320in}{0.663369in}}%
\pgfpathlineto{\pgfqpoint{4.465800in}{0.664013in}}%
\pgfpathlineto{\pgfqpoint{4.470760in}{0.664216in}}%
\pgfpathlineto{\pgfqpoint{4.472000in}{0.662807in}}%
\pgfpathlineto{\pgfqpoint{4.479440in}{0.665834in}}%
\pgfpathlineto{\pgfqpoint{4.480680in}{0.665685in}}%
\pgfpathlineto{\pgfqpoint{4.486880in}{0.669910in}}%
\pgfpathlineto{\pgfqpoint{4.490600in}{0.670675in}}%
\pgfpathlineto{\pgfqpoint{4.494320in}{0.668615in}}%
\pgfpathlineto{\pgfqpoint{4.499280in}{0.668729in}}%
\pgfpathlineto{\pgfqpoint{4.501760in}{0.667508in}}%
\pgfpathlineto{\pgfqpoint{4.504240in}{0.666740in}}%
\pgfpathlineto{\pgfqpoint{4.510440in}{0.669056in}}%
\pgfpathlineto{\pgfqpoint{4.512920in}{0.668868in}}%
\pgfpathlineto{\pgfqpoint{4.516640in}{0.668900in}}%
\pgfpathlineto{\pgfqpoint{4.519120in}{0.665777in}}%
\pgfpathlineto{\pgfqpoint{4.529040in}{0.668763in}}%
\pgfpathlineto{\pgfqpoint{4.530280in}{0.667132in}}%
\pgfpathlineto{\pgfqpoint{4.531520in}{0.667775in}}%
\pgfpathlineto{\pgfqpoint{4.532760in}{0.670346in}}%
\pgfpathlineto{\pgfqpoint{4.536480in}{0.669111in}}%
\pgfpathlineto{\pgfqpoint{4.538960in}{0.665261in}}%
\pgfpathlineto{\pgfqpoint{4.540200in}{0.664427in}}%
\pgfpathlineto{\pgfqpoint{4.542680in}{0.664806in}}%
\pgfpathlineto{\pgfqpoint{4.545160in}{0.662178in}}%
\pgfpathlineto{\pgfqpoint{4.547640in}{0.662731in}}%
\pgfpathlineto{\pgfqpoint{4.551360in}{0.663939in}}%
\pgfpathlineto{\pgfqpoint{4.553840in}{0.660536in}}%
\pgfpathlineto{\pgfqpoint{4.555080in}{0.659483in}}%
\pgfpathlineto{\pgfqpoint{4.558800in}{0.661476in}}%
\pgfpathlineto{\pgfqpoint{4.560040in}{0.660631in}}%
\pgfpathlineto{\pgfqpoint{4.565000in}{0.664199in}}%
\pgfpathlineto{\pgfqpoint{4.567480in}{0.663259in}}%
\pgfpathlineto{\pgfqpoint{4.569960in}{0.664382in}}%
\pgfpathlineto{\pgfqpoint{4.572440in}{0.664092in}}%
\pgfpathlineto{\pgfqpoint{4.574920in}{0.661219in}}%
\pgfpathlineto{\pgfqpoint{4.576160in}{0.663225in}}%
\pgfpathlineto{\pgfqpoint{4.581120in}{0.662522in}}%
\pgfpathlineto{\pgfqpoint{4.583600in}{0.664997in}}%
\pgfpathlineto{\pgfqpoint{4.584840in}{0.664530in}}%
\pgfpathlineto{\pgfqpoint{4.586080in}{0.665680in}}%
\pgfpathlineto{\pgfqpoint{4.587320in}{0.664979in}}%
\pgfpathlineto{\pgfqpoint{4.589800in}{0.665395in}}%
\pgfpathlineto{\pgfqpoint{4.591040in}{0.665580in}}%
\pgfpathlineto{\pgfqpoint{4.592280in}{0.667214in}}%
\pgfpathlineto{\pgfqpoint{4.594760in}{0.665801in}}%
\pgfpathlineto{\pgfqpoint{4.597240in}{0.663456in}}%
\pgfpathlineto{\pgfqpoint{4.608400in}{0.667661in}}%
\pgfpathlineto{\pgfqpoint{4.610880in}{0.669833in}}%
\pgfpathlineto{\pgfqpoint{4.614600in}{0.670500in}}%
\pgfpathlineto{\pgfqpoint{4.619560in}{0.668764in}}%
\pgfpathlineto{\pgfqpoint{4.622040in}{0.669174in}}%
\pgfpathlineto{\pgfqpoint{4.627000in}{0.666011in}}%
\pgfpathlineto{\pgfqpoint{4.629480in}{0.666241in}}%
\pgfpathlineto{\pgfqpoint{4.631960in}{0.668065in}}%
\pgfpathlineto{\pgfqpoint{4.634440in}{0.668601in}}%
\pgfpathlineto{\pgfqpoint{4.639400in}{0.666507in}}%
\pgfpathlineto{\pgfqpoint{4.640640in}{0.667543in}}%
\pgfpathlineto{\pgfqpoint{4.643120in}{0.664126in}}%
\pgfpathlineto{\pgfqpoint{4.653040in}{0.666831in}}%
\pgfpathlineto{\pgfqpoint{4.654280in}{0.664707in}}%
\pgfpathlineto{\pgfqpoint{4.655520in}{0.665276in}}%
\pgfpathlineto{\pgfqpoint{4.658000in}{0.667952in}}%
\pgfpathlineto{\pgfqpoint{4.659240in}{0.666734in}}%
\pgfpathlineto{\pgfqpoint{4.660480in}{0.667522in}}%
\pgfpathlineto{\pgfqpoint{4.662960in}{0.663969in}}%
\pgfpathlineto{\pgfqpoint{4.665440in}{0.663174in}}%
\pgfpathlineto{\pgfqpoint{4.666680in}{0.662784in}}%
\pgfpathlineto{\pgfqpoint{4.669160in}{0.660978in}}%
\pgfpathlineto{\pgfqpoint{4.675360in}{0.662556in}}%
\pgfpathlineto{\pgfqpoint{4.677840in}{0.659681in}}%
\pgfpathlineto{\pgfqpoint{4.679080in}{0.658448in}}%
\pgfpathlineto{\pgfqpoint{4.682800in}{0.661059in}}%
\pgfpathlineto{\pgfqpoint{4.684040in}{0.660256in}}%
\pgfpathlineto{\pgfqpoint{4.689000in}{0.663339in}}%
\pgfpathlineto{\pgfqpoint{4.690240in}{0.661906in}}%
\pgfpathlineto{\pgfqpoint{4.695200in}{0.663394in}}%
\pgfpathlineto{\pgfqpoint{4.696440in}{0.663024in}}%
\pgfpathlineto{\pgfqpoint{4.698920in}{0.660374in}}%
\pgfpathlineto{\pgfqpoint{4.700160in}{0.662133in}}%
\pgfpathlineto{\pgfqpoint{4.705120in}{0.660837in}}%
\pgfpathlineto{\pgfqpoint{4.710080in}{0.664674in}}%
\pgfpathlineto{\pgfqpoint{4.712560in}{0.664592in}}%
\pgfpathlineto{\pgfqpoint{4.715040in}{0.664618in}}%
\pgfpathlineto{\pgfqpoint{4.716280in}{0.666502in}}%
\pgfpathlineto{\pgfqpoint{4.718760in}{0.665115in}}%
\pgfpathlineto{\pgfqpoint{4.722480in}{0.662538in}}%
\pgfpathlineto{\pgfqpoint{4.727440in}{0.664454in}}%
\pgfpathlineto{\pgfqpoint{4.729920in}{0.664862in}}%
\pgfpathlineto{\pgfqpoint{4.732400in}{0.666396in}}%
\pgfpathlineto{\pgfqpoint{4.734880in}{0.668910in}}%
\pgfpathlineto{\pgfqpoint{4.738600in}{0.669394in}}%
\pgfpathlineto{\pgfqpoint{4.743560in}{0.666789in}}%
\pgfpathlineto{\pgfqpoint{4.746040in}{0.666593in}}%
\pgfpathlineto{\pgfqpoint{4.747280in}{0.665723in}}%
\pgfpathlineto{\pgfqpoint{4.749760in}{0.665761in}}%
\pgfpathlineto{\pgfqpoint{4.752240in}{0.664871in}}%
\pgfpathlineto{\pgfqpoint{4.753480in}{0.665085in}}%
\pgfpathlineto{\pgfqpoint{4.754720in}{0.666858in}}%
\pgfpathlineto{\pgfqpoint{4.765880in}{0.663778in}}%
\pgfpathlineto{\pgfqpoint{4.768360in}{0.661681in}}%
\pgfpathlineto{\pgfqpoint{4.770840in}{0.662202in}}%
\pgfpathlineto{\pgfqpoint{4.772080in}{0.662168in}}%
\pgfpathlineto{\pgfqpoint{4.774560in}{0.663986in}}%
\pgfpathlineto{\pgfqpoint{4.779520in}{0.663979in}}%
\pgfpathlineto{\pgfqpoint{4.780760in}{0.666400in}}%
\pgfpathlineto{\pgfqpoint{4.784480in}{0.665610in}}%
\pgfpathlineto{\pgfqpoint{4.788200in}{0.660269in}}%
\pgfpathlineto{\pgfqpoint{4.790680in}{0.659842in}}%
\pgfpathlineto{\pgfqpoint{4.793160in}{0.657911in}}%
\pgfpathlineto{\pgfqpoint{4.795640in}{0.658641in}}%
\pgfpathlineto{\pgfqpoint{4.799360in}{0.659294in}}%
\pgfpathlineto{\pgfqpoint{4.801840in}{0.657106in}}%
\pgfpathlineto{\pgfqpoint{4.803080in}{0.656374in}}%
\pgfpathlineto{\pgfqpoint{4.806800in}{0.658953in}}%
\pgfpathlineto{\pgfqpoint{4.808040in}{0.658001in}}%
\pgfpathlineto{\pgfqpoint{4.813000in}{0.661253in}}%
\pgfpathlineto{\pgfqpoint{4.814240in}{0.660295in}}%
\pgfpathlineto{\pgfqpoint{4.817960in}{0.662250in}}%
\pgfpathlineto{\pgfqpoint{4.822920in}{0.658111in}}%
\pgfpathlineto{\pgfqpoint{4.824160in}{0.659615in}}%
\pgfpathlineto{\pgfqpoint{4.829120in}{0.657931in}}%
\pgfpathlineto{\pgfqpoint{4.832840in}{0.661084in}}%
\pgfpathlineto{\pgfqpoint{4.834080in}{0.661983in}}%
\pgfpathlineto{\pgfqpoint{4.835320in}{0.661356in}}%
\pgfpathlineto{\pgfqpoint{4.837800in}{0.661801in}}%
\pgfpathlineto{\pgfqpoint{4.839040in}{0.661831in}}%
\pgfpathlineto{\pgfqpoint{4.841520in}{0.663762in}}%
\pgfpathlineto{\pgfqpoint{4.845240in}{0.660680in}}%
\pgfpathlineto{\pgfqpoint{4.846480in}{0.660473in}}%
\pgfpathlineto{\pgfqpoint{4.850200in}{0.662404in}}%
\pgfpathlineto{\pgfqpoint{4.853920in}{0.662863in}}%
\pgfpathlineto{\pgfqpoint{4.856400in}{0.664359in}}%
\pgfpathlineto{\pgfqpoint{4.858880in}{0.667459in}}%
\pgfpathlineto{\pgfqpoint{4.862600in}{0.667476in}}%
\pgfpathlineto{\pgfqpoint{4.867560in}{0.665051in}}%
\pgfpathlineto{\pgfqpoint{4.870040in}{0.665537in}}%
\pgfpathlineto{\pgfqpoint{4.871280in}{0.664682in}}%
\pgfpathlineto{\pgfqpoint{4.873760in}{0.664836in}}%
\pgfpathlineto{\pgfqpoint{4.876240in}{0.663473in}}%
\pgfpathlineto{\pgfqpoint{4.877480in}{0.663625in}}%
\pgfpathlineto{\pgfqpoint{4.879960in}{0.664816in}}%
\pgfpathlineto{\pgfqpoint{4.882440in}{0.665122in}}%
\pgfpathlineto{\pgfqpoint{4.887400in}{0.662907in}}%
\pgfpathlineto{\pgfqpoint{4.888640in}{0.664153in}}%
\pgfpathlineto{\pgfqpoint{4.892360in}{0.660450in}}%
\pgfpathlineto{\pgfqpoint{4.894840in}{0.660741in}}%
\pgfpathlineto{\pgfqpoint{4.896080in}{0.660398in}}%
\pgfpathlineto{\pgfqpoint{4.898560in}{0.662556in}}%
\pgfpathlineto{\pgfqpoint{4.903520in}{0.661793in}}%
\pgfpathlineto{\pgfqpoint{4.906000in}{0.664268in}}%
\pgfpathlineto{\pgfqpoint{4.907240in}{0.663254in}}%
\pgfpathlineto{\pgfqpoint{4.908480in}{0.664283in}}%
\pgfpathlineto{\pgfqpoint{4.912200in}{0.659201in}}%
\pgfpathlineto{\pgfqpoint{4.914680in}{0.658549in}}%
\pgfpathlineto{\pgfqpoint{4.917160in}{0.655905in}}%
\pgfpathlineto{\pgfqpoint{4.922120in}{0.655783in}}%
\pgfpathlineto{\pgfqpoint{4.923360in}{0.656660in}}%
\pgfpathlineto{\pgfqpoint{4.927080in}{0.653724in}}%
\pgfpathlineto{\pgfqpoint{4.930800in}{0.656570in}}%
\pgfpathlineto{\pgfqpoint{4.932040in}{0.655846in}}%
\pgfpathlineto{\pgfqpoint{4.937000in}{0.658247in}}%
\pgfpathlineto{\pgfqpoint{4.939480in}{0.657312in}}%
\pgfpathlineto{\pgfqpoint{4.940720in}{0.658833in}}%
\pgfpathlineto{\pgfqpoint{4.945680in}{0.656112in}}%
\pgfpathlineto{\pgfqpoint{4.946920in}{0.654647in}}%
\pgfpathlineto{\pgfqpoint{4.948160in}{0.656382in}}%
\pgfpathlineto{\pgfqpoint{4.953120in}{0.654914in}}%
\pgfpathlineto{\pgfqpoint{4.955600in}{0.657883in}}%
\pgfpathlineto{\pgfqpoint{4.956840in}{0.657338in}}%
\pgfpathlineto{\pgfqpoint{4.958080in}{0.658207in}}%
\pgfpathlineto{\pgfqpoint{4.959320in}{0.657477in}}%
\pgfpathlineto{\pgfqpoint{4.965520in}{0.660197in}}%
\pgfpathlineto{\pgfqpoint{4.970480in}{0.655291in}}%
\pgfpathlineto{\pgfqpoint{4.975440in}{0.656741in}}%
\pgfpathlineto{\pgfqpoint{4.977920in}{0.656243in}}%
\pgfpathlineto{\pgfqpoint{4.979160in}{0.656338in}}%
\pgfpathlineto{\pgfqpoint{4.984120in}{0.661786in}}%
\pgfpathlineto{\pgfqpoint{5.001480in}{0.658412in}}%
\pgfpathlineto{\pgfqpoint{5.003960in}{0.659213in}}%
\pgfpathlineto{\pgfqpoint{5.006440in}{0.659089in}}%
\pgfpathlineto{\pgfqpoint{5.007680in}{0.657739in}}%
\pgfpathlineto{\pgfqpoint{5.010160in}{0.658591in}}%
\pgfpathlineto{\pgfqpoint{5.011400in}{0.657553in}}%
\pgfpathlineto{\pgfqpoint{5.012640in}{0.658948in}}%
\pgfpathlineto{\pgfqpoint{5.016360in}{0.654564in}}%
\pgfpathlineto{\pgfqpoint{5.018840in}{0.655368in}}%
\pgfpathlineto{\pgfqpoint{5.020080in}{0.654982in}}%
\pgfpathlineto{\pgfqpoint{5.022560in}{0.656556in}}%
\pgfpathlineto{\pgfqpoint{5.027520in}{0.655613in}}%
\pgfpathlineto{\pgfqpoint{5.030000in}{0.658795in}}%
\pgfpathlineto{\pgfqpoint{5.031240in}{0.658186in}}%
\pgfpathlineto{\pgfqpoint{5.032480in}{0.659134in}}%
\pgfpathlineto{\pgfqpoint{5.036200in}{0.653793in}}%
\pgfpathlineto{\pgfqpoint{5.038680in}{0.654125in}}%
\pgfpathlineto{\pgfqpoint{5.041160in}{0.651918in}}%
\pgfpathlineto{\pgfqpoint{5.048600in}{0.651189in}}%
\pgfpathlineto{\pgfqpoint{5.051080in}{0.648976in}}%
\pgfpathlineto{\pgfqpoint{5.059760in}{0.653110in}}%
\pgfpathlineto{\pgfqpoint{5.061000in}{0.652995in}}%
\pgfpathlineto{\pgfqpoint{5.063480in}{0.651408in}}%
\pgfpathlineto{\pgfqpoint{5.064720in}{0.652686in}}%
\pgfpathlineto{\pgfqpoint{5.069680in}{0.651104in}}%
\pgfpathlineto{\pgfqpoint{5.070920in}{0.649487in}}%
\pgfpathlineto{\pgfqpoint{5.073400in}{0.651019in}}%
\pgfpathlineto{\pgfqpoint{5.075880in}{0.650839in}}%
\pgfpathlineto{\pgfqpoint{5.077120in}{0.649252in}}%
\pgfpathlineto{\pgfqpoint{5.079600in}{0.651910in}}%
\pgfpathlineto{\pgfqpoint{5.080840in}{0.651765in}}%
\pgfpathlineto{\pgfqpoint{5.082080in}{0.653103in}}%
\pgfpathlineto{\pgfqpoint{5.084560in}{0.652151in}}%
\pgfpathlineto{\pgfqpoint{5.087040in}{0.652663in}}%
\pgfpathlineto{\pgfqpoint{5.089520in}{0.655244in}}%
\pgfpathlineto{\pgfqpoint{5.092000in}{0.652900in}}%
\pgfpathlineto{\pgfqpoint{5.098200in}{0.655479in}}%
\pgfpathlineto{\pgfqpoint{5.101920in}{0.654935in}}%
\pgfpathlineto{\pgfqpoint{5.103160in}{0.654708in}}%
\pgfpathlineto{\pgfqpoint{5.106880in}{0.660284in}}%
\pgfpathlineto{\pgfqpoint{5.109360in}{0.660251in}}%
\pgfpathlineto{\pgfqpoint{5.114320in}{0.658389in}}%
\pgfpathlineto{\pgfqpoint{5.116800in}{0.657313in}}%
\pgfpathlineto{\pgfqpoint{5.120520in}{0.657394in}}%
\pgfpathlineto{\pgfqpoint{5.125480in}{0.655737in}}%
\pgfpathlineto{\pgfqpoint{5.127960in}{0.656535in}}%
\pgfpathlineto{\pgfqpoint{5.130440in}{0.657328in}}%
\pgfpathlineto{\pgfqpoint{5.131680in}{0.656212in}}%
\pgfpathlineto{\pgfqpoint{5.134160in}{0.656713in}}%
\pgfpathlineto{\pgfqpoint{5.135400in}{0.655452in}}%
\pgfpathlineto{\pgfqpoint{5.136640in}{0.656843in}}%
\pgfpathlineto{\pgfqpoint{5.140360in}{0.652666in}}%
\pgfpathlineto{\pgfqpoint{5.146560in}{0.655064in}}%
\pgfpathlineto{\pgfqpoint{5.151520in}{0.653618in}}%
\pgfpathlineto{\pgfqpoint{5.154000in}{0.656461in}}%
\pgfpathlineto{\pgfqpoint{5.155240in}{0.656073in}}%
\pgfpathlineto{\pgfqpoint{5.156480in}{0.656894in}}%
\pgfpathlineto{\pgfqpoint{5.158960in}{0.653016in}}%
\pgfpathlineto{\pgfqpoint{5.160200in}{0.652153in}}%
\pgfpathlineto{\pgfqpoint{5.162680in}{0.652860in}}%
\pgfpathlineto{\pgfqpoint{5.163920in}{0.650979in}}%
\pgfpathlineto{\pgfqpoint{5.168880in}{0.651311in}}%
\pgfpathlineto{\pgfqpoint{5.170120in}{0.650402in}}%
\pgfpathlineto{\pgfqpoint{5.171360in}{0.650929in}}%
\pgfpathlineto{\pgfqpoint{5.177560in}{0.648140in}}%
\pgfpathlineto{\pgfqpoint{5.178800in}{0.649676in}}%
\pgfpathlineto{\pgfqpoint{5.180040in}{0.649289in}}%
\pgfpathlineto{\pgfqpoint{5.183760in}{0.652019in}}%
\pgfpathlineto{\pgfqpoint{5.185000in}{0.652289in}}%
\pgfpathlineto{\pgfqpoint{5.187480in}{0.650738in}}%
\pgfpathlineto{\pgfqpoint{5.189960in}{0.651563in}}%
\pgfpathlineto{\pgfqpoint{5.193680in}{0.651575in}}%
\pgfpathlineto{\pgfqpoint{5.194920in}{0.649815in}}%
\pgfpathlineto{\pgfqpoint{5.196160in}{0.651433in}}%
\pgfpathlineto{\pgfqpoint{5.198640in}{0.651047in}}%
\pgfpathlineto{\pgfqpoint{5.199880in}{0.651577in}}%
\pgfpathlineto{\pgfqpoint{5.201120in}{0.650364in}}%
\pgfpathlineto{\pgfqpoint{5.206080in}{0.653856in}}%
\pgfpathlineto{\pgfqpoint{5.208560in}{0.652635in}}%
\pgfpathlineto{\pgfqpoint{5.211040in}{0.652515in}}%
\pgfpathlineto{\pgfqpoint{5.213520in}{0.654449in}}%
\pgfpathlineto{\pgfqpoint{5.217240in}{0.652299in}}%
\pgfpathlineto{\pgfqpoint{5.223440in}{0.653880in}}%
\pgfpathlineto{\pgfqpoint{5.227160in}{0.653947in}}%
\pgfpathlineto{\pgfqpoint{5.230880in}{0.658898in}}%
\pgfpathlineto{\pgfqpoint{5.233360in}{0.658505in}}%
\pgfpathlineto{\pgfqpoint{5.248240in}{0.653047in}}%
\pgfpathlineto{\pgfqpoint{5.250720in}{0.654719in}}%
\pgfpathlineto{\pgfqpoint{5.255680in}{0.652808in}}%
\pgfpathlineto{\pgfqpoint{5.258160in}{0.653187in}}%
\pgfpathlineto{\pgfqpoint{5.259400in}{0.652392in}}%
\pgfpathlineto{\pgfqpoint{5.260640in}{0.653901in}}%
\pgfpathlineto{\pgfqpoint{5.264360in}{0.650994in}}%
\pgfpathlineto{\pgfqpoint{5.270560in}{0.653674in}}%
\pgfpathlineto{\pgfqpoint{5.275520in}{0.651637in}}%
\pgfpathlineto{\pgfqpoint{5.278000in}{0.654621in}}%
\pgfpathlineto{\pgfqpoint{5.279240in}{0.654121in}}%
\pgfpathlineto{\pgfqpoint{5.280480in}{0.654835in}}%
\pgfpathlineto{\pgfqpoint{5.282960in}{0.650456in}}%
\pgfpathlineto{\pgfqpoint{5.284200in}{0.649624in}}%
\pgfpathlineto{\pgfqpoint{5.286680in}{0.650570in}}%
\pgfpathlineto{\pgfqpoint{5.287920in}{0.648531in}}%
\pgfpathlineto{\pgfqpoint{5.295360in}{0.649588in}}%
\pgfpathlineto{\pgfqpoint{5.301560in}{0.646289in}}%
\pgfpathlineto{\pgfqpoint{5.306520in}{0.650215in}}%
\pgfpathlineto{\pgfqpoint{5.309000in}{0.650774in}}%
\pgfpathlineto{\pgfqpoint{5.311480in}{0.649601in}}%
\pgfpathlineto{\pgfqpoint{5.312720in}{0.650210in}}%
\pgfpathlineto{\pgfqpoint{5.313960in}{0.649399in}}%
\pgfpathlineto{\pgfqpoint{5.317680in}{0.649972in}}%
\pgfpathlineto{\pgfqpoint{5.318920in}{0.647608in}}%
\pgfpathlineto{\pgfqpoint{5.323880in}{0.649090in}}%
\pgfpathlineto{\pgfqpoint{5.325120in}{0.648112in}}%
\pgfpathlineto{\pgfqpoint{5.330080in}{0.652534in}}%
\pgfpathlineto{\pgfqpoint{5.333800in}{0.651067in}}%
\pgfpathlineto{\pgfqpoint{5.337520in}{0.652733in}}%
\pgfpathlineto{\pgfqpoint{5.340000in}{0.651055in}}%
\pgfpathlineto{\pgfqpoint{5.347440in}{0.653690in}}%
\pgfpathlineto{\pgfqpoint{5.351160in}{0.654095in}}%
\pgfpathlineto{\pgfqpoint{5.354880in}{0.659986in}}%
\pgfpathlineto{\pgfqpoint{5.357360in}{0.659057in}}%
\pgfpathlineto{\pgfqpoint{5.373480in}{0.652387in}}%
\pgfpathlineto{\pgfqpoint{5.374720in}{0.653098in}}%
\pgfpathlineto{\pgfqpoint{5.379680in}{0.651598in}}%
\pgfpathlineto{\pgfqpoint{5.385880in}{0.653347in}}%
\pgfpathlineto{\pgfqpoint{5.388360in}{0.651813in}}%
\pgfpathlineto{\pgfqpoint{5.392080in}{0.652134in}}%
\pgfpathlineto{\pgfqpoint{5.394560in}{0.654969in}}%
\pgfpathlineto{\pgfqpoint{5.399520in}{0.653450in}}%
\pgfpathlineto{\pgfqpoint{5.402000in}{0.656119in}}%
\pgfpathlineto{\pgfqpoint{5.404480in}{0.657105in}}%
\pgfpathlineto{\pgfqpoint{5.406960in}{0.653384in}}%
\pgfpathlineto{\pgfqpoint{5.409440in}{0.653120in}}%
\pgfpathlineto{\pgfqpoint{5.410680in}{0.653776in}}%
\pgfpathlineto{\pgfqpoint{5.413160in}{0.651833in}}%
\pgfpathlineto{\pgfqpoint{5.418120in}{0.651166in}}%
\pgfpathlineto{\pgfqpoint{5.419360in}{0.652539in}}%
\pgfpathlineto{\pgfqpoint{5.421840in}{0.650389in}}%
\pgfpathlineto{\pgfqpoint{5.424320in}{0.648738in}}%
\pgfpathlineto{\pgfqpoint{5.425560in}{0.648213in}}%
\pgfpathlineto{\pgfqpoint{5.429280in}{0.651867in}}%
\pgfpathlineto{\pgfqpoint{5.433000in}{0.653766in}}%
\pgfpathlineto{\pgfqpoint{5.434240in}{0.652884in}}%
\pgfpathlineto{\pgfqpoint{5.436720in}{0.653917in}}%
\pgfpathlineto{\pgfqpoint{5.437960in}{0.652216in}}%
\pgfpathlineto{\pgfqpoint{5.441680in}{0.653349in}}%
\pgfpathlineto{\pgfqpoint{5.442920in}{0.651166in}}%
\pgfpathlineto{\pgfqpoint{5.447880in}{0.652623in}}%
\pgfpathlineto{\pgfqpoint{5.449120in}{0.652082in}}%
\pgfpathlineto{\pgfqpoint{5.454080in}{0.656492in}}%
\pgfpathlineto{\pgfqpoint{5.456560in}{0.655819in}}%
\pgfpathlineto{\pgfqpoint{5.461520in}{0.656851in}}%
\pgfpathlineto{\pgfqpoint{5.464000in}{0.655137in}}%
\pgfpathlineto{\pgfqpoint{5.473920in}{0.660016in}}%
\pgfpathlineto{\pgfqpoint{5.475160in}{0.659548in}}%
\pgfpathlineto{\pgfqpoint{5.480120in}{0.665365in}}%
\pgfpathlineto{\pgfqpoint{5.483840in}{0.662000in}}%
\pgfpathlineto{\pgfqpoint{5.486320in}{0.660076in}}%
\pgfpathlineto{\pgfqpoint{5.487560in}{0.658976in}}%
\pgfpathlineto{\pgfqpoint{5.490040in}{0.659364in}}%
\pgfpathlineto{\pgfqpoint{5.491280in}{0.657927in}}%
\pgfpathlineto{\pgfqpoint{5.493760in}{0.657843in}}%
\pgfpathlineto{\pgfqpoint{5.507400in}{0.655300in}}%
\pgfpathlineto{\pgfqpoint{5.508640in}{0.656693in}}%
\pgfpathlineto{\pgfqpoint{5.511120in}{0.653472in}}%
\pgfpathlineto{\pgfqpoint{5.514840in}{0.654535in}}%
\pgfpathlineto{\pgfqpoint{5.516080in}{0.653637in}}%
\pgfpathlineto{\pgfqpoint{5.518560in}{0.656304in}}%
\pgfpathlineto{\pgfqpoint{5.523520in}{0.654382in}}%
\pgfpathlineto{\pgfqpoint{5.526000in}{0.657668in}}%
\pgfpathlineto{\pgfqpoint{5.528480in}{0.658922in}}%
\pgfpathlineto{\pgfqpoint{5.530960in}{0.654805in}}%
\pgfpathlineto{\pgfqpoint{5.532200in}{0.654128in}}%
\pgfpathlineto{\pgfqpoint{5.534680in}{0.655964in}}%
\pgfpathlineto{\pgfqpoint{5.537160in}{0.653263in}}%
\pgfpathlineto{\pgfqpoint{5.542120in}{0.653665in}}%
\pgfpathlineto{\pgfqpoint{5.543360in}{0.655145in}}%
\pgfpathlineto{\pgfqpoint{5.549560in}{0.650755in}}%
\pgfpathlineto{\pgfqpoint{5.554520in}{0.656046in}}%
\pgfpathlineto{\pgfqpoint{5.557000in}{0.657341in}}%
\pgfpathlineto{\pgfqpoint{5.558240in}{0.656555in}}%
\pgfpathlineto{\pgfqpoint{5.560720in}{0.657257in}}%
\pgfpathlineto{\pgfqpoint{5.563200in}{0.655261in}}%
\pgfpathlineto{\pgfqpoint{5.565680in}{0.655825in}}%
\pgfpathlineto{\pgfqpoint{5.566920in}{0.654474in}}%
\pgfpathlineto{\pgfqpoint{5.571880in}{0.655646in}}%
\pgfpathlineto{\pgfqpoint{5.573120in}{0.654905in}}%
\pgfpathlineto{\pgfqpoint{5.576840in}{0.657335in}}%
\pgfpathlineto{\pgfqpoint{5.578080in}{0.657976in}}%
\pgfpathlineto{\pgfqpoint{5.579320in}{0.657118in}}%
\pgfpathlineto{\pgfqpoint{5.584280in}{0.660460in}}%
\pgfpathlineto{\pgfqpoint{5.589240in}{0.657284in}}%
\pgfpathlineto{\pgfqpoint{5.600400in}{0.663336in}}%
\pgfpathlineto{\pgfqpoint{5.602880in}{0.667729in}}%
\pgfpathlineto{\pgfqpoint{5.604120in}{0.668256in}}%
\pgfpathlineto{\pgfqpoint{5.606600in}{0.665424in}}%
\pgfpathlineto{\pgfqpoint{5.612800in}{0.661730in}}%
\pgfpathlineto{\pgfqpoint{5.619000in}{0.659488in}}%
\pgfpathlineto{\pgfqpoint{5.621480in}{0.658699in}}%
\pgfpathlineto{\pgfqpoint{5.622720in}{0.658730in}}%
\pgfpathlineto{\pgfqpoint{5.625200in}{0.657088in}}%
\pgfpathlineto{\pgfqpoint{5.626440in}{0.657809in}}%
\pgfpathlineto{\pgfqpoint{5.628920in}{0.657364in}}%
\pgfpathlineto{\pgfqpoint{5.630160in}{0.658331in}}%
\pgfpathlineto{\pgfqpoint{5.631400in}{0.657441in}}%
\pgfpathlineto{\pgfqpoint{5.632640in}{0.659110in}}%
\pgfpathlineto{\pgfqpoint{5.635120in}{0.655826in}}%
\pgfpathlineto{\pgfqpoint{5.638840in}{0.657635in}}%
\pgfpathlineto{\pgfqpoint{5.640080in}{0.656871in}}%
\pgfpathlineto{\pgfqpoint{5.642560in}{0.659530in}}%
\pgfpathlineto{\pgfqpoint{5.647520in}{0.656897in}}%
\pgfpathlineto{\pgfqpoint{5.650000in}{0.660293in}}%
\pgfpathlineto{\pgfqpoint{5.652480in}{0.662197in}}%
\pgfpathlineto{\pgfqpoint{5.654960in}{0.658277in}}%
\pgfpathlineto{\pgfqpoint{5.656200in}{0.657358in}}%
\pgfpathlineto{\pgfqpoint{5.658680in}{0.659847in}}%
\pgfpathlineto{\pgfqpoint{5.661160in}{0.657367in}}%
\pgfpathlineto{\pgfqpoint{5.667360in}{0.657370in}}%
\pgfpathlineto{\pgfqpoint{5.673560in}{0.650537in}}%
\pgfpathlineto{\pgfqpoint{5.674800in}{0.651541in}}%
\pgfpathlineto{\pgfqpoint{5.676040in}{0.651115in}}%
\pgfpathlineto{\pgfqpoint{5.678520in}{0.654613in}}%
\pgfpathlineto{\pgfqpoint{5.681000in}{0.656350in}}%
\pgfpathlineto{\pgfqpoint{5.682240in}{0.655533in}}%
\pgfpathlineto{\pgfqpoint{5.684720in}{0.656719in}}%
\pgfpathlineto{\pgfqpoint{5.687200in}{0.654796in}}%
\pgfpathlineto{\pgfqpoint{5.689680in}{0.655970in}}%
\pgfpathlineto{\pgfqpoint{5.690920in}{0.654677in}}%
\pgfpathlineto{\pgfqpoint{5.694640in}{0.655485in}}%
\pgfpathlineto{\pgfqpoint{5.698360in}{0.654898in}}%
\pgfpathlineto{\pgfqpoint{5.699600in}{0.656812in}}%
\pgfpathlineto{\pgfqpoint{5.703320in}{0.656532in}}%
\pgfpathlineto{\pgfqpoint{5.709520in}{0.659238in}}%
\pgfpathlineto{\pgfqpoint{5.712000in}{0.657116in}}%
\pgfpathlineto{\pgfqpoint{5.716960in}{0.658658in}}%
\pgfpathlineto{\pgfqpoint{5.718200in}{0.658639in}}%
\pgfpathlineto{\pgfqpoint{5.721920in}{0.661944in}}%
\pgfpathlineto{\pgfqpoint{5.723160in}{0.660379in}}%
\pgfpathlineto{\pgfqpoint{5.724400in}{0.662243in}}%
\pgfpathlineto{\pgfqpoint{5.726880in}{0.668220in}}%
\pgfpathlineto{\pgfqpoint{5.728120in}{0.668672in}}%
\pgfpathlineto{\pgfqpoint{5.730600in}{0.666643in}}%
\pgfpathlineto{\pgfqpoint{5.735560in}{0.663270in}}%
\pgfpathlineto{\pgfqpoint{5.738040in}{0.662718in}}%
\pgfpathlineto{\pgfqpoint{5.740520in}{0.662207in}}%
\pgfpathlineto{\pgfqpoint{5.741760in}{0.662643in}}%
\pgfpathlineto{\pgfqpoint{5.745480in}{0.660599in}}%
\pgfpathlineto{\pgfqpoint{5.746720in}{0.660973in}}%
\pgfpathlineto{\pgfqpoint{5.749200in}{0.660142in}}%
\pgfpathlineto{\pgfqpoint{5.756640in}{0.662036in}}%
\pgfpathlineto{\pgfqpoint{5.759120in}{0.658995in}}%
\pgfpathlineto{\pgfqpoint{5.762840in}{0.661538in}}%
\pgfpathlineto{\pgfqpoint{5.764080in}{0.660793in}}%
\pgfpathlineto{\pgfqpoint{5.766560in}{0.663016in}}%
\pgfpathlineto{\pgfqpoint{5.769040in}{0.661610in}}%
\pgfpathlineto{\pgfqpoint{5.770280in}{0.659151in}}%
\pgfpathlineto{\pgfqpoint{5.771520in}{0.659774in}}%
\pgfpathlineto{\pgfqpoint{5.774000in}{0.663166in}}%
\pgfpathlineto{\pgfqpoint{5.776480in}{0.665879in}}%
\pgfpathlineto{\pgfqpoint{5.780200in}{0.659672in}}%
\pgfpathlineto{\pgfqpoint{5.782680in}{0.661232in}}%
\pgfpathlineto{\pgfqpoint{5.785160in}{0.658139in}}%
\pgfpathlineto{\pgfqpoint{5.788880in}{0.658093in}}%
\pgfpathlineto{\pgfqpoint{5.793840in}{0.654287in}}%
\pgfpathlineto{\pgfqpoint{5.795080in}{0.651923in}}%
\pgfpathlineto{\pgfqpoint{5.798800in}{0.653972in}}%
\pgfpathlineto{\pgfqpoint{5.800040in}{0.653337in}}%
\pgfpathlineto{\pgfqpoint{5.803760in}{0.658539in}}%
\pgfpathlineto{\pgfqpoint{5.805000in}{0.659187in}}%
\pgfpathlineto{\pgfqpoint{5.807480in}{0.658629in}}%
\pgfpathlineto{\pgfqpoint{5.808720in}{0.659167in}}%
\pgfpathlineto{\pgfqpoint{5.811200in}{0.657359in}}%
\pgfpathlineto{\pgfqpoint{5.813680in}{0.658528in}}%
\pgfpathlineto{\pgfqpoint{5.814920in}{0.657455in}}%
\pgfpathlineto{\pgfqpoint{5.816160in}{0.658412in}}%
\pgfpathlineto{\pgfqpoint{5.817400in}{0.657676in}}%
\pgfpathlineto{\pgfqpoint{5.819880in}{0.658352in}}%
\pgfpathlineto{\pgfqpoint{5.822360in}{0.657327in}}%
\pgfpathlineto{\pgfqpoint{5.824840in}{0.658683in}}%
\pgfpathlineto{\pgfqpoint{5.831040in}{0.660025in}}%
\pgfpathlineto{\pgfqpoint{5.832280in}{0.662476in}}%
\pgfpathlineto{\pgfqpoint{5.833520in}{0.661936in}}%
\pgfpathlineto{\pgfqpoint{5.836000in}{0.659082in}}%
\pgfpathlineto{\pgfqpoint{5.840960in}{0.662320in}}%
\pgfpathlineto{\pgfqpoint{5.842200in}{0.662550in}}%
\pgfpathlineto{\pgfqpoint{5.845920in}{0.665845in}}%
\pgfpathlineto{\pgfqpoint{5.847160in}{0.664632in}}%
\pgfpathlineto{\pgfqpoint{5.848400in}{0.665819in}}%
\pgfpathlineto{\pgfqpoint{5.850880in}{0.672489in}}%
\pgfpathlineto{\pgfqpoint{5.852120in}{0.673376in}}%
\pgfpathlineto{\pgfqpoint{5.855840in}{0.669797in}}%
\pgfpathlineto{\pgfqpoint{5.857080in}{0.670062in}}%
\pgfpathlineto{\pgfqpoint{5.859560in}{0.667573in}}%
\pgfpathlineto{\pgfqpoint{5.860800in}{0.667931in}}%
\pgfpathlineto{\pgfqpoint{5.863280in}{0.665804in}}%
\pgfpathlineto{\pgfqpoint{5.865760in}{0.667908in}}%
\pgfpathlineto{\pgfqpoint{5.868240in}{0.665108in}}%
\pgfpathlineto{\pgfqpoint{5.870720in}{0.666904in}}%
\pgfpathlineto{\pgfqpoint{5.873200in}{0.666180in}}%
\pgfpathlineto{\pgfqpoint{5.878160in}{0.667351in}}%
\pgfpathlineto{\pgfqpoint{5.879400in}{0.666416in}}%
\pgfpathlineto{\pgfqpoint{5.880640in}{0.667348in}}%
\pgfpathlineto{\pgfqpoint{5.883120in}{0.664943in}}%
\pgfpathlineto{\pgfqpoint{5.884360in}{0.665947in}}%
\pgfpathlineto{\pgfqpoint{5.885600in}{0.665138in}}%
\pgfpathlineto{\pgfqpoint{5.886840in}{0.665692in}}%
\pgfpathlineto{\pgfqpoint{5.888080in}{0.664179in}}%
\pgfpathlineto{\pgfqpoint{5.890560in}{0.665952in}}%
\pgfpathlineto{\pgfqpoint{5.893040in}{0.665001in}}%
\pgfpathlineto{\pgfqpoint{5.894280in}{0.662488in}}%
\pgfpathlineto{\pgfqpoint{5.895520in}{0.663170in}}%
\pgfpathlineto{\pgfqpoint{5.898000in}{0.665711in}}%
\pgfpathlineto{\pgfqpoint{5.900480in}{0.667986in}}%
\pgfpathlineto{\pgfqpoint{5.904200in}{0.661945in}}%
\pgfpathlineto{\pgfqpoint{5.906680in}{0.663402in}}%
\pgfpathlineto{\pgfqpoint{5.909160in}{0.660095in}}%
\pgfpathlineto{\pgfqpoint{5.911640in}{0.659917in}}%
\pgfpathlineto{\pgfqpoint{5.912880in}{0.659768in}}%
\pgfpathlineto{\pgfqpoint{5.914120in}{0.658413in}}%
\pgfpathlineto{\pgfqpoint{5.915360in}{0.659400in}}%
\pgfpathlineto{\pgfqpoint{5.917840in}{0.656258in}}%
\pgfpathlineto{\pgfqpoint{5.919080in}{0.653870in}}%
\pgfpathlineto{\pgfqpoint{5.922800in}{0.656798in}}%
\pgfpathlineto{\pgfqpoint{5.924040in}{0.656371in}}%
\pgfpathlineto{\pgfqpoint{5.929000in}{0.662395in}}%
\pgfpathlineto{\pgfqpoint{5.931480in}{0.661395in}}%
\pgfpathlineto{\pgfqpoint{5.936440in}{0.660400in}}%
\pgfpathlineto{\pgfqpoint{5.938920in}{0.661158in}}%
\pgfpathlineto{\pgfqpoint{5.942640in}{0.662415in}}%
\pgfpathlineto{\pgfqpoint{5.946360in}{0.661392in}}%
\pgfpathlineto{\pgfqpoint{5.948840in}{0.662532in}}%
\pgfpathlineto{\pgfqpoint{5.955040in}{0.662209in}}%
\pgfpathlineto{\pgfqpoint{5.956280in}{0.664707in}}%
\pgfpathlineto{\pgfqpoint{5.957520in}{0.664548in}}%
\pgfpathlineto{\pgfqpoint{5.960000in}{0.662094in}}%
\pgfpathlineto{\pgfqpoint{5.964960in}{0.666988in}}%
\pgfpathlineto{\pgfqpoint{5.967440in}{0.668497in}}%
\pgfpathlineto{\pgfqpoint{5.969920in}{0.670364in}}%
\pgfpathlineto{\pgfqpoint{5.971160in}{0.669050in}}%
\pgfpathlineto{\pgfqpoint{5.972400in}{0.669689in}}%
\pgfpathlineto{\pgfqpoint{5.974880in}{0.676623in}}%
\pgfpathlineto{\pgfqpoint{5.976120in}{0.678100in}}%
\pgfpathlineto{\pgfqpoint{5.978600in}{0.674664in}}%
\pgfpathlineto{\pgfqpoint{5.981080in}{0.672700in}}%
\pgfpathlineto{\pgfqpoint{5.983560in}{0.670731in}}%
\pgfpathlineto{\pgfqpoint{5.984800in}{0.671549in}}%
\pgfpathlineto{\pgfqpoint{5.987280in}{0.668921in}}%
\pgfpathlineto{\pgfqpoint{5.989760in}{0.672069in}}%
\pgfpathlineto{\pgfqpoint{5.992240in}{0.670271in}}%
\pgfpathlineto{\pgfqpoint{5.997200in}{0.672961in}}%
\pgfpathlineto{\pgfqpoint{6.000920in}{0.672762in}}%
\pgfpathlineto{\pgfqpoint{6.002160in}{0.674021in}}%
\pgfpathlineto{\pgfqpoint{6.003400in}{0.672895in}}%
\pgfpathlineto{\pgfqpoint{6.004640in}{0.673976in}}%
\pgfpathlineto{\pgfqpoint{6.007120in}{0.671379in}}%
\pgfpathlineto{\pgfqpoint{6.008360in}{0.672935in}}%
\pgfpathlineto{\pgfqpoint{6.012080in}{0.672577in}}%
\pgfpathlineto{\pgfqpoint{6.014560in}{0.674215in}}%
\pgfpathlineto{\pgfqpoint{6.018280in}{0.669442in}}%
\pgfpathlineto{\pgfqpoint{6.024480in}{0.675144in}}%
\pgfpathlineto{\pgfqpoint{6.028200in}{0.668152in}}%
\pgfpathlineto{\pgfqpoint{6.030680in}{0.669309in}}%
\pgfpathlineto{\pgfqpoint{6.033160in}{0.666796in}}%
\pgfpathlineto{\pgfqpoint{6.034400in}{0.667556in}}%
\pgfpathlineto{\pgfqpoint{6.038120in}{0.665976in}}%
\pgfpathlineto{\pgfqpoint{6.039360in}{0.667192in}}%
\pgfpathlineto{\pgfqpoint{6.043080in}{0.661708in}}%
\pgfpathlineto{\pgfqpoint{6.045560in}{0.663206in}}%
\pgfpathlineto{\pgfqpoint{6.046800in}{0.664189in}}%
\pgfpathlineto{\pgfqpoint{6.048040in}{0.663131in}}%
\pgfpathlineto{\pgfqpoint{6.055480in}{0.667857in}}%
\pgfpathlineto{\pgfqpoint{6.056720in}{0.668351in}}%
\pgfpathlineto{\pgfqpoint{6.059200in}{0.667328in}}%
\pgfpathlineto{\pgfqpoint{6.064160in}{0.670379in}}%
\pgfpathlineto{\pgfqpoint{6.069120in}{0.667834in}}%
\pgfpathlineto{\pgfqpoint{6.075320in}{0.669338in}}%
\pgfpathlineto{\pgfqpoint{6.079040in}{0.668626in}}%
\pgfpathlineto{\pgfqpoint{6.080280in}{0.671669in}}%
\pgfpathlineto{\pgfqpoint{6.081520in}{0.671297in}}%
\pgfpathlineto{\pgfqpoint{6.084000in}{0.668861in}}%
\pgfpathlineto{\pgfqpoint{6.086480in}{0.669803in}}%
\pgfpathlineto{\pgfqpoint{6.088960in}{0.672207in}}%
\pgfpathlineto{\pgfqpoint{6.093920in}{0.674334in}}%
\pgfpathlineto{\pgfqpoint{6.095160in}{0.673988in}}%
\pgfpathlineto{\pgfqpoint{6.096400in}{0.675077in}}%
\pgfpathlineto{\pgfqpoint{6.098880in}{0.680573in}}%
\pgfpathlineto{\pgfqpoint{6.100120in}{0.681834in}}%
\pgfpathlineto{\pgfqpoint{6.102600in}{0.677643in}}%
\pgfpathlineto{\pgfqpoint{6.106320in}{0.674665in}}%
\pgfpathlineto{\pgfqpoint{6.107560in}{0.674017in}}%
\pgfpathlineto{\pgfqpoint{6.108800in}{0.674749in}}%
\pgfpathlineto{\pgfqpoint{6.112520in}{0.673695in}}%
\pgfpathlineto{\pgfqpoint{6.113760in}{0.674909in}}%
\pgfpathlineto{\pgfqpoint{6.116240in}{0.672889in}}%
\pgfpathlineto{\pgfqpoint{6.119960in}{0.673631in}}%
\pgfpathlineto{\pgfqpoint{6.122440in}{0.675385in}}%
\pgfpathlineto{\pgfqpoint{6.124920in}{0.674733in}}%
\pgfpathlineto{\pgfqpoint{6.126160in}{0.676376in}}%
\pgfpathlineto{\pgfqpoint{6.127400in}{0.674988in}}%
\pgfpathlineto{\pgfqpoint{6.128640in}{0.676531in}}%
\pgfpathlineto{\pgfqpoint{6.131120in}{0.673552in}}%
\pgfpathlineto{\pgfqpoint{6.132360in}{0.675804in}}%
\pgfpathlineto{\pgfqpoint{6.136080in}{0.674606in}}%
\pgfpathlineto{\pgfqpoint{6.138560in}{0.676807in}}%
\pgfpathlineto{\pgfqpoint{6.143520in}{0.672699in}}%
\pgfpathlineto{\pgfqpoint{6.147240in}{0.677315in}}%
\pgfpathlineto{\pgfqpoint{6.148480in}{0.678522in}}%
\pgfpathlineto{\pgfqpoint{6.152200in}{0.670962in}}%
\pgfpathlineto{\pgfqpoint{6.154680in}{0.673320in}}%
\pgfpathlineto{\pgfqpoint{6.157160in}{0.670646in}}%
\pgfpathlineto{\pgfqpoint{6.158400in}{0.671207in}}%
\pgfpathlineto{\pgfqpoint{6.162120in}{0.669396in}}%
\pgfpathlineto{\pgfqpoint{6.163360in}{0.670109in}}%
\pgfpathlineto{\pgfqpoint{6.165840in}{0.667349in}}%
\pgfpathlineto{\pgfqpoint{6.167080in}{0.664150in}}%
\pgfpathlineto{\pgfqpoint{6.174520in}{0.667100in}}%
\pgfpathlineto{\pgfqpoint{6.177000in}{0.670893in}}%
\pgfpathlineto{\pgfqpoint{6.178240in}{0.669947in}}%
\pgfpathlineto{\pgfqpoint{6.180720in}{0.670791in}}%
\pgfpathlineto{\pgfqpoint{6.181960in}{0.670310in}}%
\pgfpathlineto{\pgfqpoint{6.185680in}{0.672543in}}%
\pgfpathlineto{\pgfqpoint{6.186920in}{0.671977in}}%
\pgfpathlineto{\pgfqpoint{6.188160in}{0.673407in}}%
\pgfpathlineto{\pgfqpoint{6.189400in}{0.672131in}}%
\pgfpathlineto{\pgfqpoint{6.190640in}{0.672829in}}%
\pgfpathlineto{\pgfqpoint{6.193120in}{0.669342in}}%
\pgfpathlineto{\pgfqpoint{6.200560in}{0.670653in}}%
\pgfpathlineto{\pgfqpoint{6.203040in}{0.669099in}}%
\pgfpathlineto{\pgfqpoint{6.204280in}{0.671667in}}%
\pgfpathlineto{\pgfqpoint{6.206760in}{0.669987in}}%
\pgfpathlineto{\pgfqpoint{6.208000in}{0.668077in}}%
\pgfpathlineto{\pgfqpoint{6.211720in}{0.673216in}}%
\pgfpathlineto{\pgfqpoint{6.212960in}{0.672744in}}%
\pgfpathlineto{\pgfqpoint{6.215440in}{0.673767in}}%
\pgfpathlineto{\pgfqpoint{6.220400in}{0.674935in}}%
\pgfpathlineto{\pgfqpoint{6.222880in}{0.681108in}}%
\pgfpathlineto{\pgfqpoint{6.224120in}{0.682873in}}%
\pgfpathlineto{\pgfqpoint{6.226600in}{0.677578in}}%
\pgfpathlineto{\pgfqpoint{6.227840in}{0.676394in}}%
\pgfpathlineto{\pgfqpoint{6.230320in}{0.676824in}}%
\pgfpathlineto{\pgfqpoint{6.234040in}{0.676506in}}%
\pgfpathlineto{\pgfqpoint{6.235280in}{0.673326in}}%
\pgfpathlineto{\pgfqpoint{6.236520in}{0.673321in}}%
\pgfpathlineto{\pgfqpoint{6.237760in}{0.675032in}}%
\pgfpathlineto{\pgfqpoint{6.241480in}{0.672330in}}%
\pgfpathlineto{\pgfqpoint{6.246440in}{0.675058in}}%
\pgfpathlineto{\pgfqpoint{6.248920in}{0.673454in}}%
\pgfpathlineto{\pgfqpoint{6.250160in}{0.675905in}}%
\pgfpathlineto{\pgfqpoint{6.251400in}{0.675066in}}%
\pgfpathlineto{\pgfqpoint{6.252640in}{0.677548in}}%
\pgfpathlineto{\pgfqpoint{6.255120in}{0.673679in}}%
\pgfpathlineto{\pgfqpoint{6.256360in}{0.676692in}}%
\pgfpathlineto{\pgfqpoint{6.260080in}{0.674928in}}%
\pgfpathlineto{\pgfqpoint{6.262560in}{0.677302in}}%
\pgfpathlineto{\pgfqpoint{6.267520in}{0.674243in}}%
\pgfpathlineto{\pgfqpoint{6.272480in}{0.679713in}}%
\pgfpathlineto{\pgfqpoint{6.276200in}{0.671667in}}%
\pgfpathlineto{\pgfqpoint{6.278680in}{0.672492in}}%
\pgfpathlineto{\pgfqpoint{6.281160in}{0.670727in}}%
\pgfpathlineto{\pgfqpoint{6.282400in}{0.672265in}}%
\pgfpathlineto{\pgfqpoint{6.284880in}{0.671530in}}%
\pgfpathlineto{\pgfqpoint{6.286120in}{0.669360in}}%
\pgfpathlineto{\pgfqpoint{6.287360in}{0.670005in}}%
\pgfpathlineto{\pgfqpoint{6.292320in}{0.664906in}}%
\pgfpathlineto{\pgfqpoint{6.294800in}{0.666556in}}%
\pgfpathlineto{\pgfqpoint{6.297280in}{0.665638in}}%
\pgfpathlineto{\pgfqpoint{6.298520in}{0.666261in}}%
\pgfpathlineto{\pgfqpoint{6.301000in}{0.670834in}}%
\pgfpathlineto{\pgfqpoint{6.302240in}{0.670591in}}%
\pgfpathlineto{\pgfqpoint{6.304720in}{0.672221in}}%
\pgfpathlineto{\pgfqpoint{6.307200in}{0.671983in}}%
\pgfpathlineto{\pgfqpoint{6.308440in}{0.672266in}}%
\pgfpathlineto{\pgfqpoint{6.309680in}{0.674131in}}%
\pgfpathlineto{\pgfqpoint{6.310920in}{0.673440in}}%
\pgfpathlineto{\pgfqpoint{6.314640in}{0.675602in}}%
\pgfpathlineto{\pgfqpoint{6.318360in}{0.670688in}}%
\pgfpathlineto{\pgfqpoint{6.322080in}{0.672699in}}%
\pgfpathlineto{\pgfqpoint{6.323320in}{0.671097in}}%
\pgfpathlineto{\pgfqpoint{6.325800in}{0.672034in}}%
\pgfpathlineto{\pgfqpoint{6.327040in}{0.671817in}}%
\pgfpathlineto{\pgfqpoint{6.328280in}{0.674783in}}%
\pgfpathlineto{\pgfqpoint{6.332000in}{0.671551in}}%
\pgfpathlineto{\pgfqpoint{6.335720in}{0.677181in}}%
\pgfpathlineto{\pgfqpoint{6.338200in}{0.678330in}}%
\pgfpathlineto{\pgfqpoint{6.341920in}{0.680915in}}%
\pgfpathlineto{\pgfqpoint{6.344400in}{0.680608in}}%
\pgfpathlineto{\pgfqpoint{6.346880in}{0.685099in}}%
\pgfpathlineto{\pgfqpoint{6.348120in}{0.687488in}}%
\pgfpathlineto{\pgfqpoint{6.350600in}{0.681364in}}%
\pgfpathlineto{\pgfqpoint{6.353080in}{0.679445in}}%
\pgfpathlineto{\pgfqpoint{6.358040in}{0.678857in}}%
\pgfpathlineto{\pgfqpoint{6.359280in}{0.675680in}}%
\pgfpathlineto{\pgfqpoint{6.360520in}{0.676605in}}%
\pgfpathlineto{\pgfqpoint{6.361760in}{0.679151in}}%
\pgfpathlineto{\pgfqpoint{6.367960in}{0.676136in}}%
\pgfpathlineto{\pgfqpoint{6.370440in}{0.677374in}}%
\pgfpathlineto{\pgfqpoint{6.372920in}{0.674105in}}%
\pgfpathlineto{\pgfqpoint{6.374160in}{0.675449in}}%
\pgfpathlineto{\pgfqpoint{6.375400in}{0.674534in}}%
\pgfpathlineto{\pgfqpoint{6.376640in}{0.677245in}}%
\pgfpathlineto{\pgfqpoint{6.379120in}{0.673898in}}%
\pgfpathlineto{\pgfqpoint{6.380360in}{0.676821in}}%
\pgfpathlineto{\pgfqpoint{6.382840in}{0.675270in}}%
\pgfpathlineto{\pgfqpoint{6.385320in}{0.676942in}}%
\pgfpathlineto{\pgfqpoint{6.389040in}{0.677492in}}%
\pgfpathlineto{\pgfqpoint{6.390280in}{0.674950in}}%
\pgfpathlineto{\pgfqpoint{6.394000in}{0.678421in}}%
\pgfpathlineto{\pgfqpoint{6.396480in}{0.681900in}}%
\pgfpathlineto{\pgfqpoint{6.400200in}{0.671793in}}%
\pgfpathlineto{\pgfqpoint{6.405160in}{0.670271in}}%
\pgfpathlineto{\pgfqpoint{6.406400in}{0.672992in}}%
\pgfpathlineto{\pgfqpoint{6.407640in}{0.672899in}}%
\pgfpathlineto{\pgfqpoint{6.416320in}{0.665368in}}%
\pgfpathlineto{\pgfqpoint{6.421280in}{0.668986in}}%
\pgfpathlineto{\pgfqpoint{6.422520in}{0.669802in}}%
\pgfpathlineto{\pgfqpoint{6.425000in}{0.674217in}}%
\pgfpathlineto{\pgfqpoint{6.427480in}{0.674493in}}%
\pgfpathlineto{\pgfqpoint{6.433680in}{0.676177in}}%
\pgfpathlineto{\pgfqpoint{6.437400in}{0.673974in}}%
\pgfpathlineto{\pgfqpoint{6.438640in}{0.674309in}}%
\pgfpathlineto{\pgfqpoint{6.441120in}{0.669151in}}%
\pgfpathlineto{\pgfqpoint{6.442360in}{0.669374in}}%
\pgfpathlineto{\pgfqpoint{6.444840in}{0.671550in}}%
\pgfpathlineto{\pgfqpoint{6.446080in}{0.672006in}}%
\pgfpathlineto{\pgfqpoint{6.447320in}{0.670365in}}%
\pgfpathlineto{\pgfqpoint{6.449800in}{0.671227in}}%
\pgfpathlineto{\pgfqpoint{6.451040in}{0.671151in}}%
\pgfpathlineto{\pgfqpoint{6.452280in}{0.674300in}}%
\pgfpathlineto{\pgfqpoint{6.456000in}{0.670006in}}%
\pgfpathlineto{\pgfqpoint{6.457240in}{0.666462in}}%
\pgfpathlineto{\pgfqpoint{6.460960in}{0.668828in}}%
\pgfpathlineto{\pgfqpoint{6.464680in}{0.673256in}}%
\pgfpathlineto{\pgfqpoint{6.468400in}{0.671945in}}%
\pgfpathlineto{\pgfqpoint{6.472120in}{0.680844in}}%
\pgfpathlineto{\pgfqpoint{6.474600in}{0.676769in}}%
\pgfpathlineto{\pgfqpoint{6.475840in}{0.675993in}}%
\pgfpathlineto{\pgfqpoint{6.480800in}{0.678729in}}%
\pgfpathlineto{\pgfqpoint{6.482040in}{0.677981in}}%
\pgfpathlineto{\pgfqpoint{6.483280in}{0.674554in}}%
\pgfpathlineto{\pgfqpoint{6.485760in}{0.678490in}}%
\pgfpathlineto{\pgfqpoint{6.491960in}{0.673273in}}%
\pgfpathlineto{\pgfqpoint{6.494440in}{0.675025in}}%
\pgfpathlineto{\pgfqpoint{6.495680in}{0.672006in}}%
\pgfpathlineto{\pgfqpoint{6.499400in}{0.671795in}}%
\pgfpathlineto{\pgfqpoint{6.501880in}{0.674899in}}%
\pgfpathlineto{\pgfqpoint{6.503120in}{0.671584in}}%
\pgfpathlineto{\pgfqpoint{6.504360in}{0.673414in}}%
\pgfpathlineto{\pgfqpoint{6.508080in}{0.670634in}}%
\pgfpathlineto{\pgfqpoint{6.511800in}{0.675674in}}%
\pgfpathlineto{\pgfqpoint{6.513040in}{0.675430in}}%
\pgfpathlineto{\pgfqpoint{6.514280in}{0.672701in}}%
\pgfpathlineto{\pgfqpoint{6.518000in}{0.674455in}}%
\pgfpathlineto{\pgfqpoint{6.520480in}{0.677708in}}%
\pgfpathlineto{\pgfqpoint{6.524200in}{0.666037in}}%
\pgfpathlineto{\pgfqpoint{6.529160in}{0.661567in}}%
\pgfpathlineto{\pgfqpoint{6.531640in}{0.664491in}}%
\pgfpathlineto{\pgfqpoint{6.537840in}{0.659068in}}%
\pgfpathlineto{\pgfqpoint{6.539080in}{0.657487in}}%
\pgfpathlineto{\pgfqpoint{6.541560in}{0.659445in}}%
\pgfpathlineto{\pgfqpoint{6.542800in}{0.661847in}}%
\pgfpathlineto{\pgfqpoint{6.544040in}{0.661501in}}%
\pgfpathlineto{\pgfqpoint{6.549000in}{0.669373in}}%
\pgfpathlineto{\pgfqpoint{6.550240in}{0.668785in}}%
\pgfpathlineto{\pgfqpoint{6.552720in}{0.671023in}}%
\pgfpathlineto{\pgfqpoint{6.556440in}{0.670625in}}%
\pgfpathlineto{\pgfqpoint{6.558920in}{0.671870in}}%
\pgfpathlineto{\pgfqpoint{6.562640in}{0.671188in}}%
\pgfpathlineto{\pgfqpoint{6.566360in}{0.666697in}}%
\pgfpathlineto{\pgfqpoint{6.568840in}{0.667979in}}%
\pgfpathlineto{\pgfqpoint{6.570080in}{0.668390in}}%
\pgfpathlineto{\pgfqpoint{6.571320in}{0.666891in}}%
\pgfpathlineto{\pgfqpoint{6.575040in}{0.667648in}}%
\pgfpathlineto{\pgfqpoint{6.576280in}{0.670508in}}%
\pgfpathlineto{\pgfqpoint{6.580000in}{0.665179in}}%
\pgfpathlineto{\pgfqpoint{6.581240in}{0.664977in}}%
\pgfpathlineto{\pgfqpoint{6.584960in}{0.667834in}}%
\pgfpathlineto{\pgfqpoint{6.587440in}{0.669686in}}%
\pgfpathlineto{\pgfqpoint{6.589920in}{0.672098in}}%
\pgfpathlineto{\pgfqpoint{6.592400in}{0.673213in}}%
\pgfpathlineto{\pgfqpoint{6.594880in}{0.677563in}}%
\pgfpathlineto{\pgfqpoint{6.596120in}{0.679368in}}%
\pgfpathlineto{\pgfqpoint{6.598600in}{0.672942in}}%
\pgfpathlineto{\pgfqpoint{6.601080in}{0.674372in}}%
\pgfpathlineto{\pgfqpoint{6.603560in}{0.676216in}}%
\pgfpathlineto{\pgfqpoint{6.606040in}{0.675854in}}%
\pgfpathlineto{\pgfqpoint{6.607280in}{0.672559in}}%
\pgfpathlineto{\pgfqpoint{6.609760in}{0.678620in}}%
\pgfpathlineto{\pgfqpoint{6.614720in}{0.673396in}}%
\pgfpathlineto{\pgfqpoint{6.615960in}{0.673185in}}%
\pgfpathlineto{\pgfqpoint{6.618440in}{0.675322in}}%
\pgfpathlineto{\pgfqpoint{6.619680in}{0.672175in}}%
\pgfpathlineto{\pgfqpoint{6.622160in}{0.673611in}}%
\pgfpathlineto{\pgfqpoint{6.623400in}{0.671283in}}%
\pgfpathlineto{\pgfqpoint{6.625880in}{0.675836in}}%
\pgfpathlineto{\pgfqpoint{6.627120in}{0.671920in}}%
\pgfpathlineto{\pgfqpoint{6.628360in}{0.672779in}}%
\pgfpathlineto{\pgfqpoint{6.632080in}{0.666828in}}%
\pgfpathlineto{\pgfqpoint{6.634560in}{0.669907in}}%
\pgfpathlineto{\pgfqpoint{6.635800in}{0.670078in}}%
\pgfpathlineto{\pgfqpoint{6.637040in}{0.668947in}}%
\pgfpathlineto{\pgfqpoint{6.638280in}{0.665942in}}%
\pgfpathlineto{\pgfqpoint{6.640760in}{0.667727in}}%
\pgfpathlineto{\pgfqpoint{6.642000in}{0.668484in}}%
\pgfpathlineto{\pgfqpoint{6.643240in}{0.671256in}}%
\pgfpathlineto{\pgfqpoint{6.644480in}{0.670377in}}%
\pgfpathlineto{\pgfqpoint{6.646960in}{0.660676in}}%
\pgfpathlineto{\pgfqpoint{6.648200in}{0.658452in}}%
\pgfpathlineto{\pgfqpoint{6.651920in}{0.658240in}}%
\pgfpathlineto{\pgfqpoint{6.653160in}{0.656860in}}%
\pgfpathlineto{\pgfqpoint{6.654400in}{0.659571in}}%
\pgfpathlineto{\pgfqpoint{6.656880in}{0.658390in}}%
\pgfpathlineto{\pgfqpoint{6.658120in}{0.656011in}}%
\pgfpathlineto{\pgfqpoint{6.659360in}{0.656719in}}%
\pgfpathlineto{\pgfqpoint{6.664320in}{0.651156in}}%
\pgfpathlineto{\pgfqpoint{6.666800in}{0.654714in}}%
\pgfpathlineto{\pgfqpoint{6.668040in}{0.653493in}}%
\pgfpathlineto{\pgfqpoint{6.670520in}{0.656728in}}%
\pgfpathlineto{\pgfqpoint{6.673000in}{0.663217in}}%
\pgfpathlineto{\pgfqpoint{6.682920in}{0.667845in}}%
\pgfpathlineto{\pgfqpoint{6.687880in}{0.663659in}}%
\pgfpathlineto{\pgfqpoint{6.689120in}{0.659509in}}%
\pgfpathlineto{\pgfqpoint{6.690360in}{0.659730in}}%
\pgfpathlineto{\pgfqpoint{6.692840in}{0.662505in}}%
\pgfpathlineto{\pgfqpoint{6.694080in}{0.663275in}}%
\pgfpathlineto{\pgfqpoint{6.695320in}{0.659944in}}%
\pgfpathlineto{\pgfqpoint{6.699040in}{0.659985in}}%
\pgfpathlineto{\pgfqpoint{6.700280in}{0.663159in}}%
\pgfpathlineto{\pgfqpoint{6.704000in}{0.657596in}}%
\pgfpathlineto{\pgfqpoint{6.706480in}{0.660719in}}%
\pgfpathlineto{\pgfqpoint{6.708960in}{0.660129in}}%
\pgfpathlineto{\pgfqpoint{6.711440in}{0.664545in}}%
\pgfpathlineto{\pgfqpoint{6.712680in}{0.666691in}}%
\pgfpathlineto{\pgfqpoint{6.713920in}{0.665961in}}%
\pgfpathlineto{\pgfqpoint{6.715160in}{0.666585in}}%
\pgfpathlineto{\pgfqpoint{6.716400in}{0.665646in}}%
\pgfpathlineto{\pgfqpoint{6.718880in}{0.671650in}}%
\pgfpathlineto{\pgfqpoint{6.720120in}{0.673316in}}%
\pgfpathlineto{\pgfqpoint{6.722600in}{0.667019in}}%
\pgfpathlineto{\pgfqpoint{6.723840in}{0.667970in}}%
\pgfpathlineto{\pgfqpoint{6.727560in}{0.673275in}}%
\pgfpathlineto{\pgfqpoint{6.730040in}{0.675393in}}%
\pgfpathlineto{\pgfqpoint{6.731280in}{0.674140in}}%
\pgfpathlineto{\pgfqpoint{6.733760in}{0.681948in}}%
\pgfpathlineto{\pgfqpoint{6.735000in}{0.680760in}}%
\pgfpathlineto{\pgfqpoint{6.737480in}{0.676231in}}%
\pgfpathlineto{\pgfqpoint{6.742440in}{0.677181in}}%
\pgfpathlineto{\pgfqpoint{6.743680in}{0.674392in}}%
\pgfpathlineto{\pgfqpoint{6.746160in}{0.680095in}}%
\pgfpathlineto{\pgfqpoint{6.747400in}{0.679028in}}%
\pgfpathlineto{\pgfqpoint{6.749880in}{0.684553in}}%
\pgfpathlineto{\pgfqpoint{6.751120in}{0.681449in}}%
\pgfpathlineto{\pgfqpoint{6.752360in}{0.681745in}}%
\pgfpathlineto{\pgfqpoint{6.756080in}{0.673184in}}%
\pgfpathlineto{\pgfqpoint{6.757320in}{0.673137in}}%
\pgfpathlineto{\pgfqpoint{6.758560in}{0.675244in}}%
\pgfpathlineto{\pgfqpoint{6.761040in}{0.673758in}}%
\pgfpathlineto{\pgfqpoint{6.762280in}{0.670781in}}%
\pgfpathlineto{\pgfqpoint{6.766000in}{0.673533in}}%
\pgfpathlineto{\pgfqpoint{6.767240in}{0.675856in}}%
\pgfpathlineto{\pgfqpoint{6.768480in}{0.672919in}}%
\pgfpathlineto{\pgfqpoint{6.770960in}{0.663123in}}%
\pgfpathlineto{\pgfqpoint{6.773440in}{0.660826in}}%
\pgfpathlineto{\pgfqpoint{6.774680in}{0.660815in}}%
\pgfpathlineto{\pgfqpoint{6.777160in}{0.656608in}}%
\pgfpathlineto{\pgfqpoint{6.778400in}{0.659478in}}%
\pgfpathlineto{\pgfqpoint{6.780880in}{0.659401in}}%
\pgfpathlineto{\pgfqpoint{6.784600in}{0.655240in}}%
\pgfpathlineto{\pgfqpoint{6.787080in}{0.650272in}}%
\pgfpathlineto{\pgfqpoint{6.788320in}{0.651516in}}%
\pgfpathlineto{\pgfqpoint{6.790800in}{0.655073in}}%
\pgfpathlineto{\pgfqpoint{6.792040in}{0.652741in}}%
\pgfpathlineto{\pgfqpoint{6.795760in}{0.659618in}}%
\pgfpathlineto{\pgfqpoint{6.798240in}{0.662515in}}%
\pgfpathlineto{\pgfqpoint{6.799480in}{0.662382in}}%
\pgfpathlineto{\pgfqpoint{6.801960in}{0.665081in}}%
\pgfpathlineto{\pgfqpoint{6.803200in}{0.664608in}}%
\pgfpathlineto{\pgfqpoint{6.805680in}{0.667767in}}%
\pgfpathlineto{\pgfqpoint{6.806920in}{0.668264in}}%
\pgfpathlineto{\pgfqpoint{6.809400in}{0.665140in}}%
\pgfpathlineto{\pgfqpoint{6.810640in}{0.664632in}}%
\pgfpathlineto{\pgfqpoint{6.814360in}{0.658156in}}%
\pgfpathlineto{\pgfqpoint{6.815600in}{0.660689in}}%
\pgfpathlineto{\pgfqpoint{6.816840in}{0.660565in}}%
\pgfpathlineto{\pgfqpoint{6.818080in}{0.662585in}}%
\pgfpathlineto{\pgfqpoint{6.819320in}{0.661562in}}%
\pgfpathlineto{\pgfqpoint{6.821800in}{0.661624in}}%
\pgfpathlineto{\pgfqpoint{6.823040in}{0.660656in}}%
\pgfpathlineto{\pgfqpoint{6.824280in}{0.662712in}}%
\pgfpathlineto{\pgfqpoint{6.825520in}{0.662067in}}%
\pgfpathlineto{\pgfqpoint{6.828000in}{0.657155in}}%
\pgfpathlineto{\pgfqpoint{6.829240in}{0.662708in}}%
\pgfpathlineto{\pgfqpoint{6.832960in}{0.656453in}}%
\pgfpathlineto{\pgfqpoint{6.839160in}{0.666225in}}%
\pgfpathlineto{\pgfqpoint{6.840400in}{0.664337in}}%
\pgfpathlineto{\pgfqpoint{6.842880in}{0.671856in}}%
\pgfpathlineto{\pgfqpoint{6.844120in}{0.674195in}}%
\pgfpathlineto{\pgfqpoint{6.845360in}{0.668228in}}%
\pgfpathlineto{\pgfqpoint{6.847840in}{0.669228in}}%
\pgfpathlineto{\pgfqpoint{6.850320in}{0.675616in}}%
\pgfpathlineto{\pgfqpoint{6.851560in}{0.674338in}}%
\pgfpathlineto{\pgfqpoint{6.855280in}{0.673076in}}%
\pgfpathlineto{\pgfqpoint{6.857760in}{0.682568in}}%
\pgfpathlineto{\pgfqpoint{6.859000in}{0.681145in}}%
\pgfpathlineto{\pgfqpoint{6.861480in}{0.676407in}}%
\pgfpathlineto{\pgfqpoint{6.863960in}{0.679030in}}%
\pgfpathlineto{\pgfqpoint{6.865200in}{0.681319in}}%
\pgfpathlineto{\pgfqpoint{6.867680in}{0.674999in}}%
\pgfpathlineto{\pgfqpoint{6.870160in}{0.681241in}}%
\pgfpathlineto{\pgfqpoint{6.871400in}{0.681438in}}%
\pgfpathlineto{\pgfqpoint{6.873880in}{0.689442in}}%
\pgfpathlineto{\pgfqpoint{6.880080in}{0.674534in}}%
\pgfpathlineto{\pgfqpoint{6.881320in}{0.674123in}}%
\pgfpathlineto{\pgfqpoint{6.882560in}{0.675471in}}%
\pgfpathlineto{\pgfqpoint{6.886280in}{0.664917in}}%
\pgfpathlineto{\pgfqpoint{6.888760in}{0.668948in}}%
\pgfpathlineto{\pgfqpoint{6.891240in}{0.674206in}}%
\pgfpathlineto{\pgfqpoint{6.892480in}{0.672743in}}%
\pgfpathlineto{\pgfqpoint{6.894960in}{0.667374in}}%
\pgfpathlineto{\pgfqpoint{6.897440in}{0.666270in}}%
\pgfpathlineto{\pgfqpoint{6.898680in}{0.668174in}}%
\pgfpathlineto{\pgfqpoint{6.901160in}{0.662852in}}%
\pgfpathlineto{\pgfqpoint{6.902400in}{0.664922in}}%
\pgfpathlineto{\pgfqpoint{6.907360in}{0.659792in}}%
\pgfpathlineto{\pgfqpoint{6.909840in}{0.660228in}}%
\pgfpathlineto{\pgfqpoint{6.911080in}{0.656869in}}%
\pgfpathlineto{\pgfqpoint{6.912320in}{0.658092in}}%
\pgfpathlineto{\pgfqpoint{6.913560in}{0.656579in}}%
\pgfpathlineto{\pgfqpoint{6.914800in}{0.659407in}}%
\pgfpathlineto{\pgfqpoint{6.916040in}{0.658372in}}%
\pgfpathlineto{\pgfqpoint{6.923480in}{0.668067in}}%
\pgfpathlineto{\pgfqpoint{6.927200in}{0.665159in}}%
\pgfpathlineto{\pgfqpoint{6.929680in}{0.668140in}}%
\pgfpathlineto{\pgfqpoint{6.933400in}{0.660851in}}%
\pgfpathlineto{\pgfqpoint{6.935880in}{0.659419in}}%
\pgfpathlineto{\pgfqpoint{6.938360in}{0.656046in}}%
\pgfpathlineto{\pgfqpoint{6.939600in}{0.657457in}}%
\pgfpathlineto{\pgfqpoint{6.940840in}{0.654985in}}%
\pgfpathlineto{\pgfqpoint{6.942080in}{0.656136in}}%
\pgfpathlineto{\pgfqpoint{6.944560in}{0.655093in}}%
\pgfpathlineto{\pgfqpoint{6.947040in}{0.658880in}}%
\pgfpathlineto{\pgfqpoint{6.948280in}{0.662618in}}%
\pgfpathlineto{\pgfqpoint{6.949520in}{0.662592in}}%
\pgfpathlineto{\pgfqpoint{6.952000in}{0.654446in}}%
\pgfpathlineto{\pgfqpoint{6.953240in}{0.663727in}}%
\pgfpathlineto{\pgfqpoint{6.955720in}{0.657162in}}%
\pgfpathlineto{\pgfqpoint{6.956960in}{0.652623in}}%
\pgfpathlineto{\pgfqpoint{6.960680in}{0.664317in}}%
\pgfpathlineto{\pgfqpoint{6.961920in}{0.664167in}}%
\pgfpathlineto{\pgfqpoint{6.963160in}{0.667854in}}%
\pgfpathlineto{\pgfqpoint{6.964400in}{0.667978in}}%
\pgfpathlineto{\pgfqpoint{6.966880in}{0.672410in}}%
\pgfpathlineto{\pgfqpoint{6.968120in}{0.673860in}}%
\pgfpathlineto{\pgfqpoint{6.970600in}{0.664269in}}%
\pgfpathlineto{\pgfqpoint{6.973080in}{0.667682in}}%
\pgfpathlineto{\pgfqpoint{6.974320in}{0.671096in}}%
\pgfpathlineto{\pgfqpoint{6.976800in}{0.664545in}}%
\pgfpathlineto{\pgfqpoint{6.979280in}{0.667004in}}%
\pgfpathlineto{\pgfqpoint{6.981760in}{0.674852in}}%
\pgfpathlineto{\pgfqpoint{6.985480in}{0.662811in}}%
\pgfpathlineto{\pgfqpoint{6.987960in}{0.664081in}}%
\pgfpathlineto{\pgfqpoint{6.989200in}{0.670122in}}%
\pgfpathlineto{\pgfqpoint{6.991680in}{0.665489in}}%
\pgfpathlineto{\pgfqpoint{6.997880in}{0.679521in}}%
\pgfpathlineto{\pgfqpoint{6.999120in}{0.676617in}}%
\pgfpathlineto{\pgfqpoint{7.000360in}{0.678974in}}%
\pgfpathlineto{\pgfqpoint{7.002840in}{0.670837in}}%
\pgfpathlineto{\pgfqpoint{7.005320in}{0.667505in}}%
\pgfpathlineto{\pgfqpoint{7.006560in}{0.667970in}}%
\pgfpathlineto{\pgfqpoint{7.007800in}{0.663849in}}%
\pgfpathlineto{\pgfqpoint{7.009040in}{0.665490in}}%
\pgfpathlineto{\pgfqpoint{7.010280in}{0.660157in}}%
\pgfpathlineto{\pgfqpoint{7.014000in}{0.668700in}}%
\pgfpathlineto{\pgfqpoint{7.015240in}{0.673104in}}%
\pgfpathlineto{\pgfqpoint{7.016480in}{0.670323in}}%
\pgfpathlineto{\pgfqpoint{7.018960in}{0.662746in}}%
\pgfpathlineto{\pgfqpoint{7.022680in}{0.668490in}}%
\pgfpathlineto{\pgfqpoint{7.023920in}{0.665711in}}%
\pgfpathlineto{\pgfqpoint{7.026400in}{0.669168in}}%
\pgfpathlineto{\pgfqpoint{7.030120in}{0.661452in}}%
\pgfpathlineto{\pgfqpoint{7.033840in}{0.657906in}}%
\pgfpathlineto{\pgfqpoint{7.037560in}{0.657482in}}%
\pgfpathlineto{\pgfqpoint{7.038800in}{0.660372in}}%
\pgfpathlineto{\pgfqpoint{7.041280in}{0.669110in}}%
\pgfpathlineto{\pgfqpoint{7.042520in}{0.668972in}}%
\pgfpathlineto{\pgfqpoint{7.045000in}{0.671777in}}%
\pgfpathlineto{\pgfqpoint{7.047480in}{0.678054in}}%
\pgfpathlineto{\pgfqpoint{7.051200in}{0.676157in}}%
\pgfpathlineto{\pgfqpoint{7.052440in}{0.674309in}}%
\pgfpathlineto{\pgfqpoint{7.053680in}{0.676150in}}%
\pgfpathlineto{\pgfqpoint{7.058640in}{0.662412in}}%
\pgfpathlineto{\pgfqpoint{7.059880in}{0.663127in}}%
\pgfpathlineto{\pgfqpoint{7.062360in}{0.657444in}}%
\pgfpathlineto{\pgfqpoint{7.063600in}{0.662704in}}%
\pgfpathlineto{\pgfqpoint{7.066080in}{0.658913in}}%
\pgfpathlineto{\pgfqpoint{7.067320in}{0.659931in}}%
\pgfpathlineto{\pgfqpoint{7.069800in}{0.656366in}}%
\pgfpathlineto{\pgfqpoint{7.071040in}{0.658463in}}%
\pgfpathlineto{\pgfqpoint{7.073520in}{0.665979in}}%
\pgfpathlineto{\pgfqpoint{7.074760in}{0.664648in}}%
\pgfpathlineto{\pgfqpoint{7.076000in}{0.659566in}}%
\pgfpathlineto{\pgfqpoint{7.077240in}{0.676873in}}%
\pgfpathlineto{\pgfqpoint{7.080960in}{0.663359in}}%
\pgfpathlineto{\pgfqpoint{7.083440in}{0.666510in}}%
\pgfpathlineto{\pgfqpoint{7.088400in}{0.678327in}}%
\pgfpathlineto{\pgfqpoint{7.090880in}{0.678339in}}%
\pgfpathlineto{\pgfqpoint{7.092120in}{0.684444in}}%
\pgfpathlineto{\pgfqpoint{7.094600in}{0.670954in}}%
\pgfpathlineto{\pgfqpoint{7.097080in}{0.675069in}}%
\pgfpathlineto{\pgfqpoint{7.098320in}{0.685550in}}%
\pgfpathlineto{\pgfqpoint{7.099560in}{0.684555in}}%
\pgfpathlineto{\pgfqpoint{7.100800in}{0.686426in}}%
\pgfpathlineto{\pgfqpoint{7.103280in}{0.686350in}}%
\pgfpathlineto{\pgfqpoint{7.104520in}{0.693760in}}%
\pgfpathlineto{\pgfqpoint{7.105760in}{0.692313in}}%
\pgfpathlineto{\pgfqpoint{7.109480in}{0.674051in}}%
\pgfpathlineto{\pgfqpoint{7.111960in}{0.678883in}}%
\pgfpathlineto{\pgfqpoint{7.113200in}{0.688831in}}%
\pgfpathlineto{\pgfqpoint{7.116920in}{0.676220in}}%
\pgfpathlineto{\pgfqpoint{7.118160in}{0.677589in}}%
\pgfpathlineto{\pgfqpoint{7.119400in}{0.675380in}}%
\pgfpathlineto{\pgfqpoint{7.120640in}{0.678974in}}%
\pgfpathlineto{\pgfqpoint{7.123120in}{0.678483in}}%
\pgfpathlineto{\pgfqpoint{7.124360in}{0.681121in}}%
\pgfpathlineto{\pgfqpoint{7.125600in}{0.672510in}}%
\pgfpathlineto{\pgfqpoint{7.128080in}{0.680704in}}%
\pgfpathlineto{\pgfqpoint{7.130560in}{0.682185in}}%
\pgfpathlineto{\pgfqpoint{7.134280in}{0.669771in}}%
\pgfpathlineto{\pgfqpoint{7.135520in}{0.670662in}}%
\pgfpathlineto{\pgfqpoint{7.139240in}{0.693466in}}%
\pgfpathlineto{\pgfqpoint{7.142960in}{0.672030in}}%
\pgfpathlineto{\pgfqpoint{7.146680in}{0.667507in}}%
\pgfpathlineto{\pgfqpoint{7.147920in}{0.660304in}}%
\pgfpathlineto{\pgfqpoint{7.150400in}{0.663064in}}%
\pgfpathlineto{\pgfqpoint{7.151640in}{0.660852in}}%
\pgfpathlineto{\pgfqpoint{7.152880in}{0.654916in}}%
\pgfpathlineto{\pgfqpoint{7.154120in}{0.654822in}}%
\pgfpathlineto{\pgfqpoint{7.156600in}{0.667966in}}%
\pgfpathlineto{\pgfqpoint{7.157840in}{0.670473in}}%
\pgfpathlineto{\pgfqpoint{7.159080in}{0.669147in}}%
\pgfpathlineto{\pgfqpoint{7.164040in}{0.688037in}}%
\pgfpathlineto{\pgfqpoint{7.165280in}{0.690161in}}%
\pgfpathlineto{\pgfqpoint{7.167760in}{0.686318in}}%
\pgfpathlineto{\pgfqpoint{7.169000in}{0.686504in}}%
\pgfpathlineto{\pgfqpoint{7.170240in}{0.693975in}}%
\pgfpathlineto{\pgfqpoint{7.171480in}{0.691926in}}%
\pgfpathlineto{\pgfqpoint{7.175200in}{0.698540in}}%
\pgfpathlineto{\pgfqpoint{7.181400in}{0.674119in}}%
\pgfpathlineto{\pgfqpoint{7.182640in}{0.671997in}}%
\pgfpathlineto{\pgfqpoint{7.183880in}{0.680242in}}%
\pgfpathlineto{\pgfqpoint{7.185120in}{0.678864in}}%
\pgfpathlineto{\pgfqpoint{7.186360in}{0.675599in}}%
\pgfpathlineto{\pgfqpoint{7.188840in}{0.687967in}}%
\pgfpathlineto{\pgfqpoint{7.191320in}{0.686265in}}%
\pgfpathlineto{\pgfqpoint{7.193800in}{0.680232in}}%
\pgfpathlineto{\pgfqpoint{7.197520in}{0.689050in}}%
\pgfpathlineto{\pgfqpoint{7.198760in}{0.689851in}}%
\pgfpathlineto{\pgfqpoint{7.200000in}{0.687151in}}%
\pgfpathlineto{\pgfqpoint{7.200000in}{0.687151in}}%
\pgfusepath{stroke}%
\end{pgfscope}%
\begin{pgfscope}%
\pgfpathrectangle{\pgfqpoint{1.000000in}{0.350000in}}{\pgfqpoint{6.200000in}{2.800000in}} %
\pgfusepath{clip}%
\pgfsetrectcap%
\pgfsetroundjoin%
\pgfsetlinewidth{1.003750pt}%
\definecolor{currentstroke}{rgb}{1.000000,0.000000,0.000000}%
\pgfsetstrokecolor{currentstroke}%
\pgfsetdash{}{0pt}%
\pgfpathmoveto{\pgfqpoint{1.001240in}{1.774215in}}%
\pgfpathlineto{\pgfqpoint{1.002480in}{2.510200in}}%
\pgfpathlineto{\pgfqpoint{1.003720in}{2.825165in}}%
\pgfpathlineto{\pgfqpoint{1.004960in}{2.886981in}}%
\pgfpathlineto{\pgfqpoint{1.008680in}{2.826133in}}%
\pgfpathlineto{\pgfqpoint{1.014880in}{2.690100in}}%
\pgfpathlineto{\pgfqpoint{1.017360in}{2.659437in}}%
\pgfpathlineto{\pgfqpoint{1.023560in}{2.585453in}}%
\pgfpathlineto{\pgfqpoint{1.024800in}{2.586676in}}%
\pgfpathlineto{\pgfqpoint{1.027280in}{2.583759in}}%
\pgfpathlineto{\pgfqpoint{1.029760in}{2.565573in}}%
\pgfpathlineto{\pgfqpoint{1.031000in}{2.566123in}}%
\pgfpathlineto{\pgfqpoint{1.032240in}{2.565215in}}%
\pgfpathlineto{\pgfqpoint{1.033480in}{2.561869in}}%
\pgfpathlineto{\pgfqpoint{1.038440in}{2.504394in}}%
\pgfpathlineto{\pgfqpoint{1.042160in}{2.482819in}}%
\pgfpathlineto{\pgfqpoint{1.044640in}{2.490704in}}%
\pgfpathlineto{\pgfqpoint{1.045880in}{2.491285in}}%
\pgfpathlineto{\pgfqpoint{1.050840in}{2.464797in}}%
\pgfpathlineto{\pgfqpoint{1.052080in}{2.463037in}}%
\pgfpathlineto{\pgfqpoint{1.057040in}{2.427839in}}%
\pgfpathlineto{\pgfqpoint{1.059520in}{2.412494in}}%
\pgfpathlineto{\pgfqpoint{1.060760in}{2.414698in}}%
\pgfpathlineto{\pgfqpoint{1.062000in}{2.412791in}}%
\pgfpathlineto{\pgfqpoint{1.066960in}{2.392809in}}%
\pgfpathlineto{\pgfqpoint{1.068200in}{2.394792in}}%
\pgfpathlineto{\pgfqpoint{1.069440in}{2.399444in}}%
\pgfpathlineto{\pgfqpoint{1.070680in}{2.396984in}}%
\pgfpathlineto{\pgfqpoint{1.073160in}{2.380747in}}%
\pgfpathlineto{\pgfqpoint{1.074400in}{2.379257in}}%
\pgfpathlineto{\pgfqpoint{1.075640in}{2.375826in}}%
\pgfpathlineto{\pgfqpoint{1.076880in}{2.376772in}}%
\pgfpathlineto{\pgfqpoint{1.078120in}{2.372469in}}%
\pgfpathlineto{\pgfqpoint{1.079360in}{2.375690in}}%
\pgfpathlineto{\pgfqpoint{1.081840in}{2.368404in}}%
\pgfpathlineto{\pgfqpoint{1.084320in}{2.375882in}}%
\pgfpathlineto{\pgfqpoint{1.085560in}{2.374537in}}%
\pgfpathlineto{\pgfqpoint{1.090520in}{2.363940in}}%
\pgfpathlineto{\pgfqpoint{1.091760in}{2.363758in}}%
\pgfpathlineto{\pgfqpoint{1.093000in}{2.364949in}}%
\pgfpathlineto{\pgfqpoint{1.094240in}{2.369657in}}%
\pgfpathlineto{\pgfqpoint{1.096720in}{2.356760in}}%
\pgfpathlineto{\pgfqpoint{1.100440in}{2.346911in}}%
\pgfpathlineto{\pgfqpoint{1.101680in}{2.348092in}}%
\pgfpathlineto{\pgfqpoint{1.102920in}{2.349620in}}%
\pgfpathlineto{\pgfqpoint{1.104160in}{2.348469in}}%
\pgfpathlineto{\pgfqpoint{1.105400in}{2.352567in}}%
\pgfpathlineto{\pgfqpoint{1.106640in}{2.349769in}}%
\pgfpathlineto{\pgfqpoint{1.110360in}{2.336066in}}%
\pgfpathlineto{\pgfqpoint{1.114080in}{2.336196in}}%
\pgfpathlineto{\pgfqpoint{1.119040in}{2.350196in}}%
\pgfpathlineto{\pgfqpoint{1.120280in}{2.351546in}}%
\pgfpathlineto{\pgfqpoint{1.122760in}{2.338828in}}%
\pgfpathlineto{\pgfqpoint{1.126480in}{2.318693in}}%
\pgfpathlineto{\pgfqpoint{1.128960in}{2.320081in}}%
\pgfpathlineto{\pgfqpoint{1.131440in}{2.318834in}}%
\pgfpathlineto{\pgfqpoint{1.133920in}{2.313658in}}%
\pgfpathlineto{\pgfqpoint{1.135160in}{2.313844in}}%
\pgfpathlineto{\pgfqpoint{1.137640in}{2.310168in}}%
\pgfpathlineto{\pgfqpoint{1.140120in}{2.318654in}}%
\pgfpathlineto{\pgfqpoint{1.141360in}{2.317341in}}%
\pgfpathlineto{\pgfqpoint{1.142600in}{2.313382in}}%
\pgfpathlineto{\pgfqpoint{1.143840in}{2.313862in}}%
\pgfpathlineto{\pgfqpoint{1.145080in}{2.313049in}}%
\pgfpathlineto{\pgfqpoint{1.146320in}{2.319518in}}%
\pgfpathlineto{\pgfqpoint{1.148800in}{2.316136in}}%
\pgfpathlineto{\pgfqpoint{1.152520in}{2.317824in}}%
\pgfpathlineto{\pgfqpoint{1.155000in}{2.317575in}}%
\pgfpathlineto{\pgfqpoint{1.157480in}{2.326168in}}%
\pgfpathlineto{\pgfqpoint{1.161200in}{2.322117in}}%
\pgfpathlineto{\pgfqpoint{1.162440in}{2.322872in}}%
\pgfpathlineto{\pgfqpoint{1.166160in}{2.313552in}}%
\pgfpathlineto{\pgfqpoint{1.167400in}{2.319502in}}%
\pgfpathlineto{\pgfqpoint{1.168640in}{2.319890in}}%
\pgfpathlineto{\pgfqpoint{1.169880in}{2.324360in}}%
\pgfpathlineto{\pgfqpoint{1.172360in}{2.320475in}}%
\pgfpathlineto{\pgfqpoint{1.174840in}{2.317252in}}%
\pgfpathlineto{\pgfqpoint{1.176080in}{2.318274in}}%
\pgfpathlineto{\pgfqpoint{1.177320in}{2.316643in}}%
\pgfpathlineto{\pgfqpoint{1.179800in}{2.307597in}}%
\pgfpathlineto{\pgfqpoint{1.182280in}{2.306550in}}%
\pgfpathlineto{\pgfqpoint{1.183520in}{2.300807in}}%
\pgfpathlineto{\pgfqpoint{1.188480in}{2.300667in}}%
\pgfpathlineto{\pgfqpoint{1.189720in}{2.303358in}}%
\pgfpathlineto{\pgfqpoint{1.190960in}{2.302236in}}%
\pgfpathlineto{\pgfqpoint{1.192200in}{2.307678in}}%
\pgfpathlineto{\pgfqpoint{1.193440in}{2.308019in}}%
\pgfpathlineto{\pgfqpoint{1.194680in}{2.310106in}}%
\pgfpathlineto{\pgfqpoint{1.195920in}{2.308182in}}%
\pgfpathlineto{\pgfqpoint{1.197160in}{2.308846in}}%
\pgfpathlineto{\pgfqpoint{1.199640in}{2.305614in}}%
\pgfpathlineto{\pgfqpoint{1.200880in}{2.308067in}}%
\pgfpathlineto{\pgfqpoint{1.204600in}{2.305431in}}%
\pgfpathlineto{\pgfqpoint{1.205840in}{2.305365in}}%
\pgfpathlineto{\pgfqpoint{1.207080in}{2.309776in}}%
\pgfpathlineto{\pgfqpoint{1.209560in}{2.309871in}}%
\pgfpathlineto{\pgfqpoint{1.213280in}{2.314245in}}%
\pgfpathlineto{\pgfqpoint{1.214520in}{2.311079in}}%
\pgfpathlineto{\pgfqpoint{1.217000in}{2.314273in}}%
\pgfpathlineto{\pgfqpoint{1.218240in}{2.317843in}}%
\pgfpathlineto{\pgfqpoint{1.219480in}{2.315390in}}%
\pgfpathlineto{\pgfqpoint{1.220720in}{2.316275in}}%
\pgfpathlineto{\pgfqpoint{1.223200in}{2.311060in}}%
\pgfpathlineto{\pgfqpoint{1.224440in}{2.305932in}}%
\pgfpathlineto{\pgfqpoint{1.225680in}{2.305726in}}%
\pgfpathlineto{\pgfqpoint{1.228160in}{2.302882in}}%
\pgfpathlineto{\pgfqpoint{1.229400in}{2.304260in}}%
\pgfpathlineto{\pgfqpoint{1.230640in}{2.303287in}}%
\pgfpathlineto{\pgfqpoint{1.234360in}{2.290159in}}%
\pgfpathlineto{\pgfqpoint{1.235600in}{2.290296in}}%
\pgfpathlineto{\pgfqpoint{1.236840in}{2.292902in}}%
\pgfpathlineto{\pgfqpoint{1.240560in}{2.288816in}}%
\pgfpathlineto{\pgfqpoint{1.244280in}{2.304620in}}%
\pgfpathlineto{\pgfqpoint{1.245520in}{2.303987in}}%
\pgfpathlineto{\pgfqpoint{1.250480in}{2.278648in}}%
\pgfpathlineto{\pgfqpoint{1.255440in}{2.280968in}}%
\pgfpathlineto{\pgfqpoint{1.256680in}{2.280021in}}%
\pgfpathlineto{\pgfqpoint{1.261640in}{2.284595in}}%
\pgfpathlineto{\pgfqpoint{1.264120in}{2.291752in}}%
\pgfpathlineto{\pgfqpoint{1.265360in}{2.290844in}}%
\pgfpathlineto{\pgfqpoint{1.269080in}{2.283878in}}%
\pgfpathlineto{\pgfqpoint{1.270320in}{2.286429in}}%
\pgfpathlineto{\pgfqpoint{1.274040in}{2.285313in}}%
\pgfpathlineto{\pgfqpoint{1.275280in}{2.287040in}}%
\pgfpathlineto{\pgfqpoint{1.277760in}{2.284347in}}%
\pgfpathlineto{\pgfqpoint{1.279000in}{2.280880in}}%
\pgfpathlineto{\pgfqpoint{1.280240in}{2.281798in}}%
\pgfpathlineto{\pgfqpoint{1.281480in}{2.279510in}}%
\pgfpathlineto{\pgfqpoint{1.283960in}{2.282009in}}%
\pgfpathlineto{\pgfqpoint{1.285200in}{2.281071in}}%
\pgfpathlineto{\pgfqpoint{1.286440in}{2.282045in}}%
\pgfpathlineto{\pgfqpoint{1.287680in}{2.279010in}}%
\pgfpathlineto{\pgfqpoint{1.290160in}{2.279953in}}%
\pgfpathlineto{\pgfqpoint{1.291400in}{2.283121in}}%
\pgfpathlineto{\pgfqpoint{1.292640in}{2.280699in}}%
\pgfpathlineto{\pgfqpoint{1.293880in}{2.282105in}}%
\pgfpathlineto{\pgfqpoint{1.297600in}{2.279070in}}%
\pgfpathlineto{\pgfqpoint{1.298840in}{2.279457in}}%
\pgfpathlineto{\pgfqpoint{1.300080in}{2.278653in}}%
\pgfpathlineto{\pgfqpoint{1.301320in}{2.280297in}}%
\pgfpathlineto{\pgfqpoint{1.302560in}{2.278812in}}%
\pgfpathlineto{\pgfqpoint{1.305040in}{2.281507in}}%
\pgfpathlineto{\pgfqpoint{1.306280in}{2.280056in}}%
\pgfpathlineto{\pgfqpoint{1.308760in}{2.271638in}}%
\pgfpathlineto{\pgfqpoint{1.312480in}{2.273465in}}%
\pgfpathlineto{\pgfqpoint{1.318680in}{2.288480in}}%
\pgfpathlineto{\pgfqpoint{1.319920in}{2.286124in}}%
\pgfpathlineto{\pgfqpoint{1.321160in}{2.287309in}}%
\pgfpathlineto{\pgfqpoint{1.323640in}{2.285012in}}%
\pgfpathlineto{\pgfqpoint{1.326120in}{2.285580in}}%
\pgfpathlineto{\pgfqpoint{1.327360in}{2.284993in}}%
\pgfpathlineto{\pgfqpoint{1.328600in}{2.281824in}}%
\pgfpathlineto{\pgfqpoint{1.329840in}{2.282710in}}%
\pgfpathlineto{\pgfqpoint{1.331080in}{2.285674in}}%
\pgfpathlineto{\pgfqpoint{1.332320in}{2.285703in}}%
\pgfpathlineto{\pgfqpoint{1.333560in}{2.283176in}}%
\pgfpathlineto{\pgfqpoint{1.337280in}{2.287219in}}%
\pgfpathlineto{\pgfqpoint{1.338520in}{2.284424in}}%
\pgfpathlineto{\pgfqpoint{1.342240in}{2.289083in}}%
\pgfpathlineto{\pgfqpoint{1.343480in}{2.285658in}}%
\pgfpathlineto{\pgfqpoint{1.344720in}{2.286230in}}%
\pgfpathlineto{\pgfqpoint{1.345960in}{2.285379in}}%
\pgfpathlineto{\pgfqpoint{1.347200in}{2.282807in}}%
\pgfpathlineto{\pgfqpoint{1.349680in}{2.274377in}}%
\pgfpathlineto{\pgfqpoint{1.352160in}{2.271246in}}%
\pgfpathlineto{\pgfqpoint{1.354640in}{2.273232in}}%
\pgfpathlineto{\pgfqpoint{1.357120in}{2.269674in}}%
\pgfpathlineto{\pgfqpoint{1.358360in}{2.265845in}}%
\pgfpathlineto{\pgfqpoint{1.359600in}{2.265972in}}%
\pgfpathlineto{\pgfqpoint{1.360840in}{2.270122in}}%
\pgfpathlineto{\pgfqpoint{1.362080in}{2.269218in}}%
\pgfpathlineto{\pgfqpoint{1.364560in}{2.265122in}}%
\pgfpathlineto{\pgfqpoint{1.368280in}{2.276588in}}%
\pgfpathlineto{\pgfqpoint{1.369520in}{2.276275in}}%
\pgfpathlineto{\pgfqpoint{1.373240in}{2.262031in}}%
\pgfpathlineto{\pgfqpoint{1.378200in}{2.263381in}}%
\pgfpathlineto{\pgfqpoint{1.381920in}{2.261384in}}%
\pgfpathlineto{\pgfqpoint{1.383160in}{2.263172in}}%
\pgfpathlineto{\pgfqpoint{1.384400in}{2.261875in}}%
\pgfpathlineto{\pgfqpoint{1.388120in}{2.268721in}}%
\pgfpathlineto{\pgfqpoint{1.389360in}{2.269067in}}%
\pgfpathlineto{\pgfqpoint{1.393080in}{2.263664in}}%
\pgfpathlineto{\pgfqpoint{1.394320in}{2.266109in}}%
\pgfpathlineto{\pgfqpoint{1.396800in}{2.260925in}}%
\pgfpathlineto{\pgfqpoint{1.400520in}{2.264380in}}%
\pgfpathlineto{\pgfqpoint{1.401760in}{2.263721in}}%
\pgfpathlineto{\pgfqpoint{1.403000in}{2.260468in}}%
\pgfpathlineto{\pgfqpoint{1.404240in}{2.262042in}}%
\pgfpathlineto{\pgfqpoint{1.405480in}{2.260060in}}%
\pgfpathlineto{\pgfqpoint{1.406720in}{2.262831in}}%
\pgfpathlineto{\pgfqpoint{1.407960in}{2.262876in}}%
\pgfpathlineto{\pgfqpoint{1.412920in}{2.255532in}}%
\pgfpathlineto{\pgfqpoint{1.415400in}{2.260563in}}%
\pgfpathlineto{\pgfqpoint{1.416640in}{2.257685in}}%
\pgfpathlineto{\pgfqpoint{1.417880in}{2.257687in}}%
\pgfpathlineto{\pgfqpoint{1.421600in}{2.253573in}}%
\pgfpathlineto{\pgfqpoint{1.424080in}{2.253691in}}%
\pgfpathlineto{\pgfqpoint{1.425320in}{2.255941in}}%
\pgfpathlineto{\pgfqpoint{1.426560in}{2.255525in}}%
\pgfpathlineto{\pgfqpoint{1.429040in}{2.258538in}}%
\pgfpathlineto{\pgfqpoint{1.430280in}{2.257580in}}%
\pgfpathlineto{\pgfqpoint{1.432760in}{2.251670in}}%
\pgfpathlineto{\pgfqpoint{1.434000in}{2.251274in}}%
\pgfpathlineto{\pgfqpoint{1.436480in}{2.249162in}}%
\pgfpathlineto{\pgfqpoint{1.440200in}{2.258528in}}%
\pgfpathlineto{\pgfqpoint{1.442680in}{2.263319in}}%
\pgfpathlineto{\pgfqpoint{1.445160in}{2.260546in}}%
\pgfpathlineto{\pgfqpoint{1.447640in}{2.260015in}}%
\pgfpathlineto{\pgfqpoint{1.450120in}{2.259719in}}%
\pgfpathlineto{\pgfqpoint{1.453840in}{2.255201in}}%
\pgfpathlineto{\pgfqpoint{1.455080in}{2.256877in}}%
\pgfpathlineto{\pgfqpoint{1.456320in}{2.255894in}}%
\pgfpathlineto{\pgfqpoint{1.457560in}{2.253427in}}%
\pgfpathlineto{\pgfqpoint{1.461280in}{2.255667in}}%
\pgfpathlineto{\pgfqpoint{1.462520in}{2.251531in}}%
\pgfpathlineto{\pgfqpoint{1.466240in}{2.255280in}}%
\pgfpathlineto{\pgfqpoint{1.468720in}{2.250658in}}%
\pgfpathlineto{\pgfqpoint{1.469960in}{2.250760in}}%
\pgfpathlineto{\pgfqpoint{1.476160in}{2.241647in}}%
\pgfpathlineto{\pgfqpoint{1.477400in}{2.242148in}}%
\pgfpathlineto{\pgfqpoint{1.478640in}{2.244070in}}%
\pgfpathlineto{\pgfqpoint{1.481120in}{2.241417in}}%
\pgfpathlineto{\pgfqpoint{1.482360in}{2.237674in}}%
\pgfpathlineto{\pgfqpoint{1.483600in}{2.239162in}}%
\pgfpathlineto{\pgfqpoint{1.486080in}{2.243757in}}%
\pgfpathlineto{\pgfqpoint{1.488560in}{2.238881in}}%
\pgfpathlineto{\pgfqpoint{1.491040in}{2.243300in}}%
\pgfpathlineto{\pgfqpoint{1.492280in}{2.248274in}}%
\pgfpathlineto{\pgfqpoint{1.493520in}{2.246162in}}%
\pgfpathlineto{\pgfqpoint{1.497240in}{2.230735in}}%
\pgfpathlineto{\pgfqpoint{1.499720in}{2.233367in}}%
\pgfpathlineto{\pgfqpoint{1.500960in}{2.233069in}}%
\pgfpathlineto{\pgfqpoint{1.503440in}{2.230026in}}%
\pgfpathlineto{\pgfqpoint{1.505920in}{2.231536in}}%
\pgfpathlineto{\pgfqpoint{1.507160in}{2.233655in}}%
\pgfpathlineto{\pgfqpoint{1.508400in}{2.231539in}}%
\pgfpathlineto{\pgfqpoint{1.509640in}{2.232411in}}%
\pgfpathlineto{\pgfqpoint{1.512120in}{2.234796in}}%
\pgfpathlineto{\pgfqpoint{1.513360in}{2.234802in}}%
\pgfpathlineto{\pgfqpoint{1.515840in}{2.230353in}}%
\pgfpathlineto{\pgfqpoint{1.517080in}{2.230494in}}%
\pgfpathlineto{\pgfqpoint{1.518320in}{2.232644in}}%
\pgfpathlineto{\pgfqpoint{1.520800in}{2.228173in}}%
\pgfpathlineto{\pgfqpoint{1.523280in}{2.230305in}}%
\pgfpathlineto{\pgfqpoint{1.525760in}{2.230314in}}%
\pgfpathlineto{\pgfqpoint{1.527000in}{2.228037in}}%
\pgfpathlineto{\pgfqpoint{1.528240in}{2.229715in}}%
\pgfpathlineto{\pgfqpoint{1.529480in}{2.229271in}}%
\pgfpathlineto{\pgfqpoint{1.531960in}{2.231993in}}%
\pgfpathlineto{\pgfqpoint{1.533200in}{2.231040in}}%
\pgfpathlineto{\pgfqpoint{1.535680in}{2.227637in}}%
\pgfpathlineto{\pgfqpoint{1.536920in}{2.227109in}}%
\pgfpathlineto{\pgfqpoint{1.538160in}{2.227873in}}%
\pgfpathlineto{\pgfqpoint{1.539400in}{2.230327in}}%
\pgfpathlineto{\pgfqpoint{1.540640in}{2.227941in}}%
\pgfpathlineto{\pgfqpoint{1.541880in}{2.228841in}}%
\pgfpathlineto{\pgfqpoint{1.544360in}{2.226399in}}%
\pgfpathlineto{\pgfqpoint{1.546840in}{2.227249in}}%
\pgfpathlineto{\pgfqpoint{1.548080in}{2.227905in}}%
\pgfpathlineto{\pgfqpoint{1.549320in}{2.230385in}}%
\pgfpathlineto{\pgfqpoint{1.550560in}{2.229606in}}%
\pgfpathlineto{\pgfqpoint{1.553040in}{2.231106in}}%
\pgfpathlineto{\pgfqpoint{1.559240in}{2.224025in}}%
\pgfpathlineto{\pgfqpoint{1.560480in}{2.225247in}}%
\pgfpathlineto{\pgfqpoint{1.564200in}{2.235311in}}%
\pgfpathlineto{\pgfqpoint{1.566680in}{2.240199in}}%
\pgfpathlineto{\pgfqpoint{1.569160in}{2.234425in}}%
\pgfpathlineto{\pgfqpoint{1.574120in}{2.232950in}}%
\pgfpathlineto{\pgfqpoint{1.577840in}{2.228595in}}%
\pgfpathlineto{\pgfqpoint{1.580320in}{2.231513in}}%
\pgfpathlineto{\pgfqpoint{1.581560in}{2.228939in}}%
\pgfpathlineto{\pgfqpoint{1.585280in}{2.231563in}}%
\pgfpathlineto{\pgfqpoint{1.586520in}{2.228089in}}%
\pgfpathlineto{\pgfqpoint{1.589000in}{2.230225in}}%
\pgfpathlineto{\pgfqpoint{1.590240in}{2.232789in}}%
\pgfpathlineto{\pgfqpoint{1.592720in}{2.228404in}}%
\pgfpathlineto{\pgfqpoint{1.593960in}{2.229375in}}%
\pgfpathlineto{\pgfqpoint{1.595200in}{2.228873in}}%
\pgfpathlineto{\pgfqpoint{1.597680in}{2.223035in}}%
\pgfpathlineto{\pgfqpoint{1.598920in}{2.221218in}}%
\pgfpathlineto{\pgfqpoint{1.601400in}{2.221884in}}%
\pgfpathlineto{\pgfqpoint{1.602640in}{2.223452in}}%
\pgfpathlineto{\pgfqpoint{1.603880in}{2.222842in}}%
\pgfpathlineto{\pgfqpoint{1.606360in}{2.217473in}}%
\pgfpathlineto{\pgfqpoint{1.607600in}{2.218974in}}%
\pgfpathlineto{\pgfqpoint{1.610080in}{2.222843in}}%
\pgfpathlineto{\pgfqpoint{1.612560in}{2.216289in}}%
\pgfpathlineto{\pgfqpoint{1.613800in}{2.217169in}}%
\pgfpathlineto{\pgfqpoint{1.616280in}{2.223824in}}%
\pgfpathlineto{\pgfqpoint{1.617520in}{2.221623in}}%
\pgfpathlineto{\pgfqpoint{1.620000in}{2.211689in}}%
\pgfpathlineto{\pgfqpoint{1.624960in}{2.212819in}}%
\pgfpathlineto{\pgfqpoint{1.628680in}{2.207176in}}%
\pgfpathlineto{\pgfqpoint{1.629920in}{2.207297in}}%
\pgfpathlineto{\pgfqpoint{1.631160in}{2.209210in}}%
\pgfpathlineto{\pgfqpoint{1.632400in}{2.207476in}}%
\pgfpathlineto{\pgfqpoint{1.636120in}{2.211188in}}%
\pgfpathlineto{\pgfqpoint{1.637360in}{2.211643in}}%
\pgfpathlineto{\pgfqpoint{1.639840in}{2.207181in}}%
\pgfpathlineto{\pgfqpoint{1.641080in}{2.206517in}}%
\pgfpathlineto{\pgfqpoint{1.642320in}{2.208639in}}%
\pgfpathlineto{\pgfqpoint{1.646040in}{2.203848in}}%
\pgfpathlineto{\pgfqpoint{1.648520in}{2.203976in}}%
\pgfpathlineto{\pgfqpoint{1.649760in}{2.204911in}}%
\pgfpathlineto{\pgfqpoint{1.651000in}{2.203649in}}%
\pgfpathlineto{\pgfqpoint{1.655960in}{2.210335in}}%
\pgfpathlineto{\pgfqpoint{1.658440in}{2.206081in}}%
\pgfpathlineto{\pgfqpoint{1.660920in}{2.206627in}}%
\pgfpathlineto{\pgfqpoint{1.662160in}{2.207131in}}%
\pgfpathlineto{\pgfqpoint{1.663400in}{2.209592in}}%
\pgfpathlineto{\pgfqpoint{1.664640in}{2.207922in}}%
\pgfpathlineto{\pgfqpoint{1.665880in}{2.208556in}}%
\pgfpathlineto{\pgfqpoint{1.668360in}{2.205719in}}%
\pgfpathlineto{\pgfqpoint{1.677040in}{2.212013in}}%
\pgfpathlineto{\pgfqpoint{1.683240in}{2.205694in}}%
\pgfpathlineto{\pgfqpoint{1.684480in}{2.206766in}}%
\pgfpathlineto{\pgfqpoint{1.688200in}{2.215658in}}%
\pgfpathlineto{\pgfqpoint{1.690680in}{2.219720in}}%
\pgfpathlineto{\pgfqpoint{1.694400in}{2.213444in}}%
\pgfpathlineto{\pgfqpoint{1.700600in}{2.212812in}}%
\pgfpathlineto{\pgfqpoint{1.701840in}{2.211243in}}%
\pgfpathlineto{\pgfqpoint{1.704320in}{2.212851in}}%
\pgfpathlineto{\pgfqpoint{1.705560in}{2.210871in}}%
\pgfpathlineto{\pgfqpoint{1.709280in}{2.214999in}}%
\pgfpathlineto{\pgfqpoint{1.710520in}{2.211834in}}%
\pgfpathlineto{\pgfqpoint{1.713000in}{2.213763in}}%
\pgfpathlineto{\pgfqpoint{1.714240in}{2.216505in}}%
\pgfpathlineto{\pgfqpoint{1.715480in}{2.213683in}}%
\pgfpathlineto{\pgfqpoint{1.719200in}{2.215201in}}%
\pgfpathlineto{\pgfqpoint{1.721680in}{2.210181in}}%
\pgfpathlineto{\pgfqpoint{1.724160in}{2.208196in}}%
\pgfpathlineto{\pgfqpoint{1.725400in}{2.208335in}}%
\pgfpathlineto{\pgfqpoint{1.727880in}{2.210625in}}%
\pgfpathlineto{\pgfqpoint{1.731600in}{2.208775in}}%
\pgfpathlineto{\pgfqpoint{1.734080in}{2.211902in}}%
\pgfpathlineto{\pgfqpoint{1.736560in}{2.206925in}}%
\pgfpathlineto{\pgfqpoint{1.737800in}{2.208405in}}%
\pgfpathlineto{\pgfqpoint{1.740280in}{2.215520in}}%
\pgfpathlineto{\pgfqpoint{1.741520in}{2.214048in}}%
\pgfpathlineto{\pgfqpoint{1.745240in}{2.199291in}}%
\pgfpathlineto{\pgfqpoint{1.747720in}{2.201051in}}%
\pgfpathlineto{\pgfqpoint{1.750200in}{2.197314in}}%
\pgfpathlineto{\pgfqpoint{1.752680in}{2.193893in}}%
\pgfpathlineto{\pgfqpoint{1.753920in}{2.192736in}}%
\pgfpathlineto{\pgfqpoint{1.755160in}{2.194598in}}%
\pgfpathlineto{\pgfqpoint{1.756400in}{2.193410in}}%
\pgfpathlineto{\pgfqpoint{1.758880in}{2.197248in}}%
\pgfpathlineto{\pgfqpoint{1.761360in}{2.196635in}}%
\pgfpathlineto{\pgfqpoint{1.763840in}{2.191044in}}%
\pgfpathlineto{\pgfqpoint{1.765080in}{2.190968in}}%
\pgfpathlineto{\pgfqpoint{1.766320in}{2.192570in}}%
\pgfpathlineto{\pgfqpoint{1.770040in}{2.187559in}}%
\pgfpathlineto{\pgfqpoint{1.775000in}{2.188741in}}%
\pgfpathlineto{\pgfqpoint{1.778720in}{2.194654in}}%
\pgfpathlineto{\pgfqpoint{1.779960in}{2.196289in}}%
\pgfpathlineto{\pgfqpoint{1.783680in}{2.192152in}}%
\pgfpathlineto{\pgfqpoint{1.786160in}{2.191804in}}%
\pgfpathlineto{\pgfqpoint{1.787400in}{2.194652in}}%
\pgfpathlineto{\pgfqpoint{1.788640in}{2.192812in}}%
\pgfpathlineto{\pgfqpoint{1.789880in}{2.193647in}}%
\pgfpathlineto{\pgfqpoint{1.792360in}{2.191869in}}%
\pgfpathlineto{\pgfqpoint{1.799800in}{2.200074in}}%
\pgfpathlineto{\pgfqpoint{1.801040in}{2.201240in}}%
\pgfpathlineto{\pgfqpoint{1.808480in}{2.193413in}}%
\pgfpathlineto{\pgfqpoint{1.812200in}{2.200549in}}%
\pgfpathlineto{\pgfqpoint{1.814680in}{2.205491in}}%
\pgfpathlineto{\pgfqpoint{1.817160in}{2.202205in}}%
\pgfpathlineto{\pgfqpoint{1.820880in}{2.201362in}}%
\pgfpathlineto{\pgfqpoint{1.822120in}{2.202198in}}%
\pgfpathlineto{\pgfqpoint{1.825840in}{2.197172in}}%
\pgfpathlineto{\pgfqpoint{1.827080in}{2.197154in}}%
\pgfpathlineto{\pgfqpoint{1.828320in}{2.198627in}}%
\pgfpathlineto{\pgfqpoint{1.829560in}{2.197089in}}%
\pgfpathlineto{\pgfqpoint{1.833280in}{2.200586in}}%
\pgfpathlineto{\pgfqpoint{1.834520in}{2.198284in}}%
\pgfpathlineto{\pgfqpoint{1.837000in}{2.199997in}}%
\pgfpathlineto{\pgfqpoint{1.838240in}{2.202185in}}%
\pgfpathlineto{\pgfqpoint{1.839480in}{2.200062in}}%
\pgfpathlineto{\pgfqpoint{1.841960in}{2.203288in}}%
\pgfpathlineto{\pgfqpoint{1.843200in}{2.203124in}}%
\pgfpathlineto{\pgfqpoint{1.846920in}{2.196230in}}%
\pgfpathlineto{\pgfqpoint{1.849400in}{2.195411in}}%
\pgfpathlineto{\pgfqpoint{1.851880in}{2.197798in}}%
\pgfpathlineto{\pgfqpoint{1.853120in}{2.197418in}}%
\pgfpathlineto{\pgfqpoint{1.854360in}{2.195715in}}%
\pgfpathlineto{\pgfqpoint{1.855600in}{2.196332in}}%
\pgfpathlineto{\pgfqpoint{1.858080in}{2.198848in}}%
\pgfpathlineto{\pgfqpoint{1.860560in}{2.192713in}}%
\pgfpathlineto{\pgfqpoint{1.861800in}{2.193359in}}%
\pgfpathlineto{\pgfqpoint{1.864280in}{2.198857in}}%
\pgfpathlineto{\pgfqpoint{1.865520in}{2.197887in}}%
\pgfpathlineto{\pgfqpoint{1.869240in}{2.184165in}}%
\pgfpathlineto{\pgfqpoint{1.872960in}{2.184480in}}%
\pgfpathlineto{\pgfqpoint{1.875440in}{2.180576in}}%
\pgfpathlineto{\pgfqpoint{1.877920in}{2.180502in}}%
\pgfpathlineto{\pgfqpoint{1.879160in}{2.182338in}}%
\pgfpathlineto{\pgfqpoint{1.880400in}{2.181714in}}%
\pgfpathlineto{\pgfqpoint{1.882880in}{2.185643in}}%
\pgfpathlineto{\pgfqpoint{1.885360in}{2.184306in}}%
\pgfpathlineto{\pgfqpoint{1.887840in}{2.178690in}}%
\pgfpathlineto{\pgfqpoint{1.889080in}{2.178324in}}%
\pgfpathlineto{\pgfqpoint{1.890320in}{2.180804in}}%
\pgfpathlineto{\pgfqpoint{1.894040in}{2.176278in}}%
\pgfpathlineto{\pgfqpoint{1.897760in}{2.178480in}}%
\pgfpathlineto{\pgfqpoint{1.899000in}{2.177660in}}%
\pgfpathlineto{\pgfqpoint{1.903960in}{2.183123in}}%
\pgfpathlineto{\pgfqpoint{1.908920in}{2.176614in}}%
\pgfpathlineto{\pgfqpoint{1.910160in}{2.176852in}}%
\pgfpathlineto{\pgfqpoint{1.911400in}{2.179658in}}%
\pgfpathlineto{\pgfqpoint{1.912640in}{2.178961in}}%
\pgfpathlineto{\pgfqpoint{1.913880in}{2.180560in}}%
\pgfpathlineto{\pgfqpoint{1.917600in}{2.180272in}}%
\pgfpathlineto{\pgfqpoint{1.918840in}{2.181189in}}%
\pgfpathlineto{\pgfqpoint{1.921320in}{2.185276in}}%
\pgfpathlineto{\pgfqpoint{1.926280in}{2.185917in}}%
\pgfpathlineto{\pgfqpoint{1.928760in}{2.183008in}}%
\pgfpathlineto{\pgfqpoint{1.931240in}{2.179093in}}%
\pgfpathlineto{\pgfqpoint{1.932480in}{2.179667in}}%
\pgfpathlineto{\pgfqpoint{1.936200in}{2.186050in}}%
\pgfpathlineto{\pgfqpoint{1.938680in}{2.190761in}}%
\pgfpathlineto{\pgfqpoint{1.941160in}{2.187127in}}%
\pgfpathlineto{\pgfqpoint{1.946120in}{2.185556in}}%
\pgfpathlineto{\pgfqpoint{1.948600in}{2.182899in}}%
\pgfpathlineto{\pgfqpoint{1.951080in}{2.181756in}}%
\pgfpathlineto{\pgfqpoint{1.952320in}{2.182828in}}%
\pgfpathlineto{\pgfqpoint{1.953560in}{2.181321in}}%
\pgfpathlineto{\pgfqpoint{1.957280in}{2.184924in}}%
\pgfpathlineto{\pgfqpoint{1.958520in}{2.182416in}}%
\pgfpathlineto{\pgfqpoint{1.961000in}{2.185060in}}%
\pgfpathlineto{\pgfqpoint{1.962240in}{2.187346in}}%
\pgfpathlineto{\pgfqpoint{1.963480in}{2.185022in}}%
\pgfpathlineto{\pgfqpoint{1.967200in}{2.188587in}}%
\pgfpathlineto{\pgfqpoint{1.969680in}{2.183460in}}%
\pgfpathlineto{\pgfqpoint{1.972160in}{2.181561in}}%
\pgfpathlineto{\pgfqpoint{1.973400in}{2.181011in}}%
\pgfpathlineto{\pgfqpoint{1.975880in}{2.182762in}}%
\pgfpathlineto{\pgfqpoint{1.979600in}{2.181893in}}%
\pgfpathlineto{\pgfqpoint{1.980840in}{2.184769in}}%
\pgfpathlineto{\pgfqpoint{1.982080in}{2.184626in}}%
\pgfpathlineto{\pgfqpoint{1.984560in}{2.180880in}}%
\pgfpathlineto{\pgfqpoint{1.985800in}{2.181577in}}%
\pgfpathlineto{\pgfqpoint{1.988280in}{2.187091in}}%
\pgfpathlineto{\pgfqpoint{1.989520in}{2.185153in}}%
\pgfpathlineto{\pgfqpoint{1.993240in}{2.173295in}}%
\pgfpathlineto{\pgfqpoint{1.995720in}{2.173726in}}%
\pgfpathlineto{\pgfqpoint{1.999440in}{2.171647in}}%
\pgfpathlineto{\pgfqpoint{2.001920in}{2.171527in}}%
\pgfpathlineto{\pgfqpoint{2.003160in}{2.173724in}}%
\pgfpathlineto{\pgfqpoint{2.004400in}{2.172970in}}%
\pgfpathlineto{\pgfqpoint{2.006880in}{2.176654in}}%
\pgfpathlineto{\pgfqpoint{2.009360in}{2.175007in}}%
\pgfpathlineto{\pgfqpoint{2.011840in}{2.169060in}}%
\pgfpathlineto{\pgfqpoint{2.013080in}{2.168902in}}%
\pgfpathlineto{\pgfqpoint{2.014320in}{2.171065in}}%
\pgfpathlineto{\pgfqpoint{2.018040in}{2.167417in}}%
\pgfpathlineto{\pgfqpoint{2.021760in}{2.169862in}}%
\pgfpathlineto{\pgfqpoint{2.023000in}{2.170082in}}%
\pgfpathlineto{\pgfqpoint{2.027960in}{2.175811in}}%
\pgfpathlineto{\pgfqpoint{2.032920in}{2.170591in}}%
\pgfpathlineto{\pgfqpoint{2.034160in}{2.170709in}}%
\pgfpathlineto{\pgfqpoint{2.035400in}{2.173869in}}%
\pgfpathlineto{\pgfqpoint{2.036640in}{2.173920in}}%
\pgfpathlineto{\pgfqpoint{2.039120in}{2.176348in}}%
\pgfpathlineto{\pgfqpoint{2.041600in}{2.177333in}}%
\pgfpathlineto{\pgfqpoint{2.049040in}{2.182945in}}%
\pgfpathlineto{\pgfqpoint{2.051520in}{2.181606in}}%
\pgfpathlineto{\pgfqpoint{2.055240in}{2.177476in}}%
\pgfpathlineto{\pgfqpoint{2.056480in}{2.178618in}}%
\pgfpathlineto{\pgfqpoint{2.058960in}{2.183418in}}%
\pgfpathlineto{\pgfqpoint{2.060200in}{2.184015in}}%
\pgfpathlineto{\pgfqpoint{2.062680in}{2.188641in}}%
\pgfpathlineto{\pgfqpoint{2.065160in}{2.185049in}}%
\pgfpathlineto{\pgfqpoint{2.067640in}{2.185158in}}%
\pgfpathlineto{\pgfqpoint{2.072600in}{2.180065in}}%
\pgfpathlineto{\pgfqpoint{2.073840in}{2.179532in}}%
\pgfpathlineto{\pgfqpoint{2.076320in}{2.181780in}}%
\pgfpathlineto{\pgfqpoint{2.077560in}{2.180412in}}%
\pgfpathlineto{\pgfqpoint{2.081280in}{2.182321in}}%
\pgfpathlineto{\pgfqpoint{2.083760in}{2.179931in}}%
\pgfpathlineto{\pgfqpoint{2.086240in}{2.183200in}}%
\pgfpathlineto{\pgfqpoint{2.087480in}{2.181109in}}%
\pgfpathlineto{\pgfqpoint{2.091200in}{2.185933in}}%
\pgfpathlineto{\pgfqpoint{2.093680in}{2.181911in}}%
\pgfpathlineto{\pgfqpoint{2.096160in}{2.180359in}}%
\pgfpathlineto{\pgfqpoint{2.097400in}{2.179896in}}%
\pgfpathlineto{\pgfqpoint{2.101120in}{2.181597in}}%
\pgfpathlineto{\pgfqpoint{2.102360in}{2.179408in}}%
\pgfpathlineto{\pgfqpoint{2.103600in}{2.179799in}}%
\pgfpathlineto{\pgfqpoint{2.104840in}{2.182297in}}%
\pgfpathlineto{\pgfqpoint{2.106080in}{2.181847in}}%
\pgfpathlineto{\pgfqpoint{2.108560in}{2.178735in}}%
\pgfpathlineto{\pgfqpoint{2.109800in}{2.178603in}}%
\pgfpathlineto{\pgfqpoint{2.112280in}{2.183660in}}%
\pgfpathlineto{\pgfqpoint{2.113520in}{2.181941in}}%
\pgfpathlineto{\pgfqpoint{2.116000in}{2.173849in}}%
\pgfpathlineto{\pgfqpoint{2.119720in}{2.173543in}}%
\pgfpathlineto{\pgfqpoint{2.124680in}{2.171789in}}%
\pgfpathlineto{\pgfqpoint{2.125920in}{2.172067in}}%
\pgfpathlineto{\pgfqpoint{2.127160in}{2.174523in}}%
\pgfpathlineto{\pgfqpoint{2.128400in}{2.174205in}}%
\pgfpathlineto{\pgfqpoint{2.132120in}{2.177764in}}%
\pgfpathlineto{\pgfqpoint{2.133360in}{2.175650in}}%
\pgfpathlineto{\pgfqpoint{2.134600in}{2.170912in}}%
\pgfpathlineto{\pgfqpoint{2.137080in}{2.170752in}}%
\pgfpathlineto{\pgfqpoint{2.138320in}{2.172192in}}%
\pgfpathlineto{\pgfqpoint{2.142040in}{2.168542in}}%
\pgfpathlineto{\pgfqpoint{2.145760in}{2.171347in}}%
\pgfpathlineto{\pgfqpoint{2.149480in}{2.174066in}}%
\pgfpathlineto{\pgfqpoint{2.151960in}{2.177730in}}%
\pgfpathlineto{\pgfqpoint{2.156920in}{2.171662in}}%
\pgfpathlineto{\pgfqpoint{2.158160in}{2.172065in}}%
\pgfpathlineto{\pgfqpoint{2.160640in}{2.174720in}}%
\pgfpathlineto{\pgfqpoint{2.161880in}{2.176945in}}%
\pgfpathlineto{\pgfqpoint{2.164360in}{2.176676in}}%
\pgfpathlineto{\pgfqpoint{2.166840in}{2.178855in}}%
\pgfpathlineto{\pgfqpoint{2.169320in}{2.181729in}}%
\pgfpathlineto{\pgfqpoint{2.171800in}{2.181726in}}%
\pgfpathlineto{\pgfqpoint{2.175520in}{2.181924in}}%
\pgfpathlineto{\pgfqpoint{2.179240in}{2.176526in}}%
\pgfpathlineto{\pgfqpoint{2.186680in}{2.186997in}}%
\pgfpathlineto{\pgfqpoint{2.189160in}{2.184918in}}%
\pgfpathlineto{\pgfqpoint{2.191640in}{2.184541in}}%
\pgfpathlineto{\pgfqpoint{2.195360in}{2.179144in}}%
\pgfpathlineto{\pgfqpoint{2.199080in}{2.178574in}}%
\pgfpathlineto{\pgfqpoint{2.200320in}{2.179953in}}%
\pgfpathlineto{\pgfqpoint{2.201560in}{2.178385in}}%
\pgfpathlineto{\pgfqpoint{2.205280in}{2.179175in}}%
\pgfpathlineto{\pgfqpoint{2.207760in}{2.176729in}}%
\pgfpathlineto{\pgfqpoint{2.210240in}{2.180222in}}%
\pgfpathlineto{\pgfqpoint{2.211480in}{2.178391in}}%
\pgfpathlineto{\pgfqpoint{2.215200in}{2.183418in}}%
\pgfpathlineto{\pgfqpoint{2.218920in}{2.178255in}}%
\pgfpathlineto{\pgfqpoint{2.221400in}{2.177501in}}%
\pgfpathlineto{\pgfqpoint{2.225120in}{2.180202in}}%
\pgfpathlineto{\pgfqpoint{2.226360in}{2.177890in}}%
\pgfpathlineto{\pgfqpoint{2.227600in}{2.178663in}}%
\pgfpathlineto{\pgfqpoint{2.228840in}{2.180794in}}%
\pgfpathlineto{\pgfqpoint{2.230080in}{2.180506in}}%
\pgfpathlineto{\pgfqpoint{2.232560in}{2.177449in}}%
\pgfpathlineto{\pgfqpoint{2.233800in}{2.177155in}}%
\pgfpathlineto{\pgfqpoint{2.236280in}{2.183059in}}%
\pgfpathlineto{\pgfqpoint{2.237520in}{2.181326in}}%
\pgfpathlineto{\pgfqpoint{2.240000in}{2.172883in}}%
\pgfpathlineto{\pgfqpoint{2.241240in}{2.171725in}}%
\pgfpathlineto{\pgfqpoint{2.243720in}{2.171659in}}%
\pgfpathlineto{\pgfqpoint{2.247440in}{2.170567in}}%
\pgfpathlineto{\pgfqpoint{2.256120in}{2.177785in}}%
\pgfpathlineto{\pgfqpoint{2.257360in}{2.175768in}}%
\pgfpathlineto{\pgfqpoint{2.258600in}{2.170802in}}%
\pgfpathlineto{\pgfqpoint{2.262320in}{2.172190in}}%
\pgfpathlineto{\pgfqpoint{2.266040in}{2.168461in}}%
\pgfpathlineto{\pgfqpoint{2.274720in}{2.177011in}}%
\pgfpathlineto{\pgfqpoint{2.275960in}{2.178612in}}%
\pgfpathlineto{\pgfqpoint{2.282160in}{2.173570in}}%
\pgfpathlineto{\pgfqpoint{2.285880in}{2.180515in}}%
\pgfpathlineto{\pgfqpoint{2.288360in}{2.179607in}}%
\pgfpathlineto{\pgfqpoint{2.290840in}{2.181999in}}%
\pgfpathlineto{\pgfqpoint{2.293320in}{2.185426in}}%
\pgfpathlineto{\pgfqpoint{2.294560in}{2.185526in}}%
\pgfpathlineto{\pgfqpoint{2.297040in}{2.187350in}}%
\pgfpathlineto{\pgfqpoint{2.298280in}{2.187695in}}%
\pgfpathlineto{\pgfqpoint{2.302000in}{2.182662in}}%
\pgfpathlineto{\pgfqpoint{2.303240in}{2.180237in}}%
\pgfpathlineto{\pgfqpoint{2.305720in}{2.183684in}}%
\pgfpathlineto{\pgfqpoint{2.309440in}{2.188207in}}%
\pgfpathlineto{\pgfqpoint{2.310680in}{2.189892in}}%
\pgfpathlineto{\pgfqpoint{2.313160in}{2.188249in}}%
\pgfpathlineto{\pgfqpoint{2.315640in}{2.189182in}}%
\pgfpathlineto{\pgfqpoint{2.320600in}{2.183219in}}%
\pgfpathlineto{\pgfqpoint{2.321840in}{2.182355in}}%
\pgfpathlineto{\pgfqpoint{2.324320in}{2.185449in}}%
\pgfpathlineto{\pgfqpoint{2.325560in}{2.184052in}}%
\pgfpathlineto{\pgfqpoint{2.328040in}{2.185255in}}%
\pgfpathlineto{\pgfqpoint{2.329280in}{2.184666in}}%
\pgfpathlineto{\pgfqpoint{2.330520in}{2.182096in}}%
\pgfpathlineto{\pgfqpoint{2.331760in}{2.182688in}}%
\pgfpathlineto{\pgfqpoint{2.334240in}{2.187049in}}%
\pgfpathlineto{\pgfqpoint{2.335480in}{2.185912in}}%
\pgfpathlineto{\pgfqpoint{2.339200in}{2.190520in}}%
\pgfpathlineto{\pgfqpoint{2.342920in}{2.185498in}}%
\pgfpathlineto{\pgfqpoint{2.346640in}{2.185689in}}%
\pgfpathlineto{\pgfqpoint{2.347880in}{2.187301in}}%
\pgfpathlineto{\pgfqpoint{2.349120in}{2.186480in}}%
\pgfpathlineto{\pgfqpoint{2.350360in}{2.183775in}}%
\pgfpathlineto{\pgfqpoint{2.354080in}{2.185290in}}%
\pgfpathlineto{\pgfqpoint{2.356560in}{2.182882in}}%
\pgfpathlineto{\pgfqpoint{2.357800in}{2.182658in}}%
\pgfpathlineto{\pgfqpoint{2.360280in}{2.188080in}}%
\pgfpathlineto{\pgfqpoint{2.361520in}{2.186037in}}%
\pgfpathlineto{\pgfqpoint{2.365240in}{2.174856in}}%
\pgfpathlineto{\pgfqpoint{2.367720in}{2.174480in}}%
\pgfpathlineto{\pgfqpoint{2.371440in}{2.174398in}}%
\pgfpathlineto{\pgfqpoint{2.373920in}{2.175999in}}%
\pgfpathlineto{\pgfqpoint{2.375160in}{2.179459in}}%
\pgfpathlineto{\pgfqpoint{2.376400in}{2.179530in}}%
\pgfpathlineto{\pgfqpoint{2.380120in}{2.182639in}}%
\pgfpathlineto{\pgfqpoint{2.382600in}{2.175771in}}%
\pgfpathlineto{\pgfqpoint{2.383840in}{2.176858in}}%
\pgfpathlineto{\pgfqpoint{2.386320in}{2.178153in}}%
\pgfpathlineto{\pgfqpoint{2.390040in}{2.173196in}}%
\pgfpathlineto{\pgfqpoint{2.397480in}{2.179795in}}%
\pgfpathlineto{\pgfqpoint{2.399960in}{2.183774in}}%
\pgfpathlineto{\pgfqpoint{2.406160in}{2.178075in}}%
\pgfpathlineto{\pgfqpoint{2.409880in}{2.184023in}}%
\pgfpathlineto{\pgfqpoint{2.412360in}{2.182651in}}%
\pgfpathlineto{\pgfqpoint{2.414840in}{2.184601in}}%
\pgfpathlineto{\pgfqpoint{2.416080in}{2.187799in}}%
\pgfpathlineto{\pgfqpoint{2.421040in}{2.187990in}}%
\pgfpathlineto{\pgfqpoint{2.422280in}{2.188024in}}%
\pgfpathlineto{\pgfqpoint{2.428480in}{2.182810in}}%
\pgfpathlineto{\pgfqpoint{2.434680in}{2.192511in}}%
\pgfpathlineto{\pgfqpoint{2.437160in}{2.190700in}}%
\pgfpathlineto{\pgfqpoint{2.439640in}{2.191513in}}%
\pgfpathlineto{\pgfqpoint{2.443360in}{2.186215in}}%
\pgfpathlineto{\pgfqpoint{2.445840in}{2.184880in}}%
\pgfpathlineto{\pgfqpoint{2.448320in}{2.186550in}}%
\pgfpathlineto{\pgfqpoint{2.449560in}{2.185130in}}%
\pgfpathlineto{\pgfqpoint{2.452040in}{2.186007in}}%
\pgfpathlineto{\pgfqpoint{2.453280in}{2.185432in}}%
\pgfpathlineto{\pgfqpoint{2.454520in}{2.183169in}}%
\pgfpathlineto{\pgfqpoint{2.455760in}{2.184151in}}%
\pgfpathlineto{\pgfqpoint{2.458240in}{2.188333in}}%
\pgfpathlineto{\pgfqpoint{2.459480in}{2.187406in}}%
\pgfpathlineto{\pgfqpoint{2.463200in}{2.192428in}}%
\pgfpathlineto{\pgfqpoint{2.465680in}{2.190460in}}%
\pgfpathlineto{\pgfqpoint{2.468160in}{2.187763in}}%
\pgfpathlineto{\pgfqpoint{2.470640in}{2.187191in}}%
\pgfpathlineto{\pgfqpoint{2.471880in}{2.188683in}}%
\pgfpathlineto{\pgfqpoint{2.473120in}{2.187851in}}%
\pgfpathlineto{\pgfqpoint{2.474360in}{2.185468in}}%
\pgfpathlineto{\pgfqpoint{2.478080in}{2.187174in}}%
\pgfpathlineto{\pgfqpoint{2.480560in}{2.184374in}}%
\pgfpathlineto{\pgfqpoint{2.481800in}{2.183707in}}%
\pgfpathlineto{\pgfqpoint{2.484280in}{2.188813in}}%
\pgfpathlineto{\pgfqpoint{2.485520in}{2.186906in}}%
\pgfpathlineto{\pgfqpoint{2.488000in}{2.178841in}}%
\pgfpathlineto{\pgfqpoint{2.489240in}{2.177232in}}%
\pgfpathlineto{\pgfqpoint{2.490480in}{2.177756in}}%
\pgfpathlineto{\pgfqpoint{2.491720in}{2.176720in}}%
\pgfpathlineto{\pgfqpoint{2.497920in}{2.179452in}}%
\pgfpathlineto{\pgfqpoint{2.500400in}{2.183344in}}%
\pgfpathlineto{\pgfqpoint{2.504120in}{2.187505in}}%
\pgfpathlineto{\pgfqpoint{2.506600in}{2.181621in}}%
\pgfpathlineto{\pgfqpoint{2.507840in}{2.182468in}}%
\pgfpathlineto{\pgfqpoint{2.510320in}{2.183401in}}%
\pgfpathlineto{\pgfqpoint{2.514040in}{2.178616in}}%
\pgfpathlineto{\pgfqpoint{2.525200in}{2.187646in}}%
\pgfpathlineto{\pgfqpoint{2.527680in}{2.185025in}}%
\pgfpathlineto{\pgfqpoint{2.530160in}{2.183766in}}%
\pgfpathlineto{\pgfqpoint{2.533880in}{2.189269in}}%
\pgfpathlineto{\pgfqpoint{2.536360in}{2.187757in}}%
\pgfpathlineto{\pgfqpoint{2.538840in}{2.189525in}}%
\pgfpathlineto{\pgfqpoint{2.540080in}{2.193154in}}%
\pgfpathlineto{\pgfqpoint{2.547520in}{2.191480in}}%
\pgfpathlineto{\pgfqpoint{2.551240in}{2.186874in}}%
\pgfpathlineto{\pgfqpoint{2.552480in}{2.187527in}}%
\pgfpathlineto{\pgfqpoint{2.557440in}{2.193259in}}%
\pgfpathlineto{\pgfqpoint{2.558680in}{2.195150in}}%
\pgfpathlineto{\pgfqpoint{2.561160in}{2.193125in}}%
\pgfpathlineto{\pgfqpoint{2.563640in}{2.193895in}}%
\pgfpathlineto{\pgfqpoint{2.567360in}{2.188402in}}%
\pgfpathlineto{\pgfqpoint{2.569840in}{2.186891in}}%
\pgfpathlineto{\pgfqpoint{2.572320in}{2.188304in}}%
\pgfpathlineto{\pgfqpoint{2.573560in}{2.186307in}}%
\pgfpathlineto{\pgfqpoint{2.576040in}{2.187571in}}%
\pgfpathlineto{\pgfqpoint{2.579760in}{2.186117in}}%
\pgfpathlineto{\pgfqpoint{2.582240in}{2.191115in}}%
\pgfpathlineto{\pgfqpoint{2.583480in}{2.189884in}}%
\pgfpathlineto{\pgfqpoint{2.584720in}{2.190992in}}%
\pgfpathlineto{\pgfqpoint{2.587200in}{2.195780in}}%
\pgfpathlineto{\pgfqpoint{2.589680in}{2.194187in}}%
\pgfpathlineto{\pgfqpoint{2.592160in}{2.190967in}}%
\pgfpathlineto{\pgfqpoint{2.594640in}{2.190088in}}%
\pgfpathlineto{\pgfqpoint{2.595880in}{2.191365in}}%
\pgfpathlineto{\pgfqpoint{2.598360in}{2.187881in}}%
\pgfpathlineto{\pgfqpoint{2.602080in}{2.189922in}}%
\pgfpathlineto{\pgfqpoint{2.605800in}{2.185628in}}%
\pgfpathlineto{\pgfqpoint{2.608280in}{2.190633in}}%
\pgfpathlineto{\pgfqpoint{2.609520in}{2.188320in}}%
\pgfpathlineto{\pgfqpoint{2.612000in}{2.180338in}}%
\pgfpathlineto{\pgfqpoint{2.614480in}{2.182616in}}%
\pgfpathlineto{\pgfqpoint{2.615720in}{2.181417in}}%
\pgfpathlineto{\pgfqpoint{2.621920in}{2.184882in}}%
\pgfpathlineto{\pgfqpoint{2.624400in}{2.188857in}}%
\pgfpathlineto{\pgfqpoint{2.628120in}{2.193470in}}%
\pgfpathlineto{\pgfqpoint{2.629360in}{2.191948in}}%
\pgfpathlineto{\pgfqpoint{2.630600in}{2.188142in}}%
\pgfpathlineto{\pgfqpoint{2.634320in}{2.189883in}}%
\pgfpathlineto{\pgfqpoint{2.638040in}{2.185964in}}%
\pgfpathlineto{\pgfqpoint{2.640520in}{2.187719in}}%
\pgfpathlineto{\pgfqpoint{2.643000in}{2.190432in}}%
\pgfpathlineto{\pgfqpoint{2.645480in}{2.190706in}}%
\pgfpathlineto{\pgfqpoint{2.647960in}{2.196129in}}%
\pgfpathlineto{\pgfqpoint{2.649200in}{2.195411in}}%
\pgfpathlineto{\pgfqpoint{2.651680in}{2.192306in}}%
\pgfpathlineto{\pgfqpoint{2.654160in}{2.190305in}}%
\pgfpathlineto{\pgfqpoint{2.657880in}{2.194786in}}%
\pgfpathlineto{\pgfqpoint{2.660360in}{2.193393in}}%
\pgfpathlineto{\pgfqpoint{2.665320in}{2.198242in}}%
\pgfpathlineto{\pgfqpoint{2.667800in}{2.198825in}}%
\pgfpathlineto{\pgfqpoint{2.670280in}{2.198682in}}%
\pgfpathlineto{\pgfqpoint{2.674000in}{2.193867in}}%
\pgfpathlineto{\pgfqpoint{2.675240in}{2.191251in}}%
\pgfpathlineto{\pgfqpoint{2.676480in}{2.191753in}}%
\pgfpathlineto{\pgfqpoint{2.682680in}{2.199204in}}%
\pgfpathlineto{\pgfqpoint{2.685160in}{2.196140in}}%
\pgfpathlineto{\pgfqpoint{2.687640in}{2.196764in}}%
\pgfpathlineto{\pgfqpoint{2.690120in}{2.194914in}}%
\pgfpathlineto{\pgfqpoint{2.693840in}{2.191447in}}%
\pgfpathlineto{\pgfqpoint{2.701280in}{2.191565in}}%
\pgfpathlineto{\pgfqpoint{2.703760in}{2.190220in}}%
\pgfpathlineto{\pgfqpoint{2.706240in}{2.194991in}}%
\pgfpathlineto{\pgfqpoint{2.707480in}{2.194359in}}%
\pgfpathlineto{\pgfqpoint{2.713680in}{2.199730in}}%
\pgfpathlineto{\pgfqpoint{2.716160in}{2.196548in}}%
\pgfpathlineto{\pgfqpoint{2.718640in}{2.195051in}}%
\pgfpathlineto{\pgfqpoint{2.719880in}{2.195889in}}%
\pgfpathlineto{\pgfqpoint{2.722360in}{2.193201in}}%
\pgfpathlineto{\pgfqpoint{2.724840in}{2.196932in}}%
\pgfpathlineto{\pgfqpoint{2.726080in}{2.196179in}}%
\pgfpathlineto{\pgfqpoint{2.729800in}{2.190566in}}%
\pgfpathlineto{\pgfqpoint{2.732280in}{2.194597in}}%
\pgfpathlineto{\pgfqpoint{2.733520in}{2.192588in}}%
\pgfpathlineto{\pgfqpoint{2.736000in}{2.185667in}}%
\pgfpathlineto{\pgfqpoint{2.738480in}{2.187161in}}%
\pgfpathlineto{\pgfqpoint{2.740960in}{2.185652in}}%
\pgfpathlineto{\pgfqpoint{2.743440in}{2.185902in}}%
\pgfpathlineto{\pgfqpoint{2.745920in}{2.188443in}}%
\pgfpathlineto{\pgfqpoint{2.748400in}{2.193374in}}%
\pgfpathlineto{\pgfqpoint{2.752120in}{2.197320in}}%
\pgfpathlineto{\pgfqpoint{2.753360in}{2.196161in}}%
\pgfpathlineto{\pgfqpoint{2.754600in}{2.192430in}}%
\pgfpathlineto{\pgfqpoint{2.758320in}{2.193622in}}%
\pgfpathlineto{\pgfqpoint{2.762040in}{2.189847in}}%
\pgfpathlineto{\pgfqpoint{2.763280in}{2.189457in}}%
\pgfpathlineto{\pgfqpoint{2.773200in}{2.197935in}}%
\pgfpathlineto{\pgfqpoint{2.778160in}{2.193185in}}%
\pgfpathlineto{\pgfqpoint{2.781880in}{2.199837in}}%
\pgfpathlineto{\pgfqpoint{2.784360in}{2.197302in}}%
\pgfpathlineto{\pgfqpoint{2.786840in}{2.199028in}}%
\pgfpathlineto{\pgfqpoint{2.789320in}{2.202135in}}%
\pgfpathlineto{\pgfqpoint{2.794280in}{2.202613in}}%
\pgfpathlineto{\pgfqpoint{2.800480in}{2.194983in}}%
\pgfpathlineto{\pgfqpoint{2.802960in}{2.199185in}}%
\pgfpathlineto{\pgfqpoint{2.805440in}{2.200620in}}%
\pgfpathlineto{\pgfqpoint{2.806680in}{2.202097in}}%
\pgfpathlineto{\pgfqpoint{2.809160in}{2.199558in}}%
\pgfpathlineto{\pgfqpoint{2.811640in}{2.200440in}}%
\pgfpathlineto{\pgfqpoint{2.815360in}{2.196613in}}%
\pgfpathlineto{\pgfqpoint{2.819080in}{2.194731in}}%
\pgfpathlineto{\pgfqpoint{2.820320in}{2.195947in}}%
\pgfpathlineto{\pgfqpoint{2.821560in}{2.194832in}}%
\pgfpathlineto{\pgfqpoint{2.824040in}{2.196589in}}%
\pgfpathlineto{\pgfqpoint{2.827760in}{2.194343in}}%
\pgfpathlineto{\pgfqpoint{2.830240in}{2.198132in}}%
\pgfpathlineto{\pgfqpoint{2.831480in}{2.197211in}}%
\pgfpathlineto{\pgfqpoint{2.836440in}{2.201773in}}%
\pgfpathlineto{\pgfqpoint{2.837680in}{2.201776in}}%
\pgfpathlineto{\pgfqpoint{2.840160in}{2.198630in}}%
\pgfpathlineto{\pgfqpoint{2.842640in}{2.197650in}}%
\pgfpathlineto{\pgfqpoint{2.843880in}{2.198402in}}%
\pgfpathlineto{\pgfqpoint{2.846360in}{2.194672in}}%
\pgfpathlineto{\pgfqpoint{2.850080in}{2.197942in}}%
\pgfpathlineto{\pgfqpoint{2.853800in}{2.193311in}}%
\pgfpathlineto{\pgfqpoint{2.856280in}{2.196896in}}%
\pgfpathlineto{\pgfqpoint{2.857520in}{2.194915in}}%
\pgfpathlineto{\pgfqpoint{2.860000in}{2.188383in}}%
\pgfpathlineto{\pgfqpoint{2.862480in}{2.190817in}}%
\pgfpathlineto{\pgfqpoint{2.863720in}{2.189192in}}%
\pgfpathlineto{\pgfqpoint{2.869920in}{2.193045in}}%
\pgfpathlineto{\pgfqpoint{2.872400in}{2.197629in}}%
\pgfpathlineto{\pgfqpoint{2.876120in}{2.201635in}}%
\pgfpathlineto{\pgfqpoint{2.877360in}{2.200762in}}%
\pgfpathlineto{\pgfqpoint{2.878600in}{2.197754in}}%
\pgfpathlineto{\pgfqpoint{2.882320in}{2.200250in}}%
\pgfpathlineto{\pgfqpoint{2.887280in}{2.196130in}}%
\pgfpathlineto{\pgfqpoint{2.892240in}{2.200866in}}%
\pgfpathlineto{\pgfqpoint{2.893480in}{2.200965in}}%
\pgfpathlineto{\pgfqpoint{2.895960in}{2.204641in}}%
\pgfpathlineto{\pgfqpoint{2.898440in}{2.202977in}}%
\pgfpathlineto{\pgfqpoint{2.902160in}{2.200408in}}%
\pgfpathlineto{\pgfqpoint{2.905880in}{2.207821in}}%
\pgfpathlineto{\pgfqpoint{2.908360in}{2.204854in}}%
\pgfpathlineto{\pgfqpoint{2.910840in}{2.206579in}}%
\pgfpathlineto{\pgfqpoint{2.913320in}{2.210551in}}%
\pgfpathlineto{\pgfqpoint{2.919520in}{2.210088in}}%
\pgfpathlineto{\pgfqpoint{2.924480in}{2.203684in}}%
\pgfpathlineto{\pgfqpoint{2.926960in}{2.207938in}}%
\pgfpathlineto{\pgfqpoint{2.929440in}{2.208182in}}%
\pgfpathlineto{\pgfqpoint{2.930680in}{2.209114in}}%
\pgfpathlineto{\pgfqpoint{2.933160in}{2.206310in}}%
\pgfpathlineto{\pgfqpoint{2.935640in}{2.207094in}}%
\pgfpathlineto{\pgfqpoint{2.939360in}{2.202132in}}%
\pgfpathlineto{\pgfqpoint{2.941840in}{2.200389in}}%
\pgfpathlineto{\pgfqpoint{2.948040in}{2.203689in}}%
\pgfpathlineto{\pgfqpoint{2.951760in}{2.201796in}}%
\pgfpathlineto{\pgfqpoint{2.954240in}{2.205862in}}%
\pgfpathlineto{\pgfqpoint{2.955480in}{2.205126in}}%
\pgfpathlineto{\pgfqpoint{2.960440in}{2.209314in}}%
\pgfpathlineto{\pgfqpoint{2.961680in}{2.208610in}}%
\pgfpathlineto{\pgfqpoint{2.962920in}{2.205222in}}%
\pgfpathlineto{\pgfqpoint{2.964160in}{2.205432in}}%
\pgfpathlineto{\pgfqpoint{2.966640in}{2.204014in}}%
\pgfpathlineto{\pgfqpoint{2.967880in}{2.204088in}}%
\pgfpathlineto{\pgfqpoint{2.970360in}{2.200220in}}%
\pgfpathlineto{\pgfqpoint{2.974080in}{2.203528in}}%
\pgfpathlineto{\pgfqpoint{2.976560in}{2.199706in}}%
\pgfpathlineto{\pgfqpoint{2.977800in}{2.199541in}}%
\pgfpathlineto{\pgfqpoint{2.980280in}{2.203692in}}%
\pgfpathlineto{\pgfqpoint{2.981520in}{2.201961in}}%
\pgfpathlineto{\pgfqpoint{2.984000in}{2.195654in}}%
\pgfpathlineto{\pgfqpoint{2.986480in}{2.198476in}}%
\pgfpathlineto{\pgfqpoint{2.988960in}{2.197031in}}%
\pgfpathlineto{\pgfqpoint{2.993920in}{2.199995in}}%
\pgfpathlineto{\pgfqpoint{2.996400in}{2.204808in}}%
\pgfpathlineto{\pgfqpoint{3.000120in}{2.209453in}}%
\pgfpathlineto{\pgfqpoint{3.001360in}{2.208198in}}%
\pgfpathlineto{\pgfqpoint{3.002600in}{2.205243in}}%
\pgfpathlineto{\pgfqpoint{3.006320in}{2.208033in}}%
\pgfpathlineto{\pgfqpoint{3.007560in}{2.206969in}}%
\pgfpathlineto{\pgfqpoint{3.010040in}{2.203264in}}%
\pgfpathlineto{\pgfqpoint{3.011280in}{2.202684in}}%
\pgfpathlineto{\pgfqpoint{3.016240in}{2.206807in}}%
\pgfpathlineto{\pgfqpoint{3.017480in}{2.206653in}}%
\pgfpathlineto{\pgfqpoint{3.019960in}{2.210250in}}%
\pgfpathlineto{\pgfqpoint{3.022440in}{2.209177in}}%
\pgfpathlineto{\pgfqpoint{3.023680in}{2.208654in}}%
\pgfpathlineto{\pgfqpoint{3.026160in}{2.206563in}}%
\pgfpathlineto{\pgfqpoint{3.029880in}{2.213867in}}%
\pgfpathlineto{\pgfqpoint{3.032360in}{2.210801in}}%
\pgfpathlineto{\pgfqpoint{3.034840in}{2.212692in}}%
\pgfpathlineto{\pgfqpoint{3.037320in}{2.216818in}}%
\pgfpathlineto{\pgfqpoint{3.041040in}{2.217498in}}%
\pgfpathlineto{\pgfqpoint{3.043520in}{2.215364in}}%
\pgfpathlineto{\pgfqpoint{3.047240in}{2.208118in}}%
\pgfpathlineto{\pgfqpoint{3.048480in}{2.208242in}}%
\pgfpathlineto{\pgfqpoint{3.050960in}{2.211513in}}%
\pgfpathlineto{\pgfqpoint{3.053440in}{2.211481in}}%
\pgfpathlineto{\pgfqpoint{3.054680in}{2.212761in}}%
\pgfpathlineto{\pgfqpoint{3.057160in}{2.210097in}}%
\pgfpathlineto{\pgfqpoint{3.059640in}{2.211516in}}%
\pgfpathlineto{\pgfqpoint{3.064600in}{2.205649in}}%
\pgfpathlineto{\pgfqpoint{3.065840in}{2.205064in}}%
\pgfpathlineto{\pgfqpoint{3.069560in}{2.208247in}}%
\pgfpathlineto{\pgfqpoint{3.072040in}{2.209517in}}%
\pgfpathlineto{\pgfqpoint{3.075760in}{2.207380in}}%
\pgfpathlineto{\pgfqpoint{3.078240in}{2.210454in}}%
\pgfpathlineto{\pgfqpoint{3.079480in}{2.209722in}}%
\pgfpathlineto{\pgfqpoint{3.084440in}{2.215451in}}%
\pgfpathlineto{\pgfqpoint{3.085680in}{2.214657in}}%
\pgfpathlineto{\pgfqpoint{3.086920in}{2.211503in}}%
\pgfpathlineto{\pgfqpoint{3.088160in}{2.211531in}}%
\pgfpathlineto{\pgfqpoint{3.090640in}{2.209575in}}%
\pgfpathlineto{\pgfqpoint{3.093120in}{2.208207in}}%
\pgfpathlineto{\pgfqpoint{3.094360in}{2.205719in}}%
\pgfpathlineto{\pgfqpoint{3.098080in}{2.209293in}}%
\pgfpathlineto{\pgfqpoint{3.101800in}{2.205654in}}%
\pgfpathlineto{\pgfqpoint{3.104280in}{2.210024in}}%
\pgfpathlineto{\pgfqpoint{3.108000in}{2.200190in}}%
\pgfpathlineto{\pgfqpoint{3.109240in}{2.201760in}}%
\pgfpathlineto{\pgfqpoint{3.110480in}{2.202427in}}%
\pgfpathlineto{\pgfqpoint{3.111720in}{2.201212in}}%
\pgfpathlineto{\pgfqpoint{3.117920in}{2.204736in}}%
\pgfpathlineto{\pgfqpoint{3.120400in}{2.209080in}}%
\pgfpathlineto{\pgfqpoint{3.124120in}{2.212780in}}%
\pgfpathlineto{\pgfqpoint{3.127840in}{2.210001in}}%
\pgfpathlineto{\pgfqpoint{3.129080in}{2.210252in}}%
\pgfpathlineto{\pgfqpoint{3.130320in}{2.211801in}}%
\pgfpathlineto{\pgfqpoint{3.131560in}{2.211098in}}%
\pgfpathlineto{\pgfqpoint{3.134040in}{2.208225in}}%
\pgfpathlineto{\pgfqpoint{3.135280in}{2.207320in}}%
\pgfpathlineto{\pgfqpoint{3.140240in}{2.210472in}}%
\pgfpathlineto{\pgfqpoint{3.141480in}{2.210464in}}%
\pgfpathlineto{\pgfqpoint{3.143960in}{2.213848in}}%
\pgfpathlineto{\pgfqpoint{3.147680in}{2.212918in}}%
\pgfpathlineto{\pgfqpoint{3.150160in}{2.210614in}}%
\pgfpathlineto{\pgfqpoint{3.153880in}{2.217646in}}%
\pgfpathlineto{\pgfqpoint{3.156360in}{2.214490in}}%
\pgfpathlineto{\pgfqpoint{3.158840in}{2.216125in}}%
\pgfpathlineto{\pgfqpoint{3.160080in}{2.219506in}}%
\pgfpathlineto{\pgfqpoint{3.166280in}{2.218453in}}%
\pgfpathlineto{\pgfqpoint{3.168760in}{2.214141in}}%
\pgfpathlineto{\pgfqpoint{3.172480in}{2.210286in}}%
\pgfpathlineto{\pgfqpoint{3.174960in}{2.213858in}}%
\pgfpathlineto{\pgfqpoint{3.177440in}{2.213974in}}%
\pgfpathlineto{\pgfqpoint{3.178680in}{2.215563in}}%
\pgfpathlineto{\pgfqpoint{3.181160in}{2.213241in}}%
\pgfpathlineto{\pgfqpoint{3.183640in}{2.214158in}}%
\pgfpathlineto{\pgfqpoint{3.188600in}{2.208088in}}%
\pgfpathlineto{\pgfqpoint{3.189840in}{2.207477in}}%
\pgfpathlineto{\pgfqpoint{3.192320in}{2.211429in}}%
\pgfpathlineto{\pgfqpoint{3.193560in}{2.210265in}}%
\pgfpathlineto{\pgfqpoint{3.196040in}{2.210752in}}%
\pgfpathlineto{\pgfqpoint{3.199760in}{2.208126in}}%
\pgfpathlineto{\pgfqpoint{3.202240in}{2.210898in}}%
\pgfpathlineto{\pgfqpoint{3.203480in}{2.210271in}}%
\pgfpathlineto{\pgfqpoint{3.207200in}{2.215501in}}%
\pgfpathlineto{\pgfqpoint{3.209680in}{2.215433in}}%
\pgfpathlineto{\pgfqpoint{3.210920in}{2.212665in}}%
\pgfpathlineto{\pgfqpoint{3.212160in}{2.212911in}}%
\pgfpathlineto{\pgfqpoint{3.214640in}{2.210791in}}%
\pgfpathlineto{\pgfqpoint{3.217120in}{2.209741in}}%
\pgfpathlineto{\pgfqpoint{3.218360in}{2.207473in}}%
\pgfpathlineto{\pgfqpoint{3.220840in}{2.209904in}}%
\pgfpathlineto{\pgfqpoint{3.222080in}{2.209659in}}%
\pgfpathlineto{\pgfqpoint{3.225800in}{2.205372in}}%
\pgfpathlineto{\pgfqpoint{3.228280in}{2.209811in}}%
\pgfpathlineto{\pgfqpoint{3.232000in}{2.200207in}}%
\pgfpathlineto{\pgfqpoint{3.233240in}{2.201223in}}%
\pgfpathlineto{\pgfqpoint{3.234480in}{2.201796in}}%
\pgfpathlineto{\pgfqpoint{3.235720in}{2.200838in}}%
\pgfpathlineto{\pgfqpoint{3.238200in}{2.201987in}}%
\pgfpathlineto{\pgfqpoint{3.239440in}{2.201031in}}%
\pgfpathlineto{\pgfqpoint{3.243160in}{2.207793in}}%
\pgfpathlineto{\pgfqpoint{3.244400in}{2.207935in}}%
\pgfpathlineto{\pgfqpoint{3.248120in}{2.212977in}}%
\pgfpathlineto{\pgfqpoint{3.249360in}{2.211824in}}%
\pgfpathlineto{\pgfqpoint{3.250600in}{2.209016in}}%
\pgfpathlineto{\pgfqpoint{3.254320in}{2.211607in}}%
\pgfpathlineto{\pgfqpoint{3.259280in}{2.206154in}}%
\pgfpathlineto{\pgfqpoint{3.265480in}{2.208817in}}%
\pgfpathlineto{\pgfqpoint{3.267960in}{2.212210in}}%
\pgfpathlineto{\pgfqpoint{3.272920in}{2.210506in}}%
\pgfpathlineto{\pgfqpoint{3.274160in}{2.209712in}}%
\pgfpathlineto{\pgfqpoint{3.277880in}{2.216994in}}%
\pgfpathlineto{\pgfqpoint{3.280360in}{2.213915in}}%
\pgfpathlineto{\pgfqpoint{3.282840in}{2.215625in}}%
\pgfpathlineto{\pgfqpoint{3.284080in}{2.219228in}}%
\pgfpathlineto{\pgfqpoint{3.286560in}{2.219625in}}%
\pgfpathlineto{\pgfqpoint{3.289040in}{2.219897in}}%
\pgfpathlineto{\pgfqpoint{3.291520in}{2.216689in}}%
\pgfpathlineto{\pgfqpoint{3.295240in}{2.208812in}}%
\pgfpathlineto{\pgfqpoint{3.296480in}{2.209606in}}%
\pgfpathlineto{\pgfqpoint{3.298960in}{2.214032in}}%
\pgfpathlineto{\pgfqpoint{3.301440in}{2.213443in}}%
\pgfpathlineto{\pgfqpoint{3.302680in}{2.215131in}}%
\pgfpathlineto{\pgfqpoint{3.305160in}{2.213520in}}%
\pgfpathlineto{\pgfqpoint{3.307640in}{2.214393in}}%
\pgfpathlineto{\pgfqpoint{3.313840in}{2.207658in}}%
\pgfpathlineto{\pgfqpoint{3.316320in}{2.210799in}}%
\pgfpathlineto{\pgfqpoint{3.317560in}{2.209365in}}%
\pgfpathlineto{\pgfqpoint{3.320040in}{2.210836in}}%
\pgfpathlineto{\pgfqpoint{3.325000in}{2.210671in}}%
\pgfpathlineto{\pgfqpoint{3.326240in}{2.211885in}}%
\pgfpathlineto{\pgfqpoint{3.327480in}{2.211379in}}%
\pgfpathlineto{\pgfqpoint{3.331200in}{2.216170in}}%
\pgfpathlineto{\pgfqpoint{3.333680in}{2.216507in}}%
\pgfpathlineto{\pgfqpoint{3.336160in}{2.213392in}}%
\pgfpathlineto{\pgfqpoint{3.338640in}{2.210772in}}%
\pgfpathlineto{\pgfqpoint{3.341120in}{2.209878in}}%
\pgfpathlineto{\pgfqpoint{3.342360in}{2.207801in}}%
\pgfpathlineto{\pgfqpoint{3.344840in}{2.209855in}}%
\pgfpathlineto{\pgfqpoint{3.346080in}{2.209580in}}%
\pgfpathlineto{\pgfqpoint{3.349800in}{2.206510in}}%
\pgfpathlineto{\pgfqpoint{3.352280in}{2.210533in}}%
\pgfpathlineto{\pgfqpoint{3.357240in}{2.201714in}}%
\pgfpathlineto{\pgfqpoint{3.363440in}{2.201476in}}%
\pgfpathlineto{\pgfqpoint{3.367160in}{2.207779in}}%
\pgfpathlineto{\pgfqpoint{3.368400in}{2.207708in}}%
\pgfpathlineto{\pgfqpoint{3.372120in}{2.212021in}}%
\pgfpathlineto{\pgfqpoint{3.373360in}{2.211075in}}%
\pgfpathlineto{\pgfqpoint{3.374600in}{2.207878in}}%
\pgfpathlineto{\pgfqpoint{3.378320in}{2.211233in}}%
\pgfpathlineto{\pgfqpoint{3.383280in}{2.206118in}}%
\pgfpathlineto{\pgfqpoint{3.395680in}{2.211165in}}%
\pgfpathlineto{\pgfqpoint{3.398160in}{2.210232in}}%
\pgfpathlineto{\pgfqpoint{3.401880in}{2.217673in}}%
\pgfpathlineto{\pgfqpoint{3.404360in}{2.214297in}}%
\pgfpathlineto{\pgfqpoint{3.406840in}{2.216672in}}%
\pgfpathlineto{\pgfqpoint{3.408080in}{2.221366in}}%
\pgfpathlineto{\pgfqpoint{3.413040in}{2.221314in}}%
\pgfpathlineto{\pgfqpoint{3.416760in}{2.214666in}}%
\pgfpathlineto{\pgfqpoint{3.419240in}{2.209672in}}%
\pgfpathlineto{\pgfqpoint{3.420480in}{2.210716in}}%
\pgfpathlineto{\pgfqpoint{3.422960in}{2.214799in}}%
\pgfpathlineto{\pgfqpoint{3.425440in}{2.213745in}}%
\pgfpathlineto{\pgfqpoint{3.426680in}{2.215761in}}%
\pgfpathlineto{\pgfqpoint{3.429160in}{2.214242in}}%
\pgfpathlineto{\pgfqpoint{3.431640in}{2.216434in}}%
\pgfpathlineto{\pgfqpoint{3.437840in}{2.210026in}}%
\pgfpathlineto{\pgfqpoint{3.441560in}{2.212796in}}%
\pgfpathlineto{\pgfqpoint{3.445280in}{2.213122in}}%
\pgfpathlineto{\pgfqpoint{3.447760in}{2.211939in}}%
\pgfpathlineto{\pgfqpoint{3.452720in}{2.216709in}}%
\pgfpathlineto{\pgfqpoint{3.455200in}{2.218756in}}%
\pgfpathlineto{\pgfqpoint{3.457680in}{2.217718in}}%
\pgfpathlineto{\pgfqpoint{3.458920in}{2.215155in}}%
\pgfpathlineto{\pgfqpoint{3.460160in}{2.215684in}}%
\pgfpathlineto{\pgfqpoint{3.462640in}{2.213889in}}%
\pgfpathlineto{\pgfqpoint{3.465120in}{2.213385in}}%
\pgfpathlineto{\pgfqpoint{3.466360in}{2.211594in}}%
\pgfpathlineto{\pgfqpoint{3.468840in}{2.213646in}}%
\pgfpathlineto{\pgfqpoint{3.471320in}{2.211573in}}%
\pgfpathlineto{\pgfqpoint{3.473800in}{2.210543in}}%
\pgfpathlineto{\pgfqpoint{3.476280in}{2.214834in}}%
\pgfpathlineto{\pgfqpoint{3.478760in}{2.208906in}}%
\pgfpathlineto{\pgfqpoint{3.480000in}{2.206058in}}%
\pgfpathlineto{\pgfqpoint{3.482480in}{2.207534in}}%
\pgfpathlineto{\pgfqpoint{3.483720in}{2.206762in}}%
\pgfpathlineto{\pgfqpoint{3.486200in}{2.207760in}}%
\pgfpathlineto{\pgfqpoint{3.487440in}{2.206674in}}%
\pgfpathlineto{\pgfqpoint{3.491160in}{2.212898in}}%
\pgfpathlineto{\pgfqpoint{3.492400in}{2.212519in}}%
\pgfpathlineto{\pgfqpoint{3.496120in}{2.217110in}}%
\pgfpathlineto{\pgfqpoint{3.497360in}{2.216569in}}%
\pgfpathlineto{\pgfqpoint{3.498600in}{2.213306in}}%
\pgfpathlineto{\pgfqpoint{3.502320in}{2.216055in}}%
\pgfpathlineto{\pgfqpoint{3.508520in}{2.210833in}}%
\pgfpathlineto{\pgfqpoint{3.511000in}{2.212908in}}%
\pgfpathlineto{\pgfqpoint{3.513480in}{2.212840in}}%
\pgfpathlineto{\pgfqpoint{3.515960in}{2.215895in}}%
\pgfpathlineto{\pgfqpoint{3.517200in}{2.216151in}}%
\pgfpathlineto{\pgfqpoint{3.519680in}{2.214692in}}%
\pgfpathlineto{\pgfqpoint{3.522160in}{2.214500in}}%
\pgfpathlineto{\pgfqpoint{3.525880in}{2.221763in}}%
\pgfpathlineto{\pgfqpoint{3.528360in}{2.217857in}}%
\pgfpathlineto{\pgfqpoint{3.530840in}{2.221224in}}%
\pgfpathlineto{\pgfqpoint{3.532080in}{2.226230in}}%
\pgfpathlineto{\pgfqpoint{3.537040in}{2.225851in}}%
\pgfpathlineto{\pgfqpoint{3.540760in}{2.217755in}}%
\pgfpathlineto{\pgfqpoint{3.543240in}{2.212394in}}%
\pgfpathlineto{\pgfqpoint{3.544480in}{2.213715in}}%
\pgfpathlineto{\pgfqpoint{3.546960in}{2.217873in}}%
\pgfpathlineto{\pgfqpoint{3.549440in}{2.218003in}}%
\pgfpathlineto{\pgfqpoint{3.550680in}{2.220881in}}%
\pgfpathlineto{\pgfqpoint{3.553160in}{2.218994in}}%
\pgfpathlineto{\pgfqpoint{3.555640in}{2.220893in}}%
\pgfpathlineto{\pgfqpoint{3.561840in}{2.214102in}}%
\pgfpathlineto{\pgfqpoint{3.565560in}{2.217408in}}%
\pgfpathlineto{\pgfqpoint{3.573000in}{2.219640in}}%
\pgfpathlineto{\pgfqpoint{3.579200in}{2.226409in}}%
\pgfpathlineto{\pgfqpoint{3.580440in}{2.226853in}}%
\pgfpathlineto{\pgfqpoint{3.581680in}{2.225603in}}%
\pgfpathlineto{\pgfqpoint{3.582920in}{2.222571in}}%
\pgfpathlineto{\pgfqpoint{3.584160in}{2.222792in}}%
\pgfpathlineto{\pgfqpoint{3.586640in}{2.221116in}}%
\pgfpathlineto{\pgfqpoint{3.590360in}{2.219486in}}%
\pgfpathlineto{\pgfqpoint{3.592840in}{2.221780in}}%
\pgfpathlineto{\pgfqpoint{3.595320in}{2.220041in}}%
\pgfpathlineto{\pgfqpoint{3.597800in}{2.218746in}}%
\pgfpathlineto{\pgfqpoint{3.600280in}{2.223872in}}%
\pgfpathlineto{\pgfqpoint{3.601520in}{2.222168in}}%
\pgfpathlineto{\pgfqpoint{3.604000in}{2.215407in}}%
\pgfpathlineto{\pgfqpoint{3.605240in}{2.218246in}}%
\pgfpathlineto{\pgfqpoint{3.611440in}{2.217519in}}%
\pgfpathlineto{\pgfqpoint{3.613920in}{2.221503in}}%
\pgfpathlineto{\pgfqpoint{3.615160in}{2.223226in}}%
\pgfpathlineto{\pgfqpoint{3.616400in}{2.222456in}}%
\pgfpathlineto{\pgfqpoint{3.618880in}{2.226880in}}%
\pgfpathlineto{\pgfqpoint{3.620120in}{2.226855in}}%
\pgfpathlineto{\pgfqpoint{3.621360in}{2.225708in}}%
\pgfpathlineto{\pgfqpoint{3.622600in}{2.222840in}}%
\pgfpathlineto{\pgfqpoint{3.623840in}{2.224244in}}%
\pgfpathlineto{\pgfqpoint{3.625080in}{2.223603in}}%
\pgfpathlineto{\pgfqpoint{3.626320in}{2.224941in}}%
\pgfpathlineto{\pgfqpoint{3.632520in}{2.219006in}}%
\pgfpathlineto{\pgfqpoint{3.635000in}{2.221073in}}%
\pgfpathlineto{\pgfqpoint{3.637480in}{2.220791in}}%
\pgfpathlineto{\pgfqpoint{3.638720in}{2.223553in}}%
\pgfpathlineto{\pgfqpoint{3.641200in}{2.223157in}}%
\pgfpathlineto{\pgfqpoint{3.643680in}{2.220981in}}%
\pgfpathlineto{\pgfqpoint{3.646160in}{2.220919in}}%
\pgfpathlineto{\pgfqpoint{3.649880in}{2.228435in}}%
\pgfpathlineto{\pgfqpoint{3.652360in}{2.224152in}}%
\pgfpathlineto{\pgfqpoint{3.654840in}{2.227633in}}%
\pgfpathlineto{\pgfqpoint{3.656080in}{2.232929in}}%
\pgfpathlineto{\pgfqpoint{3.661040in}{2.233720in}}%
\pgfpathlineto{\pgfqpoint{3.664760in}{2.226381in}}%
\pgfpathlineto{\pgfqpoint{3.667240in}{2.221406in}}%
\pgfpathlineto{\pgfqpoint{3.668480in}{2.222729in}}%
\pgfpathlineto{\pgfqpoint{3.670960in}{2.226353in}}%
\pgfpathlineto{\pgfqpoint{3.673440in}{2.227042in}}%
\pgfpathlineto{\pgfqpoint{3.674680in}{2.229931in}}%
\pgfpathlineto{\pgfqpoint{3.677160in}{2.228608in}}%
\pgfpathlineto{\pgfqpoint{3.679640in}{2.231572in}}%
\pgfpathlineto{\pgfqpoint{3.685840in}{2.225534in}}%
\pgfpathlineto{\pgfqpoint{3.689560in}{2.228926in}}%
\pgfpathlineto{\pgfqpoint{3.695760in}{2.229498in}}%
\pgfpathlineto{\pgfqpoint{3.704440in}{2.237541in}}%
\pgfpathlineto{\pgfqpoint{3.709400in}{2.231973in}}%
\pgfpathlineto{\pgfqpoint{3.713120in}{2.231658in}}%
\pgfpathlineto{\pgfqpoint{3.714360in}{2.229804in}}%
\pgfpathlineto{\pgfqpoint{3.718080in}{2.232307in}}%
\pgfpathlineto{\pgfqpoint{3.721800in}{2.229319in}}%
\pgfpathlineto{\pgfqpoint{3.724280in}{2.234434in}}%
\pgfpathlineto{\pgfqpoint{3.726760in}{2.229305in}}%
\pgfpathlineto{\pgfqpoint{3.728000in}{2.226919in}}%
\pgfpathlineto{\pgfqpoint{3.729240in}{2.228592in}}%
\pgfpathlineto{\pgfqpoint{3.732960in}{2.228651in}}%
\pgfpathlineto{\pgfqpoint{3.735440in}{2.227582in}}%
\pgfpathlineto{\pgfqpoint{3.737920in}{2.232067in}}%
\pgfpathlineto{\pgfqpoint{3.739160in}{2.233525in}}%
\pgfpathlineto{\pgfqpoint{3.740400in}{2.232654in}}%
\pgfpathlineto{\pgfqpoint{3.742880in}{2.237886in}}%
\pgfpathlineto{\pgfqpoint{3.745360in}{2.236841in}}%
\pgfpathlineto{\pgfqpoint{3.746600in}{2.234139in}}%
\pgfpathlineto{\pgfqpoint{3.747840in}{2.234838in}}%
\pgfpathlineto{\pgfqpoint{3.749080in}{2.234077in}}%
\pgfpathlineto{\pgfqpoint{3.750320in}{2.235079in}}%
\pgfpathlineto{\pgfqpoint{3.754040in}{2.232243in}}%
\pgfpathlineto{\pgfqpoint{3.755280in}{2.230192in}}%
\pgfpathlineto{\pgfqpoint{3.756520in}{2.230354in}}%
\pgfpathlineto{\pgfqpoint{3.759000in}{2.232528in}}%
\pgfpathlineto{\pgfqpoint{3.761480in}{2.231703in}}%
\pgfpathlineto{\pgfqpoint{3.763960in}{2.235049in}}%
\pgfpathlineto{\pgfqpoint{3.765200in}{2.235259in}}%
\pgfpathlineto{\pgfqpoint{3.767680in}{2.233542in}}%
\pgfpathlineto{\pgfqpoint{3.768920in}{2.233051in}}%
\pgfpathlineto{\pgfqpoint{3.770160in}{2.233787in}}%
\pgfpathlineto{\pgfqpoint{3.773880in}{2.241969in}}%
\pgfpathlineto{\pgfqpoint{3.776360in}{2.237000in}}%
\pgfpathlineto{\pgfqpoint{3.778840in}{2.240522in}}%
\pgfpathlineto{\pgfqpoint{3.781320in}{2.245683in}}%
\pgfpathlineto{\pgfqpoint{3.785040in}{2.246448in}}%
\pgfpathlineto{\pgfqpoint{3.791240in}{2.233252in}}%
\pgfpathlineto{\pgfqpoint{3.796200in}{2.238438in}}%
\pgfpathlineto{\pgfqpoint{3.797440in}{2.238777in}}%
\pgfpathlineto{\pgfqpoint{3.798680in}{2.241010in}}%
\pgfpathlineto{\pgfqpoint{3.801160in}{2.239042in}}%
\pgfpathlineto{\pgfqpoint{3.803640in}{2.241705in}}%
\pgfpathlineto{\pgfqpoint{3.806120in}{2.239300in}}%
\pgfpathlineto{\pgfqpoint{3.809840in}{2.236863in}}%
\pgfpathlineto{\pgfqpoint{3.813560in}{2.240109in}}%
\pgfpathlineto{\pgfqpoint{3.819760in}{2.240572in}}%
\pgfpathlineto{\pgfqpoint{3.822240in}{2.242830in}}%
\pgfpathlineto{\pgfqpoint{3.823480in}{2.242932in}}%
\pgfpathlineto{\pgfqpoint{3.827200in}{2.248419in}}%
\pgfpathlineto{\pgfqpoint{3.828440in}{2.249146in}}%
\pgfpathlineto{\pgfqpoint{3.833400in}{2.243053in}}%
\pgfpathlineto{\pgfqpoint{3.835880in}{2.243144in}}%
\pgfpathlineto{\pgfqpoint{3.837120in}{2.242783in}}%
\pgfpathlineto{\pgfqpoint{3.838360in}{2.240860in}}%
\pgfpathlineto{\pgfqpoint{3.842080in}{2.243597in}}%
\pgfpathlineto{\pgfqpoint{3.843320in}{2.242979in}}%
\pgfpathlineto{\pgfqpoint{3.845800in}{2.240837in}}%
\pgfpathlineto{\pgfqpoint{3.848280in}{2.246676in}}%
\pgfpathlineto{\pgfqpoint{3.850760in}{2.241530in}}%
\pgfpathlineto{\pgfqpoint{3.852000in}{2.238729in}}%
\pgfpathlineto{\pgfqpoint{3.853240in}{2.243307in}}%
\pgfpathlineto{\pgfqpoint{3.859440in}{2.240214in}}%
\pgfpathlineto{\pgfqpoint{3.861920in}{2.245113in}}%
\pgfpathlineto{\pgfqpoint{3.863160in}{2.246570in}}%
\pgfpathlineto{\pgfqpoint{3.864400in}{2.245223in}}%
\pgfpathlineto{\pgfqpoint{3.866880in}{2.249859in}}%
\pgfpathlineto{\pgfqpoint{3.869360in}{2.248723in}}%
\pgfpathlineto{\pgfqpoint{3.870600in}{2.246201in}}%
\pgfpathlineto{\pgfqpoint{3.871840in}{2.247590in}}%
\pgfpathlineto{\pgfqpoint{3.873080in}{2.246887in}}%
\pgfpathlineto{\pgfqpoint{3.875560in}{2.247013in}}%
\pgfpathlineto{\pgfqpoint{3.879280in}{2.244022in}}%
\pgfpathlineto{\pgfqpoint{3.883000in}{2.245869in}}%
\pgfpathlineto{\pgfqpoint{3.885480in}{2.244511in}}%
\pgfpathlineto{\pgfqpoint{3.887960in}{2.247969in}}%
\pgfpathlineto{\pgfqpoint{3.889200in}{2.247893in}}%
\pgfpathlineto{\pgfqpoint{3.891680in}{2.246644in}}%
\pgfpathlineto{\pgfqpoint{3.894160in}{2.246036in}}%
\pgfpathlineto{\pgfqpoint{3.897880in}{2.253014in}}%
\pgfpathlineto{\pgfqpoint{3.900360in}{2.248851in}}%
\pgfpathlineto{\pgfqpoint{3.902840in}{2.252274in}}%
\pgfpathlineto{\pgfqpoint{3.904080in}{2.257630in}}%
\pgfpathlineto{\pgfqpoint{3.909040in}{2.258820in}}%
\pgfpathlineto{\pgfqpoint{3.915240in}{2.246093in}}%
\pgfpathlineto{\pgfqpoint{3.922680in}{2.253753in}}%
\pgfpathlineto{\pgfqpoint{3.925160in}{2.251075in}}%
\pgfpathlineto{\pgfqpoint{3.927640in}{2.255008in}}%
\pgfpathlineto{\pgfqpoint{3.935080in}{2.251576in}}%
\pgfpathlineto{\pgfqpoint{3.936320in}{2.253856in}}%
\pgfpathlineto{\pgfqpoint{3.941280in}{2.253050in}}%
\pgfpathlineto{\pgfqpoint{3.942520in}{2.252712in}}%
\pgfpathlineto{\pgfqpoint{3.946240in}{2.256232in}}%
\pgfpathlineto{\pgfqpoint{3.947480in}{2.256346in}}%
\pgfpathlineto{\pgfqpoint{3.951200in}{2.262667in}}%
\pgfpathlineto{\pgfqpoint{3.952440in}{2.263100in}}%
\pgfpathlineto{\pgfqpoint{3.957400in}{2.256439in}}%
\pgfpathlineto{\pgfqpoint{3.958640in}{2.257096in}}%
\pgfpathlineto{\pgfqpoint{3.961120in}{2.255621in}}%
\pgfpathlineto{\pgfqpoint{3.962360in}{2.253773in}}%
\pgfpathlineto{\pgfqpoint{3.967320in}{2.254681in}}%
\pgfpathlineto{\pgfqpoint{3.969800in}{2.253069in}}%
\pgfpathlineto{\pgfqpoint{3.972280in}{2.259636in}}%
\pgfpathlineto{\pgfqpoint{3.976000in}{2.252233in}}%
\pgfpathlineto{\pgfqpoint{3.977240in}{2.255669in}}%
\pgfpathlineto{\pgfqpoint{3.979720in}{2.253669in}}%
\pgfpathlineto{\pgfqpoint{3.982200in}{2.253202in}}%
\pgfpathlineto{\pgfqpoint{3.983440in}{2.251773in}}%
\pgfpathlineto{\pgfqpoint{3.987160in}{2.257916in}}%
\pgfpathlineto{\pgfqpoint{3.988400in}{2.256233in}}%
\pgfpathlineto{\pgfqpoint{3.992120in}{2.259945in}}%
\pgfpathlineto{\pgfqpoint{3.993360in}{2.259751in}}%
\pgfpathlineto{\pgfqpoint{3.994600in}{2.257538in}}%
\pgfpathlineto{\pgfqpoint{3.995840in}{2.259505in}}%
\pgfpathlineto{\pgfqpoint{3.997080in}{2.258942in}}%
\pgfpathlineto{\pgfqpoint{3.998320in}{2.260220in}}%
\pgfpathlineto{\pgfqpoint{4.002040in}{2.258028in}}%
\pgfpathlineto{\pgfqpoint{4.003280in}{2.256624in}}%
\pgfpathlineto{\pgfqpoint{4.009480in}{2.257776in}}%
\pgfpathlineto{\pgfqpoint{4.011960in}{2.261419in}}%
\pgfpathlineto{\pgfqpoint{4.014440in}{2.259722in}}%
\pgfpathlineto{\pgfqpoint{4.018160in}{2.258515in}}%
\pgfpathlineto{\pgfqpoint{4.021880in}{2.265680in}}%
\pgfpathlineto{\pgfqpoint{4.024360in}{2.261490in}}%
\pgfpathlineto{\pgfqpoint{4.026840in}{2.265468in}}%
\pgfpathlineto{\pgfqpoint{4.029320in}{2.271673in}}%
\pgfpathlineto{\pgfqpoint{4.031800in}{2.273193in}}%
\pgfpathlineto{\pgfqpoint{4.033040in}{2.272809in}}%
\pgfpathlineto{\pgfqpoint{4.036760in}{2.264614in}}%
\pgfpathlineto{\pgfqpoint{4.039240in}{2.260848in}}%
\pgfpathlineto{\pgfqpoint{4.040480in}{2.261515in}}%
\pgfpathlineto{\pgfqpoint{4.042960in}{2.264626in}}%
\pgfpathlineto{\pgfqpoint{4.046680in}{2.267282in}}%
\pgfpathlineto{\pgfqpoint{4.049160in}{2.264550in}}%
\pgfpathlineto{\pgfqpoint{4.051640in}{2.269058in}}%
\pgfpathlineto{\pgfqpoint{4.055360in}{2.265896in}}%
\pgfpathlineto{\pgfqpoint{4.057840in}{2.264229in}}%
\pgfpathlineto{\pgfqpoint{4.062800in}{2.266931in}}%
\pgfpathlineto{\pgfqpoint{4.064040in}{2.265828in}}%
\pgfpathlineto{\pgfqpoint{4.071480in}{2.269557in}}%
\pgfpathlineto{\pgfqpoint{4.075200in}{2.275351in}}%
\pgfpathlineto{\pgfqpoint{4.076440in}{2.275039in}}%
\pgfpathlineto{\pgfqpoint{4.081400in}{2.266866in}}%
\pgfpathlineto{\pgfqpoint{4.082640in}{2.266983in}}%
\pgfpathlineto{\pgfqpoint{4.086360in}{2.263827in}}%
\pgfpathlineto{\pgfqpoint{4.088840in}{2.264181in}}%
\pgfpathlineto{\pgfqpoint{4.091320in}{2.263797in}}%
\pgfpathlineto{\pgfqpoint{4.093800in}{2.262591in}}%
\pgfpathlineto{\pgfqpoint{4.096280in}{2.269411in}}%
\pgfpathlineto{\pgfqpoint{4.100000in}{2.262649in}}%
\pgfpathlineto{\pgfqpoint{4.101240in}{2.264293in}}%
\pgfpathlineto{\pgfqpoint{4.107440in}{2.260809in}}%
\pgfpathlineto{\pgfqpoint{4.111160in}{2.267427in}}%
\pgfpathlineto{\pgfqpoint{4.112400in}{2.266192in}}%
\pgfpathlineto{\pgfqpoint{4.114880in}{2.270224in}}%
\pgfpathlineto{\pgfqpoint{4.118600in}{2.264840in}}%
\pgfpathlineto{\pgfqpoint{4.119840in}{2.266058in}}%
\pgfpathlineto{\pgfqpoint{4.129760in}{2.263802in}}%
\pgfpathlineto{\pgfqpoint{4.133480in}{2.264304in}}%
\pgfpathlineto{\pgfqpoint{4.135960in}{2.267346in}}%
\pgfpathlineto{\pgfqpoint{4.138440in}{2.266597in}}%
\pgfpathlineto{\pgfqpoint{4.142160in}{2.265502in}}%
\pgfpathlineto{\pgfqpoint{4.145880in}{2.272565in}}%
\pgfpathlineto{\pgfqpoint{4.148360in}{2.267546in}}%
\pgfpathlineto{\pgfqpoint{4.150840in}{2.272012in}}%
\pgfpathlineto{\pgfqpoint{4.153320in}{2.278090in}}%
\pgfpathlineto{\pgfqpoint{4.155800in}{2.279459in}}%
\pgfpathlineto{\pgfqpoint{4.159520in}{2.276626in}}%
\pgfpathlineto{\pgfqpoint{4.162000in}{2.268966in}}%
\pgfpathlineto{\pgfqpoint{4.163240in}{2.268210in}}%
\pgfpathlineto{\pgfqpoint{4.164480in}{2.268945in}}%
\pgfpathlineto{\pgfqpoint{4.166960in}{2.272421in}}%
\pgfpathlineto{\pgfqpoint{4.170680in}{2.276232in}}%
\pgfpathlineto{\pgfqpoint{4.173160in}{2.273657in}}%
\pgfpathlineto{\pgfqpoint{4.175640in}{2.278209in}}%
\pgfpathlineto{\pgfqpoint{4.176880in}{2.277788in}}%
\pgfpathlineto{\pgfqpoint{4.179360in}{2.275047in}}%
\pgfpathlineto{\pgfqpoint{4.181840in}{2.274011in}}%
\pgfpathlineto{\pgfqpoint{4.186800in}{2.275822in}}%
\pgfpathlineto{\pgfqpoint{4.189280in}{2.274518in}}%
\pgfpathlineto{\pgfqpoint{4.190520in}{2.274121in}}%
\pgfpathlineto{\pgfqpoint{4.193000in}{2.278329in}}%
\pgfpathlineto{\pgfqpoint{4.195480in}{2.280579in}}%
\pgfpathlineto{\pgfqpoint{4.197960in}{2.285970in}}%
\pgfpathlineto{\pgfqpoint{4.199200in}{2.286618in}}%
\pgfpathlineto{\pgfqpoint{4.200440in}{2.285772in}}%
\pgfpathlineto{\pgfqpoint{4.205400in}{2.275904in}}%
\pgfpathlineto{\pgfqpoint{4.206640in}{2.275753in}}%
\pgfpathlineto{\pgfqpoint{4.210360in}{2.271703in}}%
\pgfpathlineto{\pgfqpoint{4.214080in}{2.273508in}}%
\pgfpathlineto{\pgfqpoint{4.217800in}{2.270708in}}%
\pgfpathlineto{\pgfqpoint{4.220280in}{2.278701in}}%
\pgfpathlineto{\pgfqpoint{4.222760in}{2.273981in}}%
\pgfpathlineto{\pgfqpoint{4.224000in}{2.272001in}}%
\pgfpathlineto{\pgfqpoint{4.225240in}{2.273670in}}%
\pgfpathlineto{\pgfqpoint{4.227720in}{2.273423in}}%
\pgfpathlineto{\pgfqpoint{4.231440in}{2.269474in}}%
\pgfpathlineto{\pgfqpoint{4.235160in}{2.276250in}}%
\pgfpathlineto{\pgfqpoint{4.236400in}{2.275182in}}%
\pgfpathlineto{\pgfqpoint{4.238880in}{2.280247in}}%
\pgfpathlineto{\pgfqpoint{4.242600in}{2.273563in}}%
\pgfpathlineto{\pgfqpoint{4.245080in}{2.274867in}}%
\pgfpathlineto{\pgfqpoint{4.246320in}{2.275177in}}%
\pgfpathlineto{\pgfqpoint{4.252520in}{2.270128in}}%
\pgfpathlineto{\pgfqpoint{4.255000in}{2.271975in}}%
\pgfpathlineto{\pgfqpoint{4.257480in}{2.272067in}}%
\pgfpathlineto{\pgfqpoint{4.259960in}{2.274877in}}%
\pgfpathlineto{\pgfqpoint{4.261200in}{2.274950in}}%
\pgfpathlineto{\pgfqpoint{4.264920in}{2.272020in}}%
\pgfpathlineto{\pgfqpoint{4.266160in}{2.272432in}}%
\pgfpathlineto{\pgfqpoint{4.268640in}{2.278134in}}%
\pgfpathlineto{\pgfqpoint{4.269880in}{2.279070in}}%
\pgfpathlineto{\pgfqpoint{4.272360in}{2.273525in}}%
\pgfpathlineto{\pgfqpoint{4.274840in}{2.278363in}}%
\pgfpathlineto{\pgfqpoint{4.277320in}{2.284406in}}%
\pgfpathlineto{\pgfqpoint{4.279800in}{2.285546in}}%
\pgfpathlineto{\pgfqpoint{4.281040in}{2.284909in}}%
\pgfpathlineto{\pgfqpoint{4.284760in}{2.277691in}}%
\pgfpathlineto{\pgfqpoint{4.287240in}{2.274785in}}%
\pgfpathlineto{\pgfqpoint{4.288480in}{2.275557in}}%
\pgfpathlineto{\pgfqpoint{4.290960in}{2.279999in}}%
\pgfpathlineto{\pgfqpoint{4.294680in}{2.283643in}}%
\pgfpathlineto{\pgfqpoint{4.297160in}{2.281338in}}%
\pgfpathlineto{\pgfqpoint{4.299640in}{2.284096in}}%
\pgfpathlineto{\pgfqpoint{4.305840in}{2.280608in}}%
\pgfpathlineto{\pgfqpoint{4.308320in}{2.283542in}}%
\pgfpathlineto{\pgfqpoint{4.309560in}{2.283120in}}%
\pgfpathlineto{\pgfqpoint{4.310800in}{2.284548in}}%
\pgfpathlineto{\pgfqpoint{4.314520in}{2.283048in}}%
\pgfpathlineto{\pgfqpoint{4.318240in}{2.289627in}}%
\pgfpathlineto{\pgfqpoint{4.319480in}{2.291014in}}%
\pgfpathlineto{\pgfqpoint{4.321960in}{2.296626in}}%
\pgfpathlineto{\pgfqpoint{4.323200in}{2.297795in}}%
\pgfpathlineto{\pgfqpoint{4.324440in}{2.297328in}}%
\pgfpathlineto{\pgfqpoint{4.326920in}{2.291748in}}%
\pgfpathlineto{\pgfqpoint{4.328160in}{2.290975in}}%
\pgfpathlineto{\pgfqpoint{4.330640in}{2.286360in}}%
\pgfpathlineto{\pgfqpoint{4.334360in}{2.281187in}}%
\pgfpathlineto{\pgfqpoint{4.339320in}{2.280968in}}%
\pgfpathlineto{\pgfqpoint{4.341800in}{2.279428in}}%
\pgfpathlineto{\pgfqpoint{4.344280in}{2.286383in}}%
\pgfpathlineto{\pgfqpoint{4.349240in}{2.279423in}}%
\pgfpathlineto{\pgfqpoint{4.351720in}{2.279560in}}%
\pgfpathlineto{\pgfqpoint{4.355440in}{2.275750in}}%
\pgfpathlineto{\pgfqpoint{4.359160in}{2.283249in}}%
\pgfpathlineto{\pgfqpoint{4.360400in}{2.281782in}}%
\pgfpathlineto{\pgfqpoint{4.362880in}{2.286040in}}%
\pgfpathlineto{\pgfqpoint{4.366600in}{2.279557in}}%
\pgfpathlineto{\pgfqpoint{4.367840in}{2.280997in}}%
\pgfpathlineto{\pgfqpoint{4.371560in}{2.279694in}}%
\pgfpathlineto{\pgfqpoint{4.374040in}{2.278169in}}%
\pgfpathlineto{\pgfqpoint{4.375280in}{2.275864in}}%
\pgfpathlineto{\pgfqpoint{4.376520in}{2.276246in}}%
\pgfpathlineto{\pgfqpoint{4.379000in}{2.278204in}}%
\pgfpathlineto{\pgfqpoint{4.381480in}{2.278550in}}%
\pgfpathlineto{\pgfqpoint{4.383960in}{2.281218in}}%
\pgfpathlineto{\pgfqpoint{4.385200in}{2.281442in}}%
\pgfpathlineto{\pgfqpoint{4.387680in}{2.279295in}}%
\pgfpathlineto{\pgfqpoint{4.390160in}{2.277888in}}%
\pgfpathlineto{\pgfqpoint{4.392640in}{2.283915in}}%
\pgfpathlineto{\pgfqpoint{4.393880in}{2.285046in}}%
\pgfpathlineto{\pgfqpoint{4.396360in}{2.279854in}}%
\pgfpathlineto{\pgfqpoint{4.398840in}{2.284901in}}%
\pgfpathlineto{\pgfqpoint{4.401320in}{2.291513in}}%
\pgfpathlineto{\pgfqpoint{4.403800in}{2.292146in}}%
\pgfpathlineto{\pgfqpoint{4.407520in}{2.290263in}}%
\pgfpathlineto{\pgfqpoint{4.410000in}{2.283192in}}%
\pgfpathlineto{\pgfqpoint{4.411240in}{2.282692in}}%
\pgfpathlineto{\pgfqpoint{4.412480in}{2.283471in}}%
\pgfpathlineto{\pgfqpoint{4.414960in}{2.287196in}}%
\pgfpathlineto{\pgfqpoint{4.418680in}{2.290937in}}%
\pgfpathlineto{\pgfqpoint{4.419920in}{2.288759in}}%
\pgfpathlineto{\pgfqpoint{4.421160in}{2.288872in}}%
\pgfpathlineto{\pgfqpoint{4.423640in}{2.292073in}}%
\pgfpathlineto{\pgfqpoint{4.429840in}{2.288929in}}%
\pgfpathlineto{\pgfqpoint{4.432320in}{2.291964in}}%
\pgfpathlineto{\pgfqpoint{4.433560in}{2.291506in}}%
\pgfpathlineto{\pgfqpoint{4.436040in}{2.292017in}}%
\pgfpathlineto{\pgfqpoint{4.438520in}{2.292449in}}%
\pgfpathlineto{\pgfqpoint{4.441000in}{2.296834in}}%
\pgfpathlineto{\pgfqpoint{4.448440in}{2.304682in}}%
\pgfpathlineto{\pgfqpoint{4.454640in}{2.293060in}}%
\pgfpathlineto{\pgfqpoint{4.458360in}{2.288174in}}%
\pgfpathlineto{\pgfqpoint{4.462080in}{2.289282in}}%
\pgfpathlineto{\pgfqpoint{4.465800in}{2.287374in}}%
\pgfpathlineto{\pgfqpoint{4.468280in}{2.293961in}}%
\pgfpathlineto{\pgfqpoint{4.472000in}{2.286974in}}%
\pgfpathlineto{\pgfqpoint{4.475720in}{2.289818in}}%
\pgfpathlineto{\pgfqpoint{4.478200in}{2.288701in}}%
\pgfpathlineto{\pgfqpoint{4.479440in}{2.287001in}}%
\pgfpathlineto{\pgfqpoint{4.483160in}{2.295484in}}%
\pgfpathlineto{\pgfqpoint{4.484400in}{2.293828in}}%
\pgfpathlineto{\pgfqpoint{4.486880in}{2.298645in}}%
\pgfpathlineto{\pgfqpoint{4.490600in}{2.293098in}}%
\pgfpathlineto{\pgfqpoint{4.491840in}{2.293771in}}%
\pgfpathlineto{\pgfqpoint{4.493080in}{2.292245in}}%
\pgfpathlineto{\pgfqpoint{4.495560in}{2.292761in}}%
\pgfpathlineto{\pgfqpoint{4.498040in}{2.290403in}}%
\pgfpathlineto{\pgfqpoint{4.499280in}{2.287984in}}%
\pgfpathlineto{\pgfqpoint{4.505480in}{2.289586in}}%
\pgfpathlineto{\pgfqpoint{4.507960in}{2.291653in}}%
\pgfpathlineto{\pgfqpoint{4.509200in}{2.291187in}}%
\pgfpathlineto{\pgfqpoint{4.511680in}{2.288413in}}%
\pgfpathlineto{\pgfqpoint{4.512920in}{2.287129in}}%
\pgfpathlineto{\pgfqpoint{4.514160in}{2.287715in}}%
\pgfpathlineto{\pgfqpoint{4.516640in}{2.294112in}}%
\pgfpathlineto{\pgfqpoint{4.517880in}{2.296289in}}%
\pgfpathlineto{\pgfqpoint{4.520360in}{2.290468in}}%
\pgfpathlineto{\pgfqpoint{4.522840in}{2.294410in}}%
\pgfpathlineto{\pgfqpoint{4.525320in}{2.301687in}}%
\pgfpathlineto{\pgfqpoint{4.529040in}{2.301364in}}%
\pgfpathlineto{\pgfqpoint{4.531520in}{2.299724in}}%
\pgfpathlineto{\pgfqpoint{4.534000in}{2.292578in}}%
\pgfpathlineto{\pgfqpoint{4.535240in}{2.291958in}}%
\pgfpathlineto{\pgfqpoint{4.538960in}{2.295215in}}%
\pgfpathlineto{\pgfqpoint{4.542680in}{2.299786in}}%
\pgfpathlineto{\pgfqpoint{4.545160in}{2.297254in}}%
\pgfpathlineto{\pgfqpoint{4.547640in}{2.300413in}}%
\pgfpathlineto{\pgfqpoint{4.552600in}{2.296633in}}%
\pgfpathlineto{\pgfqpoint{4.553840in}{2.296681in}}%
\pgfpathlineto{\pgfqpoint{4.556320in}{2.299832in}}%
\pgfpathlineto{\pgfqpoint{4.557560in}{2.299463in}}%
\pgfpathlineto{\pgfqpoint{4.560040in}{2.300926in}}%
\pgfpathlineto{\pgfqpoint{4.562520in}{2.299911in}}%
\pgfpathlineto{\pgfqpoint{4.565000in}{2.304993in}}%
\pgfpathlineto{\pgfqpoint{4.568720in}{2.311719in}}%
\pgfpathlineto{\pgfqpoint{4.571200in}{2.313896in}}%
\pgfpathlineto{\pgfqpoint{4.572440in}{2.313427in}}%
\pgfpathlineto{\pgfqpoint{4.578640in}{2.302601in}}%
\pgfpathlineto{\pgfqpoint{4.583600in}{2.297620in}}%
\pgfpathlineto{\pgfqpoint{4.586080in}{2.298575in}}%
\pgfpathlineto{\pgfqpoint{4.589800in}{2.297147in}}%
\pgfpathlineto{\pgfqpoint{4.592280in}{2.303009in}}%
\pgfpathlineto{\pgfqpoint{4.596000in}{2.295460in}}%
\pgfpathlineto{\pgfqpoint{4.597240in}{2.298611in}}%
\pgfpathlineto{\pgfqpoint{4.602200in}{2.297916in}}%
\pgfpathlineto{\pgfqpoint{4.603440in}{2.295558in}}%
\pgfpathlineto{\pgfqpoint{4.607160in}{2.305319in}}%
\pgfpathlineto{\pgfqpoint{4.608400in}{2.304273in}}%
\pgfpathlineto{\pgfqpoint{4.610880in}{2.309890in}}%
\pgfpathlineto{\pgfqpoint{4.614600in}{2.303874in}}%
\pgfpathlineto{\pgfqpoint{4.615840in}{2.303872in}}%
\pgfpathlineto{\pgfqpoint{4.618320in}{2.302711in}}%
\pgfpathlineto{\pgfqpoint{4.619560in}{2.302504in}}%
\pgfpathlineto{\pgfqpoint{4.624520in}{2.297135in}}%
\pgfpathlineto{\pgfqpoint{4.627000in}{2.297949in}}%
\pgfpathlineto{\pgfqpoint{4.629480in}{2.298928in}}%
\pgfpathlineto{\pgfqpoint{4.630720in}{2.300961in}}%
\pgfpathlineto{\pgfqpoint{4.638160in}{2.298368in}}%
\pgfpathlineto{\pgfqpoint{4.640640in}{2.305097in}}%
\pgfpathlineto{\pgfqpoint{4.641880in}{2.307108in}}%
\pgfpathlineto{\pgfqpoint{4.644360in}{2.300872in}}%
\pgfpathlineto{\pgfqpoint{4.646840in}{2.304848in}}%
\pgfpathlineto{\pgfqpoint{4.649320in}{2.312442in}}%
\pgfpathlineto{\pgfqpoint{4.651800in}{2.313032in}}%
\pgfpathlineto{\pgfqpoint{4.655520in}{2.309947in}}%
\pgfpathlineto{\pgfqpoint{4.658000in}{2.302308in}}%
\pgfpathlineto{\pgfqpoint{4.659240in}{2.300916in}}%
\pgfpathlineto{\pgfqpoint{4.660480in}{2.301397in}}%
\pgfpathlineto{\pgfqpoint{4.662960in}{2.303358in}}%
\pgfpathlineto{\pgfqpoint{4.666680in}{2.307784in}}%
\pgfpathlineto{\pgfqpoint{4.667920in}{2.306291in}}%
\pgfpathlineto{\pgfqpoint{4.669160in}{2.304711in}}%
\pgfpathlineto{\pgfqpoint{4.671640in}{2.307660in}}%
\pgfpathlineto{\pgfqpoint{4.675360in}{2.305417in}}%
\pgfpathlineto{\pgfqpoint{4.677840in}{2.306077in}}%
\pgfpathlineto{\pgfqpoint{4.680320in}{2.309815in}}%
\pgfpathlineto{\pgfqpoint{4.682800in}{2.309239in}}%
\pgfpathlineto{\pgfqpoint{4.684040in}{2.310087in}}%
\pgfpathlineto{\pgfqpoint{4.686520in}{2.307705in}}%
\pgfpathlineto{\pgfqpoint{4.690240in}{2.314534in}}%
\pgfpathlineto{\pgfqpoint{4.691480in}{2.316134in}}%
\pgfpathlineto{\pgfqpoint{4.693960in}{2.322626in}}%
\pgfpathlineto{\pgfqpoint{4.695200in}{2.323433in}}%
\pgfpathlineto{\pgfqpoint{4.697680in}{2.318395in}}%
\pgfpathlineto{\pgfqpoint{4.701400in}{2.311900in}}%
\pgfpathlineto{\pgfqpoint{4.707600in}{2.304492in}}%
\pgfpathlineto{\pgfqpoint{4.710080in}{2.304642in}}%
\pgfpathlineto{\pgfqpoint{4.713800in}{2.303245in}}%
\pgfpathlineto{\pgfqpoint{4.716280in}{2.308958in}}%
\pgfpathlineto{\pgfqpoint{4.720000in}{2.300940in}}%
\pgfpathlineto{\pgfqpoint{4.721240in}{2.301713in}}%
\pgfpathlineto{\pgfqpoint{4.723720in}{2.303853in}}%
\pgfpathlineto{\pgfqpoint{4.726200in}{2.302853in}}%
\pgfpathlineto{\pgfqpoint{4.727440in}{2.300620in}}%
\pgfpathlineto{\pgfqpoint{4.731160in}{2.310555in}}%
\pgfpathlineto{\pgfqpoint{4.732400in}{2.309900in}}%
\pgfpathlineto{\pgfqpoint{4.734880in}{2.316600in}}%
\pgfpathlineto{\pgfqpoint{4.739840in}{2.312053in}}%
\pgfpathlineto{\pgfqpoint{4.742320in}{2.311460in}}%
\pgfpathlineto{\pgfqpoint{4.743560in}{2.311577in}}%
\pgfpathlineto{\pgfqpoint{4.748520in}{2.305407in}}%
\pgfpathlineto{\pgfqpoint{4.751000in}{2.307449in}}%
\pgfpathlineto{\pgfqpoint{4.753480in}{2.308126in}}%
\pgfpathlineto{\pgfqpoint{4.754720in}{2.309897in}}%
\pgfpathlineto{\pgfqpoint{4.759680in}{2.308013in}}%
\pgfpathlineto{\pgfqpoint{4.760920in}{2.306297in}}%
\pgfpathlineto{\pgfqpoint{4.762160in}{2.306736in}}%
\pgfpathlineto{\pgfqpoint{4.764640in}{2.312825in}}%
\pgfpathlineto{\pgfqpoint{4.765880in}{2.314367in}}%
\pgfpathlineto{\pgfqpoint{4.768360in}{2.308441in}}%
\pgfpathlineto{\pgfqpoint{4.770840in}{2.313206in}}%
\pgfpathlineto{\pgfqpoint{4.772080in}{2.318901in}}%
\pgfpathlineto{\pgfqpoint{4.775800in}{2.319945in}}%
\pgfpathlineto{\pgfqpoint{4.777040in}{2.319393in}}%
\pgfpathlineto{\pgfqpoint{4.783240in}{2.307498in}}%
\pgfpathlineto{\pgfqpoint{4.790680in}{2.312232in}}%
\pgfpathlineto{\pgfqpoint{4.793160in}{2.309000in}}%
\pgfpathlineto{\pgfqpoint{4.796880in}{2.311538in}}%
\pgfpathlineto{\pgfqpoint{4.799360in}{2.310748in}}%
\pgfpathlineto{\pgfqpoint{4.805560in}{2.314099in}}%
\pgfpathlineto{\pgfqpoint{4.809280in}{2.313048in}}%
\pgfpathlineto{\pgfqpoint{4.810520in}{2.311394in}}%
\pgfpathlineto{\pgfqpoint{4.813000in}{2.314651in}}%
\pgfpathlineto{\pgfqpoint{4.815480in}{2.318106in}}%
\pgfpathlineto{\pgfqpoint{4.817960in}{2.325096in}}%
\pgfpathlineto{\pgfqpoint{4.819200in}{2.325946in}}%
\pgfpathlineto{\pgfqpoint{4.821680in}{2.321094in}}%
\pgfpathlineto{\pgfqpoint{4.822920in}{2.319059in}}%
\pgfpathlineto{\pgfqpoint{4.824160in}{2.319614in}}%
\pgfpathlineto{\pgfqpoint{4.829120in}{2.312478in}}%
\pgfpathlineto{\pgfqpoint{4.834080in}{2.309165in}}%
\pgfpathlineto{\pgfqpoint{4.835320in}{2.308153in}}%
\pgfpathlineto{\pgfqpoint{4.836560in}{2.305659in}}%
\pgfpathlineto{\pgfqpoint{4.837800in}{2.306050in}}%
\pgfpathlineto{\pgfqpoint{4.840280in}{2.312886in}}%
\pgfpathlineto{\pgfqpoint{4.844000in}{2.305210in}}%
\pgfpathlineto{\pgfqpoint{4.845240in}{2.302228in}}%
\pgfpathlineto{\pgfqpoint{4.847720in}{2.303313in}}%
\pgfpathlineto{\pgfqpoint{4.850200in}{2.301907in}}%
\pgfpathlineto{\pgfqpoint{4.851440in}{2.299740in}}%
\pgfpathlineto{\pgfqpoint{4.855160in}{2.308836in}}%
\pgfpathlineto{\pgfqpoint{4.856400in}{2.309160in}}%
\pgfpathlineto{\pgfqpoint{4.858880in}{2.316055in}}%
\pgfpathlineto{\pgfqpoint{4.860120in}{2.314511in}}%
\pgfpathlineto{\pgfqpoint{4.861360in}{2.314512in}}%
\pgfpathlineto{\pgfqpoint{4.863840in}{2.311624in}}%
\pgfpathlineto{\pgfqpoint{4.865080in}{2.311241in}}%
\pgfpathlineto{\pgfqpoint{4.867560in}{2.312647in}}%
\pgfpathlineto{\pgfqpoint{4.871280in}{2.307565in}}%
\pgfpathlineto{\pgfqpoint{4.872520in}{2.307818in}}%
\pgfpathlineto{\pgfqpoint{4.873760in}{2.309645in}}%
\pgfpathlineto{\pgfqpoint{4.877480in}{2.309091in}}%
\pgfpathlineto{\pgfqpoint{4.878720in}{2.310537in}}%
\pgfpathlineto{\pgfqpoint{4.883680in}{2.308666in}}%
\pgfpathlineto{\pgfqpoint{4.886160in}{2.307234in}}%
\pgfpathlineto{\pgfqpoint{4.888640in}{2.313320in}}%
\pgfpathlineto{\pgfqpoint{4.889880in}{2.315124in}}%
\pgfpathlineto{\pgfqpoint{4.892360in}{2.309713in}}%
\pgfpathlineto{\pgfqpoint{4.897320in}{2.320700in}}%
\pgfpathlineto{\pgfqpoint{4.899800in}{2.321317in}}%
\pgfpathlineto{\pgfqpoint{4.903520in}{2.317359in}}%
\pgfpathlineto{\pgfqpoint{4.907240in}{2.306722in}}%
\pgfpathlineto{\pgfqpoint{4.910960in}{2.307587in}}%
\pgfpathlineto{\pgfqpoint{4.913440in}{2.308870in}}%
\pgfpathlineto{\pgfqpoint{4.914680in}{2.310683in}}%
\pgfpathlineto{\pgfqpoint{4.917160in}{2.308467in}}%
\pgfpathlineto{\pgfqpoint{4.919640in}{2.311529in}}%
\pgfpathlineto{\pgfqpoint{4.922120in}{2.309767in}}%
\pgfpathlineto{\pgfqpoint{4.925840in}{2.311372in}}%
\pgfpathlineto{\pgfqpoint{4.928320in}{2.315431in}}%
\pgfpathlineto{\pgfqpoint{4.930800in}{2.313864in}}%
\pgfpathlineto{\pgfqpoint{4.932040in}{2.315000in}}%
\pgfpathlineto{\pgfqpoint{4.934520in}{2.312422in}}%
\pgfpathlineto{\pgfqpoint{4.937000in}{2.316722in}}%
\pgfpathlineto{\pgfqpoint{4.939480in}{2.318820in}}%
\pgfpathlineto{\pgfqpoint{4.941960in}{2.326467in}}%
\pgfpathlineto{\pgfqpoint{4.943200in}{2.327525in}}%
\pgfpathlineto{\pgfqpoint{4.944440in}{2.326030in}}%
\pgfpathlineto{\pgfqpoint{4.946920in}{2.320344in}}%
\pgfpathlineto{\pgfqpoint{4.948160in}{2.320908in}}%
\pgfpathlineto{\pgfqpoint{4.951880in}{2.315382in}}%
\pgfpathlineto{\pgfqpoint{4.955600in}{2.313183in}}%
\pgfpathlineto{\pgfqpoint{4.959320in}{2.310711in}}%
\pgfpathlineto{\pgfqpoint{4.960560in}{2.308036in}}%
\pgfpathlineto{\pgfqpoint{4.961800in}{2.308221in}}%
\pgfpathlineto{\pgfqpoint{4.964280in}{2.314577in}}%
\pgfpathlineto{\pgfqpoint{4.966760in}{2.309303in}}%
\pgfpathlineto{\pgfqpoint{4.969240in}{2.304282in}}%
\pgfpathlineto{\pgfqpoint{4.971720in}{2.305852in}}%
\pgfpathlineto{\pgfqpoint{4.974200in}{2.304095in}}%
\pgfpathlineto{\pgfqpoint{4.975440in}{2.301347in}}%
\pgfpathlineto{\pgfqpoint{4.979160in}{2.308982in}}%
\pgfpathlineto{\pgfqpoint{4.980400in}{2.309486in}}%
\pgfpathlineto{\pgfqpoint{4.982880in}{2.315295in}}%
\pgfpathlineto{\pgfqpoint{4.984120in}{2.313760in}}%
\pgfpathlineto{\pgfqpoint{4.985360in}{2.315143in}}%
\pgfpathlineto{\pgfqpoint{4.989080in}{2.312395in}}%
\pgfpathlineto{\pgfqpoint{4.991560in}{2.313218in}}%
\pgfpathlineto{\pgfqpoint{4.995280in}{2.307441in}}%
\pgfpathlineto{\pgfqpoint{5.001480in}{2.306488in}}%
\pgfpathlineto{\pgfqpoint{5.003960in}{2.308359in}}%
\pgfpathlineto{\pgfqpoint{5.005200in}{2.308867in}}%
\pgfpathlineto{\pgfqpoint{5.010160in}{2.306332in}}%
\pgfpathlineto{\pgfqpoint{5.013880in}{2.314283in}}%
\pgfpathlineto{\pgfqpoint{5.016360in}{2.308428in}}%
\pgfpathlineto{\pgfqpoint{5.020080in}{2.320338in}}%
\pgfpathlineto{\pgfqpoint{5.025040in}{2.318645in}}%
\pgfpathlineto{\pgfqpoint{5.028760in}{2.311274in}}%
\pgfpathlineto{\pgfqpoint{5.031240in}{2.305793in}}%
\pgfpathlineto{\pgfqpoint{5.033720in}{2.307268in}}%
\pgfpathlineto{\pgfqpoint{5.034960in}{2.306953in}}%
\pgfpathlineto{\pgfqpoint{5.038680in}{2.312697in}}%
\pgfpathlineto{\pgfqpoint{5.041160in}{2.310506in}}%
\pgfpathlineto{\pgfqpoint{5.043640in}{2.313978in}}%
\pgfpathlineto{\pgfqpoint{5.046120in}{2.311313in}}%
\pgfpathlineto{\pgfqpoint{5.047360in}{2.311061in}}%
\pgfpathlineto{\pgfqpoint{5.048600in}{2.312207in}}%
\pgfpathlineto{\pgfqpoint{5.049840in}{2.311638in}}%
\pgfpathlineto{\pgfqpoint{5.052320in}{2.316848in}}%
\pgfpathlineto{\pgfqpoint{5.054800in}{2.316343in}}%
\pgfpathlineto{\pgfqpoint{5.056040in}{2.317580in}}%
\pgfpathlineto{\pgfqpoint{5.058520in}{2.314984in}}%
\pgfpathlineto{\pgfqpoint{5.059760in}{2.317741in}}%
\pgfpathlineto{\pgfqpoint{5.062240in}{2.317775in}}%
\pgfpathlineto{\pgfqpoint{5.063480in}{2.319294in}}%
\pgfpathlineto{\pgfqpoint{5.067200in}{2.329065in}}%
\pgfpathlineto{\pgfqpoint{5.069680in}{2.324510in}}%
\pgfpathlineto{\pgfqpoint{5.070920in}{2.322613in}}%
\pgfpathlineto{\pgfqpoint{5.072160in}{2.323749in}}%
\pgfpathlineto{\pgfqpoint{5.075880in}{2.318572in}}%
\pgfpathlineto{\pgfqpoint{5.078360in}{2.317123in}}%
\pgfpathlineto{\pgfqpoint{5.079600in}{2.318038in}}%
\pgfpathlineto{\pgfqpoint{5.083320in}{2.315615in}}%
\pgfpathlineto{\pgfqpoint{5.085800in}{2.312313in}}%
\pgfpathlineto{\pgfqpoint{5.088280in}{2.317760in}}%
\pgfpathlineto{\pgfqpoint{5.090760in}{2.313266in}}%
\pgfpathlineto{\pgfqpoint{5.092000in}{2.308680in}}%
\pgfpathlineto{\pgfqpoint{5.093240in}{2.308370in}}%
\pgfpathlineto{\pgfqpoint{5.094480in}{2.309661in}}%
\pgfpathlineto{\pgfqpoint{5.098200in}{2.307505in}}%
\pgfpathlineto{\pgfqpoint{5.099440in}{2.304278in}}%
\pgfpathlineto{\pgfqpoint{5.103160in}{2.311501in}}%
\pgfpathlineto{\pgfqpoint{5.104400in}{2.311723in}}%
\pgfpathlineto{\pgfqpoint{5.106880in}{2.319400in}}%
\pgfpathlineto{\pgfqpoint{5.108120in}{2.318586in}}%
\pgfpathlineto{\pgfqpoint{5.109360in}{2.320527in}}%
\pgfpathlineto{\pgfqpoint{5.114320in}{2.317495in}}%
\pgfpathlineto{\pgfqpoint{5.115560in}{2.317652in}}%
\pgfpathlineto{\pgfqpoint{5.118040in}{2.314437in}}%
\pgfpathlineto{\pgfqpoint{5.120520in}{2.312163in}}%
\pgfpathlineto{\pgfqpoint{5.125480in}{2.312422in}}%
\pgfpathlineto{\pgfqpoint{5.127960in}{2.314202in}}%
\pgfpathlineto{\pgfqpoint{5.130440in}{2.315109in}}%
\pgfpathlineto{\pgfqpoint{5.134160in}{2.314958in}}%
\pgfpathlineto{\pgfqpoint{5.137880in}{2.322919in}}%
\pgfpathlineto{\pgfqpoint{5.140360in}{2.317513in}}%
\pgfpathlineto{\pgfqpoint{5.144080in}{2.326297in}}%
\pgfpathlineto{\pgfqpoint{5.149040in}{2.324523in}}%
\pgfpathlineto{\pgfqpoint{5.152760in}{2.317491in}}%
\pgfpathlineto{\pgfqpoint{5.155240in}{2.311241in}}%
\pgfpathlineto{\pgfqpoint{5.158960in}{2.312362in}}%
\pgfpathlineto{\pgfqpoint{5.162680in}{2.317528in}}%
\pgfpathlineto{\pgfqpoint{5.165160in}{2.315863in}}%
\pgfpathlineto{\pgfqpoint{5.167640in}{2.319013in}}%
\pgfpathlineto{\pgfqpoint{5.170120in}{2.315622in}}%
\pgfpathlineto{\pgfqpoint{5.171360in}{2.315130in}}%
\pgfpathlineto{\pgfqpoint{5.172600in}{2.316025in}}%
\pgfpathlineto{\pgfqpoint{5.173840in}{2.315472in}}%
\pgfpathlineto{\pgfqpoint{5.176320in}{2.321202in}}%
\pgfpathlineto{\pgfqpoint{5.177560in}{2.320278in}}%
\pgfpathlineto{\pgfqpoint{5.178800in}{2.320032in}}%
\pgfpathlineto{\pgfqpoint{5.180040in}{2.321878in}}%
\pgfpathlineto{\pgfqpoint{5.182520in}{2.318310in}}%
\pgfpathlineto{\pgfqpoint{5.183760in}{2.321385in}}%
\pgfpathlineto{\pgfqpoint{5.185000in}{2.321108in}}%
\pgfpathlineto{\pgfqpoint{5.187480in}{2.324209in}}%
\pgfpathlineto{\pgfqpoint{5.191200in}{2.333686in}}%
\pgfpathlineto{\pgfqpoint{5.194920in}{2.327298in}}%
\pgfpathlineto{\pgfqpoint{5.196160in}{2.329043in}}%
\pgfpathlineto{\pgfqpoint{5.198640in}{2.323525in}}%
\pgfpathlineto{\pgfqpoint{5.199880in}{2.323187in}}%
\pgfpathlineto{\pgfqpoint{5.202360in}{2.320829in}}%
\pgfpathlineto{\pgfqpoint{5.206080in}{2.320279in}}%
\pgfpathlineto{\pgfqpoint{5.209800in}{2.315055in}}%
\pgfpathlineto{\pgfqpoint{5.212280in}{2.320431in}}%
\pgfpathlineto{\pgfqpoint{5.213520in}{2.319577in}}%
\pgfpathlineto{\pgfqpoint{5.216000in}{2.311952in}}%
\pgfpathlineto{\pgfqpoint{5.218480in}{2.318938in}}%
\pgfpathlineto{\pgfqpoint{5.219720in}{2.318816in}}%
\pgfpathlineto{\pgfqpoint{5.222200in}{2.315528in}}%
\pgfpathlineto{\pgfqpoint{5.223440in}{2.311784in}}%
\pgfpathlineto{\pgfqpoint{5.227160in}{2.320583in}}%
\pgfpathlineto{\pgfqpoint{5.228400in}{2.319914in}}%
\pgfpathlineto{\pgfqpoint{5.233360in}{2.327448in}}%
\pgfpathlineto{\pgfqpoint{5.237080in}{2.324437in}}%
\pgfpathlineto{\pgfqpoint{5.239560in}{2.326406in}}%
\pgfpathlineto{\pgfqpoint{5.242040in}{2.322886in}}%
\pgfpathlineto{\pgfqpoint{5.244520in}{2.321340in}}%
\pgfpathlineto{\pgfqpoint{5.248240in}{2.321611in}}%
\pgfpathlineto{\pgfqpoint{5.251960in}{2.325179in}}%
\pgfpathlineto{\pgfqpoint{5.258160in}{2.326380in}}%
\pgfpathlineto{\pgfqpoint{5.261880in}{2.335427in}}%
\pgfpathlineto{\pgfqpoint{5.264360in}{2.328452in}}%
\pgfpathlineto{\pgfqpoint{5.268080in}{2.335164in}}%
\pgfpathlineto{\pgfqpoint{5.274280in}{2.332404in}}%
\pgfpathlineto{\pgfqpoint{5.275520in}{2.331048in}}%
\pgfpathlineto{\pgfqpoint{5.279240in}{2.319379in}}%
\pgfpathlineto{\pgfqpoint{5.282960in}{2.321391in}}%
\pgfpathlineto{\pgfqpoint{5.286680in}{2.326550in}}%
\pgfpathlineto{\pgfqpoint{5.289160in}{2.325257in}}%
\pgfpathlineto{\pgfqpoint{5.291640in}{2.328221in}}%
\pgfpathlineto{\pgfqpoint{5.294120in}{2.322556in}}%
\pgfpathlineto{\pgfqpoint{5.295360in}{2.321807in}}%
\pgfpathlineto{\pgfqpoint{5.296600in}{2.323124in}}%
\pgfpathlineto{\pgfqpoint{5.297840in}{2.322674in}}%
\pgfpathlineto{\pgfqpoint{5.300320in}{2.327607in}}%
\pgfpathlineto{\pgfqpoint{5.302800in}{2.325989in}}%
\pgfpathlineto{\pgfqpoint{5.304040in}{2.328049in}}%
\pgfpathlineto{\pgfqpoint{5.306520in}{2.323476in}}%
\pgfpathlineto{\pgfqpoint{5.307760in}{2.325923in}}%
\pgfpathlineto{\pgfqpoint{5.309000in}{2.325205in}}%
\pgfpathlineto{\pgfqpoint{5.311480in}{2.327707in}}%
\pgfpathlineto{\pgfqpoint{5.315200in}{2.337088in}}%
\pgfpathlineto{\pgfqpoint{5.318920in}{2.332608in}}%
\pgfpathlineto{\pgfqpoint{5.320160in}{2.335194in}}%
\pgfpathlineto{\pgfqpoint{5.322640in}{2.331169in}}%
\pgfpathlineto{\pgfqpoint{5.325120in}{2.328665in}}%
\pgfpathlineto{\pgfqpoint{5.328840in}{2.326490in}}%
\pgfpathlineto{\pgfqpoint{5.330080in}{2.326280in}}%
\pgfpathlineto{\pgfqpoint{5.333800in}{2.321617in}}%
\pgfpathlineto{\pgfqpoint{5.336280in}{2.325708in}}%
\pgfpathlineto{\pgfqpoint{5.337520in}{2.324669in}}%
\pgfpathlineto{\pgfqpoint{5.340000in}{2.318296in}}%
\pgfpathlineto{\pgfqpoint{5.342480in}{2.322384in}}%
\pgfpathlineto{\pgfqpoint{5.344960in}{2.322057in}}%
\pgfpathlineto{\pgfqpoint{5.347440in}{2.315164in}}%
\pgfpathlineto{\pgfqpoint{5.351160in}{2.323680in}}%
\pgfpathlineto{\pgfqpoint{5.352400in}{2.323384in}}%
\pgfpathlineto{\pgfqpoint{5.354880in}{2.330942in}}%
\pgfpathlineto{\pgfqpoint{5.356120in}{2.330407in}}%
\pgfpathlineto{\pgfqpoint{5.357360in}{2.331821in}}%
\pgfpathlineto{\pgfqpoint{5.361080in}{2.326651in}}%
\pgfpathlineto{\pgfqpoint{5.363560in}{2.327119in}}%
\pgfpathlineto{\pgfqpoint{5.364800in}{2.324819in}}%
\pgfpathlineto{\pgfqpoint{5.366040in}{2.325142in}}%
\pgfpathlineto{\pgfqpoint{5.367280in}{2.322961in}}%
\pgfpathlineto{\pgfqpoint{5.369760in}{2.322931in}}%
\pgfpathlineto{\pgfqpoint{5.372240in}{2.322664in}}%
\pgfpathlineto{\pgfqpoint{5.378440in}{2.328917in}}%
\pgfpathlineto{\pgfqpoint{5.380920in}{2.330116in}}%
\pgfpathlineto{\pgfqpoint{5.382160in}{2.330377in}}%
\pgfpathlineto{\pgfqpoint{5.385880in}{2.338079in}}%
\pgfpathlineto{\pgfqpoint{5.388360in}{2.332648in}}%
\pgfpathlineto{\pgfqpoint{5.389600in}{2.335692in}}%
\pgfpathlineto{\pgfqpoint{5.390840in}{2.335453in}}%
\pgfpathlineto{\pgfqpoint{5.392080in}{2.340016in}}%
\pgfpathlineto{\pgfqpoint{5.397040in}{2.339175in}}%
\pgfpathlineto{\pgfqpoint{5.399520in}{2.337206in}}%
\pgfpathlineto{\pgfqpoint{5.403240in}{2.325364in}}%
\pgfpathlineto{\pgfqpoint{5.405720in}{2.326587in}}%
\pgfpathlineto{\pgfqpoint{5.408200in}{2.328509in}}%
\pgfpathlineto{\pgfqpoint{5.410680in}{2.331158in}}%
\pgfpathlineto{\pgfqpoint{5.413160in}{2.327568in}}%
\pgfpathlineto{\pgfqpoint{5.415640in}{2.331524in}}%
\pgfpathlineto{\pgfqpoint{5.419360in}{2.324997in}}%
\pgfpathlineto{\pgfqpoint{5.420600in}{2.325781in}}%
\pgfpathlineto{\pgfqpoint{5.421840in}{2.324180in}}%
\pgfpathlineto{\pgfqpoint{5.425560in}{2.328913in}}%
\pgfpathlineto{\pgfqpoint{5.426800in}{2.328446in}}%
\pgfpathlineto{\pgfqpoint{5.428040in}{2.330680in}}%
\pgfpathlineto{\pgfqpoint{5.430520in}{2.325537in}}%
\pgfpathlineto{\pgfqpoint{5.433000in}{2.327709in}}%
\pgfpathlineto{\pgfqpoint{5.435480in}{2.330518in}}%
\pgfpathlineto{\pgfqpoint{5.439200in}{2.339112in}}%
\pgfpathlineto{\pgfqpoint{5.442920in}{2.334089in}}%
\pgfpathlineto{\pgfqpoint{5.444160in}{2.337037in}}%
\pgfpathlineto{\pgfqpoint{5.447880in}{2.331588in}}%
\pgfpathlineto{\pgfqpoint{5.450360in}{2.327999in}}%
\pgfpathlineto{\pgfqpoint{5.455320in}{2.326027in}}%
\pgfpathlineto{\pgfqpoint{5.457800in}{2.322332in}}%
\pgfpathlineto{\pgfqpoint{5.461520in}{2.325789in}}%
\pgfpathlineto{\pgfqpoint{5.464000in}{2.320947in}}%
\pgfpathlineto{\pgfqpoint{5.466480in}{2.326071in}}%
\pgfpathlineto{\pgfqpoint{5.467720in}{2.326674in}}%
\pgfpathlineto{\pgfqpoint{5.468960in}{2.325665in}}%
\pgfpathlineto{\pgfqpoint{5.471440in}{2.318271in}}%
\pgfpathlineto{\pgfqpoint{5.475160in}{2.326520in}}%
\pgfpathlineto{\pgfqpoint{5.476400in}{2.326805in}}%
\pgfpathlineto{\pgfqpoint{5.477640in}{2.328842in}}%
\pgfpathlineto{\pgfqpoint{5.478880in}{2.333922in}}%
\pgfpathlineto{\pgfqpoint{5.480120in}{2.333775in}}%
\pgfpathlineto{\pgfqpoint{5.481360in}{2.335378in}}%
\pgfpathlineto{\pgfqpoint{5.483840in}{2.330007in}}%
\pgfpathlineto{\pgfqpoint{5.485080in}{2.328284in}}%
\pgfpathlineto{\pgfqpoint{5.487560in}{2.328824in}}%
\pgfpathlineto{\pgfqpoint{5.488800in}{2.326558in}}%
\pgfpathlineto{\pgfqpoint{5.490040in}{2.326872in}}%
\pgfpathlineto{\pgfqpoint{5.491280in}{2.325037in}}%
\pgfpathlineto{\pgfqpoint{5.492520in}{2.326353in}}%
\pgfpathlineto{\pgfqpoint{5.496240in}{2.324603in}}%
\pgfpathlineto{\pgfqpoint{5.498720in}{2.328117in}}%
\pgfpathlineto{\pgfqpoint{5.499960in}{2.327257in}}%
\pgfpathlineto{\pgfqpoint{5.502440in}{2.328339in}}%
\pgfpathlineto{\pgfqpoint{5.506160in}{2.327248in}}%
\pgfpathlineto{\pgfqpoint{5.509880in}{2.333829in}}%
\pgfpathlineto{\pgfqpoint{5.512360in}{2.329290in}}%
\pgfpathlineto{\pgfqpoint{5.513600in}{2.332194in}}%
\pgfpathlineto{\pgfqpoint{5.514840in}{2.331450in}}%
\pgfpathlineto{\pgfqpoint{5.517320in}{2.336454in}}%
\pgfpathlineto{\pgfqpoint{5.518560in}{2.336080in}}%
\pgfpathlineto{\pgfqpoint{5.521040in}{2.333710in}}%
\pgfpathlineto{\pgfqpoint{5.523520in}{2.332878in}}%
\pgfpathlineto{\pgfqpoint{5.527240in}{2.322813in}}%
\pgfpathlineto{\pgfqpoint{5.528480in}{2.322614in}}%
\pgfpathlineto{\pgfqpoint{5.534680in}{2.330522in}}%
\pgfpathlineto{\pgfqpoint{5.537160in}{2.326294in}}%
\pgfpathlineto{\pgfqpoint{5.539640in}{2.329718in}}%
\pgfpathlineto{\pgfqpoint{5.543360in}{2.322974in}}%
\pgfpathlineto{\pgfqpoint{5.544600in}{2.323087in}}%
\pgfpathlineto{\pgfqpoint{5.545840in}{2.320908in}}%
\pgfpathlineto{\pgfqpoint{5.549560in}{2.326532in}}%
\pgfpathlineto{\pgfqpoint{5.550800in}{2.325570in}}%
\pgfpathlineto{\pgfqpoint{5.552040in}{2.327207in}}%
\pgfpathlineto{\pgfqpoint{5.554520in}{2.321302in}}%
\pgfpathlineto{\pgfqpoint{5.557000in}{2.323954in}}%
\pgfpathlineto{\pgfqpoint{5.558240in}{2.324094in}}%
\pgfpathlineto{\pgfqpoint{5.559480in}{2.325673in}}%
\pgfpathlineto{\pgfqpoint{5.561960in}{2.331899in}}%
\pgfpathlineto{\pgfqpoint{5.563200in}{2.332826in}}%
\pgfpathlineto{\pgfqpoint{5.566920in}{2.330000in}}%
\pgfpathlineto{\pgfqpoint{5.568160in}{2.332672in}}%
\pgfpathlineto{\pgfqpoint{5.570640in}{2.328192in}}%
\pgfpathlineto{\pgfqpoint{5.571880in}{2.327205in}}%
\pgfpathlineto{\pgfqpoint{5.574360in}{2.324503in}}%
\pgfpathlineto{\pgfqpoint{5.578080in}{2.324050in}}%
\pgfpathlineto{\pgfqpoint{5.579320in}{2.323066in}}%
\pgfpathlineto{\pgfqpoint{5.581800in}{2.320146in}}%
\pgfpathlineto{\pgfqpoint{5.584280in}{2.323540in}}%
\pgfpathlineto{\pgfqpoint{5.585520in}{2.322389in}}%
\pgfpathlineto{\pgfqpoint{5.588000in}{2.316306in}}%
\pgfpathlineto{\pgfqpoint{5.589240in}{2.316616in}}%
\pgfpathlineto{\pgfqpoint{5.591720in}{2.320034in}}%
\pgfpathlineto{\pgfqpoint{5.592960in}{2.318830in}}%
\pgfpathlineto{\pgfqpoint{5.595440in}{2.310968in}}%
\pgfpathlineto{\pgfqpoint{5.599160in}{2.318861in}}%
\pgfpathlineto{\pgfqpoint{5.601640in}{2.321544in}}%
\pgfpathlineto{\pgfqpoint{5.602880in}{2.326656in}}%
\pgfpathlineto{\pgfqpoint{5.604120in}{2.326571in}}%
\pgfpathlineto{\pgfqpoint{5.605360in}{2.328663in}}%
\pgfpathlineto{\pgfqpoint{5.607840in}{2.323189in}}%
\pgfpathlineto{\pgfqpoint{5.609080in}{2.322184in}}%
\pgfpathlineto{\pgfqpoint{5.611560in}{2.322490in}}%
\pgfpathlineto{\pgfqpoint{5.615280in}{2.317278in}}%
\pgfpathlineto{\pgfqpoint{5.616520in}{2.318058in}}%
\pgfpathlineto{\pgfqpoint{5.620240in}{2.316001in}}%
\pgfpathlineto{\pgfqpoint{5.622720in}{2.320871in}}%
\pgfpathlineto{\pgfqpoint{5.623960in}{2.321149in}}%
\pgfpathlineto{\pgfqpoint{5.626440in}{2.323233in}}%
\pgfpathlineto{\pgfqpoint{5.628920in}{2.322982in}}%
\pgfpathlineto{\pgfqpoint{5.633880in}{2.327724in}}%
\pgfpathlineto{\pgfqpoint{5.636360in}{2.322645in}}%
\pgfpathlineto{\pgfqpoint{5.637600in}{2.325750in}}%
\pgfpathlineto{\pgfqpoint{5.638840in}{2.325055in}}%
\pgfpathlineto{\pgfqpoint{5.640080in}{2.331068in}}%
\pgfpathlineto{\pgfqpoint{5.641320in}{2.331259in}}%
\pgfpathlineto{\pgfqpoint{5.645040in}{2.327402in}}%
\pgfpathlineto{\pgfqpoint{5.647520in}{2.326679in}}%
\pgfpathlineto{\pgfqpoint{5.651240in}{2.317274in}}%
\pgfpathlineto{\pgfqpoint{5.652480in}{2.316418in}}%
\pgfpathlineto{\pgfqpoint{5.658680in}{2.324428in}}%
\pgfpathlineto{\pgfqpoint{5.661160in}{2.319736in}}%
\pgfpathlineto{\pgfqpoint{5.663640in}{2.324038in}}%
\pgfpathlineto{\pgfqpoint{5.667360in}{2.316651in}}%
\pgfpathlineto{\pgfqpoint{5.668600in}{2.318029in}}%
\pgfpathlineto{\pgfqpoint{5.669840in}{2.316378in}}%
\pgfpathlineto{\pgfqpoint{5.673560in}{2.321223in}}%
\pgfpathlineto{\pgfqpoint{5.674800in}{2.318912in}}%
\pgfpathlineto{\pgfqpoint{5.676040in}{2.319958in}}%
\pgfpathlineto{\pgfqpoint{5.678520in}{2.313471in}}%
\pgfpathlineto{\pgfqpoint{5.681000in}{2.318118in}}%
\pgfpathlineto{\pgfqpoint{5.683480in}{2.319656in}}%
\pgfpathlineto{\pgfqpoint{5.685960in}{2.325914in}}%
\pgfpathlineto{\pgfqpoint{5.688440in}{2.325883in}}%
\pgfpathlineto{\pgfqpoint{5.690920in}{2.323660in}}%
\pgfpathlineto{\pgfqpoint{5.692160in}{2.325331in}}%
\pgfpathlineto{\pgfqpoint{5.694640in}{2.319165in}}%
\pgfpathlineto{\pgfqpoint{5.695880in}{2.318481in}}%
\pgfpathlineto{\pgfqpoint{5.698360in}{2.314760in}}%
\pgfpathlineto{\pgfqpoint{5.703320in}{2.312524in}}%
\pgfpathlineto{\pgfqpoint{5.704560in}{2.309628in}}%
\pgfpathlineto{\pgfqpoint{5.705800in}{2.310068in}}%
\pgfpathlineto{\pgfqpoint{5.708280in}{2.313297in}}%
\pgfpathlineto{\pgfqpoint{5.709520in}{2.313162in}}%
\pgfpathlineto{\pgfqpoint{5.712000in}{2.308526in}}%
\pgfpathlineto{\pgfqpoint{5.714480in}{2.315815in}}%
\pgfpathlineto{\pgfqpoint{5.715720in}{2.318325in}}%
\pgfpathlineto{\pgfqpoint{5.716960in}{2.317126in}}%
\pgfpathlineto{\pgfqpoint{5.719440in}{2.308245in}}%
\pgfpathlineto{\pgfqpoint{5.723160in}{2.316559in}}%
\pgfpathlineto{\pgfqpoint{5.724400in}{2.316124in}}%
\pgfpathlineto{\pgfqpoint{5.725640in}{2.317704in}}%
\pgfpathlineto{\pgfqpoint{5.726880in}{2.322628in}}%
\pgfpathlineto{\pgfqpoint{5.729360in}{2.321213in}}%
\pgfpathlineto{\pgfqpoint{5.731840in}{2.315642in}}%
\pgfpathlineto{\pgfqpoint{5.733080in}{2.315969in}}%
\pgfpathlineto{\pgfqpoint{5.738040in}{2.311794in}}%
\pgfpathlineto{\pgfqpoint{5.739280in}{2.309951in}}%
\pgfpathlineto{\pgfqpoint{5.740520in}{2.311256in}}%
\pgfpathlineto{\pgfqpoint{5.744240in}{2.310260in}}%
\pgfpathlineto{\pgfqpoint{5.747960in}{2.317960in}}%
\pgfpathlineto{\pgfqpoint{5.750440in}{2.319531in}}%
\pgfpathlineto{\pgfqpoint{5.751680in}{2.319261in}}%
\pgfpathlineto{\pgfqpoint{5.752920in}{2.317719in}}%
\pgfpathlineto{\pgfqpoint{5.757880in}{2.322533in}}%
\pgfpathlineto{\pgfqpoint{5.760360in}{2.317048in}}%
\pgfpathlineto{\pgfqpoint{5.761600in}{2.318264in}}%
\pgfpathlineto{\pgfqpoint{5.762840in}{2.316887in}}%
\pgfpathlineto{\pgfqpoint{5.765320in}{2.324688in}}%
\pgfpathlineto{\pgfqpoint{5.767800in}{2.321935in}}%
\pgfpathlineto{\pgfqpoint{5.771520in}{2.323766in}}%
\pgfpathlineto{\pgfqpoint{5.775240in}{2.313100in}}%
\pgfpathlineto{\pgfqpoint{5.776480in}{2.312832in}}%
\pgfpathlineto{\pgfqpoint{5.777720in}{2.315095in}}%
\pgfpathlineto{\pgfqpoint{5.778960in}{2.314451in}}%
\pgfpathlineto{\pgfqpoint{5.782680in}{2.320989in}}%
\pgfpathlineto{\pgfqpoint{5.785160in}{2.316528in}}%
\pgfpathlineto{\pgfqpoint{5.787640in}{2.321149in}}%
\pgfpathlineto{\pgfqpoint{5.791360in}{2.311299in}}%
\pgfpathlineto{\pgfqpoint{5.792600in}{2.312509in}}%
\pgfpathlineto{\pgfqpoint{5.793840in}{2.310123in}}%
\pgfpathlineto{\pgfqpoint{5.797560in}{2.314929in}}%
\pgfpathlineto{\pgfqpoint{5.798800in}{2.313216in}}%
\pgfpathlineto{\pgfqpoint{5.800040in}{2.313485in}}%
\pgfpathlineto{\pgfqpoint{5.802520in}{2.306715in}}%
\pgfpathlineto{\pgfqpoint{5.803760in}{2.310573in}}%
\pgfpathlineto{\pgfqpoint{5.807480in}{2.312105in}}%
\pgfpathlineto{\pgfqpoint{5.811200in}{2.319969in}}%
\pgfpathlineto{\pgfqpoint{5.814920in}{2.315857in}}%
\pgfpathlineto{\pgfqpoint{5.816160in}{2.318341in}}%
\pgfpathlineto{\pgfqpoint{5.821120in}{2.308794in}}%
\pgfpathlineto{\pgfqpoint{5.829800in}{2.304715in}}%
\pgfpathlineto{\pgfqpoint{5.832280in}{2.308821in}}%
\pgfpathlineto{\pgfqpoint{5.833520in}{2.309489in}}%
\pgfpathlineto{\pgfqpoint{5.836000in}{2.303884in}}%
\pgfpathlineto{\pgfqpoint{5.838480in}{2.312025in}}%
\pgfpathlineto{\pgfqpoint{5.839720in}{2.314525in}}%
\pgfpathlineto{\pgfqpoint{5.840960in}{2.312933in}}%
\pgfpathlineto{\pgfqpoint{5.843440in}{2.303750in}}%
\pgfpathlineto{\pgfqpoint{5.847160in}{2.312942in}}%
\pgfpathlineto{\pgfqpoint{5.848400in}{2.311512in}}%
\pgfpathlineto{\pgfqpoint{5.849640in}{2.312442in}}%
\pgfpathlineto{\pgfqpoint{5.850880in}{2.317587in}}%
\pgfpathlineto{\pgfqpoint{5.853360in}{2.315283in}}%
\pgfpathlineto{\pgfqpoint{5.855840in}{2.309778in}}%
\pgfpathlineto{\pgfqpoint{5.858320in}{2.310206in}}%
\pgfpathlineto{\pgfqpoint{5.859560in}{2.309468in}}%
\pgfpathlineto{\pgfqpoint{5.863280in}{2.303448in}}%
\pgfpathlineto{\pgfqpoint{5.864520in}{2.304457in}}%
\pgfpathlineto{\pgfqpoint{5.867000in}{2.302457in}}%
\pgfpathlineto{\pgfqpoint{5.868240in}{2.303993in}}%
\pgfpathlineto{\pgfqpoint{5.871960in}{2.312241in}}%
\pgfpathlineto{\pgfqpoint{5.874440in}{2.314144in}}%
\pgfpathlineto{\pgfqpoint{5.875680in}{2.313408in}}%
\pgfpathlineto{\pgfqpoint{5.878160in}{2.311170in}}%
\pgfpathlineto{\pgfqpoint{5.881880in}{2.316841in}}%
\pgfpathlineto{\pgfqpoint{5.884360in}{2.311788in}}%
\pgfpathlineto{\pgfqpoint{5.885600in}{2.312974in}}%
\pgfpathlineto{\pgfqpoint{5.886840in}{2.309687in}}%
\pgfpathlineto{\pgfqpoint{5.889320in}{2.315610in}}%
\pgfpathlineto{\pgfqpoint{5.891800in}{2.313220in}}%
\pgfpathlineto{\pgfqpoint{5.894280in}{2.313854in}}%
\pgfpathlineto{\pgfqpoint{5.895520in}{2.313357in}}%
\pgfpathlineto{\pgfqpoint{5.898000in}{2.304959in}}%
\pgfpathlineto{\pgfqpoint{5.900480in}{2.302658in}}%
\pgfpathlineto{\pgfqpoint{5.901720in}{2.304236in}}%
\pgfpathlineto{\pgfqpoint{5.902960in}{2.303978in}}%
\pgfpathlineto{\pgfqpoint{5.906680in}{2.310149in}}%
\pgfpathlineto{\pgfqpoint{5.909160in}{2.305675in}}%
\pgfpathlineto{\pgfqpoint{5.911640in}{2.308877in}}%
\pgfpathlineto{\pgfqpoint{5.914120in}{2.298860in}}%
\pgfpathlineto{\pgfqpoint{5.915360in}{2.297872in}}%
\pgfpathlineto{\pgfqpoint{5.916600in}{2.299630in}}%
\pgfpathlineto{\pgfqpoint{5.917840in}{2.297543in}}%
\pgfpathlineto{\pgfqpoint{5.921560in}{2.301793in}}%
\pgfpathlineto{\pgfqpoint{5.926520in}{2.291867in}}%
\pgfpathlineto{\pgfqpoint{5.929000in}{2.295064in}}%
\pgfpathlineto{\pgfqpoint{5.930240in}{2.295123in}}%
\pgfpathlineto{\pgfqpoint{5.931480in}{2.296302in}}%
\pgfpathlineto{\pgfqpoint{5.935200in}{2.305831in}}%
\pgfpathlineto{\pgfqpoint{5.937680in}{2.304093in}}%
\pgfpathlineto{\pgfqpoint{5.938920in}{2.302535in}}%
\pgfpathlineto{\pgfqpoint{5.940160in}{2.304674in}}%
\pgfpathlineto{\pgfqpoint{5.942640in}{2.298672in}}%
\pgfpathlineto{\pgfqpoint{5.943880in}{2.298964in}}%
\pgfpathlineto{\pgfqpoint{5.947600in}{2.293938in}}%
\pgfpathlineto{\pgfqpoint{5.948840in}{2.293970in}}%
\pgfpathlineto{\pgfqpoint{5.950080in}{2.296653in}}%
\pgfpathlineto{\pgfqpoint{5.953800in}{2.296307in}}%
\pgfpathlineto{\pgfqpoint{5.956280in}{2.301482in}}%
\pgfpathlineto{\pgfqpoint{5.957520in}{2.302473in}}%
\pgfpathlineto{\pgfqpoint{5.960000in}{2.295248in}}%
\pgfpathlineto{\pgfqpoint{5.963720in}{2.302003in}}%
\pgfpathlineto{\pgfqpoint{5.966200in}{2.294891in}}%
\pgfpathlineto{\pgfqpoint{5.967440in}{2.292102in}}%
\pgfpathlineto{\pgfqpoint{5.969920in}{2.297989in}}%
\pgfpathlineto{\pgfqpoint{5.971160in}{2.300737in}}%
\pgfpathlineto{\pgfqpoint{5.973640in}{2.299014in}}%
\pgfpathlineto{\pgfqpoint{5.974880in}{2.303905in}}%
\pgfpathlineto{\pgfqpoint{5.977360in}{2.302131in}}%
\pgfpathlineto{\pgfqpoint{5.979840in}{2.298318in}}%
\pgfpathlineto{\pgfqpoint{5.982320in}{2.299593in}}%
\pgfpathlineto{\pgfqpoint{5.986040in}{2.293136in}}%
\pgfpathlineto{\pgfqpoint{5.987280in}{2.291209in}}%
\pgfpathlineto{\pgfqpoint{5.988520in}{2.292505in}}%
\pgfpathlineto{\pgfqpoint{5.991000in}{2.288345in}}%
\pgfpathlineto{\pgfqpoint{5.993480in}{2.293768in}}%
\pgfpathlineto{\pgfqpoint{5.995960in}{2.297900in}}%
\pgfpathlineto{\pgfqpoint{5.998440in}{2.303391in}}%
\pgfpathlineto{\pgfqpoint{5.999680in}{2.303061in}}%
\pgfpathlineto{\pgfqpoint{6.002160in}{2.299573in}}%
\pgfpathlineto{\pgfqpoint{6.004640in}{2.302384in}}%
\pgfpathlineto{\pgfqpoint{6.005880in}{2.304932in}}%
\pgfpathlineto{\pgfqpoint{6.008360in}{2.299881in}}%
\pgfpathlineto{\pgfqpoint{6.009600in}{2.300939in}}%
\pgfpathlineto{\pgfqpoint{6.010840in}{2.296761in}}%
\pgfpathlineto{\pgfqpoint{6.013320in}{2.303723in}}%
\pgfpathlineto{\pgfqpoint{6.015800in}{2.300937in}}%
\pgfpathlineto{\pgfqpoint{6.018280in}{2.301357in}}%
\pgfpathlineto{\pgfqpoint{6.019520in}{2.299876in}}%
\pgfpathlineto{\pgfqpoint{6.022000in}{2.292358in}}%
\pgfpathlineto{\pgfqpoint{6.024480in}{2.289991in}}%
\pgfpathlineto{\pgfqpoint{6.026960in}{2.290467in}}%
\pgfpathlineto{\pgfqpoint{6.030680in}{2.297237in}}%
\pgfpathlineto{\pgfqpoint{6.034400in}{2.294444in}}%
\pgfpathlineto{\pgfqpoint{6.035640in}{2.293608in}}%
\pgfpathlineto{\pgfqpoint{6.038120in}{2.281395in}}%
\pgfpathlineto{\pgfqpoint{6.039360in}{2.280384in}}%
\pgfpathlineto{\pgfqpoint{6.040600in}{2.282716in}}%
\pgfpathlineto{\pgfqpoint{6.041840in}{2.280557in}}%
\pgfpathlineto{\pgfqpoint{6.045560in}{2.285352in}}%
\pgfpathlineto{\pgfqpoint{6.046800in}{2.283068in}}%
\pgfpathlineto{\pgfqpoint{6.048040in}{2.283586in}}%
\pgfpathlineto{\pgfqpoint{6.050520in}{2.276933in}}%
\pgfpathlineto{\pgfqpoint{6.051760in}{2.280381in}}%
\pgfpathlineto{\pgfqpoint{6.054240in}{2.281469in}}%
\pgfpathlineto{\pgfqpoint{6.055480in}{2.283539in}}%
\pgfpathlineto{\pgfqpoint{6.059200in}{2.295868in}}%
\pgfpathlineto{\pgfqpoint{6.060440in}{2.295899in}}%
\pgfpathlineto{\pgfqpoint{6.062920in}{2.291365in}}%
\pgfpathlineto{\pgfqpoint{6.064160in}{2.294597in}}%
\pgfpathlineto{\pgfqpoint{6.066640in}{2.289559in}}%
\pgfpathlineto{\pgfqpoint{6.067880in}{2.289262in}}%
\pgfpathlineto{\pgfqpoint{6.069120in}{2.285413in}}%
\pgfpathlineto{\pgfqpoint{6.070360in}{2.286005in}}%
\pgfpathlineto{\pgfqpoint{6.072840in}{2.283259in}}%
\pgfpathlineto{\pgfqpoint{6.074080in}{2.285473in}}%
\pgfpathlineto{\pgfqpoint{6.076560in}{2.284679in}}%
\pgfpathlineto{\pgfqpoint{6.081520in}{2.294103in}}%
\pgfpathlineto{\pgfqpoint{6.085240in}{2.285196in}}%
\pgfpathlineto{\pgfqpoint{6.087720in}{2.290040in}}%
\pgfpathlineto{\pgfqpoint{6.091440in}{2.283167in}}%
\pgfpathlineto{\pgfqpoint{6.093920in}{2.287477in}}%
\pgfpathlineto{\pgfqpoint{6.095160in}{2.289327in}}%
\pgfpathlineto{\pgfqpoint{6.097640in}{2.285861in}}%
\pgfpathlineto{\pgfqpoint{6.098880in}{2.290661in}}%
\pgfpathlineto{\pgfqpoint{6.101360in}{2.287853in}}%
\pgfpathlineto{\pgfqpoint{6.103840in}{2.283812in}}%
\pgfpathlineto{\pgfqpoint{6.106320in}{2.283628in}}%
\pgfpathlineto{\pgfqpoint{6.111280in}{2.277299in}}%
\pgfpathlineto{\pgfqpoint{6.112520in}{2.279020in}}%
\pgfpathlineto{\pgfqpoint{6.115000in}{2.276853in}}%
\pgfpathlineto{\pgfqpoint{6.116240in}{2.277165in}}%
\pgfpathlineto{\pgfqpoint{6.118720in}{2.283868in}}%
\pgfpathlineto{\pgfqpoint{6.119960in}{2.284229in}}%
\pgfpathlineto{\pgfqpoint{6.122440in}{2.293208in}}%
\pgfpathlineto{\pgfqpoint{6.123680in}{2.292688in}}%
\pgfpathlineto{\pgfqpoint{6.126160in}{2.289340in}}%
\pgfpathlineto{\pgfqpoint{6.127400in}{2.289648in}}%
\pgfpathlineto{\pgfqpoint{6.129880in}{2.294977in}}%
\pgfpathlineto{\pgfqpoint{6.132360in}{2.289126in}}%
\pgfpathlineto{\pgfqpoint{6.133600in}{2.289209in}}%
\pgfpathlineto{\pgfqpoint{6.134840in}{2.284759in}}%
\pgfpathlineto{\pgfqpoint{6.137320in}{2.291835in}}%
\pgfpathlineto{\pgfqpoint{6.139800in}{2.289172in}}%
\pgfpathlineto{\pgfqpoint{6.142280in}{2.290912in}}%
\pgfpathlineto{\pgfqpoint{6.143520in}{2.289769in}}%
\pgfpathlineto{\pgfqpoint{6.147240in}{2.279773in}}%
\pgfpathlineto{\pgfqpoint{6.149720in}{2.278594in}}%
\pgfpathlineto{\pgfqpoint{6.150960in}{2.279278in}}%
\pgfpathlineto{\pgfqpoint{6.153440in}{2.284937in}}%
\pgfpathlineto{\pgfqpoint{6.154680in}{2.287565in}}%
\pgfpathlineto{\pgfqpoint{6.157160in}{2.283367in}}%
\pgfpathlineto{\pgfqpoint{6.158400in}{2.284697in}}%
\pgfpathlineto{\pgfqpoint{6.159640in}{2.284457in}}%
\pgfpathlineto{\pgfqpoint{6.163360in}{2.267871in}}%
\pgfpathlineto{\pgfqpoint{6.164600in}{2.270124in}}%
\pgfpathlineto{\pgfqpoint{6.165840in}{2.268519in}}%
\pgfpathlineto{\pgfqpoint{6.168320in}{2.271405in}}%
\pgfpathlineto{\pgfqpoint{6.169560in}{2.274227in}}%
\pgfpathlineto{\pgfqpoint{6.172040in}{2.274200in}}%
\pgfpathlineto{\pgfqpoint{6.174520in}{2.265375in}}%
\pgfpathlineto{\pgfqpoint{6.175760in}{2.269239in}}%
\pgfpathlineto{\pgfqpoint{6.178240in}{2.270530in}}%
\pgfpathlineto{\pgfqpoint{6.180720in}{2.276164in}}%
\pgfpathlineto{\pgfqpoint{6.184440in}{2.284281in}}%
\pgfpathlineto{\pgfqpoint{6.185680in}{2.280868in}}%
\pgfpathlineto{\pgfqpoint{6.186920in}{2.281323in}}%
\pgfpathlineto{\pgfqpoint{6.188160in}{2.283514in}}%
\pgfpathlineto{\pgfqpoint{6.191880in}{2.280172in}}%
\pgfpathlineto{\pgfqpoint{6.193120in}{2.275427in}}%
\pgfpathlineto{\pgfqpoint{6.194360in}{2.276552in}}%
\pgfpathlineto{\pgfqpoint{6.196840in}{2.273084in}}%
\pgfpathlineto{\pgfqpoint{6.198080in}{2.274678in}}%
\pgfpathlineto{\pgfqpoint{6.200560in}{2.273390in}}%
\pgfpathlineto{\pgfqpoint{6.201800in}{2.275922in}}%
\pgfpathlineto{\pgfqpoint{6.204280in}{2.282961in}}%
\pgfpathlineto{\pgfqpoint{6.205520in}{2.286058in}}%
\pgfpathlineto{\pgfqpoint{6.209240in}{2.272579in}}%
\pgfpathlineto{\pgfqpoint{6.211720in}{2.278548in}}%
\pgfpathlineto{\pgfqpoint{6.215440in}{2.272044in}}%
\pgfpathlineto{\pgfqpoint{6.217920in}{2.275789in}}%
\pgfpathlineto{\pgfqpoint{6.219160in}{2.277327in}}%
\pgfpathlineto{\pgfqpoint{6.221640in}{2.273678in}}%
\pgfpathlineto{\pgfqpoint{6.222880in}{2.278121in}}%
\pgfpathlineto{\pgfqpoint{6.224120in}{2.277663in}}%
\pgfpathlineto{\pgfqpoint{6.229080in}{2.272662in}}%
\pgfpathlineto{\pgfqpoint{6.230320in}{2.274395in}}%
\pgfpathlineto{\pgfqpoint{6.234040in}{2.267032in}}%
\pgfpathlineto{\pgfqpoint{6.236520in}{2.268948in}}%
\pgfpathlineto{\pgfqpoint{6.237760in}{2.266751in}}%
\pgfpathlineto{\pgfqpoint{6.240240in}{2.267341in}}%
\pgfpathlineto{\pgfqpoint{6.242720in}{2.273547in}}%
\pgfpathlineto{\pgfqpoint{6.243960in}{2.272749in}}%
\pgfpathlineto{\pgfqpoint{6.246440in}{2.279046in}}%
\pgfpathlineto{\pgfqpoint{6.250160in}{2.273540in}}%
\pgfpathlineto{\pgfqpoint{6.251400in}{2.272841in}}%
\pgfpathlineto{\pgfqpoint{6.252640in}{2.273464in}}%
\pgfpathlineto{\pgfqpoint{6.253880in}{2.276118in}}%
\pgfpathlineto{\pgfqpoint{6.256360in}{2.268433in}}%
\pgfpathlineto{\pgfqpoint{6.257600in}{2.270133in}}%
\pgfpathlineto{\pgfqpoint{6.258840in}{2.266746in}}%
\pgfpathlineto{\pgfqpoint{6.261320in}{2.274421in}}%
\pgfpathlineto{\pgfqpoint{6.262560in}{2.271231in}}%
\pgfpathlineto{\pgfqpoint{6.266280in}{2.271656in}}%
\pgfpathlineto{\pgfqpoint{6.273720in}{2.262825in}}%
\pgfpathlineto{\pgfqpoint{6.274960in}{2.263692in}}%
\pgfpathlineto{\pgfqpoint{6.277440in}{2.271223in}}%
\pgfpathlineto{\pgfqpoint{6.279920in}{2.274193in}}%
\pgfpathlineto{\pgfqpoint{6.281160in}{2.271331in}}%
\pgfpathlineto{\pgfqpoint{6.283640in}{2.274153in}}%
\pgfpathlineto{\pgfqpoint{6.287360in}{2.256066in}}%
\pgfpathlineto{\pgfqpoint{6.288600in}{2.258270in}}%
\pgfpathlineto{\pgfqpoint{6.289840in}{2.256607in}}%
\pgfpathlineto{\pgfqpoint{6.291080in}{2.257400in}}%
\pgfpathlineto{\pgfqpoint{6.296040in}{2.265957in}}%
\pgfpathlineto{\pgfqpoint{6.298520in}{2.255716in}}%
\pgfpathlineto{\pgfqpoint{6.301000in}{2.258488in}}%
\pgfpathlineto{\pgfqpoint{6.302240in}{2.258818in}}%
\pgfpathlineto{\pgfqpoint{6.308440in}{2.270275in}}%
\pgfpathlineto{\pgfqpoint{6.309680in}{2.266557in}}%
\pgfpathlineto{\pgfqpoint{6.312160in}{2.267841in}}%
\pgfpathlineto{\pgfqpoint{6.313400in}{2.266207in}}%
\pgfpathlineto{\pgfqpoint{6.315880in}{2.266136in}}%
\pgfpathlineto{\pgfqpoint{6.317120in}{2.262497in}}%
\pgfpathlineto{\pgfqpoint{6.318360in}{2.265146in}}%
\pgfpathlineto{\pgfqpoint{6.320840in}{2.261159in}}%
\pgfpathlineto{\pgfqpoint{6.322080in}{2.262502in}}%
\pgfpathlineto{\pgfqpoint{6.324560in}{2.259462in}}%
\pgfpathlineto{\pgfqpoint{6.325800in}{2.262422in}}%
\pgfpathlineto{\pgfqpoint{6.328280in}{2.271813in}}%
\pgfpathlineto{\pgfqpoint{6.329520in}{2.273996in}}%
\pgfpathlineto{\pgfqpoint{6.333240in}{2.254136in}}%
\pgfpathlineto{\pgfqpoint{6.335720in}{2.261084in}}%
\pgfpathlineto{\pgfqpoint{6.338200in}{2.257136in}}%
\pgfpathlineto{\pgfqpoint{6.339440in}{2.255441in}}%
\pgfpathlineto{\pgfqpoint{6.343160in}{2.264343in}}%
\pgfpathlineto{\pgfqpoint{6.345640in}{2.259173in}}%
\pgfpathlineto{\pgfqpoint{6.346880in}{2.263949in}}%
\pgfpathlineto{\pgfqpoint{6.349360in}{2.262244in}}%
\pgfpathlineto{\pgfqpoint{6.351840in}{2.255592in}}%
\pgfpathlineto{\pgfqpoint{6.354320in}{2.259800in}}%
\pgfpathlineto{\pgfqpoint{6.356800in}{2.254571in}}%
\pgfpathlineto{\pgfqpoint{6.358040in}{2.254059in}}%
\pgfpathlineto{\pgfqpoint{6.359280in}{2.255346in}}%
\pgfpathlineto{\pgfqpoint{6.360520in}{2.254401in}}%
\pgfpathlineto{\pgfqpoint{6.361760in}{2.251418in}}%
\pgfpathlineto{\pgfqpoint{6.365480in}{2.253168in}}%
\pgfpathlineto{\pgfqpoint{6.366720in}{2.257448in}}%
\pgfpathlineto{\pgfqpoint{6.367960in}{2.257583in}}%
\pgfpathlineto{\pgfqpoint{6.370440in}{2.263454in}}%
\pgfpathlineto{\pgfqpoint{6.375400in}{2.256665in}}%
\pgfpathlineto{\pgfqpoint{6.377880in}{2.262146in}}%
\pgfpathlineto{\pgfqpoint{6.380360in}{2.256365in}}%
\pgfpathlineto{\pgfqpoint{6.381600in}{2.258626in}}%
\pgfpathlineto{\pgfqpoint{6.382840in}{2.255989in}}%
\pgfpathlineto{\pgfqpoint{6.385320in}{2.265975in}}%
\pgfpathlineto{\pgfqpoint{6.386560in}{2.263625in}}%
\pgfpathlineto{\pgfqpoint{6.387800in}{2.264330in}}%
\pgfpathlineto{\pgfqpoint{6.390280in}{2.261087in}}%
\pgfpathlineto{\pgfqpoint{6.395240in}{2.251012in}}%
\pgfpathlineto{\pgfqpoint{6.396480in}{2.251457in}}%
\pgfpathlineto{\pgfqpoint{6.398960in}{2.253180in}}%
\pgfpathlineto{\pgfqpoint{6.401440in}{2.263162in}}%
\pgfpathlineto{\pgfqpoint{6.403920in}{2.265425in}}%
\pgfpathlineto{\pgfqpoint{6.405160in}{2.262226in}}%
\pgfpathlineto{\pgfqpoint{6.407640in}{2.262799in}}%
\pgfpathlineto{\pgfqpoint{6.411360in}{2.243329in}}%
\pgfpathlineto{\pgfqpoint{6.412600in}{2.245081in}}%
\pgfpathlineto{\pgfqpoint{6.415080in}{2.244522in}}%
\pgfpathlineto{\pgfqpoint{6.416320in}{2.246439in}}%
\pgfpathlineto{\pgfqpoint{6.418800in}{2.253141in}}%
\pgfpathlineto{\pgfqpoint{6.420040in}{2.255837in}}%
\pgfpathlineto{\pgfqpoint{6.422520in}{2.245100in}}%
\pgfpathlineto{\pgfqpoint{6.425000in}{2.248738in}}%
\pgfpathlineto{\pgfqpoint{6.426240in}{2.248351in}}%
\pgfpathlineto{\pgfqpoint{6.431200in}{2.259048in}}%
\pgfpathlineto{\pgfqpoint{6.432440in}{2.259772in}}%
\pgfpathlineto{\pgfqpoint{6.433680in}{2.253112in}}%
\pgfpathlineto{\pgfqpoint{6.434920in}{2.253979in}}%
\pgfpathlineto{\pgfqpoint{6.436160in}{2.256265in}}%
\pgfpathlineto{\pgfqpoint{6.438640in}{2.253055in}}%
\pgfpathlineto{\pgfqpoint{6.439880in}{2.252598in}}%
\pgfpathlineto{\pgfqpoint{6.441120in}{2.246965in}}%
\pgfpathlineto{\pgfqpoint{6.442360in}{2.249581in}}%
\pgfpathlineto{\pgfqpoint{6.444840in}{2.246354in}}%
\pgfpathlineto{\pgfqpoint{6.447320in}{2.249126in}}%
\pgfpathlineto{\pgfqpoint{6.448560in}{2.247254in}}%
\pgfpathlineto{\pgfqpoint{6.449800in}{2.248899in}}%
\pgfpathlineto{\pgfqpoint{6.452280in}{2.257092in}}%
\pgfpathlineto{\pgfqpoint{6.453520in}{2.258827in}}%
\pgfpathlineto{\pgfqpoint{6.457240in}{2.245320in}}%
\pgfpathlineto{\pgfqpoint{6.459720in}{2.258614in}}%
\pgfpathlineto{\pgfqpoint{6.462200in}{2.253675in}}%
\pgfpathlineto{\pgfqpoint{6.463440in}{2.251384in}}%
\pgfpathlineto{\pgfqpoint{6.464680in}{2.251941in}}%
\pgfpathlineto{\pgfqpoint{6.467160in}{2.257029in}}%
\pgfpathlineto{\pgfqpoint{6.469640in}{2.250827in}}%
\pgfpathlineto{\pgfqpoint{6.472120in}{2.256475in}}%
\pgfpathlineto{\pgfqpoint{6.473360in}{2.255879in}}%
\pgfpathlineto{\pgfqpoint{6.475840in}{2.246538in}}%
\pgfpathlineto{\pgfqpoint{6.478320in}{2.250585in}}%
\pgfpathlineto{\pgfqpoint{6.480800in}{2.244868in}}%
\pgfpathlineto{\pgfqpoint{6.482040in}{2.243455in}}%
\pgfpathlineto{\pgfqpoint{6.484520in}{2.246959in}}%
\pgfpathlineto{\pgfqpoint{6.487000in}{2.247954in}}%
\pgfpathlineto{\pgfqpoint{6.489480in}{2.244238in}}%
\pgfpathlineto{\pgfqpoint{6.491960in}{2.250782in}}%
\pgfpathlineto{\pgfqpoint{6.494440in}{2.258045in}}%
\pgfpathlineto{\pgfqpoint{6.496920in}{2.250868in}}%
\pgfpathlineto{\pgfqpoint{6.500640in}{2.247058in}}%
\pgfpathlineto{\pgfqpoint{6.501880in}{2.248967in}}%
\pgfpathlineto{\pgfqpoint{6.504360in}{2.242949in}}%
\pgfpathlineto{\pgfqpoint{6.509320in}{2.252635in}}%
\pgfpathlineto{\pgfqpoint{6.510560in}{2.250830in}}%
\pgfpathlineto{\pgfqpoint{6.511800in}{2.251499in}}%
\pgfpathlineto{\pgfqpoint{6.515520in}{2.246649in}}%
\pgfpathlineto{\pgfqpoint{6.519240in}{2.234118in}}%
\pgfpathlineto{\pgfqpoint{6.520480in}{2.234929in}}%
\pgfpathlineto{\pgfqpoint{6.522960in}{2.232523in}}%
\pgfpathlineto{\pgfqpoint{6.525440in}{2.242453in}}%
\pgfpathlineto{\pgfqpoint{6.527920in}{2.244814in}}%
\pgfpathlineto{\pgfqpoint{6.530400in}{2.239198in}}%
\pgfpathlineto{\pgfqpoint{6.531640in}{2.240053in}}%
\pgfpathlineto{\pgfqpoint{6.534120in}{2.225404in}}%
\pgfpathlineto{\pgfqpoint{6.535360in}{2.225611in}}%
\pgfpathlineto{\pgfqpoint{6.536600in}{2.229302in}}%
\pgfpathlineto{\pgfqpoint{6.540320in}{2.229827in}}%
\pgfpathlineto{\pgfqpoint{6.542800in}{2.233533in}}%
\pgfpathlineto{\pgfqpoint{6.544040in}{2.236429in}}%
\pgfpathlineto{\pgfqpoint{6.546520in}{2.227726in}}%
\pgfpathlineto{\pgfqpoint{6.550240in}{2.232514in}}%
\pgfpathlineto{\pgfqpoint{6.553960in}{2.239866in}}%
\pgfpathlineto{\pgfqpoint{6.556440in}{2.243513in}}%
\pgfpathlineto{\pgfqpoint{6.557680in}{2.236982in}}%
\pgfpathlineto{\pgfqpoint{6.560160in}{2.242963in}}%
\pgfpathlineto{\pgfqpoint{6.561400in}{2.240090in}}%
\pgfpathlineto{\pgfqpoint{6.562640in}{2.241749in}}%
\pgfpathlineto{\pgfqpoint{6.563880in}{2.240403in}}%
\pgfpathlineto{\pgfqpoint{6.565120in}{2.234944in}}%
\pgfpathlineto{\pgfqpoint{6.566360in}{2.237158in}}%
\pgfpathlineto{\pgfqpoint{6.567600in}{2.234797in}}%
\pgfpathlineto{\pgfqpoint{6.568840in}{2.235278in}}%
\pgfpathlineto{\pgfqpoint{6.570080in}{2.238968in}}%
\pgfpathlineto{\pgfqpoint{6.572560in}{2.238012in}}%
\pgfpathlineto{\pgfqpoint{6.573800in}{2.238538in}}%
\pgfpathlineto{\pgfqpoint{6.575040in}{2.241993in}}%
\pgfpathlineto{\pgfqpoint{6.576280in}{2.242126in}}%
\pgfpathlineto{\pgfqpoint{6.577520in}{2.243572in}}%
\pgfpathlineto{\pgfqpoint{6.581240in}{2.228993in}}%
\pgfpathlineto{\pgfqpoint{6.583720in}{2.243289in}}%
\pgfpathlineto{\pgfqpoint{6.586200in}{2.234829in}}%
\pgfpathlineto{\pgfqpoint{6.588680in}{2.230633in}}%
\pgfpathlineto{\pgfqpoint{6.591160in}{2.235457in}}%
\pgfpathlineto{\pgfqpoint{6.593640in}{2.228115in}}%
\pgfpathlineto{\pgfqpoint{6.597360in}{2.231117in}}%
\pgfpathlineto{\pgfqpoint{6.599840in}{2.219652in}}%
\pgfpathlineto{\pgfqpoint{6.602320in}{2.224448in}}%
\pgfpathlineto{\pgfqpoint{6.603560in}{2.222633in}}%
\pgfpathlineto{\pgfqpoint{6.611000in}{2.231128in}}%
\pgfpathlineto{\pgfqpoint{6.613480in}{2.226402in}}%
\pgfpathlineto{\pgfqpoint{6.617200in}{2.239079in}}%
\pgfpathlineto{\pgfqpoint{6.618440in}{2.241980in}}%
\pgfpathlineto{\pgfqpoint{6.623400in}{2.235098in}}%
\pgfpathlineto{\pgfqpoint{6.625880in}{2.236278in}}%
\pgfpathlineto{\pgfqpoint{6.628360in}{2.232066in}}%
\pgfpathlineto{\pgfqpoint{6.629600in}{2.234347in}}%
\pgfpathlineto{\pgfqpoint{6.630840in}{2.231694in}}%
\pgfpathlineto{\pgfqpoint{6.633320in}{2.232668in}}%
\pgfpathlineto{\pgfqpoint{6.634560in}{2.228489in}}%
\pgfpathlineto{\pgfqpoint{6.635800in}{2.229878in}}%
\pgfpathlineto{\pgfqpoint{6.639520in}{2.228341in}}%
\pgfpathlineto{\pgfqpoint{6.643240in}{2.215710in}}%
\pgfpathlineto{\pgfqpoint{6.644480in}{2.216257in}}%
\pgfpathlineto{\pgfqpoint{6.645720in}{2.216279in}}%
\pgfpathlineto{\pgfqpoint{6.646960in}{2.217623in}}%
\pgfpathlineto{\pgfqpoint{6.649440in}{2.229800in}}%
\pgfpathlineto{\pgfqpoint{6.650680in}{2.229069in}}%
\pgfpathlineto{\pgfqpoint{6.651920in}{2.229886in}}%
\pgfpathlineto{\pgfqpoint{6.653160in}{2.228336in}}%
\pgfpathlineto{\pgfqpoint{6.654400in}{2.224717in}}%
\pgfpathlineto{\pgfqpoint{6.655640in}{2.226284in}}%
\pgfpathlineto{\pgfqpoint{6.658120in}{2.214439in}}%
\pgfpathlineto{\pgfqpoint{6.660600in}{2.218461in}}%
\pgfpathlineto{\pgfqpoint{6.661840in}{2.213373in}}%
\pgfpathlineto{\pgfqpoint{6.664320in}{2.214555in}}%
\pgfpathlineto{\pgfqpoint{6.666800in}{2.220320in}}%
\pgfpathlineto{\pgfqpoint{6.668040in}{2.224403in}}%
\pgfpathlineto{\pgfqpoint{6.670520in}{2.214308in}}%
\pgfpathlineto{\pgfqpoint{6.673000in}{2.212859in}}%
\pgfpathlineto{\pgfqpoint{6.680440in}{2.223164in}}%
\pgfpathlineto{\pgfqpoint{6.681680in}{2.217595in}}%
\pgfpathlineto{\pgfqpoint{6.684160in}{2.224992in}}%
\pgfpathlineto{\pgfqpoint{6.686640in}{2.224604in}}%
\pgfpathlineto{\pgfqpoint{6.687880in}{2.224014in}}%
\pgfpathlineto{\pgfqpoint{6.691600in}{2.214272in}}%
\pgfpathlineto{\pgfqpoint{6.692840in}{2.215831in}}%
\pgfpathlineto{\pgfqpoint{6.694080in}{2.220054in}}%
\pgfpathlineto{\pgfqpoint{6.695320in}{2.219963in}}%
\pgfpathlineto{\pgfqpoint{6.696560in}{2.217908in}}%
\pgfpathlineto{\pgfqpoint{6.697800in}{2.218517in}}%
\pgfpathlineto{\pgfqpoint{6.699040in}{2.223075in}}%
\pgfpathlineto{\pgfqpoint{6.700280in}{2.223254in}}%
\pgfpathlineto{\pgfqpoint{6.701520in}{2.224583in}}%
\pgfpathlineto{\pgfqpoint{6.702760in}{2.223296in}}%
\pgfpathlineto{\pgfqpoint{6.704000in}{2.217952in}}%
\pgfpathlineto{\pgfqpoint{6.705240in}{2.205239in}}%
\pgfpathlineto{\pgfqpoint{6.707720in}{2.218379in}}%
\pgfpathlineto{\pgfqpoint{6.711440in}{2.207391in}}%
\pgfpathlineto{\pgfqpoint{6.712680in}{2.206672in}}%
\pgfpathlineto{\pgfqpoint{6.715160in}{2.216014in}}%
\pgfpathlineto{\pgfqpoint{6.720120in}{2.203943in}}%
\pgfpathlineto{\pgfqpoint{6.721360in}{2.205609in}}%
\pgfpathlineto{\pgfqpoint{6.723840in}{2.189550in}}%
\pgfpathlineto{\pgfqpoint{6.727560in}{2.196977in}}%
\pgfpathlineto{\pgfqpoint{6.728800in}{2.199989in}}%
\pgfpathlineto{\pgfqpoint{6.731280in}{2.198336in}}%
\pgfpathlineto{\pgfqpoint{6.732520in}{2.198003in}}%
\pgfpathlineto{\pgfqpoint{6.735000in}{2.200343in}}%
\pgfpathlineto{\pgfqpoint{6.737480in}{2.198109in}}%
\pgfpathlineto{\pgfqpoint{6.741200in}{2.214785in}}%
\pgfpathlineto{\pgfqpoint{6.742440in}{2.215861in}}%
\pgfpathlineto{\pgfqpoint{6.743680in}{2.214271in}}%
\pgfpathlineto{\pgfqpoint{6.746160in}{2.216353in}}%
\pgfpathlineto{\pgfqpoint{6.748640in}{2.210112in}}%
\pgfpathlineto{\pgfqpoint{6.749880in}{2.212507in}}%
\pgfpathlineto{\pgfqpoint{6.751120in}{2.210160in}}%
\pgfpathlineto{\pgfqpoint{6.753600in}{2.209984in}}%
\pgfpathlineto{\pgfqpoint{6.761040in}{2.198430in}}%
\pgfpathlineto{\pgfqpoint{6.763520in}{2.201785in}}%
\pgfpathlineto{\pgfqpoint{6.767240in}{2.190468in}}%
\pgfpathlineto{\pgfqpoint{6.768480in}{2.191566in}}%
\pgfpathlineto{\pgfqpoint{6.769720in}{2.189521in}}%
\pgfpathlineto{\pgfqpoint{6.770960in}{2.190418in}}%
\pgfpathlineto{\pgfqpoint{6.773440in}{2.204996in}}%
\pgfpathlineto{\pgfqpoint{6.774680in}{2.205221in}}%
\pgfpathlineto{\pgfqpoint{6.775920in}{2.209676in}}%
\pgfpathlineto{\pgfqpoint{6.779640in}{2.204430in}}%
\pgfpathlineto{\pgfqpoint{6.782120in}{2.189556in}}%
\pgfpathlineto{\pgfqpoint{6.783360in}{2.190096in}}%
\pgfpathlineto{\pgfqpoint{6.784600in}{2.193436in}}%
\pgfpathlineto{\pgfqpoint{6.785840in}{2.187396in}}%
\pgfpathlineto{\pgfqpoint{6.787080in}{2.188380in}}%
\pgfpathlineto{\pgfqpoint{6.789560in}{2.197054in}}%
\pgfpathlineto{\pgfqpoint{6.792040in}{2.205677in}}%
\pgfpathlineto{\pgfqpoint{6.797000in}{2.181951in}}%
\pgfpathlineto{\pgfqpoint{6.799480in}{2.182631in}}%
\pgfpathlineto{\pgfqpoint{6.800720in}{2.180067in}}%
\pgfpathlineto{\pgfqpoint{6.804440in}{2.190724in}}%
\pgfpathlineto{\pgfqpoint{6.805680in}{2.184727in}}%
\pgfpathlineto{\pgfqpoint{6.809400in}{2.198029in}}%
\pgfpathlineto{\pgfqpoint{6.810640in}{2.195333in}}%
\pgfpathlineto{\pgfqpoint{6.811880in}{2.195691in}}%
\pgfpathlineto{\pgfqpoint{6.813120in}{2.191913in}}%
\pgfpathlineto{\pgfqpoint{6.814360in}{2.192516in}}%
\pgfpathlineto{\pgfqpoint{6.816840in}{2.188038in}}%
\pgfpathlineto{\pgfqpoint{6.818080in}{2.193194in}}%
\pgfpathlineto{\pgfqpoint{6.819320in}{2.192912in}}%
\pgfpathlineto{\pgfqpoint{6.821800in}{2.188239in}}%
\pgfpathlineto{\pgfqpoint{6.826760in}{2.198173in}}%
\pgfpathlineto{\pgfqpoint{6.829240in}{2.188717in}}%
\pgfpathlineto{\pgfqpoint{6.831720in}{2.199842in}}%
\pgfpathlineto{\pgfqpoint{6.834200in}{2.195199in}}%
\pgfpathlineto{\pgfqpoint{6.835440in}{2.194674in}}%
\pgfpathlineto{\pgfqpoint{6.837920in}{2.197547in}}%
\pgfpathlineto{\pgfqpoint{6.839160in}{2.202286in}}%
\pgfpathlineto{\pgfqpoint{6.841640in}{2.196591in}}%
\pgfpathlineto{\pgfqpoint{6.845360in}{2.193040in}}%
\pgfpathlineto{\pgfqpoint{6.847840in}{2.177525in}}%
\pgfpathlineto{\pgfqpoint{6.852800in}{2.199244in}}%
\pgfpathlineto{\pgfqpoint{6.855280in}{2.190642in}}%
\pgfpathlineto{\pgfqpoint{6.857760in}{2.191377in}}%
\pgfpathlineto{\pgfqpoint{6.859000in}{2.190571in}}%
\pgfpathlineto{\pgfqpoint{6.860240in}{2.187921in}}%
\pgfpathlineto{\pgfqpoint{6.861480in}{2.188103in}}%
\pgfpathlineto{\pgfqpoint{6.862720in}{2.192057in}}%
\pgfpathlineto{\pgfqpoint{6.863960in}{2.190702in}}%
\pgfpathlineto{\pgfqpoint{6.866440in}{2.198517in}}%
\pgfpathlineto{\pgfqpoint{6.868920in}{2.194296in}}%
\pgfpathlineto{\pgfqpoint{6.870160in}{2.194103in}}%
\pgfpathlineto{\pgfqpoint{6.871400in}{2.189115in}}%
\pgfpathlineto{\pgfqpoint{6.872640in}{2.190150in}}%
\pgfpathlineto{\pgfqpoint{6.873880in}{2.194665in}}%
\pgfpathlineto{\pgfqpoint{6.875120in}{2.191058in}}%
\pgfpathlineto{\pgfqpoint{6.876360in}{2.191151in}}%
\pgfpathlineto{\pgfqpoint{6.877600in}{2.193410in}}%
\pgfpathlineto{\pgfqpoint{6.882560in}{2.182543in}}%
\pgfpathlineto{\pgfqpoint{6.883800in}{2.187053in}}%
\pgfpathlineto{\pgfqpoint{6.885040in}{2.184524in}}%
\pgfpathlineto{\pgfqpoint{6.887520in}{2.188382in}}%
\pgfpathlineto{\pgfqpoint{6.890000in}{2.182477in}}%
\pgfpathlineto{\pgfqpoint{6.891240in}{2.177770in}}%
\pgfpathlineto{\pgfqpoint{6.893720in}{2.182630in}}%
\pgfpathlineto{\pgfqpoint{6.894960in}{2.181044in}}%
\pgfpathlineto{\pgfqpoint{6.897440in}{2.199463in}}%
\pgfpathlineto{\pgfqpoint{6.899920in}{2.203350in}}%
\pgfpathlineto{\pgfqpoint{6.904880in}{2.180386in}}%
\pgfpathlineto{\pgfqpoint{6.906120in}{2.169028in}}%
\pgfpathlineto{\pgfqpoint{6.907360in}{2.169582in}}%
\pgfpathlineto{\pgfqpoint{6.908600in}{2.173662in}}%
\pgfpathlineto{\pgfqpoint{6.909840in}{2.168427in}}%
\pgfpathlineto{\pgfqpoint{6.912320in}{2.176016in}}%
\pgfpathlineto{\pgfqpoint{6.916040in}{2.202886in}}%
\pgfpathlineto{\pgfqpoint{6.922240in}{2.174564in}}%
\pgfpathlineto{\pgfqpoint{6.923480in}{2.175224in}}%
\pgfpathlineto{\pgfqpoint{6.924720in}{2.174386in}}%
\pgfpathlineto{\pgfqpoint{6.925960in}{2.177173in}}%
\pgfpathlineto{\pgfqpoint{6.928440in}{2.192894in}}%
\pgfpathlineto{\pgfqpoint{6.929680in}{2.184636in}}%
\pgfpathlineto{\pgfqpoint{6.932160in}{2.201839in}}%
\pgfpathlineto{\pgfqpoint{6.937120in}{2.186907in}}%
\pgfpathlineto{\pgfqpoint{6.939600in}{2.189474in}}%
\pgfpathlineto{\pgfqpoint{6.940840in}{2.185779in}}%
\pgfpathlineto{\pgfqpoint{6.942080in}{2.188122in}}%
\pgfpathlineto{\pgfqpoint{6.944560in}{2.183245in}}%
\pgfpathlineto{\pgfqpoint{6.945800in}{2.184507in}}%
\pgfpathlineto{\pgfqpoint{6.950760in}{2.199336in}}%
\pgfpathlineto{\pgfqpoint{6.952000in}{2.198519in}}%
\pgfpathlineto{\pgfqpoint{6.954480in}{2.210922in}}%
\pgfpathlineto{\pgfqpoint{6.955720in}{2.215501in}}%
\pgfpathlineto{\pgfqpoint{6.956960in}{2.207190in}}%
\pgfpathlineto{\pgfqpoint{6.959440in}{2.212222in}}%
\pgfpathlineto{\pgfqpoint{6.961920in}{2.205106in}}%
\pgfpathlineto{\pgfqpoint{6.963160in}{2.207523in}}%
\pgfpathlineto{\pgfqpoint{6.965640in}{2.199983in}}%
\pgfpathlineto{\pgfqpoint{6.966880in}{2.199663in}}%
\pgfpathlineto{\pgfqpoint{6.969360in}{2.208847in}}%
\pgfpathlineto{\pgfqpoint{6.971840in}{2.196447in}}%
\pgfpathlineto{\pgfqpoint{6.976800in}{2.220238in}}%
\pgfpathlineto{\pgfqpoint{6.980520in}{2.204690in}}%
\pgfpathlineto{\pgfqpoint{6.981760in}{2.202435in}}%
\pgfpathlineto{\pgfqpoint{6.983000in}{2.204738in}}%
\pgfpathlineto{\pgfqpoint{6.985480in}{2.194495in}}%
\pgfpathlineto{\pgfqpoint{6.987960in}{2.206788in}}%
\pgfpathlineto{\pgfqpoint{6.990440in}{2.214391in}}%
\pgfpathlineto{\pgfqpoint{6.991680in}{2.210486in}}%
\pgfpathlineto{\pgfqpoint{6.992920in}{2.210423in}}%
\pgfpathlineto{\pgfqpoint{6.994160in}{2.213427in}}%
\pgfpathlineto{\pgfqpoint{6.995400in}{2.207530in}}%
\pgfpathlineto{\pgfqpoint{6.997880in}{2.221319in}}%
\pgfpathlineto{\pgfqpoint{6.999120in}{2.214416in}}%
\pgfpathlineto{\pgfqpoint{7.000360in}{2.214802in}}%
\pgfpathlineto{\pgfqpoint{7.005320in}{2.197459in}}%
\pgfpathlineto{\pgfqpoint{7.006560in}{2.189842in}}%
\pgfpathlineto{\pgfqpoint{7.007800in}{2.199607in}}%
\pgfpathlineto{\pgfqpoint{7.009040in}{2.197129in}}%
\pgfpathlineto{\pgfqpoint{7.010280in}{2.201069in}}%
\pgfpathlineto{\pgfqpoint{7.012760in}{2.192949in}}%
\pgfpathlineto{\pgfqpoint{7.015240in}{2.184450in}}%
\pgfpathlineto{\pgfqpoint{7.017720in}{2.193567in}}%
\pgfpathlineto{\pgfqpoint{7.018960in}{2.194362in}}%
\pgfpathlineto{\pgfqpoint{7.021440in}{2.218546in}}%
\pgfpathlineto{\pgfqpoint{7.022680in}{2.220360in}}%
\pgfpathlineto{\pgfqpoint{7.025160in}{2.210435in}}%
\pgfpathlineto{\pgfqpoint{7.027640in}{2.196101in}}%
\pgfpathlineto{\pgfqpoint{7.028880in}{2.185632in}}%
\pgfpathlineto{\pgfqpoint{7.030120in}{2.187046in}}%
\pgfpathlineto{\pgfqpoint{7.032600in}{2.197525in}}%
\pgfpathlineto{\pgfqpoint{7.033840in}{2.190261in}}%
\pgfpathlineto{\pgfqpoint{7.036320in}{2.195598in}}%
\pgfpathlineto{\pgfqpoint{7.040040in}{2.230910in}}%
\pgfpathlineto{\pgfqpoint{7.046240in}{2.173514in}}%
\pgfpathlineto{\pgfqpoint{7.047480in}{2.172969in}}%
\pgfpathlineto{\pgfqpoint{7.048720in}{2.174488in}}%
\pgfpathlineto{\pgfqpoint{7.052440in}{2.191022in}}%
\pgfpathlineto{\pgfqpoint{7.053680in}{2.184060in}}%
\pgfpathlineto{\pgfqpoint{7.056160in}{2.205795in}}%
\pgfpathlineto{\pgfqpoint{7.061120in}{2.187537in}}%
\pgfpathlineto{\pgfqpoint{7.062360in}{2.189876in}}%
\pgfpathlineto{\pgfqpoint{7.063600in}{2.189097in}}%
\pgfpathlineto{\pgfqpoint{7.064840in}{2.184756in}}%
\pgfpathlineto{\pgfqpoint{7.066080in}{2.190720in}}%
\pgfpathlineto{\pgfqpoint{7.068560in}{2.186895in}}%
\pgfpathlineto{\pgfqpoint{7.069800in}{2.188638in}}%
\pgfpathlineto{\pgfqpoint{7.071040in}{2.194172in}}%
\pgfpathlineto{\pgfqpoint{7.072280in}{2.192864in}}%
\pgfpathlineto{\pgfqpoint{7.074760in}{2.200748in}}%
\pgfpathlineto{\pgfqpoint{7.076000in}{2.195058in}}%
\pgfpathlineto{\pgfqpoint{7.079720in}{2.208375in}}%
\pgfpathlineto{\pgfqpoint{7.080960in}{2.201060in}}%
\pgfpathlineto{\pgfqpoint{7.082200in}{2.206624in}}%
\pgfpathlineto{\pgfqpoint{7.084680in}{2.203289in}}%
\pgfpathlineto{\pgfqpoint{7.085920in}{2.191100in}}%
\pgfpathlineto{\pgfqpoint{7.087160in}{2.195246in}}%
\pgfpathlineto{\pgfqpoint{7.088400in}{2.189949in}}%
\pgfpathlineto{\pgfqpoint{7.089640in}{2.190355in}}%
\pgfpathlineto{\pgfqpoint{7.093360in}{2.212066in}}%
\pgfpathlineto{\pgfqpoint{7.095840in}{2.200719in}}%
\pgfpathlineto{\pgfqpoint{7.099560in}{2.238381in}}%
\pgfpathlineto{\pgfqpoint{7.102040in}{2.258983in}}%
\pgfpathlineto{\pgfqpoint{7.105760in}{2.235137in}}%
\pgfpathlineto{\pgfqpoint{7.107000in}{2.240587in}}%
\pgfpathlineto{\pgfqpoint{7.109480in}{2.236620in}}%
\pgfpathlineto{\pgfqpoint{7.114440in}{2.275091in}}%
\pgfpathlineto{\pgfqpoint{7.116920in}{2.259153in}}%
\pgfpathlineto{\pgfqpoint{7.118160in}{2.268519in}}%
\pgfpathlineto{\pgfqpoint{7.119400in}{2.257728in}}%
\pgfpathlineto{\pgfqpoint{7.121880in}{2.278002in}}%
\pgfpathlineto{\pgfqpoint{7.126840in}{2.258769in}}%
\pgfpathlineto{\pgfqpoint{7.128080in}{2.251458in}}%
\pgfpathlineto{\pgfqpoint{7.129320in}{2.255039in}}%
\pgfpathlineto{\pgfqpoint{7.130560in}{2.240432in}}%
\pgfpathlineto{\pgfqpoint{7.131800in}{2.250745in}}%
\pgfpathlineto{\pgfqpoint{7.134280in}{2.243122in}}%
\pgfpathlineto{\pgfqpoint{7.136760in}{2.226641in}}%
\pgfpathlineto{\pgfqpoint{7.138000in}{2.225906in}}%
\pgfpathlineto{\pgfqpoint{7.141720in}{2.256427in}}%
\pgfpathlineto{\pgfqpoint{7.142960in}{2.250641in}}%
\pgfpathlineto{\pgfqpoint{7.144200in}{2.267524in}}%
\pgfpathlineto{\pgfqpoint{7.145440in}{2.268456in}}%
\pgfpathlineto{\pgfqpoint{7.146680in}{2.265809in}}%
\pgfpathlineto{\pgfqpoint{7.147920in}{2.267682in}}%
\pgfpathlineto{\pgfqpoint{7.149160in}{2.259976in}}%
\pgfpathlineto{\pgfqpoint{7.151640in}{2.227516in}}%
\pgfpathlineto{\pgfqpoint{7.152880in}{2.228663in}}%
\pgfpathlineto{\pgfqpoint{7.155360in}{2.256569in}}%
\pgfpathlineto{\pgfqpoint{7.156600in}{2.259276in}}%
\pgfpathlineto{\pgfqpoint{7.157840in}{2.252710in}}%
\pgfpathlineto{\pgfqpoint{7.164040in}{2.297110in}}%
\pgfpathlineto{\pgfqpoint{7.165280in}{2.284329in}}%
\pgfpathlineto{\pgfqpoint{7.169000in}{2.223400in}}%
\pgfpathlineto{\pgfqpoint{7.171480in}{2.205123in}}%
\pgfpathlineto{\pgfqpoint{7.175200in}{2.223434in}}%
\pgfpathlineto{\pgfqpoint{7.176440in}{2.234509in}}%
\pgfpathlineto{\pgfqpoint{7.177680in}{2.234860in}}%
\pgfpathlineto{\pgfqpoint{7.178920in}{2.244783in}}%
\pgfpathlineto{\pgfqpoint{7.180160in}{2.245194in}}%
\pgfpathlineto{\pgfqpoint{7.185120in}{2.206750in}}%
\pgfpathlineto{\pgfqpoint{7.186360in}{2.209371in}}%
\pgfpathlineto{\pgfqpoint{7.191320in}{2.246928in}}%
\pgfpathlineto{\pgfqpoint{7.193800in}{2.255428in}}%
\pgfpathlineto{\pgfqpoint{7.195040in}{2.248758in}}%
\pgfpathlineto{\pgfqpoint{7.196280in}{2.252611in}}%
\pgfpathlineto{\pgfqpoint{7.198760in}{2.233885in}}%
\pgfpathlineto{\pgfqpoint{7.200000in}{2.223781in}}%
\pgfpathlineto{\pgfqpoint{7.200000in}{2.223781in}}%
\pgfusepath{stroke}%
\end{pgfscope}%
\begin{pgfscope}%
\pgfpathrectangle{\pgfqpoint{1.000000in}{0.350000in}}{\pgfqpoint{6.200000in}{2.800000in}} %
\pgfusepath{clip}%
\pgfsetbuttcap%
\pgfsetroundjoin%
\pgfsetlinewidth{0.501875pt}%
\definecolor{currentstroke}{rgb}{0.000000,0.000000,0.000000}%
\pgfsetstrokecolor{currentstroke}%
\pgfsetdash{{1.000000pt}{3.000000pt}}{0.000000pt}%
\pgfpathmoveto{\pgfqpoint{1.000000in}{0.350000in}}%
\pgfpathlineto{\pgfqpoint{1.000000in}{3.150000in}}%
\pgfusepath{stroke}%
\end{pgfscope}%
\begin{pgfscope}%
\pgfsetbuttcap%
\pgfsetroundjoin%
\definecolor{currentfill}{rgb}{0.000000,0.000000,0.000000}%
\pgfsetfillcolor{currentfill}%
\pgfsetlinewidth{0.501875pt}%
\definecolor{currentstroke}{rgb}{0.000000,0.000000,0.000000}%
\pgfsetstrokecolor{currentstroke}%
\pgfsetdash{}{0pt}%
\pgfsys@defobject{currentmarker}{\pgfqpoint{0.000000in}{0.000000in}}{\pgfqpoint{0.000000in}{0.055556in}}{%
\pgfpathmoveto{\pgfqpoint{0.000000in}{0.000000in}}%
\pgfpathlineto{\pgfqpoint{0.000000in}{0.055556in}}%
\pgfusepath{stroke,fill}%
}%
\begin{pgfscope}%
\pgfsys@transformshift{1.000000in}{0.350000in}%
\pgfsys@useobject{currentmarker}{}%
\end{pgfscope}%
\end{pgfscope}%
\begin{pgfscope}%
\pgfsetbuttcap%
\pgfsetroundjoin%
\definecolor{currentfill}{rgb}{0.000000,0.000000,0.000000}%
\pgfsetfillcolor{currentfill}%
\pgfsetlinewidth{0.501875pt}%
\definecolor{currentstroke}{rgb}{0.000000,0.000000,0.000000}%
\pgfsetstrokecolor{currentstroke}%
\pgfsetdash{}{0pt}%
\pgfsys@defobject{currentmarker}{\pgfqpoint{0.000000in}{-0.055556in}}{\pgfqpoint{0.000000in}{0.000000in}}{%
\pgfpathmoveto{\pgfqpoint{0.000000in}{0.000000in}}%
\pgfpathlineto{\pgfqpoint{0.000000in}{-0.055556in}}%
\pgfusepath{stroke,fill}%
}%
\begin{pgfscope}%
\pgfsys@transformshift{1.000000in}{3.150000in}%
\pgfsys@useobject{currentmarker}{}%
\end{pgfscope}%
\end{pgfscope}%
\begin{pgfscope}%
\pgftext[left,bottom,x=0.946981in,y=0.168387in,rotate=0.000000]{{\sffamily\fontsize{12.000000}{14.400000}\selectfont 0}}
%
\end{pgfscope}%
\begin{pgfscope}%
\pgfpathrectangle{\pgfqpoint{1.000000in}{0.350000in}}{\pgfqpoint{6.200000in}{2.800000in}} %
\pgfusepath{clip}%
\pgfsetbuttcap%
\pgfsetroundjoin%
\pgfsetlinewidth{0.501875pt}%
\definecolor{currentstroke}{rgb}{0.000000,0.000000,0.000000}%
\pgfsetstrokecolor{currentstroke}%
\pgfsetdash{{1.000000pt}{3.000000pt}}{0.000000pt}%
\pgfpathmoveto{\pgfqpoint{2.240000in}{0.350000in}}%
\pgfpathlineto{\pgfqpoint{2.240000in}{3.150000in}}%
\pgfusepath{stroke}%
\end{pgfscope}%
\begin{pgfscope}%
\pgfsetbuttcap%
\pgfsetroundjoin%
\definecolor{currentfill}{rgb}{0.000000,0.000000,0.000000}%
\pgfsetfillcolor{currentfill}%
\pgfsetlinewidth{0.501875pt}%
\definecolor{currentstroke}{rgb}{0.000000,0.000000,0.000000}%
\pgfsetstrokecolor{currentstroke}%
\pgfsetdash{}{0pt}%
\pgfsys@defobject{currentmarker}{\pgfqpoint{0.000000in}{0.000000in}}{\pgfqpoint{0.000000in}{0.055556in}}{%
\pgfpathmoveto{\pgfqpoint{0.000000in}{0.000000in}}%
\pgfpathlineto{\pgfqpoint{0.000000in}{0.055556in}}%
\pgfusepath{stroke,fill}%
}%
\begin{pgfscope}%
\pgfsys@transformshift{2.240000in}{0.350000in}%
\pgfsys@useobject{currentmarker}{}%
\end{pgfscope}%
\end{pgfscope}%
\begin{pgfscope}%
\pgfsetbuttcap%
\pgfsetroundjoin%
\definecolor{currentfill}{rgb}{0.000000,0.000000,0.000000}%
\pgfsetfillcolor{currentfill}%
\pgfsetlinewidth{0.501875pt}%
\definecolor{currentstroke}{rgb}{0.000000,0.000000,0.000000}%
\pgfsetstrokecolor{currentstroke}%
\pgfsetdash{}{0pt}%
\pgfsys@defobject{currentmarker}{\pgfqpoint{0.000000in}{-0.055556in}}{\pgfqpoint{0.000000in}{0.000000in}}{%
\pgfpathmoveto{\pgfqpoint{0.000000in}{0.000000in}}%
\pgfpathlineto{\pgfqpoint{0.000000in}{-0.055556in}}%
\pgfusepath{stroke,fill}%
}%
\begin{pgfscope}%
\pgfsys@transformshift{2.240000in}{3.150000in}%
\pgfsys@useobject{currentmarker}{}%
\end{pgfscope}%
\end{pgfscope}%
\begin{pgfscope}%
\pgftext[left,bottom,x=2.080942in,y=0.168387in,rotate=0.000000]{{\sffamily\fontsize{12.000000}{14.400000}\selectfont 100}}
%
\end{pgfscope}%
\begin{pgfscope}%
\pgfpathrectangle{\pgfqpoint{1.000000in}{0.350000in}}{\pgfqpoint{6.200000in}{2.800000in}} %
\pgfusepath{clip}%
\pgfsetbuttcap%
\pgfsetroundjoin%
\pgfsetlinewidth{0.501875pt}%
\definecolor{currentstroke}{rgb}{0.000000,0.000000,0.000000}%
\pgfsetstrokecolor{currentstroke}%
\pgfsetdash{{1.000000pt}{3.000000pt}}{0.000000pt}%
\pgfpathmoveto{\pgfqpoint{3.480000in}{0.350000in}}%
\pgfpathlineto{\pgfqpoint{3.480000in}{3.150000in}}%
\pgfusepath{stroke}%
\end{pgfscope}%
\begin{pgfscope}%
\pgfsetbuttcap%
\pgfsetroundjoin%
\definecolor{currentfill}{rgb}{0.000000,0.000000,0.000000}%
\pgfsetfillcolor{currentfill}%
\pgfsetlinewidth{0.501875pt}%
\definecolor{currentstroke}{rgb}{0.000000,0.000000,0.000000}%
\pgfsetstrokecolor{currentstroke}%
\pgfsetdash{}{0pt}%
\pgfsys@defobject{currentmarker}{\pgfqpoint{0.000000in}{0.000000in}}{\pgfqpoint{0.000000in}{0.055556in}}{%
\pgfpathmoveto{\pgfqpoint{0.000000in}{0.000000in}}%
\pgfpathlineto{\pgfqpoint{0.000000in}{0.055556in}}%
\pgfusepath{stroke,fill}%
}%
\begin{pgfscope}%
\pgfsys@transformshift{3.480000in}{0.350000in}%
\pgfsys@useobject{currentmarker}{}%
\end{pgfscope}%
\end{pgfscope}%
\begin{pgfscope}%
\pgfsetbuttcap%
\pgfsetroundjoin%
\definecolor{currentfill}{rgb}{0.000000,0.000000,0.000000}%
\pgfsetfillcolor{currentfill}%
\pgfsetlinewidth{0.501875pt}%
\definecolor{currentstroke}{rgb}{0.000000,0.000000,0.000000}%
\pgfsetstrokecolor{currentstroke}%
\pgfsetdash{}{0pt}%
\pgfsys@defobject{currentmarker}{\pgfqpoint{0.000000in}{-0.055556in}}{\pgfqpoint{0.000000in}{0.000000in}}{%
\pgfpathmoveto{\pgfqpoint{0.000000in}{0.000000in}}%
\pgfpathlineto{\pgfqpoint{0.000000in}{-0.055556in}}%
\pgfusepath{stroke,fill}%
}%
\begin{pgfscope}%
\pgfsys@transformshift{3.480000in}{3.150000in}%
\pgfsys@useobject{currentmarker}{}%
\end{pgfscope}%
\end{pgfscope}%
\begin{pgfscope}%
\pgftext[left,bottom,x=3.320942in,y=0.168387in,rotate=0.000000]{{\sffamily\fontsize{12.000000}{14.400000}\selectfont 200}}
%
\end{pgfscope}%
\begin{pgfscope}%
\pgfpathrectangle{\pgfqpoint{1.000000in}{0.350000in}}{\pgfqpoint{6.200000in}{2.800000in}} %
\pgfusepath{clip}%
\pgfsetbuttcap%
\pgfsetroundjoin%
\pgfsetlinewidth{0.501875pt}%
\definecolor{currentstroke}{rgb}{0.000000,0.000000,0.000000}%
\pgfsetstrokecolor{currentstroke}%
\pgfsetdash{{1.000000pt}{3.000000pt}}{0.000000pt}%
\pgfpathmoveto{\pgfqpoint{4.720000in}{0.350000in}}%
\pgfpathlineto{\pgfqpoint{4.720000in}{3.150000in}}%
\pgfusepath{stroke}%
\end{pgfscope}%
\begin{pgfscope}%
\pgfsetbuttcap%
\pgfsetroundjoin%
\definecolor{currentfill}{rgb}{0.000000,0.000000,0.000000}%
\pgfsetfillcolor{currentfill}%
\pgfsetlinewidth{0.501875pt}%
\definecolor{currentstroke}{rgb}{0.000000,0.000000,0.000000}%
\pgfsetstrokecolor{currentstroke}%
\pgfsetdash{}{0pt}%
\pgfsys@defobject{currentmarker}{\pgfqpoint{0.000000in}{0.000000in}}{\pgfqpoint{0.000000in}{0.055556in}}{%
\pgfpathmoveto{\pgfqpoint{0.000000in}{0.000000in}}%
\pgfpathlineto{\pgfqpoint{0.000000in}{0.055556in}}%
\pgfusepath{stroke,fill}%
}%
\begin{pgfscope}%
\pgfsys@transformshift{4.720000in}{0.350000in}%
\pgfsys@useobject{currentmarker}{}%
\end{pgfscope}%
\end{pgfscope}%
\begin{pgfscope}%
\pgfsetbuttcap%
\pgfsetroundjoin%
\definecolor{currentfill}{rgb}{0.000000,0.000000,0.000000}%
\pgfsetfillcolor{currentfill}%
\pgfsetlinewidth{0.501875pt}%
\definecolor{currentstroke}{rgb}{0.000000,0.000000,0.000000}%
\pgfsetstrokecolor{currentstroke}%
\pgfsetdash{}{0pt}%
\pgfsys@defobject{currentmarker}{\pgfqpoint{0.000000in}{-0.055556in}}{\pgfqpoint{0.000000in}{0.000000in}}{%
\pgfpathmoveto{\pgfqpoint{0.000000in}{0.000000in}}%
\pgfpathlineto{\pgfqpoint{0.000000in}{-0.055556in}}%
\pgfusepath{stroke,fill}%
}%
\begin{pgfscope}%
\pgfsys@transformshift{4.720000in}{3.150000in}%
\pgfsys@useobject{currentmarker}{}%
\end{pgfscope}%
\end{pgfscope}%
\begin{pgfscope}%
\pgftext[left,bottom,x=4.560942in,y=0.168387in,rotate=0.000000]{{\sffamily\fontsize{12.000000}{14.400000}\selectfont 300}}
%
\end{pgfscope}%
\begin{pgfscope}%
\pgfpathrectangle{\pgfqpoint{1.000000in}{0.350000in}}{\pgfqpoint{6.200000in}{2.800000in}} %
\pgfusepath{clip}%
\pgfsetbuttcap%
\pgfsetroundjoin%
\pgfsetlinewidth{0.501875pt}%
\definecolor{currentstroke}{rgb}{0.000000,0.000000,0.000000}%
\pgfsetstrokecolor{currentstroke}%
\pgfsetdash{{1.000000pt}{3.000000pt}}{0.000000pt}%
\pgfpathmoveto{\pgfqpoint{5.960000in}{0.350000in}}%
\pgfpathlineto{\pgfqpoint{5.960000in}{3.150000in}}%
\pgfusepath{stroke}%
\end{pgfscope}%
\begin{pgfscope}%
\pgfsetbuttcap%
\pgfsetroundjoin%
\definecolor{currentfill}{rgb}{0.000000,0.000000,0.000000}%
\pgfsetfillcolor{currentfill}%
\pgfsetlinewidth{0.501875pt}%
\definecolor{currentstroke}{rgb}{0.000000,0.000000,0.000000}%
\pgfsetstrokecolor{currentstroke}%
\pgfsetdash{}{0pt}%
\pgfsys@defobject{currentmarker}{\pgfqpoint{0.000000in}{0.000000in}}{\pgfqpoint{0.000000in}{0.055556in}}{%
\pgfpathmoveto{\pgfqpoint{0.000000in}{0.000000in}}%
\pgfpathlineto{\pgfqpoint{0.000000in}{0.055556in}}%
\pgfusepath{stroke,fill}%
}%
\begin{pgfscope}%
\pgfsys@transformshift{5.960000in}{0.350000in}%
\pgfsys@useobject{currentmarker}{}%
\end{pgfscope}%
\end{pgfscope}%
\begin{pgfscope}%
\pgfsetbuttcap%
\pgfsetroundjoin%
\definecolor{currentfill}{rgb}{0.000000,0.000000,0.000000}%
\pgfsetfillcolor{currentfill}%
\pgfsetlinewidth{0.501875pt}%
\definecolor{currentstroke}{rgb}{0.000000,0.000000,0.000000}%
\pgfsetstrokecolor{currentstroke}%
\pgfsetdash{}{0pt}%
\pgfsys@defobject{currentmarker}{\pgfqpoint{0.000000in}{-0.055556in}}{\pgfqpoint{0.000000in}{0.000000in}}{%
\pgfpathmoveto{\pgfqpoint{0.000000in}{0.000000in}}%
\pgfpathlineto{\pgfqpoint{0.000000in}{-0.055556in}}%
\pgfusepath{stroke,fill}%
}%
\begin{pgfscope}%
\pgfsys@transformshift{5.960000in}{3.150000in}%
\pgfsys@useobject{currentmarker}{}%
\end{pgfscope}%
\end{pgfscope}%
\begin{pgfscope}%
\pgftext[left,bottom,x=5.800942in,y=0.168387in,rotate=0.000000]{{\sffamily\fontsize{12.000000}{14.400000}\selectfont 400}}
%
\end{pgfscope}%
\begin{pgfscope}%
\pgfpathrectangle{\pgfqpoint{1.000000in}{0.350000in}}{\pgfqpoint{6.200000in}{2.800000in}} %
\pgfusepath{clip}%
\pgfsetbuttcap%
\pgfsetroundjoin%
\pgfsetlinewidth{0.501875pt}%
\definecolor{currentstroke}{rgb}{0.000000,0.000000,0.000000}%
\pgfsetstrokecolor{currentstroke}%
\pgfsetdash{{1.000000pt}{3.000000pt}}{0.000000pt}%
\pgfpathmoveto{\pgfqpoint{7.200000in}{0.350000in}}%
\pgfpathlineto{\pgfqpoint{7.200000in}{3.150000in}}%
\pgfusepath{stroke}%
\end{pgfscope}%
\begin{pgfscope}%
\pgfsetbuttcap%
\pgfsetroundjoin%
\definecolor{currentfill}{rgb}{0.000000,0.000000,0.000000}%
\pgfsetfillcolor{currentfill}%
\pgfsetlinewidth{0.501875pt}%
\definecolor{currentstroke}{rgb}{0.000000,0.000000,0.000000}%
\pgfsetstrokecolor{currentstroke}%
\pgfsetdash{}{0pt}%
\pgfsys@defobject{currentmarker}{\pgfqpoint{0.000000in}{0.000000in}}{\pgfqpoint{0.000000in}{0.055556in}}{%
\pgfpathmoveto{\pgfqpoint{0.000000in}{0.000000in}}%
\pgfpathlineto{\pgfqpoint{0.000000in}{0.055556in}}%
\pgfusepath{stroke,fill}%
}%
\begin{pgfscope}%
\pgfsys@transformshift{7.200000in}{0.350000in}%
\pgfsys@useobject{currentmarker}{}%
\end{pgfscope}%
\end{pgfscope}%
\begin{pgfscope}%
\pgfsetbuttcap%
\pgfsetroundjoin%
\definecolor{currentfill}{rgb}{0.000000,0.000000,0.000000}%
\pgfsetfillcolor{currentfill}%
\pgfsetlinewidth{0.501875pt}%
\definecolor{currentstroke}{rgb}{0.000000,0.000000,0.000000}%
\pgfsetstrokecolor{currentstroke}%
\pgfsetdash{}{0pt}%
\pgfsys@defobject{currentmarker}{\pgfqpoint{0.000000in}{-0.055556in}}{\pgfqpoint{0.000000in}{0.000000in}}{%
\pgfpathmoveto{\pgfqpoint{0.000000in}{0.000000in}}%
\pgfpathlineto{\pgfqpoint{0.000000in}{-0.055556in}}%
\pgfusepath{stroke,fill}%
}%
\begin{pgfscope}%
\pgfsys@transformshift{7.200000in}{3.150000in}%
\pgfsys@useobject{currentmarker}{}%
\end{pgfscope}%
\end{pgfscope}%
\begin{pgfscope}%
\pgftext[left,bottom,x=7.040942in,y=0.168387in,rotate=0.000000]{{\sffamily\fontsize{12.000000}{14.400000}\selectfont 500}}
%
\end{pgfscope}%
\begin{pgfscope}%
\pgftext[left,bottom,x=3.911727in,y=-0.030045in,rotate=0.000000]{{\sffamily\fontsize{12.000000}{14.400000}\selectfont time}}
%
\end{pgfscope}%
\begin{pgfscope}%
\pgfpathrectangle{\pgfqpoint{1.000000in}{0.350000in}}{\pgfqpoint{6.200000in}{2.800000in}} %
\pgfusepath{clip}%
\pgfsetbuttcap%
\pgfsetroundjoin%
\pgfsetlinewidth{0.501875pt}%
\definecolor{currentstroke}{rgb}{0.000000,0.000000,0.000000}%
\pgfsetstrokecolor{currentstroke}%
\pgfsetdash{{1.000000pt}{3.000000pt}}{0.000000pt}%
\pgfpathmoveto{\pgfqpoint{1.000000in}{0.350000in}}%
\pgfpathlineto{\pgfqpoint{7.200000in}{0.350000in}}%
\pgfusepath{stroke}%
\end{pgfscope}%
\begin{pgfscope}%
\pgfsetbuttcap%
\pgfsetroundjoin%
\definecolor{currentfill}{rgb}{0.000000,0.000000,0.000000}%
\pgfsetfillcolor{currentfill}%
\pgfsetlinewidth{0.501875pt}%
\definecolor{currentstroke}{rgb}{0.000000,0.000000,0.000000}%
\pgfsetstrokecolor{currentstroke}%
\pgfsetdash{}{0pt}%
\pgfsys@defobject{currentmarker}{\pgfqpoint{0.000000in}{0.000000in}}{\pgfqpoint{0.055556in}{0.000000in}}{%
\pgfpathmoveto{\pgfqpoint{0.000000in}{0.000000in}}%
\pgfpathlineto{\pgfqpoint{0.055556in}{0.000000in}}%
\pgfusepath{stroke,fill}%
}%
\begin{pgfscope}%
\pgfsys@transformshift{1.000000in}{0.350000in}%
\pgfsys@useobject{currentmarker}{}%
\end{pgfscope}%
\end{pgfscope}%
\begin{pgfscope}%
\pgfsetbuttcap%
\pgfsetroundjoin%
\definecolor{currentfill}{rgb}{0.000000,0.000000,0.000000}%
\pgfsetfillcolor{currentfill}%
\pgfsetlinewidth{0.501875pt}%
\definecolor{currentstroke}{rgb}{0.000000,0.000000,0.000000}%
\pgfsetstrokecolor{currentstroke}%
\pgfsetdash{}{0pt}%
\pgfsys@defobject{currentmarker}{\pgfqpoint{-0.055556in}{0.000000in}}{\pgfqpoint{0.000000in}{0.000000in}}{%
\pgfpathmoveto{\pgfqpoint{0.000000in}{0.000000in}}%
\pgfpathlineto{\pgfqpoint{-0.055556in}{0.000000in}}%
\pgfusepath{stroke,fill}%
}%
\begin{pgfscope}%
\pgfsys@transformshift{7.200000in}{0.350000in}%
\pgfsys@useobject{currentmarker}{}%
\end{pgfscope}%
\end{pgfscope}%
\begin{pgfscope}%
\pgftext[left,bottom,x=0.361274in,y=0.286971in,rotate=0.000000]{{\sffamily\fontsize{12.000000}{14.400000}\selectfont 0.0020}}
%
\end{pgfscope}%
\begin{pgfscope}%
\pgfpathrectangle{\pgfqpoint{1.000000in}{0.350000in}}{\pgfqpoint{6.200000in}{2.800000in}} %
\pgfusepath{clip}%
\pgfsetbuttcap%
\pgfsetroundjoin%
\pgfsetlinewidth{0.501875pt}%
\definecolor{currentstroke}{rgb}{0.000000,0.000000,0.000000}%
\pgfsetstrokecolor{currentstroke}%
\pgfsetdash{{1.000000pt}{3.000000pt}}{0.000000pt}%
\pgfpathmoveto{\pgfqpoint{1.000000in}{0.661111in}}%
\pgfpathlineto{\pgfqpoint{7.200000in}{0.661111in}}%
\pgfusepath{stroke}%
\end{pgfscope}%
\begin{pgfscope}%
\pgfsetbuttcap%
\pgfsetroundjoin%
\definecolor{currentfill}{rgb}{0.000000,0.000000,0.000000}%
\pgfsetfillcolor{currentfill}%
\pgfsetlinewidth{0.501875pt}%
\definecolor{currentstroke}{rgb}{0.000000,0.000000,0.000000}%
\pgfsetstrokecolor{currentstroke}%
\pgfsetdash{}{0pt}%
\pgfsys@defobject{currentmarker}{\pgfqpoint{0.000000in}{0.000000in}}{\pgfqpoint{0.055556in}{0.000000in}}{%
\pgfpathmoveto{\pgfqpoint{0.000000in}{0.000000in}}%
\pgfpathlineto{\pgfqpoint{0.055556in}{0.000000in}}%
\pgfusepath{stroke,fill}%
}%
\begin{pgfscope}%
\pgfsys@transformshift{1.000000in}{0.661111in}%
\pgfsys@useobject{currentmarker}{}%
\end{pgfscope}%
\end{pgfscope}%
\begin{pgfscope}%
\pgfsetbuttcap%
\pgfsetroundjoin%
\definecolor{currentfill}{rgb}{0.000000,0.000000,0.000000}%
\pgfsetfillcolor{currentfill}%
\pgfsetlinewidth{0.501875pt}%
\definecolor{currentstroke}{rgb}{0.000000,0.000000,0.000000}%
\pgfsetstrokecolor{currentstroke}%
\pgfsetdash{}{0pt}%
\pgfsys@defobject{currentmarker}{\pgfqpoint{-0.055556in}{0.000000in}}{\pgfqpoint{0.000000in}{0.000000in}}{%
\pgfpathmoveto{\pgfqpoint{0.000000in}{0.000000in}}%
\pgfpathlineto{\pgfqpoint{-0.055556in}{0.000000in}}%
\pgfusepath{stroke,fill}%
}%
\begin{pgfscope}%
\pgfsys@transformshift{7.200000in}{0.661111in}%
\pgfsys@useobject{currentmarker}{}%
\end{pgfscope}%
\end{pgfscope}%
\begin{pgfscope}%
\pgftext[left,bottom,x=0.361274in,y=0.598082in,rotate=0.000000]{{\sffamily\fontsize{12.000000}{14.400000}\selectfont 0.0025}}
%
\end{pgfscope}%
\begin{pgfscope}%
\pgfpathrectangle{\pgfqpoint{1.000000in}{0.350000in}}{\pgfqpoint{6.200000in}{2.800000in}} %
\pgfusepath{clip}%
\pgfsetbuttcap%
\pgfsetroundjoin%
\pgfsetlinewidth{0.501875pt}%
\definecolor{currentstroke}{rgb}{0.000000,0.000000,0.000000}%
\pgfsetstrokecolor{currentstroke}%
\pgfsetdash{{1.000000pt}{3.000000pt}}{0.000000pt}%
\pgfpathmoveto{\pgfqpoint{1.000000in}{0.972222in}}%
\pgfpathlineto{\pgfqpoint{7.200000in}{0.972222in}}%
\pgfusepath{stroke}%
\end{pgfscope}%
\begin{pgfscope}%
\pgfsetbuttcap%
\pgfsetroundjoin%
\definecolor{currentfill}{rgb}{0.000000,0.000000,0.000000}%
\pgfsetfillcolor{currentfill}%
\pgfsetlinewidth{0.501875pt}%
\definecolor{currentstroke}{rgb}{0.000000,0.000000,0.000000}%
\pgfsetstrokecolor{currentstroke}%
\pgfsetdash{}{0pt}%
\pgfsys@defobject{currentmarker}{\pgfqpoint{0.000000in}{0.000000in}}{\pgfqpoint{0.055556in}{0.000000in}}{%
\pgfpathmoveto{\pgfqpoint{0.000000in}{0.000000in}}%
\pgfpathlineto{\pgfqpoint{0.055556in}{0.000000in}}%
\pgfusepath{stroke,fill}%
}%
\begin{pgfscope}%
\pgfsys@transformshift{1.000000in}{0.972222in}%
\pgfsys@useobject{currentmarker}{}%
\end{pgfscope}%
\end{pgfscope}%
\begin{pgfscope}%
\pgfsetbuttcap%
\pgfsetroundjoin%
\definecolor{currentfill}{rgb}{0.000000,0.000000,0.000000}%
\pgfsetfillcolor{currentfill}%
\pgfsetlinewidth{0.501875pt}%
\definecolor{currentstroke}{rgb}{0.000000,0.000000,0.000000}%
\pgfsetstrokecolor{currentstroke}%
\pgfsetdash{}{0pt}%
\pgfsys@defobject{currentmarker}{\pgfqpoint{-0.055556in}{0.000000in}}{\pgfqpoint{0.000000in}{0.000000in}}{%
\pgfpathmoveto{\pgfqpoint{0.000000in}{0.000000in}}%
\pgfpathlineto{\pgfqpoint{-0.055556in}{0.000000in}}%
\pgfusepath{stroke,fill}%
}%
\begin{pgfscope}%
\pgfsys@transformshift{7.200000in}{0.972222in}%
\pgfsys@useobject{currentmarker}{}%
\end{pgfscope}%
\end{pgfscope}%
\begin{pgfscope}%
\pgftext[left,bottom,x=0.361274in,y=0.909193in,rotate=0.000000]{{\sffamily\fontsize{12.000000}{14.400000}\selectfont 0.0030}}
%
\end{pgfscope}%
\begin{pgfscope}%
\pgfpathrectangle{\pgfqpoint{1.000000in}{0.350000in}}{\pgfqpoint{6.200000in}{2.800000in}} %
\pgfusepath{clip}%
\pgfsetbuttcap%
\pgfsetroundjoin%
\pgfsetlinewidth{0.501875pt}%
\definecolor{currentstroke}{rgb}{0.000000,0.000000,0.000000}%
\pgfsetstrokecolor{currentstroke}%
\pgfsetdash{{1.000000pt}{3.000000pt}}{0.000000pt}%
\pgfpathmoveto{\pgfqpoint{1.000000in}{1.283333in}}%
\pgfpathlineto{\pgfqpoint{7.200000in}{1.283333in}}%
\pgfusepath{stroke}%
\end{pgfscope}%
\begin{pgfscope}%
\pgfsetbuttcap%
\pgfsetroundjoin%
\definecolor{currentfill}{rgb}{0.000000,0.000000,0.000000}%
\pgfsetfillcolor{currentfill}%
\pgfsetlinewidth{0.501875pt}%
\definecolor{currentstroke}{rgb}{0.000000,0.000000,0.000000}%
\pgfsetstrokecolor{currentstroke}%
\pgfsetdash{}{0pt}%
\pgfsys@defobject{currentmarker}{\pgfqpoint{0.000000in}{0.000000in}}{\pgfqpoint{0.055556in}{0.000000in}}{%
\pgfpathmoveto{\pgfqpoint{0.000000in}{0.000000in}}%
\pgfpathlineto{\pgfqpoint{0.055556in}{0.000000in}}%
\pgfusepath{stroke,fill}%
}%
\begin{pgfscope}%
\pgfsys@transformshift{1.000000in}{1.283333in}%
\pgfsys@useobject{currentmarker}{}%
\end{pgfscope}%
\end{pgfscope}%
\begin{pgfscope}%
\pgfsetbuttcap%
\pgfsetroundjoin%
\definecolor{currentfill}{rgb}{0.000000,0.000000,0.000000}%
\pgfsetfillcolor{currentfill}%
\pgfsetlinewidth{0.501875pt}%
\definecolor{currentstroke}{rgb}{0.000000,0.000000,0.000000}%
\pgfsetstrokecolor{currentstroke}%
\pgfsetdash{}{0pt}%
\pgfsys@defobject{currentmarker}{\pgfqpoint{-0.055556in}{0.000000in}}{\pgfqpoint{0.000000in}{0.000000in}}{%
\pgfpathmoveto{\pgfqpoint{0.000000in}{0.000000in}}%
\pgfpathlineto{\pgfqpoint{-0.055556in}{0.000000in}}%
\pgfusepath{stroke,fill}%
}%
\begin{pgfscope}%
\pgfsys@transformshift{7.200000in}{1.283333in}%
\pgfsys@useobject{currentmarker}{}%
\end{pgfscope}%
\end{pgfscope}%
\begin{pgfscope}%
\pgftext[left,bottom,x=0.361274in,y=1.220305in,rotate=0.000000]{{\sffamily\fontsize{12.000000}{14.400000}\selectfont 0.0035}}
%
\end{pgfscope}%
\begin{pgfscope}%
\pgfpathrectangle{\pgfqpoint{1.000000in}{0.350000in}}{\pgfqpoint{6.200000in}{2.800000in}} %
\pgfusepath{clip}%
\pgfsetbuttcap%
\pgfsetroundjoin%
\pgfsetlinewidth{0.501875pt}%
\definecolor{currentstroke}{rgb}{0.000000,0.000000,0.000000}%
\pgfsetstrokecolor{currentstroke}%
\pgfsetdash{{1.000000pt}{3.000000pt}}{0.000000pt}%
\pgfpathmoveto{\pgfqpoint{1.000000in}{1.594444in}}%
\pgfpathlineto{\pgfqpoint{7.200000in}{1.594444in}}%
\pgfusepath{stroke}%
\end{pgfscope}%
\begin{pgfscope}%
\pgfsetbuttcap%
\pgfsetroundjoin%
\definecolor{currentfill}{rgb}{0.000000,0.000000,0.000000}%
\pgfsetfillcolor{currentfill}%
\pgfsetlinewidth{0.501875pt}%
\definecolor{currentstroke}{rgb}{0.000000,0.000000,0.000000}%
\pgfsetstrokecolor{currentstroke}%
\pgfsetdash{}{0pt}%
\pgfsys@defobject{currentmarker}{\pgfqpoint{0.000000in}{0.000000in}}{\pgfqpoint{0.055556in}{0.000000in}}{%
\pgfpathmoveto{\pgfqpoint{0.000000in}{0.000000in}}%
\pgfpathlineto{\pgfqpoint{0.055556in}{0.000000in}}%
\pgfusepath{stroke,fill}%
}%
\begin{pgfscope}%
\pgfsys@transformshift{1.000000in}{1.594444in}%
\pgfsys@useobject{currentmarker}{}%
\end{pgfscope}%
\end{pgfscope}%
\begin{pgfscope}%
\pgfsetbuttcap%
\pgfsetroundjoin%
\definecolor{currentfill}{rgb}{0.000000,0.000000,0.000000}%
\pgfsetfillcolor{currentfill}%
\pgfsetlinewidth{0.501875pt}%
\definecolor{currentstroke}{rgb}{0.000000,0.000000,0.000000}%
\pgfsetstrokecolor{currentstroke}%
\pgfsetdash{}{0pt}%
\pgfsys@defobject{currentmarker}{\pgfqpoint{-0.055556in}{0.000000in}}{\pgfqpoint{0.000000in}{0.000000in}}{%
\pgfpathmoveto{\pgfqpoint{0.000000in}{0.000000in}}%
\pgfpathlineto{\pgfqpoint{-0.055556in}{0.000000in}}%
\pgfusepath{stroke,fill}%
}%
\begin{pgfscope}%
\pgfsys@transformshift{7.200000in}{1.594444in}%
\pgfsys@useobject{currentmarker}{}%
\end{pgfscope}%
\end{pgfscope}%
\begin{pgfscope}%
\pgftext[left,bottom,x=0.361274in,y=1.531416in,rotate=0.000000]{{\sffamily\fontsize{12.000000}{14.400000}\selectfont 0.0040}}
%
\end{pgfscope}%
\begin{pgfscope}%
\pgfpathrectangle{\pgfqpoint{1.000000in}{0.350000in}}{\pgfqpoint{6.200000in}{2.800000in}} %
\pgfusepath{clip}%
\pgfsetbuttcap%
\pgfsetroundjoin%
\pgfsetlinewidth{0.501875pt}%
\definecolor{currentstroke}{rgb}{0.000000,0.000000,0.000000}%
\pgfsetstrokecolor{currentstroke}%
\pgfsetdash{{1.000000pt}{3.000000pt}}{0.000000pt}%
\pgfpathmoveto{\pgfqpoint{1.000000in}{1.905556in}}%
\pgfpathlineto{\pgfqpoint{7.200000in}{1.905556in}}%
\pgfusepath{stroke}%
\end{pgfscope}%
\begin{pgfscope}%
\pgfsetbuttcap%
\pgfsetroundjoin%
\definecolor{currentfill}{rgb}{0.000000,0.000000,0.000000}%
\pgfsetfillcolor{currentfill}%
\pgfsetlinewidth{0.501875pt}%
\definecolor{currentstroke}{rgb}{0.000000,0.000000,0.000000}%
\pgfsetstrokecolor{currentstroke}%
\pgfsetdash{}{0pt}%
\pgfsys@defobject{currentmarker}{\pgfqpoint{0.000000in}{0.000000in}}{\pgfqpoint{0.055556in}{0.000000in}}{%
\pgfpathmoveto{\pgfqpoint{0.000000in}{0.000000in}}%
\pgfpathlineto{\pgfqpoint{0.055556in}{0.000000in}}%
\pgfusepath{stroke,fill}%
}%
\begin{pgfscope}%
\pgfsys@transformshift{1.000000in}{1.905556in}%
\pgfsys@useobject{currentmarker}{}%
\end{pgfscope}%
\end{pgfscope}%
\begin{pgfscope}%
\pgfsetbuttcap%
\pgfsetroundjoin%
\definecolor{currentfill}{rgb}{0.000000,0.000000,0.000000}%
\pgfsetfillcolor{currentfill}%
\pgfsetlinewidth{0.501875pt}%
\definecolor{currentstroke}{rgb}{0.000000,0.000000,0.000000}%
\pgfsetstrokecolor{currentstroke}%
\pgfsetdash{}{0pt}%
\pgfsys@defobject{currentmarker}{\pgfqpoint{-0.055556in}{0.000000in}}{\pgfqpoint{0.000000in}{0.000000in}}{%
\pgfpathmoveto{\pgfqpoint{0.000000in}{0.000000in}}%
\pgfpathlineto{\pgfqpoint{-0.055556in}{0.000000in}}%
\pgfusepath{stroke,fill}%
}%
\begin{pgfscope}%
\pgfsys@transformshift{7.200000in}{1.905556in}%
\pgfsys@useobject{currentmarker}{}%
\end{pgfscope}%
\end{pgfscope}%
\begin{pgfscope}%
\pgftext[left,bottom,x=0.361274in,y=1.842527in,rotate=0.000000]{{\sffamily\fontsize{12.000000}{14.400000}\selectfont 0.0045}}
%
\end{pgfscope}%
\begin{pgfscope}%
\pgfpathrectangle{\pgfqpoint{1.000000in}{0.350000in}}{\pgfqpoint{6.200000in}{2.800000in}} %
\pgfusepath{clip}%
\pgfsetbuttcap%
\pgfsetroundjoin%
\pgfsetlinewidth{0.501875pt}%
\definecolor{currentstroke}{rgb}{0.000000,0.000000,0.000000}%
\pgfsetstrokecolor{currentstroke}%
\pgfsetdash{{1.000000pt}{3.000000pt}}{0.000000pt}%
\pgfpathmoveto{\pgfqpoint{1.000000in}{2.216667in}}%
\pgfpathlineto{\pgfqpoint{7.200000in}{2.216667in}}%
\pgfusepath{stroke}%
\end{pgfscope}%
\begin{pgfscope}%
\pgfsetbuttcap%
\pgfsetroundjoin%
\definecolor{currentfill}{rgb}{0.000000,0.000000,0.000000}%
\pgfsetfillcolor{currentfill}%
\pgfsetlinewidth{0.501875pt}%
\definecolor{currentstroke}{rgb}{0.000000,0.000000,0.000000}%
\pgfsetstrokecolor{currentstroke}%
\pgfsetdash{}{0pt}%
\pgfsys@defobject{currentmarker}{\pgfqpoint{0.000000in}{0.000000in}}{\pgfqpoint{0.055556in}{0.000000in}}{%
\pgfpathmoveto{\pgfqpoint{0.000000in}{0.000000in}}%
\pgfpathlineto{\pgfqpoint{0.055556in}{0.000000in}}%
\pgfusepath{stroke,fill}%
}%
\begin{pgfscope}%
\pgfsys@transformshift{1.000000in}{2.216667in}%
\pgfsys@useobject{currentmarker}{}%
\end{pgfscope}%
\end{pgfscope}%
\begin{pgfscope}%
\pgfsetbuttcap%
\pgfsetroundjoin%
\definecolor{currentfill}{rgb}{0.000000,0.000000,0.000000}%
\pgfsetfillcolor{currentfill}%
\pgfsetlinewidth{0.501875pt}%
\definecolor{currentstroke}{rgb}{0.000000,0.000000,0.000000}%
\pgfsetstrokecolor{currentstroke}%
\pgfsetdash{}{0pt}%
\pgfsys@defobject{currentmarker}{\pgfqpoint{-0.055556in}{0.000000in}}{\pgfqpoint{0.000000in}{0.000000in}}{%
\pgfpathmoveto{\pgfqpoint{0.000000in}{0.000000in}}%
\pgfpathlineto{\pgfqpoint{-0.055556in}{0.000000in}}%
\pgfusepath{stroke,fill}%
}%
\begin{pgfscope}%
\pgfsys@transformshift{7.200000in}{2.216667in}%
\pgfsys@useobject{currentmarker}{}%
\end{pgfscope}%
\end{pgfscope}%
\begin{pgfscope}%
\pgftext[left,bottom,x=0.361274in,y=2.153638in,rotate=0.000000]{{\sffamily\fontsize{12.000000}{14.400000}\selectfont 0.0050}}
%
\end{pgfscope}%
\begin{pgfscope}%
\pgfpathrectangle{\pgfqpoint{1.000000in}{0.350000in}}{\pgfqpoint{6.200000in}{2.800000in}} %
\pgfusepath{clip}%
\pgfsetbuttcap%
\pgfsetroundjoin%
\pgfsetlinewidth{0.501875pt}%
\definecolor{currentstroke}{rgb}{0.000000,0.000000,0.000000}%
\pgfsetstrokecolor{currentstroke}%
\pgfsetdash{{1.000000pt}{3.000000pt}}{0.000000pt}%
\pgfpathmoveto{\pgfqpoint{1.000000in}{2.527778in}}%
\pgfpathlineto{\pgfqpoint{7.200000in}{2.527778in}}%
\pgfusepath{stroke}%
\end{pgfscope}%
\begin{pgfscope}%
\pgfsetbuttcap%
\pgfsetroundjoin%
\definecolor{currentfill}{rgb}{0.000000,0.000000,0.000000}%
\pgfsetfillcolor{currentfill}%
\pgfsetlinewidth{0.501875pt}%
\definecolor{currentstroke}{rgb}{0.000000,0.000000,0.000000}%
\pgfsetstrokecolor{currentstroke}%
\pgfsetdash{}{0pt}%
\pgfsys@defobject{currentmarker}{\pgfqpoint{0.000000in}{0.000000in}}{\pgfqpoint{0.055556in}{0.000000in}}{%
\pgfpathmoveto{\pgfqpoint{0.000000in}{0.000000in}}%
\pgfpathlineto{\pgfqpoint{0.055556in}{0.000000in}}%
\pgfusepath{stroke,fill}%
}%
\begin{pgfscope}%
\pgfsys@transformshift{1.000000in}{2.527778in}%
\pgfsys@useobject{currentmarker}{}%
\end{pgfscope}%
\end{pgfscope}%
\begin{pgfscope}%
\pgfsetbuttcap%
\pgfsetroundjoin%
\definecolor{currentfill}{rgb}{0.000000,0.000000,0.000000}%
\pgfsetfillcolor{currentfill}%
\pgfsetlinewidth{0.501875pt}%
\definecolor{currentstroke}{rgb}{0.000000,0.000000,0.000000}%
\pgfsetstrokecolor{currentstroke}%
\pgfsetdash{}{0pt}%
\pgfsys@defobject{currentmarker}{\pgfqpoint{-0.055556in}{0.000000in}}{\pgfqpoint{0.000000in}{0.000000in}}{%
\pgfpathmoveto{\pgfqpoint{0.000000in}{0.000000in}}%
\pgfpathlineto{\pgfqpoint{-0.055556in}{0.000000in}}%
\pgfusepath{stroke,fill}%
}%
\begin{pgfscope}%
\pgfsys@transformshift{7.200000in}{2.527778in}%
\pgfsys@useobject{currentmarker}{}%
\end{pgfscope}%
\end{pgfscope}%
\begin{pgfscope}%
\pgftext[left,bottom,x=0.361274in,y=2.464749in,rotate=0.000000]{{\sffamily\fontsize{12.000000}{14.400000}\selectfont 0.0055}}
%
\end{pgfscope}%
\begin{pgfscope}%
\pgfpathrectangle{\pgfqpoint{1.000000in}{0.350000in}}{\pgfqpoint{6.200000in}{2.800000in}} %
\pgfusepath{clip}%
\pgfsetbuttcap%
\pgfsetroundjoin%
\pgfsetlinewidth{0.501875pt}%
\definecolor{currentstroke}{rgb}{0.000000,0.000000,0.000000}%
\pgfsetstrokecolor{currentstroke}%
\pgfsetdash{{1.000000pt}{3.000000pt}}{0.000000pt}%
\pgfpathmoveto{\pgfqpoint{1.000000in}{2.838889in}}%
\pgfpathlineto{\pgfqpoint{7.200000in}{2.838889in}}%
\pgfusepath{stroke}%
\end{pgfscope}%
\begin{pgfscope}%
\pgfsetbuttcap%
\pgfsetroundjoin%
\definecolor{currentfill}{rgb}{0.000000,0.000000,0.000000}%
\pgfsetfillcolor{currentfill}%
\pgfsetlinewidth{0.501875pt}%
\definecolor{currentstroke}{rgb}{0.000000,0.000000,0.000000}%
\pgfsetstrokecolor{currentstroke}%
\pgfsetdash{}{0pt}%
\pgfsys@defobject{currentmarker}{\pgfqpoint{0.000000in}{0.000000in}}{\pgfqpoint{0.055556in}{0.000000in}}{%
\pgfpathmoveto{\pgfqpoint{0.000000in}{0.000000in}}%
\pgfpathlineto{\pgfqpoint{0.055556in}{0.000000in}}%
\pgfusepath{stroke,fill}%
}%
\begin{pgfscope}%
\pgfsys@transformshift{1.000000in}{2.838889in}%
\pgfsys@useobject{currentmarker}{}%
\end{pgfscope}%
\end{pgfscope}%
\begin{pgfscope}%
\pgfsetbuttcap%
\pgfsetroundjoin%
\definecolor{currentfill}{rgb}{0.000000,0.000000,0.000000}%
\pgfsetfillcolor{currentfill}%
\pgfsetlinewidth{0.501875pt}%
\definecolor{currentstroke}{rgb}{0.000000,0.000000,0.000000}%
\pgfsetstrokecolor{currentstroke}%
\pgfsetdash{}{0pt}%
\pgfsys@defobject{currentmarker}{\pgfqpoint{-0.055556in}{0.000000in}}{\pgfqpoint{0.000000in}{0.000000in}}{%
\pgfpathmoveto{\pgfqpoint{0.000000in}{0.000000in}}%
\pgfpathlineto{\pgfqpoint{-0.055556in}{0.000000in}}%
\pgfusepath{stroke,fill}%
}%
\begin{pgfscope}%
\pgfsys@transformshift{7.200000in}{2.838889in}%
\pgfsys@useobject{currentmarker}{}%
\end{pgfscope}%
\end{pgfscope}%
\begin{pgfscope}%
\pgftext[left,bottom,x=0.361274in,y=2.775860in,rotate=0.000000]{{\sffamily\fontsize{12.000000}{14.400000}\selectfont 0.0060}}
%
\end{pgfscope}%
\begin{pgfscope}%
\pgfpathrectangle{\pgfqpoint{1.000000in}{0.350000in}}{\pgfqpoint{6.200000in}{2.800000in}} %
\pgfusepath{clip}%
\pgfsetbuttcap%
\pgfsetroundjoin%
\pgfsetlinewidth{0.501875pt}%
\definecolor{currentstroke}{rgb}{0.000000,0.000000,0.000000}%
\pgfsetstrokecolor{currentstroke}%
\pgfsetdash{{1.000000pt}{3.000000pt}}{0.000000pt}%
\pgfpathmoveto{\pgfqpoint{1.000000in}{3.150000in}}%
\pgfpathlineto{\pgfqpoint{7.200000in}{3.150000in}}%
\pgfusepath{stroke}%
\end{pgfscope}%
\begin{pgfscope}%
\pgfsetbuttcap%
\pgfsetroundjoin%
\definecolor{currentfill}{rgb}{0.000000,0.000000,0.000000}%
\pgfsetfillcolor{currentfill}%
\pgfsetlinewidth{0.501875pt}%
\definecolor{currentstroke}{rgb}{0.000000,0.000000,0.000000}%
\pgfsetstrokecolor{currentstroke}%
\pgfsetdash{}{0pt}%
\pgfsys@defobject{currentmarker}{\pgfqpoint{0.000000in}{0.000000in}}{\pgfqpoint{0.055556in}{0.000000in}}{%
\pgfpathmoveto{\pgfqpoint{0.000000in}{0.000000in}}%
\pgfpathlineto{\pgfqpoint{0.055556in}{0.000000in}}%
\pgfusepath{stroke,fill}%
}%
\begin{pgfscope}%
\pgfsys@transformshift{1.000000in}{3.150000in}%
\pgfsys@useobject{currentmarker}{}%
\end{pgfscope}%
\end{pgfscope}%
\begin{pgfscope}%
\pgfsetbuttcap%
\pgfsetroundjoin%
\definecolor{currentfill}{rgb}{0.000000,0.000000,0.000000}%
\pgfsetfillcolor{currentfill}%
\pgfsetlinewidth{0.501875pt}%
\definecolor{currentstroke}{rgb}{0.000000,0.000000,0.000000}%
\pgfsetstrokecolor{currentstroke}%
\pgfsetdash{}{0pt}%
\pgfsys@defobject{currentmarker}{\pgfqpoint{-0.055556in}{0.000000in}}{\pgfqpoint{0.000000in}{0.000000in}}{%
\pgfpathmoveto{\pgfqpoint{0.000000in}{0.000000in}}%
\pgfpathlineto{\pgfqpoint{-0.055556in}{0.000000in}}%
\pgfusepath{stroke,fill}%
}%
\begin{pgfscope}%
\pgfsys@transformshift{7.200000in}{3.150000in}%
\pgfsys@useobject{currentmarker}{}%
\end{pgfscope}%
\end{pgfscope}%
\begin{pgfscope}%
\pgftext[left,bottom,x=0.361274in,y=3.086971in,rotate=0.000000]{{\sffamily\fontsize{12.000000}{14.400000}\selectfont 0.0065}}
%
\end{pgfscope}%
\begin{pgfscope}%
\pgftext[left,bottom,x=0.291829in,y=0.932495in,rotate=90.000000]{{\sffamily\fontsize{12.000000}{14.400000}\selectfont diffusion coefficient}}
%
\end{pgfscope}%
\begin{pgfscope}%
\pgfsetrectcap%
\pgfsetroundjoin%
\pgfsetlinewidth{1.003750pt}%
\definecolor{currentstroke}{rgb}{0.000000,0.000000,0.000000}%
\pgfsetstrokecolor{currentstroke}%
\pgfsetdash{}{0pt}%
\pgfpathmoveto{\pgfqpoint{1.000000in}{3.150000in}}%
\pgfpathlineto{\pgfqpoint{7.200000in}{3.150000in}}%
\pgfusepath{stroke}%
\end{pgfscope}%
\begin{pgfscope}%
\pgfsetrectcap%
\pgfsetroundjoin%
\pgfsetlinewidth{1.003750pt}%
\definecolor{currentstroke}{rgb}{0.000000,0.000000,0.000000}%
\pgfsetstrokecolor{currentstroke}%
\pgfsetdash{}{0pt}%
\pgfpathmoveto{\pgfqpoint{7.200000in}{0.350000in}}%
\pgfpathlineto{\pgfqpoint{7.200000in}{3.150000in}}%
\pgfusepath{stroke}%
\end{pgfscope}%
\begin{pgfscope}%
\pgfsetrectcap%
\pgfsetroundjoin%
\pgfsetlinewidth{1.003750pt}%
\definecolor{currentstroke}{rgb}{0.000000,0.000000,0.000000}%
\pgfsetstrokecolor{currentstroke}%
\pgfsetdash{}{0pt}%
\pgfpathmoveto{\pgfqpoint{1.000000in}{0.350000in}}%
\pgfpathlineto{\pgfqpoint{7.200000in}{0.350000in}}%
\pgfusepath{stroke}%
\end{pgfscope}%
\begin{pgfscope}%
\pgfsetrectcap%
\pgfsetroundjoin%
\pgfsetlinewidth{1.003750pt}%
\definecolor{currentstroke}{rgb}{0.000000,0.000000,0.000000}%
\pgfsetstrokecolor{currentstroke}%
\pgfsetdash{}{0pt}%
\pgfpathmoveto{\pgfqpoint{1.000000in}{0.350000in}}%
\pgfpathlineto{\pgfqpoint{1.000000in}{3.150000in}}%
\pgfusepath{stroke}%
\end{pgfscope}%
\begin{pgfscope}%
\pgfsetrectcap%
\pgfsetroundjoin%
\definecolor{currentfill}{rgb}{1.000000,1.000000,1.000000}%
\pgfsetfillcolor{currentfill}%
\pgfsetlinewidth{1.003750pt}%
\definecolor{currentstroke}{rgb}{0.000000,0.000000,0.000000}%
\pgfsetstrokecolor{currentstroke}%
\pgfsetdash{}{0pt}%
\pgfpathmoveto{\pgfqpoint{1.069417in}{2.427606in}}%
\pgfpathlineto{\pgfqpoint{1.926808in}{2.427606in}}%
\pgfpathlineto{\pgfqpoint{1.926808in}{3.080583in}}%
\pgfpathlineto{\pgfqpoint{1.069417in}{3.080583in}}%
\pgfpathlineto{\pgfqpoint{1.069417in}{2.427606in}}%
\pgfpathclose%
\pgfusepath{stroke,fill}%
\end{pgfscope}%
\begin{pgfscope}%
\pgfsetrectcap%
\pgfsetroundjoin%
\pgfsetlinewidth{1.003750pt}%
\definecolor{currentstroke}{rgb}{0.000000,0.000000,1.000000}%
\pgfsetstrokecolor{currentstroke}%
\pgfsetdash{}{0pt}%
\pgfpathmoveto{\pgfqpoint{1.166600in}{2.968161in}}%
\pgfpathlineto{\pgfqpoint{1.360967in}{2.968161in}}%
\pgfusepath{stroke}%
\end{pgfscope}%
\begin{pgfscope}%
\pgftext[left,bottom,x=1.513683in,y=2.890691in,rotate=0.000000]{{\sffamily\fontsize{9.996000}{11.995200}\selectfont spc}}
%
\end{pgfscope}%
\begin{pgfscope}%
\pgfsetrectcap%
\pgfsetroundjoin%
\pgfsetlinewidth{1.003750pt}%
\definecolor{currentstroke}{rgb}{0.000000,0.500000,0.000000}%
\pgfsetstrokecolor{currentstroke}%
\pgfsetdash{}{0pt}%
\pgfpathmoveto{\pgfqpoint{1.166600in}{2.764385in}}%
\pgfpathlineto{\pgfqpoint{1.360967in}{2.764385in}}%
\pgfusepath{stroke}%
\end{pgfscope}%
\begin{pgfscope}%
\pgftext[left,bottom,x=1.513683in,y=2.686915in,rotate=0.000000]{{\sffamily\fontsize{9.996000}{11.995200}\selectfont spce}}
%
\end{pgfscope}%
\begin{pgfscope}%
\pgfsetrectcap%
\pgfsetroundjoin%
\pgfsetlinewidth{1.003750pt}%
\definecolor{currentstroke}{rgb}{1.000000,0.000000,0.000000}%
\pgfsetstrokecolor{currentstroke}%
\pgfsetdash{}{0pt}%
\pgfpathmoveto{\pgfqpoint{1.166600in}{2.560609in}}%
\pgfpathlineto{\pgfqpoint{1.360967in}{2.560609in}}%
\pgfusepath{stroke}%
\end{pgfscope}%
\begin{pgfscope}%
\pgftext[left,bottom,x=1.513683in,y=2.483139in,rotate=0.000000]{{\sffamily\fontsize{9.996000}{11.995200}\selectfont tip3p}}
%
\end{pgfscope}%
\end{pgfpicture}%
\makeatother%
\endgroup%
}
    		\caption{LONG}
		\end{subfigure}
        \caption{LONG} \label{fig:latticeLDA}
\end{figure}
