% Literaturbersicht
% basiert auf KOMA-Script scrbook-Klasse
\documentclass[11pt,a4paper,% BCOR8mm,
	       %twoside, onecolumn, openright, cleardoubleempty, %
	       %parindent,
	       headsepline=false, footsepline=false, notitlepage, %
	       %onelinecaption,
	       bigheadings, %
	       bibtotoc, %tocindent, listsindent, %
	       %chapterprefix, noappendixprefix,
	       %tablecaptionbelow,
	       %pointlessnumbers, % macht Probleme (Anhang ohne Punkt, aber sonst Kapitel mit Punkt
	       % abstractoff, fleqn, leqno,
	       % openbib, origlongtable,
	       final]{scrartcl}

% Satzspiegel 

% wenn keine KOMA-Klasse verwendet wird, kann so der Satzspiegel
% berechnet werden
%\usepackage[DIV15,BCOR12mm,pagesize]{typearea}

% Hier knnen Seitenhoehe und -breite individuell angepasst werden
%\areaset[BCOR]{Breite}{Hohe}
% oder 
%\usepackage[a4paper,body={15.6cm,23cm},left=3cm]{geometry}

% ============================================================================

%% Grafikpakete
% Fr einfache Einbingung von Grafiken
\usepackage{graphicx}%

% Wenn man direkt mit dem pdflatex eine PDF-Datei erzeugt, sollten diese beiden
% Pakete eingebunden werden (Hyperlinks, bessere Bildschirmschriftarten usw.)
\usepackage{color}
\definecolor{mylinkcolor}{rgb}{0.5812,0.0665,0.0659} % IndianRed
\definecolor{mycitecolor}{rgb}{0.075,0.31,0.0431} % MossGreen
\definecolor{myurlcolor}{rgb}{0.0118,0.098,0.7412} % DarkBlue

% Druckversion
% für reine pdf-Dateien noch Option colorlinks hinzufgen, um Links farbig zumachen
%\usepackage[pdftex,bookmarks,bookmarksopen,citecolor={mycitecolor},%
%linkcolor={mylinkcolor},urlcolor={myurlcolor},breaklinks=true,%
%hypertexnames=false,hyperindex=true,encap,colorlinks]{hyperref} %


%\usepackage{ae,aecompl}

% ============================================================================

%% Sprachliche Pakete
\usepackage[english]{babel}
% Neue Deutsche Rechtschreibung
%\usepackage[ngerman]{babel}
%\usepackage{ngerman} 

% Paket zur einfacheren Eingabe deutscher Umlaute
\usepackage[applemac]{inputenc} %Mac
%\usepackage[latin1]{inputenc}   %UNIX/LINUX
% \usepackage[ansinew]{inputenc}  % Windows
\usepackage[T1]{fontenc}

% ============================================================================

% Literaturverzeichnis
\usepackage[square,numbers,sort&compress]{natbib}
%\usepackage{bibmods}
%\bibpunct{(}{)}{,}{a}{}{,}
%\bibpunct{(}{)}{,}{a}{,}{;}
\setlength{\bibsep}{1ex}

% ============================================================================

% Stichwortverzeichnis
% \usepackage{makeidx}
% \makeindex

% ============================================================================

% Deutsche Zahlenkonvention (1 (Komma) 0 statt 1 (Punkt) 0)
% \usepackage{ziffer}

%% Schriftarten
\usepackage{times} % times is used to avoid bitmap fonts in PDF

%% Mathematische Packages
% Dieses Pakete definiert viele ntzliche mathematische Befehele und
% Zeichens�ze und sollte in jedem mathematischen Dokument eingebunden werden.
% Die Option "intlimits" bewirkt, dass beim Integral die Grenzenangaben oben
% und unten erscheinen und nicht seitlich.
\usepackage[intlimits]{amsmath}
\usepackage{amsfonts}
\usepackage{amsthm}
\usepackage{mathrsfs}
\usepackage{stmaryrd}


% Diese Schriftarten ermglichen schne Mengensymbole fr natrliche Zahlen, usw.
% siehe Definition von \N, \Z usw. Dies ist Geschmackssache.
%\usepackage{bbm}
%\usepackage{dsfont}

% Subfigures
\usepackage{subfigure}

% Needed for Tabular-Umgebung
\usepackage{array}

% Diese Befehle bewirken, dass Abs�ze  bei Seitenumbrchen sauberer getrennt
% werden (sog. Schusterjungen und Hurenkinder vermeiden)
\clubpenalty = 10000
\widowpenalty = 10000

% Ein Paket um "kommutative Diagramme" zu erstellen. Fuer einfhrende Beispiele
% siehe xymanual.ps und xyreference.ps
%\usepackage[all]{xy}

%%%%%%%%%%%%%%%%%%% Shortcuts %%%%%%%%%%%%%%%%%%%%%%%%%%%%%%%%%%%%%%%%%%%%
%\newcommand{\N}{\mathbbm{N}}	% Natuerliche Zahlen
%\newcommand{\Z}{\mathbbm{Z}}	% Ganze Zahlen
%\newcommand{\Q}{\mathbbm{Q}}	% Rationale Zahlen
%\newcommand{\R}{\mathbbm{R}}	% Reelle Zahlen
%\newcommand{\C}{\mathbbm{C}}	% Komplexe Zahlen
%\newcommand{\one}{\mathbbm{1}}	% Einheits Eins
%
%\newcommand{\toinf}{\to\infty}				% --> oo
%\newcommand{\tozero}{\to 0}				% --> 0
%\newcommand{\ontop}[2]{\genfrac{}{}{0pt}{}{#1}{#2}}	% Aufeinander
%\newcommand{\abs}[1]{\left|#1\right|}
%\newcommand{\argmax}{\mathop{\rm arg\,max}}

% Ein Befehl, um Abbildungen einfach einheitlich zu gestalten
% Bsp: \abb{f}{\R}{\R}{x}{x^2}

%\newcommand{\abb}[5]{%
%\setlength{\arraycolsep}{0.4ex}%
%\begin{array}{rcccc}%
%#1 &:\,& #2 & \,\,\longrightarrow\,\, & #3 \\[0.5ex]%
%     & & #4 & \longmapsto & #5%
%\end{array}%
%}

%%%%%%%%%%%%%%%%%%% Theorem definitions %%%%%%%%%%%%%%%%%%%%%%%%%%%%%%%%%%
%\theoremstyle{plain}
%\newtheorem{theorem}{Theorem}[chapter]
%\newtheorem{proposition}[theorem]{Proposition}
%\newtheorem{lemma}[theorem]{Lemma}
%\newtheorem{satz}[theorem]{Satz}
%\newtheorem{korollar}[theorem]{Korollar}
%
%\theoremstyle{definition}
%\newtheorem{definition}{Definition}[chapter]
%\newtheorem{beispiel}[theorem]{Beispiel}
%\newtheorem{bemerkung}[theorem]{Bemerkung}
%
%%%%%%%%%%%%%%%%%%%%%%%%%%%%%%%%%%%%%%%%%%%%%%%%%%%%%%%%%%%%%%%%%%%%%%%%%%

% ============================================================================
% \renewcommand*{\partpagestyle}{empty}
% \renewcommand*{\partformat}{\partname~\thepart:}
% Kopf- und Fu�eilen
%\usepackage{fancyhdr}
%%\pagestyle{headings}
%\pagestyle{fancyplain}
%%\addtolength{\headwidth}{\marginparsep}
%%\addtolength{\headwidth}{\marginparwidth}
%\renewcommand{\chaptermark}[1]{\markboth{#1}{}}
%\renewcommand{\sectionmark}[1]{\markright{\thesection\ #1}}
%%\lhead[\fancyplain{}{\bfseries\thepage}]{\fancyplain{}{\bfseries\rightmark}}
%\lhead[\fancyplain{}{\thepage}]{\fancyplain{}{\rightmark}}
%%\rhead[\fancyplain{}{\bfseries\leftmark}]{\fancyplain{}{\bfseries\thepage}}
%\rhead[\fancyplain{}{\leftmark}]{\fancyplain{}{\thepage}}
%%\chead{}
%%\rhead{\thepage}
%%\lfoot{Schnellste Pfade in geometrischen Netzwerken}
%\cfoot{}
%%\rfoot{}
%%\setlength{\headrulewidth}{0.4pt}
%%\setlength{\footrulewidth}{0.4pt}

% Kopf- und Fu�eilen (nach KOMA Style)
%\usepackage[automark]{scrpage2}
%\pagestyle{scrheadings}
%\automark[section){chapter}
%\lehead[scrplain-links-gerade]{scrheadings-links-gerade}
%\cehead[scrplain-mittig-gerade]{scrheadings-mittig-gerade}
%\rehead[scrplain-rechts-gerade]{scrheadings-rechts-gerade}
%\lefoot[scrplain-links-gerade]{scrheadings-links-gerade}
%\cefoot[scrplain-mittig-gerade]{scrheadings-mittig-gerade}
%\refoot[scrplain-rechts-gerade]{scrheadings-rechts-gerade}
%\lohead[scrplain-links-ungerade]{scrheadings-links-ungerade}
%\cohead[scrplain-mittig-ungerade]{scrheadings-mittig-ungerade}
%\rohead[scrplain-rechts-ungerade]{scrheadings-rechts-ungerade}
%\lofoot[scrplain-links-ungerade]{scrheadings-links-ungerade}
%\cofoot[scrplain-mittig-ungerade]{scrheadings-mittig-ungerade}
%\rofoot[scrplain-rechts-ungerade]{scrheadings-rechts-ungerade}
%\ihead[scrplain-innen]{scrheadings-innen}
%\chead[scrplain-zentriert]{scrheadings-zentriert}
%\ohead[scrplain-au�n]{scrheadings-au�n}
%\ifoot[scrplain-innen]{scrheadings-innen}
%\cfoot[scrplain-zentriert]{scrheadings-zentriert}
%\ofoot[scrplain-au�n]{scrheadings-au�n}
% Markierungen: \leftmark, \rightmark, \pagemark \headmark
% \manualmark \automark

% ============================================================================  

%%%%%%%%%%%%%%%%%%%%%%%%%%%%%%%%%%%%%%%%%%%%%%%%%%%%%%%%%%%%%%%%%%%%%%%%%%
\setcounter{secnumdepth}{2}
\setcounter{tocdepth}{2}
%%%%%%%%%%%%%%%%%%%%%%%%%%%%%%%%%%%%%%%%%%%%%%%%%%%%%%%%%%%%%%%%%%%%%%%%%%

%% Schriftsatz in KOMA
%\setkomafont{Element}{Befehle}
%\addtokomafont{Element}{Befehle}
%\usekomafont{Element}

%\includeonly{kapitel/sinterkeramiken, kapitel/statisch, kapitel/dynamisch, kapitel/ergebnisse, kapitel/abbildungen,
% kapitel/literatur}
%\includeonly{kapitel/titelseite2}
% ============================================================================

% fr komplette Zitate im Text
% \usepackage{bibentry}

\newbox{\taskbox}
\usepackage{ifthen}
\newenvironment{task}[1][]{%
  \bigskip
  \noindent%
  \begin{lrbox}{\taskbox}
    \begin{minipage}{0.9\textwidth}
      \textbf{Task:} \ifthenelse{\equal{#1}{1}}{\hfill(#1 point)}{\hfill(#1 points)}%
      \par\itshape%
    }{%
    \end{minipage}
  \end{lrbox}
  \smallskip
  \fbox{\usebox{\taskbox}}
  \bigskip \noindent%

}

\newenvironment{otask}[1][]{%
  \bigskip
  \noindent%
  \begin{lrbox}{\taskbox}
    \begin{minipage}{0.9\textwidth}
      \textbf{Optional task:} \ifthenelse{\equal{#1}{1}}{\hfill(#1 bonus point)}{\hfill(#1 bonus points)}%
      \par\itshape%
    }{%
    \end{minipage}
  \end{lrbox}
  \smallskip
  \fbox{\usebox{\taskbox}}
  \bigskip\noindent%

}



\begin{document}

\titlehead{Simulation Methods in Physics II (SS 2013)}
\title{Worksheet 5\\Coarse-grained simulations with ESPResSo}
\author{Jens Smiatek\thanks{smiatek@icp.uni-stuttgart.de} and Maria Fyta\thanks{mfyta@icp.uni-stuttgart.de}}
%\date{October 28, 2009}
\publishers{ICP, Uni Stuttgart}


\maketitle

%\tableofcontents
\section*{Important remarks}
\begin{itemize}
\item Due date: {\bf Tuesday, July 16$^{th}$, 2013, 8:00}  
\item You can either send a PDF file to Jens Smiatek (smiatek@icp.uni-stuttgart.de) or submit a hand-written copy.
\item If you have further questions, contact Jens Smiatek (smiatek@icp.uni-stuttgart.de) 
\end{itemize}

\section*{The program package ESPResSo}
The program package ESPResSo is developed and maintained at the Institute for Computational Physics and 
is mainly intended to perform coarse-grained simulations with Lattice-Boltzmann, Dissipative Particle Dynamics and Langevin Dynamics.
It consists of a broad variety of electrostatic algorithms, analysis tools and various other features like the support of massively parallelized hardware architectures or GPU-platforms.
In the following you will conduct coarse-grained simulations with the Lattice-Boltzmann and Dissipative Particle Dynamics method to learn how to work with ESPResSo.
The package can be downloaded at
\begin{itemize}
\item ESPResSo-Homepage: http://espressomd.org/
\item ESPResSo-Download: http://espressomd.org/wordpress/download/
\item ESPResSo-Manual: http://espressomd.org/jenkins/job/ESPResSo/lastSuccessfulBuild/artifact/doc/ug/ug.pdf
\end{itemize}
{\bf Important Remark:} It is good to have a look inside the manual to understand how it works. This tutorial is mainly intended to understand how the coarse-grained methods and the simulation package works.

\section*{Coarse-grained simulations (20 points)}
\section*{Download and install ESPResSo}
Download the ESPResSo package version 3.2.0 (espresso-3.2.0.tar.gz) and install it in your home directory. 
You can follow the commands as given in the manual on p. 14.
Before compilation and after using "./configure", please uncomment the macros in "myconfig-sample.h" for
\begin{itemize}
\item EXTERNAL\_FORCES
\item CONSTRAINTS
\item DPD
\item TUNABLE\_SLIP
\item LB
\item LENNARD\_JONES
\end{itemize}
and rename it to myconfig.h.

\section*{Lattice-Boltzmann simulations of a single particle in a fluid}
Attached to this tutorial you will find the TCL-Script "lb\_particle.tcl" which is intended to perform a Lattice-Boltzmann simulation of a single particle in a fluid. 
Have a look at the parameters in the script. What is their meaning?
You can run the simulation with the command
\begin{itemize}
\item ./Espresso lb\_particle.tcl
\end{itemize}
if your installation has been correctly conducted. Otherwise ask the tutors.
\section{Single particle in a fluid}
In the following the motion of a single particle in a Lattice-Boltzmann fluid will be considered. The coupling of the particle to the fluid will be performed by the scheme that was introduced in the lecture.
Have look into the script file. Try to interpret the parameters of most importance.
How is the system simulated? What are the interactions? Have a look into the manual to understand how it works.
Visualize the corresponding trajectory.\\
Run the simulation.
Calculate the mean square displacement of the particle by the equation
\begin{equation}
<(\vec{r}(t)-\vec{r}(t_0))^2> = MSD
\end{equation}
and determine the Diffusion coefficient. \\
Hint: Remember Simulation Methods I how to calculate it.

\section*{Plane Poiseuille flow with Dissipative Particle Dynamics (DPD)}
In the following, we will simulate a Plane Poiseuille Flow (PPF) in a confined microgeometry with Dissipative Particle Dynamics (DPD).
The Poiseuille flow occurs for low Reynolds-numbers such that the Navier-Stokes equation can be reduced to the Stokes equation
\begin{equation}
\eta\frac{\partial^2}{\partial z^2}v_x(z) = -\rho F_x
\end{equation}
with the shear viscosity $\eta$, the fluid density $\rho$, the fluid velocity $v_x(z)$ and the external force in x-direction $F_x$.
\subsection*{Plane Poiseuille flow equation}
Solve the Stokes equation to get $v_x(z)$ with the boundary conditions
\begin{itemize}
\item $v_x(z_B) = 0$ where $z_B$ denotes the position of the channel walls
\item $\partial_zv_x(z)|_{z=0} = 0$ for the velocity derivative in the middle of the channel
\end{itemize}
\section*{DPD simulations}
Look at the script "PPF.tcl". Try to interpret the parameters of most importance.
How is the system simulated? What are the interactions? Have a look into the manual to understand how it works.
Run the simulation with the command {\em ./Espresso PPF.tcl}.
Some parameters in the script are missing.
\begin{itemize}
\item We need a solvent density of $3.75 \sigma^{-3}$. The effective box lengths are given by $(10\times 10\times 8)\sigma^{-3}$. To insert $\rho=3.75 \sigma^{-3}$ is therefore not correct. Have a look into the file!
\item The DPD friction coefficient is given by $\gamma_{DPD}=5.0$.
\end{itemize}
Run the warm up and the simulation.
\section*{Analysis of the Plane Poiseuille flow}
After the simulations have finished, have a look at the output file. Try to understand the meaning of the different rows by having a look into the tcl-file.
In order to analyze the flow profile, we have to calculate a histogram for several z-positions.
Therefore we sort the particles according to their z-position to equidistant bins and sum up their velocities in x-direction. Having sorted all the particles, we average the summed velocities of the bin by taking into account 
the particle number within the bin
to get the 
average velocity. Finally the average velocities will printed out according to their z-position which gives you the flow profile.
Fit the corresponding flow profile and estimate the shear viscosity. Have a look at the PPF.tcl-file to estimate the external forces.

\end{document}

