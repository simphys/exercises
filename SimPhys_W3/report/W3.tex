%%%%%%%% Klassen-Optionen
\documentclass[12pt,a4paper]{scrartcl}

%%%%%%%% PAKETE: unverzichtbare Pakete mit Einstellungen
\usepackage[left=2.5cm, right=2cm, top=3cm, bottom=3cm, a4paper]{geometry} %Seitenrände
\usepackage[utf8x]{inputenc} % utf8-Kodierung und direkte Eingabe von Sonderzeichen
\usepackage{fixltx2e} % Verbessert einige Kernkompetenzen von LaTeX2e

%%%%%%%% PAKETE: AMS-Pakete
\usepackage{amsmath} % Mathe-Erweiterung
\usepackage{amsfonts} % Schrift-Erweiterung
\usepackage{amssymb} % Sonderzeichen-Erweiterung

%%%%%%%% PAKETE: Sonstiges
\usepackage[colorlinks, citecolor=black, filecolor=black, linkcolor=black, urlcolor=black]{hyperref} % Links
\usepackage{wrapfig} % ausgeklügekte Floatumgebung
\usepackage{float} % normale Floatumgebung
\restylefloat{figure} % ermöglicht die Verwendung von "H" (ist noch stärker als "h!")
\usepackage[small,it,singlelinecheck=false]{caption} % Bildunterschriften formatieren
\usepackage{multirow} % ermöglich Verbinden von Tabellenzeilen
\usepackage{multicol} % ermöglicht Spalten
\usepackage{fancyhdr} % ermöglicht Kopf- und Fußzeilen
\usepackage{graphicx} % Einbinden von Bildern möglich
\usepackage{units} % Einheiten
 \usepackage{booktabs}%Tabellen - Linien
\usepackage{subcaption}

%%%%%%%% DEFINITIONEN: Titelseite
\author{April Cooper, Patrick Kreissl und Sebastian Weber}
\title{Worksheet X: NAME}
\publishers{University of Stuttgart}
\date{\today}

%%%%%%%% ANPASSUNGEN: Kopf-und Fußzeile
\fancypagestyle{plain}{} % redefine the plain pagestyle to match the fancy layout
\pagestyle{fancy} % aktiviere eigenen Seitenstil
\fancyhf{} % alle Kopf- und Fußzeilen bereinigen
\fancyhead[L]{Worksheet 3:  Molecular Dynamics 2 and Observables}
\fancyhead[R]{\today}
\renewcommand{\headrulewidth}{0.6pt} % obere Trennlinie
\fancyfoot[L]{Patrick Kreissl und Sebastian Weber}
\fancyfoot[R]{Seite \thepage}
\renewcommand{\footrulewidth}{0.6pt} % untere Trennlinie

%%%%%%%% ANPASSUNGEN: Absätze
\setlength{\parindent}{0em} % keine Absatzeinzüge
\setlength{\parskip}{0em} % Absatz-Abstand

%%%%%%%% ANPASSUNGEN: Abbildungsverzeichnis
\usepackage{tocloft} % Zum Anpassen der Verzeichnisse
\renewcommand{\cftfigpresnum}{Abb. }
\renewcommand{\cfttabpresnum}{Tab. }
\renewcommand{\cftfigaftersnum}{:}
\renewcommand{\cfttabaftersnum}{:}
\setlength{\cftfignumwidth}{2cm}
\setlength{\cfttabnumwidth}{2cm}
\setlength{\cftfigindent}{0cm}
\setlength{\cfttabindent}{0cm}

%%%%%%%% SONSTIGES
\usepackage{pdfpages}
\usepackage{pgf}

% NÜTZLICH: http://truben.no/latex/table/

% Anfang des eigentlichen Dokuments
\begin{document}

\maketitle
\tableofcontents
\newpage

% =============== Section ============
\section{Running averages}
In this section we had to calculate the running averages of the measured observables.
The resulting plot can be  seen below:

%Hier fehlt ein Plot

It can be observed that the observables fluctuate during the first time steps immensely and after some time they still fluctuate a bit, but their mean values become quite constant. This is due to the fact that first the system is equilibrating wherefore the observables fluctuate quite much, after this process the system is in equilibrium and therefore the observables stay constant.
The time the system needs to be equilibrated is about $t_equi=20$ time units.
The mean values of the observables measured in one of our simulatios are:\newline
\\
\begin{tabular}{cc}
\toprule
Observable & Mean value\\
\midrule
$E_{tot}$&1329.25\\
$E_{pot}$&-1671.12\\
$E_{kin}$&3000.38\\
T&2.00\\
P&0.6\\
\bottomrule

\end{tabular}

\section{Velocity rescaling}
 \subsection{Derivation of the rescaling factor}
 It is: 
 \begin{equation}
 \frac{k_bT}{2}=\frac{0.5mv^2}{3N}
 \end{equation}
 Which obviously leads to:
  \begin{equation}\nonumber
 T_{mes}=\frac{mv_{mes}^2}{3Nk_B} \\
 \end{equation}
\begin{equation}\nonumber
  T_{des}=\frac{mv_{des}^2}{3Nk_B} \\
\end{equation}
where $T_{mes}$ is the measured and $T_{des}$ is the desired temperature - analogous for the velocities.
Calculating the factor $\frac{T_{mes}}{T_{des}}$ and solving for $v_{des}$ leads to:
\begin{equation}
v_{des}=v_{mes} \sqrt{\frac{T_{des}}{T_{mes}}}
\end{equation}
Therefore you have to multiply the measured velocities $v_{mes}$ by the factor $\sqrt{\frac{T_{des}}{T_{mes}}}$ in order to get a correct velocity rescaling.
\newline
\newline
This velocity rescaling has been implemented in the python part due to the fact that it's a simple multiplication with a factor and nothing numerically problematic.
\end{document}


% =============== Comments ============
\begin{comment}
\verb{x_init {}}

\begin{figure}[H]
	\resizebox{1\textwidth}{!{\input{../plots/NAME.pgf}}
	\caption{CAPTION}\label{fig:NAME}
\end{figure}
\end{comment}
