%%%%%%%% Klassen-Optionen
\documentclass[12pt,a4paper]{scrartcl}

%%%%%%%% PAKETE: unverzichtbare Pakete mit Einstellungen
\usepackage[left=2.5cm, right=2cm, top=3cm, bottom=3cm, a4paper]{geometry} %Seitenrände
\usepackage[utf8x]{inputenc} % utf8-Kodierung und direkte Eingabe von Sonderzeichen
\usepackage{fixltx2e} % Verbessert einige Kernkompetenzen von LaTeX2e

%%%%%%%% PAKETE: AMS-Pakete
\usepackage{amsmath} % Mathe-Erweiterung
\usepackage{amsfonts} % Schrift-Erweiterung
\usepackage{amssymb} % Sonderzeichen-Erweiterung

%%%%%%%% PAKETE: Sonstiges
\usepackage[colorlinks, citecolor=black, filecolor=black, linkcolor=black, urlcolor=black]{hyperref} % Links
\usepackage{wrapfig} % ausgeklügekte Floatumgebung
\usepackage{float} % normale Floatumgebung
\restylefloat{figure} % ermöglicht die Verwendung von "H" (ist noch stärker als "h!")
\usepackage[small,it,singlelinecheck=false]{caption} % Bildunterschriften formatieren
\usepackage{multirow} % ermöglich Verbinden von Tabellenzeilen
\usepackage{multicol} % ermöglicht Spalten
\usepackage{fancyhdr} % ermöglicht Kopf- und Fußzeilen
\usepackage{graphicx} % Einbinden von Bildern möglich
\usepackage{units} % Einheiten
\usepackage{subcaption}

%%%%%%%% DEFINITIONEN: Titelseite
\author{April Cooper, Patrick Kreissl und Sebastian Weber}
\title{Worksheet 2: Statistical Mechanics and Molecular Dynamics}
\publishers{University of Stuttgart}
\date{\today}

%%%%%%%% ANPASSUNGEN: Kopf-und Fußzeile
\fancypagestyle{plain}{} % redefine the plain pagestyle to match the fancy layout
\pagestyle{fancy} % aktiviere eigenen Seitenstil
\fancyhf{} % alle Kopf- und Fußzeilen bereinigen
\fancyhead[L]{Worksheet 2: Statistical Mechanics and Molecular Dynamics}
\fancyhead[R]{\today}
\renewcommand{\headrulewidth}{0.6pt} % obere Trennlinie
\fancyfoot[L]{April Cooper, Patrick Kreissl und Sebastian Weber}
\fancyfoot[R]{Seite \thepage}
\renewcommand{\footrulewidth}{0.6pt} % untere Trennlinie

%%%%%%%% ANPASSUNGEN: Absätze
\setlength{\parindent}{0em} % keine Absatzeinzüge
\setlength{\parskip}{0em} % Absatz-Abstand

%%%%%%%% ANPASSUNGEN: Abbildungsverzeichnis
\usepackage{tocloft} % Zum Anpassen der Verzeichnisse
\renewcommand{\cftfigpresnum}{Abb. }
\renewcommand{\cfttabpresnum}{Tab. }
\renewcommand{\cftfigaftersnum}{:}
\renewcommand{\cfttabaftersnum}{:}
\setlength{\cftfignumwidth}{2cm}
\setlength{\cfttabnumwidth}{2cm}
\setlength{\cftfigindent}{0cm}
\setlength{\cfttabindent}{0cm}

%%%%%%%% SONSTIGES
\usepackage{pdfpages}
\usepackage{pgf}

% NÜTZLICH: http://truben.no/latex/table/

% Anfang des eigentlichen Dokuments
\begin{document}

\maketitle
\tableofcontents
\newpage

% =============== Section ============
\section{Statistical Mechanics}
\subsection{Task 1}
First we had to consider a system A consisting of two subsystems $A_1$ and $A_2$ with the related numbers of configurations $\Omega_1= 10^20$ and $\Omega_2=10^22$.\\
The number of configurations available to the combined system  is $\Omega=\Omega_1\cdot \Omega_2= 10^{42}$.\\
As the entropy is defined as $S = ln(\Omega)k_B$ the entropies are:\\
$S_1=k_B ln(\Omega_1)=k_B ln(10^{20})$,
$S_2=k_B ln(\Omega_2)=k_B ln(10^{22})$,
$S=k_B ln(\Omega)=k_B ln(10^{42})$.\\
\\
Then were asked to give the factor by which the number of available configurations increases in a system with given initial $V_1,T,p_1$ if $V_1$ was expanded isothermal by 0.001\%.
%We  assume that the air behaves like an ideal gas and as temperature is constant it is:\\
%$p_1V_1=p_2V_2 \Rightarrow p_2= \frac{p_1V_1}{V_2}\approx101324\nonumber$ Pa \\
\\
With the help of the first law of thermodynamics  it follows that for T = const it is:\\
$dS=\frac{p}{T}dV$, which leads with the ideal gas equation to
 \begin{align}
 \Delta S&=\int_{V1}^{V2} \frac{k_BN}{V}dV\nonumber\\
  &= k_BNln\left(\frac{V_2}{V_1}\right)\nonumber\\
  &=10^{-7}k_BN
   \end{align}
  If you consider now the definition of S: $\Delta S = ln( \hat \Delta \Omega)k_B $ you get with equation (1):\\
 \[ \hat \Delta\Omega = exp(\frac{\Delta S}{k_B}) = exp(10^{-7}N)=\frac{\Omega_2}{\Omega_1}\]
 \\
 Finally we were asked to give the factor by which the number of available configurations increases in a system with given initial $N,V,T_1$ when an energy of 150 kJ is  added to the system at constant Volume.



\subsection{Task 2 - Thermodynamic Variables in the Canonical Ensemble}
Given the Helmholtz free energy F we were asked to derive expressions for U,p,S.
Using the Maxwell relations it follows:\\
\begin{align}
S&=\frac{-\partial F}{\partial T} = ln(Q(N,V,T))k_B+\frac{k_BT}{Q(N,V,T)}\frac{\partial Q(N,V,T)}{\partial T}\nonumber\\
p&=\frac{-\partial F}{\partial V}=\frac{1}{\beta Q(N,V,T)}\frac{\partial Q(N,V,T)}{\partial V}\nonumber
\end{align}
\newpage
The derivation of the equation for U is as follows:\\
It is known from statistical mechanics that $Q(N,V,T)= \frac{1}{h^{3N}N!}\int d\Gamma exp(-\beta H)$. The thermodynamic properties of the system can be obtained by $Q(N,V,T)=exp(-\beta F(N,V,T))$. To justify this identification we show fist that F is extensive and then that $F=U-TS$, where $U=\langle H \rangle$.\\
That F is extensive can be derived directly from $Q(N,V,T)= \frac{1}{h^{3N}N!}\int d\Gamma exp(-\beta H)$. When the system is split up into two systems with a very weak corelation, then is Q a product of two factors.
To show the second equivalence we rewrite $F = U -TS $ to $U = \langle H \rangle = A - T\left(\frac{\partial A}{\partial T} \right)$.
To show this we divide the two expressions of Q that we stated before and therefore get:\\
\[\frac{1}{h^{3N}N!}\int d\Gamma exp(\beta (F-H))=1\]
Deriving both sides by $\beta$ leads to:\\
\[ \frac{1}{h^{3N}N!}\int d\Gamma exp(\beta (F-H))(F-H+\beta \left(\frac{\partial F}{\partial \beta}\right))=0\]
This is equivalent to $F - U - T \left(\frac{\partial F}{\partial T}\right)$, wherefore it is: $U= F+TS$.

\subsection{Task 3 - Ideal Gas}
Given the partition function $Q(N,V,T)$ we had to derive expressions for the free Helmholtz energy F(N,V,T) and the pressure p(N,V,T).\\
As we know from task 2 it is $F=\frac{-ln(Q(N,V,T))}{\beta}$.
Plugging in the $Q(N,V,T)$ given on the sheet and using the hint leads to:

 \begin{align}
 F&= -\frac{ln(Q(N,V,T)}{\beta}\nonumber\\
 &= -k_BT  ln\left(\frac{V^N}{\lambda^{3N}N!}\right)\nonumber\\
 &= -k_BT\left(N ln(V) - N ln(\lambda^{3}) - ln(N!)\right)\nonumber\\
 &= -k_BTN\left(ln(V) - ln(\lambda^{3}) -ln(N) + 1\right)\nonumber\\
 &= k_BTN\left(ln\left(\frac{N\lambda^3}{V}\right)-1\right)
 \end{align}
 Applying the Maxwell relation $\frac{-\partial F}{\partial V}=p$ on equation (1) it follows:\\
  \begin{align}
 p&=\frac{-\partial F}{\partial V} \nonumber\\
 &=\frac{-\partial k_BTN\left(ln\left(\frac{N\lambda^3}{V}\right)-1\right) }{\partial V}\nonumber\\
 &=\frac{k_BTN}{V}
 \end{align}
As it is obvious equation(2) leads to the ideal gas law : $pV=Nk_BT$.



\end{document}


% =============== Comments ============
\begin{comment}
\verb{x_init {}}

\begin{figure}[H]
	\resizebox{1\textwidth}{!{\input{../plots/NAME.pgf}}
	\caption{CAPTION}\label{fig:NAME}
\end{figure}
\end{comment}
